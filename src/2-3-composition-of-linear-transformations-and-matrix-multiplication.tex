\section{Composition of Linear Transformations and Matrix Multiplication}\label{sec:2.3}

\begin{note}
  We use the more convenient notation of \(\U \T\) rather than \(\U \circ \T\) for the composite of linear transformations \(\U\) and \(\T\).
\end{note}

\begin{thm}\label{2.9}
  Let \(\V\), \(\W\), and \(\vs{Z}\) be vector spaces over the same field \(\F\), and let \(\T \in \ls(\V, \W)\) and \(\U \in \ls(\W, \vs{Z})\).
  Then \(\U \T \in \ls(\V, \vs{Z})\).
\end{thm}

\begin{proof}[\pf{2.9}]
  Let \(x, y \in \V\) and \(a \in \F\).
  Then
  \begin{align*}
    \U \T(ax + y) & = \U(\T(ax + y))                              \\
                  & = \U(a \T(x) + \T(y))      &  & \by{2.1.2}[b] \\
                  & = a \U(\T(x)) + \U(\T(y))  &  & \by{2.1.2}[b] \\
                  & = a (\U \T)(x) + \U \T(y).
  \end{align*}
\end{proof}

\begin{thm}\label{2.10}
  Let \(\V, \W, \vs{X}, \vs{Y}\) be vector spaces over \(\F\).
  Let \(\lt{S}, \lt{S}_1, \lt{S}_2 \in \ls(\vs{X}, \vs{Y})\), let \(\T \in \ls(\W, \vs{X})\) and let \(\U, \U_1, \U_2 \in \ls(\V, \W)\).
  Then
  \begin{enumerate}
    \item \(\T(\U_1 + \U_2) = \T \U_1 + \T \U_2\) and \((\lt{S}_1 + \lt{S}_2) \T = \lt{S}_1 \T + \lt{S}_2 \T\).
    \item \(\lt{S} (\T \U) = (\lt{S} \T) \U\).
    \item \(\T \IT[\W] = \IT[\vs{X}] \T = \T\).
    \item \(a(\T \U) = (a \T) \U = \T (a \U)\) for all scalars \(a \in \F\).
  \end{enumerate}
\end{thm}

\begin{proof}[\pf{2.10}(a)]
  For all \(x \in \V\), we have
  \begin{align*}
    (\T(\U_1 + \U_2))(x) & = \T((\U_1 + \U_2)(x))                        \\
                         & = \T(\U_1(x) + \U_2(x))       &  & \by{2.2.5} \\
                         & = \T(\U_1(x)) + \T(\U_2(x))   &  & \by{2.1.1} \\
                         & = (\T \U_1)(x) + (\T \U_2)(x)                 \\
                         & = (\T \U_1 + \T \U_2)(x).     &  & \by{2.2.5}
  \end{align*}
  Thus \(\T(\U_1 + \U_2) = \T \U_1 + \T \U_2\).
  For all \(x \in \W\), we have
  \begin{align*}
    ((\vs{S}_1 + \vs{S}_2) \T)(x) & = (\vs{S}_1 + \vs{S}_2)(\T(x))                        \\
                                  & = \vs{S}_1(\T(x)) + \vs{S}_2(\T(x))   &  & \by{2.2.5} \\
                                  & = (\vs{S}_1 \T)(x) + (\vs{S}_2 \T)(x)                 \\
                                  & = (\vs{S}_1 \T + \vs{S}_2 \T)(x).     &  & \by{2.2.5}
  \end{align*}
  Thus \((\vs{S}_1 + \vs{S}_2) \T = \vs{S}_1 \T + \vs{S}_2 \T\).
\end{proof}

\begin{proof}[\pf{2.10}(b)]
  For all \(x \in \V\), we have
  \begin{align*}
    (\lt{S} (\T \U))(x) & = \lt{S}((\T \U)(x))  \\
                        & = \lt{S}(\T(\U(x)))   \\
                        & = (\lt{S} \T)(\U(x))  \\
                        & = ((\lt{S} \T) \U)(x)
  \end{align*}
  and thus \(\lt{S} (\T \U) = (\lt{S} \T) \U\).
\end{proof}

\begin{proof}[\pf{2.10}(c)]
  For all \(x \in \V\), we have
  \begin{align*}
    (\T \IT[\W])(x) & = \T(\IT[\W](x))                      \\
                    & = \T(x)               &  & \by{2.1.9} \\
                    & = \IT[\vs{X}](\T(x))  &  & \by{2.1.9} \\
                    & = (\IT[\vs{X}] \T)(x)
  \end{align*}
  and thus \(\T \IT[\W] = \T = \IT[\vs{X}] \T\).
\end{proof}

\begin{proof}[\pf{2.10}(d)]
  For all \(x \in \V\), we have
  \begin{align*}
    (a (\T \U))(x) & = a (\T \U)(x)                   \\
                   & = a \T(\U(x))                    \\
                   & = (a \T)(\U(x))  &  & \by{2.2.5} \\
                   & = ((a \T) \U)(x)                 \\
                   & = \T(a \U(x))    &  & \by{2.1.1} \\
                   & = \T((a \U)(x))                  \\
                   & = (\T (a \U))(x)
  \end{align*}
  and thus \(a (\T \U) = (a \T) \U = \T (a \U)\).
\end{proof}

\begin{defn}\label{2.3.1}
  Let \(A \in \ms\) matrix and \(B \in \ms[n][p][\F]\).
  We define the \textbf{product} of \(A\) and \(B\), denoted \(AB\), to be the \(m \times p\) matrix such that
  \[
    (AB)_{i j} = \sum_{k = 1}^n A_{i k} B_{k j} \quad \text{for } 1 \leq i \leq m, 1 \leq j \leq p.
  \]
  Note that \((AB)_{i j}\) is the sum of products of corresponding entries from the \(i\)th row of \(A\) and the \(j\)th column of \(B\).
\end{defn}

\begin{note}
  The reader should observe that in order for the product \(AB\) to be defined, there are restrictions regarding the relative sizes of \(A\) and \(B\).
  The following mnemonic device is helpful:
  ``\((m \times n) \cdot (n \times p) = (m \times p)\)'';
  that is, in order for the product \(AB\) to be defined, the two ``inner'' dimensions must be equal, and the two ``outer'' dimensions yield the size of the product.
\end{note}

\begin{note}
  As in the case with composition of functions, we have that matrix multiplication is not commutative. Consider the following two products:
  \[
    \begin{pmatrix}
      1 & 1 \\
      0 & 0
    \end{pmatrix} \begin{pmatrix}
      0 & 1 \\
      1 & 0
    \end{pmatrix} = \begin{pmatrix}
      1 & 1 \\
      0 & 0
    \end{pmatrix} \quad \text{and} \quad \begin{pmatrix}
      0 & 1 \\
      1 & 0
    \end{pmatrix} \begin{pmatrix}
      1 & 1 \\
      0 & 0
    \end{pmatrix} = \begin{pmatrix}
      0 & 0 \\
      1 & 1
    \end{pmatrix}.
  \]
  Hence we see that even if both of the matrix products \(AB\) and \(BA\) are defined, it need not be true that \(AB = BA\).
\end{note}

\begin{eg}\label{2.3.2}
  If \(A \in \ms\) and \(B \in \ms[n][p][\F]\), then \(\tp{(AB)} = \tp{B} \tp{A}\).
\end{eg}

\begin{proof}[\pf{2.3.2}]
  Since
  \[
    \tp{(AB)}_{i j} = (AB)_{j i} = \sum_{k = 1}^n A_{j k} B_{k i}
  \]
  and
  \[
    (\tp{B} \tp{A})_{i j} = \sum_{k = 1}^n \tp{B}_{i k} \tp{A}_{k j} = \sum_{k = 1}^n B_{k i} A_{j k},
  \]
  we are finished.
  Therefore the transpose of a product is the product of the transposes in the \emph{opposite order}.
\end{proof}

\begin{thm}\label{2.11}
  Let \(\V\), \(\W\), and \(\vs{Z}\) be finite-dimensional vector spaces over \(\F\) with ordered bases \(\alpha\), \(\beta\), and \(\gamma\) over \(\F\), respectively.
  Let \(\T \in \ls(\V, \W)\) and \(\U \in \ls(\W, \vs{Z})\).
  Then
  \[
    [\U \T]_{\alpha}^{\gamma} = [\U]_{\beta}^{\gamma} [\T]_{\alpha}^{\beta}.
  \]
\end{thm}

\begin{proof}[\pf{2.11}]
  Define
  \begin{align*}
    \alpha & = \set{\seq{v}{1,,n}}; \\
    \beta  & = \set{\seq{w}{1,,m}}; \\
    \gamma & = \set{\seq{z}{1,,p}}.
  \end{align*}
  For \(1 \leq j \leq n\), we have
  \begin{align*}
    (\U \T)(v_j) & = \sum_{i = 1}^p ([\U \T]_{\alpha}^{\gamma})_{i j} \cdot z_i                                                      &  & \by{2.2.4}    \\
                 & = \U(\T(v_j))                                                                                                                        \\
                 & = \U\pa{\sum_{k = 1}^m ([\T]_{\alpha}^{\beta})_{k j} \cdot w_k}                                                   &  & \by{2.2.4}    \\
                 & = \sum_{k = 1}^m ([\T]_{\alpha}^{\beta})_{k j} \cdot \U(w_k)                                                      &  & \by{2.1.2}[d] \\
                 & = \sum_{k = 1}^m ([\T]_{\alpha}^{\beta})_{k j} \cdot \pa{\sum_{i = 1}^p ([\U]_{\beta}^{\gamma})_{i k} \cdot z_i}  &  & \by{2.2.4}    \\
                 & = \sum_{k = 1}^m \sum_{i = 1}^p \pa{([\T]_{\alpha}^{\beta})_{k j}  \cdot ([\U]_{\beta}^{\gamma})_{i k} \cdot z_i} &  & \by{1.2.1}    \\
                 & = \sum_{k = 1}^m \sum_{i = 1}^p \pa{([\U]_{\beta}^{\gamma})_{i k} \cdot ([\T]_{\alpha}^{\beta})_{k j} \cdot z_i}  &  & \by{1.2.1}    \\
                 & = \sum_{i = 1}^p \sum_{k = 1}^m \pa{([\U]_{\beta}^{\gamma})_{i k} \cdot ([\T]_{\alpha}^{\beta})_{k j} \cdot z_i}  &  & \by{1.2.1}    \\
                 & = \sum_{i = 1}^p \pa{\sum_{k = 1}^m ([\U]_{\beta}^{\gamma})_{i k} \cdot ([\T]_{\alpha}^{\beta})_{k j}} \cdot z_i  &  & \by{1.2.1}    \\
                 & = \sum_{i = 1}^p \pa{\pa{[\U]_{\beta}^{\gamma} [\T]_{\alpha}^{\beta}}_{i j} \cdot z_i}.                           &  & \by{2.3.1}
  \end{align*}
  Thus by \cref{1.8} we see that \([\U \T]_{\alpha}^{\gamma} = [\U]_{\beta}^{\gamma} [\T]_{\alpha}^{\beta}\).
\end{proof}

\begin{cor}\label{2.3.3}
  Let \(\V\) be a finite-dimensional vector space over \(\F\) with an ordered basis \(\beta\).
  Let \(\T, \U \in \ls(\V)\).
  Then \([\U \T]_{\beta} = [\U]_{\beta} [\T]_{\beta}\).
\end{cor}

\begin{proof}[\pf{2.3.3}]
  We have
  \begin{align*}
    [\U \T]_{\beta} & = [\U \T]_{\beta}^{\beta}                   &  & \by{2.2.4} \\
                    & = [\U]_{\beta}^{\beta} [\T]_{\beta}^{\beta} &  & \by{2.11}  \\
                    & = [\U]_{\beta} [\T]_{\beta}.                &  & \by{2.2.4}
  \end{align*}
\end{proof}

\begin{defn}\label{2.3.4}
  We define the \textbf{Kronecker delta} \(\delta_{i j}\) by \(\delta_{i j} = 1\) if \(i = j\) and \(\delta_{i j} = 0\) if \(i \neq j\).
  The \(n \times n\) \textbf{identity matrix} \(I_n\) is defined by \((I_n)_{i j} = \delta_{i j}\).
\end{defn}

\begin{thm}\label{2.12}
  Let \(A \in \ms\), let \(B, C \in \ms[n][p][\F]\) and let \(D, E \in \ms[q][m][\F]\).
  Then
  \begin{enumerate}
    \item \(A (B + C) = AB + AC\) and \((D + E) A = DA + EA\).
    \item \(a (AB) = (aA) B = A (aB)\) for any \(a \in \F\).
    \item \(I_m A = A = A I_n\).
    \item If \(\V\) is an \(n\)-dimensional vector space over \(\F\) with an ordered basis \(\beta\), then \([\IT[\V]]_{\beta} = I_n\).
  \end{enumerate}
\end{thm}

\begin{proof}[\pf{2.12}(a)]
  We have
  \begin{align*}
    (A (B + C))_{i j} & = \sum_{k = 1}^n A_{i k} (B + C)_{k j}                            &  & \by{2.3.1}                         \\
                      & = \sum_{k = 1}^n A_{i k} (B_{k j} + C_{k j})                      &  & \by{1.2.9}                         \\
                      & = \sum_{k = 1}^n (A_{i k} B_{k j} + A_{i k} C_{k j})              &  & (A_{i k}, B_{k j}, C_{k j} \in \F) \\
                      & = \sum_{k = 1}^n A_{i k} B_{k j} + \sum_{k = 1}^n A_{i k} C_{k j} &  & (A_{i k}, B_{k j}, C_{k j} \in \F) \\
                      & = (AB)_{i j} + (AC)_{i j}                                         &  & \by{2.3.1}                         \\
                      & = (AB + AC)_{i j}                                                 &  & \by{1.2.9}
  \end{align*}
  and
  \begin{align*}
    ((D + E) A)_{i j} & = \sum_{k = 1}^m (D + E)_{i k} A_{k j}                            &  & \by{2.3.1}                         \\
                      & = \sum_{k = 1}^m (D_{i k} + E_{i k}) A_{k j}                      &  & \by{1.2.9}                         \\
                      & = \sum_{k = 1}^m (D_{i k} A_{k j} + E_{i k} A_{k j})              &  & (A_{k j}, D_{i k}, E_{i k} \in \F) \\
                      & = \sum_{k = 1}^m D_{i k} A_{k j} + \sum_{k = 1}^m E_{i k} A_{k j} &  & (A_{k j}, D_{i k}, E_{i k} \in \F) \\
                      & = (DA)_{i j} + (EA)_{i j}                                         &  & \by{2.3.1}                         \\
                      & = (DA + EA)_{i j}.                                                &  & \by{1.2.9}
  \end{align*}
  Thus by \cref{1.2.8} \(A (B + C) = AB + AC\) and \((D + E) A = DA + EA\).
\end{proof}

\begin{proof}[\pf{2.12}(b)]
  We have
  \begin{align*}
    (a (AB))_{i j} & = a (AB)_{i j}                          &  & \by{1.2.9} \\
                   & = a \pa{\sum_{k = 1}^n A_{i k} B_{k j}} &  & \by{2.3.1} \\
                   & = \sum_{k = 1}^n (a A_{i k}) B_{k j}    &  & \by{1.2.9} \\
                   & = \sum_{k = 1}^n (a A)_{i k} B_{k j}    &  & \by{1.2.9} \\
                   & = ((aA) B)_{i j}                        &  & \by{2.3.1} \\
                   & = \sum_{k = 1}^n A_{i k} (a B_{k j})    &  & \by{1.2.9} \\
                   & = \sum_{k = 1}^n A_{i k} (a B)_{k j}    &  & \by{1.2.9} \\
                   & = (A (aB))_{i j}                        &  & \by{2.3.1}
  \end{align*}
  thus by \cref{1.2.8} \(a (AB) = (aA) B = A (aB)\).
\end{proof}

\begin{proof}[\pf{2.12}(c)]
  We have
  \begin{align*}
    (I_m A)_{i j} & = \sum_{k = 1}^m (I_m)_{i k} A_{k j}  &  & \by{2.3.1} \\
                  & = \sum_{k = 1}^m \delta_{i k} A_{k j} &  & \by{2.3.4} \\
                  & = A_{i j}                             &  & \by{2.3.4} \\
                  & = \sum_{k = 1}^n A_{i k} \delta_{k j} &  & \by{2.3.4} \\
                  & = \sum_{k = 1}^n A_{i k} (I_n)_{k j}  &  & \by{2.3.4} \\
                  & = (A I_n)_{i j}                       &  & \by{2.3.1}
  \end{align*}
  thus by \cref{1.2.8} \(I_m A = A = A I_n\).
\end{proof}

\begin{proof}[\pf{2.12}(d)]
  Let \(\beta = \set{\seq{v}{1,,n}}\).
  For all \(j \in \set{1, 2, \dots, n}\), we have
  \begin{align*}
             & \IT[\V](v_j) = v_j                                                                          &  & \by{2.1.9} \\
             & = \sum_{i = 1}^n ([\IT[\V]]_{\beta})_{i j} v_i                                              &  & \by{2.2.4} \\
    \implies & \forall i \in \set{1, 2, \dots, n}, ([\IT[\V]]_{\beta})_{i j} = \delta_{i j} = (I_n)_{i j}. &  & \by{2.3.4}
  \end{align*}
  Thus by \cref{1.2.8} \([\IT[\V]]_{\beta} = I_n\).
\end{proof}

\begin{note}
  \cref{2.12} provides analogs of (a), (c), and (d) of \cref{2.10}.
  \cref{2.10}(b) has its analog in \cref{2.16}.
  Observe also that part (c) of \cref{2.12} illustrates that the identity matrix acts as a multiplicative identity in \(\ms[n][n][\F]\).
  When the context is clear, we sometimes omit the subscript \(n\) from \(I_n\).
\end{note}

\begin{cor}\label{2.3.5}
  Let
  \begin{align*}
    A             & \in \ms;           \\
    \seq{B}{1,,k} & \in \ms[n][p][\F]; \\
    \seq{C}{1,,k} & \in \ms[q][m][\F]; \\
    \seq{a}{1,,k} & \in \F.
  \end{align*}
  Then
  \begin{align*}
    A \pa{\sum_{i = 1}^k a_i B_i} & = \sum_{i = 1}^k a_i A B_i, \\
    \pa{\sum_{i = 1}^k a_i C_i} A & = \sum_{i = 1}^k a_i C_i A.
  \end{align*}
\end{cor}

\begin{proof}[\pf{2.3.5}]
  We have
  \begin{align*}
    A \pa{\sum_{i = 1}^k a_i B_i} & = \sum_{i = 1}^k (A a_i B_i)  &  & \by{2.12}[a] \\
                                  & = \sum_{i = 1}^k (a_i A B_i), &  & \by{2.12}[b] \\
    \pa{\sum_{i = 1}^k a_i C_i} A & = \sum_{i = 1}^k (a_i C_i A). &  & \by{2.12}[a]
  \end{align*}
\end{proof}

\begin{defn}\label{2.3.6}
  For an \(A \in \ms[n][n][\F]\), we define \(A^1 = A\), \(A^2 = AA\), \(A^3 = A^2 A\), and, in general, \(A^k = A^{k - 1} A\) for \(k = 2, 3, \dots\).
  We define \(A^0 = I_n\).
\end{defn}

\begin{eg}\label{2.3.7}
  If
  \[
    A = \begin{pmatrix}
      0 & 0 \\
      1 & 0
    \end{pmatrix},
  \]
  then \(A^2 = \zm\) (the zero matrix) even though \(A \neq \zm\).
  Thus the cancellation property for multiplication in fields is not valid for matrices.
  To see why, assume that the cancellation law is valid.
  Then, from \(A \cdot A = A^2 = \zm = A \cdot \zm\), we would conclude that \(A = \zm\), which is false.
\end{eg}

\begin{thm}\label{2.13}
  Let \(A \in \ms\) and \(B \in \ms[n][p][\F]\).
  For each \(j\) (\(1 \leq j \leq p\)) let \(u_j\) and \(v_j\) denote the \(j\)th columns of \(AB\) and \(B\), respectively.
  Then
  \begin{enumerate}
    \item \(u_j = A v_j\).
    \item \(v_j = B e_j\), where \(e_j\) is the \(j\)th standard vector of \(\vs{F}^p\).
  \end{enumerate}
\end{thm}

\begin{proof}[\pf{2.13}(a)]
  We have
  \begin{align*}
    u_j & = \begin{pmatrix}
              (AB)_{1 j} \\
              (AB)_{2 j} \\
              \vdots     \\
              (AB)_{m j}
            \end{pmatrix}                 &  & \by{2.2.4}   \\
        & = \begin{pmatrix}
              \sum_{k = 1}^n A_{1 k} B_{k j} \\
              \sum_{k = 1}^n A_{2 k} B_{k j} \\
              \vdots                         \\
              \sum_{k = 1}^n A_{m k} B_{k j}
            \end{pmatrix} &  & \by{2.3.1}                  \\
        & = A \begin{pmatrix}
                B_{1 j} \\
                B_{2 j} \\
                \vdots  \\
                B_{n j}
              \end{pmatrix}               &  & \by{2.3.1}   \\
        & = A v_j.                          &  & \by{2.2.4}
  \end{align*}
\end{proof}

\begin{proof}[\pf{2.13}(b)]
  We have
  \begin{align*}
    v_j & = \begin{pmatrix}
              B_{1 j} \\
              B_{2 j} \\
              \vdots  \\
              B_{m j}
            \end{pmatrix}                     &  & \by{2.2.4}   \\
        & = \begin{pmatrix}
              (B I_p)_{1 j} \\
              (B I_p)_{2 j} \\
              \vdots        \\
              (B I_p)_{m j}
            \end{pmatrix}                     &  & \by{2.12}[c] \\
        & = \begin{pmatrix}
              \sum_{k = 1}^p B_{1 k} (I_p)_{k j} \\
              \sum_{k = 1}^p B_{2 k} (I_p)_{k j} \\
              \vdots                             \\
              \sum_{k = 1}^p B_{m k} (I_p)_{k j}
            \end{pmatrix} &  & \by{2.3.1}                  \\
        & = B \begin{pmatrix}
                (I_p)_{1 j} \\
                (I_p)_{2 j} \\
                \vdots      \\
                (I_p)_{p j}
              \end{pmatrix}                   &  & \by{2.3.1}   \\
        & = B e_j.                              &  & \by{2.3.4}
  \end{align*}
\end{proof}

\begin{note}
  It follows (see \cref{ex:2.3.14}) from \cref{2.13} that column \(j\) of \(AB\) is a linear combination of the columns of \(A\) with the coefficients in the linear combination being the entries of column \(j\) of \(B\).
  An analogous result holds for rows;
  that is, row \(i\) of \(AB\) is a linear combination of the rows of \(B\) with the coefficients in the linear combination being the entries of row \(i\) of \(A\).
\end{note}

\begin{thm}\label{2.14}
  Let \(\V\) and \(\W\) be finite-dimensional vector spaces over \(\F\) having ordered bases \(\beta\) and \(\gamma\) over \(\F\), respectively, and let \(\T \in \ls(\V, \W)\).
  Then, for each \(u \in \V\), we have
  \[
    [\T(u)]_{\gamma} = [\T]_{\beta}^{\gamma} [u]_{\beta}.
  \]
\end{thm}

\begin{proof}[\pf{2.14}]
  Fix \(u \in \V\), and define the linear transformations \(f : \F \to \V\) by \(f(a) = au\) and \(g : \F \to \W\) by \(g(a) = a \T(u)\) for all \(a \in \F\).
  Let \(\alpha = \set{1}\) be the standard ordered basis for \(\F\).
  Notice that \(g = \T f\).
  Identifying column vectors as matrices and using \cref{2.11}, we obtain
  \[
    [\T(u)]_{\gamma} = [g(1)]_{\gamma} = [g]_{\alpha}^{\gamma} = [\T f]_{\alpha}^{\gamma} = [\T]_{\beta}^{\gamma} [f]_{\alpha}^{\beta} = [\T]_{\beta}^{\gamma} [f(1)]_{\beta} = [\T]_{\beta}^{\gamma} [u]_{\beta}.
  \]
\end{proof}

\begin{defn}\label{2.3.8}
  Let \(A \in \ms[m][n][\F]\).
  We denote by \(\L_A\) the mapping \(\L_A : \vs{F}^n \to \vs{F}^m\) defined by \(\L_A(x) = Ax\) (the matrix product of \(A\) and \(x\)) for each column vector \(x \in \vs{F}^n\).
  We call \(\L_A\) a \textbf{left-multiplication transformation}.
\end{defn}

\begin{thm}\label{2.15}
  Let \(A \in \ms\).
  Then the left-multiplication transformation \(\L_A : \vs{F}^n \to \vs{F}^m\) is linear.
  Furthermore, if \(B \in \ms\) and \(\beta\) and \(\gamma\) are the standard ordered bases for \(\vs{F}^n\) and \(\vs{F}^m\) over \(\F\), respectively, then we have the following properties.
  \begin{enumerate}
    \item \([\L_A]_{\beta}^{\gamma} = A\).
    \item \(\L_A = \L_B\) iff \(A = B\).
    \item \(\L_{A + B} = \L_A + \L_B\) and \(\L_{aA} = a \L_A\) for all \(a \in \F\).
    \item If \(\T : \vs{F}^n \to \vs{F}^m\) is linear, then there exists an unique \(C \in \ms\) such that \(\T = \L_C\).
          In fact, \(C = [\T]_{\beta}^{\gamma}\).
    \item If \(E \in \ms[n][p][\F]\), then \(\L_{AE} = \L_A \L_E\).
    \item If \(m = n\), then \(\L_{I_n} = \IT[\vs{F}^n]\).
  \end{enumerate}
\end{thm}

\begin{proof}[\pf{2.15}(a)]
  \(\L_A\) is linear follows immediately from \cref{2.12}(a)(b).
  By \cref{2.2.4} the \(j\)th column of \([\L_A]_{\beta}^{\gamma}\) is equal to \(\L_A(e_j)\).
  However by \cref{2.3.8} we have \(\L_A(e_j) = A e_j\), which is also the \(j\)th column of \(A\) by \cref{2.13}(b).
  So \([\L_A]_{\beta}^{\gamma} = A\).
\end{proof}

\begin{proof}[\pf{2.15}(b)]
  If \(\L_A = \L_B\), then we may use \cref{2.15}(a) to write \(A = [\L_A]_{\beta}^{\gamma} = [\L_B]_{\beta}^{\gamma} = B\).
  Hence \(A = B\).
  The proof of the converse is trivial.
\end{proof}

\begin{proof}[\pf{2.15}(c)]
  For all \(x \in \vs{F}^n\), we have
  \begin{align*}
    \L_{A + B}(x) & = (A + B) x         &  & \by{2.3.8}   \\
                  & = Ax + Bx           &  & \by{2.12}[a] \\
                  & = \L_A(x) + \L_B(x) &  & \by{2.3.8}   \\
                  & = (\L_A + \L_B)(x)  &  & \by{2.2.5}
  \end{align*}
  and
  \begin{align*}
    \L_{aA}(x) & = (aA) x       &  & \by{2.3.8}   \\
               & = a (Ax)       &  & \by{2.12}[b] \\
               & = a \L_A(x)    &  & \by{2.3.8}   \\
               & = (a \L_A)(x). &  & \by{2.2.5}
  \end{align*}
  Thus \(\L_{A + B} = \L_A + \L_B\) and \(\L_{aA} = a \L_A\).
\end{proof}

\begin{proof}[\pf{2.15}(d)]
  Let \(C = [\T]_{\beta}^{\gamma}\).
  By \cref{2.14} we have \([\T(x)]_{\gamma} = [\T]_{\beta}^{\gamma} [x]_{\beta}\), or \(\T(x) = Cx = \L_C(x)\) for all \(x \in \vs{F}^n\).
  So \(\T = \L_C\).
  The uniqueness of \(C\) follows from \cref{2.15}(b).
\end{proof}

\begin{proof}[\pf{2.15}(e)]
  For any \(j\) (\(1 \leq j \leq p\)), we may apply \cref{2.13} several times to note that \((AE) e_j\) is the \(j\)th column of \(AE\) and that the \(j\)th column of \(AE\) is also equal to \(A (E e_j)\).
  So \((AE) e_j = A (Ee_j)\).
  Thus
  \[
    \L_{AE}(e_j) = (AE) e_j = A (E e_j) = \L_A(E e_j) = \L_A(L_E(e_j)).
  \]
  Hence \(\L_{AE} = \L_A \L_E\) by \cref{2.1.13}.
\end{proof}

\begin{proof}[\pf{2.15}(f)]
  For all \(x \in \vs{F}^n\), we have
  \begin{align*}
    \L_{I_n}(x) & = I_n x              &  & \by{2.3.8}   \\
                & = x                  &  & \by{2.12}[c] \\
                & = \IT_{\vs{F}^n}(x). &  & \by{2.1.9}
  \end{align*}
  Thus \(\L_{I_n} = \IT_{\vs{F}^n}\).
\end{proof}

\begin{thm}\label{2.16}
  Let \(A\), \(B\), and \(C\) be matrices such that \(A (BC)\) is defined.
  Then \((AB) C\) is also defined and \(A (BC) = (AB) C\);
  that is, matrix multiplication is associative.
\end{thm}

\begin{proof}[\pf{2.16}]
  Since \(A (BC)\) is defined, by \cref{2.3.1} we can let \(A \in \ms\) and \(BC \in \ms[n][p][\F]\) such that \(A (BC) \in \ms[m][p][\F]\).
  Since \(BC\) is also defined, by \cref{2.3.1} again we can let \(B \in \ms[n][k][\F]\) and \(C \in \ms[k][p][\F]\).
  Then we have
  \begin{align*}
             & \begin{dcases}
                 A \in \ms           \\
                 B \in \ms[n][k][\F] \\
                 C \in \ms[k][p][\F]
               \end{dcases}                      \\
    \implies & \begin{dcases}
                 AB \in \ms[m][k][\F] \\
                 C \in \ms[k][p][\F]
               \end{dcases}  &  & \by{2.3.1}            \\
    \implies & (AB) C \in \ms[m][p][\F] &  & \by{2.3.1}
  \end{align*}
  and thus \((AB) C\) is defined.

  Using \cref{2.15}(e) and the associativity of functional composition, we have
  \[
    \L_{A (BC)} = \L_A \L_{BC} = \L_A (\L_B \L_C) = (\L_A \L_B) \L_C = \L_{AB} \L_C = \L_{(AB) C}.
  \]
  So by \cref{2.15}(b), it follows that \(A (BC) = (AB) C\).
\end{proof}

\begin{defn}\label{2.3.9}
  An \textbf{incidence matrix} is a square matrix in which all the entries are either zero or one and, for convenience, all the diagonal entries are zero.
  If we have a relationship on a set of \(n\) objects that we denote by \(1, 2, \dots, n\), then we define the associated incidence matrix \(A\) by \(A_{i j} = 1\) if \(i\) is related to \(j\), and \(A_{i j} = 0\) otherwise.
\end{defn}

\begin{eg}\label{2.3.10}
  A maximal collection of three or more people with the property that any two can send to each other is called a \textbf{clique}.
  The problem of determining cliques is difficult, but there is a simple method for determining if someone belongs to a clique.
  If we define a new matrix \(B\) by \(B_{i j} = 1\) if \(i\) and \(j\) can send to each other, and \(B_{i j} = 0\) otherwise, then person \(i\) belongs to a clique iff \((B^3)_{i i} > 0\).
\end{eg}

\begin{proof}[\pf{2.3.10}]
  Suppose that \(B \in \ms[n][n][\F]\).
  We know that person \(i\) belongs to a clique iff there exists at least two other people \(j, k\) such that \(i\) can send to \(j\), \(j\) can send to \(k\) and \(k\) can send back to \(i\).
  In other words \(B_{i j} = B_{j k} = B_{k i} = 1\).
  Thus we have
  \begin{align*}
         & \exists j, k \in \set{1, \dots, n} \setminus \set{i} : B_{i j} = B_{j k} = B_{k i} = 1           &  & \by{2.3.10} \\
    \iff & \exists j, k \in \set{1, \dots, n} \setminus \set{i} : B_{i j} B_{j k} B_{k i} = 1                                \\
    \iff & (B^3)_{i i} = (BBB)_{i i}                                                                        &  & \by{2.3.6}  \\
         & = \sum_{j = 1}^n B_{i j} (BB)_{j i} = \sum_{j = 1}^n B_{i j} \pa{\sum_{k = 1}^n B_{j k} B_{k i}} &  & \by{2.3.1}  \\
         & = \sum_{j = 1}^n \sum_{k = 1}^n (B_{i j} B_{j k} B_{k i}) > 0.                                   &  & \by{1.2.1}
  \end{align*}
\end{proof}

\begin{eg}\label{2.3.11}
  A relation among a group of people is called a \textbf{dominance relation} if the associated incidence matrix \(A\) has the property that for all distinct pairs \(i\) and \(j\), \(A_{i j} = 1\) iff \(A_{j i} = 0\), that is, given any two people, exactly one of them dominates the other.
  Since \(A\) is an incidence matrix, \(A_{i i} = 0\) for all \(i\).
  For such a relation, it can be shown that the matrix \(A + A^2\) has a row [column] in which each entry is positive except for the diagonal entry.
  In other words, there is at least one person who dominates [is dominated by] all others in one or two stages.
  In fact, it can be shown that any person who dominates [is dominated by] the greatest number of people in the first stage has this property.
\end{eg}

\begin{proof}[\pf{2.3.11}]
  Let \(A \in \ms[n][n][\F]\).
  First observe that
  \begin{align*}
    (A + A^2)_{i j} & = A_{i j} + (A^2)_{i j}                     &  & \by{1.2.9} \\
                    & = A_{i j} + (AA)_{i j}                      &  & \by{2.3.6} \\
                    & = A_{i j} + \sum_{k = 1}^n A_{i k} A_{k j}. &  & \by{2.3.1}
  \end{align*}
  We see that all diagonal entries of \(A + A^2\) is \(0\) since
  \begin{align*}
             & \forall (i, j) \in \set{1, \dots, n}^2, (A_{i j} = 0) \lor (A_{j i} = 0)      &  & \by{2.3.11}                   \\
    \implies & \forall (i, j) \in \set{1, \dots, n}^2, A_{i j} A_{j i} = 0                                                      \\
    \implies & \forall i \in \set{1, \dots, n}, \sum_{j = 1}^n A_{i j} A_{j i} = 0                                              \\
    \implies & \forall i \in \set{1, \dots, n}, A_{i i} + \sum_{j = 1}^n A_{i j} A_{j i} = 0 &  & \by{2.3.9}                    \\
    \implies & \forall i \in \set{1, \dots, n}, (A + A^2)_{i i} = 0.                         &  & \text{(from the proof above)}
  \end{align*}

  Next we show that there exists an \(i \in \set{1, \dots, n}\) such that each entry in the \(i\)-th row of \(A + A^2\) is positive except for the diagonal entry.
  Let \(v_1 \in \set{1, \dots, n}\).
  If each entry in the \(v_1\)-th row of \(A + A^2\) is positive except for the diagonal entry, then we are done.
  If not, then there exists a \(v_2 \in \set{1, \dots, n} \setminus \set{v_1}\) such that \((A + A^2)_{v_1 v_2} = 0\).
  From the proof above we see that \(A_{v_1 v_2} = 0\) and thus by \cref{2.3.11} we must have \(A_{v_2 v_1} = 1\).
  We claim that if \(k \in \set{1, \dots, n} \setminus \set{v_1, v_2}\) such that \(A_{v_1 k} = 1\), then \(A_{v_2 k} = 1\).
  Otherwise we would have
  \begin{align*}
             & A_{v_2 k} = 0                                               \\
    \implies & A_{k v_2} = 1            &  & \by{2.3.11}                   \\
    \implies & A_{v_1 k} A_{k v_2} = 1                                     \\
    \implies & (A + A^2)_{v_1 v_2} > 0, &  & \text{(from the proof above)}
  \end{align*}
  a contradiction.
  From the claim above we see that the \(v_2\)-th row of \(A + A^2\) has greater number of positive entries than the \(v_1\)-th row of \(A + A^2\) (at least larger than one since \(A_{v_2 v_1} = 1\)).
  Now we can ask if each entry in the \(v_2\)-th row of \(A + A^2\) is positive except for the diagonal entry.
  If yes, then we are done.
  If not, we can find a \(v_3 \in \set{1, \dots, n} \setminus \set{v_1, v_2}\) such that \((A + A^2)_{v_2 v_3} = 0\).
  Using similar arguments as above we see that the \(v_3\)-th row of \(A + A^2\) has greater number of positive entries than the \(v_2\)-th row of \(A + A^2\) (at least larger than one since \(A_{v_3 v_2} = 1\)).
  Since there are only finitely many rows in \(A + A^2\), we see that by continuing the above process we can find an \(i \in \set{1, \dots, n}\) such that each entry in the \(i\)-th row of \(A + A^2\) is positive except for the diagonal entry.

  Finally we show that if \(i\) is the person who dominates the most in \(A\), then each entry in the \(i\)-th row of \(A + A^2\) is positive except for the diagonal entry.
  Suppose for sake of contradiction that there exists a \(j \in \set{1, \dots, n} \setminus \set{i}\) such that \((A + A^2)_{i j} = 0\).
  This means
  \begin{align*}
             & A_{i j} + \sum_{k = 1}^n A_{i k} A_{k j} = 0                                              \\
    \implies & \begin{dcases}
                 A_{i j} = 0 \\
                 \sum_{k = 1}^n A_{i k} A_{k j} = 0
               \end{dcases}                                       &  & \by{2.3.11}                       \\
    \implies & \forall k \in \set{1, \dots, n},                                                          \\
             & (A_{i k} = 0 \land A_{k j} = 1) \lor (A_{i k} = 1 \land A_{k j} = 0)     &  & \by{2.3.9}  \\
    \implies & \forall k \in \set{1, \dots, n},                                                          \\
             & (A_{i k} = 0 \land A_{k j} = 1 \land A_{k i} = 1 \land A_{j k} = 0)                       \\
             & \lor (A_{i k} = 1 \land A_{k j} = 0 \land A_{k i} = 0 \land A_{j k} = 1) &  & \by{2.3.11} \\
    \implies & \sum_{k = 1}^n A_{i k} = \sum_{k = 1}^n A_{j k} > 0.                     &  & \by{2.3.11}
  \end{align*}
  But this contradicts to the fact that \(i\) is the person who dominates the most in \(A\).
  Thus such \(j\) does not exist and each entry in the \(i\)-th row of \(A + A^2\) is positive except for the diagonal entry.
  One can replace the role of \(i\)-th row with \(j\)-th column and derive similar arguments.
\end{proof}

\exercisesection

\setcounter{ex}{9}
\begin{ex}\label{ex:2.3.10}
  Let \(A \in \ms[n][n][\F]\).
  Prove that \(A\) is a diagonal matrix iff \(A_{i j} = \delta_{i j} A_{i j}\) for all \(i\) and \(j\).
\end{ex}

\begin{proof}[\pf{ex:2.3.10}]
  We have
  \begin{align*}
         & A \text{ is diagonal matrix}                                                                                \\
    \iff & A_{i j} = 0 \text{ for all } i, j \in \set{1, 2, \dots, n} \text{ and } i \neq j            &  & \by{1.3.8} \\
    \iff & A_{i j} = \delta_{i j} \text{ for all } i, j \in \set{1, 2, \dots, n} \text{ and } i \neq j &  & \by{2.3.4} \\
    \iff & A_{i j} = \delta_{i j} A_{i j} \text{ for all } i, j \in \set{1, 2, \dots, n}.              &  & \by{2.3.4}
  \end{align*}
\end{proof}

\begin{ex}\label{ex:2.3.11}
  Let \(\V\) be a vector space over \(\F\), and let \(\T : \V \to \V\) be linear.
  Prove that \(\T^2 = \zT\) iff \(\rg{\T} \subseteq \ns{\T}\).
\end{ex}

\begin{proof}[\pf{ex:2.3.11}]
  We have
  \begin{align*}
         & \T^2 = \zT                                                  \\
    \iff & \forall x \in \V, \T(\T(x)) = \zT(x) = \zv &  & \by{2.1.9}  \\
    \iff & \forall x \in \V, \T(x) \in \ns{\T}        &  & \by{2.1.10} \\
    \iff & \rg{\T} \subseteq \ns{\T}.                 &  & \by{2.1.10}
  \end{align*}
\end{proof}

\begin{ex}\label{ex:2.3.12}
  Let \(\V\), \(\W\), and \(\vs{Z}\) be vector spaces over \(\F\), and let \(\T \in \ls(\V, \W)\) and \(\U \in \ls(\W, \vs{Z})\).
  \begin{enumerate}
    \item Prove that if \(\U \T\) is one-to-one, then \(\T\) is one-to-one.
          Must \(\U\) also be one-to-one?
    \item Prove that if \(\U \T\) is onto, then \(\U\) is onto.
          Must \(\T\) also be onto?
    \item Prove that if \(\U\) and \(\T\) are one-to-one and onto, then \(\U \T\) is also.
  \end{enumerate}
\end{ex}

\begin{proof}[\pf{ex:2.3.12}(a)]
  Let \(x, y \in \V\) such that \(x \neq y\).
  Then we have
  \begin{align*}
             & (\U \T)(x) \neq (\U \T)(y) &  & \text{(\(\U \T\) is one-to-one)}            \\
    \implies & \U(\T(x)) \neq \U(\T(y))                                                    \\
    \implies & \T(x) \neq \T(y)           &  & \text{(this is the definition of function)} \\
    \implies & \T \text{ is one-to-one}.
  \end{align*}
  From the proof above we see that is doesn't matter whether \(\U\) is one-to-one or not.
\end{proof}

\begin{proof}[\pf{ex:2.3.12}(b)]
  Let \(z \in \vs{Z}\).
  Then we have
  \begin{align*}
             & \exists x \in \V : (\U \T)(x) = z                &  & \text{(\(\U \T\) is onto)} \\
    \implies & \exists (x, y) \in \V \times \W : \begin{dcases}
                                                   \T(x) = y \\
                                                   \U(\T(x)) = \U(y) = z
                                                 \end{dcases} &  & \text{(\(\U \T\) is onto)}   \\
    \implies & \U \text{ is onto}.
  \end{align*}
  From the proof above we see that is doesn't matter whether \(\T\) is onto or not.
\end{proof}

\begin{proof}[\pf{ex:2.3.12}(c)]
  First we show that \(\U \T\) is one-to-one.
  Let \(x, y \in \V\) such that \(x \neq y\).
  Then we have
  \begin{align*}
             & \T(x) \neq \T(y)             &  & \text{(\(\T\) is one-to-one)} \\
    \implies & \U(\T(x)) \neq \U(\T(y))     &  & \text{(\(\U\) is one-to-one)} \\
    \implies & \U \T \text{ is one-to-one}.
  \end{align*}

  Now we show that \(\U \T\) is onto.
  This is true since
  \begin{align*}
             & \begin{dcases}
                 \forall y \in \W, \exists x \in \V : \T(x) = y \\
                 \forall z \in \vs{Z}, \exists y \in \W : \U(y) = z
               \end{dcases}     &  & \text{(\(\U, \T\) are onto)}     \\
    \implies & \forall z \in \vs{Z}, \exists x \in \V : \U(\T(x)) = z \\
    \implies & \U \T \text{ is onto}.
  \end{align*}
\end{proof}

\begin{ex}\label{ex:2.3.13}
  Let \(A, B \in \ms[n][n][\F]\).
  Prove that \(\tr[AB] = \tr[BA]\) and \(\tr[A] = \tr[\tp{A}]\).
\end{ex}

\begin{proof}[\pf{ex:2.3.13}]
  We have
  \begin{align*}
    \tr[AB] & = \sum_{i = 1}^n (AB)_{i i}                          &  & \by{1.3.9} \\
            & = \sum_{i = 1}^n \pa{\sum_{k = 1}^n A_{i k} B_{k i}} &  & \by{2.3.1} \\
            & = \sum_{i = 1}^n \pa{\sum_{k = 1}^n B_{k i} A_{i k}} &  & \by{1.2.1} \\
            & = \sum_{k = 1}^n \pa{\sum_{i = 1}^n B_{k i} A_{i k}} &  & \by{1.2.1} \\
            & = \sum_{k = 1}^n (BA)_{k k}                          &  & \by{2.3.1} \\
            & = \tr[BA]                                            &  & \by{1.3.9}
  \end{align*}
  and
  \begin{align*}
    \tr[A] & = \sum_{i = 1}^n A_{i i}        &  & \by{1.3.9} \\
           & = \sum_{i = 1}^n (\tp{A})_{i i} &  & \by{1.3.3} \\
           & = \tr[\tp{A}].                  &  & \by{1.3.9}
  \end{align*}
\end{proof}

\begin{ex}\label{ex:2.3.14}
  Assume the notation in \cref{2.13}.
  \begin{enumerate}
    \item Suppose that \(z\) is a (column) vector in \(\vs{F}^p\).
          Use \cref{2.13}(b) to prove that \(Bz\) is a linear combination of the columns of \(B\).
          In particular, if \(z = \tp{\tuple{a}{1,,p}}\), then show that
          \[
            Bz = \sum_{j = 1}^p a_j v_j.
          \]
    \item Extend (a) to prove that column \(j\) of \(AB\) is a linear combination of the columns of \(A\) with the coefficients in the linear combination being the entries of column \(j\) of \(B\).
    \item For any row vector \(w \in \vs{F}^m\), prove that \(wA\) is a linear combination of the rows of \(A\) with the coefficients in the linear combination being the coordinates of \(w\).
    \item Prove the analogous result to (b) about rows:
          Row \(i\) of \(AB\) is a linear combination of the rows of \(B\) with the coefficients in the linear combination being the entries of row \(i\) of \(A\).
  \end{enumerate}
\end{ex}

\begin{proof}[\pf{ex:2.3.14}(a)]
  We have
  \begin{align*}
    Bz & = B \pa{\sum_{j = 1}^p a_j e_j} &  & \by{1.6.3} \\
       & = \sum_{j = 1}^p (a_j B e_j)    &  & \by{2.3.5} \\
       & = \sum_{j = 1}^p a_j v_j.       &  & \by{2.13}
  \end{align*}
\end{proof}

\begin{proof}[\pf{ex:2.3.14}(b)]
  We have
  \begin{align*}
    u_j & = A v_j = A \begin{pmatrix}
                        B_{1 j} \\
                        B_{2 j} \\
                        \vdots  \\
                        B_{n j}
                      \end{pmatrix}              &  & \by{2.13}[a]       \\
        & = \sum_{k = 1}^n B_{k j} \begin{pmatrix}
                                     A_{1 k} \\
                                     A_{2 k} \\
                                     \vdots  \\
                                     A_{n k}
                                   \end{pmatrix}. &  & \by{ex:2.3.14}[a]
  \end{align*}
\end{proof}

\begin{proof}[\pf{ex:2.3.14}(c)]
  We have
  \begin{align*}
    wA & = \begin{pmatrix}
             (wA)_{1 1} & (wA)_{1 2} & \cdots & (wA)_{1 n}
           \end{pmatrix}                                                             &  & \by{2.3.1}                                \\
       & = \begin{pmatrix}
             \sum_{k = 1}^m w_{1 k} A_{k 1} & \sum_{k = 1}^m w_{1 k} A_{k 2} & \cdots & \sum_{k = 1}^m w_{1 k} A_{k n}
           \end{pmatrix} &  & \by{2.3.1}                \\
       & = \sum_{k = 1}^m \begin{pmatrix}
                            w_{1 k} A_{k 1} & w_{1 k} A_{k 2} & \cdots & w_{1 k} A_{k n}
                          \end{pmatrix}                                              &  & \by{1.2.9}                                \\
       & = \sum_{k = 1}^m \begin{pmatrix}
                            w_k A_{k 1} & w_k A_{k 2} & \cdots & w_k A_{k n}
                          \end{pmatrix}                                                          &  & \by{2.3.1}                    \\
       & = \sum_{k = 1}^m \pa{w_k \begin{pmatrix}
                                      A_{k 1} & A_{k 2} & \cdots & A_{k n}
                                    \end{pmatrix}}.                                                                   &  & \by{1.2.9} \\
  \end{align*}
\end{proof}

\begin{proof}[\pf{ex:2.3.14}(d)]
  We have
  \begin{align*}
     & \begin{pmatrix}
         (AB)_{i 1} & (AB)_{i 2} & \cdots & (AB)_{i p}
       \end{pmatrix}                                                                              \\
     & = \begin{pmatrix}
           \sum_{k = 1}^n A_{i k} B_{k 1} & \sum_{k = 1}^n A_{i k} B_{k 2} & \cdots & \sum_{k = 1}^n A_{i k} B_{k p}
         \end{pmatrix} &  & \by{2.3.1}                \\
     & = \sum_{k = 1}^n \begin{pmatrix}
                          A_{i k} B_{k 1} & A_{i k} B_{k 2} & \cdots & A_{i k} B_{k p}
                        \end{pmatrix}                                              &  & \by{1.2.9}                                \\
     & = \sum_{k = 1}^n \pa{A_{i k} \begin{pmatrix}
                                        B_{k 1} & B_{k 2} & \cdots & B_{k p}
                                      \end{pmatrix}}.                                                               &  & \by{1.2.9}
  \end{align*}
\end{proof}

\begin{ex}\label{ex:2.3.15}
  Let \(M\) and \(A\) be matrices for which the product matrix \(MA\) is defined.
  If the \(j\)th column of \(A\) is a linear combination of a set of columns of \(A\), prove that the \(j\)th column of \(MA\) is a linear combination of the corresponding columns of \(MA\) with the same corresponding coefficients.
\end{ex}

\begin{proof}[\pf{ex:2.3.15}]
  Let \(M \in \ms\) and let \(A \in \ms[n][p][\F]\).
  For all \(i \in \set{1, 2, \dots, p}\) we define \(v_i\) to be the \(i\)th column of \(A\).
  By hypothesis we know that
  \[
    \exists \seq{c}{1,,p} \in \F : v_j = \sum_{i = 1}^p c_i v_i.
  \]
  Then we have
  \begin{align*}
    \begin{pmatrix}
      (MA)_{1 j} \\
      (MA)_{2 j} \\
      \vdots     \\
      (MA)_{p j}
    \end{pmatrix} & = M v_j = M \pa{\sum_{i = 1}^p c_i v_i} &  & \by{2.13}[a]  \\
                    & = \sum_{i = 1}^p c_i (M v_i)            &  & \by{2.3.5}  \\
                    & = \sum_{i = 1}^p c_i \begin{pmatrix}
                                             (MA)_{1 i} \\
                                             (MA)_{2 i} \\
                                             \vdots     \\
                                             (MA)_{p i}
                                           \end{pmatrix}.    &  & \by{2.13}[a]
  \end{align*}
\end{proof}

\begin{ex}\label{ex:2.3.16}
  Let \(V\) be a finite-dimensional vector space over \(\F\), and let \(\T \in \ls(\V, \V)\).
  \begin{enumerate}
    \item If \(\rk{\T} = \rk{\T^2}\), prove that \(\rg{\T} \cap \ns{\T} = \set{\zv}\).
          Deduce that \(\V = \rg{\T} \oplus \ns{\T}\).
    \item Prove that \(\V = \rg{\T^k} \oplus \ns{\T^k}\) for some positive integer \(k\).
  \end{enumerate}
\end{ex}

\begin{proof}[\pf{ex:2.3.16}(a)]
  First observe that
  \begin{align*}
             & \begin{dcases}
                 \T(\V) \subseteq \V \\
                 \rk{\T^2} = \rk{\T}
               \end{dcases}                                                                               \\
    \implies & \begin{dcases}
                 \rg{\T^2} = \T^{2}(\V) = \T(\T(\V)) \subseteq \T(\V) = \rg{\T} \\
                 \rk{\T^2} = \rk{\T}
               \end{dcases} &  & \by{2.1.10}                   \\
    \implies & \rg{\T^2} = \rg{\T}.                                                               &  & \by{1.11}
  \end{align*}
  Let \(\beta = \set{\seq{v}{1,,n}}\) be a basis for \(\rg{\T}\) over \(\F\).
  Since \(\rg{\T^2} = \rg{\T}\), by \cref{2.2} we know that \(\T(\beta)\) is also a basis for \(\rg{\T}\) over \(\F\).
  Let \(x \in \rg{\T} \cap \ns{\T}\).
  Since \(x \in \rg{\T}\), by \cref{1.6.1} we know that
  \[
    \exists \seq{a}{1,,n} \in \F : x = \sum_{i = 1}^n a_i v_i.
  \]
  By fixing \(\seq{a}{1,,n}\) we see that
  \begin{align*}
             & x \in \ns{\T}                                                                       \\
    \implies & \T(x) = \zv                                                      &  & \by{2.1.10}   \\
    \implies & \T\pa{\sum_{i = 1}^n a_i v_i} = \sum_{i = 1}^n a_i \T(v_i) = \zv &  & \by{2.1.2}[d] \\
    \implies & \seq[=]{a}{1,,n} = 0                                             &  & \by{1.5.3}    \\
    \implies & x = \zv.                                                         &  & \by{1.2}[a]
  \end{align*}
  Since \(x\) is arbitrary, we conclude that \(\rg{\T} \cap \ns{\T} = \set{\zv}\).
  Since
  \begin{align*}
     & \dim(\rg{\T} + \ns{\T})                                                          \\
     & = \dim(\rg{\T}) + \dim(\ns{\T}) - \dim(\rg{\T} \cap \ns{\T}) &  & \by{ex:1.6.29} \\
     & = \dim(\rg{\T}) + \dim(\ns{\T})                              &  & \by{1.6.9}     \\
     & = \rk{\T} + \nt{\T} = \dim(\V)                               &  & \by{2.3}
  \end{align*}
  and by \cref{ex:1.3.23}(a) \(\rg{\T} + \ns{\T}\) is a subspace of \(\V\) over \(\F\), by \cref{1.11} we know that \(\V = \rg{\T} + \ns{\T}\).
  Thus by \cref{1.3.11} we know that \(\V = \rg{\T} \oplus \ns{\T}\).
\end{proof}

\begin{proof}[\pf{ex:2.3.16}(b)]
  First observe that
  \begin{align*}
             & \forall k \in \Z^+, \rg{\T^{k + 1}} = \T^{k + 1}(\V) \subseteq \T^k(\V) = \rg{\T^k} &  & \by{2.1.10}   \\
    \implies & \forall k \in \Z^+, 0 \leq \rk{\T^{k + 1}} \leq \rk{\T^k} \leq \dim(\V).            &  & \by{1.11,2.3}
  \end{align*}
  Since \(\V\) is finite-dimensional, we know that there must exists a \(k \in \Z^+\) such that \(\rk{\T^{k + 1}} = \rk{\T^k}\).
  By \cref{ex:2.3.16}(a) we see that \(\rg{\T^k} \cap \ns{\T^k} = \set{\zv}\) and \(\V = \rg{\T^k} \oplus \ns{\T^k}\).
\end{proof}

\begin{ex}\label{ex:2.3.17}
  Let \(\V\) be a vector space over \(\F\).
  Determine all linear transformations \(\T : \V \to \V\) such that \(\T = \T^2\).
\end{ex}

\begin{proof}[\pf{ex:2.3.17}]
  First observe that
  \[
    \forall x \in \V, x = \T(x) + (x - \T(x)).
  \]
  Suppose that \(\T = \T^2\).
  Then we have
  \begin{align*}
             & \forall x \in \V, \T(x) = \T(\T(x))                                                                       \\
    \implies & \forall x \in \V, \T(x - \T(x)) = \T(x) - \T(\T(x)) = \zv                          &  & \by{2.1.2}[c]     \\
    \implies & \forall x \in \V, \begin{dcases}
                                   \T(x) \in \set{y \in \V : \T(y) = y} \\
                                   x - \T(x) \in \ns{\T}
                                 \end{dcases}                                            &  & \by{2.1.10}                \\
    \implies & \forall x \in \V, x = \T(x) + (x - \T(x)) \in \set{y \in \V : \T(y) = y} + \ns{\T} &  & \by{1.3.10}       \\
    \implies & \V \subseteq \set{y \in \V : \T(y) = y} + \ns{\T}                                                         \\
    \implies & \V = \set{y \in \V : \T(y) = y} + \ns{\T}                                          &  & \by{ex:1.3.23}[a]
  \end{align*}
  and
  \begin{align*}
             & \forall x \in \set{y \in \V : \T(y) = y} \cap \ns{\T}, \begin{dcases}
                                                                        \T(x) = x \\
                                                                        \T(x) = \zv
                                                                      \end{dcases} &  & \by{2.1.10}         \\
    \implies & \forall x \in \set{y \in \V : \T(y) = y} \cap \ns{\T}, x = \zv                               \\
    \implies & \set{y \in \V : \T(y) = y} \cap \ns{\T} = \set{\zv}                                          \\
    \implies & \V = \set{y \in \V : \T(y) = y} \oplus \ns{\T}.                       &  & \by{ex:2.3.16}[a]
  \end{align*}
  By \cref{ex:2.1.26} we see that \(\T\) is a projection.
\end{proof}

\begin{ex}\label{ex:2.3.18}
  Using only the definition of matrix multiplication, prove that multipli- cation of matrices is associative.
\end{ex}

\begin{proof}[\pf{ex:2.3.18}]
  Let \(A \in \ms\), \(B \in \ms[n][p][\F]\) and \(C \in \ms[p][q][\F]\).
  Then we have
  \begin{align*}
    ((AB) C)_{i j} & = \sum_{k_2 = 1}^p (AB)_{i k_2} C_{k_2 j}                                     &  & \by{2.3.1} \\
                   & = \sum_{k_2 = 1}^p \pa{\pa{\sum_{k_1 = 1}^n A_{i k_1} B_{k_1 k_2}} C_{k_2 j}} &  & \by{2.3.1} \\
                   & = \sum_{k_2 = 1}^p \pa{\sum_{k_1 = 1}^n (A_{i k_1} B_{k_1 k_2}) C_{k_2 j}}    &  & \by{1.2.1} \\
                   & = \sum_{k_2 = 1}^p \pa{\sum_{k_1 = 1}^n A_{i k_1} (B_{k_1 k_2} C_{k_2 j})}    &  & \by{1.2.1} \\
                   & = \sum_{k_1 = 1}^n \pa{\sum_{k_2 = 1}^p A_{i k_1} (B_{k_1 k_2} C_{k_2 j})}    &  & \by{1.2.1} \\
                   & = \sum_{k_1 = 1}^n \pa{A_{i k_1} \pa{\sum_{k_2 = 1}^p B_{k_1 k_2} C_{k_2 j}}} &  & \by{1.2.1} \\
                   & = \sum_{k_1 = 1}^n A_{i k_1} (BC)_{k_1 j}                                     &  & \by{2.3.1} \\
                   & = (A (BC))_{i j}                                                              &  & \by{2.3.1}
  \end{align*}
  and thus by \cref{1.2.9} \((AB) C = A (BC)\).
\end{proof}

\setcounter{ex}{22}
\begin{ex}\label{ex:2.3.23}
  Let \(A \in \ms[n][n][\F]\) be an incidence matrix that corresponds to a dominance relation.
  Determine the number of nonzero entries of \(A\).
\end{ex}

\begin{proof}[\pf{ex:2.3.23}]
  By \cref{2.3.11} we see that the number of nonzero entries is
  \[
    \dfrac{n (n + 1)}{2} - n = \dfrac{n (n - 1)}{2}.
  \]
\end{proof}

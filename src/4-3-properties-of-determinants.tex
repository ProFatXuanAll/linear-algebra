\section{Properties of Determinants}\label{sec:4.3}

\begin{thm}\label{4.7}
  For any \(A, B \in \ms{n}{n}{\F}\), \(\det(AB) = \det(A) \cdot \det(B)\).
\end{thm}

\begin{proof}[\pf{4.7}]
  We begin by establishing the result when \(A\) is an elementary matrix.
  If \(A\) is an elementary matrix obtained by interchanging two rows of \(I_n\), then by \cref{ex:4.2.28} we have \(\det(A) = -1\).
  But by \cref{3.1}, \(AB\) is a matrix obtained by interchanging two rows of \(B\).
  Hence by \cref{4.5}, \(\det(AB) = -\det(B) = \det(A) \cdot \det(B)\).
  Similar arguments establish the result when \(A\) is an elementary matrix of type 2 or type 3.
  (See \cref{ex:4.3.18}.)

  If \(\rk{A} < n\), then \(\det(A) = 0\) by \cref{4.2.7}.
  Since \(\rk{AB} \leq \rk{A} < n\) by \cref{3.7}, we have \(\det(AB) = 0\).
  Thus \(\det(AB) = \det(A) \cdot \det(B)\) in this case.

  On the other hand, if \(\rk{A} = n\), then \(A\) is invertible and hence is the product of elementary matrices (\cref{3.2.6}), say, \(A = \seq[]{E}{m,,1}\).
  The first paragraph of this proof shows that
  \begin{align*}
    \det(AB) & = \det(\seq[]{E}{m,,1} B)                     \\
             & = \det(E_m) \cdot \det(\seq[]{E}{m - 1,,1} B) \\
             & \vdots                                        \\
             & = \det(E_m) \cdots \det(E_1) \cdot \det(B)    \\
             & = \det(\seq[]{E}{m,,1}) \cdot \det(B)         \\
             & = \det(A) \cdot \det(B).
  \end{align*}
\end{proof}

\begin{note}
  The determinant is a \emph{multiplicative} function.
\end{note}

\begin{cor}\label{4.3.1}
  A matrix \(A \in \ms{n}{n}{\F}\) is invertible iff \(\det(A) \neq 0\).
  Furthermore, if \(A\) is invertible, then \(\det(A^{-1}) = \frac{1}{\det(A)}\).
\end{cor}

\begin{proof}[\pf{4.3.1}]
  If \(A \in \ms{n}{n}{\F}\) is not invertible, then by \cref{3.2.2} the rank of \(A\) is less than \(n\).
  So \(\det(A) = 0\) by \cref{4.2.7}.
  On the other hand, if \(A \in \ms{n}{n}{\F}\) is invertible, then
  \[
    \det(A) \cdot \det(A^{-1}) = \det(A A^{-1}) = \det(I_n) = 1
  \]
  by \cref{4.7}.
  Hence \(\det(A) \neq 0\) and \(\det(A^{-1}) = \frac{1}{\det(A)}\).
\end{proof}

\begin{thm}\label{4.8}
  For any \(A \in \ms{n}{n}{\F}\), \(\det(\tp{A}) = \det(A)\).
\end{thm}

\begin{proof}[\pf{4.8}]
  If \(A\) is not invertible, then \(\rk{A} < n\).
  But \(\rk{\tp{A}} = \rk{A}\) by \cref{3.2.5}(a), and so \(\tp{A}\) is not invertible.
  Thus \(\det(\tp{A}) = 0 = \det(A)\) in this case.

  On the other hand, if \(A\) is invertible, then \(A\) is a product of elementary matrices, say \(A = \seq[]{E}{m,,1}\).
  This means
  \begin{align*}
    \det(\tp{A}) & = \det(\tp{E_1} \cdots \tp{E_m})       &  & \text{(by \cref{2.3.2})}     \\
                 & = \det(\tp{E_1}) \cdots \det(\tp{E_m}) &  & \text{(by \cref{4.7})}       \\
                 & = \det(E_1) \cdots \det(E_m)           &  & \text{(by \cref{ex:4.2.29})} \\
                 & = \det(E_m) \cdots \det(E_1)           &  & (\det(\tp{E_i}) \in \F)      \\
                 & = \det(E_m \cdots E_1)                 &  & \text{(by \cref{4.7})}       \\
                 & = \det(A).
  \end{align*}
  Thus, in either case, \(\det(\tp{A}) = \det(A)\).
\end{proof}

\begin{note}
  In our discussion of determinants until now, we have used only the rows of a matrix.
  For example, the recursive definition of a determinant involved cofactor expansion along a row, and the more efficient method developed in \cref{sec:4.2} used elementary row operations.
  \cref{4.8} shows that the determinants of \(A\) and \(\tp{A}\) are always equal.
  Since the rows of \(A\) are the columns of \(\tp{A}\), this fact enables us to translate any statement about determinants that involves the rows of a matrix into a corresponding statement that involves its columns.
\end{note}

\begin{thm}[Cramer's Rule]\label{4.9}
  Let \(Ax = b\) be the matrix form of a system of \(n\) linear equations in \(n\) unknowns, where \(x = \tp{\tuple{x}{1,,n}}\).
  If \(\det(A) \neq 0\), then this system has a unique solution, and for each \(k \in \set{1, \dots, n}\),
  \[
    x_k = \frac{\det(M_k)}{\det(A)},
  \]
  where \(M_k \in \ms{n}{n}{\F}\) is obtained from \(A\) by replacing column \(k\) of \(A\) by \(b\).
\end{thm}

\begin{proof}[\pf{4.9}]
  If \(\det(A) \neq 0\), then the system \(Ax = b\) has a unique solution by the \cref{4.3.1,3.10}.
  For each integer \(k \in \set{1, \dots, n}\), let \(a_k\) denote the \(k\)th column of \(A\) and \(X_k\) denote the matrix obtained from the \(I_n\) by replacing column \(k\) by \(x\).
  Then by \cref{2.13}, \(A X_k\) is the \(n \times n\) matrix whose \(i\)th column is
  \[
    A e_i = a_i \text{ if } i \neq k \quad \text{and} \quad Ax = b \text{ if } i = k.
  \]
  Thus \(A X_k = M_k\).
  Evaluating \(X_k\) by cofactor expansion along row \(k\) produces
  \[
    \det(X_k) = x_k \cdot \det(I_{n - 1}) = x_k.
  \]
  Hence by \cref{4.7},
  \[
    \det(M_k) = \det(A X_k) = \det(A) \cdot \det(X_k) = \det(A) \cdot x_k.
  \]
  Therefore
  \[
    x_k = \pa{\det(A)}^{-1} \cdot \det(M_k).
  \]
\end{proof}

\begin{note}
  In applications involving systems of linear equations, we sometimes need to know that there is a solution in which the unknowns are integers.
  In this situation, Cramer's rule can be useful because it implies that a system of linear equations with integral coefficients has an integral solution if the determinant of its coefficient matrix is \(\pm 1\).
  On the other hand, Cramer's rule is not useful for computation because it requires evaluating \(n + 1\) determinants of \(n \times n\) matrices to solve a system of \(n\) linear equations in \(n\) unknowns.
  The amount of computation to do this is far greater than that required to solve the system by the method of Gaussian-Jordan elimination, which was discussed in \cref{sec:3.4}.
  Thus Cramer's rule is primarily of theoretical and aesthetic interest, rather than of computational value.
\end{note}

\begin{note}
  As in \cref{sec:4.1}, it is possible to interpret the determinant of a matrix \(A \in \ms{n}{n}{\F}\) geometrically.
  If the rows of \(A\) are \(\seq{a}{1,,n}\), respectively, then \(\abs{\det(A)}\) is the \textbf{\(n\)-dimensional volume} (the generalization of area in \(\R^2\) and volume in \(\R^3\)) of the parallelepiped having the vectors \(\seq{a}{1,,n}\) as adjacent sides.
\end{note}

\begin{note}
  In our earlier discussion of the geometric significance of the determinant formed from the vectors in an ordered basis for \(\R^2\), we also saw that this determinant is positive iff the basis induces a right-handed coordinate system.
  A similar statement is true in \(\R^n\).
  Specifically, if \(\gamma\) is any ordered basis for \(\R^n\) and \(\beta\) is the standard ordered basis for \(\R^n\), then \(\gamma\) induces a \emph{right-handed coordinate system} iff \(\det(Q) > 0\), where \(Q\) is the change of coordinate matrix changing \(\gamma\)-coordinates into \(\beta\)-coordinates, i.e., \(Q = [I_n]_{\gamma}^{\beta}\)
  (see \cref{2.5.1}).
  More generally, if \(\beta\) and \(\gamma\) are two ordered bases for \(\R^n\), then the coordinate systems induced by \(\beta\) and \(\gamma\) have the same \textbf{orientation} (either both are right-handed or both are left-handed) iff \(\det(Q) > 0\), where \(Q\) is the change of coordinate matrix changing \(\gamma\)-coordinates into \(\beta\)-coordinates.
\end{note}

\exercisesection

\setcounter{ex}{7}
\begin{ex}\label{ex:4.3.8}
  Use \cref{4.8} to prove a result analogous to \cref{4.3}, but for columns.
\end{ex}

\begin{proof}[\pf{ex:4.3.8}]
  Let \(A \in \ms{n}{n}{\F}\) and let \(\seq{a}{1,,n}\) be columns of \(A\).
  Let \(u, v \in \vs{F}^n\) and let \(k \in \F\).
  If \(a_r = u + kv\) for some \(r \in \set{1, \dots, n}\), then we have
  \begin{align*}
    \det(A) & = \det\begin{pmatrix}
                      a_1 & \cdots & a_{r - 1} & u + kv & a_{r + 1} & \cdots & a_n
                    \end{pmatrix}                            \\
            & = \det\begin{pmatrix}
                      \tp{a_1}          \\
                      \vdots            \\
                      \tp{a_{r - 1}}    \\
                      \tp{u} + k \tp{v} \\
                      \tp{a_{r + 1}}    \\
                      \vdots            \\
                      \tp{a_n}
                    \end{pmatrix}                      &  & \text{(by \cref{4.8})}                          \\
            & = \det\begin{pmatrix}
                      \tp{a_1}       \\
                      \vdots         \\
                      \tp{a_{r - 1}} \\
                      \tp{u}         \\
                      \tp{a_{r + 1}} \\
                      \vdots         \\
                      \tp{a_n}
                    \end{pmatrix} + k \det\begin{pmatrix}
                                            \tp{a_1}       \\
                                            \vdots         \\
                                            \tp{a_{r - 1}} \\
                                            \tp{v}         \\
                                            \tp{a_{r + 1}} \\
                                            \vdots         \\
                                            \tp{a_n}
                                          \end{pmatrix}                         &  & \text{(by \cref{4.3})} \\
            & = \det\begin{pmatrix}
                      a_1 & \cdots & a_{r - 1} & u & a_{r + 1} & \cdots & a_n
                    \end{pmatrix}                                 \\
            & \quad + k \det\begin{pmatrix}
                              a_1 & \cdots & a_{r - 1} & v & a_{r + 1} & \cdots & a_n
                            \end{pmatrix}.      &  & \text{(by \cref{4.8})}
  \end{align*}
\end{proof}

\begin{ex}\label{ex:4.3.9}
  Prove that an upper triangular matrix \(A \in \ms{n}{n}{\F}\) is invertible iff all its diagonal entries are nonzero.
\end{ex}

\begin{proof}[\pf{ex:4.3.9}]
  We have
  \begin{align*}
         & A \text{ is invertible}                                                            \\
    \iff & \det(A) \neq 0                                   &  & \text{(by \cref{4.3.1})}     \\
    \iff & \prod_{j = 1}^n A_{j j} \neq 0                   &  & \text{(by \cref{ex:4.2.23})} \\
    \iff & \forall j \in \set{1, \dots, n}, A_{j j} \neq 0.
  \end{align*}
\end{proof}

\begin{ex}\label{ex:4.3.18}

\end{ex}

\chapter{Polynomial}\label{ch:e}

\begin{defn}\label{e.0.1}
	A polynomial \(f\) \textbf{divides} a polynomial \(g\) if there exists a polynomial \(q\) such that \(g(x) = f(x) q(x)\).
\end{defn}

\begin{thm}[The Division Algorithm for Polynomials]\label{e.1}
	Let \(f \in \ps[n]{\F}\) and let \(g \in \ps[m]{\F}\).
	Then there exist unique polynomials \(q\) and \(r\) such that
	\begin{equation}\label{eq:e.0.1}
		\forall x \in \F, f(x) = q(x) g(x) + r(x),
	\end{equation}
	where \(r \in \ps[m - 1]{\F}\).
\end{thm}

\begin{proof}[\pf{e.1}]
	We begin by establishing the existence of \(q\) and \(r\) that satisfy \cref{eq:e.0.1}.
	\begin{description}
		\item[Case 1.]
			If \(n < m\), take \(q = \zv\) and \(r = f\) to satisfy \cref{eq:e.0.1}.
		\item[Case 2.]
			When \(0 \leq m \leq n\), we apply mathematical induction on \(n\).
			First suppose that \(n = 0\).
			Then \(m = 0\), and it follows that \(f\) and \(g\) are nonzero constants.
			Hence we may take \(q = f / g\) and \(r = \zv\) to satisfy \cref{eq:e.0.1}.

			Now suppose that the result is valid for all polynomials with degree less than \(n\) for some fixed \(n > 0\), and assume that \(f\) has degree \(n\).
			Suppose that
			\[
				\forall x \in \F, f(x) = a_n x^n + a_{n - 1} x^{n - 1} + \cdots + a_1 x + a_0
			\]
			and
			\[
				\forall x \in \F, g(x) = b_m x^m + b_{m - 1} x^{m - 1} + \cdots + b_1 x + b_0
			\]
			and let \(h\) be the polynomial defined by
			\begin{equation}\label{eq:e.0.2}
				\forall x \in \F, h(x) = f(x) - a_n b_m^{-1} x^{n - m} g(x).
			\end{equation}
			Then \(h\) is a polynomial of degree less than \(n\), and therefore we may apply the induction hypothesis or Case 1 (whichever is relevant) to obtain polynomials \(q_1\) and \(r\) such that \(r\) has degree less than \(m\) and
			\begin{equation}\label{eq:e.0.3}
				\forall x \in \F, h(x) = q_1(x) g(x) + r(x).
			\end{equation}
			Combining \cref{eq:e.0.2,eq:e.0.3} and solving for \(f\) gives us \(f = q g + r\) with
			\[
				\forall x \in \F, q(x) = a_n b_m^{-1} x^{n - m} + q_1(x),
			\]
			which establishes Case I and Case II for any \(n \geq 0\) by mathematical induction.
			This establishes the existence of \(q\) and \(r\).

			We now show the uniqueness of \(q\) and \(r\).
			Suppose that \(q_1, q_2, r_1\), and \(r_2\) exist such that \(r_1\) and \(r_2\) each has degree less than \(m\) and
			\[
				\forall x \in \F, f(x) = q_1(x) g(x) + r_1(x) = q_2(x) g(x) + r_2(x).
			\]
			Then
			\begin{equation}\label{eq:e.0.4}
				\forall x \in \F, (q_1(x) - q_2(x)) g(x) = r_2(x) - r_1(x).
			\end{equation}
			The right side of \cref{eq:e.0.4} is a polynomial of degree less than \(m\).
			Since \(g\) has degree \(m\), it must follow that \(q_1 - q_2\) is the zero polynomial.
			Hence \(q_1 = q_2\);
			thus \(r_1 = r_2\) by \cref{eq:e.0.4}.
	\end{description}
\end{proof}

\begin{defn}\label{e.0.2}
	In the context of \cref{e.1}, we call \(q\) and \(r\) the \textbf{quotient} and \textbf{remainder}, respectively, for the division of \(f\) by \(g\).
\end{defn}

\begin{cor}\label{e.0.3}
	Let \(f\) be a polynomial of positive degree, and let \(a \in \F\).
	Then \(f(a) = 0\) iff \(x - a\) divides \(f\).
\end{cor}

\begin{proof}[\pf{e.0.3}]
	Suppose that \(x - a\) divides \(f\).
	Then there exists a polynomial \(q\) such that \(f(x) = (x - a) q(x)\) for all \(x \in \F\).
	Thus \(f(a) = (a - a) q(a) = 0 \cdot q(a) = 0\).

	Conversely, suppose that \(f(a) = 0\).
	By the division algorithm, there exist polynomials \(q\) and \(r\) such that \(r\) has degree less than one and
	\[
		\forall x \in \F, f(x) = q(x) (x - a) + r(x).
	\]
	Substituting \(a\) for \(x\) in the equation above, we obtain \(r(a) = 0\).
	Since \(r\) has degree less than \(1\), it must be the constant polynomial \(r = \zv\).
	Thus \(f(x) = q(x) (x - a)\).
\end{proof}

\begin{defn}\label{e.0.4}
	For any polynomial \(f\) with coefficients from a field \(\F\), an element \(a \in \F\) is called a \textbf{zero} of \(f\) if \(f(a) = 0\).
	With this terminology, \cref{e.0.3} states that \(a\) is a zero of \(f\) iff \(x - a\) divides \(f\).
\end{defn}

\begin{cor}\label{e.0.5}
	Any polynomial of degree \(n \geq 1\) has at most \(n\) distinct zeros.
\end{cor}

\begin{proof}[\pf{e.0.5}]
	The proof is by mathematical induction on \(n\).
	The result is obvious if \(n = 1\).
	Now suppose that the result is true for some positive integer \(n\), and let \(f\) be a polynomial of degree \(n + 1\).
	If \(f\) has no zeros, then there is nothing to prove.
	Otherwise, if \(a\) is a zero of \(f\), then by \cref{e.0.5} we may write \(f(x) = (x - a) q(x)\) for some polynomial \(q\).
	Note that \(q\) must be of degree \(n\);
	therefore, by the induction hypothesis, \(q\) can have at most \(n\) distinct zeros.
	Since any zero of \(q\) distinct from \(a\) is also a zero of \(f\), it follows that \(f\) can have at most \(n + 1\) distinct zeros.
\end{proof}

\begin{defn}\label{e.0.6}
	Two nonzero polynomials are called \textbf{relatively prime} if no polynomial of positive degree divides each of them.
\end{defn}

\begin{thm}\label{e.2}
	If \(f_1\) and \(f_2\) are relatively prime polynomials over \(\F\), there exist polynomials \(q_1\) and \(q_2\) such that
	\[
		\forall x \in \F, q_1(x) f_1(x) + q_2(x) f_2(x) = 1,
	\]
	where \(1 \in \F\).
\end{thm}

\begin{proof}[\pf{e.2}]
	Without loss of generality, assume that the degree of \(f_1\) is greater than or equal to the degree of \(f_2\).
	The proof is by mathematical induction on the degree of \(f_2\).
	If \(f_2\) has degree \(0\), then \(f_2\) is a nonzero constant \(c\).
	In this case, we can take \(q_1 = \zv\) and \(q_2 = 1 / c\).

	Now suppose that the theorem holds whenever the polynomial of lesser degree has degree less than \(n\) for some positive integer \(n\), and suppose that \(f_2\) has degree \(n\).
	By the division algorithm, there exist polynomials \(q\) and \(r\) such that \(r\) has degree less than \(n\) and
	\begin{equation}\label{eq:e.0.5}
		\forall x \in \F, f_1(x) = q(x) f_2(x) + r(x).
	\end{equation}
	Since \(f_1\) and \(f_2\) are relatively prime, \(r\) is not the zero polynomial.
	We claim that \(f_2\) and \(r\) are relatively prime.
	Suppose otherwise;
	then there exists a polynomial \(g\) of positive degree that divides both \(f_2\) and \(r\).
	Hence, by (5), \(g\) also divides \(f_1\), contradicting the fact that \(f_1\) and \(f_2\) are relatively prime.
	Since \(r\) has degree less than \(n\), we may apply the induction hypothesis to \(f_2\) and \(r\).
	Thus there exist polynomials \(g_1\) and \(g_2\) such that
	\begin{equation}\label{eq:e.0.6}
		\forall x \in \F, g_1(x) f_2(x) + g_2(x) r(x) = 1.
	\end{equation}
	Combining \cref{eq:e.0.5,eq:e.0.6}, we have
	\begin{align*}
		1 & = g_1(x) f_2(x) + g_2(x) (f_1(x) - q(x) f_2(x))  \\
		  & = g_2(x) f_1(x) + (g_1(x) - g_2(x) q(x)) f_2(x).
	\end{align*}
	Thus, setting \(q_1 = g_2\) and \(q_2 = g_1 - g_2 q\), we obtain the desired result.
\end{proof}

\begin{defn}\label{e.0.7}
	Let
	\[
		f(x) = a_0 + a_1 x + \cdots + a_n x^n
	\]
	be a polynomial with coefficients from a field \(\F\).
	If \(\T\) is a linear operator on a vector space \(\V\) over \(\F\), we define
	\[
		f(\T) = a_0 \IT[\V] + a_1 \T + \cdots + a_n \T^n.
	\]
	Similarly, if \(A \in \ms[n][n][\F]\), we define
	\[
		f(A) = a_0 I_n + a_1 A + \cdots + a_n A^n.
	\]
\end{defn}

\begin{thm}\label{e.3}
	Let \(f\) be a polynomial with coefficients from a field \(\F\), and let \(\T\) be a linear operator on a vector space \(\V\) over \(\F\).
	Then the following statements are true.
	\begin{enumerate}
		\item \(f(\T)\) is a linear operator on \(\V\).
		\item If \(\beta\) is a finite ordered basis for \(\V\) over \(\F\) and \(A = [\T]_{\beta}\), then \([f(\T)]_{\beta} = f(A)\).
	\end{enumerate}
\end{thm}

\begin{proof}[\pf{e.3}]
	Let \(f(x) = a_0 + a_1 x + \cdots + a_n x^n\).
	By \cref{e.0.7} we have
	\[
		f(\T) = a_0 \IT[\V] + a_1 \T + \cdots + a_n \T^n \quad \text{and} \quad f(A) = a_0 I_n + a_1 A + \cdots + a_n A^n.
	\]
	By \cref{2.7,2.9} we see that \(f(\T) \in \ls(\V)\).
	Suppose that \(A = [\T]_{\beta} \in \ms[n][n][\F]\).
	Then we have
	\begin{align*}
		[f(\T)]_{\beta} & = [a_0 \IT[\V] + a_1 \T + \cdots + a_n \T^n]_{\beta}                       &  & \by{e.0.7}   \\
		                & = a_0 [\IT[\V]]_{\beta} + a_1 [\T]_{\beta} + \cdots + a_n [\T^n]_{\beta}   &  & \by{2.8}     \\
		                & = a_0 [\IT[\V]]_{\beta} + a_1 [\T]_{\beta} + \cdots + a_n ([\T]_{\beta})^n &  & \by{2.11}    \\
		                & = a_0 I_n + a_1 A + \cdots + a_n A^n                                       &  & \by{2.12}[d] \\
		                & = f(A).                                                                    &  & \by{e.0.7}
	\end{align*}
\end{proof}

\begin{thm}\label{e.4}
	Let \(\T\) be a linear operator on a vector space \(\V\) over field \(\F\), and let \(A \in \ms[n][n][\F]\).
	Then, for any polynomials \(f_1\) and \(f_2\) with coefficients from \(\F\),
	\begin{enumerate}
		\item \(f_1(\T) f_2(\T) = f_2(\T) f_1(\T)\);
		\item \(f_1(A) f_2(A) = f_2(A) f_1(A)\).
	\end{enumerate}
\end{thm}

\begin{proof}[\pf{e.4}(a)]
	Let \(f_1(x) = a_0 + a_1 x + \cdots + a_n x^n\) and \(f_2(x) = b_0 + b_1 x + \cdots + b_m x^m\).
	Since
	\begin{align*}
		\forall i \in \set{0, \dots, n}, & (a_i \T^i) f_2(\T)                                                                  \\
		                                 & = (a_i \T^i) (b_0 \IT[\V] + b_1 \T + \cdots + b_m \T^m)       &  & \by{e.0.7}       \\
		                                 & = b_0 a_i \T^i + b_1 a_i \T^{1 + i} + \cdots + b_m \T^{m + i} &  & \by{2.10}[a,c,d] \\
		                                 & = (b_0 \IT[\V] + b_1 \T + \cdots + b_m \T^m) (a_i \T^i)       &  & \by{2.10}[b,c]   \\
		                                 & = f_2(\T) (a_i \T^i),                                         &  & \by{e.0.7}
	\end{align*}
	we have
	\begin{align*}
		f_1(\T) f_2(\T) & = (a_0 \IT[\V] + a_1 \T + \cdots + a_n \T^n) f_2(\T)                     &  & \by{e.0.7}                    \\
		                & = (a_0 \IT[\V]) f_2(\T) + (a_1 \T) f_2(\T) + \cdots + (a_n \T^n) f_2(\T) &  & \by{2.10}[a,d]                \\
		                & = f_2(\T) (a_0 \IT[\V]) + f_2(\T) (a_1 \T) + \cdots + f_2(\T) (a_n \T^n) &  & \text{(from the proof above)} \\
		                & = f_2(\T) (a_0 \IT[\V] + a_1 \T + \cdots + a_n \T^n)                     &  & \by{2.10}[a,d]                \\
		                & = f_2(\T) f_1(\T).                                                       &  & \by{e.0.7}
	\end{align*}
\end{proof}

\begin{proof}[\pf{e.4}(b)]
	Let \(\beta\) be the standard ordered basis for \(\vs{F}^n\) over \(\F\).
	Then we have
	\begin{align*}
		f_1(A) f_2(A) & = f_1([\L_A]_{\beta}) f_2([\L_A]_{\beta}) &  & \by{2.15}[a] \\
		              & = [f_1(\L_A)]_{\beta} [f_2(\L_A)]_{\beta} &  & \by{e.3}[b]  \\
		              & = [f_1(\L_A) f_2(\L_A)]_{\beta}           &  & \by{2.3.3}   \\
		              & = [f_2(\L_A) f_1(\L_A)]_{\beta}           &  & \by{e.4}[a]  \\
		              & = [f_2(\L_A)]_{\beta} [f_1(\L_A)]_{\beta} &  & \by{2.3.3}   \\
		              & = f_2([\L_A]_{\beta}) f_1([\L_A]_{\beta}) &  & \by{2.15}[a] \\
		              & = f_2(A) f_1(A).                          &  & \by{e.3}[b]
	\end{align*}
\end{proof}

\begin{thm}\label{e.5}
	Let \(\T\) be a linear operator on a vector space \(\V\) over field \(\F\), and let \(A \in \ms[n][n][\F]\).
	If \(f_1\) and \(f_2\) are relatively prime polynomials with entries from \(\F\), then there exist polynomials \(q_1\) and \(q_2\) with entries from \(\F\) such that
	\begin{enumerate}
		\item \(q_1(\T) f_1(\T) + q_2(\T) f_2(\T) = \IT[\V]\);
		\item \(q_1(A) f_1(A) + q_2(A) f_2(A) = I_n\).
	\end{enumerate}
\end{thm}

\begin{proof}[\pf{e.5}]
	By \cref{e.2} there exist polynomials \(q_1\) and \(q_2\) such that \(q_1(x) f_1(x) + q_2(x) f_2(x) = 1\).
	Thus by \cref{e.0.7} we have \(q_1(\T) f_1(\T) + q_2(\T) f_2(\T) = \IT[\V]\) and \(q_1(A) f_1(A) + q_2(A) f_2(A) = I_n\).
\end{proof}

\begin{defn}\label{e.0.8}
	A polynomial \(f\) with coefficients from a field \(\F\) is called \textbf{monic} if its leading coefficient is \(1\).
	If \(f\) has positive degree and cannot be expressed as a product of polynomials with coefficients from \(\F\) each having positive degree, then \(f\) is called \textbf{irreducible}.

	Observe that whether a polynomial is irreducible depends on the field \(\F\) from which its coefficients come.
	Clearly any polynomial of degree \(1\) is irreducible.
	Moreover, for polynomials with coefficients from an algebraically closed field, the polynomials of degree \(1\) are the only irreducible polynomials.
\end{defn}

\begin{thm}\label{e.6}
	Let \(\phi\) and \(f\) be polynomials.
	If \(\phi\) is irreducible and \(\phi\) does not divide \(f\), then \(\phi\) and \(f\) are relatively prime.
\end{thm}

\begin{proof}[\pf{e.6}]
	Suppose for sake of contradiction that \(\phi\) and \(f\) are not relatively prime.
	Then there exists a polynomial \(g\) such that \(g\) divides \(\phi\) and \(f\).
	Since \(\phi\) is irreducible, by \cref{e.0.8} we must have \(\phi = cg\) for some \(c \in \F\).
	Since \(g\) divides \(f\), we know that \(\phi\) divides \(f\).
	But this contradict to the fact that \(\phi\) does not divide \(f\).
	Thus \(\phi\) and \(f\) are relatively prime.
\end{proof}

\begin{thm}\label{e.7}
	Any two distinct irreducible monic polynomials are relatively prime.
\end{thm}

\begin{proof}[\pf{e.7}]
	Let \(\phi\) and \(f\) be distinct irreducible monic polynomials.
	Since \(\phi\) and \(f\) are irreducible and monic, by \cref{e.0.8} we know that \(\phi\) does not divide \(f\).
	Thus by \cref{e.6} \(\phi\) and \(f\) are relatively prime.
\end{proof}

\begin{thm}\label{e.8}
	Let \(f, g\), and \(\phi\) be polynomials.
	If \(\phi\) is irreducible and divides the product \(fg\), then \(\phi\) divides \(f\) or \(\phi\) divides \(g\).
\end{thm}

\begin{proof}[\pf{e.8}]
	Suppose that \(\phi\) does not divide \(f\).
	Then \(\phi\) and \(f\) are relatively prime by \cref{e.6}, and so there exist polynomials \(q_1\) and \(q_2\) such that
	\[
		1 = q_1(x) \phi(x) + q_2(x) f(x).
	\]
	Multiplying both sides of this equation by \(g\) yields
	\begin{equation}\label{eq:e.0.7}
		g(x) = q_1(x) \phi(x) g(x) + q_2(x) f(x) g(x).
	\end{equation}
	Since \(\phi\) divides \(fg\), there is a polynomial \(h\) such that \(fg = \phi h\).
	Thus \cref{eq:e.0.7} becomes
	\[
		g(x) = q_1(x) \phi(x) g(x) + q_2(x) \phi(x) h(x) = \phi(x) (q_1(x) g(x) + q_2(x) h(x)).
	\]
	So \(\phi\) divides \(g\).
\end{proof}

\begin{cor}\label{e.0.9}
	Let \(\phi, \seq{\phi}{1,,n}\) be irreducible monic polynomials.
	If \(\phi\) divides the product \(\seq[]{\phi}{1,,n}\), then \(\phi = \phi_i\) for some \(i \in \set{1, 2, . . . , n}\).
\end{cor}

\begin{proof}[\pf{e.0.9}]
	We prove the corollary by mathematical induction on \(n\).
	For \(n = 1\), the result is an immediate consequence of \cref{e.7}.
	Suppose then that for some \(n > 1\), the corollary is true for any \(n - 1\) irreducible monic polynomials, and let \(\seq{\phi}{1,,n}\) be \(n\) irreducible polynomials.
	If \(\phi\) divides
	\[
		\seq[]{\phi}{1,,n} = (\seq[]{\phi}{1,,n-1}) \phi_n,
	\]
	then \(\phi\) divides the product \(\seq[]{\phi}{1,,n-1}\) or \(\phi\) divides \(\phi_n\) by \cref{e.8}.
	In the first case, \(\phi = \phi_i\) for some \(i \in \set{1, \dots, n - 1}\) by the induction hypothesis;
	in the second case, \(\phi = \phi_n\) by \cref{e.7}.
\end{proof}

\begin{thm}[Unique Factorization Theorem for Polynomials]\label{e.9}
	For any polynomial \(f\) of positive degree, there exist a unique constant \(c\);
	unique distinct irreducible monic polynomials \(\seq{\phi}{1,,k}\);
	and unique positive integers \(\seq{n}{1,,k}\) such that
	\[
		f(x) = c (\phi_1(x))^{n_1} \cdots (\phi_k(x))^{n_k}.
	\]
\end{thm}

\begin{proof}[\pf{e.9}]
	We begin by showing the existence of such a factorization using mathematical induction on the degree of \(f\).
	If \(f\) is of degree \(1\), then \(f(x) = ax + b\) for some constants \(a\) and \(b\) with \(a \neq 0\).
	Setting \(\phi(x) = x + b / a\), we have \(f = a \phi\).
	Since \(\phi\) is an irreducible monic polynomial, the result is proved in this case.
	Now suppose that the conclusion is true for any polynomial with positive degree less than some integer \(n > 1\), and let \(f\) be a polynomial of degree \(n\).
	Then
	\[
		f(x) = a_n x^n + \cdots + a_1 x + a_0
	\]
	for some constants \(a_i\) with \(a_n \neq 0\).
	If \(f\) is irreducible, then
	\[
		f(x) = a_n \pa{x^n + \dfrac{a_{n - 1}}{a_n} x^{n - 1} + \cdots + \dfrac{a_1}{a_n} x + \dfrac{a_0}{a_n}}
	\]
	is a representation of \(f\) as a product of \(a_n\) and an irreducible monic polynomial.
	If \(f\) is not irreducible, then \(f = gh\) for some polynomials \(g\) and \(h\), each of positive degree less than \(n\).
	The induction hypothesis guarantees that both \(g\) and \(h\) factor as products of a constant and powers of distinct irreducible monic polynomials.
	Consequently \(f = gh\) also factors in this way.
	Thus, in either case, \(f\) can be factored as a product of a constant and powers of distinct irreducible monic polynomials.

	It remains to establish the uniqueness of such a factorization.
	Suppose that
	\begin{equation}\label{eq:e.0.8}
		\begin{aligned}
			f(x) & = c (\phi_1(x))^{n_1} \cdots (\phi_k(x))^{n_k}  \\
			     & = d (\psi_1(x))^{m_1} \cdots (\psi_r(x))^{m_r},
		\end{aligned}
	\end{equation}
	where \(c\) and \(d\) are constants, \(\phi_i\) and \(\psi_j\) are irreducible monic polynomials, and \(n_i\) and \(m_j\) are positive integers for \(i \in \set{1, \dots, k}\) and \(j \in \set{1, \dots, r}\).
	Clearly both \(c\) and \(d\) must be the leading coefficient of \(f\);
	hence \(c = d\).
	Dividing by \(c\), we find that \cref{eq:e.0.8} becomes
	\begin{equation}\label{eq:e.0.9}
		(\phi_1)^{n_1} \cdots (\phi_k)^{n_k} = (\psi_1)^{m_1} \cdots (\psi_r)^{m_r}.
	\end{equation}
	So \(\phi_i\) divides the right side of \cref{eq:e.0.9} for \(i \in \set{1, \dots, k}\).
	Consequently, by \cref{e.0.9}, each \(\phi_i\) equals some \(\psi_j\), and similarly, each \(\psi_j\) equals some \(\phi_i\).
	We conclude that \(r = k\) and that, by renumbering if necessary, \(\phi_i = \psi_i\) for \(i \in \set{1, \dots, k}\).
	Suppose that \(n_i \neq m_i\) for some \(i \in \set{1, \dots, k}\).
	Without loss of generality, we may suppose that \(i = 1\) and \(n_1 > m_1\).
	Then by canceling \((\phi_1)^{m_1}\) from both sides of \cref{eq:e.0.9}, we obtain
	\begin{equation}\label{eq:e.0.10}
		(\phi_1)^{n_1 - m_1} (\phi_2)^{n_2} \cdots (\phi_k)^{n_k} = (\phi_2)^{m_2} \cdots (\phi_k)^{m_k}.
	\end{equation}
	Since \(n_1 - m_1 > 0\), \(\phi_1\) divides the left side of \cref{eq:e.0.10} and hence divides the right side also.
	So \(\phi_1 = \phi_i\) for some \(i \in \set{2, \dots, k}\) by \cref{e.0.9}.
	But this contradicts that \(\seq{\phi}{1,,k}\) are distinct.
	Hence the factorizations of \(f\) in \cref{eq:e.0.8} are the same.
\end{proof}

\begin{note}
	It is often useful to regard a polynomial \(f(x) = a_n x^n + \cdots + a_1 x + a_0\) with coefficients from a field \(\F\) as a function \(f : \F \to \F\).
	In this case, the value of \(f\) at \(c \in \F\) is \(f(c) = a_n c^n + \cdots + a_1 c + a_0\).
	Unfortunately, for arbitrary fields there is not a one-to-one correspondence between polynomials and polynomial functions.
	For example, if \(f(x) = x^2\) and \(g(x) = x\) are two polynomials over the field \(\Z^2\) (defined in \cref{c.0.4}), then \(f\) and \(g\) have different degrees and hence are not equal as polynomials.
	But \(f(a) = g(a)\) for all \(a \in \Z^2\), so that \(f\) and \(g\) are equal polynomial functions.
	Our final result shows that this anomaly cannot occur over an infinite field.
\end{note}

\begin{thm}\label{e.10}
	Let \(f\) and \(g\) be polynomials with coefficients from an infinite field \(\F\).
	If \(f(a) = g(a)\) for all \(a \in \F\), then \(f\) and \(g\) are equal.
\end{thm}

\begin{proof}[\pf{e.10}]
	Suppose that \(f(a) = g(a)\) for all \(a \in \F\).
	Define \(h = f - g\), and suppose that \(h\) is of degree \(n \geq 1\).
	It follows from \cref{e.0.5} that \(h\) can have at most \(n\) zeroes.
	But
	\[
		h(a) = f(a) - g(a) = 0
	\]
	for every \(a \in \F\), contradicting the assumption that \(h\) has positive degree.
	Thus \(h\) is a constant polynomial, and since \(h(a) = 0\) for each \(a \in \F\), it follows that \(h\) is the zero polynomial.
	Hence \(f = g\).
\end{proof}

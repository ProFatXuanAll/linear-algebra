\chapter{Polynomial}\label{ch:e}

\begin{defn}\label{e.0.1}
	A polynomial \(f\) \textbf{divides} a polynomial \(g\) if there exists a polynomial \(q\) such that \(g(x) = f(x) q(x)\).
\end{defn}

\begin{thm}[The Division Algorithm for Polynomials]\label{e.1}
	Let \(f \in \ps[n]{\F}\) and let \(g \in \ps[m]{\F}\).
	Then there exist unique polynomials \(q\) and \(r\) such that
	\begin{equation}\label{eq:e.0.1}
		\forall x \in \F, f(x) = q(x) g(x) + r(x),
	\end{equation}
	where \(r \in \ps[m - 1]{\F}\).
\end{thm}

\begin{proof}[\pf{e.1}]
	We begin by establishing the existence of \(q\) and \(r\) that satisfy \cref{eq:e.0.1}.
	\begin{description}
		\item[Case 1.]
			If \(n < m\), take \(q = \zv\) and \(r = f\) to satisfy \cref{eq:e.0.1}.
		\item[Case 2.]
			When \(0 \leq m \leq n\), we apply mathematical induction on \(n\).
			First suppose that \(n = 0\).
			Then \(m = 0\), and it follows that \(f\) and \(g\) are nonzero constants.
			Hence we may take \(q = f / g\) and \(r = \zv\) to satisfy \cref{eq:e.0.1}.

			Now suppose that the result is valid for all polynomials with degree less than \(n\) for some fixed \(n > 0\), and assume that \(f\) has degree \(n\).
			Suppose that
			\[
				\forall x \in \F, f(x) = a_n x^n + a_{n - 1} x^{n - 1} + \cdots + a_1 x + a_0
			\]
			and
			\[
				\forall x \in \F, g(x) = b_m x^m + b_{m - 1} x^{m - 1} + \cdots + b_1 x + b_0
			\]
			and let \(h\) be the polynomial defined by
			\begin{equation}\label{eq:e.0.2}
				\forall x \in \F, h(x) = f(x) - a_n b_m^{-1} x^{n - m} g(x).
			\end{equation}
			Then \(h\) is a polynomial of degree less than \(n\), and therefore we may apply the induction hypothesis or Case 1 (whichever is relevant) to obtain polynomials \(q_1\) and \(r\) such that \(r\) has degree less than \(m\) and
			\begin{equation}\label{eq:e.0.3}
				\forall x \in \F, h(x) = q_1(x) g(x) + r(x).
			\end{equation}
			Combining \cref{eq:e.0.2,eq:e.0.3} and solving for \(f\) gives us \(f = q g + r\) with
			\[
				\forall x \in \F, q(x) = a_n b_m^{-1} x^{n - m} + q_1(x),
			\]
			which establishes Case I and Case II for any \(n \geq 0\) by mathematical induction.
			This establishes the existence of \(q\) and \(r\).

			We now show the uniqueness of \(q\) and \(r\).
			Suppose that \(q_1, q_2, r_1\), and \(r_2\) exist such that \(r_1\) and \(r_2\) each has degree less than \(m\) and
			\[
				\forall x \in \F, f(x) = q_1(x) g(x) + r_1(x) = q_2(x) g(x) + r_2(x).
			\]
			Then
			\begin{equation}\label{eq:e.0.4}
				\forall x \in \F, (q_1(x) - q_2(x)) g(x) = r_2(x) - r_1(x).
			\end{equation}
			The right side of \cref{eq:e.0.4} is a polynomial of degree less than \(m\).
			Since \(g\) has degree \(m\), it must follow that \(q_1 - q_2\) is the zero polynomial.
			Hence \(q_1 = q_2\);
			thus \(r_1 = r_2\) by \cref{eq:e.0.4}.
	\end{description}
\end{proof}

\begin{defn}\label{e.0.2}
	In the context of \cref{e.1}, we call \(q\) and \(r\) the \textbf{quotient} and \textbf{remainder}, respectively, for the division of \(f\) by \(g\).
\end{defn}

\begin{cor}\label{e.0.3}
	Let \(f\) be a polynomial of positive degree, and let \(a \in \F\).
	Then \(f(a) = 0\) iff \(x - a\) divides \(f\).
\end{cor}

\begin{proof}[\pf{e.0.3}]
	Suppose that \(x - a\) divides \(f\).
	Then there exists a polynomial \(q\) such that \(f(x) = (x - a) q(x)\) for all \(x \in \F\).
	Thus \(f(a) = (a - a) q(a) = 0 \cdot q(a) = 0\).

	Conversely, suppose that \(f(a) = 0\).
	By the division algorithm, there exist polynomials \(q\) and \(r\) such that \(r\) has degree less than one and
	\[
		\forall x \in \F, f(x) = q(x) (x - a) + r(x).
	\]
	Substituting \(a\) for \(x\) in the equation above, we obtain \(r(a) = 0\).
	Since \(r\) has degree less than \(1\), it must be the constant polynomial \(r = \zv\).
	Thus \(f(x) = q(x) (x - a)\).
\end{proof}

\begin{defn}\label{e.0.4}
	For any polynomial \(f\) with coefficients from a field \(\F\), an element \(a \in \F\) is called a \textbf{zero} of \(f\) if \(f(a) = 0\).
	With this terminology, \cref{e.0.3} states that \(a\) is a zero of \(f\) iff \(x - a\) divides \(f\).
\end{defn}

\begin{cor}\label{e.0.5}
	Any polynomial of degree \(n \geq 1\) has at most \(n\) distinct zeros.
\end{cor}

\begin{proof}[\pf{e.0.5}]
	The proof is by mathematical induction on \(n\).
	The result is obvious if \(n = 1\).
	Now suppose that the result is true for some positive integer \(n\), and let \(f\) be a polynomial of degree \(n + 1\).
	If \(f\) has no zeros, then there is nothing to prove.
	Otherwise, if \(a\) is a zero of \(f\), then by \cref{e.0.5} we may write \(f(x) = (x - a) q(x)\) for some polynomial \(q\).
	Note that \(q\) must be of degree \(n\);
	therefore, by the induction hypothesis, \(q\) can have at most \(n\) distinct zeros.
	Since any zero of \(q\) distinct from \(a\) is also a zero of \(f\), it follows that \(f\) can have at most \(n + 1\) distinct zeros.
\end{proof}

\begin{defn}\label{e.0.6}
	Two nonzero polynomials are called \textbf{relatively prime} if no polynomial of positive degree divides each of them.
\end{defn}

\begin{thm}\label{e.2}
	If \(f_1\) and \(f_2\) are relatively prime polynomials over \(\F\), there exist polynomials \(q_1\) and \(q_2\) such that
	\[
		\forall x \in \F, q_1(x) f_1(x) + q_2(x) f_2(x) = 1,
	\]
	where \(1 \in \F\).
\end{thm}

\begin{proof}[\pf{e.2}]
	Without loss of generality, assume that the degree of \(f_1\) is greater than or equal to the degree of \(f_2\).
	The proof is by mathematical induction on the degree of \(f_2\).
	If \(f_2\) has degree \(0\), then \(f_2\) is a nonzero constant \(c\).
	In this case, we can take \(q_1 = \zv\) and \(q_2 = 1 / c\).

	Now suppose that the theorem holds whenever the polynomial of lesser degree has degree less than \(n\) for some positive integer \(n\), and suppose that \(f_2\) has degree \(n\).
	By the division algorithm, there exist polynomials \(q\) and \(r\) such that \(r\) has degree less than \(n\) and
	\begin{equation}\label{eq:e.0.5}
		\forall x \in \F, f_1(x) = q(x) f_2(x) + r(x).
	\end{equation}
	Since \(f_1\) and \(f_2\) are relatively prime, \(r\) is not the zero polynomial.
	We claim that \(f_2\) and \(r\) are relatively prime.
	Suppose otherwise;
	then there exists a polynomial \(g\) of positive degree that divides both \(f_2\) and \(r\).
	Hence, by (5), \(g\) also divides \(f_1\), contradicting the fact that \(f_1\) and \(f_2\) are relatively prime.
	Since \(r\) has degree less than \(n\), we may apply the induction hypothesis to \(f_2\) and \(r\).
	Thus there exist polynomials \(g_1\) and \(g_2\) such that
	\begin{equation}\label{eq:e.0.6}
		\forall x \in \F, g_1(x) f_2(x) + g_2(x) r(x) = 1.
	\end{equation}
	Combining \cref{eq:e.0.5,eq:e.0.6}, we have
	\begin{align*}
		1 & = g_1(x) f_2(x) + g_2(x) (f_1(x) - q(x) f_2(x))  \\
		  & = g_2(x) f_1(x) + (g_1(x) - g_2(x) q(x)) f_2(x).
	\end{align*}
	Thus, setting \(q_1 = g_2\) and \(q_2 = g_1 - g_2 q\), we obtain the desired result.
\end{proof}

\begin{defn}\label{e.0.7}
	Let
	\[
		f(x) = a_0 + a_1(x) + \cdots + a_n x^n
	\]
	be a polynomial with coefficients from a field \(\F\).
	If \(\T\) is a linear operator on a vector space \(\V\) over \(\F\), we define
	\[
		f(\T) = a_0 \IT + a_1 \T + \cdots + a_n \T^n.
	\]
	Similarly, if \(A \in \ms[n][n][\F]\), we define
	\[
		f(A) = a_0 I + a_1 A + \cdots + a_n A^n.
	\]
\end{defn}

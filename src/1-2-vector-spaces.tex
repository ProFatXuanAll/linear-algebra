\section{Vector Spaces}\label{sec:1.2}

\begin{defn}\label{1.2.1}
	A \textbf{vector space} (or \textbf{linear space}) \(\V\) over a field \(\F\) consists of a set on which two operations (called \textbf{addition} and \textbf{scalar multiplication}, respectively) are defined so that for each pair of elements \(x\), \(y\) in \(\V\) there is a unique element \(x + y\) in \(\V\), and for each element \(a\) in \(\F\) and each element \(x\) in \(\V\) there is a unique element \(ax\) in \(\V\), such that the following conditions hold.
	\begin{enumerate}[label=(VS \arabic*), ref=VS \arabic*]
		\item\label{vs1}
		For all \(x, y\) in \(\V\), \(x + y = y + x\)
		(commutativity of addition).
		\item\label{vs2}
		For all \(x, y, z\) in \(\V\), \((x + y) + z = x + (y + z)\)
		(associativity of addition).
		\item\label{vs3}
		There exists an element in \(\V\) denoted by \(\zv\) such that \(x + \zv = x\) for each \(x\) in \(\V\).
		\item\label{vs4}
		For each element \(x\) in \(\V\) there exists an element \(y\) in \(\V\) such that \(x + y = \zv\).
		\item\label{vs5}
		For each element \(x\) in \(\V\), \(1x = x\).
		\item\label{vs6}
		For each pair of elements \(a, b\) in \(\F\) and each element \(x\) in \(\V\), \((ab) x = a (bx)\).
		\item\label{vs7}
		For each element \(a\) in \(\F\) and each pair of elements \(x, y\) in \(\V\), \(a (x + y) = ax + ay\).
		\item\label{vs8}
		For each pair of elements \(a, b\) in \(\F\) and each element \(x\) in \(\V\), \((a + b) x = ax + bx\).
	\end{enumerate}
	The elements \(x + y\) and \(ax\) are called the \textbf{sum} of \(x\) and \(y\) and the \textbf{product} of \(a\) and \(x\), respectively.
\end{defn}

\begin{defn}\label{1.2.2}
	The elements of the field \(\F\) are called \textbf{scalars} and the elements of the vector space \(\V\) are called \textbf{vectors}.
\end{defn}

\begin{note}
	A vector space is frequently discussed in the text without explicitly mentioning its field of scalars.
	The reader is cautioned to remember, however, that \emph{every vector space is regarded as a vector space over a given field, which is denoted by \(\F\)}.
	Occasionally we restrict our attention to the fields of real and complex numbers, which are denoted \(\R\) and \(\C\), respectively.
\end{note}

\begin{note}
	\ref{vs2} permits us to unambiguously define the addition of any finite number of vectors
	(without the use of parentheses).
\end{note}

\begin{defn}\label{1.2.3}
	An object of the form \(\tuple{a}{1,2,,n}\), where the entries \(\seq{a}{1,2,,n}\) are elements of a field \(\F\), is called an \textbf{\(n\)-tuple} with entries from \(\F\).
	The elements \(\seq{a}{1,2,,n}\) are called the \textbf{entries} or \textbf{components} of the \(n\)-tuple.
	Two \(n\)-tuples \(\tuple{a}{1,2,,n}\) and \(\tuple{b}{1,2,,n}\) with entries from a field \(\F\) are called \textbf{equal} if \(a_i = b_i\) for \(i = 1, 2, \dots, n\).
\end{defn}

\begin{eg}\label{1.2.4}
	The set of all \(n\)-tuples with entries from a field \(\F\) is denoted by \(\vs{F}^n\).
	This set is a vector space over \(\F\) with the operations of coordinatewise addition and scalar multiplication;
	that is, if \(u = \tuple{a}{1,2,,n} \in \vs{F}^n\), \(v = \tuple{b}{1,2,,n} \in \vs{F}^n\), and \(c \in \F\), then
	\[
		u + v = (a_1 + b_1, a_2 + b_2, \dots, a_n + b_n) \quad \text{and} \quad cu = \tuple{ca}{1,2,,n}.
	\]
\end{eg}

\begin{proof}[\pf{1.2.4}]
	Clearly we have
	\[
		\forall u, v \in \vs{F}^n, u + v \in \vs{F}^n
	\]
	and
	\[
		\begin{dcases}
			\forall u \in \vs{F}^n \\
			\forall c \in \F
		\end{dcases}, cu \in \vs{F}^n.
	\]

	Let \(I_n = \set{i \in \N : 1 \leq i \leq n}\).
	First we show that addition in \(\vs{F}^n\) over \(\F\) is unique.
	Let \(u, v \in \vs{F}^n\).
	Suppose for sake of contradiction that
	\[
		\begin{dcases}
			u + v = w \in \vs{F}^n  \\
			u + v = w' \in \vs{F}^n \\
			w \neq w'
		\end{dcases}.
	\]
	Then we have
	\begin{align*}
		         & w \neq w'                                                    \\
		\implies & \exists i \in I_n : w_i \neq w_i'            &  & \by{1.2.3} \\
		\implies & \exists i \in I_n : u_i + v_i \neq u_i + v_i &  & \by{1.2.4}
	\end{align*}
	which contradicts to the fact that \(u_i + v_i \in \F\) is unique.
	Thus addition in \(\vs{F}^n\) over \(\F\) is unique.

	Next we show that scalar multiplication in \(\vs{F}^n\) over \(\F\) is unique.
	Let \(u \in \vs{F}^n\) and \(c \in \F\).
	Suppose for sake of contradiction that
	\[
		\begin{dcases}
			cu = w \in \vs{F}^n  \\
			cu = w' \in \vs{F}^n \\
			w \neq w'
		\end{dcases}.
	\]
	Then we have
	\begin{align*}
		         & w \neq w'                                            \\
		\implies & \exists i \in I_n : w_i \neq w_i'    &  & \by{1.2.3} \\
		\implies & \exists i \in I_n : c u_i \neq c u_i &  & \by{1.2.4}
	\end{align*}
	which contradicts to the fact that \(c u_i \in \F\) is unique.
	Thus scalar multiplication in \(\vs{F}^n\) over \(\F\) is unique.

	Now we show that \ref{vs1}--\ref{vs8} holds for \cref{1.2.4}.
	\begin{description}
		\item[For \ref{vs1}:]
			For all \(u, v \in \vs{F}^n\), we have
			\begin{align*}
				u + v & = (u_1 + v_1, u_2 + v_2, \dots, u_n + v_n) &  & \by{1.2.4}                 \\
				      & = (v_1 + u_1, v_2 + u_2, \dots, v_n + u_n) &  & \text{(\(\F\) is a field)} \\
				      & = v + u.                                   &  & \by{1.2.4}
			\end{align*}
		\item[For \ref{vs2}:]
			For all \(u, v, w \in \vs{F}^n\), we have
			\begin{align*}
				 & (u + v) + w                                                                                        \\
				 & = (u_1 + v_1, u_2 + v_2, \dots, u_n + v_n) + w                     &  & \by{1.2.4}                 \\
				 & = ((u_1 + v_1) + w_1, (u_2 + v_2) + w_2, \dots, (u_n + v_n) + w_n) &  & \by{1.2.4}                 \\
				 & = (u_1 + (v_1 + w_1), u_2 + (v_2 + w_2), \dots, u_n + (v_n + w_n)) &  & \text{(\(\F\) is a field)} \\
				 & = u + (v_1 + w_1, v_2 + w_2, \dots, v_n + w_n)                     &  & \by{1.2.4}                 \\
				 & = u + (v + w).                                                     &  & \by{1.2.4}
			\end{align*}
		\item[For \ref{vs3}:]
			Since \(\F\) is a field, we know that \(0 \in \F\) and thus \((0, \dots, 0) \in \vs{F}^n\).
			Then for all \(u \in \vs{F}^n\), we have
			\begin{align*}
				u + (0, \dots, 0) & = (u_1 + 0, u_2 + 0, \dots, u_n + 0) &  & \by{1.2.4}                 \\
				                  & = \tuple{u}{1,2,,n}                  &  & \text{(\(\F\) is a field)} \\
				                  & = u.
			\end{align*}
			We denote \(\zv = (0, \dots, 0)\).
		\item[For \ref{vs4}:]
			For all \(u \in \vs{F}^n\), we have
			\begin{align*}
				         & \forall i \in I_n, u_i \in \F                                                                     \\
				\implies & \forall i \in I_n, \exists v_i \in \F : u_i + v_i = 0          &  & \text{(\(\F\) is a field)}    \\
				\implies & \exists \seq{v}{1,2,,n} \in \F :                                                                  \\
				         & (u_1 + v_1, u_2 + v_2, \dots, u_n + v_n) = (0, \dots, 0) = \zv                                    \\
				\implies & \exists y \in \vs{F}^n : u + v = \zv.                          &  & \text{(from the proof above)}
			\end{align*}
		\item[For \ref{vs5}:]
			Since \(\F\) is a field, we know that \(1 \in \F\).
			Then for all \(u \in \vs{F}^n\), we have
			\begin{align*}
				1u & = \tuple{1u}{1,2,,n} &  & \by{1.2.4}                 \\
				   & = \tuple{u}{1,2,,n}  &  & \text{(\(\F\) is a field)} \\
				   & = u.
			\end{align*}
		\item[For \ref{vs6}:]
			For all \(a, b \in \F\) and \(u \in \vs{F}^n\), we have
			\begin{align*}
				(ab) u & = \tuple{(ab) u}{1,2,,n}                   &  & \by{1.2.4}                 \\
				       & = (a (b u_1), a (b u_2), \dots, a (b u_n)) &  & \text{(\(\F\) is a field)} \\
				       & = a \tuple{bu}{1,2,,n}                     &  & \by{1.2.4}                 \\
				       & = a (bu).                                  &  & \by{1.2.4}
			\end{align*}
		\item[For \ref{vs7}:]
			For all \(a \in \F\) and \(u, v \in \vs{F}^n\), we have
			\begin{align*}
				a (u + v) & = a (u_1 + v_1, u_2 + v_2, \dots, u_n + v_n)           &  & \by{1.2.4}                 \\
				          & = (a (u_1 + v_1), a (u_2 + v_2), \dots, a (u_n + v_n)) &  & \by{1.2.4}                 \\
				          & = (a u_1 + a v_1, a u_2 + a v_2, \dots, a u_n + a v_n) &  & \text{(\(\F\) is a field)} \\
				          & = \tuple{au}{1,2,,n} + \tuple{av}{1,2,,n}              &  & \by{1.2.4}                 \\
				          & = a u + a v.                                           &  & \by{1.2.4}
			\end{align*}
		\item[For \ref{vs8}:]
			For all \(a, b \in \F\) and \(u \in \vs{F}^n\), we have
			\begin{align*}
				(a + b) u & = \tuple{(a + b) u}{1,2,,n}                            &  & \by{1.2.4}                 \\
				          & = (a u_1 + b u_1, a u_2 + b u_2, \dots, a u_n + b u_n) &  & \text{(\(\F\) is a field)} \\
				          & = \tuple{au}{1,2,,n} + \tuple{bu}{1,2,,n}              &  & \by{1.2.4}                 \\
				          & = a u + b u.                                           &  & \by{1.2.4}
			\end{align*}
	\end{description}
	From all proofs above we conclude by \cref{1.2.1} that \cref{1.2.4} is indeed a vector space over \(\F\).
\end{proof}

\begin{defn}\label{1.2.5}
	Vectors in \(\vs{F}^n\) may be written as \textbf{column vectors}
	\[
		\begin{pmatrix}
			a_1    \\
			a_2    \\
			\vdots \\
			a_n
		\end{pmatrix}
	\]
	rather than as \textbf{row vectors} \(\tuple{a}{1,2,,n}\).
	Since a \(1\)-tuple whose only entry is from \(\F\) can be regarded as an element of \(\F\), we usually write \(\F\) rather than \(\vs{F}^1\) for the vector space of \(1\)-tuples with entry from \(\F\).
\end{defn}

\begin{defn}\label{1.2.6}
	An \(m \times n\) \textbf{matrix} with entries from a field \(\F\) is a rectangular array of the form
	\[
		\begin{pmatrix}
			a_{1 1} & a_{1 2} & \cdots & a_{1 n} \\
			a_{2 1} & a_{2 2} & \cdots & a_{2 n} \\
			\vdots  & \vdots  & \ddots & \vdots  \\
			a_{m 1} & a_{m 2} & \cdots & a_{m n}
		\end{pmatrix},
	\]
	where each entry \(a_{i j}\) (\(1 \leq i \leq m\), \(1 \leq j \leq n\)) is an element of \(\F\).
	We call the entries \(a_{i j}\) with \(i = j\) the \textbf{diagonal entries} of the matrix.
	The entries \(a_{i 1} ,a_{i 2} , \dots, a_{i n}\) compose the \textbf{\(i\)th row} of the matrix, and the entries \(a_{1 j}, a_{2 j}, \dots, a_{m j}\) compose the \textbf{\(j\)th column} of the matrix.
	The rows of the preceding matrix are regarded as vectors in \(\vs{F}^n\), and the columns are regarded as vectors in \(\vs{F}^m\).
	The \(m \times n\) matrix in which each entry equals zero is called the \textbf{zero matrix} and is denoted by \(\zm\).
\end{defn}

\begin{defn}\label{1.2.7}
	In this book, we denote matrices by capital italic letters (e.g., \(A\), \(B\), and \(C\)), and we denote the entry of a matrix \(A\) that lies in row \(i\) and column \(j\) by \(A_{i j}\).
	In addition, if the number of rows and columns of a matrix are equal, the matrix is called \textbf{square}.
\end{defn}

\begin{defn}\label{1.2.8}
	Two \(m \times n\) matrices \(A\) and \(B\) are called \textbf{equal} if all their corresponding entries are equal, that is, if \(A_{i j} = B_{i j}\) for \(1 \leq i \leq m\) and \(1 \leq j \leq n\).
\end{defn}

\begin{eg}\label{1.2.9}
	The set of all \(m \times n\) matrices with entries from a field \(\F\) is a vector space over \(\F\), which we denote by \(\MS\), with the following operations of \textbf{matrix addition} and \textbf{scalar multiplication}:
	For \(A, B \in \MS\) and \(c \in \F\),
	\[
		(A + B)_{i j} = A_{i j} + B_{i j} \quad \text{and} \quad (c A)_{i j} = c A_{i j}
	\]
	for \(1 \leq i \leq m\) and \(1 \leq j \leq n\).
\end{eg}

\begin{proof}[\pf{1.2.9}]
	Clearly we have
	\[
		\forall A, B \in \MS, A + B \in \MS
	\]
	and
	\[
		\begin{dcases}
			\forall A \in \MS \\
			\forall c \in \F
		\end{dcases}, cA \in \MS.
	\]

	Let \(I_m = \set{i \in \N : 1 \leq i \leq m}\) and \(J_n = \set{j \in \N : 1 \leq j \leq n}\).
	First we show that addition in \(\MS\) over \(\F\) is unique.
	Let \(A, B \in \MS\).
	Suppose for sake of contradiction that
	\[
		\begin{dcases}
			A + B = C \in \MS  \\
			A + B = C' \in \MS \\
			C \neq C'
		\end{dcases}.
	\]
	Then we have
	\begin{align*}
		         & C \neq C'                                                                                     \\
		\implies & \exists (i, j) \in I_m \times J_n :  C_{i j} \neq C_{i j}'                    &  & \by{1.2.8} \\
		\implies & \exists (i, j) \in I_m \times J_n :  A_{i j} + B_{i j} \neq A_{i j} + B_{i j} &  & \by{1.2.9}
	\end{align*}
	which contradicts to the fact that \(A_{i j} + B_{i j} \in \F\) is unique.
	Thus addition in \(\MS\) over \(\F\) is unique.

	Next we show that scalar multiplication in \(\MS\) over \(\F\) is unique.
	Let \(A \in \MS\) and \(c \in \F\).
	Suppose for sake of contradiction that
	\[
		\begin{dcases}
			cA = B \in \MS  \\
			cA = B' \in \MS \\
			B \neq B'
		\end{dcases}.
	\]
	Then we have
	\begin{align*}
		         & B \neq B'                                                                    \\
		\implies & \exists (i, j) \in I_m \times J_n : B_{i j} \neq B_{i j}     &  & \by{1.2.8} \\
		\implies & \exists (i, j) \in I_m \times J_n : c A_{i j} \neq c A_{i j} &  & \by{1.2.9}
	\end{align*}
	which contradicts to the fact that \(c A_{i j} \in \F\) is unique.
	Thus scalar multiplication in \(\MS\) over \(\F\) is unique.

	Now we show that \ref{vs1}--\ref{vs8} holds for \cref{1.2.9}.
	\begin{description}
		\item[For \ref{vs1}:]
			For all \(A, B \in \MS\) and \((i, j) \in I_m \times J_n\), we have
			\begin{align*}
				(A + B)_{i j} & = A_{i j} + B_{i j} &  & \by{1.2.9}                 \\
				              & = B_{i j} + A_{i j} &  & \text{(\(\F\) is a field)} \\
				              & = (B + A)_{i j}     &  & \by{1.2.9}
			\end{align*}
			and thus by \cref{1.2.8} \(A + B = B + A\).
		\item[For \ref{vs2}:]
			For all \(A, B, C \in \MS\) and \((i, j) \in I_m \times J_n\), we have
			\begin{align*}
				((A + B) + C)_{i j} & = (A + B)_{i j} + C_{i + j}     &  & \by{1.2.9}                 \\
				                    & = (A_{i j} + B_{i j}) + C_{i j} &  & \by{1.2.9}                 \\
				                    & = A_{i j} + (B_{i j} + C_{i j}) &  & \text{(\(\F\) is a field)} \\
				                    & = A_{i j} + (B + C)_{i j}       &  & \by{1.2.9}                 \\
				                    & = (A + (B + C))_{i j}           &  & \by{1.2.9}
			\end{align*}
			and thus by \cref{1.2.8} \((A + B) + C = A + (B + C)\).
		\item[For \ref{vs3}:]
			Since \(\F\) is a field, we know that \(0 \in \F\) and thus
			\[
				\zm = \begin{pmatrix}
					0      & 0      & \cdots & 0      \\
					0      & 0      & \cdots & 0      \\
					\vdots & \vdots & \ddots & \vdots \\
					0      & 0      & \cdots & 0
				\end{pmatrix} \in \MS.
			\]
			Then for all \(A \in \MS\) and \((i, j) \in I_m \times J_n\), we have
			\begin{align*}
				(A + \zm)_{i j} & = A_{i j} + \zm_{i j} &  & \by{1.2.9}                 \\
				                & = A_{i j} + 0         &  & \by{1.2.6}                 \\
				                & = A_{i j}             &  & \text{(\(\F\) is a field)}
			\end{align*}
			and thus by \cref{1.2.8} \(A + \zm = A\).
		\item[For \ref{vs4}:]
			For all \(A \in \MS\), we have
			\begin{align*}
				         & \forall (i, j) \in I_m \times J_n, A_{i j} \in \F                                                                    \\
				\implies & \forall (i, j) \in I_m \times J_n, \exists B_{i j} \in \F : A_{i j} + B_{i j} = 0 &  & \text{(\(\F\) is a field)}    \\
				\implies & \exists B \in \MS : A + B = \zm.                                                  &  & \text{(from the proof above)}
			\end{align*}
		\item[For \ref{vs5}:]
			Since \(\F\) is a field, we know that \(1 \in \F\).
			Then for all \(A \in \MS\) and \((i, j) \in I_m \times J_n\), we have
			\begin{align*}
				(1A)_{i j} & = 1 A_{i j} &  & \by{1.2.9}                 \\
				           & = A_{i j}   &  & \text{(\(\F\) is a field)}
			\end{align*}
			and thus by \cref{1.2.8} \(1A = A\).
		\item[For \ref{vs6}:]
			For all \(c, d \in \F\), \(A \in \MS\) and \((i, j) \in I_m \times J_n\), we have
			\begin{align*}
				((cd) A)_{i j} & = (cd) A_{i j}    &  & \by{1.2.9}                 \\
				               & = c (d A_{i j})   &  & \text{(\(\F\) is a field)} \\
				               & = c (d A)_{i j}   &  & \by{1.2.9}                 \\
				               & = (c (d A))_{i j} &  & \by{1.2.9}
			\end{align*}
			and thus by \cref{1.2.8} \((cd) A = c (dA)\).
		\item[For \ref{vs7}:]
			For all \(c \in \F\), \(A, B \in \MS\) and \((i, j) \in I_m \times J_n\), we have
			\begin{align*}
				(c (A + B))_{i j} & = c (A + B)_{i j}         &  & \by{1.2.9}                 \\
				                  & = c (A_{i j} + B_{i j})   &  & \by{1.2.9}                 \\
				                  & = c A_{i j} + c B_{i j}   &  & \text{(\(\F\) is a field)} \\
				                  & = (cA)_{i j} + (cB)_{i j} &  & \by{1.2.9}
			\end{align*}
			and thus by \cref{1.2.8} \(c (A + B) = cA + cB\).
		\item[For \ref{vs8}:]
			For all \(c, d \in \F\), \(A \in \MS\) and \((i, j) \in I_m \times J_n\), we have
			\begin{align*}
				((c + d) A)_{i j} & = (c + d) A_{i j}         &  & \by{1.2.9}                 \\
				                  & = c A_{i j} + d A_{i j}   &  & \text{(\(\F\) is a field)} \\
				                  & = (cA)_{i j} + (dA)_{i j} &  & \by{1.2.9}
			\end{align*}
			and thus by \cref{1.2.8} \((c + d) A = cA + dA\).
	\end{description}
	From all proofs above we conclude by \cref{1.2.1} that \cref{1.2.9} is indeed a vector space over \(\F\).
\end{proof}

\begin{eg}\label{1.2.10}
	Let \(S\) be any nonempty set and \(\F\) be any field, and let \(\FS\) denote the set of all functions from \(S\) to \(\F\).
	Two functions \(f\) and \(g\) in \(\FS\) are called \textbf{equal} if \(f(s) = g(s)\) for each \(s \in S\).
	The set \(\FS\) is a vector space over \(\F\) with the operations of addition and scalar multiplication defined for \(f, g \in \FS\) and \(c \in \F\) by
	\[
		(f + g)(s) = f(s) + g(s) \quad \text{and} \quad (cf)(s) = cf(s)
	\]
	for each \(s \in S\).
\end{eg}

\begin{proof}[\pf{1.2.10}]
	Clearly we have
	\[
		\forall f, g \in \FS, f + g \in \FS
	\]
	and
	\[
		\begin{dcases}
			\forall f \in \FS \\
			\forall c \in \F
		\end{dcases}, cf \in \FS.
	\]

	First we show that addition in \(\FS\) over \(\F\) is unique.
	Let \(f, g \in \FS\).
	Suppose for sake of contradiction that
	\[
		\begin{dcases}
			f + g = h \in \FS  \\
			f + g = h' \in \FS \\
			h \neq h'
		\end{dcases}.
	\]
	Then we have
	\begin{align*}
		         & h \neq h'                                                       \\
		\implies & \exists s \in S : h(s) \neq h'(s)              &  & \by{1.2.10} \\
		\implies & \exists s \in S : (f + g)(s) \neq (f + g)(s)   &  & \by{1.2.10} \\
		\implies & \exists s \in S : f(s) + g(s) \neq f(s) + g(s) &  & \by{1.2.10}
	\end{align*}
	which contradicts to the fact that \(f(s) + g(s) \in \F\) is unique.
	Thus addition in \(\FS\) over \(\F\) is unique.

	Next we show that scalar multiplication in \(\FS\) over \(\F\) is unique.
	Let \(f \in \FS\) and \(c \in \F\).
	Suppose for sake of contradiction that
	\[
		\begin{dcases}
			cf = h \in \FS  \\
			cf = h' \in \FS \\
			h \neq h'
		\end{dcases}.
	\]
	Then we have
	\begin{align*}
		         & h \neq h'                                               \\
		\implies & \exists s \in S : h(s) \neq h'(s)      &  & \by{1.2.10} \\
		\implies & \exists s \in S : (cf)(s) \neq (cf)(s) &  & \by{1.2.10} \\
		\implies & \exists s \in S : cf(s) \neq cf(s)     &  & \by{1.2.10}
	\end{align*}
	which contradicts to the fact that \(cf(s) \in \F\) is unique.
	Thus scalar multiplication in \(\FS\) over \(\F\) is unique.

	Now we show that \ref{vs1}--\ref{vs8} holds for \cref{1.2.10}.
	\begin{description}
		\item[For \ref{vs1}:]
			For all \(f, g \in \FS\) and \(s \in S\), we have
			\begin{align*}
				(f + g)(s) & = f(s) + g(s) &  & \by{1.2.10}                \\
				           & = g(s) + f(s) &  & \text{(\(\F\) is a field)} \\
				           & = (g + f)(s)  &  & \by{1.2.10}
			\end{align*}
			and thus by \cref{1.2.10} \(f + g = g + f\).
		\item[For \ref{vs2}:]
			For all \(f, g, h \in \FS\) and \(s \in S\), we have
			\begin{align*}
				 & ((f + g) + h)(s)                                       \\
				 & = (f + g)(s) + h(s)    &  & \by{1.2.10}                \\
				 & = (f(s) + g(s)) + h(s) &  & \by{1.2.10}                \\
				 & = f(s) + (g(s) + h(s)) &  & \text{(\(\F\) is a field)} \\
				 & = f(s) + (g + h)(s)    &  & \by{1.2.10}                \\
				 & = (f + (g + h))(s)     &  & \by{1.2.10}
			\end{align*}
			and thus by \cref{1.2.10} \((f + g) + h = f + (g + h)\).
		\item[For \ref{vs3}:]
			Since \(\F\) is a field, we know that \(0 \in \F\).
			If we define \(\zv : S \to \{0\}\), then we know that \(\zv \in \F\).
			Thus for all \(f \in \FS\) and \(s \in S\), we have
			\begin{align*}
				(f + \zv)(s) & = f(s) + \zv(s) &  & \by{1.2.10}                \\
				             & = f(s) + 0                                      \\
				             & = f(s)          &  & \text{(\(\F\) is a field)}
			\end{align*}
			and by \cref{1.2.10} \(f + \zv = f\).
		\item[For \ref{vs4}:]
			For all \(f \in \FS\), we have
			\begin{align*}
				         & \forall s \in S, f(s) \in \F                                                          \\
				\implies & \forall s \in S, \exists s' \in \F : f(s) + s' = 0 &  & \text{(\(\F\) is a field)}    \\
				\implies & \exists g \in \FS : \begin{dcases}
					                               g = s \mapsto s' \\
					                               f(s) + g(s) = 0
				                               \end{dcases}                                                  \\
				\implies & \exists g \in \FS : (f + g)(s) = 0                 &  & \by{1.2.10}                   \\
				\implies & \exists g \in \FS : f + g = \zv.                   &  & \text{(from the proof above)}
			\end{align*}
		\item[For \ref{vs5}:]
			Since \(\F\) is a field, we know that \(1 \in \F\).
			Then for all \(f \in \FS\) and \(s \in S\), we have
			\begin{align*}
				(1f)(s) & = 1f(s) &  & \by{1.2.10}                \\
				        & = f(s)  &  & \text{(\(\F\) is a field)}
			\end{align*}
			and thus by \cref{1.2.10} \(1f = f\).
		\item[For \ref{vs6}:]
			For all \(a, b \in \F\), \(f \in \FS\) and \(s \in S\), we have
			\begin{align*}
				((ab) f)(s) & = (ab) f(s)   &  & \by{1.2.10}                \\
				            & = a (b f(s))  &  & \text{(\(\F\) is a field)} \\
				            & = a ((bf)(s)) &  & \by{1.2.10}                \\
				            & = (a (bf))(s) &  & \by{1.2.10}
			\end{align*}
			and thus by \cref{1.2.10} \((ab) f = a (bf)\).
		\item[For \ref{vs7}:]
			For all \(a \in \F\), \(f, g \in \FS\) and \(s \in S\), we have
			\begin{align*}
				(a (f + g))(s) & = a ((f + g)(s))    &  & \by{1.2.10}                \\
				               & = a (f(s) + g(s))   &  & \by{1.2.10}                \\
				               & = af(s) + ag(s)     &  & \text{(\(\F\) is a field)} \\
				               & = (af)(s) + (ag)(s) &  & \by{1.2.10}                \\
				               & = (af + ag)(s)      &  & \by{1.2.10}
			\end{align*}
			and thus by \cref{1.2.10} \(a (f + g) = af + ag\).
		\item[For \ref{vs8}:]
			For all \(a, b \in \F\), \(f \in \FS\) and \(s \in S\), we have
			\begin{align*}
				((a + b) f)(s) & = (a + b) f(s)      &  & \by{1.2.10}                \\
				               & = af(s) + bf(s)     &  & \text{(\(\F\) is a field)} \\
				               & = (af)(s) + (bf)(s) &  & \by{1.2.10}                \\
				               & = (af + bf)(s)      &  & \by{1.2.10}
			\end{align*}
			and thus by \cref{1.2.10} \((a + b) f = af + bf\).
	\end{description}
	From all proofs above we conclude by \cref{1.2.1} that \cref{1.2.10} is indeed a vector space over \(\F\).
\end{proof}

\begin{defn}\label{1.2.11}
	A \textbf{polynomial} with coefficients from a field \(\F\) is an expression of the form
	\[
		f(x) = a_n x^n + a_{n - 1} x^{n - 1} + \cdots + a_1 x + a_0,
	\]
	where \(n\) is a nonnegative integer and each \(a_k\), called the \textbf{coefficient} of \(x^k\), is in \(\F\).
	If \(f(x) = 0\), that is, if \(a_n = a_{n - 1} = \cdots = a_0 = 0\), then \(f(x)\) is called the \textbf{zero polynomial} and, for convenience, its degree is defined to be \(-1\);
	otherwise, the \textbf{degree} of a polynomial is defined to be the largest exponent of \(x\) that appears in the representation
	\[
		f(x) = a_n x^n + a_{n - 1} x^{n - 1} + \cdots + a_1 x + a_0
	\]
	with a nonzero coefficient.
	Note that the polynomials of degree zero may be written in the form \(f(x) = c\) for some nonzero scalar \(c\).
	Two polynomials,
	\[
		f(x) = a_n x^n + a_{n - 1} x^{n - 1} + \cdots + a_1 x + a_0
	\]
	and
	\[
		g(x) = b_m x^m + b_{m - 1} x^{m - 1} + \cdots + b_1 x + b_0,
	\]
	are called \textbf{equal} if \(m = n\) and \(a_i = b_i\) for \(i = 0, 1, \dots, n\).
\end{defn}

\begin{eg}\label{1.2.12}
	Let
	\[
		f(x) = a_n x^n + a_{n - 1} x^{n - 1} + \cdots + a_1 x + a_0
	\]
	and
	\[
		g(x) = b_m x^m + b_{m - 1} x^{m - 1} + \cdots + b_1 x + b_0
	\]
	be polynomials with coefficients from a field \(\F\).
	Suppose that \(m \leq n\), and define \(b_{m + 1} = b_{m + 2} = \cdots = b_n = 0\).
	Then \(g(x)\) can be written as
	\[
		g(x) = b_n x^n + b_{n - 1} x^{n - 1} + \cdots + b_1 x + b_0.
	\]

	Define
	\[
		f(x) + g(x) = (a_n + b_n) x^n + (a_{n - 1} + b_{n - 1}) x^{n - 1} + \cdots + (a_1 + b_1) x^1 + (a_0 + b_0)
	\]
	and for any scalar \(c \in \F\), define
	\[
		cf(x) = ca_n x^n + ca_{n - 1} x^{n - 1} + \cdots + ca_1 x + ca_0.
	\]
	With these operations of addition and scalar multiplication, the set of all polynomials with coefficients from \(\F\) is a vector space over \(\F\), which we denote by \(\ps{\F}\).
\end{eg}

\begin{proof}[\pf{1.2.12}]
	Note that we prove \cref{1.2.12} without circularity.
	Observe that \(\ps{\F} \subseteq \fs(\F, \F)\) (see \cref{1.2.10}), thus we can use \cref{1.3} to prove \cref{1.2.12}.
	Since \(\F\) is a field, we know that \(0 \in F\).
	Let \(\zv\) be the zero vector in \(\fs(\F, \F)\) (cf. \cref{1.2.10}).
	Then we have
	\begin{align*}
		         & \forall f \in \fs(\F, \F), \forall s \in \F, (f + \zv)(s) = f(s) \\
		\implies & \forall p \in \ps{\F}, \forall s \in \F, (p + \zv)(s) = p(s).
	\end{align*}
	Now we define \(\zv' \in \ps{F}\) as follow:
	\[
		\forall s \in \F, \zv'(s) = 0.
	\]
	Since
	\begin{align*}
		         & \forall p \in \ps{\F}, \forall s \in \F, p(s) + \zv'(s) = p(s) = p(s) + \zv(s)                    \\
		\implies & \forall s \in \F, \zv'(s) = \zv(s)                                             &  & (p(s) \in \F) \\
		\implies & \zv' = \zv,                                                                    &  & \by{1.2.10}
	\end{align*}
	we know that \(\zv \in \ps{\F}\).
	Since \(\ps{\F}\) is closed under addition and scalar multiplication for obvious reason, by \cref{1.3} we conclude that \(\ps{\F}\) is a subspace of \(\fs(\F, \F)\) over \(\F\) and thus \(\ps{\F}\) is a vector space over \(\F\).
\end{proof}

\begin{eg}\label{1.2.13}
	Let \(\F\) be any field.
	A sequence in \(\F\) is a function \(\sigma\) from the positive integers into \(\F\).
	In this book, the sequence \(\sigma\) such that \(\sigma(n) = a_n\) for \(n = 1, 2, \dots\) is denoted \(\set{a_n}\).
	Let \(\V\) consist of all sequences \(\set{a_n}\) in \(\F\) that have only a finite number of nonzero terms \(a_n\).
	If \(\set{a_n}\) and \(\set{b_n}\) are in \(\V\) and \(t \in \F\), define
	\[
		\set{a_n} + \set{b_n} = \set{a_n + b_n} \quad \text{and} \quad t\set{a_n} = \set{ta_n}.
	\]
	With these operations \(\V\) is a vector space over \(\F\).
\end{eg}

\begin{proof}[\pf{1.2.13}]
	First observe that \(\V \subseteq \fs(\N, \F)\).
	We know by \cref{1.2.10} that there exists a function \(\zv \in \fs(\N, \F)\) such that \(\zv(n) = 0\) for all \(n \in \N\).
	Since \(\zv\) only have finite number of nonzero terms (actually, \(\zv\) has \(0\) nonzero term), we know that \(\zv \in \V\).

	Let \(\B{a_n}, \B{b_n} \in \V\) and \(c \in \R\).
	Since \(\B{a_n}, \B{b_n}\) have finite numbers of nonzero terms, we know that \(\B{a_n} + \B{b_n} = \B{a_n + b_n}\) also has a finite number of nonzero terms and thus \(\B{a_n} + \B{b_n} \in \V\).
	Similarly we know that \(c\B{a_n} = \B{ca_n}\) also has a finite number of nonzero terms and thus \(c\B{a_n} \in \V\).
	Therefore by \cref{1.3} we know that \(\V\) is a subspace of \(\fs(\N, \F)\) over \(\F\) and \(\V\) is a vector space over \(\F\).
\end{proof}

\begin{thm}[Cancellation Law for Vector Addition]\label{1.1}
	If \(x, y\), and \(z\) are vectors in a vector space \(\V\) over \(\F\) such that \(x + z = y + z\), then \(x = y\).
\end{thm}

\begin{proof}[\pf{1.1}]
	There exists a vector \(v\) in \(\V\) such that \(z + v = \zv\) (\ref{vs4}).
	Thus
	\begin{align*}
		x & = x + \zv = x + (z + v) = (x + z) + v     \\
		  & = (y + z) + v = y + (z + v) = y + \zv = y
	\end{align*}
	by \ref{vs2} and \ref{vs3}.
\end{proof}

\begin{cor}\label{1.2.14}
	The vector \(\zv\) described in \ref{vs3} is unique.
\end{cor}

\begin{proof}[\pf{1.2.14}]
	Suppose for sake of contradiction that
	\[
		\exists \zv' \in \V : \begin{dcases}
			\zv \neq \zv' \\
			\forall x \in \V, x + \zv' = x
		\end{dcases}.
	\]
	But then we have
	\begin{align*}
		         & x + \zv = x = x + \zv' &  & \text{(by \ref{vs3})} \\
		\implies & \zv = \zv',            &  & \by{1.1}
	\end{align*}
	a contradiction.
	Thus \(\zv \in \V\) is unique.
\end{proof}

\begin{cor}\label{1.2.15}
	The vector \(y\) described in \ref{vs4} is unique.
\end{cor}

\begin{proof}[\pf{1.2.15}]
	Suppose for sake of contradiction that
	\[
		\forall x \in \V, \exists y, y' \in \V : \begin{dcases}
			y \neq y' \\
			x + y = 0 = x + y'
		\end{dcases}.
	\]
	But then we have
	\begin{align*}
		         & x + y = x + y'               \\
		\implies & y = y',        &  & \by{1.1}
	\end{align*}
	a contradiction.
	Thus \(y\) described in \ref{vs4} is unique.
\end{proof}

\begin{defn}\label{1.2.16}
	The vector \(\zv\) in \ref{vs3} is called the \textbf{zero vector} of \(\V\), and the vector \(y\) in \ref{vs4} (that is, the unique vector such that \(x + y = \zv\)) is called the \textbf{additive inverse} of \(x\) and is denoted by \(-x\).
\end{defn}

\begin{thm}\label{1.2}
	In any vector space \(\V\) over \(\F\), the following statements are true:
	\begin{enumerate}
		\item \(0x = \zv\) for each \(x \in \V\).
		\item \((-a)x = -(ax) = a(-x)\) for each \(a \in \F\) and each \(x \in \V\).
		\item \(a\zv = \zv\) for each \(a \in F\).
	\end{enumerate}
\end{thm}

\begin{proof}[\pf{1.2}(a)]
	By \ref{vs8}, \ref{vs3}, and \ref{vs1}, it follows that
	\[
		0x + 0x = (0 + 0) x = 0x = 0x + \zv = \zv + 0x.
	\]
	Hence \(0x = \zv\) by \cref{1.1}.
\end{proof}

\begin{proof}[\pf{1.2}(b)]
	The vector \(-(ax)\) is the unique element of \(\V\) such that \(ax + [-(ax)] = \zv\).
	Thus if \(ax + (-a)x = 0\), \cref{1.2.15} to \cref{1.1} implies that \((-a)x = -(ax)\).
	But by \ref{vs8},
	\[
		ax + (-a)x = [a + (-a)]x = 0x = \zv
	\]
	by \cref{1.2}(a).
	Consequently \((-a)x = -(ax)\).
	In particular, \((-1)x = -x\).
	So, by \ref{vs6},
	\[
		a(-x) = a[(-1)x] = [a(-1)]x = (-a)x.
	\]
\end{proof}

\begin{proof}[\pf{1.2}(c)]
	We have
	\begin{align*}
		a\zv + \zv & = a\zv         &  & \text{(by \ref{vs3})} \\
		           & = a(\zv + \zv) &  & \text{(by \ref{vs3})} \\
		           & = a\zv + a\zv  &  & \text{(by \ref{vs7})}
	\end{align*}
	and thus by \cref{1.1} \(a\zv = \zv\).
\end{proof}

\exercisesection

\setcounter{ex}{7}
\begin{ex}\label{ex:1.2.8}
	In any vector space \(\V\) over \(\F\), show that \((a + b)(x + y) = ax + ay + bx + by\) for any \(x, y \in \V\) and any \(a, b \in \F\).
\end{ex}

\begin{proof}[\pf{ex:1.2.8}]
	We have
	\begin{align*}
		(a + b)(x + y) & = (a + b) x + (a + b) y &  & \text{(by \ref{vs7})} \\
		               & = ax + bx + ay + by     &  & \text{(by \ref{vs8})} \\
		               & = ax + ay + bx + by.    &  & \text{(by \ref{vs1})}
	\end{align*}
\end{proof}

\setcounter{ex}{9}
\begin{ex}\label{ex:1.2.10}
	Let \(\V\) denote the set of all differentiable real-valued functions defined on the real line.
	Prove that \(\V\) is a vector space over \(\R\) with the operations of addition and scalar multiplication defined in \cref{1.2.10}.
\end{ex}

\begin{proof}[\pf{ex:1.2.10}]
	First observe that \(\V \subseteq \fs(S, \R)\) for arbitrary domain \(S\).
	Since \(\zv \in \fs(S, \R)\) defined by \(\zv(x) = 0\) is differentiable on \(\R\), we know that \(\zv \in \V\).

	Let \(f, g \in \V\) and \(c \in \R\).
	Since
	\[
		f' + g' = (f + g)' \in \V
	\]
	and
	\[
		cf' = (cf)' \in \V,
	\]
	by \cref{1.3} we know that \(\V\) is a subspace of \(\fs(S, \R)\) over \(\R\) and thus \(\V\) is a vector space over \(\R\).
\end{proof}

\begin{ex}\label{ex:1.2.11}
	Let \(\V = \set{\zv}\) consist of a single vector \(\zv\) and define \(\zv + \zv = \zv\) and \(c\zv = \zv\) for each scalar \(c\) in \(\F\).
	Prove that \(\V\) is a vector space over \(\F\).
	(\(\V\) is called the \textbf{zero vector space}.)
\end{ex}

\begin{proof}[\pf{ex:1.2.11}]
	Clearly addition and scalar multiplication result in one unique element in \(\V\), namely \(\zv\).
	We only need to show that \ref{vs1}--\ref{vs8} holds for \cref{ex:1.2.11}.
	\begin{description}
		\item[For \ref{vs1}:]
			By definition we have
			\[
				\zv + \zv = \zv = \zv + \zv.
			\]
		\item[For \ref{vs2}:]
			We have
			\[
				(\zv + \zv) + \zv = \zv + \zv = \zv + (\zv + \zv).
			\]
		\item[For \ref{vs3}:]
			By definition we have
			\[
				\zv + \zv = \zv.
			\]
		\item[For \ref{vs4}:]
			By definition we have
			\[
				\zv + \zv = \zv.
			\]
		\item[For \ref{vs5}:]
			By definition we have
			\[
				1\zv = \zv.
			\]
		\item[For \ref{vs6}:]
			For all \(a, b \in \F\), we have
			\begin{align*}
				(ab) \zv & = \zv      &  & \text{(by definition)} \\
				         & = b\zv     &  & (b \in \F)             \\
				         & = a(b\zv). &  & (a \in \F)
			\end{align*}
		\item[For \ref{vs7}:]
			For all \(a \in \F\), we have
			\begin{align*}
				a(\zv + \zv) & = a\zv         &  & \text{(by definition)} \\
				             & = \zv          &  & \text{(by definition)} \\
				             & = \zv + \zv    &  & \text{(by definition)} \\
				             & = a\zv + a\zv. &  & (a \in \F)
			\end{align*}
		\item[For \ref{vs8}:]
			For all \(a, b \in \F\), \(f \in \FS\) and \(s \in S\), we have
			\begin{align*}
				(a + b) \zv & = \zv          &  & \text{(by definition)} \\
				            & = \zv + \zv    &  & \text{(by definition)} \\
				            & = a\zv + b\zv. &  & (a, b \in \F)
			\end{align*}
	\end{description}
	From all proofs above we conclude by \cref{1.2.1} that \cref{ex:1.2.11} is indeed a vector space.
\end{proof}

\begin{ex}\label{ex:1.2.12}
	A real-valued function \(f\) defined on the real line is called an \textbf{even function} if \(f(-t) = f(t)\) for each real number \(t\).
	Prove that the set of even functions defined on the real line with the operations of addition and scalar multiplication defined in \cref{1.2.10} is a vector space over \(\R\).
\end{ex}

\begin{proof}[\pf{ex:1.2.12}]
	First observe that \(\V \subseteq \fs(\R, \R)\).
	We know by \cref{1.2.10} that there exists a function \(\zv \in \fs(\R, \R)\) such that \(\zv(x) = 0\) for all \(x \in \R\).
	Since \(\zv(t) = 0 = \zv(-t)\) for all \(t \in \R\), we know that \(\zv \in \V\).

	Let \(f, g \in \V\) and \(t, c \in \R\).
	Since
	\begin{align*}
		(f + g)(t) & = f(t) + g(t)   &  & \by{1.2.10}    \\
		           & = f(-t) + g(-t) &  & \by{ex:1.2.12} \\
		           & = (f + g)(-t)   &  & \by{1.2.10}
	\end{align*}
	and
	\begin{align*}
		(cf)(t) & = cf(t)     &  & \by{1.2.10}    \\
		        & = cf(-t)    &  & \by{ex:1.2.12} \\
		        & = (cf)(-t), &  & \by{1.2.10}
	\end{align*}
	we know that \(f + g \in \V\) and \(cf \in \V\).
	Thus by \cref{1.3} we know that \(\V\) is a subspace of \(\fs(\R, \R)\) over \(\R\) and \(\V\) is a vector space over \(\R\).
\end{proof}

\setcounter{ex}{20}
\begin{ex}\label{ex:1.2.21}
	Let \(\V\) and \(\W\) be vector spaces over a field \(\F\).
	Let
	\[
		\V \times \W = \set{(v, w) : v \in \V \land w \in \W}.
	\]
	Prove that \(\V \times \W\) is a vector space over \(\F\) with the operations
	\[
		(v_1, w_1) + (v_2, w_2) = (v_1 + v_2, w_1 + w_2) \quad \text{and} \quad c(v_1, w_1) = (cv_1, cw_1).
	\]
\end{ex}

\begin{proof}[\pf{ex:1.2.21}]
	Clearly we have
	\[
		\forall (v_1, w_1), (v_2, w_2) \in \V \times \W, (v_1, w_1) + (v_2, w_2) \in \V \times \W
	\]
	and
	\[
		\begin{dcases}
			\forall (v, w) \in \V \times \W \\
			\forall c \in \F
		\end{dcases}, c (v, w) \in \V \times \W.
	\]

	First we show that addition in \(\V \times \W\) over \(\F\) is unique.
	Let \((v_1, w_1), (v_2, w_2) \in \V \times \W\).
	Suppose for sake of contradiction that
	\[
		(v_1, w_1) + (v_2, w_2) \neq (v_1, w_1) + (v_2, w_2).
	\]
	Then we have
	\begin{align*}
		         & (v_1, w_1) + (v_2, w_2) \neq (v_1, w_1) + (v_2, w_2)                                                                \\
		\implies & (v_1 + v_2, w_1 + w_2) \neq (v_1 + v_2, w_1 + w_2)         &  & \by{ex:1.2.21}                                      \\
		\implies & (v_1 + v_2 \neq v_1 + v_2) \lor (w_1 + w_2 \neq w_1 + w_2) &  & \text{(this is the definition of \(\V \times \W\))}
	\end{align*}
	which contradicts to the fact that \(v_1 + v_2 \in \V\) is unique and \(w_1 + w_2 \in \W\) is unique.
	Thus addition in \(\V \times \W\) over \(\F\) is unique.

	Next we show that scalar multiplication in \(\V \times \W\) over \(\F\) is unique.
	Let \((v, w) \in \V \times \W\) and \(c \in \F\).
	Suppose for sake of contradiction that
	\[
		c(v, w) \neq c(v, w).
	\]
	Then we have
	\begin{align*}
		         & c(v, w) \neq c(v, w)                                                                    \\
		\implies & (cv, cw) \neq (cv, cw)         &  & \by{ex:1.2.21}                                      \\
		\implies & (cv \neq cv) \lor (cw \neq cw) &  & \text{(this is the definition of \(\V \times \W\))}
	\end{align*}
	which contradicts to the fact that \(cv \in \V\) is unique and \(cw \in \W\) is unique.
	Thus scalar multiplication in \(\V \times \W\) over \(\F\) is unique.

	Now we show that \ref{vs1}--\ref{vs8} holds for \cref{ex:1.2.21}.
	\begin{description}
		\item[For \ref{vs1}:]
			For all \((v_1, w_1), (v_2, w_2) \in \V \times \W\), we have
			\begin{align*}
				(v_1, w_1) + (v_2, w_2) & = (v_1 + v_2, w_1 + w_2)   &  & \by{ex:1.2.21}        \\
				                        & = (v_2 + v_1, w_2 + w_1)   &  & \text{(bt \ref{vs1})} \\
				                        & = (v_2, w_2) + (v_1, w_1). &  & \by{ex:1.2.21}
			\end{align*}
		\item[For \ref{vs2}:]
			For all \((v_1, w_1), (v_2, w_2), (v_3, w_3) \in \V \times \W\), we have
			\begin{align*}
				 & ((v_1, w_1) + (v_2, w_2)) + (v_3, w_3)                               \\
				 & = (v_1 + v_2, w_1 + w_2) + (v_3, w_3)     &  & \by{ex:1.2.21}        \\
				 & = ((v_1 + v_2) + v_3, (w_1 + w_2) + w_3)  &  & \by{ex:1.2.21}        \\
				 & = (v_1 + (v_2 + v_3), w_1 + (w_2 + w_3))  &  & \text{(by \ref{vs2})} \\
				 & = (v_1, w_1) + (v_2 + v_3, w_2 + w_3)     &  & \by{ex:1.2.21}        \\
				 & = (v_1, w_1) + ((v_2, w_2) + (v_3, w_3)). &  & \by{ex:1.2.21}
			\end{align*}
		\item[For \ref{vs3}:]
			Since \(\V, \W\) are vector spaces, we know that \(\zv_{\V} \in \V\) and \(\zv_{\W} \in \W\).
			Thus for all \((v, w) \in \V \times \W\), we have
			\begin{align*}
				(v, w) + (\zv_{\V}, \zv_{\W}) & = (v + \zv_{\V}, w + \zv_{\W}) &  & \by{ex:1.2.21}        \\
				                              & = (v, w).                      &  & \text{(by \ref{vs3})}
			\end{align*}
			We denote \(\zv = (\zv_{\V}, \zv_{\W})\).
		\item[For \ref{vs4}:]
			For all \((v_1, w_1) \in \V \times \W\), we have
			\begin{align*}
				         & (v_1 \in \V) \land (w_1 \in \W)                                      &  & \by{ex:1.2.21}                \\
				\implies & \begin{dcases}
					           \exists v_2 \in \V : v_1 + v_2 = \zv_{\V} \\
					           \exists w_2 \in \V : w_1 + w_2 = \zv_{\W}
				           \end{dcases}                         &  & \text{(by \ref{vs4})}                                         \\
				\implies & \exists (v_2, w_2) \in \V \times \W : (v_1, w_1) + (v_2, w_2) = \zv. &  & \text{(from the proof above)}
			\end{align*}
		\item[For \ref{vs5}:]
			Since \(\F\) is a field, we know that \(1 \in \F\).
			Thus for all \((v, w) \in \V \times \W\), we have
			\begin{align*}
				1(v, w) & = (1v, 1w) &  & \by{ex:1.2.21}        \\
				        & = (v, w).  &  & \text{(by \ref{vs5})}
			\end{align*}
		\item[For \ref{vs6}:]
			For all \(c, d \in \F\), \((v, w) \in \V \times \W\), we have
			\begin{align*}
				(cd) (v, w) & = ((cd) v, (cd) w) &  & \by{ex:1.2.21}        \\
				            & = (c(dv), c(dw))   &  & \text{(by \ref{vs6})} \\
				            & = c (dv, dw)       &  & \by{ex:1.2.21}        \\
				            & = c(d(v, w)).      &  & \by{ex:1.2.21}
			\end{align*}
		\item[For \ref{vs7}:]
			For all \(c \in \F\), \((v_1, w_1), (v_2, w_2) \in \V \times \W\), we have
			\begin{align*}
				c ((v_1, w_1) + (v_2, w_2)) & = c (v_1 + v_2, w_1, w_2)     &  & \by{ex:1.2.21}        \\
				                            & = (c(v_1 + v_2), c(w_1, w_2)) &  & \by{ex:1.2.21}        \\
				                            & = (cv_1 + cv_2, cw_1 + cw_2)  &  & \text{(by \ref{vs7})} \\
				                            & = (cv_1, cw_1) + (cv_2, cw_2) &  & \by{ex:1.2.21}        \\
				                            & = c(v_1, w_1) + c(v_2, w_2).  &  & \by{ex:1.2.21}
			\end{align*}
		\item[For \ref{vs8}:]
			For all \(c, d \in \F\), \((v, w) \in \V \times \W\), we have
			\begin{align*}
				(c + d) (v, w) & = ((c + d) v, (c + d) w) &  & \by{ex:1.2.21}        \\
				               & = (cv + dv, cw + dw)     &  & \text{(by \ref{vs8})} \\
				               & = (cv, cw) + (dv, dw)    &  & \by{ex:1.2.21}        \\
				               & = c(v, w) + d(v, w).     &  & \by{ex:1.2.21}
			\end{align*}
	\end{description}
	From all proofs above we conclude by \cref{1.2.1} that \cref{ex:1.2.21} is indeed a vector space over \(\F\).
\end{proof}

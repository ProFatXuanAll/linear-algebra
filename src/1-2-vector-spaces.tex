\section{Vector Spaces}\label{sec:1.2}

\begin{definition}\label{def:1.2.1}
    A \textbf{vector space} (or \textbf{linear space}) $V$ over a field $F$ consists of a set on which two operations (called \textbf{addition} and \textbf{scalar multiplication}, respectively) are defined so that for each pair of elements $x$, $y$ in $V$ there is a unique element $x + y$ in $V$, and for each element $a$ in $F$ and each element $x$ in $V$ there is a unique element $ax$ in $V$, such that the following conditions hold.
    \begin{enumerate}[label=(VS \arabic*), ref=VS \arabic*]
        \item \label{1.2.1.1}
              For all $x, y$ in $V$, $x + y = y + x$
              (commutativity of addition).
        \item \label{1.2.1.2}
              For all $x, y, z$ in $V$, $(x + y) + z = x + (y + z)$
              (associativity of addition).
        \item \label{1.2.1.3}
              There exists an element in $V$ denoted by $0$ such that $x + 0 = x$ for each $x$ in $V$.
        \item \label{1.2.1.4}
              For each element $x$ in $V$ there exists an element $y$ in $V$ such that $x + y = 0$.
        \item \label{1.2.1.5}
              For each element $x$ in $V$, $1x = x$.
        \item \label{1.2.1.6}
              For each pair of elements $a, b$ in $F$ and each element $x$ in $V$, $(ab)x = a(bx)$.
        \item \label{1.2.1.7}
              For each element $a$ in $F$ and each pair of elements $x, y$ in $V$, $a(x + y) = ax + ay$.
        \item \label{1.2.1.8}
              For each pair of elements $a, b$ in $F$ and each element $x$ in $V$, $(a + b)x = ax + bx$.
    \end{enumerate}
    The elements $x + y$ and $ax$ are called the \textbf{sum} of $x$ and $y$ and the \textbf{product} of $a$ and $x$, respectively.
\end{definition}

\begin{definition}\label{def:1.2.2}
    The elements of the field $F$ are called \textbf{scalars} and the elements of the vector space $V$ are called \textbf{vectors}.
\end{definition}

\begin{note}
    A vector space is frequently discussed in the text without explicitly mentioning its field of scalars.
    The reader is cautioned to remember, however, that \emph{every vector space is regarded as a vector space over a given field, which is denoted by $F$}.
    Occasionally we restrict our attention to the fields of real and complex numbers, which are denoted $\mathbf{R}$ and $\mathbf{C}$, respectively.
\end{note}

\begin{note}
    \ref{1.2.1.2} permits us to unambiguously define the addition of any finite number of vectors
    (without the use of parentheses).
\end{note}

\begin{definition}\label{def:1.2.3}
    An object of the form $(a_1, a_2,  \dots, a_n)$, where the entries $a_1, a_2, \dots, a_n$ are elements of a field $F$, is called an \textbf{$n$-tuple} with entries from $F$.
    The elements $a_1, a_2, \dots, a_n$ are called the \textbf{entries} or \textbf{components} of the $n$-tuple.
    Two $n$-tuples $(a_1, a_2, \dots, a_n)$ and $(b_1, b_2, \dots, b_n)$ with entries from a field $F$ are called \textbf{equal} if $a_i = b_i$ for $i = 1, 2, \dots, n$.
\end{definition}

\begin{example}\label{eg:1.2.1}
    The set of all $n$-tuples with entries from a field $F$ is denoted by $F^n$.
    This set is a vector space over $F$ with the operations of coordinatewise addition and scalar multiplication;
    that is, if $u = (a_1, a_2, \dots, a_n) \in F^n$, $v = (b_1, b_2  \dots, b_n) \in F^n$, and $c \in F$ , then
\end{example}

\section{Vector Spaces}\label{sec:1.2}

\begin{defn}\label{1.2.1}
    A \textbf{vector space} (or \textbf{linear space}) \(\V\) over a field \(\F\) consists of a set on which two operations (called \textbf{addition} and \textbf{scalar multiplication}, respectively) are defined so that for each pair of elements \(x\), \(y\) in \(\V\) there is a unique element \(x + y\) in \(\V\), and for each element \(a\) in \(\F\) and each element \(x\) in \(\V\) there is a unique element \(ax\) in \(\V\), such that the following conditions hold.
    \begin{enumerate}[label=(VS \arabic*), ref=VS \arabic*]
        \item\label{vs1}
        For all \(x, y\) in \(\V\), \(x + y = y + x\)
        (commutativity of addition).
        \item\label{vs2}
        For all \(x, y, z\) in \(\V\), \(\p{x + y} + z = x + \p{y + z}\)
        (associativity of addition).
        \item\label{vs3}
        There exists an element in \(\V\) denoted by \(0\) such that \(x + 0 = x\) for each \(x\) in \(\V\).
        \item\label{vs4}
        For each element \(x\) in \(\V\) there exists an element \(y\) in \(\V\) such that \(x + y = 0\).
        \item\label{vs5}
        For each element \(x\) in \(\V\), \(1x = x\).
        \item\label{vs6}
        For each pair of elements \(a, b\) in \(\F\) and each element \(x\) in \(\V\), \(\p{ab} x = a \p{bx}\).
        \item\label{vs7}
        For each element \(a\) in \(\F\) and each pair of elements \(x, y\) in \(\V\), \(a \p{x + y} = ax + ay\).
        \item\label{vs8}
        For each pair of elements \(a, b\) in \(\F\) and each element \(x\) in \(\V\), \(\p{a + b} x = ax + bx\).
    \end{enumerate}
    The elements \(x + y\) and \(ax\) are called the \textbf{sum} of \(x\) and \(y\) and the \textbf{product} of \(a\) and \(x\), respectively.
\end{defn}

\begin{defn}\label{1.2.2}
    The elements of the field \(\F\) are called \textbf{scalars} and the elements of the vector space \(\V\) are called \textbf{vectors}.
\end{defn}

\begin{note}
    A vector space is frequently discussed in the text without explicitly mentioning its field of scalars.
    The reader is cautioned to remember, however, that \emph{every vector space is regarded as a vector space over a given field, which is denoted by \(\F\)}.
    Occasionally we restrict our attention to the fields of real and complex numbers, which are denoted \(\R\) and \(\C\), respectively.
\end{note}

\begin{note}
    \ref{vs2} permits us to unambiguously define the addition of any finite number of vectors
    (without the use of parentheses).
\end{note}

\begin{defn}\label{1.2.3}
    An object of the form \(\tp{a}{n}\), where the entries \(a_{1}, a_{2}, \dots, a_{n}\) are elements of a field \(\F\), is called an \textbf{\(n\)-tuple} with entries from \(\F\).
    The elements \(a_{1}, a_{2}, \dots, a_{n}\) are called the \textbf{entries} or \textbf{components} of the \(n\)-tuple.
    Two \(n\)-tuples \(\tp{a}{n}\) and \(\tp{b}{n}\) with entries from a field \(\F\) are called \textbf{equal} if \(a_i = b_i\) for \(i = 1, 2, \dots, n\).
\end{defn}

\begin{eg}\label{1.2.4}
    The set of all \(n\)-tuples with entries from a field \(\F\) is denoted by \(\vs{F}^{n}\).
    This set is a vector space over \(\F\) with the operations of coordinatewise addition and scalar multiplication;
    that is, if \(u = \tp{a}{n} \in \vs{F}^{n}\), \(v = \tp{b}{n} \in \vs{F}^{n}\), and \(c \in \F\), then
    \[
        u + v = (a_{1} + b_{1}, a_{2} + b_{2}, \dots, a_{n} + b_{n}) \quad \text{and} \quad cu = \tp{ca}{n}.
    \]
\end{eg}

\begin{proof}
    Clearly we have
    \[
        \forall u, v \in \vs{F}^{n}, u + v \in \vs{F}^{n}
    \]
    and
    \[
        \begin{dcases}
            \forall u \in \vs{F}^{n} \\
            \forall c \in \F
        \end{dcases}, cu \in \vs{F}^{n}.
    \]

    Let \(I_{n} = \set{i \in \N : 1 \leq i \leq n}\).
    First we show that addition in \(\vs{F}^{n}\) over \(\F\) is unique.
    Let \(u, v \in \vs{F}^{n}\).
    Suppose for sake of contradiction that
    \[
        \begin{dcases}
            u + v = w \in \vs{F}^{n}  \\
            u + v = w' \in \vs{F}^{n} \\
            w \neq w'
        \end{dcases}.
    \]
    Then we have
    \begin{align*}
                 & w \neq w'                                                                            \\
        \implies & \exists i \in I_{n} : w_i \neq w_i'                    &  & \text{(by \cref{1.2.3})} \\
        \implies & \exists i \in I_{n} : u_{i} + v_{i} \neq u_{i} + v_{i} &  & \text{(by \cref{1.2.4})}
    \end{align*}
    which contradict to the fact that \(u_{i} + v_{i} \in \F\) is unique.
    Thus the addition in \(\vs{F}^{n}\) over \(\F\) is unique.

    Next we show that scalar multiplication in \(\vs{F}^{n}\) over \(\F\) is unique.
    Let \(u \in \vs{F}^{n}\) and \(c \in \F\).
    Suppose for sake of contradiction that
    \[
        \begin{dcases}
            cu = w \in \vs{F}^n  \\
            cu = w' \in \vs{F}^n \\
            w \neq w'
        \end{dcases}.
    \]
    Then we have
    \begin{align*}
                 & w \neq w'                                                                \\
        \implies & \exists i \in I_{n} : w_i \neq w_i'        &  & \text{(by \cref{1.2.3})} \\
        \implies & \exists i \in I_{n} : c u_{i} \neq c u_{i} &  & \text{(by \cref{1.2.4})}
    \end{align*}
    which contradict to the fact that \(c u_{i} \in \F\) is unique.
    Thus the scalar multiplication in \(\vs{F}^{n}\) over \(\F\) is unique.

    Now we show that \ref{vs1}--\ref{vs8} holds for \cref{1.2.4}.
    \begin{description}
        \item[For \ref{vs1}:]
            For all \(u, v \in \vs{F}^{n}\), we have
            \begin{align*}
                u + v & = \p{u_{1} + v_{1}, u_{2} + v_{2}, \dots, u_{n} + v_{n}} &  & \text{(by \cref{1.2.4})}   \\
                      & = \p{v_{1} + u_{1}, v_{2} + u_{2}, \dots, v_{n} + u_{n}} &  & \text{(\(\F\) is a field)} \\
                      & = v + u.                                                 &  & \text{(by \cref{1.2.4})}
            \end{align*}
        \item[For \ref{vs2}:]
            For all \(u, v, w \in \vs{F}^{n}\), we have
            \begin{align*}
                 & \p{u + v} + w                                                                                                                \\
                 & = \p{u_{1} + v_{1}, u_{2} + v_{2}, \dots, u_{n} + v_{n}} + w                                 &  & \text{(by \cref{1.2.4})}   \\
                 & = \p{\p{u_{1} + v_{1}} + w_{1}, \p{u_{2} + v_{2}} + w_{2}, \dots, \p{u_{n} + v_{n}} + w_{n}} &  & \text{(by \cref{1.2.4})}   \\
                 & = \p{u_{1} + \p{v_{1} + w_{1}}, u_{2} + \p{v_{2} + w_{2}}, \dots, u_{n} + \p{v_{n} + w_{n}}} &  & \text{(\(\F\) is a field)} \\
                 & = u + \p{v_{1} + w_{1}, v_{2} + w_{2}, \dots, v_{n} + w_{n}}                                 &  & \text{(by \cref{1.2.4})}   \\
                 & = u + \p{v + w}.                                                                             &  & \text{(by \cref{1.2.4})}
            \end{align*}
        \item[For \ref{vs3}:]
            Since \(\F\) is a field, we know that \(0 \in \F\) and thus \(\p{0, \dots, 0} \in \vs{F}^{n}\).
            Then for all \(u \in \vs{F}^{n}\), we have
            \begin{align*}
                u + \p{0, \dots, 0} & = \p{u_{1} + 0, u_{2} + 0, \dots, u_{n} + 0} &  & \text{(by \cref{1.2.4})}   \\
                                    & = \tp{u}{n}                                  &  & \text{(\(\F\) is a field)} \\
                                    & = u.
            \end{align*}
            We denote \(\zv = \p{0, \dots, 0}\).
        \item[For \ref{vs4}:]
            For all \(u \in \vs{F}^{n}\), we have
            \begin{align*}
                         & \forall i \in I_{n}, u_{i} \in \F                                                                                 \\
                \implies & \forall i \in I_{n}, \exists v_{i} \in \F : u_{i} + v_{i} = 0                  &  & \text{(\(\F\) is a field)}    \\
                \implies & \exists v_{1}, v_{2}, \dots, v_{n} \in \F :                                                                       \\
                         & \p{u_{1} + v_{1}, u_{2} + v_{2}, \dots, u_{n} + v_{n}} = \p{0, \dots, 0} = \zv                                    \\
                \implies & \exists y \in \vs{F}^{n} : u + v = \zv.                                        &  & \text{(from the proof above)}
            \end{align*}
        \item[For \ref{vs5}:]
            Since \(\F\) is a field, we know that \(1 \in \F\).
            Then for all \(u \in \vs{F}^{n}\), we have
            \begin{align*}
                1u & = \p{1 u_{1}, 1 u_{2}, \dots, 1 u_{n}} &  & \text{(by \cref{1.2.4})}   \\
                   & = \tp{u}{n}                            &  & \text{(\(\F\) is a field)} \\
                   & = u.
            \end{align*}
        \item[For \ref{vs6}:]
            For all \(a, b \in \F\) and \(u \in \vs{F}^{n}\), we have
            \begin{align*}
                \p{ab} u & = \p{\p{ab} u_{1}, \p{ab} u_{2}, \dots, \p{ab} u_{n}}    &  & \text{(by \cref{1.2.4})}   \\
                         & = \p{a \p{b u_{1}}, a \p{b u_{2}}, \dots, a \p{b u_{n}}} &  & \text{(\(\F\) is a field)} \\
                         & = a \p{b u_{1}, b u_{2}, \dots, b u_{n}}                 &  & \text{(by \cref{1.2.4})}   \\
                         & = a \p{bu}.                                              &  & \text{(by \cref{1.2.4})}
            \end{align*}
        \item[For \ref{vs7}:]
            For all \(a \in \F\) and \(u, v \in \vs{F}^{n}\), we have
            \begin{align*}
                a \p{u + v} & = a \p{u_{1} + v_{1}, u_{2} + v_{2}, \dots, u_{n} + v_{n}}                    &  & \text{(by \cref{1.2.4})}   \\
                            & = \p{a \p{u_{1} + v_{1}}, a \p{u_{2} + v_{2}}, \dots, a \p{u_{n} + v_{n}}}    &  & \text{(by \cref{1.2.4})}   \\
                            & = \p{a u_{1} + a v_{1}, a u_{2} + a v_{2}, \dots, a u_{n} + a v_{n}}          &  & \text{(\(\F\) is a field)} \\
                            & = \p{a u_{1}, a u_{2}, \dots, a u_{n}} + \p{a v_{1}, a v_{2}, \dots, a v_{n}} &  & \text{(by \cref{1.2.4})}   \\
                            & = a u + a v.                                                                  &  & \text{(by \cref{1.2.4})}
            \end{align*}
        \item[For \ref{vs8}:]
            For all \(a, b \in \F\) and \(u \in \vs{F}^{n}\), we have
            \begin{align*}
                \p{a + b} x & = \p{\p{a + b} u_{1}, \p{a + b} u_{2}, \dots, \p{a + b} u_{n}}                &  & \text{(by \cref{1.2.4})}   \\
                            & = \p{a u_{1} + b u_{1}, a u_{2} + b u_{2}, \dots, a u_{n} + b u_{n}}          &  & \text{(\(\F\) is a field)} \\
                            & = \p{a u_{1}, a u_{2}, \dots, a u_{n}} + \p{b u_{1}, b u_{2}, \dots, b u_{n}} &  & \text{(by \cref{1.2.4})}   \\
                            & = a u + b u.                                                                  &  & \text{(by \cref{1.2.4})}
            \end{align*}
    \end{description}
    From all proofs above we conclude by \cref{1.2.1} that \cref{1.2.4} is indeed a vector space.
\end{proof}

\begin{defn}\label{1.2.5}
    Vectors in \(\F^n\) may be written as \textbf{column vectors}
    \[
        \begin{pmatrix}
            a_{1}  \\
            a_{2}  \\
            \vdots \\
            a_{n}
        \end{pmatrix}
    \]
    rather than as \textbf{row vectors} \(\tp{a}{n}\).
    Since a \(1\)-tuple whose only entry is from \(\F\) can be regarded as an element of \(\F\), we usually write \(\F\) rather than \(\vs{F}^{1}\) for the vector space of \(1\)-tuples with entry from \(\F\).
\end{defn}

\begin{defn}\label{1.2.6}
    An \(m \times n\) \textbf{matrix} with entries from a field \(\F\) is a rectangular array of the form
    \[
        \begin{pmatrix}
            a_{1 1} & a_{1 2} & \dots  & a_{1 n} \\
            a_{2 1} & a_{2 2} & \dots  & a_{2 n} \\
            \vdots  & \vdots  & \ddots & \vdots  \\
            a_{m 1} & a_{m 2} & \dots  & a_{m n}
        \end{pmatrix},
    \]
    where each entry \(a_{i j}\) (\(1 \leq i \leq m\), \(1 \leq j \leq n\)) is an element of \(\F\).
    We call the entries \(a_{i j}\) with \(i = j\) the \textbf{diagonal entries} of the matrix.
    The entries \(a_{i 1} ,a_{i 2} , \dots, a_{i n}\) compose the \textbf{\(i\)th row} of the matrix, and the entries \(a_{1 j}, a_{2 j}, \dots, a_{m j}\) compose the \textbf{\(j\)th column} of the matrix.
    The rows of the preceding matrix are regarded as vectors in \(\vs{F}^{n}\), and the columns are regarded as vectors in \(\vs{F}^{m}\).
    The \(m \times n\) matrix in which each entry equals zero is called the \textbf{zero matrix} and is denoted by \(\zm\).
\end{defn}

\begin{defn}\label{1.2.7}
    In this book, we denote matrices by capital italic letters (e.g., \(A\), \(B\), and \(C\)), and we denote the entry of a matrix \(A\) that lies in row \(i\) and column \(j\) by \(A_{i j}\).
    In addition, if the number of rows and columns of a matrix are equal, the matrix is called \textbf{square}.
\end{defn}

\begin{defn}\label{1.2.8}
    Two \(m \times n\) matrices \(A\) and \(B\) are called \textbf{equal} if all their corresponding entries are equal, that is, if \(A_{i j} = B_{i j}\) for \(1 \leq i \leq m\) and \(1 \leq j \leq n\).
\end{defn}

\begin{eg}\label{1.2.9}
    The set of all \(m \times n\) matrices with entries from a field \(\F\) is a vector space, which we denote by \(\ms{m}{n}{\F}\), with the following operations of \textbf{matrix addition} and \textbf{scalar multiplication}:
    For \(A, B \in \ms{m}{n}{\F}\) and \(c \in \F\),
    \[
        \p{A + B}_{i j} = A_{i j} + B_{i j} \quad \text{and} \quad \p{c A}_{i j} = c A_{i j}
    \]
    for \(1 \leq i \leq m\) and \(1 \leq j \leq n\).
\end{eg}

\begin{proof}
    Clearly we have
    \[
        \forall A, B \in \ms{m}{n}{\F}, A + B \in \ms{m}{n}{\F}
    \]
    and
    \[
        \begin{dcases}
            \forall A \in \ms{m}{n}{\F} \\
            \forall c \in \F
        \end{dcases}, cA \in \ms{m}{n}{\F}.
    \]

    Let \(I_{m} = \set{i \in \N : 1 \leq i \leq m}\) and \(J_{n} = \set{j \in \N : 1 \leq j \leq n}\).
    First we show that addition and scalar multiplication in \(\ms{m}{n}{\F}\) over \(\F\) are unique.
    Suppose that \(A, A', B, B' \in \ms{m}{n}{\F}\) such that \(A = A'\) and \(B = B'\).
    Then we have
    \begin{align*}
                 & \p{A = A'} \land \p{B = B'}                                                                                      \\
        \implies & \forall \p{i, j} \in I_{m} \times J_{n}, \begin{dcases}
            A_{i j} = A_{i j}' \\
            B_{i j} = B_{i j}'
        \end{dcases}              &  & \text{(by \cref{1.2.9})}   \\
        \implies & \forall \p{i, j} \in I_{m} \times J_{n}, A_{i j} + B_{i j} = A_{i j}' + B_{i j}' &  & \text{(\(\F\) is a field)} \\
        \implies & A + B = A' + B'                                                                  &  & \text{(by \cref{1.2.9})}
    \end{align*}
    and thus the addition in \(\ms{m}{n}{\F}\) over \(\F\) is unique.
    Now suppose that \(A, A' \in \ms{m}{n}{\F}\) and \(c, c' \in \F\) such that \(A = A'\) and \(c = c'\).
    Then we have
    \begin{align*}
                 & \p{A = A'} \land \p{c = c'}                                                                      \\
        \implies & \begin{dcases}
            \forall \p{i, j} \in I_{m} \times J_{n}, A_{i j} = A_{i j}' \\
            c = c'
        \end{dcases}                                       &  & \text{(by \cref{1.2.9})}   \\
        \implies & \forall \p{i, j} \in I_{m} \times J_{n}, c A_{i j} = c' A_{i j}' &  & \text{(\(\F\) is a field)} \\
        \implies & cA = c' A'                                                       &  & \text{(by \cref{1.2.9})}
    \end{align*}
    and thus the scalar multiplication in \(\ms{m}{n}{\F}\) over \(\F\) is unique.

    Now we show that \ref{vs1}--\ref{vs8} holds for \cref{1.2.9}.
    \begin{description}
        \item[For \ref{vs1}:]
            For all \(A, B \in \ms{m}{n}{\F}\) and \(\p{i, j} \in I_{m} \times J_{n}\), we have
            \begin{align*}
                \p{A + B}_{i j} & = A_{i j} + B_{i j} &  & \text{(by \cref{1.2.9})}   \\
                                & = B_{i j} + A_{i j} &  & \text{(\(\F\) is a field)} \\
                                & = \p{B + A}_{i j}   &  & \text{(by \cref{1.2.9})}
            \end{align*}
            and thus by \cref{1.2.8} \(A + B = B + A\).
        \item[For \ref{vs2}:]
            For all \(A, B, C \in \ms{m}{n}{\F}\) and \(\p{i, j} \in I_{m} \times J_{n}\), we have
            \begin{align*}
                \p{\p{A + B} + C}_{i j} & = \p{A + B}_{i j} + C_{i + j}     &  & \text{(by \cref{1.2.9})}   \\
                                        & = \p{A_{i j} + B_{i j}} + C_{i j} &  & \text{(by \cref{1.2.9})}   \\
                                        & = A_{i j} + \p{B_{i j} + C_{i j}} &  & \text{(\(\F\) is a field)} \\
                                        & = A_{i j} + \p{B + C}_{i j}       &  & \text{(by \cref{1.2.9})}   \\
                                        & = \p{A + \p{B + C}}_{i j}         &  & \text{(by \cref{1.2.9})}
            \end{align*}
            and thus by \cref{1.2.8} \(\p{A + B} + C = A + \p{B + C}\).
        \item[For \ref{vs3}:]
            Since \(\F\) is a field, we know that \(0 \in \F\) and thus
            \[
                \zm = \begin{pmatrix}
                    0      & 0      & \dots  & 0      \\
                    0      & 0      & \dots  & 0      \\
                    \vdots & \vdots & \ddots & \vdots \\
                    0      & 0      & \dots  & 0
                \end{pmatrix} \in \ms{m}{n}{\F}.
            \]
            Then for all \(A \in \ms{m}{n}{\F}\) and \(\p{i, j} \in I_{m} \times J_{n}\), we have
            \begin{align*}
                \p{A + \zm}_{i j} & = A_{i j} + \zm_{i j} &  & \text{(by \cref{1.2.9})}   \\
                                  & = A_{i j} + 0         &  & \text{(by \cref{1.2.6})}   \\
                                  & = A_{i j}             &  & \text{(\(\F\) is a field)}
            \end{align*}
            and thus by \cref{1.2.8} \(A + \zm = A\).
        \item[For \ref{vs4}:]
            For all \(A \in \ms{m}{n}{\F}\), we have
            \begin{align*}
                         & \forall \p{i, j} \in I_{m} \times J_{n}, A_{i j} \in \F                                                                    \\
                \implies & \forall \p{i, j} \in I_{m} \times J_{n}, \exists B_{i j} \in \F : A_{i j} + B_{i j} = 0 &  & \text{(\(\F\) is a field)}    \\
                \implies & \exists B \in \ms{m}{n}{\F} : A + B = \zm.                                              &  & \text{(from the proof above)}
            \end{align*}
        \item[For \ref{vs5}:]
            Since \(\F\) is a field, we know that \(1 \in \F\).
            Then for all \(A \in \ms{m}{n}{\F}\) and \(\p{i, j} \in I_{m} \times J_{n}\), we have
            \begin{align*}
                \p{1A}_{i j} & = 1 A_{i j} &  & \text{(by \cref{1.2.9})}   \\
                             & = A_{i j}   &  & \text{(\(\F\) is a field)}
            \end{align*}
            and thus by \cref{1.2.8} \(1A = A\).
        \item[For \ref{vs6}:]
            For all \(c, d \in \F\), \(A \in \ms{m}{n}{\F}\) and \(\p{i, j} \in I_{m} \times J_{n}\), we have
            \begin{align*}
                \p{\p{cd} A}_{i j} & = \p{cd} A_{i j}      &  & \text{(by \cref{1.2.9})}   \\
                                   & = c \p{d A_{i j}}     &  & \text{(\(\F\) is a field)} \\
                                   & = c \p{d A}_{i j}     &  & \text{(by \cref{1.2.9})}   \\
                                   & = \p{c \p{d A}}_{i j} &  & \text{(by \cref{1.2.9})}
            \end{align*}
            and thus by \cref{1.2.8} \(\p{cd} A = c \p{dA}\).
        \item[For \ref{vs7}:]
            For all \(c \in \F\), \(A, B \in \ms{m}{n}{\F}\) and \(\p{i, j} \in I_{m} \times J_{n}\), we have
            \begin{align*}
                \p{c \p{A + B}}_{i j} & = c \p{A + B}_{i j}           &  & \text{(by \cref{1.2.9})}   \\
                                      & = c \p{A_{i j} + B_{i j}}     &  & \text{(by \cref{1.2.9})}   \\
                                      & = c A_{i j} + c B_{i j}       &  & \text{(\(\F\) is a field)} \\
                                      & = \p{cA}_{i j} + \p{cB}_{i j} &  & \text{(by \cref{1.2.9})}
            \end{align*}
            and thus by \cref{1.2.8} \(c \p{A + B} = cA + cB\).
        \item[For \ref{vs8}:]
            For all \(c, d \in \F\), \(A \in \ms{m}{n}{\F}\) and \(\p{i, j} \in I_{m} \times J_{n}\), we have
            \begin{align*}
                \p{\p{c + d} A}_{i j} & = \p{c + d} A_{i j}       &  & \text{(by \cref{1.2.9})}   \\
                                      & = c A_{i j} + d A_{i j}   &  & \text{(\(\F\) is a field)} \\
                                      & = (cA)_{i j} + (dA)_{i j} &  & \text{(by \cref{1.2.9})}
            \end{align*}
            and thus by \cref{1.2.8} \(\p{c + d} A = cA + dA\).
    \end{description}
    From all proofs above we conclude by \cref{1.2.1} that \cref{1.2.9} is indeed a vector space.
\end{proof}

\begin{eg}\label{1.2.10}
    Let \(S\) be any nonempty set and \(\F\) be any field, and let \(\mathcal{F}\p{S, \F}\) denote the set of all functions from \(S\) to \(\F\).
    Two functions \(f\) and \(g\) in \(\mathcal{F}\p{S, \F}\) are called \textbf{equal} if \(f(s) = g(s)\) for each \(s \in S\).
    The set \(\mathcal{F}\p{S, \F}\) is a vector space with the operations of addition and scalar multiplication defined for \(f, g \in \mathcal{F}\p{S, \F}\) and \(c \in \F\) by
    \[
        \p{f + g}\p{s} = f\p{s} + g\p{s} \quad \text{and} \quad \p{cf}\p{s} = cf\p{s}
    \]
    for each \(s \in S\).
\end{eg}

\begin{proof}
    Clearly we have
    \[
        \forall f, g \in \mathcal{F}\p{S, \F}, f + g \in \mathcal{F}\p{S, \F}
    \]
    and
    \[
        \begin{dcases}
            \forall f \in \mathcal{F}\p{S, \F} \\
            \forall c \in \F
        \end{dcases}, cf \in \mathcal{F}\p{S, \F}.
    \]

    First we show that addition and scalar multiplication in \(\mathcal{F}\p{S, \F}\) over \(\F\) are unique.
    Suppose that \(f, f', g, g' \in \mathcal{F}\p{S, \F}\) such that \(f = f'\) and \(g = g'\).
    Then we have
    \begin{align*}
                 & \p{f = f'} \land \p{g = g'}                                                          \\
        \implies & \forall s \in S, \begin{dcases}
            f\p{s} = f'\p{s} \\
            g\p{s} = g'\p{s}
        \end{dcases}          &  & \text{(by \cref{1.2.10})}  \\
        \implies & \forall s \in S, f\p{s} + g\p{s} = f'\p{s} + g'\p{s} &  & \text{(\(\F\) is a field)} \\
        \implies & \forall s \in S, (f + g)\p{s} = (f' + g')\p{s}       &  & \text{(by \cref{1.2.10})}  \\
        \implies & f + g = f' + g'                                      &  & \text{(by \cref{1.2.10})}
    \end{align*}
    and thus the addition in \(\mathcal{F}\p{S, \F}\) over \(\F\) is unique.
    Now suppose that \(f, f' \in \mathcal{F}\p{S, \F}\) and \(c, c' \in \F\) such that \(f = f'\) and \(c = c'\).
    Then we have
    \begin{align*}
                 & \p{f = f'} \land \p{c = c'}                                                   \\
        \implies & \begin{dcases}
            \forall s \in S, f\p{s} = f'\p{s} \\
            c = c'
        \end{dcases}                    &  & \text{(by \cref{1.2.10})}  \\
        \implies & \forall s \in S, c f\p{s} = c' f'\p{s}        &  & \text{(\(\F\) is a field)} \\
        \implies & \forall s \in S, \p{cf}\p{s} = \p{c' f'}\p{s} &  & \text{(by \cref{1.2.10})}  \\
        \implies & cf = c' f'                                    &  & \text{(by \cref{1.2.10})}
    \end{align*}
    and thus the scalar multiplication in \(\mathcal{F}\p{S, \F}\) over \(\F\) is unique.

    Now we show that \ref{vs1}--\ref{vs8} holds for \cref{1.2.10}.
    \begin{description}
        \item[For \ref{vs1}:]
            For all \(f, g \in \mathcal{F}\p{S, \F}\) and \(s \in S\), we have
            \begin{align*}
                \p{f + g}\p{s} & = f\p{s} + g\p{s} &  & \text{(by \cref{1.2.10})}  \\
                               & = g\p{s} + f\p{s} &  & \text{(\(\F\) is a field)} \\
                               & = \p{g + f}\p{s}  &  & \text{(by \cref{1.2.10})}
            \end{align*}
            and thus by \cref{1.2.10} \(f + g = g + f\).
        \item[For \ref{vs2}:]
            For all \(f, g, h \in \mathcal{F}\p{S, \F}\) and \(s \in S\), we have
            \begin{align*}
                 & \p{\p{f + g} + h}\p{s}                                         \\
                 & = \p{f + g}\p{s} + h\p{s}      &  & \text{(by \cref{1.2.10})}  \\
                 & = \p{f\p{s} + g\p{s}} + h\p{s} &  & \text{(by \cref{1.2.10})}  \\
                 & = f\p{s} + \p{g\p{s} + h\p{s}} &  & \text{(\(\F\) is a field)} \\
                 & = f\p{s} + \p{g + h}\p{s}      &  & \text{(by \cref{1.2.10})}  \\
                 & = \p{f + \p{g + h}}\p{s}       &  & \text{(by \cref{1.2.10})}
            \end{align*}
            and thus by \cref{1.2.10} \(\p{f + g} + h = f + \p{g + h}\).
        \item[For \ref{vs3}:]
            Since \(\F\) is a field, we know that \(0 \in \F\).
            If we define \(\zv : S \to \{0\}\), then we know that \(\zv \in \F\).
            Thus for all \(f \in \mathcal{F}\p{S, \F}\) and \(s \in S\), we have
            \begin{align*}
                \p{f + \zv}\p{s} & = f\p{s} + \zv\p{s} &  & \text{(by \cref{1.2.10})}  \\
                                 & = f\p{s} + 0                                        \\
                                 & = f\p{s}            &  & \text{(\(\F\) is a field)}
            \end{align*}
            and by \cref{1.2.10} \(f + \zv = f\).
        \item[For \ref{vs4}:]
            For all \(f \in \mathcal{F}\p{S, \F}\), we have
            \begin{align*}
                         & \forall s \in S, f\p{s} \in \F                                                                     \\
                \implies & \forall s \in S, \exists s' \in \F : f\p{s} + s' = 0            &  & \text{(\(\F\) is a field)}    \\
                \implies & \exists g \in \mathcal{F}\p{S, \F} : \begin{dcases}
                    g = s \mapsto s' \\
                    f\p{s} + g\p{s} = 0
                \end{dcases}                                    \\
                \implies & \exists g \in \mathcal{F}\p{S, \F} : \p{f + g}\p{s} = 0         &  & \text{(by \cref{1.2.10})}     \\
                \implies & \exists g \in \mathcal{F}\p{S, \F} : f + g = \zv.               &  & \text{(from the proof above)}
            \end{align*}
        \item[For \ref{vs5}:]
            Since \(\F\) is a field, we know that \(1 \in \F\).
            Then for all \(f \in \mathcal{F}\p{S, \F}\) and \(s \in S\), we have
            \begin{align*}
                \p{1f}\p{s} & = 1f\p{s} &  & \text{(by \cref{1.2.10})}  \\
                            & = f\p{s}  &  & \text{(\(\F\) is a field)}
            \end{align*}
            and thus by \cref{1.2.10} \(1f = f\).
        \item[For \ref{vs6}:]
            For all \(a, b \in \F\), \(f \in \mathcal{F}\p{S, \F}\) and \(s \in S\), we have
            \begin{align*}
                \p{\p{ab} f}\p{s} & = \p{ab} f\p{s}     &  & \text{(by \cref{1.2.10})}  \\
                                  & = a \p{b f\p{s}}    &  & \text{(\(\F\) is a field)} \\
                                  & = a \p{\p{bf}\p{s}} &  & \text{(by \cref{1.2.10})}  \\
                                  & = \p{a \p{bf}}\p{s} &  & \text{(by \cref{1.2.10})}
            \end{align*}
            and thus by \cref{1.2.10} \(\p{ab} f = a \p{bf}\).
        \item[For \ref{vs7}:]
            For all \(a \in \F\), \(f, g \in \mathcal{F}\p{S, \F}\) and \(s \in S\), we have
            \begin{align*}
                \p{a \p{f + g}}\p{s} & = a \p{\p{f + g}\p{s}}      &  & \text{(by \cref{1.2.10})}  \\
                                     & = a \p{f\p{s} + g\p{s}}     &  & \text{(by \cref{1.2.10})}  \\
                                     & = af\p{s} + ag\p{s}         &  & \text{(\(\F\) is a field)} \\
                                     & = \p{af}\p{s} + \p{ag}\p{s} &  & \text{(by \cref{1.2.10})}  \\
                                     & = \p{af + ag}\p{s}          &  & \text{(by \cref{1.2.10})}
            \end{align*}
            and thus by \cref{1.2.10} \(a \p{f + g} = af + ag\).
        \item[For \ref{vs8}:]
            For all \(a, b \in \F\), \(f \in \mathcal{F}\p{S, \F}\) and \(s \in S\), we have
            \begin{align*}
                \p{\p{a + b} f}\p{s} & = \p{a + b} f\p{s}          &  & \text{(by \cref{1.2.10})}  \\
                                     & = af\p{s} + bf\p{s}         &  & \text{(\(\F\) is a field)} \\
                                     & = \p{af}\p{s} + \p{bf}\p{s} &  & \text{(by \cref{1.2.10})}  \\
                                     & = \p{af + bf}\p{s}          &  & \text{(by \cref{1.2.10})}
            \end{align*}
            and thus by \cref{1.2.10} \(\p{a + b} f = af + bf\).
    \end{description}
    From all proofs above we conclude by \cref{1.2.1} that \cref{1.2.10} is indeed a vector space.
\end{proof}

\begin{defn}\label{1.2.11}
    A \textbf{polynomial} with coefficients from a field \(\F\) is an expression of the form
    \[
        f\p{x} = a_{n} x^{n} + a_{n - 1} x^{n - 1} + \cdots + a_{1} x + a_{0},
    \]
    where \(n\) is a nonnegative integer and each \(a_{k}\), called the \textbf{coefficient} of \(x^{k}\), is in \(\F\).
    If \(f\p{x} = 0\), that is, if \(a_{n} = a_{n - 1} = \cdots = a_{0} = 0\), then \(f\p{x}\) is called the \textbf{zero polynomial} and, for convenience, its degree is defined to be \(-1\);
    otherwise, the \textbf{degree} of a polynomial is defined to be the largest exponent of \(x\) that appears in the representation
    \[
        f\p{x} = a_{n} x^{n} + a_{n - 1} x^{n - 1} + \cdots + a_{1} x + a_{0}
    \]
    with a nonzero coefficient.
    Note that the polynomials of degree zero may be written in the form \(f\p{x} = c\) for some nonzero scalar \(c\).
    Two polynomials,
    \[
        f\p{x} = a_{n} x^{n} + a_{n - 1} x^{n - 1} + \cdots + a_{1} x + a_{0}
    \]
    and
    \[
        g\p{x} = b_{m} x^{m} + b_{m - 1} x^{m - 1} + \cdots + b_{1} x + b_{0},
    \]
    are called \textbf{equal} if \(m = n\) and \(a_i = b_i\) for \(i = 0, 1, \dots, n\).
\end{defn}

\begin{eg}\label{1.2.12}
    Let
    \[
        f\p{x} = a_{n} x^{n} + a_{n - 1} x^{n - 1} + \cdots + a_{1} x + a_{0}
    \]
    and
    \[
        g\p{x} = b_{m} x^{m} + b_{m - 1} x^{m - 1} + \cdots + b_{1} x + b_{0}
    \]
    be polynomials with coefficients from a field \(\F\).
    Suppose that \(m \leq n\), and define \(b_{m + 1} = b_{m + 2} = \cdots = b_{n} = 0\).
    Then \(g\p{x}\) can be written as
    \[
        g\p{x} = b_{n} x^{n} + b_{n - 1} x^{n - 1} + \cdots + b_{1} x + b_{0}.
    \]

    Define
    \[
        f\p{x} + g\p{x} = \p{a_{n} + b_{n}} x^{n} + \p{a_{n - 1} + b_{n - 1}} x^{n - 1} + \cdots + \p{a_{1} + b_{1}} x^{1} + \p{a_{0} + b_{0}}
    \]
    and for any scalar \(c \in \F\), define
    \[
        cf\p{x} = ca_{n} x^{n} + ca_{n - 1} x^{n - 1} + \cdots + ca_{1} x + ca_{0}.
    \]
    With these operations of addition and scalar multiplication, the set of all polynomials with coefficients from \(\F\) is a vector space, which we denote by \(\ps{\F}\).
\end{eg}

\begin{proof}
    Observe that \(\ps{\F} \subseteq \mathcal{F}\p{\F, \F}\) (see \cref{1.2.10}).
\end{proof}

\begin{eg}\label{1.2.13}
    Let \(\F\) be any field.
    A sequence in \(\F\) is a function \(\sigma\) from the positive integers into \(\F\).
    In this book, the sequence \(\sigma\) such that \(\sigma\p{n} = a_{n}\) for \(n = 1, 2, \dots\) is denoted \(\set{a_{n}}\).
    Let \(\V\) consist of all sequences \(\set{a_{n}}\) in \(\F\) that have only a finite number of nonzero terms \(a_{n}\).
    If \(\set{a_{n}}\) and \(\set{b_{n}}\) are in \(\V\) and \(t \in \F\), define
    \[
        \set{a_{n}} + \set{b_{n}} = \set{a_{n} + b_{n}} \quad \text{and} \quad t\set{a_{n}} = \set{ta_{n}}.
    \]
    With these operations \(\V\) is a vector space.
\end{eg}

\begin{thm}[Cancellation Law for Vector Addition]\label{1.2.14}
    If \(x, y\), and \(z\) are vectors in a vector space \(\V\) such that \(x + z = y + z\), then \(x = y\).
\end{thm}

\begin{proof}
    There exists a vector \(v\) in \(\V\) such that \(z + v = \zv\) (\ref{vs4}).
    Thus
    \begin{align*}
        x & = x + \zv = x + \p{z + v} = \p{x + z} + v     \\
          & = \p{y + z} + v = y + \p{z + v} = y + \zv = y
    \end{align*}
    by \ref{vs2} and \ref{vs3}.
\end{proof}

\begin{cor}\label{1.2.15}
    The vector \(\zv\) described in \ref{vs3} is unique.
\end{cor}

\begin{proof}
    Suppose for sake of contradiction that
    \[
        \exists \zv' \in V : \begin{dcases}
            \zv \neq \zv' \\
            \forall x \in \V, x + \zv' = x
        \end{dcases}.
    \]
    But then we have
    \begin{align*}
                 & x + \zv = x = x + \zv' &  & \text{(by \ref{vs3})}     \\
        \implies & \zv = \zv',            &  & \text{(by \cref{1.2.14})}
    \end{align*}
    a contradiction.
    Thus \(\zv \in V\) is unique.
\end{proof}

\begin{cor}\label{1.2.16}
    The vector \(y\) described in \ref{vs4} is unique.
\end{cor}

\begin{proof}
    Suppose for sake of contradiction that
    \[
        \forall x \in \V, \exists y, y' \in \V : \begin{dcases}
            y \neq y' \\
            x + y = 0 = x + y'
        \end{dcases}.
    \]
    But then we have
    \begin{align*}
                 & x + y = x + y'                                \\
        \implies & y = y',        &  & \text{(by \cref{1.2.14})}
    \end{align*}
    a contradiction.
    Thus \(y\) described in \ref{vs4} is unique.
\end{proof}

\begin{defn}\label{1.2.17}
    The vector \(\zv\) in \ref{vs3} is called the \textbf{zero vector} of \(\V\), and the vector \(y\) in \ref{vs4} (that is, the unique vector such that \(x + y = \zv\)) is called the \textbf{additive inverse} of \(x\) and is denoted by \(-x\).
\end{defn}

\begin{thm}\label{1.2.18}
    In any vector space \(\V\), the following statements are true:
    \begin{enumerate}
        \item \(0x = \zv\) for each \(x \in \V\).
        \item \(\p{-a}x = -\p{ax} = a\p{-x}\) for each \(a \in \F\) and each \(x \in \V\).
        \item \(a\zv = \zv\) for each \(a \in F\).
    \end{enumerate}
\end{thm}

\begin{proof}{(a)}
    By \ref{vs8}, \ref{vs3}, and \ref{vs1}, it follows that
    \[
        0x + 0x = \p{0 + 0} x = 0x = 0x + \zv = \zv + 0x.
    \]
    Hence \(0x = \zv\) by \cref{1.2.14}.
\end{proof}

\begin{proof}{(b)}
    The vector \(-\p{ax}\) is the unique element of \(\V\) such that \(ax + \bk{-\p{ax}} = \zv\).
    Thus if \(ax + \p{-a}x = 0\), \cref{1.2.16} to \cref{1.2.14} implies that \(\p{-a}x = -\p{ax}\).
    But by \ref{vs8},
    \[
        ax + \p{-a}x = \bk{a + \p{-a}}x = 0x = \zv
    \]
    by \cref{1.2.18}(a).
    Consequently \(\p{-a}x = -\p{ax}\).
    In particular, \(\p{-1}x = -x\).
    So, by \ref{vs6},
    \[
        a\p{-x} = a\bk{\p{-1}x} = \bk{a\p{-1}}x = \p{-a}x.
    \]
\end{proof}

\begin{proof}{(c)}
    We have
    \begin{align*}
        a\zv + \zv & = a\zv           &  & \text{(by \ref{vs3})} \\
                   & = a\p{\zv + \zv} &  & \text{(by \ref{vs3})} \\
                   & = a\zv + a\zv    &  & \text{(by \ref{vs7})}
    \end{align*}
    and thus by \cref{1.2.14} \(a\zv = \zv\).
\end{proof}

\exercisesection

\setcounter{ex}{7}
\begin{ex}\label{ex:1.2.8}
    In any vector space \(\V\), show that \(\p{a + b}\p{x + y} = ax + ay + bx + by\) for any \(x, y \in \V\) and any \(a, b \in \F\).
\end{ex}

\begin{proof}
    We have
    \begin{align*}
        \p{a + b}\p{x + y} & = \p{a + b} x + \p{a + b} y &  & \text{(by \ref{vs7})} \\
                           & = ax + bx + ay + by         &  & \text{(by \ref{vs8})} \\
                           & = ax + ay + bx + by.        &  & \text{(by \ref{vs1})}
    \end{align*}
\end{proof}

\setcounter{ex}{9}
\begin{ex}\label{ex:1.2.10}
    Let \(\V\) denote the set of all differentiable real-valued functions defined on the real line.
    Prove that \(\V\) is a vector space with the operations of addition and scalar multiplication defined in \cref{1.2.10}.
\end{ex}

\begin{ex}\label{ex:1.2.11}
    Let \(\V = \set{\zv}\) consist of a single vector \(\zv\) and define \(\zv + \zv = \zv\) and \(c\zv = \zv\) for each scalar \(c\) in \(\F\).
    Prove that \(\V\) is a vector space over \(\F\).
    (\(\V\) is called the \textbf{zero vector space}.)
\end{ex}

\begin{proof}
    Clearly addition and scalar multiplication result in one unique element in \(\V\), namely \(\zv\).
    We only need to show that \ref{vs1}--\ref{vs8} holds for \cref{ex:1.2.11}.
    \begin{description}
        \item[For \ref{vs1}:]
            By definition we have
            \[
                \zv + \zv = \zv = \zv + \zv.
            \]
        \item[For \ref{vs2}:]
            We have
            \[
                \p{\zv + \zv} + \zv = \zv + \zv = \zv + \p{\zv + \zv}.
            \]
        \item[For \ref{vs3}:]
            By definition we have
            \[
                \zv + \zv = \zv.
            \]
        \item[For \ref{vs4}:]
            By definition we have
            \[
                \zv + \zv = \zv.
            \]
        \item[For \ref{vs5}:]
            By definition we have
            \[
                1\zv = \zv.
            \]
        \item[For \ref{vs6}:]
            For all \(a, b \in \F\), we have
            \begin{align*}
                \p{ab} \zv & = \zv        &  & \text{(by definition)} \\
                           & = b\zv       &  & (b \in \F)             \\
                           & = a\p{b\zv}. &  & (a \in \F)
            \end{align*}
        \item[For \ref{vs7}:]
            For all \(a \in \F\), we have
            \begin{align*}
                a\p{\zv + \zv} & = a\zv         &  & \text{(by definition)} \\
                               & = \zv          &  & \text{(by definition)} \\
                               & = \zv + \zv    &  & \text{(by definition)} \\
                               & = a\zv + a\zv. &  & (a \in \F)
            \end{align*}
        \item[For \ref{vs8}:]
            For all \(a, b \in \F\), \(f \in \mathcal{F}\p{S, \F}\) and \(s \in S\), we have
            \begin{align*}
                \p{a + b} \zv & = \zv          &  & \text{(by definition)} \\
                              & = \zv + \zv    &  & \text{(by definition)} \\
                              & = a\zv + b\zv. &  & (a, b \in \F)
            \end{align*}
    \end{description}
    From all proofs above we conclude by \cref{1.2.1} that \cref{ex:1.2.11} is indeed a vector space.
\end{proof}

\begin{ex}\label{ex:1.2.12}
    A real-valued function \(f\) defined on the real line is called an \textbf{even function} if \(f\p{-t} = f\p{t}\) for each real number \(t\).
    Prove that the set of even functions defined on the real line with the operations of addition and scalar multiplication defined in \cref{1.2.10} is a vector space.
\end{ex}

\setcounter{ex}{20}
\begin{ex}\label{ex:1.2.21}
    Let \(\V\) and \(\W\) be vector spaces over a field \(\F\).
    Let
    \[
        \vs{Z} = \set{\p{v, w} : v \in \V \land w \in \W}.
    \]
    Prove that \(\vs{Z}\) is a vector space over \(\F\) with the operations
    \[
        \p{v_1, w_1} + \p{v_2, w_2} = \p{v_1 + v_2, w_1 + w_2} \quad \text{and} \quad c\p{v_1, w_1} = \p{cv_1, cw_1}.
    \]
\end{ex}

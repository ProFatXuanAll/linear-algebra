\section{Linear Combinations and Systems of Linear Equations}\label{sec:1.4}

\begin{defn}\label{1.4.1}
  Let \(\V\) be a vector space over \(\F\) and \(S\) a nonempty subset of \(\V\).
  A vector \(v \in \V\) is called a \textbf{linear combination} of vectors of \(S\) if there exist a finite number of vectors \(\seq{u}{1,2,,n}\) in \(S\) and scalars \(\seq{a}{1,2,,n}\) in \(\F\) such that \(v = \seq[+]{a,u}{1,2,,n}\).
  In this case we also say that \(v\) is a linear combination of \(\seq{u}{1,2,,n}\) and call \(\seq{a}{1,2,,n}\) the \textbf{coefficients} of the linear combination.
\end{defn}

\begin{eg}\label{1.4.2}
  Observe that in any vector space \(\V\), \(0v = \zv\) for each \(v \in \V\).
  Thus the zero vector is a linear combination of any nonempty subset of \(\V\).
\end{eg}

\begin{defn}\label{1.4.3}
  Let \(S\) be a nonempty subset of a vector space \(\V\) over \(\F\).
  The \textbf{span} of \(S\), denoted \(\spn{S}\), is the set consisting of all linear combinations of the vectors in \(S\).
  For convenience, we define \(\spn{\varnothing} = \set{\zv}\).
\end{defn}

\begin{thm}\label{1.5}
  The span of any subset \(S\) of a vector space \(\V\) over \(\F\) is a subspace of \(\V\) over \(\F\).
  Moreover, any subspace of \(\V\) over \(\F\) that contains \(S\) must also contain the span of \(S\).
\end{thm}

\begin{proof}[\pf{1.5}]
  This result is immediate if \(S = \varnothing\) because \(\spn{\varnothing} = \set{\zv}\), which is a subspace that is contained in any subspace of \(\V\) over \(\F\)
  (see \cref{1.3.2}).

  If \(S \neq \varnothing\), then \(S\) contains a vector \(z\).
  So \(0z = \zv\) is in \(\spn{S}\).
  Let \(x, y \in \spn{S}\).
  Then there exist vectors \(\seq{u}{1,2,,m}\), \(\seq{v}{1,2,,n}\) in \(S\) and scalars \(\seq{a}{1,2,,m}\), \(\seq{b}{1,2,,n}\) in \(\F\) such that
  \[
    x = \seq[+]{a,u}{1,2,,m} \quad \text{and} \quad y = \seq[+]{b,v}{1,2,,n}
  \]
  Then
  \[
    x + y = \seq[+]{a,u}{1,2,,m} + \seq[+]{b,v}{1,2,,n}
  \]
  and, for any scalar \(c \in \F\),
  \[
    cx = (ca_1) u_1 + (ca_2) u_2 + \cdots + (ca_m) u_m
  \]
  are clearly linear combinations of the vectors in \(S\);
  so \(x + y\) and \(cx\) are in \(\spn{S}\).
  Thus \(\spn{S}\) is a subspace of \(\V\) over \(\F\) (by \cref{1.3}).

  Now let \(\W\) denote any subspace of \(\V\) over \(\F\) that contains \(S\).
  If \(w \in \spn{S}\), then \(w\) has the form \(w = \seq[+]{c,w}{1,2,,k}\) for some vectors \(\seq{w}{1,2,,k}\) in \(S\) and some scalars \(\seq{c}{1,2,,k}\).
  Since \(S \subseteq \W\), we have \(\seq{w}{1,2,,k} \in \W\).
  Therefore \(w = \seq[+]{c,w}{1,2,,k}\) is in \(\W\) by \cref{ex:1.3.20} of \cref{sec:1.3}.
  Because \(w\), an arbitrary vector in \(\spn{S}\), belongs to \(\W\), it follows that \(\spn{S} \subseteq \W\).
\end{proof}

\begin{defn}\label{1.4.4}
  A subset \(S\) of a vector space \(\V\) over \(\F\) \textbf{generates} (or \textbf{spans}) \(\V\) if \(\spn{S} = \V\).
  In this case, we also say that the vectors of \(S\) generate (or span) \(\V\).
\end{defn}

\begin{note}
  Usually there are many different subsets that generate a subspace \(\W\).
  It is natural to seek a subset of \(\W\) that generates \(\W\) and is as small as possible.
\end{note}

\exercisesection

\setcounter{ex}{6}
\begin{ex}\label{ex:1.4.7}
  In \(\vs{F}^n\), let \(e_j\) denote the vector whose \(j\)th coordinate is \(1\) and whose other coordinates are \(0\).
  Prove that \(\set{\seq{e}{1,2,,n}}\) generates \(\vs{F}^n\).
\end{ex}

\begin{proof}[\pf{ex:1.4.7}]
  We use induction on \(n\).
  For \(n = 1\), we have
  \begin{align*}
             & \forall x \in \vs{F}^1, x = x \cdot e_1 \in \spn{\set{e_1}} &  & \text{(by \cref{1.4.3})} \\
    \implies & \vs{F}^1 \subseteq \spn{\set{e_1}}                                                        \\
    \implies & \vs{F}^1 = \spn{\set{e_1}}                                  &  & \text{(by \cref{1.5})}
  \end{align*}
  and thus the base case holds.
  Suppose inductively that \(\vs{F}^n = \spn{\set{\seq{e}{1,2,,n}}}\) for some \(n \geq 1\).
  Then for \(n + 1\), we have
  \begin{align*}
             & \forall x \in \vs{F}^{n + 1}, x = \tuple{x}{1,2,,n,n + 1}  &  & \text{(by \cref{1.2.4})}         \\
             & = (\seq{x}{1,2,,n}, 0) + (0, 0, \dots, 0, x_{n + 1})                                             \\
             & = (\seq[+]{x,e}{1,2,,n}) + x_{n + 1} e_{n + 1}             &  & \text{(by induction hypothesis)} \\
             & \in \spn{\set{\seq{x}{1,2,,n,n + 1}}}                      &  & \text{(by \cref{1.4.3})}         \\
    \implies & \vs{F}^{n + 1} \subseteq \spn{\set{\seq{x}{1,2,,n,n + 1}}}                                       \\
    \implies & \vs{F}^{n + 1} = \spn{\set{\seq{x}{1,2,,n,n + 1}}}         &  & \text{(by \cref{1.5})}
  \end{align*}
  and this closes the induction.
\end{proof}

\begin{ex}\label{ex:1.4.8}
  Show that \(\ps[n]{\F}\) is generated by \(\set{1, x, \dots, x_n}\).
\end{ex}

\begin{proof}[\pf{ex:1.4.8}]
  By \cref{1.3.6} we know that \(\ps[n]{\F}\) is a vector space over \(\F\), and we have
  \begin{align*}
             & \forall f \in \ps[n]{\F}, \exists \seq{a}{0,1,2,,n} \in \F :                               \\
             & \forall x \in \F, f(x) = a_0 1 + a_1 x + \cdots + a_n x^n                                  \\
    \implies & \ps[n]{\F} \subseteq \spn{\set{1, x, \dots, x_n}}            &  & \text{(by \cref{1.4.3})} \\
    \implies & \ps[n]{\F} = \spn{\set{1, x, \dots, x_n}}                    &  & \text{(by \cref{1.5})}
  \end{align*}
\end{proof}

\setcounter{ex}{10}
\begin{ex}\label{ex:1.4.11}
  Prove that \(\spn{\set{x}} = \set{ax : a \in \F}\) for any vector \(x\) in a vector space \(\V\) over \(\F\).
  Interpret this result geometrically in \(\R^3\).
\end{ex}

\begin{proof}[\pf{ex:1.4.11}]
  We have
  \begin{align*}
             & \forall v \in \spn{\set{x}}, \begin{dcases}
      \exists \seq{x}{1,2,,n} \in \set{x} \\
      \exists \seq{a}{1,2,,n} \in \F
    \end{dcases} :                                                     \\
             & v = \seq[+]{a,x}{1,2,,n} \in \spn{\set{x}}                                      &  & \text{(by \cref{1.4.3})} \\
    \implies & \forall v \in \spn{\set{x}}, \exists \seq{a}{1,2,,n} \in \F :                                                 \\
             & v = a_1 x + a_2 x + \cdots a_n x = (\seq[+]{a}{1,2,,n}) x                       &  & \text{(by \ref{vs8})}    \\
    \implies & \forall v \in \spn{\set{x}}, \exists a \in \F : v = (a + 0 + \cdots + 0) x = ax                               \\
    \implies & \spn{\set{x}} = \set{ax : a \in \F}.
  \end{align*}

  In \(\R^3\), \(\spn{x}\) is a line in the direction \(x\).
\end{proof}

\begin{ex}\label{ex:1.4.12}
  Show that a subset \(\W\) of a vector space \(\V\) over \(\F\) is a subspace of \(\V\) over \(\F\) if and only if \(\spn{\W} = \W\).
\end{ex}

\begin{proof}[\pf{ex:1.4.12}]
  We have
  \begin{align*}
         & \begin{dcases}
      \W \subseteq \spn{\W} \\
      \W \subseteq \W \text{ and }\W \text{ is a subspace of } \V \text{ over } \F
    \end{dcases} &  & \text{(by \cref{1.4.3})} \\
    \iff & \begin{dcases}
      \W \subseteq \spn{\W} \\
      \spn{\W} \subseteq \W
    \end{dcases} &  & \text{(by \cref{1.5})}   \\
    \iff & \spn{\W} = \W.
  \end{align*}
\end{proof}

\begin{ex}\label{ex:1.4.13}
  Show that if \(S_1\) and \(S_2\) are subsets of a vector space \(\V\) over \(\F\) such that \(S_1 \subseteq S_2\), then \(\spn{S_1} \subseteq \spn{S_2}\).
  In particular, if \(S_1 \subseteq S_2\) and \(\spn{S_1} = \V\), deduce that \(\spn{S_2} = \V\).
\end{ex}

\begin{proof}[\pf{ex:1.4.13}]
  We have
  \begin{align*}
             & \forall v \in \spn{S_1}, \begin{dcases}
      \exists \seq{u}{1,2,,n} \in S_1 \subseteq S_2 \\
      \exists \seq{a}{1,2,,n} \in \F
    \end{dcases} :                               \\
             & v = \seq[+]{a,u}{1,2,,n} \in \spn{S_2}                &  & \text{(by \cref{1.4.3})} \\
    \implies & \spn{S_1} \subseteq \spn{S_2}.
  \end{align*}

  If \(S_1 \subseteq S_2\) and \(\spn{S_1} = \V\), then
  \begin{align*}
             & \V = \spn{S_1} \subseteq \spn{S_2}  &  & \text{(from the proof above)} \\
    \implies & \V \subseteq \spn{S_2} \subseteq \V &  & \text{(by \cref{1.5})}        \\
    \implies & \V = \spn{S_2}.
  \end{align*}
\end{proof}

\begin{ex}\label{ex:1.4.14}
  Show that if \(S_1\) and \(S_2\) are arbitrary subsets of a vector space \(\V\) over \(\F\), then \(\spn{S_1 \cup S_2} = \spn{S_1} + \spn{S_2}\).
  (The sum of two subsets is defined in \cref{1.3.10}.)
\end{ex}

\begin{proof}[\pf{ex:1.4.14}]
  We have
  \begin{align*}
         & v \in \spn{S_1 \cup S_2}                                                                                   \\
    \iff & \begin{dcases}
      \exists \seq{u}{1,2,,n + m} \in S_1 \cup S_2 \\
      \exists \seq{a}{1,2,,n + m} \in \F
    \end{dcases} :                                                                               \\
         & v = \seq[+]{a,u}{1,2,,n + m}                                                &  & \text{(by \cref{1.4.3})}  \\
    \iff & \begin{dcases}
      \exists \seq{u}{1,2,,n} \in S_1             \\
      \exists \seq{u}{n + 1,n + 2,,n + m} \in S_2 \\
      \exists \seq{a}{1,2,,n + m} \in \F
    \end{dcases} :                                                                               \\
         & \begin{dcases}
      v_1 = \seq[+]{a,u}{1,2,,n}             \\
      v_2 = \seq[+]{a,u}{n + 1,n + 2,,n + m} \\
      v = v_1 + v_2
    \end{dcases}                                                  &  & \text{(by \cref{1.4.3})}  \\
    \iff & \exists \tuple{v}{1,2} \in \spn{S_1} \times \spn{S_2} : v = \seq[+]{v}{1,2} &  & \text{(by \cref{1.4.3})}  \\
    \iff & v \in \spn{S_1} + \spn{S_2}                                                 &  & \text{(by \cref{1.3.10})}
  \end{align*}
  and thus \(\spn{S_1 \cup S_2} = \spn{S_1} + \spn{S_2}\).
\end{proof}

\begin{ex}\label{ex:1.4.15}
  Let \(S_1\) and \(S_2\) be subsets of a vector space \(\V\).
  Prove that \\
  \(\spn{S_1 \cap S_2} \subseteq \spn{S_1} \cap \spn{S_2}\).
  Give an example in which \(\spn{S_1 \cap S_2}\) and \(\spn{S_1} \cap \spn{S_2}\) are equal and one in which they are unequal.
\end{ex}

\begin{proof}[\pf{ex:1.4.15}]
  We have
  \begin{align*}
             & \forall v \in \spn{S_1 \cap S_2}, \begin{dcases}
      \exists \seq{u}{1,2,,n} \in S_1 \cap S_2 \\
      \exists \seq{a}{1,2,,n} \in \F
    \end{dcases} :                               \\
             & \begin{dcases}
      \seq[+]{a,u}{1,2,,n} \in \spn{S_1} \\
      \seq[+]{a,u}{1,2,,n} \in \spn{S_2} \\
      v = \seq[+]{a,u}{1,2,,n}
    \end{dcases}                                     &  & \text{(by \cref{1.4.3})} \\
    \implies & \spn{S_1 \cap S_2} \subseteq \spn{S_1} \cap \spn{S_2}.
  \end{align*}

  For equal example, consider \(S_1 = S_2\).
  In this case we have
  \[
    \spn{S_1 \cap S_1} = \spn{S_1} \quad \text{and} \quad \spn{S_1} \cap \spn{S_1} = \spn{S_1}.
  \]

  For unequal example, consider \(S_1 = \set{v_1, v_2}\) and \(S_2 = \set{v_1 + v_2}\) where
  \[
    v_1, v_2 \in \V \setminus \set{\zv} \quad \text{and} \quad v_1 + v_2 \neq \zv.
  \]
  In this case we have
  \[
    \spn{\set{v_1, v_2} \cap \set{v_1 + v_2}} = \spn{\varnothing} = \set{\zv}
  \]
  and
  \[
    \spn{\set{v_1, v_2}} \cap \spn{\set{v_1 + v_2}} = \spn{\set{v_1 + v_2}}.
  \]
\end{proof}

\begin{ex}\label{ex:1.4.16}
  Let \(\V\) be a vector space over \(\F\) and \(S\) a subset of \(\V\) with the property that whenever \(\seq{v}{1,2,,n} \in S\) and \(\seq[+]{a,v}{1,2,,n} = \zv\), then \(\seq[=]{a}{1,2,,n} = 0\).
  Prove that every vector in the span of \(S\) can be \emph{uniquely} written as a linear combination of vectors of \(S\).
\end{ex}

\begin{proof}[\pf{ex:1.4.16}]
  Let \(x \in \spn{S}\).
  If \(x = \zv\), then by hypothesis we know that
  \begin{align*}
             & \begin{dcases}
      \forall \seq{v}{1,2,,n} \in S \\
      \forall \seq{a}{1,2,,n} \in \F
    \end{dcases}, \seq[+]{a,v}{1,2,,n} = \zv \\
    \implies & \seq[=]{a}{1,2,,n} = 0
  \end{align*}
  and thus \(\zv\) can be uniquely written as a linear combination of vectors of \(S\).

  Now suppose that \(x \neq \zv\).
  Suppose for sake of contradiction that \(x\) can be written as two different linear combination of vectors of \(S\).
  By \cref{1.4.3} this means
  \[
    \begin{dcases}
      \exists \seq{v}{1,2,,n} \in S  \\
      \exists \seq{a}{1,2,,n} \in \F \\
      \exists \seq{u}{1,2,,m} \in S  \\
      \exists \seq{b}{1,2,,m} \in \F
    \end{dcases} : \begin{dcases}
      x = \seq[+]{a,v}{1,2,,n} \\
      x = \seq[+]{b,u}{1,2,,m}
    \end{dcases}.
  \]
  But then we have
  \begin{align*}
             & x = \seq[+]{a,v}{1,2,,n}                                       \\
             & \quad = \seq[+]{b,u}{1,2,,m}                                   \\
    \implies & \seq[+]{a,v}{1,2,,n}                                           \\
             & \quad - \seq[+]{b,u}{1,2,,m} = \zv                             \\
    \implies & \seq[=]{a}{1,2,,n}                                             \\
             & \quad = \seq[=]{b}{1,2,,m} = 0     &  & \text{(by hypothesis)} \\
    \implies & x = 0v_1 + 0v_2 + \cdots + 0v_n                                \\
             & = \zv + \zv + \cdots + \zv         &  & \text{(by \cref{1.2})} \\
             & = \zv,                             &  & \text{(by \ref{vs3})}
  \end{align*}
  a contradiction.
  Thus \(x\) can be uniquely written as a linear combination of vectors of \(S\).
\end{proof}

\begin{ex}\label{ex:1.4.17}
  Let \(\W\) be a subspace of a vector space \(\V\) over \(\F\).
  Under what conditions are there only a finite number of distinct subsets \(S\) of \(\W\) such that \(S\) generates \(\W\)?
\end{ex}

\begin{proof}[\pf{ex:1.4.17}]
  The condition is when \(\W\) is a finite set.
  To proof this, suppose for sake of contradiction that \(\W\) is an infinite set and \(\W\) has only a finite number of distinct subsets \(S\) of \(\W\) such that \(\spn{S} = \W\).

  Let \(v \in \W\).
  We claim that \(\spn{\W \setminus \set{v}} = \W\).
  Since \(\W \setminus (\W \setminus \set{v}) = \set{v}\), by \cref{1.4.3} we only need to find a linear combination in \(\W \setminus \set{v}\) which equals to \(v\).
  Now we split into two cases:
  \begin{itemize}
    \item If \(v = \zv\), then we know that there exists a vector \(x\) in \(\W \setminus \set{\zv}\) since \(\W\) is infinite.
          By \ref{vs4} we know that there exist another vector \(y\) in \(\W \setminus \set{\zv}\) such that \(x + y = \zv\).
          Thus we have found a linear combination of \(\zv\) in \(\W \setminus \set{\zv}\).
    \item If \(v \neq \zv\), then by \ref{vs4} we know that \(-v \in \W\).
          By \cref{1.2} we know that \(-(-v) = v\), thus we have found a linear combination of \(v\) in \(\W \setminus \set{v}\).
  \end{itemize}
  From all cases above we conclude that \(\spn{\W \setminus \set{v}} = \W\).
  But \(\W\) is infinite implies we have infinite subset \(\W \setminus \set{v}\) which generate \(\W\), a contradiction.
  Thus \(\W\) must be finite.
\end{proof}

\section{Invertibility and Isomorphisms}\label{sec:2.4}

\begin{defn}\label{2.4.1}
  Let \(\V\) and \(\W\) be vector spaces over \(\F\), and let \(\T : \V \to \W\) be linear.
  A function \(\U : \W \to \V\) is said to be an \textbf{inverse} of \(\T\) if \(\T \U = \IT[\W]\) and \(\U \T = \IT[\V]\).
  If \(\T\) has an inverse, then \(\T\) is said to be \textbf{invertible}.
  If \(\T\) is invertible, then the inverse of \(\T\) is unique and is denoted by \(\T^{-1}\).
  The following facts hold for invertible functions T and U.
  \begin{itemize}
    \item \((\T \U)^{-1} = \U^{-1} \T^{-1}\).
    \item \((\T^{-1})^{-1} = \T\);
          in particular, \(\T^{-1}\) is invertible.
  \end{itemize}
  We often use the fact that a function is invertible iff it is both one-to-one and onto.
  We can therefore restate \cref{2.5} as follows.
  \begin{itemize}
    \item Let \(\T : \V \to \W\) be a linear transformation, where \(\V\) and \(\W\) are finite-dimensional spaces over \(\F\) of equal dimension.
          Then \(\T\) is invertible if and only if \(\rk{\T} = \dim(\V)\).
  \end{itemize}
\end{defn}

\begin{thm}\label{2.17}
  Let \(\V\) and \(\W\) be vector spaces over \(\F\), and let \(\T : \V \to \W\) be linear and invertible.
  Then \(\T^{-1} : \W \to \V\) is linear.
\end{thm}

\begin{proof}[\pf{2.17}]
  Let \(y_1, y_2 \in \W\) and \(c \in \F\).
  Since \(\T\) is onto and one-to-one, there exist unique vectors \(x_1\) and \(x_2\) such that \(\T(x_1) = y_1\) and \(\T(x_2) = y_2\).
  Thus \(x_1 = \T^{-1}(y_1)\) and \(x_2 = \T^{-1}(y_2)\);
  so
  \begin{align*}
    \T^{-1}(c y_1 + y_2) & = \T^{-1}(c \T(x_1) + \T(x_2))   &  & \text{(by \cref{2.4.1})}    \\
                         & = \T^{-1}(\T(c x_1 + x_2))       &  & \text{(by \cref{2.1.2}(b))} \\
                         & = c x_1 + x_2                    &  & \text{(by \cref{2.4.1})}    \\
                         & = c \T^{-1}(x_1) + \T^{-1}(x_2). &  & \text{(by \cref{2.4.1})}
  \end{align*}
\end{proof}

\begin{cor}\label{2.4.2}
  If \(\T\) is a linear transformation between vector spaces of equal (finite) dimension, then the conditions of being invertible, one-to-one, and onto are all equivalent.
\end{cor}

\begin{proof}[\pf{2.4.2}]
  By \cref{2.5} we see that this is true.
\end{proof}

\begin{defn}\label{2.4.3}
  Let \(A \in \ms{n}{n}{\F}\).
  Then \(A\) is \textbf{invertible} if there exists an \(B \in \ms{n}{n}{\F}\) such that \(AB = BA = I_n\).
\end{defn}

\begin{cor}\label{2.4.4}
  If \(A\) is invertible, then the matrix \(B\) such that \(AB = BA = I\) is unique.
  The matrix \(B\) is called the \textbf{inverse} of \(A\) and is denoted by \(A^{-1}\).
\end{cor}

\begin{proof}[\pf{2.4.4}]
  If \(C\) were another such matrix, then
  \[
    C = CI = C(AB) = (CA)B = IB = B.
  \]
  Thus \(B\) is unique.
\end{proof}

\begin{lem}\label{2.4.5}
  Let \(\T\) be an invertible linear transformation from \(\V\) to \(\W\).
  Then \(\V\) is finite-dimensional iff \(\W\) is finite-dimensional.
  In this case, \(\dim(\V) = \dim(\W)\).
\end{lem}

\begin{proof}[\pf{2.4.5}]
  Suppose that \(\V\) is finite-dimensional.
  Let \(\beta = \set{\seq{x}{1,,n}}\) be a basis for \(\V\) over \(\F\).
  By \cref{2.2} \(\T(\beta)\) spans \(\rg{\T} = \W\);
  hence \(\W\) is finite-dimensional by \cref{1.9}.
  Conversely, if \(\W\) is finite-dimensional, then so is \(\V\) by a similar argument, using \(\T^{-1}\).

  Now suppose that \(\V\) and \(\W\) are finite-dimensional.
  Because \(\T\) is one-to-one and onto, we have
  \[
    \nt{\T} = 0 \quad \text{and} \quad \rk{\T} = \dim(\rg{\T}) = \dim(\W).
  \]
  So by the dimension theorem (\cref{2.3}), it follows that \(\dim(\V) = \dim(\W)\).
\end{proof}

\begin{thm}\label{2.18}
  Let \(\V\) and \(\W\) be finite-dimensional vector spaces over \(\F\) with ordered bases \(\beta\) and \(\gamma\) over \(\F\), respectively.
  Let \(\T : \V \to \W\) be linear.
  Then \(\T\) is invertible iff \([\T]_{\beta}^{\gamma}\) is invertible.
  Furthermore, \([\T^{-1}]_{\gamma}^{\beta} = ([\T]_{\beta}^{\gamma})^{-1}\).
\end{thm}

\begin{proof}[\pf{2.18}]
  Suppose that \(\T\) is invertible.
  By \cref{2.4.5}, we have \(\dim(\V) = \dim(\W)\).
  Let \(n = \dim(\V)\).
  So \([\T]_{\beta}^{\gamma} \in \ms{n}{n}{\F}\).
  Now \(\T^{-1} : \W \to \V\) satisfies \(\T \T^{-1} = \IT[\W]\) and \(\T^{-1} \T = \IT[\V]\).
  Thus
  \[
    I_n = [\IT[\V]]_{\beta} = [\T^{-1} \T]_{\beta} = [\T^{-1}]_{\gamma}^{\beta} [\T]_{\beta}^{\gamma}.
  \]
  Similarly, \([\T]_{\beta}^{\gamma} [\T^{-1}]_{\gamma}^{\beta} = I_n\).
  So \([\T]_{\beta}^{\gamma}\) is invertible and \(\pa{[\T]_{\beta}^{\gamma}}^{-1} = [\T^{-1}]_{\gamma}^{\beta}\).

  Now suppose that \(A = [\T]_{\beta}^{\gamma}\) is invertible.
  Then there exists an \(B \in \ms{n}{n}{\F}\) such that \(AB = BA = I_n\).
  By \cref{2.6} there exists \(U \in \ls(\W, \V)\) such that
  \[
    \U(w_j) = \sum_{i = 1}^n B_{i j} v_i \quad \text{for } j \in \set{1, \dots, n},
  \]
  where \(\gamma = \set{\seq{w}{1,,n}}\) and \(\beta = \set{\seq{v}{1,,n}}\).
  It follows that \([\U]_{\gamma}^{\beta} = B\).
  To show that \(\U = \T^{-1}\), observe that
  \[
    [\U \T]_{\beta} = [\U]_{\gamma}^{\beta} [\T]_{\beta}^{\gamma} = BA = I_n = [\IT[\V]]_{\beta}
  \]
  by \cref{2.11}.
  So \(\U \T = \IT[\V]\), and similarly, \(\T \U = \IT[\W]\).
\end{proof}

\begin{cor}\label{2.4.6}
  Let \(\V\) be a finite-dimensional vector space over \(\F\) with an ordered basis \(\beta\), and let \(\T : \V \to \V\) be linear.
  Then \(\T\) is invertible iff \([\T]_{\beta}\) is invertible.
  Furthermore, \([\T^{-1}]_{\beta} = ([\T]_{\beta})^{-1}\).
\end{cor}

\begin{proof}[\pf{2.4.6}]
  This is done by \cref{2.18}.
\end{proof}

\begin{cor}\label{2.4.7}
  Let \(A \in \ms{n}{n}{\F}\).
  Then \(A\) is invertible iff \(\L_A\) is invertible.
  Furthermore, \((\L_A)^{-1} = \L_{A^{-1}}\).
\end{cor}

\begin{proof}[\pf{2.4.7}]
  By \cref{2.15} we know that \(\L_A\) is linear and \([\L_A]_{\beta} = A\) where \(\beta\) is the standard ordered basis for \(\vs{F}^n\) over \(\F\).
  Thus by \cref{2.4.7} \(A\) is invertible iff \([\L_A]_{\beta}\) is invertible iff \(\L_A\) is invertible.
  And we have \([\L_A^{-1}]_{\beta} = ([\L_A]_{\beta})^{-1} = A^{-1} = [\L_{A^{-1}}]_{\beta}\).
  By \cref{2.1.13} this means \((\L_A)^{-1} = \L_{A^{-1}}\).
\end{proof}

\begin{defn}\label{2.4.8}
  Let \(\V\) and \(\W\) be vector spaces over \(\F\).
  We say that \(\V\) is \textbf{isomorphic} to \(\W\) if there exists a linear transformation \(\T : \V \to \W\) that is invertible.
  Such a linear transformation is called an \textbf{isomorphism} from \(\V\) onto \(\W\).
\end{defn}

\begin{note}
  Since ``is isomorphic to'' is an equivalence relation.
  So we need only say that \(\V\) and \(\W\) are isomorphic.
\end{note}

\begin{thm}\label{2.19}
  Let \(\V\) and \(\W\) be finite-dimensional vector spaces (over the same field).
  Then \(\V\) is isomorphic to \(\W\) iff \(\dim(\V) = \dim(\W)\).
\end{thm}

\begin{proof}[\pf{2.19}]
  Suppose that \(\V\) is isomorphic to \(\W\) and that \(\T : \V \to \W\) is an isomorphism from \(\V\) to \(\W\).
  By \cref{2.4.5} we have that \(\dim(\V) = \dim(\W)\).

  Now suppose that \(\dim(\V) = \dim(\W)\), and let \(\beta = \set{\seq{v}{1,,n}}\) and \(\gamma = \set{\seq{w}{1,,n}}\) be bases for \(\V\) and \(\W\) over \(\F\), respectively.
  By \cref{2.6} there exists \(\T : \V \to \W\) such that \(\T\) is linear and \(\T(v_i) = w_i\) for \(i = 1, 2, \dots, n\).
  Using \cref{2.2} we have
  \[
    \rg{\T} = \spn{\T(\beta)} = \spn{\gamma} = \W.
  \]
  So \(\T\) is onto.
  From \cref{2.5} we have that \(\T\) is also one-to-one.
  Hence \(\T\) is an isomorphism.
\end{proof}

\begin{note}
  By \cref{2.4.5} if \(\V\) and \(\W\) are isomorphic, then either both of \(\V\) and \(\W\) are finite-dimensional or both are infinite-dimensional.
\end{note}

\begin{cor}\label{2.4.9}
  Let \(\V\) be a vector space over \(\F\).
  Then \(\V\) is isomorphic to \(\vs{F}^n\) iff \(\dim(\V) = n\).
\end{cor}

\begin{proof}[\pf{2.4.9}]
  We have
  \begin{align*}
         & \V \text{ is isomorphic to } \vs{F}^n &  & \text{(by \cref{2.19})}   \\
    \iff & \dim(\V) = \dim(\vs{F}^n) = n.        &  & \text{(by \cref{1.6.10})}
  \end{align*}
\end{proof}

\begin{thm}\label{2.20}
  Let \(\V\) and \(\W\) be finite-dimensional vector spaces over \(\F\) of dimensions \(n\) and \(m\), respectively, and let \(\beta\) and \(\gamma\) be ordered bases for \(\V\) and \(\W\) over \(\F\), respectively.
  Then the function \(\Phi : \ls(\V, \W) \to \MS\), defined by \(\Phi(\T) = [\T]_{\beta}^{\gamma}\) for \(\T \in \ls(\V, \W)\), is an isomorphism.
\end{thm}

\begin{proof}[\pf{2.20}]
  By \cref{2.8} \(\Phi\) is linear.
  Hence we must show that \(\Phi\) is one-to-one and onto.
  This is accomplished if we show that for every \(A \in \MS\), there exists a unique linear transformation \(\T : \V \to \W\) such that \(\Phi(\T) = A\).
  Let \(\beta = \set{\seq{v}{1,,n}}\), \(\gamma = \set{\seq{w}{1,,m}}\), and let \(A \in \MS\).
  By \cref{2.6} there exists a unique linear transformation \(\T : \V \to \W\) such that
  \[
    \T(v_j) = \sum_{i = 1}^m A_{i j} w_i \quad \text{for } 1 \leq j \leq n.
  \]
  But this means that \([\T]_{\beta}^{\gamma} = A\), or \(\Phi(\T) = A\).
  Thus \(\Phi\) is an isomorphism.
\end{proof}

\begin{cor}\label{2.4.10}
  Let \(\V\) and \(\W\) be finite-dimensional vector spaces over \(\F\) of dimensions \(n\) and \(m\), respectively.
  Then \(\ls(\V, \W)\) is finite-dimensional of dimension \(mn\).
\end{cor}

\begin{proof}[\pf{2.4.10}]
  The proof follows from \cref{2.20} and \cref{2.19} and the fact that \(\dim(\MS) = mn\).
\end{proof}

\begin{defn}\label{2.4.11}
  Let \(\beta\) be an ordered basis for an \(n\)-dimensional vector space \(\V\) over the field \(\F\).
  The \textbf{standard representation of \(\V\) with respect to \(\beta\)} is the function \(\phi_{\beta} : \V \to \vs{F}^n\) defined by \(\phi_{\beta}(x) = [x]_{\beta}\) for each \(x \in \V\).
\end{defn}

\begin{thm}\label{2.21}
  For any finite-dimensional vector space \(\V\) with ordered basis \(\beta\) over \(\F\), \(\phi_{\beta}\) is an isomorphism.
\end{thm}

\begin{proof}[\pf{2.21}]
  Let \(n = \dim(\V)\), let \(\beta = \set{\seq{v}{1,,n}}\), let \(\tuple{a}{1,,n} \in \vs{F}^n\) and let \(x = \sum_{i = 1}^n a_i v_i\).
  Since \(x \in \V\), by \cref{2.2.3} we know that \([x]_{\beta} = \tp{\tuple{a}{1,,n}}\) and thus \(\phi_{\beta}\) is onto.
  Since \(\sum_{i = 1}^n a_i v_i\) is unique, we know that \(\phi_{\beta}\) is one-to-one.
  Thus by \cref{2.4.8} \(\phi_{\beta}\) is an isomorphism from \(\V\) onto \(\vs{\F}^n\).
\end{proof}

\begin{note}
  \cref{2.21} provides us with an alternate proof that an \(n\)-dimensional vector space is isomorphic to \(\vs{F}^n\)
  (see \cref{2.4.9}).
\end{note}

\begin{cor}\label{2.4.12}
  Let \(\V\) and \(\W\) be vector spaces of dimension \(n\) and \(m\) over \(\F\), respectively, and let \(\T : \V \to \W\) be a linear transformation.
  Define \(A = [\T]_{\beta}^{\gamma}\), where \(\beta\) and \(\gamma\) are arbitrary ordered bases of \(\V\) and \(\W\) over \(\F\), respectively.
  We are now able to use \(\phi_{\beta}\) and \(\phi_{\gamma}\) to study the relationship between the linear transformations \(\T\) and \(\L_A : \vs{F}^n \to \vs{F}^m\).
  There are two composites of linear transformations that map \(\V\) into \(\vs{F}^m\):
  \begin{itemize}
    \item Map \(\V\) into \(\vs{F}^n\) with \(\phi_{\beta}\) and follow this transformation with \(\L_A\);
          this yields the composite \(\L_A \phi_{\beta}\).
    \item Map \(\V\) into \(\W\) with \(\T\) and follow it by \(\phi_{\gamma}\) to obtain the composite \(\phi_{\gamma} \T\).
  \end{itemize}
  By a simple reformulation of \cref{2.14}, we may conclude that
  \[
    \L_A \phi_{\beta} = \phi_{\gamma} \T.
  \]
  Heuristically, this relationship indicates that after \(\V\) and \(\W\) are identified with \(\vs{F}^n\) and \(\vs{F}^m\) via \(\phi_{\beta}\) and \(\phi_{\gamma}\), respectively, we may ``identify'' \(\T\) with \(\L_A\).
\end{cor}

\begin{proof}[\pf{2.4.12}]
  For all \(x \in \V\), we have
  \begin{align*}
    (\L_A \phi_{\beta})(x) & = \L_A(\phi_{\beta}(x))                                            \\
                           & = \L_A([x]_{\beta})                 &  & \text{(by \cref{2.4.11})} \\
                           & = A [x]_{\beta}                     &  & \text{(by \cref{2.3.8})}  \\
                           & = [\T]_{\beta}^{\gamma} [x]_{\beta}                                \\
                           & = [\T(x)]_{\gamma}                  &  & \text{(by \cref{2.14})}   \\
                           & = \phi_{\gamma}(\T(x))              &  & \text{(by \cref{2.4.11})} \\
                           & = (\phi_{\gamma} \T)(x).
  \end{align*}
  Thus \(\L_A \phi_{\beta} = \phi_{\gamma} \T\).
\end{proof}

\exercisesection

\setcounter{ex}{3}
\begin{ex}\label{ex:2.4.4}
  Let \(A, B \in \ms{n}{n}{\F}\) be invertible.
  Prove that \(AB\) is invertible and \((AB)^{-1} = B^{-1} A^{-1}\).
\end{ex}

\begin{proof}[\pf{ex:2.4.4}]
  Since
  \begin{align*}
    (AB) (B^{-1} A^{-1}) & = A (B B^{-1}) A^{-1} &  & \text{(by \cref{2.16})}    \\
                         & = A I_n A^{-1}        &  & \text{(by \cref{2.4.3})}   \\
                         & = A A^{-1}            &  & \text{(by \cref{2.12}(c))} \\
                         & = I_n
  \end{align*}
  and
  \begin{align*}
    (B^{-1} A^{-1}) (AB) & = B^{-1} (A^{-1} A) B &  & \text{(by \cref{2.16})}    \\
                         & = B^{-1} I_n B        &  & \text{(by \cref{2.4.3})}   \\
                         & = B^{-1} B            &  & \text{(by \cref{2.12}(c))} \\
                         & = I_n,
  \end{align*}
  by \cref{2.4.3} we see that \(AB\) is invertible and \((AB)^{-1} = B^{-1} A^{-1}\).
\end{proof}

\begin{ex}\label{ex:2.4.5}
  Let \(A\) be invertible.
  Prove that \(\tp{A}\) is invertible and \((\tp{A})^{-1} = \tp{(A^{-1})}\).
\end{ex}

\begin{proof}[\pf{ex:2.4.5}]
  Since
  \begin{align*}
    \tp{A} \tp{(A^{-1})} & = \tp{(A^{-1} A)} &  & \text{(by \cref{2.3.2})} \\
                         & = \tp{I_n}        &  & \text{(by \cref{2.4.3})} \\
                         & = I_n             &  & \text{(by \cref{2.3.4})}
  \end{align*}
  and
  \begin{align*}
    \tp{(A^{-1})} \tp{A} & = \tp{(A A^{-1})} &  & \text{(by \cref{2.3.2})} \\
                         & = \tp{I_n}        &  & \text{(by \cref{2.4.3})} \\
                         & = I_n,            &  & \text{(by \cref{2.3.4})}
  \end{align*}
  by \cref{2.4.3} we see that \(\tp{A}\) is invertible and \((\tp{A})^{-1} = \tp{(A^{-1})}\).
\end{proof}

\begin{ex}\label{ex:2.4.6}
  Prove that if \(A\) is invertible and \(AB = \zm\), then \(B = \zm\).
\end{ex}

\begin{proof}[\pf{ex:2.4.6}]
  We have
  \begin{align*}
             & AB = \zm                                                   \\
    \implies & A^{-1} AB = A^{-1} \zm = \zm &  & \text{(by \cref{2.3.1})} \\
    \implies & IB = B = \zm.                &  & \text{(by \cref{2.3.4})}
  \end{align*}
\end{proof}

\begin{ex}\label{ex:2.4.7}
  Let \(A \in \ms{n}{n}{\F}\).
  \begin{enumerate}
    \item Suppose that \(A^2 = \zm\).
          Prove that \(A\) is not invertible.
    \item Suppose that \(AB = \zm\) for some \(B \in \ms{n}{n}{\F} \setminus \set{\zm}\).
          Could \(A\) be invertible?
          Explain.
  \end{enumerate}
\end{ex}

\begin{proof}[\pf{ex:2.4.7}(a)]
  Suppose for sake of contradiction that \(A\) is invertible.
  Then we have
  \begin{align*}
             & A^2 = \zm                                                   \\
    \implies & A^{-1} A^2 = A^{-1} \zm = \zm &  & \text{(by \cref{2.3.1})} \\
    \implies & (A^{-1} A) A = \zm            &  & \text{(by \cref{2.16})}  \\
    \implies & I_n A = A = \zm.              &  & \text{(by \cref{2.3.4})}
  \end{align*}
  But we know that \(\zm\) is not invertible since \(B \zm = \zm B = \zm \neq I_n\) for all \(B \in \ms{n}{n}{\F}\).
  Thus \(A\) is not invertible.
\end{proof}

\begin{proof}[\pf{ex:2.4.7}(b)]
  \(A\) cannot be invertible for which it contradict to \cref{ex:2.4.6}.
\end{proof}

\setcounter{ex}{8}
\begin{ex}\label{ex:2.4.9}
  Let \(A, B \in \ms{n}{n}{\F}\) such that \(AB\) is invertible.
  Prove that \(A\) and \(B\) are invertible.
  Give an example to show that arbitrary matrices \(A\) and \(B\) need not be invertible if \(AB\) is invertible.
\end{ex}

\begin{proof}[\pf{ex:2.4.9}]
  First we show that if \(A, B \in \ms{n}{n}{\F}\) and \(AB\) is invertible, then \(A, B\) are invertible.
  This is true since
  \begin{align*}
             & AB \text{ is invertible}                                                             \\
    \implies & \L_{AB} \text{ is invertible}                        &  & \text{(by \cref{2.4.7})}   \\
    \implies & \L_A \L_B \text{ is invertible}                      &  & \text{(by \cref{2.15}(e))} \\
    \implies & \L_A \text{ is onto and } \L_B \text{ is one-to-one}                                 \\
    \implies & \L_A \text{ is invertible}                           &  & \text{(by \cref{2.5})}     \\
    \implies & \L_{A^{-1}} \L_A \L_B \text{ is invertible}          &  & \text{(by \cref{2.4.7})}   \\
    \implies & \L_B \text{ is invertible}                           &  & \text{(by \cref{2.5})}     \\
    \implies & A, B \text{ are invertible}.                         &  & \text{(by \cref{2.4.7})}
  \end{align*}

  Now we show that if \(A \in \ms{n}{m}{\F}\) and \(B \in \MS\), then \(AB\) can be invertible.
  We define \(A, B\) as follow:
  \[
    A = \begin{pmatrix}
      1 & 0 & 0 \\
      0 & 1 & 0
    \end{pmatrix} \quad \text{and} \quad B = \begin{pmatrix}
      1 & 0 \\
      0 & 1 \\
      0 & 0
    \end{pmatrix}.
  \]
  Then we see that \(AB = I_2\) which is invertible.
\end{proof}

\begin{ex}\label{ex:2.4.10}
  Let \(A, B \in \ms{n}{n}{\F}\) such that \(AB = I_n\).
  \begin{enumerate}
    \item Use \cref{ex:2.4.9} to conclude that \(A\) and \(B\) are invertible.
    \item Prove \(A = B^{-1}\) (and hence \(B = A^{-1}\)).
          (We are, in effect, saying that for square matrices, a ``one-sided'' inverse is a ``two-sided'' inverse.)
    \item State and prove analogous results for linear transformations defined on finite-dimensional vector spaces.
  \end{enumerate}
\end{ex}

\begin{proof}[\pf{ex:2.4.10}(a)]
  Since \(I_n\) is invertible, by \cref{ex:2.4.9} we see that \(A, B\) are invertible.
\end{proof}

\begin{proof}[\pf{ex:2.4.10}(b)]
  We have
  \begin{align*}
    B^{-1} & = I_n B^{-1} = (AB) B^{-1} &  & \text{(by \cref{2.12}(c))} \\
           & = A (B B^{-1})             &  & \text{(by \cref{2.16})}    \\
           & = A I_n                    &  & \text{(by \cref{2.4.3})}   \\
           & = A                        &  & \text{(by \cref{2.12}(c))}
  \end{align*}
  and
  \begin{align*}
    A^{-1} & = A^{-1} I_n = A^{-1} (AB) &  & \text{(by \cref{2.12}(c))} \\
           & = (A^{-1} A) B             &  & \text{(by \cref{2.16})}    \\
           & = I_n B                    &  & \text{(by \cref{2.4.3})}   \\
           & = B.                       &  & \text{(by \cref{2.12}(c))}
  \end{align*}
\end{proof}

\begin{proof}[\pf{ex:2.4.10}(c)]
  Let \(\V, \W\) be finite dimensional vector spaces over \(\F\) with equal dimension.
  Let \(\T \in \ls(\V, \W), \U \in \ls(\W, \V)\) such that \(\U \T = \IT[\V]\) and \(\T \U = \IT[\W]\).
  We claim that \(\T, \U\) are invertible and \(\T = \U^{-1}\).
  Let \(\beta\) and \(\gamma\) be bases for \(\V\) and \(\W\) over \(\F\), respectively.
  Since
  \begin{align*}
             & \U \T = \IT[\V]                                                                                    \\
    \implies & [\U \T]_{\beta} = [\U]_{\gamma}^{\beta} [\T]_{\beta}^{\gamma} &  & \text{(by \cref{2.11})}         \\
             & = [\IT[\V]]_{\beta} = I_n                                     &  & \text{(by \cref{2.12}(d))}      \\
    \implies & [\U]_{\gamma}^{\beta} = ([\T]_{\beta}^{\gamma})^{-1}          &  & \text{(by \cref{ex:2.4.10}(b))} \\
    \implies & \U = \T^{-1},                                                 &  & \text{(by \cref{2.18})}
  \end{align*}
  we see that \(\T = \U^{-1}\).
\end{proof}

\setcounter{ex}{12}
\begin{ex}\label{ex:2.4.13}
  Let \(\sim\) mean ``is isomorphic to.''
  Prove that \(\sim\) is an equivalence relation on the class of vector spaces over \(\F\).
\end{ex}

\begin{proof}[\pf{ex:2.4.13}]
  To show that \(\sim\) is an equivalence relation, we need to show that \(\sim\) is reflexive, symmetric and transitive.
  \begin{description}
    \item[For reflexive:]
      Let \(\V\) be a vector spaces over \(\F\).
      Since \(\IT[\V] \in \ls(\V)\) is invertible, by \cref{2.4.8} we see that \(\IT[\V]\) is an isomorphism from \(\V\) onto \(\V\).
      Thus \(\V \sim \V\).
    \item[For symmetric:]
      Let \(\V, \W\) be vector spaces over \(\F\) such that \(\V \sim \W\).
      By \cref{2.4.8} there exists a \(\T \in \ls(\V, \W)\) such that \(\T\) is invertible.
      Since \(\T\) is invertible, by \cref{2.17} we know that \(\T^{-1} \in \ls(\W, \V)\) and \(\T^{-1}\) is invertible.
      Thus by \cref{2.4.8} \(\T^{-1}\) is an isomorphism from \(\W\) onto \(\V\) and \(\W \sim \V\).
    \item[For transitive:]
      Let \(\V, \W, \vs{X}\) be vector spaces over \(\F\) such that \(\V \sim \W\) and \(\W \sim \vs{X}\).
      By \cref{2.4.8} there exist \(\T \in \ls(\V, \W)\) and \(\U \in \ls(\W, \vs{X})\) such that \(\T, \U\) are invertible.
      By \cref{2.9} we know that \(\U \T \in \ls(\V, \vs{X})\) is invertible.
      Thus by \cref{2.4.8} \(\U \T\) is an isomorphism from \(\V\) onto \(\vs{X}\) and \(\V \sim \vs{X}\).
  \end{description}
  From proof above we see that \(\sim\) is an equivalence relation.
\end{proof}

\setcounter{ex}{14}
\begin{ex}\label{ex:2.4.15}
  Let \(\V\) and \(\W\) be finite-dimensional vector spaces over \(\F\), and let \(\T : \V \to \W\) be a linear transformation.
  Suppose that \(\beta\) is a basis for \(\V\) over \(\F\).
  Prove that \(\T\) is an isomorphism iff \(\T(\beta)\) is a basis for \(\W\) over \(\F\).
\end{ex}

\begin{proof}[\pf{ex:2.4.15}]
  We have
  \begin{align*}
         & \T \text{ is an isomorphism from } \V \text{ onto } \W                                   \\
    \iff & \begin{dcases}
             \dim(\V) = \dim(\W) \\
             \T \text{ is one-to-one}
           \end{dcases}                               &  & \text{(by \cref{2.19})}                  \\
    \iff & \dim(\W) = \rk{\T}                                     &  & \text{(by \cref{2.5}(c))}    \\
    \iff & \dim(\W) = \dim(\spn{\T(\beta)})                       &  & \text{(by \cref{2.2})}       \\
    \iff & \T(\beta) \text{ is a basis for } \W \text{ over } \F. &  & \text{(by \cref{1.6.15}(a))}
  \end{align*}
\end{proof}

\begin{ex}\label{ex:2.4.16}
  Let \(B \in \ms{n}{n}{\F}\) be invertible.
  Define \(\Phi : \ms{n}{n}{\F} \to \ms{n}{n}{\F}\) by \(\Phi(A) = B^{-1} AB\).
  Prove that \(\Phi\) is an isomorphism.
\end{ex}

\begin{proof}[\pf{ex:2.4.16}]
  First we show that \(\Phi\) is linear.
  Let \(A, C \in \ms{n}{n}{\F}\) and let \(c \in \F\).
  Since
  \begin{align*}
    \Phi(cA + C) & = B^{-1} (cA + C) B                                               \\
                 & = (B^{-1} (cA + C)) B             &  & \text{(by \cref{2.16})}    \\
                 & = ((B^{-1} c A) + (B^{-1} C)) B   &  & \text{(by \cref{2.12}(a))} \\
                 & = (B^{-1} c A) B + (B^{-1} C) B   &  & \text{(by \cref{2.12}(a))} \\
                 & = (c (B^{-1} A)) B + (B^{-1} C) B &  & \text{(by \cref{2.12}(b))} \\
                 & = c ((B^{-1} A) B) + (B^{-1} C) B &  & \text{(by \cref{2.12}(b))} \\
                 & = c (B^{-1} A B) + B^{-1} CB      &  & \text{(by \cref{2.16})}    \\
                 & = c \Phi(A) + \Phi(C),
  \end{align*}
  by \cref{2.1.2}(b) we see that \(\Phi\) is linear.

  Now we show that \(\Phi\) is invertible.
  We define \(\phi : \ms{n}{n}{\F} \to \ms{n}{n}{\F}\) as follow:
  \[
    \forall A \in \ms{n}{n}{\F}, \phi(A) = BAB^{-1}.
  \]
  Since
  \begin{align*}
    \forall A \in \ms{n}{n}{\F}, (\phi \Phi)(A) & = \phi(\Phi(A))                                           \\
                                                & = \phi(B^{-1} AB)                                         \\
                                                & = B(B^{-1} AB) B^{-1}                                     \\
                                                & = (BB^{-1}) A (BB^{-1})   &  & \text{(by \cref{2.16})}    \\
                                                & = I_n A I_n               &  & \text{(by \cref{2.4.4})}   \\
                                                & = A                       &  & \text{(by \cref{2.12}(c))} \\
                                                & = (B^{-1} B) A (B^{-1} B) &  & \text{(by \cref{2.4.4})}   \\
                                                & = B^{-1} (BAB^{-1}) B     &  & \text{(by \cref{2.16})}    \\
                                                & = \Phi(BAB^{-1})                                          \\
                                                & = \Phi(\phi(A))                                           \\
                                                & = (\Phi \phi)(A),
  \end{align*}
  we know that \(\Phi^{-1} = \phi\) and thus by \cref{2.4.8} \(\Phi\) is an isomorphism from \(\ms{n}{n}{\F}\) onto \(\ms{n}{n}{\F}\).
\end{proof}

\begin{ex}\label{ex:2.4.17}
  Let \(\V\) and \(\W\) be finite-dimensional vector spaces over \(\F\) and \(\T : \V \to \W\) be an isomorphism.
  Let \(\V_0\) be a subspace of \(\V\) over \(\F\).
  \begin{enumerate}
    \item Prove that \(\T(\V_0)\) is a subspace of \(\W\) over \(\F\).
    \item Prove that \(\dim(\V_0) = \dim(\T(\V_0))\).
  \end{enumerate}
\end{ex}

\begin{proof}[\pf{ex:2.4.17}(a)]
  Let \(x, y \in \T(\V_0)\) and let \(c \in \F\).
  Since
  \begin{align*}
             & x, y \in \T(\V_0)                                                  \\
    \implies & \exists u, v \in \V_0 : \begin{dcases}
                                         \T(u) = x \\
                                         \T(v) = y
                                       \end{dcases}                              \\
    \implies & \exists u, v \in \V_0 : \begin{dcases}
                                         \T(u + v) = \T(u) + \T(v) = x + y \\
                                         \T(cu) = c \T(u) = cx
                                       \end{dcases} &  & \text{(by \cref{2.1.1})} \\
    \implies & \begin{dcases}
                 x + y \in \T(\V_0) \\
                 cx \in \T(\V_0)
               \end{dcases}                  &  & \text{(by \cref{1.3}(b)(c))}
  \end{align*}
  and
  \begin{align*}
             & \zv_{\V} \in \V_0                     &  & \text{(by \cref{1.3}(a))}   \\
    \implies & \T(\zv_{\V}) = \zv_{\W} \in \T(\V_0), &  & \text{(by \cref{2.1.2}(a))}
  \end{align*}
  by \cref{1.3} we know that \(\T(\V_0)\) is a subspace of \(\W\) over \(\F\).
\end{proof}

\begin{proof}[\pf{ex:2.4.17}(b)]
  Since \(\T\) is an isomorphism from \(\V\) onto \(\W\), by \cref{2.4.8} we know that \(\T\) is invertible.
  Since \(\T\) is invertible, we know that \(\T|_{\V_0}\) is also invertible.
  Since \(\V\) is finite-dimensional, we know that \(\V_0\) is finite-dimensional.
  Thus by \cref{2.4.8} \(\T|_{\V_0}\) is an isomorphism from \(\V_0\) onto \(\T(\V_0)\).
  By \cref{2.19} we have \(\dim(\V_0) = \dim(\T(\V_0))\).
\end{proof}

\setcounter{ex}{19}
\begin{ex}\label{ex:2.4.20}
  Let \(\T : \V \to \W\) be a linear transformation from an \(n\)-dimensional vector space \(\V\) over \(\F\) to an \(m\)-dimensional vector space \(\W\) over \(\F\).
  Let \(\beta\) and \(\gamma\) be ordered bases for \(\V\) and \(\W\) over \(\F\), respectively.
  Prove that \(\rk{\T} = \rk{\L_A}\) and that \(\nt{\T} = \nt{\L_A}\), where \(A = [\T]_{\beta}^{\gamma}\).
\end{ex}

\begin{proof}[\pf{ex:2.4.20}]
  By \cref{2.4.12} we have \(\L_A \phi_{\beta} = \phi_{\gamma} \T\).
  By \cref{2.9,2.21} we know that \(\phi_{\gamma} \T\) is linear.
  Since
  \begin{align*}
             & \phi_{\gamma} \text{ is an isomorphism from } \V \text{ onto } \vs{F}^n &  & \text{(by \cref{2.21})}         \\
    \implies & \rk{\T} = \dim(\T(\V))                                                  &  & \text{(by \cref{2.1.12})}       \\
             & = \dim(\phi_{\gamma}(\T(\V)))                                           &  & \text{(by \cref{ex:2.4.17})}    \\
             & = \dim(\L_A(\phi_{\beta}(\V)))                                          &  & \text{(by \cref{2.4.12})}       \\
             & = \dim(\L_A(\vs{F}^n))                                                  &  & \text{(by \cref{2.21})}         \\
             & = \rk{\L_A},                                                            &  & \text{(by \cref{2.1.12,2.3.8})}
  \end{align*}
  we know that
  \begin{align*}
             & \dim(\V) = \dim(\phi_{\beta}(\V)) = \dim(\vs{F}^n) &  & \text{(by \cref{ex:2.4.17})} \\
    \implies & \rk{\T} + \nt{\T} = \rk{\L_A} + \nt{\L_A}          &  & \text{(by \cref{2.3})}       \\
    \implies & \nt{\T} = \nt{\L_A}.                               &  & (\rk{\T} = \rk{\L_A})
  \end{align*}
\end{proof}

\begin{ex}\label{ex:2.4.21}
  Let \(\V\) and \(\W\) be finite-dimensional vector spaces over \(\F\) with ordered bases \(\beta = \set{\seq{v}{1,,n}}\) and \(\gamma = \set{\seq{w}{1,,m}}\) over \(\F\), respectively.
  By \cref{2.6} there exist linear transformations \(\T_{i j} : \V \to \W\) such that
  \[
    \T_{i j}(v_k) = \begin{dcases}
      w_i      & \text{if } k = j    \\
      \zv_{\W} & \text{if } k \neq j
    \end{dcases}.
  \]
  First prove that \(\set{\T_{i j} : 1 \leq i \leq m, 1 \leq j \leq n}\) is a basis for \(\ls(\V, \W)\) over \(\F\).
  Then let \(E^{i j} \in \MS\) with \(1\) in the \(i\)th row and \(j\)th column and \(0\) elsewhere, and prove that \([\T_{i j}]_{\beta}^{\gamma} = E^{i j}\).
  Again by \cref{2.6}, there exists a linear transformation \(\Phi : \ls(\V, \W) \to \MS\) such that \(\Phi(\T_{i j}) = E^{i j}\).
  Prove that \(\Phi\) is an isomorphism.
\end{ex}

\begin{proof}[\pf{ex:2.4.21}]
  First we show that \(S = \set{\T_{i j} : 1 \leq i \leq m, 1 \leq j \leq n}\) is a basis for \(\ls(\V, \W)\) over \(\F\).
  For all \(i \in \set{1, \dots, m}\) and \(j \in \set{1, \dots, n}\), let \(a_{i j} \in \F\) such that
  \[
    \sum_{i = 1}^m \sum_{j = 1}^n a_{i j} \T_{i j} = \zv.
  \]
  Note that \(\zv\) is the zero function in \(\ls(\V, \W)\).
  Since
  \begin{align*}
             & \forall k \in \set{1, \dots, n}, \T_{i j}(v_k) = \begin{dcases}
                                                                  w_i      & \text{if } k = j    \\
                                                                  \zv_{\W} & \text{if } k \neq j
                                                                \end{dcases}                                                                            \\
    \implies & \forall k \in \set{1, \dots, n}, \T_{i j}(a_{i j} v_k) = a_{i j} \T_{i j}(v_k) = \begin{dcases}
                                                                                                  a_{i k} w_i & \text{if } k = j    \\
                                                                                                  \zv_{\W}    & \text{if } k \neq j
                                                                                                \end{dcases}                              &  & \text{(by \cref{2.1.1})}   \\
    \implies & \forall k \in \set{1, \dots, n}, \sum_{j = 1}^n a_{i j} \T_{i j}(v_k) = a_{i k} w_i                                                                        \\
    \implies & \forall k \in \set{1, \dots, n}, \sum_{i = 1}^m \sum_{j = 1}^n a_{i j} \T_{i j}(v_k) = \sum_{i = 1}^m a_{i k} w_i = \zv_{\W}                               \\
    \implies & \forall (i, k) \in \set{1, \dots, m} \times \set{1, \dots, n}, a_{i k} = 0                                                   &  & \text{(by \cref{1.6.1})}
  \end{align*}
  and \(\#(S) = mn\), by \cref{1.6.15}(a) and \cref{2.4.10} we know that \(S\) is a basis for \(\ls(\V, \W)\) over \(\F\).

  Next we show that \([\T_{i j}]_{\beta}^{\gamma} = E^{i j}\).
  This is true since
  \begin{align*}
             & \forall k \in \set{1, \dots, n}, [\T_{i j}(v_k)]_{\gamma} = \begin{dcases}
                                                                             [w_i]_{\gamma} = e_i                 & \text{if } k = j    \\
                                                                             [\zv_{\W}]_{\gamma} = \zv_{\vs{F}^m} & \text{if } k \neq j
                                                                           \end{dcases} &  & \text{(by \cref{2.2.3})} \\
    \implies & [\T_{i j}]_{\beta}^{\gamma} = E^{i j}.                                       &  & \text{(by \cref{2.2.4})}
  \end{align*}

  Finally we show that \(\Phi\) is an isomorphism from \(\ls(\V, \W)\) onto \(\MS\).
  Let \(\U \in \ls(\V, \W)\).
  Since
  \begin{align*}
             & \U \in \ls(\V, \W)                                                                                                                                 \\
    \implies & \forall (i, j) \in \set{1, \dots, m} \times \set{1, \dots, n}, \exists a_{i j} \in \F :                                                            \\
             & \U = \sum_{i = 1}^m \sum_{j = 1}^n a_{i j} \T_{i j}                                                             &  & \text{(by \cref{1.6.1})}      \\
    \implies & \forall (i, j) \in \set{1, \dots, m} \times \set{1, \dots, n}, \exists a_{i j} \in \F :                                                            \\
             & \Phi(\U) = \sum_{i = 1}^m \sum_{j = 1}^n a_{i j} \Phi(\T_{i j}) = \sum_{i = 1}^m \sum_{j = 1}^n a_{i j} E^{i j} &  & \text{(by \cref{2.1.2}(d))}   \\
             & = \sum_{i = 1}^m \sum_{j = 1}^n a_{i j} [\T_{i j}]_{\beta}^{\gamma}                                             &  & \text{(from the proof above)} \\
             & = \br{\sum_{i = 1}^m \sum_{j = 1}^n a_{i j} \T_{i j}}_{\beta}^{\gamma} = [\U]_{\beta}^{\gamma},                 &  & \text{(by \cref{2.8})}
  \end{align*}
  by \cref{2.20} we see that \(\Phi\) is an isomorphism from \(\ls(\V, \W)\) onto \(\MS\).
\end{proof}

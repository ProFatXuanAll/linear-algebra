\section{Invertibility and Isomorphisms}\label{sec:2.4}

\begin{defn}\label{2.4.1}
  Let \(\V\) and \(\W\) be vector spaces over \(\F\), and let \(\T : \V \to \W\) be linear.
  A function \(\U : \W \to \V\) is said to be an \textbf{inverse} of \(\T\) if \(\T \U = \IT[\W]\) and \(\U \T = \IT[\V]\).
  If \(\T\) has an inverse, then \(\T\) is said to be \textbf{invertible}.
  If \(\T\) is invertible, then the inverse of \(\T\) is unique and is denoted by \(\T^{-1}\).
  The following facts hold for invertible functions T and U.
  \begin{itemize}
    \item \((\T \U)^{-1} = \U^{-1} \T^{-1}\).
    \item \((\T^{-1})^{-1} = \T\);
          in particular, \(\T^{-1}\) is invertible.
  \end{itemize}
  We often use the fact that a function is invertible iff it is both one-to-one and onto.
  We can therefore restate \cref{2.5} as follows.
  \begin{itemize}
    \item Let \(\T : \V \to \W\) be a linear transformation, where \(\V\) and \(\W\) are finite-dimensional spaces over \(\F\) of equal dimension.
          Then \(\T\) is invertible if and only if \(\rk{\T} = \dim(\V)\).
  \end{itemize}
\end{defn}

\begin{thm}\label{2.17}
  Let \(\V\) and \(\W\) be vector spaces over \(\F\), and let \(\T : \V \to \W\) be linear and invertible.
  Then \(\T^{-1} : \W \to \V\) is linear.
\end{thm}

\begin{proof}[\pf{2.17}]
  Let \(y_1, y_2 \in \W\) and \(c \in \F\).
  Since \(\T\) is onto and one-to-one, there exist unique vectors \(x_1\) and \(x_2\) such that \(\T(x_1) = y_1\) and \(\T(x_2) = y_2\).
  Thus \(x_1 = \T^{-1}(y_1)\) and \(x_2 = \T^{-1}(y_2)\);
  so
  \begin{align*}
    \T^{-1}(c y_1 + y_2) & = \T^{-1}(c \T(x_1) + \T(x_2))   &  & \text{(by \cref{2.4.1})}    \\
                         & = \T^{-1}(\T(c x_1 + x_2))       &  & \text{(by \cref{2.1.2}(b))} \\
                         & = c x_1 + x_2                    &  & \text{(by \cref{2.4.1})}    \\
                         & = c \T^{-1}(x_1) + \T^{-1}(x_2). &  & \text{(by \cref{2.4.1})}
  \end{align*}
\end{proof}

\begin{cor}\label{2.4.2}
  If \(\T\) is a linear transformation between vector spaces of equal (finite) dimension, then the conditions of being invertible, one-to-one, and onto are all equivalent.
\end{cor}

\begin{proof}[\pf{2.4.2}]
  By \cref{2.5} we see that this is true.
\end{proof}

\begin{defn}\label{2.4.3}
  Let \(A \in \ms{n}{n}{\F}\).
  Then \(A\) is \textbf{invertible} if there exists an \(B \in \ms{n}{n}{\F}\) such that \(AB = BA = I_n\).
\end{defn}

\begin{cor}\label{2.4.4}
  If \(A\) is invertible, then the matrix \(B\) such that \(AB = BA = I\) is unique.
  The matrix \(B\) is called the \textbf{inverse} of \(A\) and is denoted by \(A^{-1}\).
\end{cor}

\begin{proof}[\pf{2.4.4}]
  If \(C\) were another such matrix, then
  \[
    C = CI = C(AB) = (CA)B = IB = B.
  \]
  Thus \(B\) is unique.
\end{proof}

\begin{lem}\label{2.4.5}
  Let \(\T\) be an invertible linear transformation from \(\V\) to \(\W\).
  Then \(\V\) is finite-dimensional iff \(\W\) is finite-dimensional.
  In this case, \(\dim(\V) = \dim(\W)\).
\end{lem}

\begin{proof}[\pf{2.4.5}]
  Suppose that \(\V\) is finite-dimensional.
  Let \(\beta = \set{\seq{x}{1,,n}}\) be a basis for \(\V\) over \(\F\).
  By \cref{2.2} \(\T(\beta)\) spans \(\rg{\T} = \W\);
  hence \(\W\) is finite-dimensional by \cref{1.9}.
  Conversely, if \(\W\) is finite-dimensional, then so is \(\V\) by a similar argument, using \(T^{-1}\).

  Now suppose that \(\V\) and \(\W\) are finite-dimensional.
  Because \(\T\) is one-to-one and onto, we have
  \[
    \nt{\T} = 0 \quad \text{and} \quad \rk{\T} = \dim(\rg{\T}) = \dim(\W).
  \]
  So by the dimension theorem (\cref{2.3}), it follows that \(\dim(\V) = \dim(\W)\).
\end{proof}

\begin{thm}\label{2.18}
  Let \(\V\) and \(\W\) be finite-dimensional vector spaces over \(\F\) with ordered bases \(\beta\) and \(\gamma\) over \(\F\), respectively.
  Let \(\T : \V \to \W\) be linear.
  Then \(\T\) is invertible iff \([\T]_{\beta}^{\gamma}\) is invertible.
  Furthermore, \([\T^{-1}]_{\gamma}^{\beta} = ([\T]_{\beta}^{\gamma})^{-1}\).
\end{thm}

\begin{proof}[\pf{2.18}]
  Suppose that \(\T\) is invertible.
  By \cref{2.4.5}, we have \(\dim(\V) = \dim(\W)\).
  Let \(n = \dim(\V)\).
  So \([\T]_{\beta}^{\gamma} \in \ms{n}{n}{\F}\).
  Now \(\T^{-1} : \W \to \V\) satisfies \(\T \T^{-1} = \IT[\W]\) and \(\T^{-1} \T = \IT[\V]\).
  Thus
  \[
    I_n = [\IT[\V]]_{\beta} = [\T^{-1} \T]_{\beta} = [\T^{-1}]_{\gamma}^{\beta} [\T]_{\beta}^{\gamma}.
  \]
  Similarly, \([\T]_{\beta}^{\gamma} [\T^{-1}]_{\gamma}^{\beta} = I_n\).
  So \([\T]_{\beta}^{\gamma}\) is invertible and \(\pa{[\T]_{\beta}^{\gamma}}^{-1} = [\T^{-1}]_{\gamma}^{\beta}\).

  Now suppose that \(A = [\T]_{\beta}^{\gamma}\) is invertible.
  Then there exists an \(B \in \ms{n}{n}{\F}\) such that \(AB = BA = I_n\).
  By \cref{2.6} there exists \(U \in \ls(\W, \V)\) such that
  \[
    \U(w_j) = \sum_{i = 1}^n B_{i j} v_i \quad \text{for } j \in \set{1, \dots, n},
  \]
  where \(\gamma = \set{\seq{w}{1,,n}}\) and \(\beta = \set{\seq{v}{1,,n}}\).
  It follows that \([\U]_{\gamma}^{\beta} = B\).
  To show that \(\U = \T^{-1}\), observe that
  \[
    [\U \T]_{\beta} = [\U]_{\gamma}^{\beta} [\T]_{\beta}^{\gamma} = BA = I_n = [\IT[\V]]_{\beta}
  \]
  by \cref{2.11}.
  So \(\U \T = \IT[\V]\), and similarly, \(\T \U = \IT[\W]\).
\end{proof}

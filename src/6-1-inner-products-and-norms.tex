\section{Inner Products and Norms}\label{sec:6.1}

\begin{defn}\label{6.1.1}
  Let \(\V\) be a vector space over \(\F\).
  An \textbf{inner product} on \(\V\) over \(\F\) is a function that assigns, to every ordered pair of vectors \(x\) and \(y\) in \(\V\), a scalar in \(\F\), denoted \(\inn{x, y}\), such that for all \(x\), \(y\), and \(z\) in \(\V\) and all \(c\) in \(\F\), the following hold:
  \begin{enumerate}
    \item \(\inn{x + z, y} = \inn{x, y} + \inn{z, y}\).
    \item \(\inn{cx, y} = c \inn{x, y}\).
    \item \(\conj{\inn{x, y}} = \inn{y, x}\), where the bar denotes complex conjugation.
    \item \(\inn{x, x} > 0\) if \(x \neq \zv\).
  \end{enumerate}
\end{defn}

\begin{note}
  Note that \cref{6.1.1}(c) reduces to \(\inn{x, y} = \inn{y, x}\) if \(\F = \R\).
  \cref{6.1.1}(a)(b) simply require that the inner product be linear in the first component.
  It is easily shown that if \(\seq{a}{1,,n} \in \F\) and \(y, \seq{v}{1,,n} \in \V\), then
  \[
    \inn{\sum_{i = 1}^n a_i v_i, y} = \sum_{i = 1}^n a_i \inn{v_i, y}.
  \]
\end{note}

\begin{eg}\label{6.1.2}
  For \(x = \tuple{a}{1,,n}\) and \(y = \tuple{b}{1,,n}\) in \(\vs{F}^n\), define
  \[
    \inn{x, y} = \sum_{i = 1}^n a_i \conj{b_i}.
  \]
  The inner product defined above is called the \textbf{standard inner product} on \(\vs{F}^n\).
  When \(\F = \R\) the conjugations are not needed, and in early courses this standard inner product is usually called the \emph{dot product} and is denoted by \(x \cdot y\) instead of \(\inn{x, y}\).
\end{eg}

\begin{proof}[\pf{6.1.2}]
  Let \(x, y, z \in \vs{F}^n\) and let \(c \in \F\).
  Since
  \begin{align*}
    \inn{x + y, z}    & = \sum_{i = 1}^n (x + y)_i \conj{z_i}                           &  & \text{(by \cref{6.1.2})}        \\
                      & = \sum_{i = 1}^n x_i \conj{z_i} + \sum_{i = 1}^n y_i \conj{z_i} &  & \text{(by \cref{c.0.1})}        \\
                      & = \inn{x, z} + \inn{y, z}                                       &  & \text{(by \cref{6.1.2})}        \\
    \inn{cx, y}       & = \sum_{i = 1}^n (cx)_i \conj{y_i}                              &  & \text{(by \cref{6.1.2})}        \\
                      & = c \sum_{i = 1}^n x_i \conj{y_i}                               &  & \text{(by \cref{c.0.1})}        \\
                      & = c \inn{x, y}                                                  &  & \text{(by \cref{6.1.2})}        \\
    \conj{\inn{x, y}} & = \conj{\sum_{i = 1}^n x_i \conj{y_i}}                          &  & \text{(by \cref{6.1.2})}        \\
                      & = \sum_{i = 1}^n \conj{x_i} y_i                                 &  & \text{(by \cref{d.2}(a)(b)(c))} \\
                      & = \sum_{i = 1}^n y_i \conj{x_i}                                 &  & \text{(by \cref{c.0.1})}        \\
                      & = \inn{y, x}                                                    &  & \text{(by \cref{6.1.2})}
  \end{align*}
  and
  \begin{align*}
             & x \neq \zv                                                                                       \\
    \implies & \exists i \in \set{1, \dots, n} : x_i \neq 0                                                     \\
    \implies & \exists i \in \set{1, \dots, n} : \abs{x_i}^2 = x_i \conj{x_i} > 0 &  & \text{(by \cref{d.0.5})} \\
    \implies & \inn{x, x} = \sum_{i = 1}^n x_i \conj{x_i} > 0,                    &  & \text{(by \cref{6.1.2})}
  \end{align*}
  by \cref{6.1.1} we know that \(\inn{\cdot, \cdot}\) is an inner product on \(\vs{F}^n\) over \(\F\).
\end{proof}

\begin{eg}\label{6.1.3}
  If \(\inn{x, y}\) is any inner product on a vector space \(\V\) over \(\F\) and \(r > 0\), we may define another inner product by the rule \(\inn{x, y}' = r \inn{x, y}\).
  If \(r \leq 0\), then \cref{6.1.1}(d) would not hold.
\end{eg}

\begin{proof}[\pf{6.1.3}]
  Let \(x, y, z \in \V\), let \(c \in \F\) and let \(r \in \R^+\).
  Since
  \begin{align*}
    \inn{x + y, z}'    & = r\inn{x + y, z}             &  & \text{(by \cref{6.1.3})}    \\
                       & = r (\inn{x, z} + \inn{y, z}) &  & \text{(by \cref{6.1.1}(a))} \\
                       & = r \inn{x, z} + r \inn{y, z} &  & \text{(by \cref{c.0.1})}    \\
                       & = \inn{x, z}' + \inn{y, z}'   &  & \text{(by \cref{6.1.3})}    \\
    \inn{cx, y}'       & = r \inn{cx, y}               &  & \text{(by \cref{6.1.3})}    \\
                       & = rc \inn{x, y}               &  & \text{(by \cref{6.1.1}(b))} \\
                       & = cr \inn{x, y}               &  & \text{(by \cref{c.0.1})}    \\
                       & = c \inn{x, y}'               &  & \text{(by \cref{6.1.3})}    \\
    \conj{\inn{x, y}'} & = \conj{r \inn{x, y}}         &  & \text{(by \cref{6.1.3})}    \\
                       & = \conj{r} \conj{\inn{x, y}}  &  & \text{(by \cref{d.2}(c))}   \\
                       & = \conj{r} \inn{y, x}         &  & \text{(by \cref{6.1.1}(c))} \\
                       & = r \inn{y, x}                &  & (r \in \R^+)                \\
                       & = \inn{y, x}'                 &  & \text{(by \cref{6.1.3})}
  \end{align*}
  and
  \begin{align*}
             & \begin{dcases}
                 x \neq \zv \\
                 r > 0
               \end{dcases}                                                    \\
    \implies & \inn{x, x}' = r \inn{x, x} > 0, &  & \text{(by \cref{6.1.1}(d))}
  \end{align*}
  by \cref{6.1.1} we see that \(\inn{\cdot, \cdot}'\) is an inner product on \(\V\) over \(\F\).
\end{proof}

\begin{eg}\label{6.1.4}
  Let \(\V = \cfs([0, 1], \R)\), the vector space of real-valued continuous functions on \([0, 1]\).
  For \(f, g \in \V\), define \(\inn{f, g} = \int_0^1 f(t) g(t) \; dt\).
  Since the preceding integral is linear in \(f\), \cref{6.1.1}(a)(b) are immediate, and \cref{6.1.1}(c) is trivial.
  If \(f \neq \zv\), then \(f^2\) is bounded away from zero on some subinterval of \([0, 1]\) (continuity is used here), and hence \(\inn{f, f} = \int_0^1 f^2(t) \; dt > 0\).
\end{eg}

\begin{defn}\label{6.1.5}
  Let \(A \in \MS\).
  We define the \textbf{conjugate transpose} or \textbf{adjoint} of \(A\) to be the \(n \times m\) matrix \(A^*\) such that \((A^*)_{i j} = \conj{A_{j i}}\) for all \(i \in \set{1, \dots, m}\) and \(j \in \set{1, \dots, n}\).
\end{defn}

\begin{note}
  If \(x\) and \(y\) are viewed as column vectors in \(\vs{F}^n\), then \(\inn{x, y} = y^* x\) where \(\inn{\cdot, \cdot}\) is the standard inner product.
  The conjugate transpose of a matrix plays a very important role in the remainder of \cref{ch:6}.
  In the case that \(A\) has real entries, \(A^*\) is simply the transpose of \(A\).
\end{note}

\begin{eg}\label{6.1.6}
  Let \(\V = \ms{n}{n}{\F}\), and define \(\inn{A, B} = \tr(B^* A)\) for \(A, B \in \V\).
  Then \(\inn{\cdot, \cdot}\) is called the \textbf{Frobenius inner product} and is an inner product on \(\V\).
\end{eg}

\begin{proof}[\pf{6.1.6}]
  Let \(A, B, C \in \V\) and let \(k \in \F\).
  Then
  \begin{align*}
    \inn{A + B, C}    & = \tr(C^* (A + B))                                                      &  & \text{(by \cref{6.1.6})}     \\
                      & = \tr(C^* A + C^* B)                                                    &  & \text{(by \cref{2.3.5})}     \\
                      & = \tr(C^* A) + \tr(C^* B)                                               &  & \text{(by \cref{ex:1.3.6})}  \\
                      & = \inn{A, C} + \inn{B, C}                                               &  & \text{(by \cref{6.1.6})}     \\
    \inn{kA, B}       & = \tr(B^* (kA))                                                         &  & \text{(by \cref{6.1.6})}     \\
                      & = k \tr(B^* A)                                                          &  & \text{(by \cref{ex:1.3.6})}  \\
                      & = k \inn{A, B}                                                          &  & \text{(by \cref{6.1.6})}     \\
    \conj{\inn{A, B}} & = \conj{\tr(B^* A)}                                                     &  & \text{(by \cref{6.1.6})}     \\
                      & = \conj{\sum_{i = 1}^n (B^* A)_{i i}}                                   &  & \text{(by \cref{1.3.9})}     \\
                      & = \conj{\sum_{i = 1}^n \sum_{j = 1}^n (B^*)_{i j} \cdot A_{j i}}        &  & \text{(by \cref{2.3.1})}     \\
                      & = \sum_{i = 1}^n \sum_{j = 1}^n \conj{(B^*)_{i j}} \cdot \conj{A_{j i}} &  & \text{(by \cref{d.2}(b)(c))} \\
                      & = \sum_{i = 1}^n \sum_{j = 1}^n B_{j i} \cdot (A^*)_{i j}               &  & \text{(by \cref{6.1.5})}     \\
                      & = \sum_{i = 1}^n (A^* B)_{i i}                                          &  & \text{(by \cref{2.3.1})}     \\
                      & = \tr(A^* B)                                                            &  & \text{(by \cref{1.3.9})}     \\
                      & = \inn{B, A}.                                                           &  & \text{(by \cref{6.1.6})}
  \end{align*}
  Also
  \begin{align*}
    \inn{A, A} & = \tr(A^* A)                                           &  & \text{(by \cref{6.1.6})} \\
               & = \sum_{i = 1}^n (A^* A)_{i i}                         &  & \text{(by \cref{1.3.9})} \\
               & = \sum_{i = 1}^n \sum_{k = 1}^n (A^*)_{i k} A_{k i}    &  & \text{(by \cref{2.3.1})} \\
               & = \sum_{i = 1}^n \sum_{k = 1}^n \conj{A_{k i}} A_{k i} &  & \text{(by \cref{6.1.5})} \\
               & = \sum_{i = 1}^n \sum_{k = 1}^n \abs{A_{k i}}^2.       &  & \text{(by \cref{d.0.5})}
  \end{align*}
  Now if \(A \neq \zm\), then \(A_{k i} \neq 0\) for some \(i, k \in \set{1, \dots, n}\).
  So \(\inn{A, A} > 0\).
  Thus by \cref{6.1.1} we see that \(\inn{\cdot, \cdot}\) is an inner product on \(\V\).
\end{proof}

\begin{defn}\label{6.1.7}
  A vector space \(\V\) over \(\F\) endowed with a specific inner product is called an \textbf{inner product space}.
  If \(\F = \C\), we call \(\V\) a \textbf{complex inner product space}, whereas if \(\F = \R\), we call \(\V\) a \textbf{real inner product space}.

  It is clear that if \(\V\) has an inner product \(\inn{x, y}\) and \(\W\) is a subspace of \(\V\) over \(\F\), then \(\W\) is also an inner product space when the same function \(\inn{x, y}\) is restricted to the vectors \(x, y \in \W\).
\end{defn}

\begin{note}
  For the remainder of \cref{ch:6}, \(\vs{F}^n\) denotes the inner product space with the standard inner product as defined in \cref{6.1.2}.
  Likewise, \(\ms{n}{n}{\F}\) denotes the inner product space with the Frobenius inner product as defined in \cref{6.1.6}.
\end{note}

\exercisesection

\begin{ex}\label{ex:6.1.15}

\end{ex}

\section{Subspaces}\label{sec:1.3}

\begin{defn}\label{1.3.1}
    A subset \(\W\) of a vector space \(\V\) over a field \(\F\) is called a \textbf{subspace} of \(\V\) if \(\W\) is a vector space over \(\F\) with the operations of addition and scalar multiplication defined on \(\V\).
\end{defn}

\begin{eg}\label{1.3.2}
    In any vector space \(\V\), note that \(\V\) and \(\set{\zv}\) are subspaces.
    The latter is called the \textbf{zero subspace} of \(\V\).
\end{eg}

\begin{proof}
    Since \(\V \subseteq \V\) and \(\V\) is a vector space over \(\F\) with the operations of addition and scalar multiplication defined on \(\V\), by \cref{1.3.1} we know that \(\V\) is a subspace of \(\V\).
    Since \(\zv \in \V\) (by \ref{vs3}), we know that \(\set{\zv} \subseteq \V\).
    Thus by \cref{ex:1.2.11} and \cref{1.3.1} \(\set{\zv}\) is a subspace of \(\V\).
\end{proof}
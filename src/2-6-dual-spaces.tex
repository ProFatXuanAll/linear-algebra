\section{Dual Spaces}\label{sec:2.6}

\begin{defn}\label{2.6.1}
  In this section, we are concerned exclusively with linear transformations from a vector space \(\V\) into its field of scalars \(\F\), which is itself a vector space of dimension \(1\) over \(\F\).
  Such a linear transformation is called a \textbf{linear functional} on \(\V\).
\end{defn}

\begin{eg}\label{2.6.2}
  Let \(\V\) be the vector space of continuous real-valued functions on the interval \([0, 2\pi]\).
  Fix a function \(g \in \V\).
  The function \(h : \V \to \R\) defined by
  \[
    h(x) = \frac{1}{2\pi} \int_{0}^{2\pi} x(t) g(t) \; dt
  \]
  is a linear functional on \(\V\).
  In the cases that \(g(t)\) equals \(\sin(nt)\) or \(\cos(nt)\), \(h(x)\) is often called the \textbf{\(n\)th Fourier coefficient of \(x\)}.
\end{eg}

\begin{proof}[\pf{2.6.2}]
  Let \(f_1, f_2 \in \V\) and let \(c \in \R\).
  Then we have
  \begin{align*}
    h(cf_1 + f_2) & = \frac{1}{2\pi} \int_{0}^{2\pi} (cf_1 + f_2)(t) g(t) \; dt                                           &  & \text{(by \cref{2.6.2})} \\
                  & = \frac{1}{2\pi} \int_{0}^{2\pi} cf_1(t) g(t) + f_2(t) g(t) \; dt                                                                   \\
                  & = \frac{c}{2\pi} \int_{0}^{2\pi} f_1(t) g(t) \; dt + \frac{1}{2\pi} \int_{0}^{2\pi} f_2(t) g(t) \; dt                               \\
                  & = c h(f_1) + h(f_2)                                                                                   &  & \text{(by \cref{2.6.2})}
  \end{align*}
  and thus by \cref{2.1.2}(b) \(h \in \ls(\V, \F)\).
\end{proof}

\begin{eg}\label{2.6.3}
  The trace function \(\tr : \ms{n}{n}{\F} \to \F\) is a linear functional.
\end{eg}

\begin{proof}[\pf{2.6.3}]
  By \cref{ex:1.3.6} we see that this is true.
\end{proof}

\begin{eg}\label{2.6.4}
  Let \(\V\) be a finite-dimensional vector space over \(\F\), and let \(\beta = \set{\seq{x}{1,,n}}\) be an ordered basis for \(\V\) over \(\F\).
  For each \(i \in \set{1, \dots, n}\), define \(f_i(x) = a_i\), where
  \[
    [x]_{\beta} = \begin{pmatrix}
      a_1    \\
      \vdots \\
      a_n
    \end{pmatrix}
  \]
  is the coordinate vector of \(x\) relative to \(\beta\).
  Then \(f_i\) is a linear functional on \(\V\) called the \textbf{\(i\)th coordinate function with respect to the basis \(\beta\)}.
  Note that \(f_i(x_j) = \delta_{i j}\), where \(\delta_{i j}\) is the Kronecker delta.
  These linear functionals play an important role in the theory of dual spaces (see \cref{2.24}).
\end{eg}

\begin{proof}[\pf{2.6.4}]
  Let \(a, b \in \V\) and let \(c \in \F\).
  By \cref{1.8} there exist some \(\seq{a}{1,,n}, \seq{b}{1,,n} \in \F\) such that
  \[
    a = \sum_{j = 1}^n a_j x_j \quad \text{and} \quad b = \sum_{j = 1}^n b_j x_j.
  \]
  Then we have
  \begin{align*}
    f_i(ca + b) & = f_i\pa{c \pa{\sum_{j = 1}^n a_j x_j} + \sum_{j = 1}^n b_j x_j}                               \\
                & = f_i\pa{\sum_{j = 1}^n (c a_j + b_j) x_j}                       &  & \text{(by \cref{1.2.1})} \\
                & = c a_i + b_i                                                    &  & \text{(by \cref{2.6.4})} \\
                & = c f_i(a) + f_i(b)                                              &  & \text{(by \cref{2.6.4})}
  \end{align*}
  and thus by \cref{2.1.2}(b) \(f_i \in \ls(\V, \F)\).
\end{proof}

\begin{defn}\label{2.6.5}
  For a vector space \(\V\) over \(\F\), we define the \textbf{dual space} of \(\V\) to be the vector space \(\ls(\V, \F)\), denoted by \(\V^*\).
  Thus \(\V^*\) is the vector space consisting of all linear functionals on \(\V\) with the operations of addition and scalar multiplication as defined in \cref{sec:2.2}.
  Note that if \(\V\) is finite-dimensional, then by the \cref{2.4.10}
  \[
    \dim(\V^*) = \dim(\ls(\V, \F)) = \dim(\V) \cdot \dim(\F) = \dim(\V).
  \]
  Hence by \cref{2.19} \(\V\) and \(\V^*\) are isomorphic.
  We also define the \textbf{double dual} \(\V^{**}\) of \(\V\) to be the dual of \(\V^*\).
  In \cref{2.26}, we show, in fact, that there is a natural identification of \(\V\) and \(\V^{**}\) in the case that \(\V\) is finite-dimensional.
\end{defn}

\begin{thm}\label{2.24}
  Suppose that \(\V\) is a finite-dimensional vector space over \(\F\) with the ordered basis \(\beta = \set{\seq{x}{1,,n}}\).
  Let \(f_i\) (\(1 \leq i \leq n\)) be the \(i\)th coordinate function with respect to \(\beta\) as defined in \cref{2.6.4}, and let \(\beta^* = \set{\seq{f}{1,,n}}\).
  Then \(\beta^*\) is an ordered basis for \(\V^*\), and, for any \(f \in \V^*\), we have
  \[
    f = \sum_{i = 1}^n f(x_i) f_i.
  \]
\end{thm}

\begin{proof}[\pf{2.24}]
  Let \(f \in \V^*\).
  Since \(\dim(\V^*) = n\), we need only show that
  \[
    f = \sum_{i = 1}^n f(x_i) f_i,
  \]
  from which it follows that \(\beta^*\) generates \(\V^*\), and hence is a basis by \cref{1.6.15}(a).
  Let
  \[
    g = \sum_{i = 1}^n f(x_i) f_i.
  \]
  For \(1 \leq j \leq n\), we have
  \begin{align*}
    g(x_j) & = \pa{\sum_{i = 1}^n f(x_i) f_i}(x_j)                               \\
           & = \sum_{i = 1}^n f(x_i) f_i(x_j)      &  & \text{(by \cref{2.2.5})} \\
           & = \sum_{i = 1}^n f(x_i) \delta_{i j}  &  & \text{(by \cref{2.6.4})} \\
           & = f(x_j).
  \end{align*}
  Therefore \(f = g\) by \cref{2.1.13}.
\end{proof}

\begin{defn}\label{2.6.6}
  Using the notation of \cref{2.24}, we call the ordered basis \(\beta^* = \set{\seq{f}{1,,n}}\) of \(\V^*\) over \(\F\) that satisfies \(f_i(x_j) = \delta_{i j}\) (\(1 \leq i, j \leq n\)) the \textbf{dual basis} of \(\beta\).
\end{defn}

\begin{thm}\label{2.26}

\end{thm}

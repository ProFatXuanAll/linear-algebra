\section{Diagonalizability}\label{sec:5.2}

\begin{thm}\label{5.5}
  Let \(\T\) be a linear operator on a vector space \(\V\) over \(\F\), and let \(\seq{\lambda}{1,,k}\) be distinct eigenvalues of \(\T\).
  If \(\seq{v}{1,,k}\) are eigenvectors of \(\T\) such that \(\lambda_i\) corresponds to \(v_i\) for all \(i \in \set{1, \dots, k}\), then \(\set{\seq{v}{1,,k}}\) is linearly independent.
\end{thm}

\begin{proof}[\pf{5.5}]
  The proof is by mathematical induction on \(k\).
  Suppose that \(k = 1\).
  Then \(v_1 \neq \zv\) since \(v_1\) is an eigenvector, and hence \(\set{v_1}\) is linearly independent.
  Now assume that the theorem holds for \(k\) distinct eigenvalues, where \(k \geq 1\), and that we have \(k + 1\) eigenvectors \(\seq{v}{1,,k+1}\) corresponding to the distinct eigenvalues \(\seq{\lambda}{1,,k+1}\).
  We wish to show that \(\set{\seq{v}{1,,k+1}}\) is linearly independent.
  Suppose that \(\seq{a}{1,,k+1} \in \F\) such that
  \[
    \seq[+]{a,v}{1,2,,k+1} = \zv.
  \]
  Applying \(\T - \lambda_{k + 1} \IT[\V]\) to both sides of the above equation, we obtain
  \[
    a_1 (\lambda_1 - \lambda_{k + 1}) v_1 + a_2 (\lambda_2 - \lambda_{k + 1}) v_2 + \cdots + a_k (\lambda_k - \lambda_{k + 1}) v_k = \zv.
  \]
  By the induction hypothesis \(\set{\seq{v}{1,,k}}\) is linearly independent, and
  hence
  \[
    a_1 (\lambda_1 - \lambda_{k + 1}) = a_2 (\lambda_2 - \lambda_{k + 1}) = \cdots = a_k (\lambda_k - \lambda_{k + 1}) = 0.
  \]
  Since \(\seq{\lambda}{1,,k+1}\) are distinct, it follows that \(\lambda_i - \lambda_{k + 1} \neq 0\) for \(i \in \set{1, \dots, k}\).
  So \(\seq[=]{a}{1,,k} = 0\), and therefore \(\seq[+]{a,v}{1,,k+1} = \zv\) reduces to \(a_{k + 1} v_{k + 1} = \zv\).
  But \(v_{k + 1} \neq \zv\) and therefore \(a_{k + 1} = 0\).
  Consequently \(\seq[=]{a}{1,,k+1} = 0\), and it follows that \(\set{\seq{v}{1,,k+1}}\) is linearly independent.
\end{proof}

\begin{cor}\label{5.2.1}
  Let \(\T\) be a linear operator on an \(n\)-dimensional vector space \(\V\) over \(\F\).
  If \(\T\) has \(n\) distinct eigenvalues, then \(\T\) is diagonalizable.
\end{cor}

\begin{proof}[\pf{5.2.1}]
  Suppose that \(\T\) has \(n\) distinct eigenvalues \(\seq{\lambda}{1,,n}\).
  For each \(i \in \set{1, \dots, n}\) choose an eigenvector \(v_i\) corresponding to \(\lambda_i\).
  By \cref{5.5}, \(\set{\seq{v}{1,,n}}\) is linearly independent, and since \(\dim(\V) = n\), this set is a basis for \(\V\) over \(\F\).
  Thus by \cref{5.1} \(\T\) is diagonalizable.
\end{proof}

\begin{note}
  The converse of \cref{5.5} is false.
  That is, it is not true that if \(\T\) is diagonalizable, then it has \(n\) distinct eigenvalues.
  For example, the identity operator is diagonalizable even though it has only one eigenvalue, namely, \(\lambda = 1\).
\end{note}

\begin{defn}\label{5.2.2}
  A polynomial \(f\) in \(\ps{\F}\) splits over \(\F\) if there are scalars \(c, \seq{a}{1,,n}\) (not necessarily distinct) in \(\F\) such that
  \[
    f(t) = c (t - a_1) (t - a_2) \cdots (t - a_n).
  \]
  If \(f\) is the characteristic polynomial of a linear operator or a matrix over a field \(\F\), then the statement that \(f\) splits is understood to mean that it splits over \(\F\).
\end{defn}

\begin{thm}\label{5.6}
  The characteristic polynomial of any diagonalizable linear operator splits.
\end{thm}

\begin{proof}[\pf{5.6}]
  Let \(\T\) be a diagonalizable linear operator on the \(n\)-dimensional vector space \(\V\) over \(\F\), and let \(\beta\) be an ordered basis for \(\V\) over \(\F\) such that \([\T]_{\beta} = D\) is a diagonal matrix.
  Suppose that
  \[
    D = \begin{pmatrix}
      \lambda_1 & 0         & \cdots & 0         \\
      0         & \lambda_2 & \cdots & 0         \\
      \vdots    & \vdots    &        & \vdots    \\
      0         & 0         & \cdots & \lambda_n
    \end{pmatrix},
  \]
  and let \(f\) be the characteristic polynomial of \(\T\).
  Then
  \begin{align*}
    f(t) & = \det(D - t I_n)                                                &  & \text{(by \cref{5.1.6})}     \\
         & = \det\begin{pmatrix}
                   \lambda_1 - t & 0             & \cdots & 0             \\
                   0             & \lambda_2 - t & \cdots & 0             \\
                   \vdots        & \vdots        &        & \vdots        \\
                   0             & 0             & \cdots & \lambda_n - t
                 \end{pmatrix}                                       \\
         & = (\lambda_1 - t) (\lambda_2 - t) \cdots (\lambda_n - t)         &  & \text{(by \cref{ex:4.2.23})} \\
         & = (-1)^n (t - \lambda_1) (t - \lambda_2) \cdots (t - \lambda_n).
  \end{align*}
\end{proof}

\begin{note}
  From \cref{5.6}, it is clear that if \(\T\) is a diagonalizable linear operator on an \(n\)-dimensional vector space that fails to have distinct eigenvalues, then the characteristic polynomial of \(\T\) must have repeated zeros.

  The converse of \cref{5.6} is false;
  that is, the characteristic polynomial of \(\T\) may split, but \(\T\) need not be diagonalizable.
\end{note}

\begin{defn}\label{5.2.3}
  Let \(\lambda\) be an eigenvalue of a linear operator or matrix with characteristic polynomial \(f\).
  The \textbf{(algebraic) multiplicity} of \(\lambda\) is the largest positive integer \(k\) for which \((t - \lambda)^k\) is a factor of \(f\).
\end{defn}

\begin{note}
  If \(\T\) is a diagonalizable linear operator on a \(n\)-dimensional vector space \(\V\) over \(\F\), then there is an ordered basis \(\beta\) for \(\V\) over \(\F\) consisting of eigenvectors of \(\T\).
  We know from \cref{5.1} that \([\T]_{\beta}\) is a diagonal matrix in which the diagonal entries are the eigenvalues of \(\T\).
  Since the characteristic polynomial of \(\T\) is \(\det([\T]_{\beta} - t I_n)\), it is easily seen that each eigenvalue of \(\T\) must occur as a diagonal entry of \([\T]_{\beta}\) exactly as many times as its multiplicity.
  Hence \(\beta\) contains as many (linearly independent) eigenvectors corresponding to an eigenvalue as the multiplicity of that eigenvalue.
  So the number of linearly independent eigenvectors corresponding to a given eigenvalue is of interest in determining whether an operator can be diagonalized.
  Recalling from \cref{5.4} that the eigenvectors of \(\T\) corresponding to the eigenvalue \(\lambda\) are the nonzero vectors in the null space of \(\T - \lambda \IT[\V]\), we are led naturally to the study of this set.
\end{note}

\begin{defn}\label{5.2.4}
  Let \(\T\) be a linear operator on a vector space \(\V\) over \(\F\), and let \(\lambda\) be an eigenvalue of \(\T\).
  Define \(E_\lambda = \set{x \in \V : \T(x) = \lambda x} = \ns{\T - \lambda \IT[\V]}\).
  The set \(E_\lambda\) is called the \textbf{eigenspace} of \(\T\) corresponding to the eigenvalue \(\lambda\).
  Analogously, we define the \textbf{eigenspace} of a square matrix \(A\) to be the eigenspace of \(\L_A\).
\end{defn}

\begin{note}
  Clearly, \(E_{\lambda}\) is a subspace of \(\V\) over \(\F\) consisting of the zero vector and the eigenvectors of \(\T\) corresponding to the eigenvalue \(\lambda\).
  The maximum number of linearly independent eigenvectors of \(\T\) corresponding to the eigenvalue \(\lambda\) is therefore the dimension of \(E_{\lambda}\).
  \cref{5.7} relates this dimension to the multiplicity of \(\lambda\).
\end{note}

\begin{thm}\label{5.7}
  Let \(\T\) be a linear operator on a finite-dimensional vector space \(\V\) over \(\F\), and let \(\lambda\) be an eigenvalue of \(\T\) having multiplicity \(m\).
  Then \(1 \leq \dim(E_{\lambda}) \leq m\).
\end{thm}

\begin{proof}[\pf{5.7}]
  Choose an ordered basis \(\set{\seq{v}{1,,p}}\) for \(E_{\lambda}\), extend it to an ordered basis \(\beta = \set{\seq{v}{1,,p,p+1,,n}}\) for \(\V\) over \(\F\), and let \(A = [\T]_{\beta}\).
  Observe that for each \(i \in \set{1, \dots p}\), \(v_i\) is an eigenvector of \(\T\) corresponding to \(\lambda\), and therefore
  \[
    A = \begin{pmatrix}
      \lambda I_p & B \\
      \zm         & C
    \end{pmatrix}.
  \]
  By \cref{ex:4.3.21}, the characteristic polynomial of \(\T\) is
  \begin{align*}
    f(t) & = \det(A - t I_n)                               &  & \text{(by \cref{5.1.6})}     \\
         & = \det\begin{pmatrix}
                   (\lambda - t) I_p & B               \\
                   \zm               & C - t I_{n - p}
                 \end{pmatrix}                                         \\
         & = \det((\lambda - t) I_p) \det(C - t I_{n - p}) &  & \text{(by \cref{ex:4.3.21})} \\
         & = (\lambda - t)^p g(t),                         &  & \text{(by \cref{ex:4.2.23})}
  \end{align*}
  where \(g\) is a polynomial.
  Thus \((\lambda - t)^p\) is a factor of \(f\), and hence the multiplicity of \(\lambda\) is at least \(p\).
  But \(\dim(E_{\lambda}) = p\), and so \(\dim(E_{\lambda}) \leq m\).
\end{proof}

\begin{lem}\label{5.2.5}
  Let \(\T\) be a linear operator, and let \(\seq{\lambda}{1,,k}\) be distinct eigenvalues of \(\T\).
  For each \(i \in \set{1, \dots, k}\), let \(v_i \in E_{\lambda_i}\), the eigenspace corresponding to \(\lambda_i\).
  If
  \[
    \seq[+]{v}{1,,k} = \zv,
  \]
  then \(v_i = \zv\) for all \(i \in \set{1, \dots, k}\).
\end{lem}

\begin{proof}[\pf{5.2.5}]
  Suppose otherwise.
  By renumbering if necessary, suppose that, for \(m \in \set{1, \dots, k}\), we have \(v_i \neq \zv\) for \(i \in \set{1, \dots, m}\), and \(v_i = \zv\) for \(i \in \set{m + 1, \dots, k}\).
  Then, for each \(i \leq m\), \(v_i\) is an eigenvector of \(\T\) corresponding to \(\lambda_i\) and
  \[
    \seq[+]{v}{1,,m} = \zv.
  \]
  But this contradicts \cref{5.5}, which states that these \(v_i\)'s are linearly independent.
  We conclude, therefore, that \(v_i = \zv\) for all \(i \in \set{1, \dots, k}\).
\end{proof}

\begin{thm}\label{5.8}
  Let \(\T\) be a linear operator on a vector space \(\V\) over \(\F\), and let \(\seq{\lambda}{1,,k}\) be distinct eigenvalues of \(\T\).
  For each \(i \in \set{1, \dots, k}\), let \(S_i\) be a finite linearly independent subset of the eigenspace \(E_{\lambda_i}\).
  Then \(S = \bigcup_{i = 1}^k S_i\) is a linearly independent subset of \(\V\).
\end{thm}

\begin{proof}[\pf{5.8}]
  Suppose that for each \(i \in \set{1, \dots, k}\)
  \[
    S_i = \set{v_{i 1}, v_{i 2}, \dots v_{i n_i}}.
  \]
  Then \(S = \set{v_{i j} : (1 \leq j \leq n_i) \land (1 \leq i \leq k)}\).
  Consider any scalars \(\set{a_{i j}} \subseteq \F\) such that
  \[
    \sum_{i = 1}^k \sum_{j = 1}^{n_i} a_{i j} v_{i j} = \zv.
  \]
  For each \(i \in \set{1, \dots, k}\), let
  \[
    w_i = \sum_{j = 1}^{n_i} a_{i j} v_{i j}.
  \]
  Then \(w_i \in E_{\lambda_i}\) for each \(i \in \set{1, \dots, k}\), and \(\seq[+]{w}{1,,k} = \zv\).
  Therefore, by \cref{5.2.5}, \(w_i = \zv\) for all \(i \in \set{1, \dots, k}\).
  But each \(S_i\) is linearly independent, and hence \(a_{i j} = 0\) for all \(j \in \set{1, \dots, n_i}\).
  We conclude that \(S\) is linearly independent.
\end{proof}

\begin{thm}\label{5.9}
  Let \(\T\) be a linear operator on a finite-dimensional vector space \(\V\) over \(\F\) such that the characteristic polynomial of \(\T\) splits.
  Let \(\seq{\lambda}{1,,k}\) be the distinct eigenvalues of \(\T\).
  Then
  \begin{enumerate}
    \item \(\T\) is diagonalizable iff the multiplicity of \(\lambda_i\) is equal to \(\dim(E_{\lambda_i})\) for all \(i \in \set{1, \dots, k}\).
    \item If \(\T\) is diagonalizable and \(\beta_i\) is an ordered basis for \(E_{\lambda_i}\) for each \(i \in \set{1, \dots, k}\), then \(\beta = \bigcup_{i = 1}^k \beta_i\) is an ordered basis for \(\V\) over \(\F\) consisting of eigenvectors of \(\T\).
  \end{enumerate}
\end{thm}

\begin{proof}[\pf{5.9}]
  For each \(i \in \set{1, \dots, k}\), let \(m_i\) denote the multiplicity of \(\lambda_i\), \(d_i = \dim(E_{\lambda_i})\), and \(n = \dim(\V)\).

  First, suppose that \(\T\) is diagonalizable.
  Let \(\beta\) be a basis for \(\V\) over \(\F\) consisting of eigenvectors of \(\T\).
  For each \(i \in \set{1, \dots, k}\), let \(\beta_i = \beta \cap E_{\lambda_i}\), the set of vectors in \(\beta\) that are eigenvectors corresponding to \(\lambda_i\), and let \(n_i\) denote the number of vectors in \(\beta_i\).
  Then \(n_i \leq d_i\) for each \(i \in \set{1, \dots, k}\) because \(\beta_i\) is a linearly independent subset of a subspace of dimension \(d_i\), and \(d_i \leq m_i\) by \cref{5.7}.
  The \(n_i\)'s sum to \(n\) because \(\beta\) contains \(n\) vectors.
  The \(m_i\)'s also sum to \(n\) because the degree of the characteristic polynomial of \(\T\) is equal to the sum of the multiplicities of the eigenvalues.
  Thus
  \[
    n = \sum_{i = 1}^k n_i \leq \sum_{i = 1}^k d_i \leq \sum_{i = 1}^k m_i = n.
  \]
  It follows that
  \[
    \sum_{i = 1}^k (m_i - d_i) = 0.
  \]
  Since \((m_i - d_i) \geq 0\) for all \(i \in \set{1, \dots, k}\), we conclude that \(m_i = d_i\) for all \(i \in \set{1, \dots, k}\).

  Conversely, suppose that \(m_i = d_i\) for all \(i \in \set{1, \dots, k}\).
  We simultaneously show that \(\T\) is diagonalizable and prove (b).
  For each \(i \in \set{1, \dots, k}\), let \(\beta_i\) be an ordered basis for \(E_{\lambda_i}\), and let \(\beta = \bigcup_{i = 1}^k \beta_i\).
  By \cref{5.8}, \(\beta\) is linearly independent.
  Furthermore, since \(d_i = m_i\) for all \(i \in \set{1, \dots, k}\), \(\beta\) contains
  \[
    \sum_{i = 1}^k d_i = \sum_{i = 1}^k m_i = n
  \]
  vectors.
  Therefore \(\beta\) is an ordered basis for \(\V\) over \(\F\) consisting of eigenvectors of \(\V\), and we conclude that \(\T\) is diagonalizable.
\end{proof}

\begin{note}
  Let \(\T\) be a linear operator on an \(n\)-dimensional vector space \(\V\) over \(\F\).
  Then \(\T\) is diagonalizable iff both of the following conditions hold.
  \begin{itemize}
    \item The characteristic polynomial of \(\T\) splits.
    \item For each eigenvalue \(\lambda\) of \(\T\), the multiplicity of \(\lambda\) equals \(n - \rk{\T - \lambda \IT[\V]}\).
  \end{itemize}
  These same conditions can be used to test if a square matrix \(A\) is diagonalizable because diagonalizability of \(A\) is equivalent to diagonalizability of the operator \(\L_A\).

  If \(\T\) is a diagonalizable operator and \(\seq{\beta}{1,,k}\) are ordered bases for the eigenspaces of \(\T\), then the union \(\beta = \bigcup_{i = 1}^k \beta_i\) is an ordered basis for \(\V\) over \(\F\) consisting of eigenvectors of \(\T\), and hence \([\T]_{\beta}\) is a diagonal matrix.

  When testing \(\T\) for diagonalizability, it is usually easiest to choose a convenient basis \(\alpha\) for \(\V\) over \(\F\) and work with \(B = [\T]_{\alpha}\).
  If the characteristic polynomial of \(B\) splits, then use condition 2 above to check if the multiplicity of each \emph{repeated} eigenvalue of \(B\) equals \(n - \rk{B - \lambda I_n}\).
  (By \cref{5.7}, condition 2 is automatically satisfied for eigenvalues of multiplicity \(1\).)
  If so, then \(B\), and hence \(\T\), is diagonalizable.

  If \(\T\) is diagonalizable and a basis \(\beta\) for \(\V\) over \(\F\) consisting of eigenvectors of \(\T\) is desired, then we first find a basis for each eigenspace of \(B\).
  The union of these bases is a basis \(\gamma\) for \(\vs{F}^n\) consisting of eigenvectors of \(B\).
  Each vector in \(\gamma\) is the coordinate vector relative to \(\alpha\) of an eigenvector of \(\T\).
  The set consisting of these \(n\) eigenvectors of \(\T\) is the desired basis \(\beta\).

  Furthermore, if \(A\) is an \(n \times n\) diagonalizable matrix, we can use the \cref{2.5.3} to find an invertible \(n \times n\) matrix \(Q\) and a diagonal \(n \times n\) matrix \(D\) such that \(Q^{-1} A Q = D\).
  The matrix \(Q\) has as its columns the vectors in a basis of eigenvectors of \(A\), and \(D\) has as its \(j\)th diagonal entry the eigenvalue of \(A\) corresponding to the \(j\)th column of \(Q\).
\end{note}

\begin{defn}\label{5.2.6}
  Let \(\seq{\W}{1,,k}\) be subspaces of a vector space \(\V\) over \(\F\).
  We define the \textbf{sum} of these subspaces to be the set
  \[
    \set{\seq[+]{v}{1,,k} : v_i \in \W_i \text{ for } i \in \set{1, \dots, k}},
  \]
  which we denote by \(\seq[+]{\W}{1,,k}\) or \(\sum_{i = 1}^k \W_i\).
\end{defn}

\begin{defn}\label{5.2.7}
  Let \(\seq{\W}{1,,k}\) be subspaces of a vector space \(\V\) over \(\F\).
  We call \(\V\) the \textbf{direct sum} of the subspaces \(\seq{\W}{1,,k}\) and write \(\V = \seq[\oplus]{\W}{1,,k}\), if
  \[
    \V = \sum_{i = 1}^k \W_i
  \]
  and
  \[
    \W_j \cap \sum_{i \neq j} \W_i = \set{\zv} \quad \text{for each } j \in \set{1, \dots, k}.
  \]
\end{defn}

\begin{thm}\label{5.10}
  Let \(\seq{\W}{1,,k}\) be subspaces of a finite-dimensional vector space \(\V\) over \(\F\).
  The following conditions are equivalent.
  \begin{enumerate}
    \item \(\V = \seq[\oplus]{\W}{1,,k}\).
    \item \(\V = \sum_{i = 1}^k \W_i\) and, for any vectors \(\seq{v}{1,,k}\) such that \(v_i \in \W\) for \(i \in \set{1, \dots, k}\), if \(\seq[+]{v}{1,,k} = \zv\), then \(v_i = \zv\) for all \(i \in \set{1, \dots, k}\).
    \item Each vector \(v \in \V\) can be uniquely written as \(v = \seq[+]{v}{1,,k}\), where \(v_i \in \W_i\) for all \(i \in \set{1, \dots, k}\).
    \item For each \(i \in \set{1, \dots, k}\), if \(\gamma_i\) is an ordered basis for \(\W_i\) over \(\F\), then \(\bigcup_{i = 1}^k \gamma_i\) is an ordered basis for \(\V\) over \(\F\).
    \item For each \(i \in \set{1, \dots, k}\), there exists an ordered basis \(\gamma_i\) for \(\W_i\) over \(\F\) such that \(\bigcup_{i = 1}^k \gamma_i\) is an ordered basis for \(\V\) over \(\F\).
  \end{enumerate}
\end{thm}

\begin{proof}[\pf{5.10}]
  Assume (a).
  We prove (b).
  Clearly
  \[
    \V = \sum_{i = 1}^k \W_i.
  \]
  Now suppose that \(\seq{v}{1,,k}\) are vectors such that \(v_i \in \W_i\) for all \(i \in \set{1, \dots, k}\) and \(\seq[+]{v}{1,,k} = \zv\).
  Then for any \(j \in \set{1, \dots, k}\)
  \[
    -v_j = \sum_{\substack{i = 1 \\ i \neq j}}^k v_i \in \sum_{\substack{i = 1 \\ i \neq j}}^k \W_i.
  \]
  But \(-v_j \in \W_j\) and hence
  \[
    -v_j \in \W_j \cap \sum_{\substack{i = 1 \\ i \neq j}}^k \W_i = \set{\zv}.
  \]
  So \(v_j = \zv\), proving (b).

  Now assume (b).
  We prove (c).
  Let \(v \in \V\).
  By (b), there exist vectors \(\seq{v}{1,,k}\) such that \(v_i \in \W_i\) for all \(i \in \set{1, \dots, k}\) and \(v = \seq[+]{v}{1,,k}\).
  We must show that this representation is unique.
  Suppose also that \(v = \seq[+]{w}{1,,k}\), where \(w_i \in \W_i\) for all \(i \in \set{1, \dots, k}\).
  Then
  \[
    (v_1 - w_1) + \cdots + (v_k - w_k) = \zv.
  \]
  But \(v_i - w_i \in \W_i\) for all \(i \in \set{1, \dots, k}\), and therefore \(v_i - w_i = \zv\) for all \(i \in \set{1, \dots, k}\) by (b).
  Thus \(v_i = w_i\) for all \(i \in \set{1, \dots, k}\), proving the uniqueness of the representation.

  Now assume (c).
  We prove (d).
  For each \(i \in \set{1, \dots, k}\), let \(\gamma_i\) be an ordered basis for \(\W_i\) over \(\F\).
  Since
  \[
    \V = \sum_{i = 1}^k \W_i
  \]
  by (c), it follows that \(\bigcup_{i = 1}^k \gamma_i\) generates \(\V\).
  To show that this set is linearly independent, consider vectors \(v_{i j} \in \gamma_i\) (where \(j \in \set{1, \dots, m_i}\) and \(i \in \set{1, \dots, k}\)) and scalars \(a_{i j} \in \F\) such that
  \[
    \sum_{i = 1}^k \sum_{j = 1}^{m_i} a_{i j} v_{i j} = \zv.
  \]
  For each \(i \in \set{1, \dots, k}\), set
  \[
    w_i = \sum_{j = 1}^{m_i} a_{i j} v_{i j}.
  \]
  Then for each \(i \in \set{1, \dots, k}\), \(w_i \in \spn{\gamma_i} = \W_i\) and
  \[
    \seq[+]{w}{1,,k} = \sum_{i = 1}^k \sum_{j = 1}^{m_i} a_{i j} v_{i j} = \zv.
  \]
  Since \(\zv \in \W_i\) for each \(i \in \set{1, \dots, k}\) and \(\zv + \cdots + \zv = \seq[+]{w}{1,,k}\), (c) implies that \(w_i = \zv\) for all \(i \in \set{1, \dots, k}\).
  Thus
  \[
    \zv = w_i = \sum_{j = 1}^{m_i} a_{i j} v_{i j}
  \]
  for each \(i \in \set{1, \dots, k}\).
  But each \(\gamma_i\) is linearly independent, and hence \(a_{i j} = 0\) for all \(i \in \set{1, \dots, k}\) and \(j \in \set{1, \dots, m_i}\).
  Consequently \(\bigcup_{i = 1}^k \gamma_i\) is linearly independent and therefore is a basis for \(\V\) over \(\F\).

  Clearly (e) follows immediately from (d).

  Finally, we assume (e) and prove (a).
  For each \(i \in \set{1, \dots, k}\), let \(\gamma_i\) be an ordered basis for \(\W_i\) over \(\F\) such that \(\bigcup_{i = 1}^k \gamma_i\) is an ordered basis for \(\V\) over \(\F\).
  Then
  \[
    \V = \spn{\bigcup_{i = 1}^k \gamma_i} = \sum_{i = 1}^k \spn{\gamma_i} = \sum_{i = 1}^k \W_i
  \]
  by repeated applications of \cref{ex:1.4.14}.
  Fix \(j \in \set{1, \dots, k}\), and suppose that, for some nonzero vector \(v \in \V\),
  \[
    v \in \W_j \cap \sum_{\substack{i = 1 \\ i \neq j}}^k \W_i.
  \]
  Then
  \[
    v \in \W_j = \spn{\gamma_j} \quad \text{and} \quad v \in \sum_{\substack{i = 1 \\ i \neq j}}^k \W_i = \spn{\bigcup_{\substack{i = 1 \\ i \neq j}}^k \gamma_i}.
  \]
  Hence \(v\) is a nontrivial linear combination of both \(\gamma_j\) and \(\bigcup_{i \neq j} \gamma_i\), so that \(v\) can be expressed as a linear combination of \(\bigcup_{i = 1}^k \gamma_i\) in more than one way.
  But these representations contradict \cref{1.8}, and so we conclude that
  \[
    \W_j \cap \sum_{\substack{i = 1 \\ i \neq j}}^k \W_i = \set{\zv},
  \]
  proving (a).
\end{proof}

\begin{thm}\label{5.11}
  A linear operator \(\T\) on a finite-dimensional vector space \(\V\) over \(\F\) is diagonalizable iff \(\V\) is the direct sum of the eigenspaces of \(\T\).
\end{thm}

\begin{proof}[\pf{5.11}]
  Let \(\seq{\lambda}{1,,k}\) be the distinct eigenvalues of \(\T\).

  First suppose that \(\T\) is diagonalizable, and for each \(i \in \set{1, \dots, k}\) choose an ordered basis \(\gamma_i\) for the eigenspace \(E_{\lambda_i}\) over \(\F\).
  By \cref{5.9}, \(\bigcup_{i = 1}^k \gamma_i\) is a basis for \(\V\) over \(\F\), and hence \(\V\) is a direct sum of the \(E_{\lambda_i}\)'s by \cref{5.10}.

  Conversely, suppose that \(\V\) is a direct sum of the eigenspaces of \(\T\).
  For each \(i \in \set{1, \dots, k}\), choose an ordered basis \(\gamma_i\) for \(E_{\lambda_i}\) over \(\F\).
  By \cref{5.10}, the union \(\bigcup_{i = 1}^k \gamma_i\) is a basis for \(\V\) over \(\F\).
  Since this basis consists of eigenvectors of \(\T\), we conclude that \(\T\) is diagonalizable.
\end{proof}

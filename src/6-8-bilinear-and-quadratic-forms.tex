\section{Bilinear and Quadratic Forms}\label{sec:6.8}

\begin{defn}\label{6.8.1}
  Let \(\V\) be a vector space over a field \(\F\).
  A function \(H\) from the set \(\V \times \V\) of ordered pairs of vectors to \(\F\) is called a \textbf{bilinear form} on \(\V\) if \(H\) is linear in each variable when the other variable is held fixed;
  that is, \(H\) is a bilinear form on \(\V\) if
  \begin{enumerate}
    \item \(H(a x_1 + x_2, y) = a H(x_1, y) + H(x_2, y)\) for all \(x_1, x_2, y \in \V\) and \(a \in \F\).
    \item \(H(x, a y_1 + y_2) = a H(x, y_1) + H(x, y_2)\) for all \(x, y_1, y_2 \in \V\) and \(a \in \F\).
  \end{enumerate}
  We denote the set of all bilinear forms on \(\V\) by \(\bi(\V)\).
  Observe that an inner product on a vector space is a bilinear form if the underlying field is real, but not if the underlying field is complex (\cref{6.1}(b)).
\end{defn}

\begin{eg}\label{6.8.2}
  Let \(\V = \vs{F}^n\), where the vectors are considered as column vectors.
  For any \(A \in \ms[n][n][\F]\), define \(H : \V \times \V \to \F\) by
  \[
    H(x, y) = \tp{x} A y \quad \text{for } x, y \in \V.
  \]
  Notice that since \(x\) and \(y\) are \(n \times 1\) matrices and \(A \in \ms[n][n][\F]\), \(H(x, y)\) is a \(1 \times 1\) matrix.
  Then \(H \in \bi(\V)\).
\end{eg}

\begin{proof}[\pf{6.8.2}]
  Let \(x, y, z \in \V\) and let \(c \in \F\).
  Since
  \begin{align*}
    H(ax + y, z) & = \tp{(ax + y)} A z         &  & \by{6.8.2}     \\
                 & = (a \tp{x} + \tp{y}) A z   &  & \by{ex:1.3.3}  \\
                 & = a \tp{x} A z + \tp{y} A z &  & \by{2.12}[a,b] \\
                 & = a H(x, z) + H(y, z);      &  & \by{6.8.2}     \\
    H(z, ax + y) & = \tp{z} A (ax + y)         &  & \by{6.8.2}     \\
                 & = a \tp{z} A x + \tp{z} A y &  & \by{2.12}[a,b] \\
                 & = a H(z, x) + H(z, y),      &  & \by{6.8.2}
  \end{align*}
  by \cref{6.8.1} we know that \(H \in \bi(\V)\).
\end{proof}

\begin{prop}\label{6.8.3}
  For any bilinear form \(H\) on a vector space \(\V\) over a field \(\F\), the following properties hold.
  \begin{enumerate}
    \item If, for any \(x \in \V\), the functions \(\lt{L}_x, \lt{R}_x : \V \to \F\) are defined by
          \[
            \lt{L}_x(y) = H(x, y) \quad \text{and} \quad \lt{R}_x(y) = H(y, x) \quad \text{for all } y \in \V,
          \]
          then \(\lt{L}_x, \lt{R}_x \in \ls(\V, \F)\).
    \item \(H(\zv, x) = H(x, \zv) = 0\) for all \(x \in \V\).
    \item For all \(x, y, z, w \in \V\),
          \[
            H(x + y, z + w) = H(x, z) + H(x, w) + H(y, z) + H(y, w).
          \]
    \item If \(J : \V \times \V \to \F\) is defined by \(J(x, y) = H(y, x)\), then \(J\) is a bilinear form.
  \end{enumerate}
\end{prop}

\begin{proof}[\pf{6.8.3}(a)]
  Let \(y, z \in \V\) and let \(c \in \F\).
  Since
  \begin{align*}
    \lt{L}_x(cy + z) & = H(x, cy + z)                 &  & \by{6.8.3}[a] \\
                     & = c H(x, y) + H(x, z)          &  & \by{6.8.1}[b] \\
                     & = c \lt{L}_x(y) + \lt{L}_x(z); &  & \by{6.8.3}[a] \\
    \lt{R}_x(cy + z) & = H(cy + z, x)                 &  & \by{6.8.3}[a] \\
                     & = c H(y, x) + H(z, x)          &  & \by{6.8.1}[a] \\
                     & = c \lt{R}_x(y) + \lt{R}_x(z), &  & \by{6.8.3}[a]
  \end{align*}
  by \cref{2.1.2}(b) we know that \(\lt{L}_x, \lt{R}_x \in \ls(\V, \F)\).
\end{proof}

\begin{proof}[\pf{6.8.3}(b)]
  Since
  \begin{align*}
    H(\zv, x) & = H(\zv + \zv, x)        &  & \by{1.2.1}    \\
              & = H(\zv, x) + H(\zv, x); &  & \by{6.8.1}[a] \\
    H(x, \zv) & = H(x, \zv + \zv)        &  & \by{1.2.1}    \\
              & = H(x, \zv) + H(x, \zv), &  & \by{6.8.1}[b]
  \end{align*}
  by \cref{c.1} we know that
  \[
    H(\zv, x) = 0 = H(x, \zv).
  \]
\end{proof}

\begin{proof}[\pf{6.8.3}(c)]
  We have
  \begin{align*}
    H(x + y, z + w) & = H(x, z + w) + H(y, z + w)              &  & \by{6.8.1}[a] \\
                    & = H(x, z) + H(x, w) + H(y, z) + H(y, w). &  & \by{6.8.1}[b]
  \end{align*}
\end{proof}

\begin{proof}[\pf{6.8.3}(d)]
  Let \(x, y, z \in \V\) and \(c \in \F\).
  Since
  \begin{align*}
    J(cx + y, z) & = H(z, cx + y)         &  & \by{6.8.3}[d] \\
                 & = c H(z, x) + H(z, y)  &  & \by{6.8.1}[b] \\
                 & = c J(x, z) + J(y, z); &  & \by{6.8.3}[d] \\
    J(z, cx + y) & = H(cx + y, z)         &  & \by{6.8.3}[d] \\
                 & = c H(x, z) + H(y, z)  &  & \by{6.8.1}[a] \\
                 & = c J(z, x) + J(z, y), &  & \by{6.8.3}[d]
  \end{align*}
  by \cref{6.8.1} we know that \(J \in \bi(\V)\).
\end{proof}

\begin{defn}\label{6.8.4}
  Let \(\V\) be a vector space over \(\F\), let \(H_1, H_2 \in \bi(\V)\), and let \(a \in \F\).
  We define the \textbf{sum} \(H_1 + H_2\) and the \textbf{scalar product} \(a H_1\) by the equations
  \[
    (H_1 + H_2)(x, y) = H_1(x, y) + H_2(x, y)
  \]
  and
  \[
    (a H_1)(x, y) = a (H_1(x, y)) \quad \text{for all } x, y \in \V.
  \]
\end{defn}

\begin{thm}\label{6.31}
  For any vector space \(\V\) over \(\F\), the sum of two bilinear forms and the product of a scalar and a bilinear form on \(\V\) are again bilinear forms on \(\V\).
  Furthermore, \(\bi(\V)\) is a vector space with respect to these operations.
\end{thm}

\begin{proof}[\pf{6.31}]
  By \cref{1.2.10} we know that the set \(\fs(\V \times \V, \F)\) is a vector space and \(\bi(\V) \subseteq \fs(\V \times \V, \F)\).
  Let \(\zT\) be the zero function of \(\fs(\V \times \V, \F)\).
  Let \(f, g \in \bi(\V)\), let \(x, y, z \in \V\) and let \(a, c \in \F\).
  Since
  \begin{align*}
    (cf + g)(ax + y, z) & = c f(ax + y, z) + g(ax + y, z)                 &  & \by{6.8.4}    \\
                        & = c (a f(x, z) + f(y, z)) + a g(x, z) + g(y, z) &  & \by{6.8.1}[a] \\
                        & = a (c f(x, z) + g(x, z)) + c f(y, z) + g(y, z)                    \\
                        & = a (cf + g)(x, z) + (cf + g)(y, z);            &  & \by{6.8.4}    \\
    (cf + g)(z, ax + y) & = c f(z, ax + y) + g(z, ax + y)                 &  & \by{6.8.4}    \\
                        & = c (a f(z, x) + f(z, y)) + a g(z, x) + g(z, y) &  & \by{6.8.1}[b] \\
                        & = a (c f(z, x) + g(z, x)) + c f(z, y) + g(z, y)                    \\
                        & = a (cf + g)(z, x) + (cf + g)(z, y),            &  & \by{6.8.4}
  \end{align*}
  by \cref{6.8.1} we have \(cf + g \in \bi(\V)\).
  Since
  \begin{align*}
    \zT(ax + y, z) & = 0                        \\
                   & = a 0 + 0                  \\
                   & = a \zT(x, z) + \zT(y, z); \\
    \zT(z, ax + y) & = 0                        \\
                   & = a 0 + 0                  \\
                   & = a \zT(z, x) + \zT(z, y),
  \end{align*}
  by \cref{6.8.1} we have \(\zT \in \bi(\V)\).
  Thus by \cref{ex:1.3.17} we know that \(\bi(\V)\) is a vector space over \(\F\).
\end{proof}

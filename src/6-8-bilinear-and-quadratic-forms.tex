\section{Bilinear and Quadratic Forms}\label{sec:6.8}

\begin{defn}\label{6.8.1}
  Let \(\V\) be a vector space over a field \(\F\).
  A function \(H\) from the set \(\V \times \V\) of ordered pairs of vectors to \(\F\) is called a \textbf{bilinear form} on \(\V\) if \(H\) is linear in each variable when the other variable is held fixed;
  that is, \(H\) is a bilinear form on \(\V\) if
  \begin{enumerate}
    \item \(H(a x_1 + x_2, y) = a H(x_1, y) + H(x_2, y)\) for all \(x_1, x_2, y \in \V\) and \(a \in \F\).
    \item \(H(x, a y_1 + y_2) = a H(x, y_1) + H(x, y_2)\) for all \(x, y_1, y_2 \in \V\) and \(a \in \F\).
  \end{enumerate}
  We denote the set of all bilinear forms on \(\V\) by \(\bi(\V)\).
  Observe that an inner product on a vector space is a bilinear form if the underlying field is real, but not if the underlying field is complex (\cref{6.1}(b)).
\end{defn}

\begin{eg}\label{6.8.2}
  Let \(\V = \vs{F}^n\), where the vectors are considered as column vectors.
  For any \(A \in \ms[n][n][\F]\), define \(H : \V \times \V \to \F\) by
  \[
    H(x, y) = \tp{x} A y \quad \text{for } x, y \in \V.
  \]
  Notice that since \(x\) and \(y\) are \(n \times 1\) matrices and \(A \in \ms[n][n][\F]\), \(H(x, y)\) is a \(1 \times 1\) matrix.
  Then \(H \in \bi(\V)\).
\end{eg}

\begin{proof}[\pf{6.8.2}]
  Let \(x, y, z \in \V\) and let \(c \in \F\).
  Since
  \begin{align*}
    H(ax + y, z) & = \tp{(ax + y)} A z         &  & \by{6.8.2}     \\
                 & = (a \tp{x} + \tp{y}) A z   &  & \by{ex:1.3.3}  \\
                 & = a \tp{x} A z + \tp{y} A z &  & \by{2.12}[a,b] \\
                 & = a H(x, z) + H(y, z);      &  & \by{6.8.2}     \\
    H(z, ax + y) & = \tp{z} A (ax + y)         &  & \by{6.8.2}     \\
                 & = a \tp{z} A x + \tp{z} A y &  & \by{2.12}[a,b] \\
                 & = a H(z, x) + H(z, y),      &  & \by{6.8.2}
  \end{align*}
  by \cref{6.8.1} we know that \(H \in \bi(\V)\).
\end{proof}

\begin{prop}\label{6.8.3}
  For any bilinear form \(H\) on a vector space \(\V\) over a field \(\F\), the following properties hold.
  \begin{enumerate}
    \item If, for any \(x \in \V\), the functions \(\lt{L}_x, \lt{R}_x : \V \to \F\) are defined by
          \[
            \lt{L}_x(y) = H(x, y) \quad \text{and} \quad \lt{R}_x(y) = H(y, x) \quad \text{for all } y \in \V,
          \]
          then \(\lt{L}_x, \lt{R}_x \in \ls(\V, \F)\).
    \item \(H(\zv, x) = H(x, \zv) = 0\) for all \(x \in \V\).
    \item For all \(x, y, z, w \in \V\),
          \[
            H(x + y, z + w) = H(x, z) + H(x, w) + H(y, z) + H(y, w).
          \]
    \item If \(J : \V \times \V \to \F\) is defined by \(J(x, y) = H(y, x)\), then \(J\) is a bilinear form.
  \end{enumerate}
\end{prop}

\begin{proof}[\pf{6.8.3}(a)]
  Let \(y, z \in \V\) and let \(c \in \F\).
  Since
  \begin{align*}
    \lt{L}_x(cy + z) & = H(x, cy + z)                 &  & \by{6.8.3}[a] \\
                     & = c H(x, y) + H(x, z)          &  & \by{6.8.1}[b] \\
                     & = c \lt{L}_x(y) + \lt{L}_x(z); &  & \by{6.8.3}[a] \\
    \lt{R}_x(cy + z) & = H(cy + z, x)                 &  & \by{6.8.3}[a] \\
                     & = c H(y, x) + H(z, x)          &  & \by{6.8.1}[a] \\
                     & = c \lt{R}_x(y) + \lt{R}_x(z), &  & \by{6.8.3}[a]
  \end{align*}
  by \cref{2.1.2}(b) we know that \(\lt{L}_x, \lt{R}_x \in \ls(\V, \F)\).
\end{proof}

\begin{proof}[\pf{6.8.3}(b)]
  Since
  \begin{align*}
    H(\zv, x) & = H(\zv + \zv, x)        &  & \by{1.2.1}    \\
              & = H(\zv, x) + H(\zv, x); &  & \by{6.8.1}[a] \\
    H(x, \zv) & = H(x, \zv + \zv)        &  & \by{1.2.1}    \\
              & = H(x, \zv) + H(x, \zv), &  & \by{6.8.1}[b]
  \end{align*}
  by \cref{c.1} we know that
  \[
    H(\zv, x) = 0 = H(x, \zv).
  \]
\end{proof}

\begin{proof}[\pf{6.8.3}(c)]
  We have
  \begin{align*}
    H(x + y, z + w) & = H(x, z + w) + H(y, z + w)              &  & \by{6.8.1}[a] \\
                    & = H(x, z) + H(x, w) + H(y, z) + H(y, w). &  & \by{6.8.1}[b]
  \end{align*}
\end{proof}

\begin{proof}[\pf{6.8.3}(d)]
  Let \(x, y, z \in \V\) and \(c \in \F\).
  Since
  \begin{align*}
    J(cx + y, z) & = H(z, cx + y)         &  & \by{6.8.3}[d] \\
                 & = c H(z, x) + H(z, y)  &  & \by{6.8.1}[b] \\
                 & = c J(x, z) + J(y, z); &  & \by{6.8.3}[d] \\
    J(z, cx + y) & = H(cx + y, z)         &  & \by{6.8.3}[d] \\
                 & = c H(x, z) + H(y, z)  &  & \by{6.8.1}[a] \\
                 & = c J(z, x) + J(z, y), &  & \by{6.8.3}[d]
  \end{align*}
  by \cref{6.8.1} we know that \(J \in \bi(\V)\).
\end{proof}

\begin{defn}\label{6.8.4}
  Let \(\V\) be a vector space over \(\F\), let \(H_1, H_2 \in \bi(\V)\), and let \(a \in \F\).
  We define the \textbf{sum} \(H_1 + H_2\) and the \textbf{scalar product} \(a H_1\) by the equations
  \[
    (H_1 + H_2)(x, y) = H_1(x, y) + H_2(x, y)
  \]
  and
  \[
    (a H_1)(x, y) = a (H_1(x, y)) \quad \text{for all } x, y \in \V.
  \]
\end{defn}

\begin{thm}\label{6.31}
  For any vector space \(\V\) over \(\F\), the sum of two bilinear forms and the product of a scalar and a bilinear form on \(\V\) are again bilinear forms on \(\V\).
  Furthermore, \(\bi(\V)\) is a vector space with respect to these operations.
\end{thm}

\begin{proof}[\pf{6.31}]
  By \cref{1.2.10} we know that the set \(\fs(\V \times \V, \F)\) is a vector space and \(\bi(\V) \subseteq \fs(\V \times \V, \F)\).
  Let \(\zT\) be the zero function of \(\fs(\V \times \V, \F)\).
  Let \(f, g \in \bi(\V)\), let \(x, y, z \in \V\) and let \(a, c \in \F\).
  Since
  \begin{align*}
    (cf + g)(ax + y, z) & = c f(ax + y, z) + g(ax + y, z)                 &  & \by{6.8.4}    \\
                        & = c (a f(x, z) + f(y, z)) + a g(x, z) + g(y, z) &  & \by{6.8.1}[a] \\
                        & = a (c f(x, z) + g(x, z)) + c f(y, z) + g(y, z)                    \\
                        & = a (cf + g)(x, z) + (cf + g)(y, z);            &  & \by{6.8.4}    \\
    (cf + g)(z, ax + y) & = c f(z, ax + y) + g(z, ax + y)                 &  & \by{6.8.4}    \\
                        & = c (a f(z, x) + f(z, y)) + a g(z, x) + g(z, y) &  & \by{6.8.1}[b] \\
                        & = a (c f(z, x) + g(z, x)) + c f(z, y) + g(z, y)                    \\
                        & = a (cf + g)(z, x) + (cf + g)(z, y),            &  & \by{6.8.4}
  \end{align*}
  by \cref{6.8.1} we have \(cf + g \in \bi(\V)\).
  Since
  \begin{align*}
    \zT(ax + y, z) & = 0                        \\
                   & = a 0 + 0                  \\
                   & = a \zT(x, z) + \zT(y, z); \\
    \zT(z, ax + y) & = 0                        \\
                   & = a 0 + 0                  \\
                   & = a \zT(z, x) + \zT(z, y),
  \end{align*}
  by \cref{6.8.1} we have \(\zT \in \bi(\V)\).
  Thus by \cref{ex:1.3.17} we know that \(\bi(\V)\) is a vector space over \(\F\).
\end{proof}

\begin{defn}\label{6.8.5}
  Let \(\beta = \set{\seq{v}{1,,n}}\) be an ordered basis for an \(n\)-dimensional vector space \(\V\) over \(\F\), and let \(H \in \bi(\V)\).
  We can associate \(H\) with an \(n \times n\) matrix \(A\) whose entry in row \(i\) and column \(j\) is defined by
  \[
    A_{i j} = H(v_i, v_j) \quad \text{for } i, j \in \set{1, \dots, n}.
  \]
  The matrix \(A\) above is called the \textbf{matrix representation} of \(H\) with respect to the ordered basis \(\beta\) and is denoted by \(\psi_{\beta}(H)\).
  We can therefore regard \(\psi_{\beta}\) as a mapping from \(\bi(\V)\) to \(\ms[n][n][\F]\), where \(\F\) is the field of scalars for \(\V\), that takes a bilinear form \(H\) into its matrix representation \(\psi_{\beta}(H)\).
\end{defn}

\begin{thm}\label{6.32}
  For any \(n\)-dimensional vector space \(\V\) over \(\F\) and any ordered basis \(\beta\) for \(\V\) over \(\F\), \(\psi_{\beta} : \bi(\V) \to \ms[n][n][\F]\) is an isomorphism.
\end{thm}

\begin{proof}[\pf{6.32}]
  First we show that \(\psi_{\beta} \in \ls(\bi(\V), \ms[n][n][\F])\).
  Let \(f, g \in \bi(\V)\) and let \(c \in \F\).
  Since
  \begin{align*}
    \forall i, j \in \set{1, \dots, n}, \pa{\psi_{\beta}(cf + g)}_{i j} & = (cf + g)(v_i, v_j)                                        &  & \by{6.8.5} \\
                                                                        & = c f(v_i, v_j) + g(v_i, v_j)                               &  & \by{6.8.4} \\
                                                                        & = c \pa{\psi_{\beta}(f)}_{i j} + \pa{\psi_{\beta}(g)}_{i j} &  & \by{6.8.5} \\
                                                                        & = \pa{c \psi_{\beta}(f) + \psi_{\beta}(g)}_{i j},           &  & \by{1.2.9}
  \end{align*}
  by \cref{1.2.8} we know that \(\psi_{\beta}(cf + g) = c \psi_{\beta}(f) + \psi_{\beta}(g)\) and thus by \cref{2.1.2}(b) \(\psi_{\beta} \in \ls(\bi(\V), \ms[n][n][\F])\).

  To show that \(\psi_{\beta}\) is one-to-one, suppose that \(\psi_{\beta}(H) = \zm\) for some \(H \in \bi(\V)\).
  Fix \(v_i \in \beta\), and recall the mapping \(\lt{L}_{v_i} : \V \to \F\), which is linear by \cref{6.8.3}(a).
  By hypothesis, \(\lt{L}_{v_i}(v_j) = H(v_i, v_j) = 0\) for all \(v_j \in \beta\).
  Hence \(\lt{L}_{v_i}\) is the zero transformation from \(\V\) to \(\F\) by \cref{2.1.13}.
  So
  \[
    H(v_i, x) = \lt{L}_{v_i}(x) = 0 \quad \text{for all } x \in \V \text{ and } v_i \in \beta.
  \]
  Next fix an arbitrary \(y \in \V\), and recall the linear mapping \(\lt{R}_y : \V \to \F\) defined in \cref{6.8.3}(a).
  From the equation above we have \(\lt{R}_y(v_i) = H(v_i, y) = 0\) for all \(v_i \in \beta\), and hence \(\lt{R}_y\) is the zero transformation by \cref{2.1.13}.
  So \(H(x, y) = \lt{R}_y(x) = 0\) for all \(x, y \in \V\).
  Thus \(H\) is the zero bilinear form, and therefore \(\psi_{\beta}\) is one-to-one by \cref{2.4}.

  To show that \(\psi_{\beta}\) is onto, consider any \(A \in \ms[n][n][\F]\).
  Recall the isomorphism \(\phi_{\beta} \in \ls(\V, \vs{\F}^n)\) defined in \cref{2.4.11}.
  For \(x \in \V\), we view \(\phi_{\beta}(x) \in \vs{F}^n\) as a column vector.
  Let \(H : \V \times \V \to \F\) be the mapping defined by
  \[
    H(x, y) = \tp{(\phi_{\beta}(x))} A (\phi_{\beta}(y)) \quad \text{for all } x, y \in \V.
  \]
  By \cref{6.8.2} we know that \(H \in \bi(\V)\).
  We show that \(\psi_{\beta}(H) = A\).
  Let \(v_i, v_j \in \beta\).
  Then \(\phi_{\beta}(v_i) = e_i\) and \(\phi_{\beta}(v_j) = e_j\);
  hence, for any \(i, j \in \set{1, \dots, n}\),
  \[
    H(v_i, v_j) = \tp{(\phi_{\beta}(v_i))} A (\phi_{\beta}(v_j)) = \tp{e_i} A e_j = A_{i j}.
  \]
  We conclude that \(\psi_{\beta}(H) = A\) and \(\psi_{\beta}\) is onto.
\end{proof}

\begin{cor}\label{6.8.6}
  For any \(n\)-dimensional vector space \(\V\), \(\bi(\V)\) has dimension \(n^2\).
\end{cor}

\begin{proof}[\pf{6.8.6}]
  By \cref{1.6.11,2.19,6.32} we see that this is true.
\end{proof}

\begin{cor}\label{6.8.7}
  Let \(\V\) be an \(n\)-dimensional vector space over \(\F\) with ordered basis \(\beta\).
  If \(H \in \bi(\V)\) and \(A \in \ms[n][n][\F]\), then \(\psi_{\beta}(H) = A\) iff \(H(x, y) = \tp{(\phi_{\beta}(x))} A (\phi_{\beta}(y))\) for all \(x, y \in \V\).
\end{cor}

\begin{proof}[\pf{6.8.7}]
  Let \(\beta = \set{\seq{v}{1,,n}}\) and let \(x, y \in \V\).
  Since \(\beta\) is a basis for \(\V\) over \(\F\), there exist \(\seq{a}{1,,n}, \seq{b}{1,,n} \in \F\) such that \(x = \sum_{i = 1}^n a_i v_i\) and \(y = \sum_{i = 1}^n b_i v_i\).
  Then we have
  \begin{align*}
         & \psi_{\beta}(H) = A                                                                                                                   \\
    \iff & H(x, y) = H\pa{\sum_{i = 1}^n a_i v_i, \sum_{j = 1}^n b_j v_j} = \sum_{i = 1}^n \sum_{j = 1}^n a_i b_j H(v_i, v_j) &  & \by{6.8.3}[c] \\
         & = \sum_{i = 1}^n \sum_{j = 1}^n a_i b_j A_{i j}                                                                    &  & \by{6.32}     \\
         & = \sum_{i = 1}^n \sum_{j = 1}^n a_i b_j \tp{e_i} A e_j                                                             &  & \by{2.3.1}    \\
         & = \pa{\sum_{i = 1}^n a_i \tp{e_i}} A \pa{\sum_{j = 1}^n b_j e_j}                                                   &  & \by{2.3.5}    \\
         & = \tp{\pa{\sum_{i = 1}^n a_i e_i}} A \pa{\sum_{j = 1}^n b_j e_j}                                                   &  & \by{ex:1.3.3} \\
         & = \tp{(\phi_{\beta}(x))} A (\phi_{\beta}(y)).                                                                      &  & \by{2.4.11}
  \end{align*}
\end{proof}

\begin{cor}\label{6.8.8}
  Let \(\F\) be a field, \(n\) a positive integer, and \(\beta\) be the standard ordered basis for \(\vs{F}^n\) over \(\F\).
  Then for any \(H \in \bi(\vs{F}^n)\), there exists a unique matrix \(A \in \ms[n][n][\F]\), namely, \(A = \psi_{\beta}(H)\), such that
  \[
    H(x, y) = \tp{x} A y \quad \text{for all } x, y \in \vs{F}^n.
  \]
\end{cor}

\begin{proof}[\pf{6.8.8}]
  This is an immediate consequence of \cref{6.8.7}.
\end{proof}

\begin{defn}\label{6.8.9}
  Let \(A, B \in \ms[n][n][\F]\).
  Then \(B\) is said to be \textbf{congruent} to \(A\) if there exists an invertible matrix \(Q \in \ms[n][n][\F]\) such that \(B = \tp{Q} A Q\).
  Observe that the relation of congruence is an equivalence relation (see \cref{ex:6.8.12}).
\end{defn}

\begin{thm}\label{6.33}
  Let \(\V\) be a finite-dimensional vector space over \(\F\) with ordered bases \(\beta = \set{\seq{v}{1,,n}}\) and \(\gamma = \set{\seq{w}{1,,n}}\), and let \(Q\) be the change of coordinate matrix changing \(\gamma\)-coordinates into \(\beta\)-coordinates.
  Then, for any \(H \in \bi(\V)\), we have \(\psi_{\gamma}(H) = \tp{Q} \psi_{\beta}(H) Q\).
  Therefore \(\psi_{\gamma}(H)\) is congruent to \(\psi_{\beta}(H)\).
\end{thm}

\begin{proof}[\pf{6.33}]
  There are essentially two proofs of this theorem.
  One involves a direct computation, while the other follows immediately from a clever observation.
  We give the more direct proof here, leaving the other proof for the exercises (see \cref{ex:6.8.13}).

  Suppose that \(A = \psi_{\beta}(H)\) and \(B = \psi_{\gamma}(H)\).
  Then for \(i, j \in \set{1, \dots, n}\),
  \[
    w_i = \sum_{k = 1}^n Q_{k i} v_k \quad \text{and} \quad w_j = \sum_{r = 1}^n Q_{r j} v_r.
  \]
  (See \cref{2.2.4}.)
  Thus
  \begin{align*}
    B_{i j} & = H(w_i, w_j)                                                  &  & \by{6.8.5}    \\
            & = H\pa{\sum_{k = 1}^n Q_{k i} v_k, w_j}                        &  & \by{2.2.4}    \\
            & = \sum_{k = 1}^n Q_{k i} H(v_k, w_j)                           &  & \by{6.8.1}[a] \\
            & = \sum_{k = 1}^n Q_{k i} H\pa{v_k, \sum_{r = 1}^n Q_{r j} v_r} &  & \by{2.2.4}    \\
            & = \sum_{k = 1}^n Q_{k i} \sum_{r = 1}^n Q_{r j} H(v_k, v_r)    &  & \by{6.8.1}[b] \\
            & = \sum_{k = 1}^n Q_{k i} \sum_{r = 1}^n Q_{r j} A_{k r}        &  & \by{6.8.5}    \\
            & = \sum_{k = 1}^n Q_{k i} \sum_{r=  1}^n A_{k r} Q_{r j}                           \\
            & = \sum_{k = 1}^n Q_{k i} (AQ)_{k j}                            &  & \by{2.3.1}    \\
            & = \sum_{k = 1}^n (\tp{Q})_{i k} (AQ)_{k j}                     &  & \by{1.3.3}    \\
            & = (\tp{Q} A Q)_{i j}.                                          &  & \by{2.3.1}
  \end{align*}
  Hence by \cref{1.2.9} \(B = \tp{Q} A Q\).
\end{proof}

\begin{cor}\label{6.8.10}
  Let \(\V\) be an \(n\)-dimensional vector space over \(\F\) with ordered basis \(\beta\), and let \(H \in \bi(\V)\).
  For any \(B \in \ms[n][n][\F]\), if \(B\) is congruent to \(\psi_{\beta}(H)\), then there exists an ordered basis \(\gamma\) for \(\V\) over \(\F\) such that \(\psi_{\gamma}(H) = B\).
  Furthermore, if \(B = \tp{Q} \psi_{\beta}(H) Q\) for some invertible matrix \(Q\), then \(Q\) changes \(\gamma\)-coordinates into \(\beta\)-coordinates.
\end{cor}

\begin{proof}[\pf{6.8.10}]
  Suppose that \(B = \tp{Q} \psi_{\beta}(H) Q\) for some invertible matrix \(Q\) and that \(\beta = \set{\seq{v}{1,,n}}\).
  Let \(\gamma = \set{\seq{w}{1,,n}}\), where
  \[
    w_j = \sum_{i = 1}^n Q_{i j} v_i \quad \text{for } j \in \set{1, \dots, n}.
  \]
  Since \(Q\) is invertible, \(\gamma\) is an ordered basis for \(\V\) over \(\F\) (\cref{2.2,2.3}), and \(Q\) is the change of coordinate matrix that changes \(\gamma\)-coordinates into \(\beta\)-coordinates (\cref{2.5.1}).
  Therefore, by \cref{6.33},
  \[
    B = \tp{Q} \psi_{\beta}(H) Q = \psi_{\gamma}(H).
  \]
\end{proof}

\exercisesection

\begin{ex}\label{ex:6.8.12}
  Prove that the relation of congruence is an equivalence relation.
\end{ex}

\begin{proof}[\pf{ex:6.8.12}]
  Let \(A, B, C \in \ms[n][n][\F]\).
  Since \(A = \tp{I_n} A I_n\), by \cref{6.8.9} we know that the congruence relation is reflexive.
  Since
  \begin{align*}
             & B \text{ is congruent to } A                                           \\
    \implies & \exists Q \in \ms[n][n][\F] : \begin{dcases}
                                               Q \text{ is invertible} \\
                                               B = \tp{Q} A Q
                                             \end{dcases}        &  & \by{6.8.9}      \\
    \implies & A = (\tp{Q})^{-1} B Q^{-1} = \tp{(Q^{-1})} B Q^{-1} &  & \by{ex:2.4.5} \\
    \implies & A \text{ is congruent to } B,
  \end{align*}
  by \cref{6.8.9} we know that the congruence relation is symmetric.
  Since
  \begin{align*}
             & \begin{dcases}
                 B \text{ is congruent to } A \\
                 C \text{ is congruent to } B
               \end{dcases}                                    \\
    \implies & \exists P, Q \in \ms[n][n][\F] : \begin{dcases}
                                                  P, Q \text{ are invertible} \\
                                                  B = \tp{Q} A Q              \\
                                                  C = \tp{P} B P
                                                \end{dcases} &  & \by{6.8.9}   \\
    \implies & C = \tp{P} \tp{Q} A Q P = \tp{QP} A QP          &  & \by{2.3.2} \\
    \implies & C \text{ is congruent to } A,                   &  & \by{6.8.9}
  \end{align*}
  by \cref{6.8.9} we know that the congruence relation is transitive.
  Thus the congruence relation is an equivalence relation.
\end{proof}

\begin{ex}\label{ex:6.8.13}
\end{ex}

\section{Unitary and Orthogonal Operators and Their Matrices}\label{sec:6.5}

\begin{note}
	In \cref{sec:6.5}, we study those linear operators \(\T\) on an inner product space \(\V\) over \(\F\) such that \(\T \T^* = \T^* \T = \IT[\V]\).
	We will see that these are precisely the linear operators that ``preserve length'' in the sense that \(\norm{\T(x)} = \norm{x}\) for all \(x \in \V\).
	As another characterization, we prove that, on a finite-dimensional complex inner product space, these are the normal operators whose eigenvalues all have absolute value \(1\).

	In past chapters, we were interested in studying those functions that preserve the structure of the underlying space.
	In particular, linear operators preserve the operations of vector addition and scalar multiplication, and isomorphisms preserve all the vector space structure.
	It is now natural to consider those linear operators \(\T\) on an inner product space that preserve length.
	We will see that this condition guarantees, in fact, that \(\T\) preserves the inner product.
\end{note}

\begin{defn}\label{6.5.1}
	Let \(\T\) be a linear operator on a finite-dimensional inner product space \(\V\) over \(\F\).
	If \(\norm{\T(x)} = \norm{x}\) for all \(x \in \V\), we call \(\T\) a \textbf{unitary operator} if \(\F = \C\) and an \textbf{orthogonal operator} if \(\F = \R\).

	in the infinite-dimensional case, an operator satisfying the preceding norm requirement is generally called an \textbf{isometry}.
	If, in addition, the operator is onto (the condition guarantees one-to-one, see \cref{ex:6.1.17}), then the operator is called a \textbf{unitary} or \textbf{orthogonal operator}.
\end{defn}

\begin{note}
	Clearly, any rotation or reflection in \(\R^2\) preserves length and hence is an orthogonal operator.
	We study these operators in much more detail in \cref{sec:6.11}.
\end{note}

\begin{eg}\label{6.5.2}
	Let \(h \in \vs{H}\) (see \cref{6.1.8}) satisfy \(\abs{h(x)} = 1\) for all \(x \in [0, 2 \pi]\).
	Define \(\T \in \ls(\vs{H})\) by \(\T(f) = hf\).
	Then
	\begin{align*}
		\norm{\T(f)}^2 & = \norm{hf}^2                                                                     \\
		               & = \inn{hf, hf}                                                    &  & \by{6.1.9} \\
		               & = \frac{1}{2 \pi} \int_0^{2 \pi} h(t) f(t) \conj{h(t) f(t)} \; dt &  & \by{6.1.8} \\
		               & = \inn{f, f}                                                      &  & \by{6.1.8} \\
		               & = \norm{f}^2                                                      &  & \by{6.1.9}
	\end{align*}
	since \(\abs{h(t)}^2 = 1\) for all \(t \in [0, 2 \pi]\).
	So \(\T\) is a unitary operator.
\end{eg}

\begin{lem}\label{6.5.3}
	Let \(\U\) be a self-adjoint operator on a finite-dimensional inner product space \(\V\) over \(\F\).
	If \(\inn{x, \U(x)} = 0\) for all \(x \in \V\), then \(\U = \zT\).
\end{lem}

\begin{proof}[\pf{6.5.3}]
	By either \cref{6.16} or \cref{6.17}, we may choose an orthonormal basis \(\beta\) for \(\V\) over \(\F\) consisting of eigenvectors of \(\U\).
	If \(x \in \beta\), then \(\U(x) = \lambda x\) for some \(\lambda\).
	Thus
	\begin{align*}
		0 & = \inn{x, \U(x)}                              \\
		  & = \inn{x, \lambda x}         &  & \by{5.1.2}  \\
		  & = \conj{\lambda} \inn{x, x}; &  & \by{6.1}[b]
	\end{align*}
	so \(\lambda = 0\).
	Hence \(\U(x) = \zv\) for all \(x \in \beta\), and thus \(\U = \zT\).
\end{proof}

\begin{note}
	Compare \cref{6.5.3} to \cref{ex:6.4.11}(b).
\end{note}

\begin{thm}\label{6.18}
	Let \(\T\) be a linear operator on a finite-dimensional inner product space \(\V\) over \(\F\).
	Then the following statements are equivalent.
	\begin{enumerate}
		\item \(\T \T^* = \T^* \T = \IT[\V]\).
		\item \(\inn{\T(x), \T(y)} = \inn{x, y}\) for all \(x, y \in \V\).
		\item If \(\beta\) is an orthonormal basis for \(\V\) over \(\F\), then \(\T(\beta)\) is an orthonormal basis for \(\V\) over \(\F\).
		\item There exists an orthonormal basis \(\beta\) for \(\V\) over \(\F\) such that \(\T(\beta)\) is an orthonormal basis for \(\V\) over \(\F\).
		\item \(\norm{\T(x)} = \norm{x}\) for all \(x \in \V\).
	\end{enumerate}
	Thus all the conditions above are equivalent to the definition of a unitary or orthogonal operator.
	From (a), it follows that unitary or orthogonal operators are normal.
\end{thm}

\begin{proof}[\pf{6.18}]
	We prove first that (a) implies (b).
	Let \(x, y \in \V\).
	Then \(\inn{x, y} = \inn{\T^* \T(x), y} = \inn{\T(x), \T(y)}\).

	Second, we prove that (b) implies (c).
	Let \(\beta = \set{\seq{v}{1,,n}}\) be an orthonormal basis for \(\V\) over \(\F\);
	so \(\T(\beta) = \set{\T(v_1), \dots, \T(v_n)}\).
	It follows that \(\inn{\T(v_i), \T(v_j)} = \inn{v_i, v_j} = \delta_{i j}\).
	Therefore \(\T(\beta)\) is an orthonormal basis for \(\V\) over \(\F\).

	That (c) implies (d) is obvious.

	Next we prove that (d) implies (e).
	Let \(x \in \V\), and let \(\beta = \set{\seq{v}{1,,n}}\).
	Now
	\[
		x = \sum_{i = 1}^n a_i v_i
	\]
	for some scalars \(a_i \in \F\), and so
	\begin{align*}
		\norm{x}^2 & = \inn{\sum_{i = 1}^n a_i v_i, \sum_{j = 1}^n a_j v_j}         &  & \by{6.1.9}     \\
		           & = \sum_{i = 1}^n  a_i \inn{v_i, \sum_{j = 1}^n \conj{a_j} v_j} &  & \by{6.1.1}[ab] \\
		           & = \sum_{i = 1}^n \sum_{j = 1}^n a_i \conj{a_j} \inn{v_i, v_j}  &  & \by{6.1}[ab]   \\
		           & = \sum_{i = 1}^n \sum_{j = 1}^n a_i \conj{a_j} \delta_{i j}    &  & \by{6.1.12}    \\
		           & = \sum_{i = 1}^n a_i \conj{a_i}                                &  & \by{2.3.4}     \\
		           & = \sum_{i = 1}^n \abs{a_i}^2                                   &  & \by{d.0.5}
	\end{align*}
	since \(\beta\) is orthonormal.

	Applying the same manipulations to
	\[
		\T(x) = \sum_{i = 1}^n a_i \T(v_i)
	\]
	and using the fact that \(\T(\beta)\) is also orthonormal, we obtain
	\[
		\norm{\T(x)}^2 = \sum_{i = 1}^n \abs{a_i}^2.
	\]
	Hence \(\norm{\T(x)} = \norm{x}\).

	Finally, we prove that (e) implies (a).
	For any \(x \in \V\), we have
	\begin{align*}
		\inn{x, x} & = \norm{x}^2           &  & \by{6.1.9} \\
		           & = \norm{\T(x)}^2                       \\
		           & = \inn{\T(x), \T(x)}   &  & \by{6.1.9} \\
		           & = \inn{x, \T^* \T(x)}. &  & \by{6.9}
	\end{align*}
	So \(\inn{x, (\IT[\V] - \T^* \T)(x)} = 0\) for all \(x \in \V\).
	Let \(\U = \IT[\V] - \T^* \T\);
	then \(\U\) is self-adjoint (by \cref{6.11}) and \(\inn{x, \U(x)} = 0\) for all \(x \in \V\).
	Hence, by \cref{6.5.3}, we have \(\zT = \U = \IT[\V] - \T^* \T\), and therefore \(\T^* \T = \IT[\V]\).
	Since \(\V\) is finite-dimensional, we may use \cref{ex:2.4.10} to conclude that \(\T \T^* = \IT[\V]\).
\end{proof}

\begin{cor}\label{6.5.4}
	Let \(\T\) be a linear operator on a finite-dimensional real inner product space \(\V\).
	Then \(\V\) has an orthonormal basis of eigenvectors of \(\T\) with corresponding eigenvalues of absolute value \(1\) iff \(\T\) is both self-adjoint and orthogonal.
\end{cor}

\begin{proof}[\pf{6.5.4}]
	Suppose that \(\V\) has an orthonormal basis \(\set{\seq{v}{1,,n}}\) over \(\R\) such that \(\T(v_i) = \lambda_i v_i\) and \(\abs{\lambda_i} = 1\) for all \(i \in \set{1, \dots, n}\).
	By \cref{6.17}, \(\T\) is self-adjoint.
	Thus
	\begin{align*}
		(\T \T^*)(v_i) & = (\T \T)(v_i)            &  & \by{6.4.8}                                          \\
		               & = \T(\lambda_i v_i)       &  & \by{5.1.2}                                          \\
		               & = \lambda_i \lambda_i v_i &  & \by{5.1.2}                                          \\
		               & = \lambda_i^2 v_i                                                                  \\
		               & = v_i                     &  & (\abs{\lambda_i} = 1 \text{ and } \lambda_i \in \R)
	\end{align*}
	for each \(i \in \set{1, \dots, n}\).
	So \(\T \T^* = \IT[\V]\), and again by \cref{ex:2.4.10}, \(\T\) is orthogonal by \cref{6.18}(a).

	If \(\T\) is self-adjoint, then, by \cref{6.17}, we have that \(\V\) possesses an orthonormal basis \(\set{\seq{v}{1,,n}}\) over \(\R\) such that \(\T(v_i) = \lambda_i v_i\) for all \(i \in \set{1, \dots, n}\).
	If \(\T\) is also orthogonal, we have
	\begin{align*}
		\abs{\lambda_i} \cdot \norm{v_i} & = \norm{\lambda_i v_i} &  & \by{6.2}[a] \\
		                                 & = \norm{\T(v_i)}       &  & \by{5.1.2}  \\
		                                 & = \norm{v_i};          &  & \by{6.5.1}
	\end{align*}
	so \(\abs{\lambda_i} = 1\) for every \(i \in \set{1, \dots, n}\).
\end{proof}

\begin{cor}\label{6.5.5}
	Let \(\T\) be a linear operator on a finite-dimensional complex inner product space \(\V\).
	Then \(\V\) has an orthonormal basis of eigenvectors of \(\T\) with corresponding eigenvalues of absolute value \(1\) iff \(\T\) is unitary.
\end{cor}

\begin{proof}[\pf{6.5.5}]
	Suppose that \(\V\) has an orthonormal basis \(\set{\seq{v}{1,,n}}\) over \(\C\) such that \(\T(v_i) = \lambda_i v_i\) and \(\abs{\lambda_i} = 1\) for all \(i \in \set{1, \dots, n}\).
	By \cref{6.16}, \(\T\) is normal.
	Thus
	\begin{align*}
		(\T^* \T)(v_i) & = (\T \T^*)(v_i)                 &  & \by{6.4.3}            \\
		               & = \T(\conj{\lambda_i} v_i)       &  & \by{6.15}[c]          \\
		               & = \conj{\lambda_i} \lambda_i v_i &  & \by{5.1.2}            \\
		               & = \abs{\lambda_i}^2 v_i          &  & \by{d.0.5}            \\
		               & = v_i                            &  & (\abs{\lambda_i} = 1)
	\end{align*}
	for each \(i \in \set{1, \dots, n}\).
	So \(\T^* \T = \T \T^* = \IT[\V]\) by \cref{2.1.13}.
	Thus by \cref{6.18}(a) \(\T\) is unitary.

	If \(\T\) is unitary, then by \cref{6.18}(a)(e) we know that \(\T\) is normal.
	By \cref{6.16}, we have that \(\V\) possesses an orthonormal basis \(\set{\seq{v}{1,,n}}\) over \(\C\) such that \(\T(v_i) = \lambda_i v_i\) for all \(i \in \set{1, \dots, n}\).
	Then we have
	\begin{align*}
		\abs{\lambda_i} \cdot \norm{v_i} & = \norm{\lambda_i v_i} &  & \by{6.2}[a] \\
		                                 & = \norm{\T(v_i)}       &  & \by{5.1.2}  \\
		                                 & = \norm{v_i};          &  & \by{6.5.1}
	\end{align*}
	so \(\abs{\lambda_i} = 1\) for every \(i \in \set{1, \dots, n}\).
\end{proof}

\section{Unitary and Orthogonal Operators and Their Matrices}\label{sec:6.5}

\begin{note}
  In \cref{sec:6.5}, we study those linear operators \(\T\) on an inner product space \(\V\) over \(\F\) such that \(\T \T^* = \T^* \T = \IT[\V]\).
  We will see that these are precisely the linear operators that ``preserve length'' in the sense that \(\norm{\T(x)} = \norm{x}\) for all \(x \in \V\).
  As another characterization, we prove that, on a finite-dimensional complex inner product space, these are the normal operators whose eigenvalues all have absolute value \(1\).

  In past chapters, we were interested in studying those functions that preserve the structure of the underlying space.
  In particular, linear operators preserve the operations of vector addition and scalar multiplication, and isomorphisms preserve all the vector space structure.
  It is now natural to consider those linear operators \(\T\) on an inner product space that preserve length.
  We will see that this condition guarantees, in fact, that \(\T\) preserves the inner product.
\end{note}

\begin{defn}\label{6.5.1}
  Let \(\T\) be a linear operator on a finite-dimensional inner product space \(\V\) over \(\F\).
  If \(\norm{\T(x)} = \norm{x}\) for all \(x \in \V\), we call \(\T\) a \textbf{unitary operator} if \(\F = \C\) and an \textbf{orthogonal operator} if \(\F = \R\).

  in the infinite-dimensional case, an operator satisfying the preceding norm requirement is generally called an \textbf{isometry}.
  If, in addition, the operator is onto (the condition guarantees one-to-one, see \cref{ex:6.1.17}), then the operator is called a \textbf{unitary} or \textbf{orthogonal operator}.
\end{defn}

\begin{note}
  Clearly, any rotation or reflection in \(\R^2\) preserves length and hence is an orthogonal operator.
  We study these operators in much more detail in \cref{sec:6.11}.
\end{note}

\begin{eg}\label{6.5.2}
  Let \(h \in \vs{H}\) (see \cref{6.1.8}) satisfy \(\abs{h(x)} = 1\) for all \(x \in [0, 2 \pi]\).
  Define \(\T \in \ls(\vs{H})\) by \(\T(f) = hf\).
  Then
  \begin{align*}
    \norm{\T(f)}^2 & = \norm{hf}^2                                                                                   \\
                   & = \inn{hf, hf}                                                    &  & \text{(by \cref{6.1.9})} \\
                   & = \frac{1}{2 \pi} \int_0^{2 \pi} h(t) f(t) \conj{h(t) f(t)} \; dt &  & \text{(by \cref{6.1.8})} \\
                   & = \inn{f, f}                                                      &  & \text{(by \cref{6.1.8})} \\
                   & = \norm{f}^2                                                      &  & \text{(by \cref{6.1.9})}
  \end{align*}
  since \(\abs{h(t)}^2 = 1\) for all \(t \in [0, 2 \pi]\).
  So \(\T\) is a unitary operator.
\end{eg}

\section{The Change of Coordinate Matrix}\label{sec:2.5}

\begin{note}
  Geometrically, the change of variable
  \[
    \begin{pmatrix}
      x \\
      y
    \end{pmatrix} \to \begin{pmatrix}
      x' \\
      y'
    \end{pmatrix}
  \]
  is a change in the way that the position of a point \(P\) in the plane is described.
  This is done by introducing a new frame of reference, an \(x' y'\)-coordinate system with coordinate axes rotated from the original \(xy\)-coordinate axes.
\end{note}

\begin{thm}\label{2.22}
  Let \(\beta\) and \(\beta'\) be two ordered bases over \(\F\) for a finite-dimensional vector space \(\V\) over \(\F\), and let \(Q = [\IT[\V]]_{\beta'}^{\beta}\).
  Then
  \begin{enumerate}
    \item \(Q\) is invertible.
    \item For any \(v \in \V\), \([v]_{\beta} = Q [v]_{\beta'}\).
  \end{enumerate}
\end{thm}

\begin{proof}[\pf{2.22}(a)]
  Since \(\IT[\V]\) is invertible, \(Q\) is invertible by \cref{2.18}.
\end{proof}

\begin{proof}[\pf{2.22}(b)]
  For any \(v \in \V\),
  \[
    [v]_{\beta} = [\IT[\V](v)]_{\beta} = [\IT[\V]]_{\beta'}^{\beta} [v]_{\beta'} = Q [v]_{\beta'}
  \]
  by \cref{2.14}.
\end{proof}

\begin{defn}\label{2.5.1}
  The matrix \(Q = [\IT[\V]]_{\beta'}^{\beta}\) defined in \cref{2.22} is called a \textbf{change of coordinate matrix}.
  Because of \cref{2.22}(b), we say that \(Q\) \textbf{changes \(\beta'\)-coordinates into \(\beta\)-coordinates}.
  Observe that if \(\beta = \set{\seq{x}{1,2,,n}}\) and \(\beta' = \set{x_1', x_2', \dots, x_n'}\), then
  \[
    x_j' = \sum_{i = 1}^n Q_{i j} x_i
  \]
  for \(j \in \set{1, 2, \dots, n}\);
  that is, the \(j\)th column of \(Q\) is \([x_j']_{\beta}\).
\end{defn}

\begin{note}
  If \(Q\) changes \(\beta'\)-coordinates into \(\beta\)-coordinates, then \(Q^{-1}\) changes \(\beta\)-coordinates into \(\beta'\)-coordinates
  (See \cref{ex:2.5.11}).
\end{note}

\begin{defn}\label{2.5.2}
  For the remainder of this section, we consider only linear transformations that map a vector space \(\V\) over \(\F\) into itself.
  Such a linear transformation is called a \textbf{linear operator} on \(\V\) over \(\F\).
  Suppose now that \(\T\) is a linear operator on a finite-dimensional vector space \(\V\) over \(\F\) and that \(\beta\) and \(\beta'\) are ordered bases for \(\V\) over \(\F\).
  Then \(\V\) can be represented by the matrices \([\T]_{\beta}\) and \([\T]_{\beta'}\)
  (See \cref{2.2.4}).
\end{defn}

\begin{thm}\label{2.23}
  Let \(\T\) be a linear operator on a finite-dimensional vector space \(\V\) over \(\F\), and let \(\beta\) and \(\beta'\) be ordered bases for \(\V\) over \(\F\).
  Suppose that \(Q\) is the change of coordinate matrix that changes \(\beta'\)-coordinates into \(\beta\)-coordinates.
  Then
  \[
    [\T]_{\beta'} = Q^{-1} [\T]_{\beta} Q.
  \]
\end{thm}

\begin{proof}[\pf{2.23}]
  Let \(\IT[\V]\) be the identity transformation on \(\V\).
  Then \(\T = \IT[\V] \T = \T \IT[\V]\);
  hence, by \cref{2.11},
  \[
    Q [\T]_{\beta'} = [\IT[\V]]_{\beta'}^{\beta} [\T]_{\beta'}^{\beta'} = [\IT[\V] \T]_{\beta'}^{\beta} = [\T \IT[\V]]_{\beta'}^{\beta} = [\T]_{\beta}^{\beta} [\IT[\V]]_{\beta'}^{\beta} = [\T] Q.
  \]
  Therefore \([\T]_{\beta'} = Q^{-1} [\T]_{\beta} Q\).
\end{proof}

\begin{cor}\label{2.5.3}
  Let \(A \in \ms{n}{n}{\F}\), and let \(\gamma\) be an ordered basis for \(\vs{F}^n\) over \(\F\).
  Then \([\L_A]_{\gamma} = Q^{-1} A Q\), where \(Q \in \ms{n}{n}{\F}\) and the \(j\)th column of \(Q\) is the \(j\)th vector of \(\gamma\).
\end{cor}

\begin{proof}[\pf{2.5.3}]
  Let \(\gamma = \set{\seq{v}{1,,n}}\) and let \(\beta = \set{\seq{e}{1,,n}}\) be the standard ordered basis for \(\vs{F}^n\) over \(\F\).
  Then we have
  \begin{align*}
             & \forall j \in \set{1, \dots, n}, v_j = \begin{pmatrix}
                                                        Q_{1 j} \\
                                                        \vdots  \\
                                                        Q_{n j}
                                                      \end{pmatrix} = \sum_{i = 1}^n Q_{i j} e_i &  & \text{(by \cref{1.6.3})} \\
    \implies & Q = [\IT_{\vs{F}^n}]_{\gamma}^{\beta}                  &  & \text{(by \cref{2.5.1})}                            \\
    \implies & [\L_A]_{\gamma} = Q^{-1} [\L_A]_{\beta} Q              &  & \text{(by \cref{2.23})}                             \\
             & = Q^{-1} A Q.                                          &  & \text{(by \cref{2.15}(a))}
  \end{align*}
\end{proof}

\begin{defn}\label{2.5.4}
  Let \(A, B \in \ms{n}{n}{\F}\).
  We say that \(B\) is \textbf{similar} to \(A\) if there exists an invertible matrix \(Q\) such that \(B = Q^{-1} A Q\).
\end{defn}

\begin{note}
  Observe that the relation of similarity is an equivalence relation
  (see \cref{ex:2.5.9}).
  So we need only say that \(A\) and \(B\) are similar.
\end{note}

\begin{note}
  In term of \cref{2.5.4}, \cref{2.23} can be stated as follows:
  If \(\T\) is a linear operator on a finite-dimensional vector space \(\V\) over \(\F\), and if \(\beta\) and \(\beta'\) are any ordered bases for \(\V\) over \(\F\), then \([\T]_{\beta}\) and \([\T]_{\beta'}\) are similar.
\end{note}

\begin{note}
  \cref{2.23} can be generalized to allow \(\T \in \ls(\V, \W)\), where \(\V, \W\) are vector spaces over \(\F\) and \(\V\) is distinct from \(\W\).
  In this case, we can change bases in \(\V\) as well as in \(\W\)
  (see \cref{ex:2.5.8}).
\end{note}

\exercisesection

\setcounter{ex}{7}
\begin{ex}\label{ex:2.5.8}
  Prove the following generalization of \cref{2.23}.
  Let \(\T \in \ls(\V, \W)\) where \(\V, \W\) are finite-dimensional vector spaces over \(\F\).
  Let \(\beta\) and \(\beta'\) be ordered bases for \(\V\) over \(\F\), and let \(\gamma\) and \(\gamma'\) be ordered bases for \(\W\) over \(\F\).
  Then \([\T]_{\beta'}^{\gamma'} = P^{-1} [\T]_{\beta}^{\gamma} Q\), where \(Q\) is the matrix that changes \(\beta'\)-coordinates into \(\beta\)-coordinates and \(P\) is the matrix that changes \(\gamma'\)-coordinates into \(\gamma\)-coordinates.
\end{ex}

\begin{proof}[\pf{ex:2.5.8}]
  We have
  \begin{align*}
    P [\T]_{\beta'}^{\gamma'} & = [\IT[\W]]_{\gamma'}^{\gamma} [\T]_{\beta'}^{\gamma'} &  & \text{(by \cref{2.5.1})} \\
                              & = [\IT[\W] \T]_{\beta'}^{\gamma}                       &  & \text{(by \cref{2.11})}  \\
                              & = [\T]_{\beta'}^{\gamma}                               &  & \text{(by \cref{2.1.9})} \\
                              & = [\T \IT[\V]]_{\beta'}^{\gamma}                       &  & \text{(by \cref{2.1.9})} \\
                              & = [\T]_{\beta}^{\gamma} [\IT[\V]]_{\beta'}^{\beta}     &  & \text{(by \cref{2.11})}  \\
                              & = [\T]_{\beta}^{\gamma} Q                              &  & \text{(by \cref{2.5.1})}
  \end{align*}
  and thus
  \begin{align*}
             & P [\T]_{\beta'}^{\gamma'} = [\T]_{\beta}^{\gamma} Q                                               \\
    \implies & P^{-1} P [\T]_{\beta'}^{\gamma'} = P^{-1} [\T]_{\beta}^{\gamma} Q &  & \text{(by \cref{2.22}(a))} \\
    \implies & [\T]_{\beta'}^{\gamma'} = P^{-1} [\T]_{\beta}^{\gamma} Q.         &  & \text{(by \cref{2.4.3})}
  \end{align*}
\end{proof}

\begin{ex}\label{ex:2.5.9}
  Prove that ``is similar to'' is an equivalence relation on \(\ms{n}{n}{\F}\).
\end{ex}

\begin{proof}[\pf{ex:2.5.9}]
  Let \(A, B, C \in \ms{n}{n}{\F}\).
  \begin{description}
    \item[For reflexive:]
      We have
      \[
        A = I_n A I_n = I_n^{-1} A I_n.
      \]
    \item[For symmetric:]
      Suppose that \(B = Q^{-1} A Q\).
      Let \(P = Q^{-1}\).
      Then we have
      \begin{align*}
             & B = Q^{-1} A Q                                                                 \\
        \iff & Q B Q^{-1} = Q Q^{-1} A Q Q^{-1} = I_n A I_n = A &  & \text{(by \cref{2.4.3})} \\
        \iff & P^{-1} B P = A.                                  &  & \text{(by \cref{2.4.3})}
      \end{align*}
    \item[For transitive:]
      Suppose that \(B = Q^{-1} A Q\) and \(C = P^{-1} B P\).
      Then we have
      \begin{align*}
        C & = P^{-1} B P                                           \\
          & = P^{-1} Q^{-1} A Q P                                  \\
          & = (QP)^{-1} A (QP).   &  & \text{(by \cref{ex:2.4.4})}
      \end{align*}
  \end{description}
  From all cases above we see that ``is similar to'' defined in \cref{2.5.4} is an equivalence relation on \(\ms{n}{n}{\F}\).
\end{proof}

\begin{ex}\label{ex:2.5.10}
  Prove that if \(A, B \in \ms{n}{n}{\F}\) and \(A, B\) are similar, then \(\tr[A] = \tr[B]\).
\end{ex}

\begin{proof}[\pf{ex:2.5.10}]
  By \cref{2.5.4} we know that there exists a \(Q \in \ms{n}{n}{\F}\) such that \(B = Q^{-1} A Q\).
  Thus we have
  \begin{align*}
    \tr[B] & = \tr[Q^{-1} A Q]                                     \\
           & = \tr[(Q^{-1} A) Q] &  & \text{(by \cref{2.16})}      \\
           & = \tr[Q (Q^{-1} A)] &  & \text{(by \cref{ex:2.3.13})} \\
           & = \tr[(Q Q^{-1}) A] &  & \text{(by \cref{2.16})}      \\
           & = \tr[I_n A]        &  & \text{(by \cref{2.4.3})}     \\
           & = \tr[A].           &  & \text{(by \cref{2.4.3})}
  \end{align*}
\end{proof}

\begin{ex}\label{ex:2.5.11}
  Let \(\V\) be a finite-dimensional vector space over \(\F\) with ordered bases \(\alpha\), \(\beta\), and \(\gamma\).
  \begin{enumerate}
    \item Prove that if \(Q\) and \(R\) are the change of coordinate matrices that change \(\alpha\)-coordinates into \(\beta\)-coordinates and \(\beta\)-coordinates into \(\gamma\)-coordinates, respectively, then \(RQ\) is the change of coordinate matrix that changes \(\alpha\)-coordinates into \(\gamma\)-coordinates.
    \item Prove that if \(Q\) changes \(\alpha\)-coordinates into \(\beta\)-coordinates, then \(Q^{-1}\) changes \(\beta\)-coordinates into \(\alpha\)-coordinates.
  \end{enumerate}
\end{ex}

\begin{proof}[\pf{ex:2.5.11}(a)]
  We have
  \begin{align*}
    QR & = [\IT[\V]]_{\alpha}^{\beta} [\IT[\V]]_{\beta}^{\gamma} &  & \text{(by \cref{2.5.1})} \\
       & = [\IT[\V] \IT[\V]]_{\alpha}^{\gamma}                   &  & \text{(by \cref{2.11})}  \\
       & = [\IT[\V]]_{\alpha}^{\gamma}                           &  & \text{(by \cref{2.1.9})}
  \end{align*}
  and thus by \cref{2.5.1} \(QR\) is the change of coordinate matrix that changes \(\alpha\)-coordinates into \(\gamma\)-coordinates.
\end{proof}

\begin{proof}[\pf{ex:2.5.11}(b)]
  We have
  \begin{align*}
    Q^{-1} & = \pa{[\IT[\V]]_{\alpha}^{\beta}}^{-1} &  & \text{(by \cref{2.5.1})} \\
           & = [\IT[\V]^{-1}]_{\beta}^{\alpha}      &  & \text{(by \cref{2.18})}  \\
           & = [\IT[\V]]_{\beta}^{\alpha}.          &  & \text{(by \cref{2.1.9})}
  \end{align*}
\end{proof}

\setcounter{ex}{12}
\begin{ex}\label{ex:2.5.13}
  Let \(\V\) be a finite-dimensional vector space over a field \(\F\), and let \(\beta = \set{\seq{x}{1,2,,n}}\) be an ordered basis for \(\V\) over \(\F\).
  Let \(Q \in \ms{n}{n}{\F}\) and \(Q\) is invertible.
  Define
  \[
    x_j' = \sum_{i = 1}^n Q_{i j} x_i \quad \text{for } 1 \leq j \leq n,
  \]
  and set \(\beta' = \set{x_1', x_2', \dots, x_n'}\).
  Prove that \(\beta'\) is a basis for \(\V\) over \(\F\) and hence that \(Q\) is the change of coordinate matrix changing \(\beta'\)-coordinates into \(\beta\)-coordinates.
\end{ex}

\begin{proof}[\pf{ex:2.5.13}]
  Let \(a = \tp{\tuple{a}{1,,n}} \in \vs{F}^n\) such that
  \[
    \sum_{j = 1}^n a_j x_j' = \zv.
  \]
  Since
  \begin{align*}
             & \sum_{j = 1}^n a_j x_j' = \sum_{j = 1}^n \pa{a_j \sum_{i = 1}^n Q_{i j} x_i}                                                                           \\
             & = \sum_{j = 1}^n \pa{\sum_{i = 1}^n a_j Q_{i j} x_i} = \sum_{i = 1}^n \pa{\sum_{j = 1}^n a_j Q_{i j}} x_i = \zv &  & \text{(by \cref{1.2.1})}          \\
    \implies & \forall i \in \set{1, \dots, n}, \sum_{j = 1}^n a_j Q_{i j} = 0                                                 &  & \text{(by \cref{1.6.1})}          \\
    \implies & Qa = \begin{pmatrix}
                      Q_{1 1} & \cdots & Q_{1 n} \\
                      \vdots  & \ddots & \vdots  \\
                      Q_{n 1} & \cdots & Q_{n n}
                    \end{pmatrix} \cdot \begin{pmatrix}
                                          a_1    \\
                                          \vdots \\
                                          a_n
                                        \end{pmatrix} = \zm                                                                             &  & \text{(by \cref{2.3.1})} \\
    \implies & a = \zv,                                                                                                        &  & \text{(by \cref{ex:2.4.6})}
  \end{align*}
  by \cref{1.5.3} we know that \(\beta'\) is linearly independent.
  Thus by \cref{1.6.15}(b) \(\beta'\) is a basis for \(\V\) over \(\F\).
  We conclude by \cref{2.5.1} that \(Q\) is the change of coordinate matrix changin \(\beta'\)-coordinates into \(\beta\)-coordinates.
\end{proof}

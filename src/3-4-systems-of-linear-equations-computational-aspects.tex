\section{Systems of Linear Equations --- Computational Aspects}\label{sec:3.4}

\begin{defn}\label{3.4.1}
  Two systems of linear equations are called \textbf{equivalent} if they have the same solution set.
\end{defn}

\begin{thm}\label{3.13}
  Let \(Ax = b\) be a system of \(m\) linear equations in \(n\) unknowns, and let \(C \in \ms{m}{m}{\F}\) be invertible.
  Then the system \((CA)x = Cb\) is equivalent to \(Ax = b\).
\end{thm}

\begin{proof}[\pf{3.13}]
  Let \(K\) be the solution set for \(Ax = b\) and \(K'\) the solution set for \((CA)x = Cb\).
  If \(w \in K\), then \(Aw = b\).
  So \((CA)w = Cb\), and hence \(w \in K'\).
  Thus \(K \subseteq K'\).

  Conversely, if \(w \in K'\), then \((CA)w = Cb\).
  Hence
  \[
    Aw = C^{-1} (CAw) = C^{-1} (Cb) = b;
  \]
  so \(w \in K\).
  Thus \(K' \subseteq K\), and therefore, \(K = K'\).
\end{proof}

\begin{cor}\label{3.4.2}
  Let \(Ax = b\) be a system of \(m\) linear equations in \(n\) unknowns.
  If \((A' | b')\) is obtained from \((A | b)\) by a finite number of elementary row operations, then the system \(A' x = b'\) is equivalent to the original system.
\end{cor}

\begin{proof}[\pf{3.4.2}]
  Suppose that \((A' | b')\) is obtained from \((A | b)\) by elementary row operations.
  These may be executed by multiplying \((A | b)\) by elementary \(m \times m\) matrices \(\seq{E}{1,,p}\).
  Let \(C = \seq[]{E}{p,,1}\);
  then by \cref{ex:3.2.15}
  \[
    (A' | b') = C (A | b) = (CA | Cb).
  \]
  Since each \(E_i\) is invertible, so is \(C\)
  (see \cref{ex:2.4.4}).
  Now \(A' = CA\) and \(b' = Cb\).
  Thus by \cref{3.13}, the system \(A' x = b'\) is equivalent to the system \(Ax = b\).
\end{proof}

\begin{defn}\label{3.4.3}
  A matrix is said to be in \textbf{reduced row echelon form} if the following three conditions are satisfied.
  \begin{enumerate}
    \item Any row containing a nonzero entry precedes any row in which all the entries are zero (if any).
    \item The first nonzero entry in each row is the only nonzero entry in its column.
    \item The first nonzero entry in each row is \(1\) and it occurs in a column to the right of the first nonzero entry in the preceding row.
  \end{enumerate}
\end{defn}

\begin{note}
  It can be shown (see \cref{3.4.5}) that the reduced row echelon form of a matrix is unique;
  that is, if different sequences of elementary row operations are used to transform a matrix into matrices \(Q\) and \(Q'\) in reduced row echelon form, then \(Q = Q'\).
  Thus, although there are many different sequences of elementary row operations that can be used to transform a given matrix into reduced row echelon form, they all produce the same result.
\end{note}

\begin{defn}\label{3.4.4}
  The following procedure for reducing an augmented matrix to reduced row echelon form is called \textbf{Gaussian-Jordan elimination}.
  \begin{enumerate}[label=\arabic*.]
    \item In the leftmost nonzero column, create a \(1\) in the first row.
    \item By means of type 3 row operations, use the first row to obtain zeros in the remaining positions of the leftmost nonzero column.
    \item Create a \(1\) in the next row in the leftmost possible column, without using previous row(s).
    \item Now use type 3 elementary row operations to obtain zeros below the \(1\) created in the preceding step.
    \item Repeat steps 3 and 4 on each succeeding row until no nonzero rows remain.
          (This creates zeros below the first nonzero entry in each row.)
    \item Work upward, beginning with the last nonzero row, and add multiples of each row to the rows above.
    \item Repeat the process described in step 6 for each preceding row until it is performed with the second row, at which time the reduction process is complete.
  \end{enumerate}
  Note that
  \begin{itemize}
    \item In the forward pass (steps 1 -- 5), the augmented matrix is transformed into an upper triangular matrix in which the first nonzero entry of each row is \(1\), and it occurs in a column to the right of the first nonzero entry of each preceding row.
    \item In the backward pass or back-substitution (steps 6 -- 7), the upper triangular matrix is transformed into reduced row echelon form by making the first nonzero entry of each row the only nonzero entry of its column.
  \end{itemize}
\end{defn}

\begin{thm}\label{3.14}
  Gaussian elimination transforms any matrix into its reduced row echelon form.
\end{thm}

\begin{proof}[\pf{3.14}]
  First, however, we note that every matrix can be transformed into reduced row echelon form by Gaussian elimination.
  In the forward pass, we satisfy conditions \cref{3.4.3}(a) and (c) in the definition of reduced row echelon form and thereby make zero all entries below the first nonzero entry in each row.
  Then in the backward pass, we make zero all entries above the first nonzero entry in each row, thereby satisfying condition \cref{3.4.3}(b) in the definition of reduced row echelon form.
\end{proof}

\begin{thm}\label{3.15}

\end{thm}

\begin{thm}\label{3.16}

\end{thm}

\begin{cor}\label{3.4.5}

\end{cor}

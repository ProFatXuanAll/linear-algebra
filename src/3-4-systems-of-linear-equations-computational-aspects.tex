\section{Systems of Linear Equations --- Computational Aspects}\label{sec:3.4}

\begin{defn}\label{3.4.1}
  Two systems of linear equations are called \textbf{equivalent} if they have the same solution set.
\end{defn}

\begin{thm}\label{3.13}
  Let \(Ax = b\) be a system of \(m\) linear equations in \(n\) unknowns, and let \(C \in \ms{m}{m}{\F}\) be invertible.
  Then the system \((CA)x = Cb\) is equivalent to \(Ax = b\).
\end{thm}

\begin{proof}[\pf{3.13}]
  Let \(K\) be the solution set for \(Ax = b\) and \(K'\) the solution set for \((CA)x = Cb\).
  If \(w \in K\), then \(Aw = b\).
  So \((CA)w = Cb\), and hence \(w \in K'\).
  Thus \(K \subseteq K'\).

  Conversely, if \(w \in K'\), then \((CA)w = Cb\).
  Hence
  \[
    Aw = C^{-1} (CAw) = C^{-1} (Cb) = b;
  \]
  so \(w \in K\).
  Thus \(K' \subseteq K\), and therefore, \(K = K'\).
\end{proof}

\begin{cor}\label{3.4.2}
  Let \(Ax = b\) be a system of \(m\) linear equations in \(n\) unknowns.
  If \((A' | b')\) is obtained from \((A | b)\) by a finite number of elementary row operations, then the system \(A' x = b'\) is equivalent to the original system.
\end{cor}

\begin{proof}[\pf{3.4.2}]
  Suppose that \((A' | b')\) is obtained from \((A | b)\) by elementary row operations.
  These may be executed by multiplying \((A | b)\) by elementary \(m \times m\) matrices \(\seq{E}{1,,p}\).
  Let \(C = \seq[]{E}{p,,1}\);
  then by \cref{ex:3.2.15}
  \[
    (A' | b') = C (A | b) = (CA | Cb).
  \]
  Since each \(E_i\) is invertible, so is \(C\)
  (see \cref{ex:2.4.4}).
  Now \(A' = CA\) and \(b' = Cb\).
  Thus by \cref{3.13}, the system \(A' x = b'\) is equivalent to the system \(Ax = b\).
\end{proof}

\begin{defn}\label{3.4.3}
  A matrix is said to be in \textbf{reduced row echelon form} if the following three conditions are satisfied.
  \begin{enumerate}
    \item Any row containing a nonzero entry precedes any row in which all the entries are zero (if any).
    \item The first nonzero entry in each row is the only nonzero entry in its column.
    \item The first nonzero entry in each row is \(1\) and it occurs in a column to the right of the first nonzero entry in the preceding row.
  \end{enumerate}
\end{defn}

\begin{note}
  It can be shown (see \cref{3.4.5}) that the reduced row echelon form of a matrix is unique;
  that is, if different sequences of elementary row operations are used to transform a matrix into matrices \(Q\) and \(Q'\) in reduced row echelon form, then \(Q = Q'\).
  Thus, although there are many different sequences of elementary row operations that can be used to transform a given matrix into reduced row echelon form, they all produce the same result.
\end{note}

\begin{defn}\label{3.4.4}
  The following procedure for reducing an augmented matrix to reduced row echelon form is called \textbf{Gaussian-Jordan elimination}.
  \begin{enumerate}[label=\arabic*.]
    \item In the leftmost nonzero column, create a \(1\) in the first row.
    \item By means of type 3 row operations, use the first row to obtain zeros in the remaining positions of the leftmost nonzero column.
    \item Create a \(1\) in the next row in the leftmost possible column, without using previous row(s).
    \item Now use type 3 elementary row operations to obtain zeros below the \(1\) created in the preceding step.
    \item Repeat steps 3 and 4 on each succeeding row until no nonzero rows remain.
          (This creates zeros below the first nonzero entry in each row.)
    \item Work upward, beginning with the last nonzero row, and add multiples of each row to the rows above.
    \item Repeat the process described in step 6 for each preceding row until it is performed with the second row, at which time the reduction process is complete.
  \end{enumerate}
  Note that
  \begin{itemize}
    \item In the forward pass (steps 1 -- 5), the augmented matrix is transformed into an upper triangular matrix in which the first nonzero entry of each row is \(1\), and it occurs in a column to the right of the first nonzero entry of each preceding row.
    \item In the backward pass or back-substitution (steps 6 -- 7), the upper triangular matrix is transformed into reduced row echelon form by making the first nonzero entry of each row the only nonzero entry of its column.
  \end{itemize}
\end{defn}

\begin{thm}\label{3.14}
  Gaussian-Jordan elimination transforms any matrix into its reduced row echelon form.
\end{thm}

\begin{proof}[\pf{3.14}]
  In the forward pass, we satisfy \cref{3.4.3}(a)(c) and thereby make zero all entries below the first nonzero entry in each row.
  Then in the backward pass, we make zero all entries above the first nonzero entry in each row, thereby satisfying \cref{3.4.3}(b).
\end{proof}

\begin{note}
  To solve a system for which the augmented matrix is in reduced row echelon form, divide the variables into two sets.
  The first set consists of those variables that appear as leftmost variables in one of the equations of the system.
  The second set consists of all the remaining variables.
  To each variable in the second set, assign a parametric value \(\seq{t}{1,2,}\) and then solve for the variables of the first set in terms of those in the second set.

  Therefore, in simplifying the augmented matrix of the system to reduced row echelon form, we are in effect simultaneously finding a particular solution to the original system and a basis for the solution set of the associated homogeneous system.
  Moreover, this procedure detects when a system is inconsistent, for by \cref{ex:3.4.3}, solutions exist iff, in the reduction of the augmented matrix to reduced row echelon form, we do not obtain a row in which the only nonzero entry lies in the last column.

  Thus to use this procedure for solving a system \(Ax = b\) of \(m\) linear equations in \(n\) unknowns, we need only begin to transform the augmented matrix \((A | b)\) into its reduced row echelon form \((A' | b')\) by means of Gaussian-Jordan elimination.
  If a row is obtained in which the only nonzero entry lies in the last column, then the original system is inconsistent.
  Otherwise, discard any zero rows from \((A' | b')\), and write the corresponding system of equations.
  Solve this system as described above to obtain an arbitrary solution of the form
  \[
    s = s_0 + \seq[+]{t,u}{1,2,,n-r},
  \]
  where \(r\) is the number of nonzero rows in \(A'\) (\(r \leq m\)).
  The preceding equation is called a \textbf{general solution} of the system \(Ax = b\).
  It expresses an arbitrary solution \(s\) of \(Ax = b\) in terms of \(n - r\) parameters.
  \cref{3.15} states that \(s\) cannot be expressed in fewer than \(n - r\) parameters.
\end{note}

\begin{thm}\label{3.15}
  Let \(Ax = b\) be a system of \(r\) nonzero equations in \(n\) unknowns.
  Suppose that \(\rk{A} = \rk{A | b}\) and that \((A | b)\) is in reduced row echelon form.
  Then
  \begin{enumerate}
    \item \(\rk{A} = r\).
    \item If the general solution obtained by the procedure above is of the form
          \[
            s = s_0 + \seq[+]{t,u}{1,2,,n-r},
          \]
          then \(\set{\seq{u}{1,,n-r}}\) is a basis for the solution set of the corresponding homogeneous system, and \(s_0\) is a solution to the original system.
  \end{enumerate}
\end{thm}

\begin{proof}[\pf{3.15}]
  Since \((A | b)\) is in reduced row echelon form, \((A | b)\) must have \(r\) nonzero rows.
  Clearly these rows are linearly independent by the definition of the reduced row echelon form, and so \(\rk{A | b} = r\).
  Thus \(\rk{A} = r\).

  Let \(K\) be the solution set for \(Ax = b\), and let \(\vs{K}_{\vs{H}}\) be the solution set for \(Ax = \zv\).
  Setting \(\seq[=]{t}{1,,n-r} = 0\), we see that \(s = s_0 \in K\).
  But by \cref{3.9}, \(K = \set{s_0} + \vs{K}_{\vs{H}}\).
  Hence
  \[
    \vs{K}_{\vs{H}} = \set{-s_0} + K = \spn{\set{\seq{u}{1,,u-r}}}.
  \]
  Because \(\rk{A} = r\), by \cref{3.8} we have \(\dim(\vs{K}_{\vs{H}}) = n - r\).
  Thus since \(\dim(\vs{K}_{\vs{H}}) = n - r\) and \(\vs{K}_{\vs{H}}\) is generated by a set \(\set{\seq{u}{1,,n-r}}\) containing at most \(n - r\) vectors, we conclude that this set is a basis for \(\vs{K}_{\vs{H}}\).
\end{proof}

\begin{thm}\label{3.16}

\end{thm}

\begin{cor}\label{3.4.5}

\end{cor}

\exercisesection

\setcounter{ex}{2}
\begin{ex}\label{ex:3.4.3}
  Suppose that the augmented matrix of a system \(Ax = b\) is transformed into a matrix \((A' | b')\) in reduced row echelon form by a finite sequence of elementary row operations.
  \begin{enumerate}
    \item Prove that \(\rk{A'} \neq \rk{A' | b'}\) iff \((A' | b')\) contains a row in which the only nonzero entry lies in the last column.
    \item Deduce that \(Ax = b\) is consistent iff \((A' | b')\) contains no row in which the only nonzero entry lies in the last column.
  \end{enumerate}
\end{ex}

\begin{proof}[\pf{ex:3.4.3}(a)]
  First suppose that \(\rk{A'} \neq \rk{A' | b'}\).
  By \cref{3.5} we have \(\rk{A'} + 1 = \rk{A' | b'}\).
  Since \((A' | b')\) is in reduced row echelon form, we know that \(A'\) is also in reduced row echelon form.
  Let \(r\) be the number of nonzero rows in \(A'\).
  By \cref{3.4.3} we know that these rows are linearly independent and thus by \cref{3.2.5}(b) \(\rk{A'} = r\).
  But this means the first \(r\) rows of \((A' | b')\) is also linearly independent.
  Thus to have \(\rk{A'} + 1 = \rk{A' | b'}\), we must have one more nonzero rows in the remaining \(n - r\) rows of \((A' | b')\).
  By the definition of \(r\) we know that the extra nonzero row must have nonzero entry in the column of \(b'\).
  Since \((A' | b')\) is in reduced row echelon form, we know that \(b_{r + 1} = 1\) and \(b_{r + 1}\) is the only nonzero entry in \(b\), i.e., the last column of \((A' | b')\).

  Now suppose that \((A' | b')\) contains a row in which the only nonzero entry lies in the last column.
  Since \((A' | b')\) is in reduced row echelon form, we know that \(b' \notin \spn{\set{\text{columns of } A'}}\).
  Thus by \cref{3.5} we have \(\rk{A'} \neq \rk{A' | b'}\).
\end{proof}

\begin{proof}[\pf{ex:3.4.3}(b)]
  This is the immediate consequence of \cref{3.11,ex:3.4.3}(a).
\end{proof}

\section{The Jordan Canonical Form II}\label{sec:7.2}

\begin{defn}\label{7.2.1}
  For the purposes of this section, we fix a linear operator \(\T\) on an \(n\)-dimensional vector space \(\V\) over \(\F\) such that the characteristic polynomial of \(\T\) splits.
  Let \(\seq{\lambda}{1,,k}\) be the distinct eigenvalues of \(\T\).

  By \cref{7.7}, each generalized eigenspace \(\vs{K}_{\lambda_i}\) contains an ordered basis \(\beta_i\) consisting of a union of disjoint cycles of generalized eigenvectors corresponding to \(\lambda_i\).
  So by \cref{7.4}(b) and \cref{7.5}, the union \(\beta = \bigcup_{i = 1}^k \beta_i\) is a Jordan canonical basis for \(\T\).
  For each \(i \in \set{1, \dots, k}\), let \(\T_i\) be the restriction of \(\T\) to \(\vs{K}_{\lambda_i}\), and let \(A_i = [\T_i]_{\beta_i}\).
  Then \(A_i\) is a Jordan canonical form of \(\T_i\), and
  \[
    J = [\T]_{\beta} = \begin{pmatrix}
      A_1    & \zm    & \cdots & \zm    \\
      \zm    & A_2    & \cdots & \zm    \\
      \vdots & \vdots &        & \vdots \\
      \zm    & \zm    & \cdots & A_k
    \end{pmatrix}
  \]
  is a Jordan canonical form of \(\T\).
  In this matrix, each \(\zm\) is a zero matrix of appropriate size.

  In this section, we compute the matrices \(A_i\) and the bases \(\beta_i\), thereby computing \(J\) and \(\beta\) as well.
  While developing a method for finding \(J\), it becomes evident that in some sense the matrices \(A_i\) are unique.

  To aid in formulating the uniqueness theorem for \(J\), we adopt the following convention:
  The basis \(\beta_i\) for \(\vs{K}_{\lambda_i}\) will henceforth be ordered in such a way that the cycles appear in order of decreasing length.
  That is, if \(\beta_i\) is a disjoint union of cycles \(\seq{\gamma}{1,,n_i}\) and if the length of the cycle \(\gamma_j\) is \(p_j\), we index the cycles so that \(\seq[\geq]{p}{1,,n_i}\).
  This ordering of the cycles limits the possible orderings of vectors in \(\beta_i\), which in turn determines the matrix \(A_i\).
  It is in this sense that \(A_i\) is unique.
  It then follows that the Jordan canonical form for \(\T\) is unique up to an ordering of the eigenvalues of \(\T\).
  As we will see, there is no uniqueness theorem for the bases \(\beta_i\) or for \(\beta\)
  (See \cref{ex:7.2.8}, it is for this reason that we associate the dot diagram with \(\T_i\) rather than with \(\beta_i\)).
  Specifically, we show that for each \(i \in \set{1, \dots, k}\), the number \(n_i\) of cycles that form \(\beta_i\), and the length \(p_j\) (\(j \in \set{1, \dots, n_i}\)) of each cycle, is completely determined by \(\T\).

  To help us visualize each of the matrices \(A_i\) and ordered bases \(\beta_i\), we use an array of dots called a \textbf{dot diagram} of \(\T_i\), where \(\T_i\) is the restriction of \(\T\) to \(\vs{K}_{\lambda_i}\).
  Suppose that \(\beta_i\) is a disjoint union of cycles of generalized eigenvectors \(\seq{\gamma}{1,,n_i}\) with lengths \(\seq[\geq]{p}{1,,n_i}\), respectively.
  The dot diagram of \(\T_i\) contains one dot for each vector in \(\beta_i\), and the dots are configured according to the following rules.
  \begin{itemize}
    \item The array consists of \(n_i\) columns (one column for each cycle).
    \item Counting from left to right, the \(j\)th column consists of the \(p_j\) dots that correspond to the vectors of \(\gamma_j\) starting with the initial vector at the top and continuing down to the end vector.
  \end{itemize}

  Denote the end vectors of the cycles by \(\seq{v}{1,,n_i}\).
  In the following dot diagram of \(\T_i\), each dot is labeled with the name of the vector in \(\beta_i\) to which it corresponds.
  \begin{align*}
     & \bullet (\T - \lambda_i \IT[\V])^{p_1 - 1}(v_1) &  & \bullet (\T - \lambda_i \IT[\V])^{p_2 - 1}(v_2) &  & \cdots &  & \bullet (\T - \lambda_i \IT[\V])^{p_{n_i} - 1}(v_{n_i}) \\
     & \bullet (\T - \lambda_i \IT[\V])^{p_1 - 2}(v_1) &  & \bullet (\T - \lambda_i \IT[\V])^{p_2 - 2}(v_2) &  & \cdots &  & \bullet (\T - \lambda_i \IT[\V])^{p_{n_i} - 2}(v_{n_i}) \\
     & \vdots                                          &  & \vdots                                          &  &        &  & \vdots                                                  \\
     &                                                 &  &                                                 &  &        &  & \bullet (\T - \lambda_i \IT[\V])(v_{n_i})               \\
     &                                                 &  &                                                 &  &        &  & \bullet v_{n_i}                                         \\
     &                                                 &  & \bullet (\T - \lambda_i \IT[\V])(v_2)           &  &        &  &                                                         \\
     &                                                 &  & \bullet v_2                                     &  &        &  &                                                         \\
     & \bullet (\T - \lambda_i \IT[\V])(v_1)           &  &                                                 &  &        &  &                                                         \\
     & \bullet v_1                                     &  &                                                 &  &        &  &
  \end{align*}
  Notice that the dot diagram of \(\T_i\) has \(n_i\) columns (one for each cycle) and \(p_1\) rows.
  Since \(\seq[\geq]{p}{1,,n_i}\), the columns of the dot diagram become shorter (or at least not longer) as we move from left to right.

  Now let \(r_j\) denote the number of dots in the \(j\)th row of the dot diagram.
  Observe that \(\seq[\geq]{r}{1,,p_1}\).
  Furthermore, the diagram can be reconstructed from the values of the \(r_i\)'s.
  The proofs of these facts, which are combinatorial in nature, are treated in \cref{ex:7.2.9}.
\end{defn}

\begin{thm}\label{7.9}
  Using the notations in \cref{7.2.1}.
  For any positive integer \(r\), the vectors in \(\beta_i\) that are associated with the dots in the first \(r\) rows of the dot diagram of \(\T_i\) constitute a basis for \(\ns{(\T - \lambda_i \IT[\V])^r}\) over \(\F\).
  Hence the number of dots in the first \(r\) rows of the dot diagram equals \(\nt{(\T - \lambda_i \IT[\V])^r}\).
\end{thm}

\begin{proof}[\pf{7.9}]
  By \cref{7.1.4} \(\ns{(\T - \lambda_i \IT[\V])^r} \subseteq \vs{K}_{\lambda_i}\), and \(\vs{K}_{\lambda_i}\) is invariant under \((\T - \lambda_i \IT[\V])^r\).
  Let \(\U\) denote the restriction of \((\T - \lambda_i \IT[\V])^r\) to \(\vs{K}_{\lambda_i}\).
  By \cref{7.1}(b), \(\ns{(\T - \lambda_i \IT[\V])^r} = \ns{\U}\), and hence it suffices to establish the theorem for \(\U\).
  Now define
  \[
    S_1 = \set{x \in \beta_i : \U(x) = \zv} \quad \text{and} \quad S_2 = \set{x \in \beta_i : \U(x) \neq \zv}.
  \]
  Let \(a\) and \(b\) denote the number of vectors in \(S_1\) and \(S_2\), respectively, and let \(m_i = \dim(\vs{K}_{\lambda_i})\).
  Then \(a + b = m_i\).
  For any \(x \in \beta_i\), \(x \in S_1\) iff \(x\) is one of the first \(r\) vectors of a cycle (\cref{7.1.6}), and this is true iff \(x\) corresponds to a dot in the first \(r\) rows of the dot diagram (\cref{7.2.1}).
  Hence \(a\) is the number of dots in the first \(r\) rows of the dot diagram.
  For any \(x \in S_2\), the effect of applying \(\U\) to \(x\) is to move the dot corresponding to \(x\) exactly \(r\) places up its column to another dot.
  It follows that \(\U\) maps \(S_2\) in a one-to-one fashion into \(\beta_i\) (since \(\beta_i\) is linearly independent).
  Thus \(\set{\U(x) : x \in S_2}\) is a basis for \(\rg{\U}\) over \(\F\) consisting of \(b\) vectors (\cref{2.2}).
  Hence \(\rk{\U} = b\), and so \(\nt{\U} = m_i - b = a\) (\cref{2.3}).
  But \(S_1\) is a linearly independent subset of \(\ns{\U}\) consisting of \(a\) vectors;
  therefore \(S_1\) is a basis for \(\ns{\U}\).
\end{proof}

\begin{note}
  In the case that \(r = 1\), \cref{7.9} yields \cref{7.2.2}.
\end{note}

\begin{cor}\label{7.2.2}
  Using the notations in \cref{7.2.1}.
  The dimension of \(\vs{E}_{\lambda_i}\) is \(n_i\).
  Hence in a Jordan canonical form of \(\T\), the number of Jordan blocks corresponding to \(\lambda_i\) equals the dimension of \(\vs{E}_{\lambda_i}\).
\end{cor}

\begin{proof}[\pf{7.2.2}]
  By \cref{5.4} we have \(\vs{E}_{\lambda_i} = \ns{\T - \lambda_i \IT[\V]}\).
  Thus by \cref{7.9} the first row of the dot diagram of \(\T_i\) constitute a basis for \(\vs{E}_{\lambda_i}\) over \(\F\).
\end{proof}

\begin{thm}\label{7.10}
  Using the notations in \cref{7.2.1}.
  Let \(r_j\) denote the number of dots in the \(j\)th row of the dot diagram of \(\T_i\), the restriction of \(\T\) to \(\vs{K}_{\lambda_i}\).
  Then the following statements are true.
  \begin{enumerate}
    \item \(r_1 = \dim(\V) - \rk{\T - \lambda_i \IT[\V]}\).
    \item \(r_j = \rk{(\T - \lambda_i \IT[\V])^{j - 1}} - \rk{(\T - \lambda_i \IT[\V])^j}\) if \(j \in \set{2, \dots, p_1}\).
  \end{enumerate}
\end{thm}

\begin{proof}[\pf{7.10}]
  By \cref{7.9}, for \(j \in \set{1, \dots, p_1}\), we have
  \begin{align*}
    \seq[+]{r}{1,,j} & = \nt{(\T - \lambda_i \IT[\V])^j}             &  & \by{7.9} \\
                     & = \dim(\V) - \rk{(\T - \lambda_i \IT[\V])^j}. &  & \by{2.3}
  \end{align*}
  Hence \(r_1 = \dim(\V) - \rk{\T - \lambda_i \IT[\V]}\), and for \(j \in \set{2, \dots, p_1}\),
  \begin{align*}
    r_j & = (\seq[+]{r}{1,,j}) - (\seq[+]{r}{1,,j-1})                                                               \\
        & = \pa{\dim(\V) - \rk{(\T - \lambda_i \IT[\V])^j}} - \pa{\dim(\V) - \rk{(\T - \lambda_i \IT[\V])^{j - 1}}} \\
        & = \rk{(\T - \lambda_i \IT[\V])^{j - 1}} - \rk{(\T - \lambda_i \IT[\V])^j}.
  \end{align*}
\end{proof}

\begin{cor}\label{7.2.3}
  Using the notations in \cref{7.2.1}.
  For any eigenvalue \(\lambda_i\) of \(\T\), the dot diagram of \(\T_i\) is unique.
  Thus, subject to the convention that the cycles of generalized eigenvectors for the bases of each generalized eigenspace are listed in order of decreasing length, the Jordan canonical form of a linear operator or a matrix is unique up to the ordering of the eigenvalues.
\end{cor}

\begin{proof}[\pf{7.2.3}]
  \cref{7.10} shows that the dot diagram of \(\T_i\) is completely determined by \(\T\) and \(\lambda_i\).
\end{proof}

\begin{thm}\label{7.11}
  Let \(A\) and \(B\) be \(n \times n\) matrices, each having Jordan canonical forms computed according to the conventions of this section.
  Then \(A\) and \(B\) are similar iff they have (up to an ordering of their eigenvalues) the same Jordan canonical form.
\end{thm}

\begin{proof}[\pf{7.11}]
  If \(A\) and \(B\) have the same Jordan canonical form \(J\), then \(A\) and \(B\) are each similar to \(J\) and hence are similar to each other.

  Conversely, suppose that \(A\) and \(B\) are similar.
  Then \(A\) and \(B\) have the same eigenvalues (\cref{ex:5.1.12}).
  Let \(J_A\) and \(J_B\) denote the Jordan canonical forms of \(A\) and \(B\), respectively, with the same ordering of their eigenvalues.
  Then \(A\) is similar to both \(J_A\) and \(J_B\), and therefore, by the \cref{2.5.3}, \(J_A\) and \(J_B\) are matrix representations of \(\L_A\).
  Hence \(J_A\) and \(J_B\) are Jordan canonical forms of \(\L_A\). Thus \(J_A = J_B\) by the \cref{7.2.3}.
\end{proof}

\begin{cor}\label{7.2.4}
  A linear operator \(\T\) on a finite-dimensional vector space \(\V\) over \(\F\) is diagonalizable iff its Jordan canonical form is a diagonal matrix.
  Hence \(\T\) is diagonalizable iff the Jordan canonical basis for \(\T\) consists of eigenvectors of \(\T\).
\end{cor}

\begin{proof}[\pf{7.2.4}]
  By \cref{7.1.5,7.2.3} we see that this is true.
\end{proof}

\exercisesection

\setcounter{ex}{5}
\begin{ex}\label{ex:7.2.6}
  Let \(A \in \ms[n][n][\F]\) whose characteristic polynomial splits.
  Prove that \(A\) and \(\tp{A}\) have the same Jordan canonical form, and conclude that \(A\) and \(\tp{A}\) are similar.
\end{ex}

\begin{proof}[\pf{ex:7.2.6}]
  Let \(\lambda \in \F\) be an eigenvalue of \(A\).
  Since
  \begin{align*}
    \forall r \in \Z^+, \rk{\pa{A - \lambda I_n}^r} & = \rk{\tp{\pa{(A - \lambda I_n)^r}}} &  & \by{3.2.5}[a] \\
                                                    & = \rk{\pa{\tp{(A - \lambda I_n)}}^r} &  & \by{2.3.2}    \\
                                                    & = \rk{\pa{\tp{A} - \lambda I_n}^r},  &  & \by{ex:1.3.3}
  \end{align*}
  by \cref{7.10} we see that the dot diagrams of \(A\) and \(\tp{A}\) correspond to \(\lambda\) are the same.
  Since \(\lambda\) is arbitrary, by \cref{7.2.3} we conclude that \(A\) and \(\tp{A}\) have the same Jordan canonical form.
  By \cref{7.11} this means \(A\) and \(\tp{A}\) are similar.
\end{proof}

\begin{ex}\label{ex:7.2.7}
  Let \(\T\) be a linear operator on a finite-dimensional vector space \(\V\) over \(\F\) such that the characteristic polynomial of \(\T\) splits.
  Let \(\gamma\) be a cycle of generalized eigenvectors corresponding to an eigenvalue \(\lambda\), and \(\W\) be the subspace spanned by \(\gamma\).
  Define \(\gamma'\) to be the ordered set obtained from \(\gamma\) by reversing the order of the vectors in \(\gamma\).
  \begin{enumerate}
    \item Prove that \([\T_{\W}]_{\gamma'} = \tp{([\T_{\W}]_{\gamma})}\).
    \item Let \(J\) be the Jordan canonical form of \(\T\).
          Use (a) to prove that \(J\) and \(\tp{J}\) are similar.
    \item Let \(A \in \ms[n][n][\F]\) whose characteristic polynomial splits.
          Use (b) to prove that \(A\) and \(\tp{A}\) are similar.
  \end{enumerate}
\end{ex}

\begin{proof}[\pf{ex:7.2.7}(a)]
  Let \(\gamma = \set{\seq{v}{1,,m}}\).
  By \cref{7.1.6} we have
  \[
    \forall i \in \set{1, \dots, m}, \T(v_i) = \begin{dcases}
      \lambda v_i             & \text{if } i = 1                   \\
      \lambda v_i + v_{i - 1} & \text{if } i \in \set{2, \dots, m}
    \end{dcases}.
  \]
  Now define \(\gamma' = \set{v_1', \dots, v_m'}\).
  By definition we have \(v_i' = v_{m + 1 - i}\) for all \(i \in \set{1, \dots, m}\) and
  \begin{align*}
    \forall i \in \set{1, \dots, m}, \T(v_i') & = \T(v_{m + 1 - i})                                                              \\
                                              & = \begin{dcases}
                                                    \lambda v_{m + 1 - i}             & \text{if } m + 1 - i = 1                   \\
                                                    \lambda v_{m + 1 - i} + v_{m - i} & \text{if } m + 1 - i \in \set{2, \dots, m}
                                                  \end{dcases} \\
                                              & = \begin{dcases}
                                                    \lambda v_i'              & \text{if } i = m                       \\
                                                    \lambda v_i' + v_{i + 1}' & \text{if } i \in \set{1, \dots, m - 1}
                                                  \end{dcases}.
  \end{align*}
  Thus we have
  \begin{align*}
    \forall i, j \in \set{1, \dots, m}, \pa{\tp{([\T_{\W}]_{\gamma})}}_{i j} & = ([\T_{\W}]_{\gamma})_{j i}      &  & \by{1.3.3} \\
                                                                             & = \begin{dcases}
                                                                                   \lambda & \text{if } i = j     \\
                                                                                   1       & \text{if } i + 1 = j \\
                                                                                   0       & \text{otherwise}
                                                                                 \end{dcases} &  & \by{7.1.1}                  \\
                                                                             & = ([\T_{\W}]_{\gamma'})_{i j}.    &  & \by{2.2.4}
  \end{align*}
  By \cref{1.2.8} this means \(\tp{([\T_{\W}]_{\beta})} = [\T_{\W}]_{\gamma'}\).
\end{proof}

\begin{proof}[\pf{ex:7.2.7}(b)]
  Let \(\beta\) be an Jordan canonical basis following the convention in \cref{7.2.1}.
  Let \(J = [\T]_{\beta}\).
  Let \(\beta'\) be the ordered set obtained from \(\beta\) by reversing the ordered of each disjoint cycles in \(\beta\).
  By \cref{5.25} and \cref{ex:7.2.7}(a) we see that \([\T]_{\beta'} = \tp{([\T]_{\beta})} = \tp{J}\).
  By \cref{2.23} we hve \([\T]_{\beta'} = \pa{[\IT[\V]]_{\beta'}^{\beta}}^{-1} [\T]_{\beta} [\IT[\V]]_{\beta'}^{\beta}\).
  Thus by \cref{2.5.4} \(J\) and \(\tp{J}\) are similar.
\end{proof}

\begin{proof}[\pf{ex:7.2.7}(c)]
  Let \(\beta\) be the standard ordered basis for \(\vs{F}^n\) over \(\F\) and let \(\alpha\) be a Jordan canonical basis for \(A\).
  By \cref{ex:7.2.7}(b) we see that \([\L_A]_{\alpha}\) and \(\tp{([\L_A]_{\alpha})}\) are similar.
  By \cref{2.23} we know that \(A = [\L_A]_{\beta}\) and \([\L_A]_{\alpha}\) are similar.
  Thus by \cref{ex:2.5.9} we know that \(A\) and \(\tp{([\L_A]_{\alpha})}\) are similar.
  If we can show that \(\tp{A}\) and \(\tp{([\L_A]_{\alpha})}\) are similar, then by \cref{ex:2.5.9} again we see that \(A\) and \(\tp{A}\) are similar.
  This is true since
  \begin{align*}
    \tp{A} & = \tp{([\L_A]_{\beta})}                                                                                       &  & \by{2.15}[a]  \\
           & = \tp{\pa{\pa{[\IT[\V]]_{\beta}^{\alpha}}^{-1} [\L_A]_{\alpha} [\IT[\V]]_{\beta}^{\alpha}}}                   &  & \by{2.23}     \\
           & = \tp{([\IT[\V]]_{\beta}^{\alpha})} \tp{([\L_A]_{\alpha})} \tp{\pa{\pa{[\IT[\V]]_{\beta}^{\alpha}}^{-1}}}     &  & \by{2.3.2}    \\
           & = \tp{\pa{\pa{[\IT[\V]]_{\alpha}^{\beta}}^{-1}}} \tp{([\L_A]_{\alpha})} \tp{\pa{[\IT[\V]]_{\alpha}^{\beta}}}  &  & \by{2.23}     \\
           & = \pa{\tp{\pa{[\IT[\V]]_{\alpha}^{\beta}}}}^{-1} \tp{([\L_A]_{\alpha})} \tp{\pa{[\IT[\V]]_{\alpha}^{\beta}}}. &  & \by{ex:2.4.5}
  \end{align*}
\end{proof}

\begin{ex}\label{ex:7.2.8}

\end{ex}

\begin{ex}\label{ex:7.2.9}

\end{ex}

\begin{ex}\label{ex:7.2.20}
\end{ex}

\begin{ex}\label{ex:7.2.21}
\end{ex}

\section{Homogeneous Linear Differential Equations with Constant Coefficients}\label{sec:2.7}

\begin{defn}\label{2.7.1}
  A \textbf{differential equation} in an unknown function \(y = y(t)\) is an equation involving \(y\), \(t\), and derivatives of \(y\).
  If the differential equation is of the form
  \begin{equation}\label{eq:2.7.1}
    a_n y^{(n)} + a_{n - 1} y^{(n - 1)} + \cdots + a_1 y^{(1)} + a_0 y = f,
  \end{equation}
  where \(\seq{a}{0,,n}\) and \(f\) are functions of \(t\) and \(y^{(k)}\) denotes the \(k\)th derivative of \(y\), then the equation is said to be \textbf{linear}.
  The functions \(a_i\) are called the \textbf{coefficients} of the differential equation \cref{eq:2.7.1}.
  When \(f\) is identically zero, \cref{eq:2.7.1} is called \textbf{homogeneous}.

  If \(a_n \neq 0\), we say that differential equation \cref{eq:2.7.1} is of \textbf{order \(n\)}.
  In this case, we divide both sides by \(a_n\) to obtain a new, but equivalent, equation
  \[
    y^{(n)} + b_{n - 1} y^{(n - 1)} + \cdots + b_1 y^{(1)} + b_0 y = \zv,
  \]
  where \(b_i = a_i / a_n\) for \(i \in \set{0, \dots, n - 1}\).
  Because of this observation, we always assume that the coefficient \(a_n\) in \cref{eq:2.7.1} is \(1\).

  A \textbf{solution} to \cref{eq:2.7.1} is a function that when substituted for \(y\) reduces \cref{eq:2.7.1} to an identity.
\end{defn}

\begin{defn}\label{2.7.2}
  In our study of differential equations, it is useful to regard solutions as complex-valued functions of a real variable even though the solutions that are meaningful to us in a physical sense are real-valued.
  The convenience of this viewpoint will become clear later.
  Thus we are concerned with the vector space \(\fs(\R, \C)\).
  In order to consider complex-valued functions of a real variable as solutions to differential equations, we must define what it means to differentiate such functions.
  Given a complex-valued function \(x \in \fs(\R, \C)\) of a real variable \(t\), there exist unique real-valued functions \(x_1\) and \(x_2\) of \(t\), such that
  \[
    x(t) = x_1(t) + i x_2(t) \quad \text{for} \quad t \in \R,
  \]
  where \(i\) is the imaginary number such that \(i^2 = -1\).
  We call \(x_1\) the \textbf{real part} and \(x_2\) the \textbf{imaginary part} of \(x\).
\end{defn}

\begin{defn}\label{2.7.3}
  Given a function \(x \in \fs(\R, \C)\) with real part \(x_1\) and imaginary part \(x_2\), we say that \(x\) is \textbf{differentiable} if \(x_1\) and \(x_2\) are differentiable.
  If \(x\) is differentiable, we define the derivative \(x'\) of \(x\) by
  \[
    x' = x_1' + i x_2'.
  \]
\end{defn}

\begin{thm}\label{2.27}
  Any solution to a homogeneous linear differential equation with constant coefficients has derivatives of all orders;
  that is, if \(x\) is a solution to such an equation, then \(x^{(k)}\) exists for every positive integer \(k\).
\end{thm}

\begin{proof}[\pf{2.27}]
  Let
  \[
    y^{(n)} + a_{n - 1} y^{(n - 1)} + \cdots + a_1 y^{(1)} + a_0 y = 0
  \]
  be a homogeneous linear differential equation of order \(n\) with constant coefficients.
  Clearly \(y^{(k)}\) exists for all \(k \in \set{0, \dots, n}\).
  Now we prove that \(y^{(k)}\) exists for all \(k \in \N \setminus \set{0, \dots, n}\).
  We can rewrite the equation above as
  \[
    y^{(n)} = -a_{n - 1} y^{(n - 1)} - \cdots - a_1 y^{(1)} - a_0 y.
  \]
  Since \(y^{(n)}\) exists, each term on the right hand side of the above equation can be differentiated at least one more time.
  Thus \(y^{(n + 1)}\) must exist and is equal to the follow:
  \[
    y^{(n + 1)} = (y^{(n)})' = -a_{n - 1} y^{(n)} - \cdots - a_1 y^{(2)} - a_0 y^{(1)}.
  \]
  But again \(y^{(n + 1)}\) exists, therefore each term on the right hand side of the above equation can be differentiated at least one more time.
  Thus \(y^{(n + 2)}\) must exist and is equal to the follow:
  \[
    y^{(n + 2)} = (y^{(n + 1)})' = -a_{n - 1} y^{(n + 1)} - \cdots - a_1 y^{(3)} - a_0 y^{(2)}.
  \]
  In general we see that for all \(k \in \N\), we have
  \[
    y^{(n + k)} = (y^{(n)})^{(k)} = -a_{n - 1} y^{(n - 1 + k)} - \cdots - a_1 y^{(1 + k)} - a_0 y^{(k)}.
  \]
\end{proof}

\begin{defn}\label{2.7.4}
  We use \(\cfs[\infty](\R, \C)\) to denote the set of all functions in \(\fs(\R, \C)\) that have derivatives of all orders.
  By \cref{ex:2.7.5} \(\cfs[\infty](\R, \C)\) is a subspace of \(\fs(\R, \C)\) over \(\C\) and hence is a vector space over \(\C\).
\end{defn}

\begin{defn}\label{2.7.5}
  For \(x \in \cfs[\infty](\R, \C)\), the derivative \(x'\) of \(x\) also lies in \(\cfs[\infty](\R, \C)\).
  We can use the derivative operation to define a mapping \(\Dop : \cfs[\infty](\R, \C) \to \cfs[\infty](\R, \C)\) by
  \[
    \Dop(x) = x' \quad \text{for } x \in \cfs[\infty](\R, \C).
  \]
  It is easy to show that \(\Dop\) is a linear operator.
  More generally, consider any polynomial over \(\C\) of the form
  \[
    p(t) = a_n t^n + a_{n - 1} t^{n - 1} + \cdots + a_1 t + a_0.
  \]
  If we define
  \[
    p(\Dop) = a_n \Dop^n + a_{n - 1} \Dop^{n - 1} + \cdots + a_1 \Dop + a_0 \IT,
  \]
  then \(p(\Dop)\) is a linear operator on \(\cfs[\infty](\R, \C)\).
  (See \cref{e.0.7,ex:2.7.6}.)

  For any polynomial \(p\) over \(\C\) of positive degree, \(p(\Dop)\) is called a \textbf{differential operator}.
  The \textbf{order} of the differential operator \(p(\Dop)\) is the degree of the polynomial \(p\).

  Differential operators are useful since they provide us with a means of reformulating a differential equation in the context of linear algebra.
  Any homogeneous linear differential equation with constant coefficients,
  \[
    y^{(n)} + a_{n - 1} y^{(n - 1)} + \cdots + a_1 y^{(1)} + a_0 y = \zv,
  \]
  can be rewritten using differential operators as
  \[
    \pa{\Dop^{(n)} + a_{n - 1} \Dop^{(n - 1)} + \cdots + a_1 \Dop^{(1)} + a_0 \IT}(y) = \zv.
  \]
  Given the differential equation above, the complex polynomial
  \[
    p(t) = t^n + a_{n - 1} t^{n - 1} + \cdots + a_1 t + a_0
  \]
  is called the \textbf{auxiliary polynomial} associated with the equation.
  Any homogeneous linear differential equation with constant coefficients can be rewritten as
  \[
    p(\Dop)(y) = \zv,
  \]
  where \(p(t)\) is the auxiliary polynomial associated with the equation.
\end{defn}

\begin{thm}\label{2.28}
  The set of all solutions to a homogeneous linear differential equation with constant coefficients coincides with the null space of \(p(\Dop)\), where \(p(t)\) is the auxiliary polynomial associated with the equation.
\end{thm}

\begin{proof}[\pf{2.28}]
  Let
  \[
    y^{(n)} + a_{n - 1} y^{(n - 1)} + \cdots + a_1 y^{(1)} + a_0 y = \zv
  \]
  be a homogeneous linear differential equation where \(\seq{a}{0,,n-1} \in \C\).
  By \cref{2.7.5}
  \[
    p(t) = t^n + a_{n - 1} t^{n - 1} + \cdots + a_1 t + a_0
  \]
  is the auxiliary polynomial associated with the above homogeneous linear differential equation.
  Then we have
  \begin{align*}
         & x \in \cfs[\infty](\R, \C) \text{ is a solution of }                                                \\
         & y^{(n)} + a_{n - 1} y^{(n - 1)} + \cdots + a_1 y^{(1)} + a_0 y = \zv                                \\
    \iff & x^{(n)} + a_{n - 1} x^{(n - 1)} + \cdots + a_1 x^{(1)} + a_0 x = \zv &  & \text{(by \cref{2.7.1})}  \\
    \iff & p(\Dop)(x) = \zv                                                     &  & \text{(by \cref{2.7.5})}  \\
    \iff & x \in \ns{p(\Dop)}                                                   &  & \text{(by \cref{2.1.10})}
  \end{align*}
  and thus \(\set{x \in \cfs[\infty](\R, \C) : x^{(n)} + a_{n - 1} x^{(n - 1)} + \cdots + a_1 x^{(1)} + a_0 x = \zv} = \ns{p(\Dop)}\).
\end{proof}

\begin{cor}\label{2.7.6}
  The set of all solutions to a homogeneous linear differential equation with constant coefficients is a subspace of \(\cfs[\infty](\R, \C)\).
\end{cor}

\begin{proof}[\pf{2.7.6}]
  By \cref{2.1,2.28} we see that this is true.
\end{proof}

\begin{defn}\label{2.7.7}
  In view of \cref{2.7.6}, we call the set of solutions to a homogeneous linear differential equation with constant coefficients the \textbf{solution space} of the equation.
  A practical way of describing such a space is in terms of a basis.
\end{defn}

\exercisesection

\setcounter{ex}{4}
\begin{ex}\label{ex:2.7.5}
  Show that \(\cfs[\infty](\R, \C)\) is a subspace of \(\fs(\R, \C)\).
\end{ex}

\begin{proof}[\pf{ex:2.7.5}]
  Clearly we have \(\cfs[\infty](\R, \C) \subseteq \fs(\R, \C)\).
  Since the zero function \(\zv\) is continously differentiable and \(\zv' = \zv\), we know that \(\zv \in \cfs[\infty](\R, \C)\).
  Thus by \cref{ex:1.3.18} we only need to show that \(cf + g \in \cfs[\infty](\R, \C)\) for any \(f, g \in \cfs[\infty](\R, \C)\) and \(c \in \C\).
  Since
  \[
    \forall k \in \N, (cf + g)^{(k)} = c f^{(k)} + g^{(k)},
  \]
  we see that \(cf + g \in \cfs[\infty](\R, \C)\).
\end{proof}

\begin{ex}\label{ex:2.7.6}
  \begin{enumerate}
    \item Show that \(\Dop : \cfs[\infty](\R, \C) \to \cfs[\infty](\R, \C)\) is a linear operator.
    \item Show that any differential operator is a linear operator on \(\cfs[\infty](\R, \C)\).
  \end{enumerate}
\end{ex}

\begin{proof}[\pf{ex:2.7.6}(a)]
  Let \(f, g \in \cfs[\infty](\R, \C)\) and let \(c \in \C\).
  Since
  \begin{align*}
    \Dop(cf + g) & = (cf + g)'            &  & \text{(by \cref{2.7.5})} \\
                 & = c f' + g'                                          \\
                 & = c \Dop(f) + \Dop(g), &  & \text{(by \cref{2.7.5})}
  \end{align*}
  by \cref{2.1.2}(b) we know that \(\Dop \in \ls(\cfs[\infty](\R, \C))\).
\end{proof}

\begin{proof}[\pf{ex:2.7.6}(b)]
  Let
  \[
    p(\Dop) = a_n \Dop^n + \cdots + a_1 \Dop + a_0 \IT
  \]
  be a differential operator where \(\seq{a}{0,,n} \in \C\).
  Let \(f, g \in \cfs[\infty](\R, \C)\) and let \(c \in \C\).
  Since
  \begin{align*}
     & p(\Dop)(cf + g)                                                                                                                     \\
     & = \pa{a_n \Dop^n + \cdots + a_1 \Dop + a_0 \IT}(cf + g)                                               &  & \text{(by \cref{2.7.5})} \\
     & = a_n \Dop^n(cf + g) + \cdots + a_1 \Dop(cf + g) + a_0 \IT(cf + g)                                    &  & \text{(by \cref{2.2.5})} \\
     & = a_n (cf + g)^{(n)} + \cdots + a_1 (cf + g)^{(1)} + a_0 (cf + g)                                                                   \\
     & = c a_n f^{(n)} + a_n g^{(n)} + \cdots + c a_1 f^{(1)} + a_1 g^{(1)} + c a_0 f + a_0 g                                              \\
     & = c \pa{a_n f^{(n)} + \cdots + a_1 f^{(1)} + a_0 f} + \pa{a_n g^{(n)} + \cdots + a_1 g^{(1)} + a_0 g}                               \\
     & = c p(\Dop)(f) + p(\Dop)(g),                                                                          &  & \text{(by \cref{2.7.5})}
  \end{align*}
  by \cref{2.1.2}(b) we know that \(p(\Dop) \in \ls(\cfs[\infty](\R, \C))\).
\end{proof}

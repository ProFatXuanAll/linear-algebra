\section{Homogeneous Linear Differential Equations with Constant Coefficients}\label{sec:2.7}

\begin{defn}\label{2.7.1}
  A \textbf{differential equation} in an unknown function \(y = y(t)\) is an equation involving \(y\), \(t\), and derivatives of \(y\).
  If the differential equation is of the form
  \begin{equation}\label{eq:2.7.1}
    a_n y^{(n)} + a_{n - 1} y^{(n - 1)} + \cdots + a_1 y^{(1)} + a_0 y = f,
  \end{equation}
  where \(\seq{a}{0,,n}\) and \(f\) are functions of \(t\) and \(y^{(k)}\) denotes the \(k\)th derivative of \(y\), then the equation is said to be \textbf{linear}.
  The functions \(a_i\) are called the \textbf{coefficients} of the differential equation \cref{eq:2.7.1}.
  When \(f\) is identically zero, \cref{eq:2.7.1} is called \textbf{homogeneous}.

  If \(a_n \neq 0\), we say that differential equation \cref{eq:2.7.1} is of \textbf{order \(n\)}.
  In this case, we divide both sides by \(a_n\) to obtain a new, but equivalent, equation
  \[
    y^{(n)} + b_{n - 1} y^{(n - 1)} + \cdots + b_1 y^{(1)} + b_0 y = \zv,
  \]
  where \(b_i = a_i / a_n\) for \(i \in \set{0, \dots, n - 1}\).
  Because of this observation, we always assume that the coefficient \(a_n\) in \cref{eq:2.7.1} is \(1\).

  A \textbf{solution} to \cref{eq:2.7.1} is a function that when substituted for \(y\) reduces \cref{eq:2.7.1} to an identity.
\end{defn}

\begin{defn}\label{2.7.2}
  In our study of differential equations, it is useful to regard solutions as complex-valued functions of a real variable even though the solutions that are meaningful to us in a physical sense are real-valued.
  The convenience of this viewpoint will become clear later.
  Thus we are concerned with the vector space \(\fs{\R}{\C}\).
  In order to consider complex-valued functions of a real variable as solutions to differential equations, we must define what it means to differentiate such functions.
  Given a complex-valued function \(x \in \fs{\R}{\C}\) of a real variable \(t\), there exist unique real-valued functions \(x_1\) and \(x_2\) of \(t\), such that
  \[
    x(t) = x_1(t) + i x_2(t) \quad \text{for} \quad t \in \R,
  \]
  where \(i\) is the imaginary number such that \(i^2 = -1\).
  We call \(x_1\) the \textbf{real part} and \(x_2\) the \textbf{imaginary part} of \(x\).
\end{defn}

\begin{defn}\label{2.7.3}
  Given a function \(x \in \fs{\R}{\C}\) with real part \(x_1\) and imaginary part \(x_2\), we say that \(x\) is \textbf{differentiable} if \(x_1\) and \(x_2\) are differentiable.
  If \(x\) is differentiable, we define the derivative \(x'\) of \(x\) by
  \[
    x' = x_1' + i x_2'.
  \]
\end{defn}

\begin{thm}\label{2.27}
  Any solution to a homogeneous linear differential equation with constant coefficients has derivatives of all orders;
  that is, if \(x\) is a solution to such an equation, then \(x^{(k)}\) exists for every positive integer \(k\).
\end{thm}

\begin{proof}[\pf{2.27}]
  Let
  \[
    y^{(n)} + a_{n - 1} y^{(n - 1)} + \cdots + a_1 y^{(1)} + a_0 y = 0
  \]
  be a homogeneous linear differential equation of order \(n\) with constant coefficients.
  Clearly \(y^{(k)}\) exists for all \(k \in \set{0, \dots, n}\).
  Now we prove that \(y^{(k)}\) exists for all \(k \in \N \setminus \set{0, \dots, n}\).
  We can rewrite the equation above as
  \[
    y^{(n)} = -a_{n - 1} y^{(n - 1)} - \cdots - a_1 y^{(1)} - a_0 y.
  \]
  Since \(y^{(n)}\) exists, each term on the right hand side of the above equation can be differentiated at least one more time.
  Thus \(y^{(n + 1)}\) must exist and is equal to the follow:
  \[
    y^{(n + 1)} = (y^{(n)})' = -a_{n - 1} y^{(n)} - \cdots - a_1 y^{(2)} - a_0 y^{(1)}.
  \]
  But again \(y^{(n + 1)}\) exists, therefore each term on the right hand side of the above equation can be differentiated at least one more time.
  Thus \(y^{(n + 2)}\) must exist and is equal to the follow:
  \[
    y^{(n + 2)} = (y^{(n + 1)})' = -a_{n - 1} y^{(n + 1)} - \cdots - a_1 y^{(3)} - a_0 y^{(2)}.
  \]
  In general we see that for all \(k \in \N\), we have
  \[
    y^{(n + k)} = (y^{(n)})^{(k)} = -a_{n - 1} y^{(n - 1 + k)} - \cdots - a_1 y^{(1 + k)} - a_0 y^{(k)}.
  \]
\end{proof}

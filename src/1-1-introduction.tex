\section{Introduction}\label{sec 1.1}

\begin{note}
    Experiments show that if two like quantities act together, their effect is predictable.
    In this case, the vectors used to represent these quantities can be combined to form a resultant vector that represents the combined effects of the original quantities.
    This resultant vector is called the \emph{sum} of the original vectors, and the rule for their combination is called the \emph{parallelogram law}.
\end{note}

\begin{axiom}[Parallelogram Law for Vector Addition]\label{ac 1.1.1}
    The sum of two vectors $x$ and $y$ that act at the same point $P$ is the vector beginning at $P$ that is represented by the diagonal of parallelogram having $x$ and $y$ as adjacent sides.
\end{axiom}

\begin{note}
    Since a vector beginning at the origin is completely determined by its endpoint, we sometimes refer to \emph{the point $x$} rather than \emph{the endpoint of the vector $x$} if $x$ is a vector emanating from the origin.
\end{note}

\begin{note}
    Besides the operation of vector addition, there is another natural operation that can be performed on vectors
    --- the length of a vector may be magnified or contracted.
    This operation, called \emph{scalar multiplication}, consists of multiplying the vector by a real number.
    If the vector $x$ is represented by an arrow, then for any real number $t$, the vector $tx$ is represented by an arrow in the same direction if $t \geq 0$ and in the opposite direction if $t < 0$.
    The length of the arrow $tx$ is $\abs*{t}$ times the length of the arrow $x$.
    Two nonzero vectors $x$ and $y$ are called \textbf{parallel} if $y = tx$ for some nonzero real number $t$.
    (Thus nonzero vectors having the same or opposite directions are parallel.)
\end{note}
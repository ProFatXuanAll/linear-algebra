\section{The Geometry of Orthogonal Operators}\label{sec:6.11}

\begin{note}
  By \cref{6.22}, any rigid motion on a finite-dimensional real inner product space is the composite of an orthogonal operator and a translation.
  Thus, to understand the geometry of rigid motions thoroughly, we must analyze the structure of orthogonal operators.
  Such is the aim of this section.
  We show that any orthogonal operator on a finite-dimensional real inner product space is the composite of rotations and reflections.
\end{note}

\begin{defn}\label{6.11.1}
  Let \(\T\) be a linear operator on a finite-dimensional real inner product space \(\V\).
  The operator \(\T\) is called a \textbf{rotation} if \(\T\) is the identity on \(\V\) or if there exists a two-dimensional subspace \(\W\) of \(\V\) over \(\R\), an orthonormal basis \(\beta = \set{\seq{x}{1,2}}\) for \(\W\) over \(\R\), and a real number \(\theta\) such that
  \[
    \T(x_1) = \cos(\theta) x_1 + \sin(\theta) x_2, \quad \T(x_2) = -\sin(\theta) x_1 + \cos(\theta) x_2,
  \]
  and \(\T(y) = y\) for all \(y \in \W^{\perp}\).
  In this context, \(\T\) is called a \textbf{rotation of \(\W\) about \(\W^{\perp}\)}.
  The subspace \(\W^{\perp}\) is called the \textbf{axis of rotation}.
\end{defn}

\begin{defn}\label{6.11.2}
  Let \(\T\) be a linear operator on a finite-dimensional real inner product space \(\V\).
  The operator \(\T\) is called a \textbf{reflection} if there exists a one-dimensional subspace \(\W\) of \(\V\) over \(\R\) such that \(\T(x) = -x\) for all \(x \in \W\) and \(\T(y) = y\) for all \(y \in \W^{\perp}\).
  In this context, \(\T\) is called a \textbf{reflection of \(\V\) about \(\W^{\perp}\)}.
\end{defn}

\begin{note}
  It should be noted that rotations and reflections (or composites of these) are orthogonal operators (see \cref{ex:6.11.2}).
  The principal aim of this section is to establish that the converse is also true, that is, any orthogonal operator on a finite-dimensional real inner product space is the composite of rotations and reflections.
\end{note}

\begin{eg}\label{6.11.3}
  A Characterization of Orthogonal Operators on a One-Dimensional Real Inner Product Space.

  Let \(\T\) be an orthogonal operator on a one-dimensional real inner product space \(\V\).
  Choose any nonzero vector \(x\) in \(\V\).
  Then \(\V = \spn{\set{x}}\), and so \(\T(x) = \lambda x\) for some \(\lambda \in \R\).
  Since \(\T\) is orthogonal and \(\lambda\) is an eigenvalue of \(\T\), \(\lambda = \pm 1\) (see \cref{6.2}(a) and \cref{6.5.1}).
  If \(\lambda = 1\), then \(\T\) is the identity on \(\V\), and hence \(\T\) is a rotation (\cref{6.11.1}).
  If \(\lambda = -1\), then \(\T(x) = -x\) for all \(x \in \V\);
  so \(\T\) is a reflection of \(\V\) about \(\V^{\perp} = \set{\zv}\) (\cref{6.2.11,6.11.2}).
  Thus \(\T\) is either a rotation or a reflection.
  Note that in the first case, \(\det(\T) = 1\), and in the second case, \(\det(\T) = -1\) (\cref{ex:5.1.7}).
\end{eg}

\exercisesection

\setcounter{ex}{1}
\begin{ex}\label{ex:6.11.2}
  Prove that rotations, reflections, and composites of rotations and reflections are orthogonal operators.
\end{ex}

\begin{proof}[\pf{ex:6.11.2}]
  Let \(\T\) be a linear operator on a \(n\)-dimensional real inner product space \(\V\).

  First we show that rotations are orthogonal operators.
  Suppose that \(\T\) is a rotation.
  By \cref{6.11.1} there exists a two-dimensional subspace \(\W\) of \(\V\) over \(\R\), an orthonormal basis \(\beta = \set{\seq{x}{1,2}}\) for \(\W\) over \(\R\), and a \(\theta \in \R\) such that
  \begin{align*}
    \T(x_1)                         & = \cos(\theta) x_1 + \sin(\theta) x_2;  \\
    \T(x_2)                         & = -\sin(\theta) x_1 + \cos(\theta) x_2; \\
    \forall y \in \W^{\perp}, \T(y) & = y.
  \end{align*}
  Let \(\gamma = \set{\seq{v}{1,,n-2}}\) be an orthonormal basis for \(\W^{\perp}\) over \(\R\).
  By \cref{6.2.4} we know that \(\beta \cup \gamma\) is an orthonormal basis for \(\V\) over \(\R\).
  Since
  \begin{align*}
    \inn{\T(x_1), \T(x_1)} & = \inn{\cos(\theta) x_1 + \sin(\theta) x_2, \cos(\theta) x_1 + \sin(\theta) x_2}        &  & \by{6.11.1}     \\
                           & = \cos(\theta) \inn{x_1, \cos(\theta) x_1 + \sin(\theta) x_2}                           &  & \by{6.1.1}[a,b] \\
                           & \quad + \sin(\theta) \inn{x_2, \cos(\theta) x_1 + \sin(\theta) x_2}                     &  & \by{6.1.1}[a,b] \\
                           & = \cos(\theta) \inn{x_1, \cos(\theta) x_1} + \sin(\theta) \inn{x_2, \sin(\theta) x_2}   &  & \by{6.1.12}     \\
                           & = (\cos(\theta))^2 \inn{x_1, x_1} + (\sin(\theta))^2 \inn{x_2, x_2}                     &  & \by{6.1.1}[b]   \\
                           & = (\cos(\theta))^2 + (\sin(\theta))^2                                                   &  & \by{6.1.12}     \\
                           & = 1;                                                                                                         \\
    \inn{\T(x_1), \T(x_2)} & = \inn{\cos(\theta) x_1 + \sin(\theta) x_2, -\sin(\theta) x_1 + \cos(\theta) x_2}       &  & \by{6.11.1}     \\
                           & = \cos(\theta) \inn{x_1, -\sin(\theta) x_1 + \cos(\theta) x_2}                          &  & \by{6.1.1}[a,b] \\
                           & \quad + \sin(\theta) \inn{x_2, -\sin(\theta) x_1 + \cos(\theta) x_2}                    &  & \by{6.1.1}[a,b] \\
                           & = \cos(\theta) \inn{x_1, -\sin(\theta) x_1} + \sin(\theta) \inn{x_2, \cos(\theta) x_2}  &  & \by{6.1.12}     \\
                           & = -\sin(\theta) \cos(\theta) \inn{x_1, x_1} + \sin(\theta) \cos(\theta) \inn{x_2, x_2}  &  & \by{6.1}[b]     \\
                           & = -\sin(\theta) \cos(\theta) + \sin(\theta) \cos(\theta)                                &  & \by{6.1.12}     \\
                           & = 0;                                                                                                         \\
    \inn{\T(x_2), \T(x_2)} & = \inn{-\sin(\theta) x_1 + \cos(\theta) x_2, -\sin(\theta) x_1 + \cos(\theta) x_2}      &  & \by{6.11.1}     \\
                           & = -\sin(\theta) \inn{x_1, -\sin(\theta) x_1 + \cos(\theta) x_2}                         &  & \by{6.1.1}[a,b] \\
                           & \quad + \cos(\theta) \inn{x_2, -\sin(\theta) x_1 + \cos(\theta) x_2}                    &  & \by{6.1.1}[a,b] \\
                           & = -\sin(\theta) \inn{x_1, -\sin(\theta) x_1} + \cos(\theta) \inn{x_2, \cos(\theta) x_2} &  & \by{6.1.12}     \\
                           & = (\sin(\theta))^2 \inn{x_1, x_1} + (\cos(\theta))^2 \inn{x_2, x_2}                     &  & \by{6.1.1}[b]   \\
                           & = (\sin(\theta))^2 + (\cos(\theta))^2                                                   &  & \by{6.1.12}     \\
                           & = 1,
  \end{align*}
  we know that \(\T(\beta) \subseteq \W\) is orthonormal.
  Since \(\T(\gamma) = \gamma\), we know that \(\T(\beta) \cup \T(\gamma)\) is an orthonormal basis for \(\V\) over \(\R\).
  Thus by \cref{6.18}(c)(e) we know that \(\T\) is an orthogonal operator.

  Next we show that reflections are orthogonal operators.
  Suppose that \(\T\) is a reflection.
  By \cref{6.11.2} there exists a one-dimensional subspace \(\W\) of \(\V\) over \(\R\) such that
  \begin{align*}
     & \forall x \in \W, \T(x) = -x;        \\
     & \forall x \in \W^{\perp}, \T(x) = x.
  \end{align*}
  Let \(y \in \V\).
  By \cref{6.6} there exists an unique tuple \(\tuple{y}{1,2} \in \W \times \W^{\perp}\) such that \(y = y_1 + y_2\).
  Since
  \begin{align*}
    \norm{\T(y)} & = \norm{\T(y_1 + y_2)}     &  & \by{6.6}       \\
                 & = \norm{\T(y_1) + \T(y_2)} &  & \by{2.1.1}[a]  \\
                 & = \norm{-y_1 + y_2}        &  & \by{6.11.2}    \\
                 & = \norm{-y_1} + \norm{y_2} &  & \by{ex:6.1.10} \\
                 & = \norm{y_1} + \norm{y_2}  &  & \by{6.2}[a]    \\
                 & = \norm{y_1 + y_2}         &  & \by{ex:6.1.10} \\
                 & = \norm{y},                &  & \by{6.6}
  \end{align*}
  by \cref{6.5.1} we know that \(\T\) is an orthogonal operator.

  Finally we show that composites of rotations and reflections are orthogonal operators.
  From the proof above we see that rotations and reflections are orthogonal operators.
  Thus we only need to prove that the composites of orthogonal operators are orthogonal operators.
  This is done by \cref{ex:6.5.3}.
\end{proof}

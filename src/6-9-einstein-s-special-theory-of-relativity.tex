\section{Einstein's Special Theory of Relativity}\label{sec:6.9}

\begin{note}
  As a consequence of physical experiments performed in the latter half of the nineteenth century (most notably the Michelson--Morley experiment of 1887), physicists concluded that \emph{the results obtained in measuring the speed of light are independent of the velocity of the instrument used to measure the speed of light}.

  This revelation led to a new way of relating coordinate systems used to locate events in space--time.
  The result was Albert Einstein's \emph{special theory of relativity}.
  In this section, we develop via a linear algebra viewpoint the essence of Einstein's theory.
\end{note}

\begin{defn}\label{6.9.1}
  The basic problem is to compare two different inertial (nonaccelerating) coordinate systems \(S\) and \(S'\) in three-space (\(\R^3\)) that are in motion relative to each other under the assumption that the speed of light is the same when measured in either system.
  We assume that \(S'\) moves at a constant velocity in relation to \(S\) as measured from \(S\).
  To simplify matters, let us suppose that the following conditions hold:
  \begin{enumerate}
    \item The corresponding axes of \(S\) and \(S'\) (\(x\) and \(x'\), \(y\) and \(y'\), \(z\) and \(z'\)) are parallel, and the origin of \(S'\) moves in the positive direction of the \(x\)-axis of \(S\) at a constant velocity \(v > 0\) relative to \(S\).
    \item Two clocks \(C\) and \(C'\) are placed in space---the first stationary relative to the coordinate system \(S\) and the second stationary relative to the coordinate system \(S'\).
          These clocks are designed to give real numbers in units of seconds as readings.
          The clocks are calibrated so that at the instant the origins of \(S\) and \(S'\) coincide, both clocks give the reading zero.
    \item The unit of length is the \textbf{light second} (the distance light travels in \(1\) second), and the unit of time is the second.
          Note that, with respect to these units, the speed of light is \(1\) light second per second.
  \end{enumerate}

  Given any event (something whose position and time of occurrence can be described), we may assign a set of \emph{space--time coordinates} to it.
  For example, if \(p\) is an event that occurs at position
  \[
    \begin{pmatrix}
      x \\
      y \\
      z
    \end{pmatrix}
  \]
  relative to \(S\) and at time \(t\) as read on clock \(C\), we can assign to \(p\) the set of coordinates
  \[
    \begin{pmatrix}
      x \\
      y \\
      z \\
      t
    \end{pmatrix}.
  \]
  This ordered \(4\)-tuple is called the \textbf{space--time coordinates} of \(p\) relative to \(S\) and \(C\).
  Likewise, \(p\) has a set of space--time coordinates
  \[
    \begin{pmatrix}
      x' \\
      y' \\
      z' \\
      t'
    \end{pmatrix}
  \]
  relative to \(S'\) and \(C'\).
\end{defn}

\begin{ax}[Axioms of the Special Theory of Relativity]\label{6.9.2}
  Einstein made certain assumptions about \(\T_v : \R^4 \to \R^4\) that led to his special theory of relativity.
  We formulate an equivalent set of assumptions.
  \begin{enumerate}[label=(R \arabic*)]
    \item The speed of any light beam, when measured in either coordinate system using a clock stationary relative to that coordinate system, is \(1\).
    \item The mapping \(\T_v : \R^4 \to \R^4\) is an isomorphism.
    \item If
          \[
            \T_v\begin{pmatrix}
              x \\
              y \\
              z \\
              t
            \end{pmatrix} = \begin{pmatrix}
              x' \\
              y' \\
              z' \\
              t'
            \end{pmatrix},
          \]
          then \(y' = y\) and \(z' = z\).
    \item If
          \[
            \T_v\begin{pmatrix}
              x   \\
              y_1 \\
              z_1 \\
              t
            \end{pmatrix} = \begin{pmatrix}
              x' \\
              y' \\
              z' \\
              t'
            \end{pmatrix} \quad \text{and} \quad \T_v\begin{pmatrix}
              x   \\
              y_2 \\
              z_2 \\
              t
            \end{pmatrix} = \begin{pmatrix}
              x'' \\
              y'' \\
              z'' \\
              t''
            \end{pmatrix},
          \]
          then \(x'' = x'\) and \(t'' = t'\).
    \item The origin of \(S\) moves in the negative direction of the \(x'\)-axis of \(S'\) at the constant velocity \(-v < 0\) as measured from \(S'\).
  \end{enumerate}
  Axioms (R 3) and (R 4) tell us that for \(p \in \R^4\), the second and third coordinates of \(\T_v(p)\) are unchanged and the first and fourth coordinates of \(\T_v(p)\) are independent of the second and third coordinates of \(p\).

  As we will see, these five axioms completely characterize \(\T_v\).
  The operator \(\T_v\) is called the \textbf{Lorentz transformation} in direction \(x\).
  We intend to compute \(\T_v\) and use it to study the curious phenomenon of time contraction.
\end{ax}

\begin{thm}\label{6.39}
  On \(\R^4\), the following statements are true.
  \begin{enumerate}
    \item \(\T_v(e_i) = e_i\) for \(i \in \set{2, 3}\).
    \item \(\spn{\set{\seq{e}{2,3}}}\) is \(\T_v\)-invariant.
    \item \(\spn{\set{\seq{e}{1,4}}}\) is \(\T_v\)-invariant.
    \item Both \(\spn{\set{\seq{e}{2,3}}}\) and \(\spn{\set{\seq{e}{1,4}}}\) are \(\T_v^*\)-invariant.
    \item \(\T_v^*(e_i) = e_i\) for \(i \in \set{2, 3}\).
  \end{enumerate}
\end{thm}

\begin{proof}[\pf{6.39}(a)]
  By \cref{6.9.2} (R 2),
  \[
    \T_v\begin{pmatrix}
      0 \\
      0 \\
      0 \\
      0
    \end{pmatrix} = \begin{pmatrix}
      0 \\
      0 \\
      0 \\
      0
    \end{pmatrix},
  \]
  and hence, by \cref{6.9.2} (R 4), the first and fourth coordinates of
  \[
    \T_v\begin{pmatrix}
      0 \\
      a \\
      b \\
      0
    \end{pmatrix}
  \]
  are both zero for any \(a, b \in \R\).
  Thus, by \cref{6.9.2} (R 3),
  \[
    \T_v\begin{pmatrix}
      0 \\
      1 \\
      0 \\
      0
    \end{pmatrix} = \begin{pmatrix}
      0 \\
      1 \\
      0 \\
      0
    \end{pmatrix} \quad \text{and} \quad \T_v\begin{pmatrix}
      0 \\
      0 \\
      1 \\
      0
    \end{pmatrix} = \begin{pmatrix}
      0 \\
      0 \\
      1 \\
      0
    \end{pmatrix}.
  \]
\end{proof}

\begin{proof}[\pf{6.39}(b)]
  Since
  \begin{align*}
    \T_v(\spn{\set{\seq{e}{2,3}}}) & = \spn{\set{\T_v(e_2), \T_v(e_3)}} &  & \by{2.2}     \\
                                   & = \spn{\set{\seq{e}{2,3}}},        &  & \by{6.39}[a]
  \end{align*}
  by \cref{5.4.1} we see that \(\spn{\set{\seq{e}{2,3}}}\) is \(\T\)-invariant.
\end{proof}

\begin{proof}[\pf{6.39}(c)]
  By \cref{6.9.2} (R 2) we have
  \[
    \T_v(e_1) = \begin{pmatrix}
      a \\
      0 \\
      0 \\
      b
    \end{pmatrix} = a e_1 + b e_4 \quad \text{and} \quad \T_v(e_4) = \begin{pmatrix}
      c \\
      0 \\
      0 \\
      d
    \end{pmatrix} = c e_1 + d e_4
  \]
  for some \(a, b, c, d \in \R\).
  Thus by \cref{5.4.1} we see that \(\spn{\set{\seq{e}{1,4}}}\) is \(\T\)-invariant.
\end{proof}

\begin{proof}[\pf{6.39}(d)]
  Since \(\set{\seq{e}{1,2,3,4}}\) is orthonormal with respect to the standard inner product on \(\R^4\) over \(\R\), by \cref{6.7}(b) we see that
  \[
    \spn{\set{\seq{e}{1,4}}}^{\perp} = \spn{\set{\seq{e}{2,3}}} \quad \text{and} \quad \spn{\set{\seq{e}{2,3}}}^{\perp} = \spn{\set{\seq{e}{1,4}}}.
  \]
  Thus by \cref{ex:6.4.7}(b) we know that \(\spn{\set{\seq{e}{1,4}}}\) and \(\spn{\set{\seq{e}{2,3}}}\) are \(\T_v^*\) invariant.
\end{proof}

\begin{proof}[\pf{6.39}(e)]
  For any \(j \in \set{1, 3, 4}\),
  \begin{align*}
    \inn{\T_v^*(e_2), e_j} & = \inn{e_2, \T_v(e_j)} &  & \by{6.9}       \\
                           & = 0;                   &  & \by{6.39}[a,c]
  \end{align*}
  for \(j = 2\),
  \begin{align*}
    \inn{\T_v^*(e_2), e_2} & = \inn{e_2, \T_v(e_2)} &  & \by{6.9}     \\
                           & = \inn{e_2, e_2}       &  & \by{6.39}[a] \\
                           & = 1.                   &  & \by{6.1.2}
  \end{align*}
  We conclude by \cref{6.2.6} that \(\T_v^*(e_2)\) is a multiple of \(e_2\) (i.e., that \(\T_v^*(e_2) = k e_2\) for some \(k \in \R\)).
  Thus,
  \begin{align*}
    1 & = \inn{e_2, e_2}         &  & \by{6.1.2}    \\
      & = \inn{e_2, \T_v(e_2)}   &  & \by{6.39}[a]  \\
      & = \inn{\T_v^*(e_2), e_2} &  & \by{6.9}      \\
      & = \inn{k e_2, e_2}                          \\
      & = k,                     &  & \by{6.1.1}[b]
  \end{align*}
  and hence \(\T_v^*(e_2) = e_2\).
  Similarly, \(\T_v^*(e_3) = e_3\).
\end{proof}

\begin{thm}\label{6.40}
  Suppose that, at the instant the origins of \(S\) and \(S'\) coincide, a light flash is emitted from their common origin.
  The event of the light flash when measured either relative to \(S\) and \(C\) or relative to \(S'\) and \(C'\) has space--time
  coordinates
  \[
    \begin{pmatrix}
      0 \\
      0 \\
      0 \\
      0
    \end{pmatrix}.
  \]
  Let \(P\) be the set of all events whose space--time coordinates
  \[
    \begin{pmatrix}
      x \\
      y \\
      z \\
      t
    \end{pmatrix}
  \]
  relative to \(S\) and \(C\) are such that the flash is observable from the point with coordinates
  \[
    \begin{pmatrix}
      x \\
      y \\
      z
    \end{pmatrix}
  \]
  (as measured relative to \(S\)) at the time \(t\) (as measured on \(C\)).
  Let us characterize \(P\) in terms of \(x, y, z\), and \(t\).
  Since the speed of light is \(1\), at any time \(t \geq 0\) the light flash is observable from any point whose distance to the origin of \(S\) (as measured on \(S\)) is \(t \cdot 1 = t\).
  These are precisely the points that lie on the sphere of radius \(t\) with center at the origin.
  The coordinates (relative to \(S\)) of such points satisfy the equation \(x^2 + y^2 + z^2 - t^2 = 0\).
  Hence an event lies in \(P\) iff its space--time coordinates
  \[
    \begin{pmatrix}
      x \\
      y \\
      z \\
      t
    \end{pmatrix} \quad (t \geq 0)
  \]
  relative to \(S\) and \(C\) satisfy the equation \(x^2 + y^2 + z^2 - t^2 = 0\).
  By virtue of \cref{6.9.2} (R 1), we can characterize \(P\) in terms of the space--time coordinates relative to \(S'\) and \(C'\) similarly:
  An event lies in \(P\) iff, relative to \(S'\) and \(C'\), its space--time coordinates
  \[
    \begin{pmatrix}
      x' \\
      y' \\
      z' \\
      t'
    \end{pmatrix} \quad (t' \geq 0)
  \]
  satisfy the equation \((x')^2 + (y')^2 + (z')^2 - (t')^2 = 0\).
  Let
  \[
    A = \begin{pmatrix}
      1 & 0 & 0 & 0  \\
      0 & 1 & 0 & 0  \\
      0 & 0 & 1 & 0  \\
      0 & 0 & 0 & -1
    \end{pmatrix}.
  \]
  If \(\inn{\L_A(w), w} = 0\) for some \(w \in \R^4\), then
  \[
    \inn{\T_v^* \L_A \T_v(w), w} = 0.
  \]
\end{thm}

\begin{proof}[\pf{6.40}]
  Let
  \[
    w = \begin{pmatrix}
      x \\
      y \\
      z \\
      t
    \end{pmatrix} \in \R^4,
  \]
  and suppose that \(\inn{\L_A(w), w} = 0\).
  \begin{description}
    \item[Case 1.]
      \(t \geq 0\).
      Since \(\inn{\L_A(w), w} = x^2 + y^2 + z^2 - t^2\), the vector \(w\) gives the coordinates of an event in \(P\) relative to \(S\) and \(C\).
      Because
      \[
        \begin{pmatrix}
          x \\
          y \\
          z \\
          t
        \end{pmatrix} \quad \text{and} \quad \begin{pmatrix}
          x' \\
          y' \\
          z' \\
          t'
        \end{pmatrix}
      \]
      are the space--time coordinates of the same event relative to \(S'\) and \(C'\), \cref{6.9.2} (R 1) yields
      \[
        (x')^2 + (y')^2 + (z')^2 - (t')^2 = 0.
      \]
      Thus \(\inn{\T_v^* \L_A \T_v(w), w} = \inn{\L_A \T_v(w), \T_v(w)} = (x')^2 + (y')^2 + (z')^2 - (t')^2 = 0\), and the conclusion follows.
    \item[Case 2.]
      \(t < 0\).
      The proof follows by applying Case 1 to \(-w\), i.e.,
      \begin{align*}
        0 & = \inn{\T_v^* \L_A \T_v(-w), -w} &  & \text{(by Case 1)} \\
          & = \inn{-\T_v^* \L_A \T_v(w), -w} &  & \by{2.1.1}[b]      \\
          & = -\inn{\T_v^* \L_A \T_v(w), -w} &  & \by{6.1.1}[b]      \\
          & = \inn{\T_v^* \L_A \T_v(w), w}.  &  & \by{6.1}[b]
      \end{align*}
  \end{description}
\end{proof}

\begin{thm}\label{6.41}
  We now proceed to deduce information about \(\T_v\).
  Let
  \[
    w_1 = \begin{pmatrix}
      1 \\
      0 \\
      0 \\
      1
    \end{pmatrix} \quad \text{and} \quad w_2 = \begin{pmatrix}
      1 \\
      0 \\
      0 \\
      -1
    \end{pmatrix}.
  \]
  By \cref{ex:6.9.3}, \(\set{\seq{w}{1,2}}\) is an orthogonal basis for \(\spn{\set{\seq{e}{1,4}}}\) over \(\R\), and \(\spn{\set{\seq{e}{1,4}}}\) is \(\T_v^* \L_A \T_v\)-invariant.
  Then there exist nonzero scalars \(a, b \in \R\) such that
  \begin{enumerate}
    \item \(\T_v^* \L_A \T_v(w_1) = a w_2\).
    \item \(\T_v^* \L_A \T_v(w_2) = b w_1\).
  \end{enumerate}
\end{thm}

\begin{proof}[\pf{6.41}]
  Because \(\inn{\L_A(w_1), w_1} = 0\), \(\inn{\T_v^* \L_A \T_v(w_1), w_1} = 0\) by \cref{6.40}.
  Thus \(\T_v^* \L_A \T_v(w_1)\) is orthogonal to \(w_1\).
  Since \(\spn{\set{\seq{e}{1,4}}} = \spn{\set{\seq{w}{1,2}}}\) is \(\T_v^* \L_A \T_v\)-invariant, \(\T_v^* \L_A \T_v(w_1)\) must lie in this set.
  But \(\set{\seq{w}{1,2}}\) is an orthogonal basis for this subspace, and so \(\T_v^* \L_A \T_v(w_1)\) must be a multiple of \(w_2\).
  Thus \(\T_v^* \L_A \T_v (w_1) = a w_2\) for some scalar \(a \in \R\).
  Since \(\T_v\) and \(A\) are invertible, so is \(\T_v^* \L_A \T_v\) (\cref{ex:6.3.8}).
  Thus \(a \neq 0\), proving (a).

  The proof of (b) is similar to (a).
\end{proof}

\begin{cor}\label{6.9.3}
  Let \(B_v = [\T_v]_{\beta}\), where \(\beta\) is the standard ordered basis for \(\R^4\) over \(\R\).
  Then
  \begin{enumerate}
    \item \(B_v^* A B_v = A\).
    \item \(\T_v^* \L_A \T_v = \L_A\).
  \end{enumerate}
\end{cor}

\begin{proof}[\pf{6.9.3}]
  By \cref{6.41} we know that there exist \(a, b \in \R \setminus \set{0}\) such that
  \begin{align*}
    \T_v^* \L_A \T_v(e_1 + e_4) & = \T_v^* \L_A \T_v(e_1) + \T_v^* \L_A \T_v(e_4) &  & \by{2.1.1}[a] \\
                                & = a e_1 - a e_4;                                &  & \by{6.41}     \\
    \T_v^* \L_A \T_v(e_1 - e_4) & = \T_v^* \L_A \T_v(e_1) - \T_v^* \L_A \T_v(e_4) &  & \by{2.1.2}[c] \\
                                & = b e_1 + b e_4.                                &  & \by{6.41}
  \end{align*}
  Thus we have
  \begin{align*}
    \T_v^* \L_A \T_v(e_1) & = \dfrac{a + b}{2} e_1 - \dfrac{a - b}{2} e_4; \\
    \T_v^* \L_A \T_v(e_4) & = \dfrac{a - b}{2} e_1 - \dfrac{a + b}{2} e_4.
  \end{align*}
  Since
  \begin{align*}
    \forall i \in \set{2, 3}, \T_v^* \L_A \T_v(e_i) & = \T_v^* \L_A(e_i) &  & \by{6.39}[a] \\
                                                    & = \T_v^*(e_i)      &  & \by{2.3.8}   \\
                                                    & = e_i,             &  & \by{6.39}[e]
  \end{align*}
  we have
  \begin{align*}
    [\T_v^* \L_A \T_v]_{\beta} & = [\T_v^*]_{\beta} [\L_A]_{\beta} [\T_v]_{\beta} &  & \by{2.3.3}   \\
                               & = [\T_v]_{\beta}^* [\L_A]_{\beta} [\T_v]_{\beta} &  & \by{6.10}    \\
                               & = [\T_v]_{\beta}^* A [\T_v]_{\beta}              &  & \by{2.15}[a] \\
                               & = B_v^* A B_v                                                      \\
                               & = \begin{pmatrix}
                                     \dfrac{a + b}{2}  & 0 & 0 & \dfrac{a - b}{2}  \\
                                     0                 & 1 & 0 & 0                 \\
                                     0                 & 0 & 1 & 0                 \\
                                     -\dfrac{a - b}{2} & 0 & 0 & -\dfrac{a + b}{2}
                                   \end{pmatrix}. &  & \by{2.2.4}
  \end{align*}
  Then we have
  \begin{align*}
             & (B_v^* A B_v)^* = B_v^* A^* B_v = B_v^* A B_v   &  & \by{6.3.2}[c,d] \\
    \implies & \dfrac{a - b}{2} = -\dfrac{a - b}{2}            &  & \by{6.1.5}      \\
    \implies & \dfrac{a - b}{2} = 0                                                 \\
    \implies & B_v^* A B_v = \begin{pmatrix}
                               \dfrac{a + b}{2} & 0 & 0 & 0                 \\
                               0                & 1 & 0 & 0                 \\
                               0                & 0 & 1 & 0                 \\
                               0                & 0 & 0 & -\dfrac{a + b}{2}
                             \end{pmatrix}.
  \end{align*}
  But
  \begin{align*}
    \inn{\L_A(e_2 + e_4), e_2 + e_4} & = \inn{\L_A(e_2) + \L_A(e_4), e_2 + e_4} &  & \by{2.1.1}[a] \\
                                     & = \inn{e_2 - e_4, e_2 + e_4}             &  & \by{6.39}[a]  \\
                                     & = 0                                      &  & \by{6.1.2}
  \end{align*}
  implies
  \begin{align*}
             & \inn{\T_v^* \L_A \T_v(e_2 + e_4), e_2 + e_4} = 0 &  & \by{6.40}       \\
    \implies & \inn{\begin{pmatrix}
                        0 \\
                        1 \\
                        0 \\
                        -\dfrac{a + b}{2}
                      \end{pmatrix}, e_2 + e_4} = 0                    &  & \by{2.3.1} \\
    \implies & 1 - \dfrac{a + b}{2} = 0                         &  & \by{6.1.2}      \\
    \implies & \dfrac{a + b}{2} = 1                                                  \\
    \implies & B_v^* A B_v = \begin{pmatrix}
                               1 & 0 & 0 & 0  \\
                               0 & 1 & 0 & 0  \\
                               0 & 0 & 1 & 0  \\
                               0 & 0 & 0 & -1
                             \end{pmatrix} = A.
  \end{align*}
  Thus by \cref{2.3.3} we have \(\T_v^* \L_A \T_v = \L_A\).
\end{proof}

\begin{thm}\label{6.42}
  Let \(\beta\) be the standard ordered basis for \(\R^4\).
  Then
  \[
    [\T_v]_{\beta} = B_v = \begin{pmatrix}
      \dfrac{1}{\sqrt{1 - v^2}}  & 0 & 0 & \dfrac{-v}{\sqrt{1 - v^2}} \\
      0                          & 1 & 0 & 0                          \\
      0                          & 0 & 1 & 0                          \\
      \dfrac{-v}{\sqrt{1 - v^2}} & 0 & 0 & \dfrac{1}{\sqrt{1 - v^2}}
    \end{pmatrix}.
  \]
\end{thm}

\begin{proof}[\pf{6.42}]
  Consider the situation \(1\) second after the origins of \(S\) and \(S'\) have coincided as measured by the clock \(C\).
  Since the origin of \(S'\) is moving along the \(x\)-axis at a velocity \(v\) as measured in \(S\), its space--time coordinates relative to \(S\) and \(C\) are
  \[
    \begin{pmatrix}
      v \\
      0 \\
      0 \\
      1
    \end{pmatrix}.
  \]
  Similarly, the space--time coordinates for the origin of \(S'\) relative to \(S'\) and \(C'\) must be
  \[
    \begin{pmatrix}
      0 \\
      0 \\
      0 \\
      t'
    \end{pmatrix}
  \]
  for some \(t' > 0\).
  Thus we have
  \[
    \T_v\begin{pmatrix}
      v \\
      0 \\
      0 \\
      1
    \end{pmatrix} = \begin{pmatrix}
      0 \\
      0 \\
      0 \\
      t'
    \end{pmatrix} \quad \text{for some } t' > 0.
  \]
  By \cref{6.9.3},
  \begin{align*}
    \inn{\T_v^* \L_A \T_v\begin{pmatrix}
                             v \\
                             0 \\
                             0 \\
                             1
                           \end{pmatrix}, \begin{pmatrix}
                                            v \\
                                            0 \\
                                            0 \\
                                            1
                                          \end{pmatrix}} & = \inn{\L_A\begin{pmatrix}
                                                                        v \\
                                                                        0 \\
                                                                        0 \\
                                                                        1
                                                                      \end{pmatrix}, \begin{pmatrix}
                                                                                       v \\
                                                                                       0 \\
                                                                                       0 \\
                                                                                       1
                                                                                     \end{pmatrix}} &  & \by{6.9.3}[b] \\
                                         & = \inn{\begin{pmatrix}
                                                      v \\
                                                      0 \\
                                                      0 \\
                                                      -1
                                                    \end{pmatrix}, \begin{pmatrix}
                                                                     v \\
                                                                     0 \\
                                                                     0 \\
                                                                     1
                                                                   \end{pmatrix}} &  & \by{2.3.1}                      \\
                                         & = v^2 - 1.                     &  & \by{6.1.2}
  \end{align*}
  But also
  \begin{align*}
    \inn{\T_v^* \L_A \T_v\begin{pmatrix}
                             v \\
                             0 \\
                             0 \\
                             1
                           \end{pmatrix}, \begin{pmatrix}
                                            v \\
                                            0 \\
                                            0 \\
                                            1
                                          \end{pmatrix}} & = \inn{\L_A \T_v\begin{pmatrix}
                                                                             v \\
                                                                             0 \\
                                                                             0 \\
                                                                             1
                                                                           \end{pmatrix}, \T_v\begin{pmatrix}
                                                                                                v \\
                                                                                                0 \\
                                                                                                0 \\
                                                                                                1
                                                                                              \end{pmatrix}} &  & \by{6.9} \\
                                         & = \inn{\L_A\begin{pmatrix}
                                                          0 \\
                                                          0 \\
                                                          0 \\
                                                          t'
                                                        \end{pmatrix}, \begin{pmatrix}
                                                                         0 \\
                                                                         0 \\
                                                                         0 \\
                                                                         t'
                                                                       \end{pmatrix}}                                      \\
                                         & = \inn{\begin{pmatrix}
                                                      0 \\
                                                      0 \\
                                                      0 \\
                                                      -t'
                                                    \end{pmatrix}, \begin{pmatrix}
                                                                     0 \\
                                                                     0 \\
                                                                     0 \\
                                                                     t'
                                                                   \end{pmatrix}}     &  & \by{2.3.1}                      \\
                                         & = -(t')^2.                         &  & \by{6.1.2}
  \end{align*}
  Combining equations above, we conclude that \(v^2 - 1 = -(t')^2\), or
  \[
    t' = \sqrt{1 - v^2}.
  \]
  Thus we obtain
  \[
    \T_v\begin{pmatrix}
      v \\
      0 \\
      0 \\
      1
    \end{pmatrix} = \begin{pmatrix}
      0 \\
      0 \\
      0 \\
      \sqrt{1 - v^2}
    \end{pmatrix}.
  \]

  Next recall that the origin of \(S\) moves in the negative direction of the \(x'\)-axis of \(S'\) at the constant velocity \(-v < 0\) as measured from \(S'\).
  (This fact is \cref{6.9.2} (R 5))
  Consequently, \(1\) second after the origins of \(S\) and \(S'\) have coincided as measured on clock \(C\), there exists a time \(t'' > 0\) as measured on clock \(C'\) such that
  \[
    \T_v\begin{pmatrix}
      0 \\
      0 \\
      0 \\
      1
    \end{pmatrix} = \begin{pmatrix}
      -v t'' \\
      0      \\
      0      \\
      t''
    \end{pmatrix}.
  \]
  Since
  \begin{align*}
    \inn{\T_v^* \L_A \T_v\begin{pmatrix}
                             0 \\
                             0 \\
                             0 \\
                             1
                           \end{pmatrix}, \begin{pmatrix}
                                            0 \\
                                            0 \\
                                            0 \\
                                            1
                                          \end{pmatrix}} & = \inn{\L_A\begin{pmatrix}
                                                                        0 \\
                                                                        0 \\
                                                                        0 \\
                                                                        1
                                                                      \end{pmatrix}, \begin{pmatrix}
                                                                                       0 \\
                                                                                       0 \\
                                                                                       0 \\
                                                                                       1
                                                                                     \end{pmatrix}} &  & \by{6.9.3}[b] \\
                                         & = \inn{\begin{pmatrix}
                                                      0 \\
                                                      0 \\
                                                      0 \\
                                                      -1
                                                    \end{pmatrix}, \begin{pmatrix}
                                                                     0 \\
                                                                     0 \\
                                                                     0 \\
                                                                     1
                                                                   \end{pmatrix}} &  & \by{2.3.1}                      \\
                                         & = -1                           &  & \by{6.1.2}
  \end{align*}
  and
  \begin{align*}
    \inn{\T_v^* \L_A \T_v\begin{pmatrix}
                             0 \\
                             0 \\
                             0 \\
                             1
                           \end{pmatrix}, \begin{pmatrix}
                                            0 \\
                                            0 \\
                                            0 \\
                                            1
                                          \end{pmatrix}} & = \inn{\L_A \T_v\begin{pmatrix}
                                                                             0 \\
                                                                             0 \\
                                                                             0 \\
                                                                             1
                                                                           \end{pmatrix}, \T_v\begin{pmatrix}
                                                                                                0 \\
                                                                                                0 \\
                                                                                                0 \\
                                                                                                1
                                                                                              \end{pmatrix}} &  & \by{6.9} \\
                                         & = \inn{\L_A\begin{pmatrix}
                                                          -v t'' \\
                                                          0      \\
                                                          0      \\
                                                          t''
                                                        \end{pmatrix}, \begin{pmatrix}
                                                                         -v t'' \\
                                                                         0      \\
                                                                         0      \\
                                                                         t''
                                                                       \end{pmatrix}}     &  & \by{6.9.2}[R 5]             \\
                                         & = \inn{\begin{pmatrix}
                                                      -v t'' \\
                                                      0      \\
                                                      0      \\
                                                      -t''
                                                    \end{pmatrix}, \begin{pmatrix}
                                                                     -v t'' \\
                                                                     0      \\
                                                                     0      \\
                                                                     t''
                                                                   \end{pmatrix}}     &  & \by{2.3.1}                      \\
                                         & =  v^2 (t'')^2 - (t'')^2,          &  & \by{6.1.2}
  \end{align*}
  it follows that
  \begin{align*}
             & -1 = v^2 (t'')^2 - (t'')^2 = (v^2 - 1) (t'')^2 \\
    \implies & \dfrac{1}{1 - v^2} = (t'')^2                   \\
    \implies & t'' = \dfrac{1}{\sqrt{1 - v^2}};
  \end{align*}
  hence
  \[
    \T_v\begin{pmatrix}
      0 \\
      0 \\
      0 \\
      1
    \end{pmatrix} = \begin{pmatrix}
      \dfrac{-v}{\sqrt{1 - v^2}} \\
      0                          \\
      0                          \\
      \dfrac{1}{\sqrt{1 - v^2}}
    \end{pmatrix}.
  \]

  Since
  \begin{align*}
    \T_v(e_1) & = \dfrac{1}{v} \T_v(v e_1)                                &  & \by{2.1.1}[b]    \\
              & = \dfrac{1}{v} \T_v(v e_1 + e_4 - e_4)                                          \\
              & = \dfrac{1}{v} \T_v(v e_1 + e_4) - \dfrac{1}{v} \T_v(e_4) &  & \by{2.1.2}[c]    \\
              & = \dfrac{1}{v} \begin{pmatrix}
                                 0 \\
                                 0 \\
                                 0 \\
                                 \sqrt{1 - v^2}
                               \end{pmatrix} - \dfrac{1}{v} \begin{pmatrix}
                                                              \dfrac{-v}{\sqrt{1 - v^2}} \\
                                                              0                          \\
                                                              0                          \\
                                                              \dfrac{1}{\sqrt{1 - v^2}}
                                                            \end{pmatrix} \\
              & = \begin{pmatrix}
                    \dfrac{1}{\sqrt{1 - v^2}} \\
                    0                         \\
                    0                         \\
                    \dfrac{-v}{\sqrt{1 - v^2}}
                  \end{pmatrix};                                            \\
    \T_v(e_2) & = e_2;                                                    &  & \by{6.39}[a]     \\
    \T_v(e_3) & = e_3;                                                    &  & \by{6.39}[a]     \\
    \T_v(e_4) & = \begin{pmatrix}
                    \dfrac{-v}{\sqrt{1 - v^2}} \\
                    0                          \\
                    0                          \\
                    \dfrac{1}{\sqrt{1 - v^2}}
                  \end{pmatrix},
  \end{align*}
  by \cref{2.2.4} we conclude that
  \[
    [\T_v]_{\beta} = \begin{pmatrix}
      \dfrac{1}{\sqrt{1 - v^2}}  & 0 & 0 & \dfrac{-v}{\sqrt{1 - v^2}} \\
      0                          & 1 & 0 & 0                          \\
      0                          & 0 & 1 & 0                          \\
      \dfrac{-v}{\sqrt{1 - v^2}} & 0 & 0 & \dfrac{1}{\sqrt{1 - v^2}}
    \end{pmatrix}.
  \]
\end{proof}

\begin{cor}\label{6.9.4}
  A most curious and paradoxical conclusion follows if we accept Einstein's theory.
  Suppose that an astronaut leaves our solar system in a space vehicle traveling at a fixed velocity \(v\) as measured relative to our solar system.
  It follows from Einstein's theory that, at the end of time \(t\) as measured on Earth, the time that passes on the space vehicle is only \(t \sqrt{1 - v^2}\).
\end{cor}

\begin{proof}[\pf{6.9.4}]
  To establish this result, consider the coordinate systems \(S\) and \(S'\) and clocks \(C\) and \(C'\) that we have been studying.
  Suppose that the origin of \(S'\) coincides with the space vehicle and the origin of \(S\) coincides with a point in the solar system (stationary relative to the sun) so that the origins of \(S\) and \(S'\) coincide and clocks \(C\) and \(C'\) read zero at the moment the astronaut embarks on the trip.

  As viewed from \(S\), the space--time coordinates of the vehicle at any time \(t > 0\) as measured by \(C\) are
  \[
    \begin{pmatrix}
      vt \\
      0  \\
      0  \\
      t
    \end{pmatrix},
  \]
  whereas, as viewed from \(S'\), the space--time coordinates of the vehicle at any time \(t' > 0\) as measured by \(C'\) are
  \[
    \begin{pmatrix}
      0 \\
      0 \\
      0 \\
      t'
    \end{pmatrix}.
  \]
  But if two sets of space--time coordinates
  \[
    \begin{pmatrix}
      vt \\
      0  \\
      0  \\
      t
    \end{pmatrix} \quad \text{and} \quad \begin{pmatrix}
      0 \\
      0 \\
      0 \\
      t'
    \end{pmatrix}
  \]
  are to describe the same event, it must follow that
  \[
    \T_v\begin{pmatrix}
      vt \\
      0  \\
      0  \\
      t
    \end{pmatrix} = \begin{pmatrix}
      0 \\
      0 \\
      0 \\
      t'
    \end{pmatrix}.
  \]
  Thus
  \begin{align*}
    [\T_v]_{\beta} \begin{pmatrix}
                     vt \\
                     0  \\
                     0  \\
                     t
                   \end{pmatrix} & = \begin{pmatrix}
                                       \dfrac{1}{\sqrt{1 - v^2}}  & 0 & 0 & \dfrac{-v}{\sqrt{1 - v^2}} \\
                                       0                          & 1 & 0 & 0                          \\
                                       0                          & 0 & 1 & 0                          \\
                                       \dfrac{-v}{\sqrt{1 - v^2}} & 0 & 0 & \dfrac{1}{\sqrt{1 - v^2}}
                                     \end{pmatrix} \begin{pmatrix}
                                                     vt \\
                                                     0  \\
                                                     0  \\
                                                     t
                                                   \end{pmatrix} &  & \by{6.42} \\
                                   & = \begin{pmatrix}
                                         0 \\
                                         0 \\
                                         0 \\
                                         t'
                                       \end{pmatrix}.
  \end{align*}
  From the preceding equation, we obtain \(\dfrac{-v^2 t}{\sqrt{1 - v^2}} + \dfrac{t}{\sqrt{1 - v^2}} = t'\), or
  \[
    t' = t \sqrt{1 - v^2}.
  \]
  This is the desired result.
\end{proof}

\begin{note}
  A dramatic consequence of time contraction is that distances are contracted along the line of motion (see \cref{ex:6.9.9}).
\end{note}

\begin{cor}\label{6.9.5}
  Let us make one additional point.
  Suppose that we consider units of distance and time more commonly used than the light second and second, such as the mile and hour, or the kilometer and second.
  Let \(c\) denote the speed of light relative to our chosen units of distance.
  It is easily seen that if an object travels at a velocity \(v\) relative to a set of units, then it is traveling at a velocity \(v / c\) in units of light seconds per second.
  Thus, for an arbitrary set of units of distance and time, we have
  \[
    t' = t \sqrt{1 - \dfrac{v^2}{c^2}}.
  \]
\end{cor}

\begin{proof}[\pf{6.9.5}]
  Let the distance and time units of \(v\) be \(d_v\) and \(s_v\).
  Let the unit of light second and second be \(d_c\) and \(s_c\).
  So the object is traveling at velocity \(v \, d_v\) unit per \(s_v\) time and light is traveling at velocity \(1 \, d_c\) per \(s_c\) time.
  If light is traveling at velocity \(c \, d_v\) unit per \(s_v\) time, then the object must be traveling at velocity \(v / c \, d_c\) per \(s_c\) time since
  \[
    \dfrac{1}{c} = \dfrac{\dfrac{v}{c}}{v}.
  \]
  Thus by \cref{6.9.4} we have
  \[
    t' = t \sqrt{1 - \dfrac{v^2}{c^2}}.
  \]
\end{proof}

\exercisesection

\setcounter{ex}{2}
\begin{ex}\label{ex:6.9.3}
  For
  \[
    w_1 = \begin{pmatrix}
      1 \\
      0 \\
      0 \\
      1
    \end{pmatrix} \quad \text{and} \quad w_2 = \begin{pmatrix}
      1 \\
      0 \\
      0 \\
      -1
    \end{pmatrix},
  \]
  show that
  \begin{enumerate}
    \item \(\set{\seq{w}{1,2}}\) is an orthogonal basis for \(\spn{\set{\seq{e}{1,4}}}\) over \(\R\);
    \item \(\spn{\set{\seq{e}{1,4}}}\) is \(\T_v^* \L_A \T_v\)-invariant, where \(\L_A\) is defined in \cref{6.40}.
  \end{enumerate}
\end{ex}

\begin{proof}[\pf{ex:6.9.3}(a)]
  By \cref{6.1.2} \(\set{\seq{w}{1,2}}\) is orthogonal with respect to the standard inner product on \(\R^4\) over \(\R\).
  Since
  \begin{align*}
    w_1                            & = e_1 + e_4; \\
    w_2                            & = e_1 - e_4; \\
    \dim(\spn{\set{\seq{e}{1,4}}}) & = 2,
  \end{align*}
  by \cref{1.6.15}(b) and \cref{6.2.4} we know that \(\set{\seq{w}{1,2}}\) is an orthogonal basis for \(\spn{\set{\seq{e}{1,4}}}\) over \(\R\).
\end{proof}

\begin{proof}[\pf{ex:6.9.3}(b)]
  Since
  \begin{align*}
    (\T_v^* \L_A \T_v)(\spn{\set{\seq{e}{1,4}}}) & = \T_v^*(\L_A(\T_v(\spn{\set{\seq{e}{1,4}}})))                         \\
                                                 & \subseteq \T_v^*(\L_A(\spn{\set{\seq{e}{1,4}}})) &  & \by{6.39}[c]     \\
                                                 & = \T_v^*(\L_A(\spn{\set{\seq{w}{1,2}}}))         &  & \by{ex:6.9.3}[a] \\
                                                 & = \T_v^*(\spn{\set{\seq{w}{1,2}}})               &  & \by{2.3.1}       \\
                                                 & = \T_v^*(\spn{\set{\seq{e}{1,4}}})               &  & \by{ex:6.9.3}[a] \\
                                                 & \subseteq \spn{\set{\seq{e}{1,4}}},              &  & \by{6.39}[d]
  \end{align*}
  by \cref{5.4.1} we know that \(\spn{\set{\seq{e}{1,4}}}\) is \(\T_v^* \L_A \T_v\)-invariant.
\end{proof}

\setcounter{ex}{5}
\begin{ex}\label{ex:6.9.6}
  Consider three coordinate systems \(S\), \(S'\), and \(S''\) with the corresponding axes (\(x, x', x''\); \(y, y', y''\); and \(z, z', z''\)) parallel and such that the \(x\)-, \(x'\)-, and \(x''\)-axes coincide.
  Suppose that \(S'\) is moving past \(S\) at a velocity \(v_1 > 0\) (as measured on \(S\)), \(S''\) is moving past \(S'\) at a velocity \(v_2 > 0\) (as measured on \(S'\)), and \(S''\) is moving past \(S\) at a velocity \(v_3 > 0\) (as measured on \(S\)), and that there are three clocks \(C\), \(C'\), and \(C''\) such that \(C\) is stationary relative to \(S\), \(C'\) is stationary relative to \(S'\), and \(C''\) is stationary relative to \(S''\).
  Suppose that when measured on any of the three clocks, all the origins of \(S\), \(S'\), and \(S''\) coincide at time \(0\).
  Assuming that \(\T_{v_3} = \T_{v_2} \T_{v_1}\) (i.e., \(B_{v_3} = B_{v_2} B_{v_1}\)), prove that
  \[
    v_3 = \dfrac{v_1 + v_2}{1 + v_1 v_2}.
  \]
  Note that substituting \(v_2 = 1\) in this equation yields \(v_3 = 1\).
  This tells us that the speed of light as measured in \(S\) or \(S'\) is the same.
  Why would we be surprised if this were not the case?
\end{ex}

\begin{proof}[\pf{ex:6.9.6}]
  Let \(\beta\) be the standard ordered basis for \(\R^4\) over \(\R\).
  Then we have
  \begin{align*}
             & \T_{v_3} = \T_{v_2} \T_{v_1}                                                                                                                                           \\
    \implies & B_{v_3} = [\T_{v_3}]_{\beta} = [\T_{v_2} \T_{v_1}]_{\beta}                                                                                             &  & \by{6.9.3} \\
             & = [\T_{v_2}]_{\beta} [\T_{v_1}]_{\beta} = B_{v_2} B_{v_1}                                                                                              &  & \by{2.3.3} \\
    \implies & \begin{pmatrix}
                 \dfrac{1}{\sqrt{1 - v_3^2}}    & 0 & 0 & \dfrac{-v_3}{\sqrt{1 - v_3^2}} \\
                 0                              & 1 & 0 & 0                              \\
                 0                              & 0 & 1 & 0                              \\
                 \dfrac{-v_3}{\sqrt{1 - v_3^2}} & 0 & 0 & \dfrac{1}{\sqrt{1 - v_3^2}}
               \end{pmatrix}                                                               &  & \by{6.42}                                                                             \\
             & = \begin{pmatrix}
                   \dfrac{1}{\sqrt{1 - v_2^2}}    & 0 & 0 & \dfrac{-v_2}{\sqrt{1 - v_2^2}} \\
                   0                              & 1 & 0 & 0                              \\
                   0                              & 0 & 1 & 0                              \\
                   \dfrac{-v_2}{\sqrt{1 - v_2^2}} & 0 & 0 & \dfrac{1}{\sqrt{1 - v_2^2}}
                 \end{pmatrix} \begin{pmatrix}
                                 \dfrac{1}{\sqrt{1 - v_1^2}}    & 0 & 0 & \dfrac{-v_1}{\sqrt{1 - v_1^2}} \\
                                 0                              & 1 & 0 & 0                              \\
                                 0                              & 0 & 1 & 0                              \\
                                 \dfrac{-v_1}{\sqrt{1 - v_1^2}} & 0 & 0 & \dfrac{1}{\sqrt{1 - v_1^2}}
                               \end{pmatrix}                                                               &  & \by{6.42}                                                             \\
             & = \begin{pmatrix}
                   \dfrac{1 + v_1 v_2}{\sqrt{1 - v_2^2} \sqrt{1 - v_1^2}} & 0 & 0 & -\dfrac{v_1 + v_2}{\sqrt{1 - v_2^2} \sqrt{1 - v_1^2}}  \\
                   0                                                      & 1 & 0 & 0                                                      \\
                   0                                                      & 0 & 1 & 0                                                      \\
                   -\dfrac{v_1 + v_2}{\sqrt{1 - v_2^2} \sqrt{1 - v_1^2}}  & 0 & 0 & \dfrac{1 + v_1 v_2}{\sqrt{1 - v_2^2} \sqrt{1 - v_1^2}}
                 \end{pmatrix} &  & \by{2.3.1}                  \\
    \implies & \begin{dcases}
                 \dfrac{1}{\sqrt{1 - v_3^2}} = \dfrac{1 + v_1 v_2}{\sqrt{1 - v_2^2} \sqrt{1 - v_1^2}} \\
                 \dfrac{v_3}{\sqrt{1 - v_3^2}} = \dfrac{v_1 + v_2}{\sqrt{1 - v_2^2} \sqrt{1 - v_1^2}}
               \end{dcases}                                           &  & \by{1.2.8}                                                              \\
    \implies & v_3 = \dfrac{v_3}{1} = \dfrac{v_1 + v_2}{1 + v_1 v_2}.
  \end{align*}
  If \(v_1 = v_2 = 1\) but \(v_3 \neq 1\), then \cref{6.9.2} (R 1) does not hold.
\end{proof}

\begin{ex}\label{ex:6.9.7}
  Compute \((B_v)^{-1}\).
  Show \((B_v)^{-1} = B_{(-v)}\).
  Conclude that if \(S'\) moves at a negative velocity \(v\) relative to \(S\), then \([\T_v]_{\beta} = B_v\), where \(B_v\) is of the form given in \cref{6.42}.
\end{ex}

\begin{proof}[\pf{ex:6.9.7}]
  First observe that
  \begin{align*}
             & \begin{pmatrix}
                 \dfrac{1}{\sqrt{1 - v^2}} & 0 & 0 & \dfrac{v}{\sqrt{1 - v^2}} \\
                 0                         & 1 & 0 & 0                         \\
                 0                         & 0 & 1 & 0                         \\
                 \dfrac{v}{\sqrt{1 - v^2}} & 0 & 0 & \dfrac{1}{\sqrt{1 - v^2}}
               \end{pmatrix} \begin{pmatrix}
                               \dfrac{1}{\sqrt{1 - v^2}}  & 0 & 0 & \dfrac{-v}{\sqrt{1 - v^2}} \\
                               0                          & 1 & 0 & 0                          \\
                               0                          & 0 & 1 & 0                          \\
                               \dfrac{-v}{\sqrt{1 - v^2}} & 0 & 0 & \dfrac{1}{\sqrt{1 - v^2}}
                             \end{pmatrix}            \\
             & = I_4                                                                                      &  & \by{2.3.1} \\
    \implies & (B_v)^{-1} = \begin{pmatrix}
                              \dfrac{1}{\sqrt{1 - v^2}} & 0 & 0 & \dfrac{v}{\sqrt{1 - v^2}} \\
                              0                         & 1 & 0 & 0                         \\
                              0                         & 0 & 1 & 0                         \\
                              \dfrac{v}{\sqrt{1 - v^2}} & 0 & 0 & \dfrac{1}{\sqrt{1 - v^2}}
                            \end{pmatrix}         &  & \by{ex:2.4.10}[b]               \\
             & = \begin{pmatrix}
                   \dfrac{1}{\sqrt{1 - (-v)^2}}     & 0 & 0 & \dfrac{-(-v)}{\sqrt{1 - (-v)^2}} \\
                   0                                & 1 & 0 & 0                                \\
                   0                                & 0 & 1 & 0                                \\
                   \dfrac{-(-v)}{\sqrt{1 - (-v)^2}} & 0 & 0 & \dfrac{1}{\sqrt{1 - (-v)^2}}
                 \end{pmatrix} = B_{(-v)}. &  & \by{6.42}
  \end{align*}
  If \(S'\) moves at a negative velocity \(v\) relative to \(S\), then \(S\) moves at a positive velocity \(-v\) relative to \(S'\).
  Thus
  \begin{align*}
    [\T_v]_{\beta} & = [\T_{-v}^{-1}]_{\beta}                                    \\
                   & = [\T_{-v}]_{\beta}^{-1} &  & \by{2.4.6}                    \\
                   & = B_{(-v)}^{-1}          &  & \by{6.42}                     \\
                   & = B_v.                   &  & \text{(from the proof above)}
  \end{align*}
\end{proof}

\begin{ex}\label{ex:6.9.9}

\end{ex}

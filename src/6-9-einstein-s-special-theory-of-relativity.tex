\section{Einstein's Special Theory of Relativity}\label{sec:6.9}

\begin{note}
  As a consequence of physical experiments performed in the latter half of the nineteenth century (most notably the Michelson--Morley experiment of 1887), physicists concluded that \emph{the results obtained in measuring the speed of light are independent of the velocity of the instrument used to measure the speed of light}.

  This revelation led to a new way of relating coordinate systems used to locate events in space--time.
  The result was Albert Einstein's \emph{special theory of relativity}.
  In this section, we develop via a linear algebra viewpoint the essence of Einstein's theory.
\end{note}

\begin{defn}\label{6.9.1}
  The basic problem is to compare two different inertial (nonaccelerating) coordinate systems \(S\) and \(S'\) in three-space (\(\R^3\)) that are in motion relative to each other under the assumption that the speed of light is the same when measured in either system.
  We assume that \(S'\) moves at a constant velocity in relation to \(S\) as measured from \(S\).
  To simplify matters, let us suppose that the following conditions hold:
  \begin{enumerate}
    \item The corresponding axes of \(S\) and \(S'\) (\(x\) and \(x'\), \(y\) and \(y'\), \(z\) and \(z'\)) are parallel, and the origin of \(S'\) moves in the positive direction of the \(x\)-axis of \(S\) at a constant velocity \(v > 0\) relative to \(S\).
    \item Two clocks \(C\) and \(C'\) are placed in space---the first stationary relative to the coordinate system \(S\) and the second stationary relative to the coordinate system \(S'\).
          These clocks are designed to give real numbers in units of seconds as readings.
          The clocks are calibrated so that at the instant the origins of \(S\) and \(S'\) coincide, both clocks give the reading zero.
    \item The unit of length is the \textbf{light second} (the distance light travels in \(1\) second), and the unit of time is the second.
          Note that, with respect to these units, the speed of light is \(1\) light second per second.
  \end{enumerate}

  Given any event (something whose position and time of occurrence can be described), we may assign a set of \emph{space--time coordinates} to it.
  For example, if \(p\) is an event that occurs at position
  \[
    \begin{pmatrix}
      x \\
      y \\
      z
    \end{pmatrix}
  \]
  relative to \(S\) and at time \(t\) as read on clock \(C\), we can assign to \(p\) the set of coordinates
  \[
    \begin{pmatrix}
      x \\
      y \\
      z \\
      t
    \end{pmatrix}.
  \]
  This ordered \(4\)-tuple is called the \textbf{space--time coordinates} of \(p\) relative to \(S\) and \(C\).
  Likewise, \(p\) has a set of space--time coordinates
  \[
    \begin{pmatrix}
      x' \\
      y' \\
      z' \\
      t'
    \end{pmatrix}
  \]
  relative to \(S'\) and \(C'\).
\end{defn}

\begin{ax}[Axioms of the Special Theory of Relativity]\label{6.9.2}
  Einstein made certain assumptions about \(\T_v : \R^4 \to \R^4\) that led to his special theory of relativity.
  We formulate an equivalent set of assumptions.
  \begin{enumerate}[label=(R \arabic*)]
    \item The speed of any light beam, when measured in either coordinate system using a clock stationary relative to that coordinate system, is \(1\).
    \item The mapping \(\T_v : \R^4 \to \R^4\) is an isomorphism.
    \item If
          \[
            \T_v\begin{pmatrix}
              x \\
              y \\
              z \\
              t
            \end{pmatrix} = \begin{pmatrix}
              x' \\
              y' \\
              z' \\
              t'
            \end{pmatrix},
          \]
          then \(y' = y\) and \(z' = z\).
    \item If
          \[
            \T_v\begin{pmatrix}
              x   \\
              y_1 \\
              z_1 \\
              t
            \end{pmatrix} = \begin{pmatrix}
              x' \\
              y' \\
              z' \\
              t'
            \end{pmatrix} \quad \text{and} \quad \T_v\begin{pmatrix}
              x   \\
              y_2 \\
              z_2 \\
              t
            \end{pmatrix} = \begin{pmatrix}
              x'' \\
              y'' \\
              z'' \\
              t''
            \end{pmatrix},
          \]
          then \(x'' = x'\) and \(t'' = t'\).
    \item The origin of \(S\) moves in the negative direction of the \(x'\)-axis of \(S'\) at the constant velocity \(-v < 0\) as measured from \(S'\).
  \end{enumerate}
  Axioms (R 3) and (R 4) tell us that for \(p \in \R^4\), the second and third coordinates of \(\T_v(p)\) are unchanged and the first and fourth coordinates of \(\T_v(p)\) are independent of the second and third coordinates of \(p\).

  As we will see, these five axioms completely characterize \(\T_v\).
  The operator \(\T_v\) is called the \textbf{Lorentz transformation} in direction \(x\).
  We intend to compute \(\T_v\) and use it to study the curious phenomenon of time contraction.
\end{ax}

\begin{thm}\label{6.39}
  On \(\R^4\), the following statements are true.
  \begin{enumerate}
    \item \(\T_v(e_i) = e_i\) for \(i \in \set{2, 3}\).
    \item \(\spn{\set{\seq{e}{2,3}}}\) is \(\T_v\)-invariant.
    \item \(\spn{\set{\seq{e}{1,4}}}\) is \(\T_v\)-invariant.
    \item Both \(\spn{\set{\seq{e}{2,3}}}\) and \(\spn{\set{\seq{e}{1,4}}}\) are \(\T_v^*\)-invariant.
    \item \(\T_v^*(e_i) = e_i\) for \(i \in \set{2, 3}\).
  \end{enumerate}
\end{thm}

\begin{proof}[\pf{6.39}(a)]
  By \cref{6.9.2} (R 2),
  \[
    \T_v\begin{pmatrix}
      0 \\
      0 \\
      0 \\
      0
    \end{pmatrix} = \begin{pmatrix}
      0 \\
      0 \\
      0 \\
      0
    \end{pmatrix},
  \]
  and hence, by \cref{6.9.2} (R 4), the first and fourth coordinates of
  \[
    \T_v\begin{pmatrix}
      0 \\
      a \\
      b \\
      0
    \end{pmatrix}
  \]
  are both zero for any \(a, b \in \R\).
  Thus, by \cref{6.9.2} (R 3),
  \[
    \T_v\begin{pmatrix}
      0 \\
      1 \\
      0 \\
      0
    \end{pmatrix} = \begin{pmatrix}
      0 \\
      1 \\
      0 \\
      0
    \end{pmatrix} \quad \text{and} \quad \T_v\begin{pmatrix}
      0 \\
      0 \\
      1 \\
      0
    \end{pmatrix} = \begin{pmatrix}
      0 \\
      0 \\
      1 \\
      0
    \end{pmatrix}.
  \]
\end{proof}

\begin{proof}[\pf{6.39}(b)]
  Since
  \begin{align*}
    \T_v(\spn{\set{\seq{e}{2,3}}}) & = \spn{\set{\T_v(e_2), \T_v(e_3)}} &  & \by{2.2}     \\
                                   & = \spn{\set{\seq{e}{2,3}}},        &  & \by{6.39}[a]
  \end{align*}
  by \cref{5.4.1} we see that \(\spn{\set{\seq{e}{2,3}}}\) is \(\T\)-invariant.
\end{proof}

\begin{proof}[\pf{6.39}(c)]
  By \cref{6.9.2} (R 2) we have
  \[
    \T_v(e_1) = \begin{pmatrix}
      a \\
      0 \\
      0 \\
      b
    \end{pmatrix} = a e_1 + b e_4 \quad \text{and} \quad \T_v(e_4) = \begin{pmatrix}
      c \\
      0 \\
      0 \\
      d
    \end{pmatrix} = c e_1 + d e_4
  \]
  for some \(a, b, c, d \in \R\).
  Thus by \cref{5.4.1} we see that \(\spn{\set{\seq{e}{1,4}}}\) is \(\T\)-invariant.
\end{proof}

\begin{proof}[\pf{6.39}(d)]
  Since \(\set{\seq{e}{1,2,3,4}}\) is orthonormal with respect to the standard inner product on \(\R^4\) over \(\R\), by \cref{6.7}(b) we see that
  \[
    \spn{\set{\seq{e}{1,4}}}^{\perp} = \spn{\set{\seq{e}{2,3}}} \quad \text{and} \quad \spn{\set{\seq{e}{2,3}}}^{\perp} = \spn{\set{\seq{e}{1,4}}}.
  \]
  Thus by \cref{ex:6.4.7}(b) we know that \(\spn{\set{\seq{e}{1,4}}}\) and \(\spn{\set{\seq{e}{2,3}}}\) are \(\T_v^*\) invariant.
\end{proof}

\begin{proof}[\pf{6.39}(e)]
  For any \(j \in \set{1, 3, 4}\),
  \begin{align*}
    \inn{\T_v^*(e_2), e_j} & = \inn{e_2, \T_v(e_j)} &  & \by{6.9}       \\
                           & = 0;                   &  & \by{6.39}[a,c]
  \end{align*}
  for \(j = 2\),
  \begin{align*}
    \inn{\T_v^*(e_2), e_2} & = \inn{e_2, \T_v(e_2)} &  & \by{6.9}     \\
                           & = \inn{e_2, e_2}       &  & \by{6.39}[a] \\
                           & = 1.                   &  & \by{6.1.2}
  \end{align*}
  We conclude by \cref{6.2.6} that \(\T_v^*(e_2)\) is a multiple of \(e_2\) (i.e., that \(\T_v^*(e_2) = k e_2\) for some \(k \in \R\)).
  Thus,
  \begin{align*}
    1 & = \inn{e_2, e_2}         &  & \by{6.1.2}    \\
      & = \inn{e_2, \T_v(e_2)}   &  & \by{6.39}[a]  \\
      & = \inn{\T_v^*(e_2), e_2} &  & \by{6.9}      \\
      & = \inn{k e_2, e_2}                          \\
      & = k,                     &  & \by{6.1.1}[b]
  \end{align*}
  and hence \(\T_v^*(e_2) = e_2\).
  Similarly, \(\T_v^*(e_3) = e_3\).
\end{proof}

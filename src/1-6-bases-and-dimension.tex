\section{Bases and Dimension}\label{sec:1.6}

\begin{defn}\label{1.6.1}
	A \textbf{basis} \(\beta\) for a vector space \(\V\) over \(\F\) is a linearly independent subset of \(\V\) that generates \(\V\).
	If \(\beta\) is a basis for \(\V\) over \(\F\), we also say that the vectors of \(\beta\) form a basis for \(\V\) over \(\F\).
\end{defn}

\begin{eg}\label{1.6.2}
	Recalling that \(\spn{\varnothing} = \set{\zv}\) and \(\varnothing\) is linearly independent, we see that \(\varnothing\) is a basis for the zero vector space over \(\F\).
\end{eg}

\begin{eg}\label{1.6.3}
	In \(\vs{F}^n\), let \(e_1 = (1, 0, 0, \dots, 0)\), \(e_2 = (0, 1, 0, \dots, 0)\), \dots, \(e_n = (0, 0, \dots, 0, 1)\);
	\(\set{\seq{e}{1,2,,n}}\) is readily seen to be a basis for \(\vs{F}^n\) over \(\F\) and is called the \textbf{standard basis} for \(\vs{F}^n\) over \(\F\).
\end{eg}

\begin{proof}[\pf{1.6.3}]
	By \cref{ex:1.5.4} we know that \(\set{\seq{e}{1,2,,n}}\) is linearly independent.
	By \cref{ex:1.4.7} we know that \(\vs{F}^n = \spn{\set{\seq{e}{1,2,,n}}}\).
	Thus by \cref{1.6.1} \(\set{\seq{e}{1,2,,n}}\) is a basis for \(\vs{F}^n\) over \(\F\).
\end{proof}

\begin{eg}\label{1.6.4}
	In \(\ms\), let \(E^{i j}\) denote the matrix whose only nonzero entry is a \(1\) in the \(i\)th row and \(j\)th column.
	Then \(\set{E^{i j} : 1 \leq i \leq m, 1 \leq j \leq n}\) is a basis for \(\ms\) over \(\F\).
\end{eg}

\begin{proof}[\pf{1.6.4}]
	By \cref{ex:1.5.6} we know that \(\set{E^{i j} : 1 \leq i \leq m, 1 \leq j \leq n}\) is linearly independent.
	Since
	\begin{align*}
		         & \forall A \in \ms, A = \begin{pmatrix}
			                                  A_{1 1} & A_{1 2} & \cdots & A_{1 n} \\
			                                  A_{2 1} & A_{2 2} & \cdots & A_{2 n} \\
			                                  \vdots  & \vdots  & \ddots & \vdots  \\
			                                  A_{m 1} & A_{m 2} & \cdots & A_{m n}
		                                  \end{pmatrix}                                     \\
		         & = \sum_{i = 1}^m \sum_{j = 1}^n A_{i j} E^{i j}                                 &  & \by{1.2.9} \\
		\implies & \forall A \in \ms, A \in \spn{\set{E^{i j} : 1 \leq i \leq m, 1 \leq j \leq n}} &  & \by{1.4.3} \\
		\implies & \ms = \spn{\set{E^{i j} : 1 \leq i \leq m, 1 \leq j \leq n}},                   &  & \by{1.5}
	\end{align*}
	by \cref{1.6.1} we know that \(\set{E^{i j} : 1 \leq i \leq m, 1 \leq j \leq n}\) is a basis for \(\ms\) over \(\F\).
\end{proof}

\begin{eg}\label{1.6.5}
	In \(\ps[n]{\F}\) the set \(\set{1, x, x^2, \dots, x^n}\) is a basis.
	We call this basis the \textbf{standard basis} for \(\ps[n]{\F}\) over \(\F\).
\end{eg}

\begin{proof}[\pf{1.6.5}]
	By \cref{ex:1.5.5} we know that \(\set{1, x, x^2, \dots, x^n}\) is linearly independent.
	By \cref{ex:1.4.8} we know that \(\ps[n]{\F} = \spn{\set{1, x, x^2, \dots, x^n}}\).
	Thus by \cref{1.6.1} \(\set{1, x, x^2, \dots, x^n}\) is a basis for \(\ps[n]{\F}\) over \(\F\).
\end{proof}

\begin{eg}\label{1.6.6}
	In \(\ps{\F}\) the set \(\set{1, x, x^2, \dots}\) is a basis.
\end{eg}

\begin{proof}[\pf{1.6.6}]
	Suppose for sake of contradiction that \(\set{1, x, x^2, \dots}\) is not a basis for \(\ps{\F}\) over \(\F\).
	Then by \cref{1.6.1} we can split into two cases:
	\begin{itemize}
		\item If \(\set{1, x, x^2, \dots}\) is linearly dependent, then by \cref{ex:1.5.14} we have
		      \[
			      \begin{dcases}
				      \exists x^n \in \set{1, x, x^2, \dots} \\
				      \exists \set{\seq{a}{0,1,2,}} \subseteq \F
			      \end{dcases} : x^n = \sum_{i \in \N : i \neq n} a_i x^i.
		      \]
		      By setting \(a_n = -1\) we have
		      \begin{align*}
			               & \sum_{i \in \N} a_i x^i = \zv = \sum_{i \in \N} 0x^i                  \\
			      \implies & \seq[=]{a}{0,1,2,} = 0.                              &  & \by{1.2.11}
		      \end{align*}
		      But this means \(a_n = 0\), a contradiction.
		\item If \(\ps{\F} \neq \spn{\set{1, x, x^2, \dots}}\), then by \cref{1.4.3} we have
		      \[
			      \exists f \in \ps{\F} : \forall \set{\seq{a}{0,1,2,}} \subseteq \F, f(x) \neq \sum_{i \in \N} a_i x^i.
		      \]
		      Let \(m\) be the degree of \(f\).
		      Then by \cref{1.2.11} we have
		      \[
			      \exists \seq{c}{0,1,,m} \in \F : f(x) = c_0 + c_1 x + \cdots + c_m x^m.
		      \]
		      But by setting
		      \[
			      \begin{dcases}
				      a_i = c_i & \text{if } i \leq m \\
				      a_i = 0   & \text{if } i > m
			      \end{dcases}
		      \]
		      we have \(f(x) = \sum_{i \in \N} a_i x^i\), a contradiction.
	\end{itemize}
	From all cases above we derived contradictions.
	Thus \(\set{1, x, x^2, \dots}\) is a basis for \(\ps{\F}\) over \(\F\).
\end{proof}

\begin{note}
	Observe that \cref{1.6.6} shows that a basis need not be finite.
	In fact, later in \cref{sec:1.6} it is shown that no basis for \(\ps{\F}\) over \(\F\) can be finite.
	Hence not every vector space has a finite basis.
\end{note}

\begin{thm}\label{1.8}
	Let \(\V\) be a vector space over \(\F\) and \(\beta = \set{\seq{u}{1,2,,n}}\) be a subset of \(\V\).
	Then \(\beta\) is a basis for \(\V\) over \(\F\) iff each \(v \in \V\) can be uniquely expressed as a linear combination of vectors of \(\beta\), that is, can be expressed in the form
	\[
		v = \seq[+]{a,u}{1,2,,n}
	\]
	for unique scalars \(\seq{a}{1,2,,n} \in \F\).
\end{thm}

\begin{proof}[\pf{1.8}]
	First suppose that \(\beta\) be a basis for \(\V\) over \(\F\).
	If \(v \in V\), then \(v \in \spn{\beta}\) because \(\spn{\beta} = \V\).
	Thus \(v\) is a linear combination of the vectors of \(\beta\).
	Suppose that
	\[
		v = \seq[+]{a,u}{1,2,,n} \quad \text{and} \quad v = \seq[+]{b,u}{1,2,,n}
	\]
	are two such representations of \(v\).
	Subtracting the second equation from the first gives
	\[
		\zv = (a_1 - b_1) u_1 + (a_2 - b_2) u_2 + \cdots + (a_n - b_n) u_n.
	\]
	Since \(\beta\) is linearly independent, it follows that \(a_1 - b_1 = a_2 - b_2 = \cdots = a_n - b_n = 0\).
	Hence \(a_1 = b_1, a_2 = b_2, \dots, a_n = b_n\), and so \(v\) is uniquely expressible as a linear combination of the vectors of \(\beta\).

	Now suppose that each \(v \in \V\) can be uniquely expressed as a linear combination of vectors of \(\beta\).
	By \cref{1.4.3} and \cref{1.5} this means \(\V = \spn{\beta}\).
	Thus to show that \(\beta\) is a basis for \(\V\) over \(\F\), by \cref{1.6.1} we need to show that \(\beta\) is linearly independent.
	This is true since
	\begin{align*}
		         & \zv \in \V                                             &  & \text{(by \ref{vs3})}  \\
		\implies & \exists! \seq{a}{1,2,,n} \in \F :                                                  \\
		         & \zv = \seq[+]{a,u}{1,2,,n} = \seq[+]{0u}{1,2,,n}       &  & \text{(by hypothesis)} \\
		\implies & \seq[=]{a}{1,2,,n} = 0                                                             \\
		\implies & \set{\seq{u}{1,2,,n}} \text{ is linearly independent}. &  & \by{1.5.3}
	\end{align*}
\end{proof}

\begin{note}
	\cref{1.8} shows that if the vectors \(\seq{u}{1,2,,n}\) form a basis for a vector space \(\V\) over \(\F\), then every vector in \(\V\) can be uniquely expressed in the form
	\[
		v = \seq[+]{a,u}{1,2,,n}
	\]
	for appropriately chosen scalars \(\seq{a}{1,2,,n}\).
	Thus \(v\) determines a unique \(n\)-tuple of scalars \(\tuple{a}{1,2,,n}\) and, conversely, each \(n\)-tuple of scalars determines a unique vector \(v \in \V\) by using the entries of the \(n\)-tuple as the coefficients of a linear combination of \(\seq{u}{1,2,,n}\).
	This fact suggests that \(\V\) is like the vector space \(\vs{F}^n\), where \(n\) is the number of vectors in the basis for \(\V\) over \(\F\).
	We see in \cref{sec:2.4} that this is indeed the case.
\end{note}

\begin{thm}\label{1.9}
	If a vector space \(\V\) over \(\F\) is generated by a finite set \(S\), then some subset of \(S\) is a basis for \(\V\) over \(\F\).
	Hence \(\V\) has a finite basis.
\end{thm}

\begin{proof}[\pf{1.9}]
	If \(S = \varnothing\) or \(S = \set{\zv}\), then \(\V = \set{\zv}\) and \(\varnothing\) is a subset of \(S\) that is a basis for \(\V\) over \(\F\).
	Otherwise \(S\) contains a nonzero vector \(u_1\).
	By \cref{1.5.4}(b), \(\set{u_1}\) is a linearly independent set.
	Continue, if possible, choosing vectors \(\seq{u}{2,,k}\) in \(S\) such that \(\set{\seq{u}{1,2,,k}}\) is linearly independent.
	Since \(S\) is a finite set, we must eventually reach a stage at which \(\beta = \set{\seq{u}{1,2,,k}}\) is a linearly independent subset of \(S\), but adjoining to \(\beta\) any vector in \(S \setminus \beta\) produces a linearly dependent set.
	We claim that \(\beta\) is a basis for \(\V\) over \(\F\).
	Because \(\beta\) is linearly independent by construction, it suffices to show that \(\beta\) spans \(\V\).
	By \cref{1.5} we need to show that \(S \subseteq \spn{\beta}\).
	Let \(v \in S\).
	If \(v \in \beta\), then clearly \(v \in \spn{\beta}\).
	Otherwise, if \(v \notin \beta\), then the preceding construction shows that \(\beta \cup \set{v}\) is linearly dependent.
	So \(v \in \spn{\beta}\) by \cref{1.7}.
	Thus \(S \subseteq \spn{\beta}\).
\end{proof}

\begin{note}
	Because of the method by which the basis \(\beta\) was obtained in the proof of \cref{1.9}, this theorem is often remembered as saying that \emph{a finite spanning set for \(\V\) can be reduced to a basis for \(\V\) over \(\F\).}
\end{note}

\begin{thm}[Replacement Theorem]\label{1.10}
	Let \(\V\) be a vector space over \(\F\) that is generated by a set \(G\) containing exactly \(n\) vectors, and let \(L\) be a linearly independent subset of \(\V\) containing exactly \(m\) vectors.
	Then \(m \leq n\) and there exists a subset \(H\) of \(G\) containing exactly \(n - m\) vectors such that \(L \cup H\) generates \(\V\).
\end{thm}

\begin{proof}[\pf{1.10}]
	The proof is by mathematical induction on \(m\).
	The induction begins with \(m = 0\);
	for in this case \(L = \varnothing\), and so taking \(H = G\) gives the desired result.

	Now suppose that the theorem is true for some integer \(m \geq 0\).
	We prove that the theorem is true for \(m + 1\).
	Let \(L = \set{\seq{v}{1,2,,m + 1}}\) be a linearly independent subset of \(\V\) consisting of \(m + 1\) vectors.
	By \cref{1.5.6} \(\set{\seq{v}{1,2,,m}} \subseteq L\) is linearly independent, and so we may apply the induction hypothesis to conclude that \(m \leq n\) and that there is a subset \(\set{\seq{u}{1,2,,n - m}}\) of \(G\) such that \(\set{\seq{v}{1,2,,m}} \cup \set{\seq{u}{1,2,,n - m}}\) generates \(\V\).
	Thus there exist scalars \(\seq{a}{1,2,,m}, \seq{b}{1,2,,n - m} \in \F\) such that
	\begin{equation}\label{eq:1.6.1}
		\seq[+]{a,v}{1,2,,m} + \seq[+]{b,u}{1,2,,n - m} = v_{m + 1}
	\end{equation}
	Note that \(n - m > 0\), otherwise \(n - m = 0\) implies \(v_{m + 1}\) is a linear combination of \(\seq{v}{1,2,,m}\), which by \cref{1.7} contradicts the assumption that \(L\) is linearly independent.
	Hence \(n > m\);
	that is, \(n \geq m + 1\).
	Moreover, some \(b_i\), say \(b_1\), is nonzero, for otherwise we obtain the same contradiction.
	Solving \cref{eq:1.6.1} for \(u_1\) gives
	\begin{multline*}
		u_1 = (-b_1^{-1} a_1) v_1 + (-b_1^{-1} a_2) v_2 + \cdots + (-b_1^{-1} a_m) v_m + (b_1^{-1}) v_{m + 1} \\
		+ (-b_1^{-1} b_2) u_2 + \cdots + (-b_1^{-1} b_{n - m}) u_{n - m}.
	\end{multline*}
	Let \(H = \set{\seq{u}{2,,n - m}}\).
	Then \(u_1 \in \spn{L \cup H}\), and because \(\seq{v}{1,2,,m}\), \\
	\(\seq{u}{2,,n - m}\) are clearly in \(\spn{L \cup H}\), it follows that
	\[
		\set{\seq{v}{1,2,,m}, \seq{u}{1,2,,n - m}} \subseteq \spn{L \cup H}.
	\]
	Because \(\set{\seq{v}{1,2,,m}, \seq{u}{1,2,n - m}}\) generates \(\V\), \cref{1.5} implies that \\
	\(\spn{L \cup H} = \V\).
	Since \(H\) is a subset of \(G\) that contains \((n - m) - 1 = n - (m + 1)\) vectors, the theorem is true for \(m + 1\).
	This completes the induction.
\end{proof}

\begin{cor}\label{1.6.7}
	Let \(\V\) be a vector space over \(\F\) having a finite basis.
	Then every basis for \(\V\) over \(\F\) contains the same number of vectors.
\end{cor}

\begin{proof}[\pf{1.6.7}]
	Suppose that \(\beta\) is a finite basis for \(\V\) over \(\F\) that contains exactly \(n\) vectors, and let \(\gamma\) be any other basis for \(\V\) over \(\F\).
	If \(\gamma\) contains more than \(n\) vectors, then we can select a subset \(S\) of \(\gamma\) containing exactly \(n + 1\) vectors.
	Since \(S\) is linearly independent and \(\beta\) generates \(\V\), the replacement theorem (\cref{1.10}) implies that \(n + 1 \leq n\), a contradiction.
	Therefore \(\gamma\) is finite, and the number \(m\) of vectors in \(\gamma\) satisfies \(m \leq n\).
	Reversing the roles of \(\beta\) and \(\gamma\) and arguing as above, we obtain \(n \leq m\).
	Hence \(m = n\).
\end{proof}

\begin{note}
	If a vector space has a finite basis, \cref{1.6.7} asserts that the number of vectors in \emph{any} basis for \(\V\) over \(\F\) is an intrinsic property of \(\V\).
\end{note}

\begin{defn}\label{1.6.8}
	A vector space is called \textbf{finite-dimensional} if it has a basis consisting of a finite number of vectors.
	The unique number of vectors in each basis for \(\V\) over \(\F\) is called the \textbf{dimension} of \(\V\) and is denoted by \(\dim(\V)\).
	A vector space that is not finite-dimensional is called \textbf{infinite-dimensional}.
\end{defn}

\begin{eg}\label{1.6.9}
	The vector space \(\set{\zv}\) over \(\F\) has dimension zero.
\end{eg}

\begin{proof}[\pf{1.6.9}]
	By \cref{1.6.2} and \cref{1.6.7} we are done.
\end{proof}

\begin{eg}\label{1.6.10}
	The vector space \(\vs{F}^n\) over \(\F\) has dimension \(n\).
\end{eg}

\begin{proof}[\pf{1.6.10}]
	By \cref{1.6.3} and \cref{1.6.7} we are done.
\end{proof}

\begin{eg}\label{1.6.11}
	The vector space \(\ms\) over \(\F\) has dimension \(mn\).
\end{eg}

\begin{proof}[\pf{1.6.11}]
	By \cref{1.6.4} and \cref{1.6.7} we are done.
\end{proof}

\begin{eg}\label{1.6.12}
	The vector space \(\ps[n]{\F}\) over \(\F\) has dimension \(n + 1\).
\end{eg}

\begin{proof}[\pf{1.6.12}]
	By \cref{1.6.5} and \cref{1.6.7} we are done.
\end{proof}

\begin{eg}\label{1.6.13}
	Over the field of complex numbers, the vector space of complex numbers has dimension \(1\).
	(A basis is \(\set{1}\).)
\end{eg}

\begin{proof}[\pf{1.6.13}]
	We have
	\begin{align*}
		         & \forall c \in \C, c = c \cdot 1                 \\
		\implies & \spn{\set{1}} = \C              &  & \by{1.5}   \\
		\implies & \#\pa{\set{1}} = 1 = \dim(\C).  &  & \by{1.6.8}
	\end{align*}
\end{proof}

\begin{eg}\label{1.6.14}
	Over the field of real numbers, the vector space of complex numbers has dimension \(2\).
	(A basis is \(\set{1, i}\).)
\end{eg}

\begin{proof}[\pf{1.6.14}]
	We have
	\begin{align*}
		         & \forall c \in \C, c = \Re(c) + i \Im(c) = \Re(c) \cdot 1 + \Im(c) \cdot i &  & (\Re(c), \Im(c) \in \R) \\
		\implies & \spn{\set{1, i}} = \C                                                     &  & \by{1.5}                \\
		\implies & \#\pa{\set{1, i}} = 2 = \dim(\C).                                         &  & \by{1.6.8}
	\end{align*}
\end{proof}

\begin{note}
	From \cref{1.6.13} and \cref{1.6.14} we see that the dimension of a vector space depends on its field of scalars.
\end{note}

\begin{note}
	In the terminology of dimension, the first conclusion in the replacement theorem states that if \(\V\) is a finite-dimensional vector space over \(\F\), then no linearly independent subset of \(\V\) can contain more than \(\dim(\V)\) vectors.
	From this fact it follows that the vector space \(\ps{\F}\) over \(\F\) is infinite-dimensional because it has an infinite linearly independent set, namely \(\set{1, x, x^2, \dots}\).
	This set is, in fact, a basis for \(\ps{\F}\) over \(\F\).
	Yet nothing that we have proved in this section guarantees an infinite-dimensional vector space must have a basis.
	In \cref{sec:1.7} it is shown, however, that \emph{every vector space has a basis}.
\end{note}

\begin{cor}\label{1.6.15}
	Let \(\V\) be a vector space over \(\F\) with dimension \(n\).
	\begin{enumerate}
		\item Any finite generating set for \(\V\) contains at least \(n\) vectors, and a generating set for \(\V\) that contains exactly \(n\) vectors is a basis for \(\V\) over \(\F\).
		\item Any linearly independent subset of \(\V\) that contains exactly \(n\) vectors is a basis for \(\V\) over \(\F\).
		\item Every linearly independent subset of \(\V\) can be extended to a basis for \(\V\) over \(\F\).
	\end{enumerate}
\end{cor}

\begin{proof}[\pf{1.6.15}]
	Let \(\beta\) be a basis for \(\V\) over \(\F\).
	\begin{enumerate}
		\item Let \(G\) be a finite generating set for \(\V\).
		      By \cref{1.9} some subset \(H\) of \(G\) is a basis for \(\V\) over \(\F\).
		      \cref{1.6.7} implies that \(H\) contains exactly \(n\) vectors.
		      Since a subset of \(G\) contains \(n\) vectors, \(G\) must contain at least \(n\) vectors.
		      Moreover, if \(G\) contains exactly \(n\) vectors, then we must have \(H = G\), so that \(G\) is a basis for \(\V\) over \(\F\).
		\item Let \(L\) be a linearly independent subset of \(\V\) containing exactly \(n\) vectors.
		      It follows from the replacement theorem that there is a subset \(H\) of \(\beta\) containing \(n - n = 0\) vectors such that \(L \cup H\) generates \(\V\).
		      Thus \(H = \varnothing\), and \(L\) generates \(\V\).
		      Since \(L\) is also linearly independent, \(L\) is a basis for \(\V\) over \(\F\).
		\item If \(L\) is a linearly independent subset of \(\V\) containing \(m\) vectors, then the replacement theorem asserts that there is a subset \(H\) of \(\beta\) containing exactly \(n - m\) vectors such that \(L \cup H\) generates \(\V\).
		      Now \(L \cup H\) contains at most \(n\) vectors;
		      therefore (a) implies that \(L \cup H\) contains exactly \(n\) vectors and that \(L \cup H\) is a basis for \(\V\) over \(\F\).
	\end{enumerate}
\end{proof}

\begin{eg}\label{1.6.16}
	For \(k = 0, 1, \dots, n\), let \(p_k(x) = x^k + x^{k + 1} + \cdots + x^n\).
	It follows from \cref{1.5.5} and \cref{1.6.15}(b) that
	\[
		\set{p_0(x), p_1(x), \dots, p_n(x)}
	\]
	is a basis for \(\ps[n]{\F}\) over \(\F\).
\end{eg}

\begin{thm}\label{1.11}
	Let \(\W\) be a subspace of a finite-dimensional vector space \(\V\) over \(\F\).
	Then \(\W\) is finite-dimensional and \(\dim(\W) \leq \dim(\V)\).
	Moreover, if \(\dim(\W) = \dim(\V)\), then \(\V = \W\).
\end{thm}

\begin{proof}[\pf{1.11}]
	Let \(\dim(\V) = n\).
	If \(\W = \set{\zv}\), then \(\W\) is finite-dimensional and \(\dim(\W) = 0 \leq n\).
	Otherwise, \(\W\) contains a nonzero vector \(x_1\);
	so \(\set{x_1}\) is a linearly independent set.
	Continue choosing vectors, \(\seq{x}{1,2,,k}\) in \(\W\) such that \(\set{\seq{x}{1,2,,k}}\) is linearly independent.
	Since no linearly independent subset of \(\V\) can contain more than \(n\) vectors, this process must stop at a stage where \(k \leq n\) and \(\set{\seq{x}{1,2,,k}}\) is linearly independent but adjoining any other vector from \(\W\) produces a linearly dependent set.
	\cref{1.7} implies that \(\set{\seq{x}{1,2,,k}}\) generates \(\W\), and hence it is a basis for \(\W\) over \(\F\).
	Therefore \(\dim(\W) = k \leq n\).

	If \(\dim(\W) = n\), then a basis for \(\W\) over \(\F\) is a linearly independent subset of \(\V\) containing \(n\) vectors.
	But \cref{1.6.15}(b) implies that this basis for \(\W\) over \(\F\) is also a basis for \(\V\) over \(\F\);
	so \(\W = \V\).
\end{proof}

\begin{eg}\label{1.6.17}
	The set of diagonal \(n \times n\) matrices is a subspace \(\W\) of \(\ms[n][n][\F]\) over \(\F\)
	(See \cref{1.3.8}).
	A basis for \(\W\) over \(\F\) is
	\[
		\set{E^{1 1}, E^{2 2}, \dots, E^{n n}},
	\]
	where \(E^{i j}\) is the matrix in which the only nonzero entry is a \(1\) in the \(i\)th row and \(j\)th column.
	Thus \(\dim(\W) = n\).
\end{eg}

\begin{proof}[\pf{1.6.17}]
	By \cref{ex:1.5.6} we know that \(\set{E^{1 1}, E^{2 2}, \dots, E^{n n}}\) is linearly independent.
	Since \(\W = \spn{\set{E^{1 1}, E^{2 2}, \dots, E^{n n}}}\), by \cref{1.6.15}(a) we know that \(\dim(\W) \leq n\).
	By \cref{1.6.15}(c) we also know that \(\dim(\W) \geq n\).
	Thus we have \(\dim(\W) = n\).
\end{proof}

\begin{eg}\label{1.6.18}
	The set of symmetric \(n \times n\) matrices is a subspace \(\W\) of \(\ms[n][n][\F]\) over \(\F\).
	A basis for \(\W\) over \(\F\) is
	\[
		\set{A^{i j} : 1 \leq i \leq j \leq n}
	\]
	where \(A^{i j}\) is the \(n \times n\) matrix having \(1\) in the \(i\)th row and \(j\)th column, \(1\) in the \(j\)th row and \(i\)th column, and \(0\) elsewhere.
	It follows that
	\[
		\dim(\W) = n + (n - 1) + \cdots + 1 = \frac{1}{2} n(n + 1).
	\]
\end{eg}

\begin{proof}[\pf{1.6.18}]
	By \cref{ex:1.5.6} we see that each \(A^{i j}\) can only express as \(E^{i j} + E^{j i}\).
	Thus \(\set{A^{i j} : 1 \leq i \leq j \leq n}\) is linearly independent and by \cref{1.6.15}(c)we have  \(\dim(\W) \geq \#\pa{\set{A^{i j} : 1 \leq i \leq j \leq n}}\).
	Since
	\begin{align*}
		         & \forall M \in \W, M_{i j} = M_{j i} = M_{i j} \cdot 1                                   \\
		\implies & \forall M \in \W, M = \sum_{i = 1}^n \sum_{j = i}^n M_{i j} A^{i j} &  & \by{1.2.9}     \\
		\implies & \W = \spn{\set{A^{i j} : 1 \leq i \leq j \leq n}}                   &  & \by{1.5}       \\
		\implies & \dim(\W) \leq \#\pa{\set{A^{i j} : 1 \leq i \leq j \leq n}},        &  & \by{1.6.15}[a]
	\end{align*}
	we have \(\dim(\W) = \#\pa{\set{A^{i j} : 1 \leq i \leq j \leq n}} = \frac{1}{2} n(n + 1)\).
\end{proof}

\begin{cor}\label{1.6.19}
	If \(\W\) is a subspace of a finite-dimensional vector space \(\V\) over \(\F\), then any basis for \(\W\) over \(\F\) can be extended to a basis for \(\V\) over \(\F\).
\end{cor}

\begin{proof}[\pf{1.6.19}]
	Let \(S\) be a basis for \(\W\) over \(\F\).
	Because \(S\) is a linearly independent subset of \(\V\), \cref{1.6.15}(c) guarantees that \(S\) can be extended to a basis for \(\V\) over \(\F\).
\end{proof}

\begin{defn}[The Lagrange Interpolation Formula]\label{1.6.20}
	Let \(\seq{c}{0,1,,n}\) be distinct scalars in an infinite field \(\F\).
	The polynomials \(f_0(x), f_1(x), \dots, f_n(x)\) defined by
	\[
		f_i(x) = \frac{(x - c_0) \cdots (x - c_{i - 1}) (x - c_{i + 1}) \cdots (x - c_n)}{(c_i - c_0) \cdots (c_i - c_{i - 1}) (c_i - c_{i + 1}) \cdots (c_i - c_n)} = \prod_{\substack{k = 0 \\ k \neq i}}^n \frac{x - c_k}{c_i - c_k}
	\]
	are called the \textbf{Lagrange polynomials} (associated with \(\seq{c}{0,1,,n}\)).
	Note that each \(f_i(x)\) is a polynomial of degree \(n\) and hence is in \(\ps[n]{\F}\).
	By regarding \(f_i(x)\) as a polynomial function \(f_i : \F \to \F\), we see that
	\begin{equation}\label{eq:1.6.2}
		f_i(c_j) = \begin{dcases}
			0 & \text{if } i \neq j \\
			1 & \text{if } i = j
		\end{dcases}.
	\end{equation}

	This property of Lagrange polynomials can be used to show that \(\beta = \set{\seq{f}{0,1,,n}}\) is a linearly independent subset of \(\ps[n]{\F}\).
	Suppose that
	\[
		\sum_{i = 0}^n a_i f_i = \zv \quad \text{for some scalars } \seq{a}{0,1,,n},
	\]
	where \(\zv\) denotes the zero function.
	Then
	\[
		\sum_{i = 0}^n a_i f_i(c_j) = 0 \quad \text{for } j = 0, 1, \dots, n.
	\]
	But also
	\[
		\sum_{i = 0}^n a_i f_i(c_j) = a_j
	\]
	by \cref{eq:1.6.2}.
	Hence \(a_j = 0\) for \(j = 0, 1, \dots, n\);
	so \(\beta\) is linearly independent.
	Since the dimension of \(\ps[n]{\F}\) is \(n + 1\), it follows from \cref{1.6.15} that \(\beta\) is a basis for \(\ps[n]{\F}\) over \(\F\).

	Because \(\beta\) is a basis for \(\ps[n]{\F}\) over \(\F\), every polynomial function \(g\) in \(\ps[n]{\F}\) is a linear combination of polynomial functions of \(\beta\), say,
	\[
		g = \sum_{i = 0}^n b_i f_i.
	\]
	It follows that
	\[
		g(c_j) = \sum_{i = 0}^n b_i f_i(c_j) = b_j;
	\]
	so
	\[
		g = \sum_{i = 0}^n g(c_i) f_i
	\]
	is the unique representation of \(g\) as a linear combination of elements of \(\beta\).
	This representation is called the \textbf{Lagrange interpolation formula}.
	Notice that the preceding argument shows that if \(\seq{b}{0,1,,n}\) are any \(n + 1\) scalars in \(\F\) (not necessarily distinct), then the polynomial function
	\[
		g = \sum_{i = 0}^n b_i f_i
	\]
	is the unique polynomial in \(\ps[n]{\F}\) such that \(g(c_j) = b_j\).
	Thus we have found the unique polynomial of degree not exceeding \(n\) that has specified values \(b_j\) at given points \(c_j\) in its domain (\(j = 0, 1, \dots, n\)).

	An important consequence of the Lagrange interpolation formula is the following result:
	If \(f \in \ps[n]{\F}\) and \(f(c_i) = 0\) for \(n + 1\) distinct scalars \(\seq{c}{0,1,,n}\) in \(\F\), then \(f\) is the zero function.
\end{defn}

\exercisesection

\setcounter{ex}{10}
\begin{ex}\label{ex:1.6.11}
	Let \(u\) and \(v\) be distinct vectors of a vector space \(\V\) over \(\F\).
	Show that if \(\set{u, v}\) is a basis for \(\V\) over \(\F\) and \(a\) and \(b\) are nonzero scalars, then both \(\set{u + v, au}\) and \(\set{au, bv}\) are also bases for \(\V\) over \(\F\).
\end{ex}

\begin{proof}[\pf{ex:1.6.11}]
	Let \(c_1, c_2 \in \F\).
	Since
	\begin{align*}
		         & c_1 (u + v) + c_2 (au) = \zv                  \\
		\implies & (c_1 + c_2 a) u + c_1 v = \zv &  & \by{1.2.1} \\
		\implies & \begin{dcases}
			           c_1 + c_2 a = 0 \\
			           c_1 = 0
		           \end{dcases}            &  & \by{1.5.3}       \\
		\implies & \begin{dcases}
			           c_2 a = 0 \\
			           c_1 = 0
		           \end{dcases}                                 \\
		\implies & c_1 = c_2 = 0                 &  & (a \neq 0)
	\end{align*}
	and
	\begin{align*}
		         & c_1 (au) + c_2 (bv) = \zv                          \\
		\implies & (c_1 a) u + (c_2 b) v = \zv &  & \by{1.2.1}        \\
		\implies & \begin{dcases}
			           c_1 a = 0 \\
			           c_2 b = 0
		           \end{dcases}              &  & \by{1.5.3}          \\
		\implies & c_1 = c_2 = 0,              &  & (a \neq 0 \neq b)
	\end{align*}
	by \cref{1.5.3} we know that \(\set{u + v, au}\) and \(\set{au, bv}\) are linearly independent.
	Since
	\[
		\#(\set{u + v, au}) = \#(\set{au, bv}) = 2 = \#(\set{u, v}),
	\]
	by \cref{1.6.15}(a) we know that \(\set{u + v, au}\) and \(\set{au, bv}\) are basis for \(\V\) over \(\F\).
\end{proof}

\begin{ex}\label{ex:1.6.12}
	Let \(u, v\), and \(w\) be distinct vectors of a vector space \(\V\) over \(\F\).
	Show that if \(\set{u, v, w}\) is a basis for \(\V\) over \(\F\), then \(\set{u + v + w, v + w, w}\) is also a basis for \(\V\) over \(\F\).
\end{ex}

\begin{proof}[\pf{ex:1.6.12}]
	Let \(a, b, c \in \F\).
	Since
	\begin{align*}
		         & a(u + v + w) + b(v + w) + cw = \zv                 \\
		\implies & au + (a + b)v + (a + b + c)w = \zv &  & \by{1.2.1} \\
		\implies & \begin{dcases}
			           a = 0     \\
			           a + b = 0 \\
			           a + b + c = 0
		           \end{dcases}                     &  & \by{1.5.3}   \\
		\implies & a = b = c = 0,
	\end{align*}
	by \cref{1.5.3} we know that \(\set{u + v + w, v + w, w}\) is linearly independent.
	Since
	\[
		\#(\set{u + v + w, v + w, w}) = 3 = \#(\set{u, v, w}),
	\]
	by \cref{1.6.15}(a) we know that \(\set{u + v + w, v + w, w}\) is a basis for \(\V\) over \(\F\).
\end{proof}

\setcounter{ex}{14}
\begin{ex}\label{ex:1.6.15}
	The set of all \(n \times n\) matrices having trace equal to zero is a subspace \(\W\) of \(\ms[n][n][\F]\) over \(\F\) (See \cref{1.3.9}).
	Find a basis for \(\W\) over \(\F\).
	What is the dimension of \(\W\)?
\end{ex}

\begin{proof}[\pf{ex:1.6.15}]
	Let \(E^{i j} \in \ms[n][n][\F]\) be matrix defined as in \cref{ex:1.5.6} and let \(\beta\) be the set
	\[
		\beta = \set{E^{i j} : i, j \in \set{1, \dots, n}, i \neq j} \cup \set{E^{i i} - E^{1 1} : 2 \leq i \leq n}.
	\]
	Observe that
	\begin{align*}
		         & \forall A \in \W, \begin{dcases}
			                             \tr[A] = 0 \\
			                             A = \sum_{i = 1}^n \sum_{j = 1}^n A_{i j} E^{i j}
		                             \end{dcases}                                                                                                                                                                         &  & \by{ex:1.5.6}                  \\
		\implies & \forall A \in \W, \begin{dcases}
			                             A_{1 1} + A_{2 2} + \cdots + A_{n n} = 0 \\
			                             A = \sum_{i = 1}^n \sum_{j = 1}^n A_{i j} E^{i j}
		                             \end{dcases}                                                                                                                                                                         &  & \by{1.3.9}                     \\
		\implies & \forall A \in \W, \begin{dcases}
			                             A_{2 2} + \cdots + A_{n n} = -A_{1 1}        \\
			                             A = \pa{\sum_{i = 1}^n \sum_{\substack{j = 1 \\ j \neq i}}^n A_{i j} E^{i j}} + \pa{\sum_{i = 1}^n A_{i i} E^{i i}}
		                             \end{dcases} \\
		\implies & \forall A \in \W, A = \pa{\sum_{i = 1}^n \sum_{\substack{j = 1                                                                                                                                                                             \\ j \neq i}}^n A_{i j} E^{i j}} + \pa{\sum_{i = 2}^n A_{i i} (E^{i i} - E^{1 1})} \\
		\implies & \forall A \in \W, A \in \spn{\beta}                                                                                                                                                                                       &  & \by{1.4.3}  \\
		\implies & \W \subseteq \spn{\beta}.
	\end{align*}
	Since
	\[
		\tr[E^{i j}] = 0 \quad \forall i, j \in \set{1, \dots, n} \text{ and } i \neq j
	\]
	and
	\[
		\tr[E^{i i} - E^{1 1}] = 1 + (-1) = 0 \quad \forall i \in \set{2, \dots, n},
	\]
	we know that \(\beta \subseteq \W\).
	Thus by \cref{1.5} we have \(\W = \spn{\beta}\).
	By \cref{ex:1.5.6} we know that \(\beta\) is linearly independent, thus by \cref{1.6.1} \(\beta\) is a basis for \(\W\) over \(\F\), and by \cref{1.6.8} we know that \(\dim(\W) = n^2 - 1\).
\end{proof}

\begin{ex}\label{ex:1.6.16}
	The set of all upper triangular \(n \times n\) matrices is a subspace \(\W\) of \(\ms[n][n][\F]\) over \(\F\) (See \cref{ex:1.3.12}).
	Find a basis for \(\W\) over \(\F\).
	What is the dimension of \(\W\)?
\end{ex}

\begin{proof}[\pf{ex:1.6.16}]
	Let \(E^{i j} \in \ms[n][n][\F]\) be matrix defined as in \cref{ex:1.5.6} and let \(\beta\) be the set
	\[
		\beta = \set{E^{i j} : i, j \in \set{1, \dots, n}, i \leq j}.
	\]
	Clearly \(\beta \subseteq \W\).
	Since
	\begin{align*}
		         & \forall A \in \W, A = \sum_{i = 1}^n \sum_{j = 1}^n A_{i j} E^{i j} = \sum_{i = 1}^n \sum_{j = i}^n A_{i j} E^{i j} &  & \by{ex:1.3.12} \\
		\implies & \forall A \in \W, A \in \spn{\beta}                                                                                 &  & \by{1.4.3}     \\
		\implies & \W = \spn{\beta}                                                                                                    &  & \by{1.5}
	\end{align*}
	and \(\beta\) is linearly independent (by \cref{ex:1.5.6}), by \cref{1.6.1} we know that \(\beta\) is a basis for \(\W\) over \(\F\).
	Thus by \cref{1.6.8} we have \(\dim(\W) = \frac{1}{2} n(n + 1)\).
\end{proof}

\begin{ex}\label{ex:1.6.17}
	The set of all skew-symmetric \(n \times n\) matrices is a subspace \(\W\) of \(\ms[n][n][\F]\) over \(\F\) (See \cref{ex:1.3.28}).
	Find a basis for \(\W\) over \(\F\).
	What is the dimension of \(\W\)?
\end{ex}

\begin{proof}[\pf{ex:1.6.17}]
	Let \(E^{i j} \in \ms[n][n][\F]\) be matrix defined as in \cref{ex:1.5.6} and let \(\beta\) be the set
	\[
		\beta = \set{E^{i j} - E^{j i} : i, j \in \set{1, \dots, n}, i < j}.
	\]
	By \cref{ex:1.3.28} we know that \(\beta \subseteq \W\).
	Since
	\begin{align*}
		         & \forall A \in \W, \tp{A} = -A                                                  &  & \by{ex:1.3.28} \\
		\implies & \forall A \in \W, A_{j i} = -A_{i j} \text{ where } i, j \in \set{1, \dots, n} &  & \by{1.3.3}     \\
		\implies & \forall A \in \W, \begin{dcases}
			                             A_{i j} = -A_{j i} & \text{if } i \neq j \\
			                             A_{i j} = 0        & \text{if } i = j
		                             \end{dcases}                                         \\
		\implies & \forall A \in \W, A = \sum_{i = 1}^n \sum_{j = 1}^n A_{i j} E^{i j}                                \\
		         & = \sum_{i = 1}^n \sum_{j = 1}^{i - 1} (A_{i j} E^{i j} + A_{j i} E^{j i})                          \\
		         & = \sum_{i = 1}^n \sum_{j = 1}^{i - 1} (A_{i j} E^{i j} - A_{i j} E^{j i})                          \\
		         & = \sum_{i = 1}^n \sum_{j = 1}^{i - 1} A_{i j} (E^{i j} - E^{j i})                                  \\
		\implies & \forall A \in \W, A \in \spn{\beta}                                            &  & \by{1.4.3}     \\
		\implies & \W = \spn{\beta}                                                               &  & \by{1.5}
	\end{align*}
	and \(\beta\) is linearly independent (by \cref{ex:1.5.6}), by \cref{1.6.1} we know that \(\beta\) is a basis for \(\W\) over \(\F\).
	Thus by \cref{1.6.8} we have \(\dim(\W) = \frac{1}{2} n(n - 1)\).
\end{proof}

\begin{ex}\label{ex:1.6.18}
	Find a basis for the vector space in \cref{1.2.13}.
	Justify your answer.
\end{ex}

\begin{proof}[\pf{ex:1.6.18}]
	Let \(\V\) be the vector space in \cref{1.2.13} over field \(\F\).
	We claim that the set
	\[
		\beta = \set{\set{e_n^i} : e_n^i = 0 \text{ when } n \neq i ; e_n^i = 1 \text{ when } n = i}
	\]
	is a basis for \(\V\) over \(\F\).
	Clearly \(\beta \subseteq \V\).
	Since
	\begin{align*}
		         & \forall \set{a_n} \in \V, \set{a_n} = \sum_{i \in \N} a_n \set{e_n^i} &  & \by{1.2.13} \\
		\implies & \V = \spn{\beta}                                                      &  & \by{1.5}
	\end{align*}
	and \(\beta\) is linearly independent (for obvious reason), by \cref{1.6.1} we know that \(\beta\) is a basis for \(\V\) over \(\F\).
\end{proof}

\setcounter{ex}{19}
\begin{ex}\label{ex:1.6.20}
	Let \(\V\) be a vector space over \(\F\) having dimension \(n\), and let \(S\) be a subset of \(\V\) that generates \(\V\).
	\begin{enumerate}
		\item Prove that there is a subset of \(S\) that is a basis for \(\V\) over \(\F\).
		      (Be careful not to assume that \(S\) is finite.)
		\item Prove that \(S\) contains at least \(n\) vectors.
	\end{enumerate}
\end{ex}

\begin{proof}[\pf{ex:1.6.20}(a)]
	Note that the hypothesis of \cref{ex:1.6.20}(a) is different from \cref{1.9} that in \cref{1.9} \(S\) is assumed to be finite.
	Suppose for sake of contradiction that such subset does not exist.
	By \cref{1.6.1} this means every subset of \(S\) must either be linearly dependent or cannot generate \(\V\).
	Since \(S\) is a subset of \(S\) and \(\spn{S} = \V\), we know that \(S\) must be linearly dependent.

	Now we let \(\beta_0\) be a finite, linearly independent subset of \(S\).
	From previous paragraph we know that \(\beta_0 \neq S\) and \(\V \neq \spn{\beta_0}\).
	By \cref{1.10} we know that \(\#(\beta_0) \leq n\).
	Then there exists a \(v_1 \in S\) such that \(v_1 \cup \beta_0\) is linearly independent.
	If such \(v_1\) does not exist, then we would have \(S \subseteq \spn{\beta_0}\), which implies \(\spn{\beta_0} = \V\), a contradiction.
	So such \(v_1\) exists and we let \(\beta_1 = \beta_0 \cup v_1\).
	Again we must have \(\beta_1 \neq S\), \(\V \neq \spn{\beta_1}\) and \(\#(\beta_1) \leq n\).
	Using the same argument as above there must exist a \(v_2 \in S\) such that \(v_2 \cup \beta_1\) is linearly independent.
	Continue this definition we can define \(\beta_n = \set{\seq{v}{1,2,,n}}\).
	But by \cref{1.6.15}(b) we know that \(\beta_n\) must be a basis for \(\V\) over \(\F\), a contradiction.
	Thus there must exists a subset of \(S\) which is a basis for \(\V\) over \(\F\).
\end{proof}

\begin{proof}[\pf{ex:1.6.20}(b)]
	From \cref{ex:1.6.20}(a) we know that there exists a subset of \(S\) which is a basis for \(\V\) over \(\F\).
	Thus by \cref{1.6.15}(a) \(S\) must has at least \(n\) vectors.
\end{proof}

\begin{ex}\label{ex:1.6.21}
	Prove that a vector space is infinite-dimensional iff it contains an infinite linearly independent subset.
\end{ex}

\begin{proof}[\pf{ex:1.6.21}]
	Let \(\V\) be a vector spaces over \(\F\).
	Then
	\begin{align*}
		     & \V \text{ is infinite-dimensional}                                      \\
		\iff & \V \text{ has an infinite basis}                        &  & \by{1.6.8} \\
		\iff & \V \text{ has an infinite linearly independent subset}. &  & \by{1.6.1}
	\end{align*}
\end{proof}

\begin{ex}\label{ex:1.6.22}
	Let \(\W_1\) and \(\W_2\) be subspaces of a finite-dimensional vector space \(\V\) over \(\F\).
	Determine necessary and sufficient conditions on \(\W_1\) and \(\W_2\) so that \(\dim(\W_1 \cap \W_2) = \dim(\W_1)\).
\end{ex}

\begin{proof}[\pf{ex:1.6.22}]
	We have
	\begin{align*}
		     & \dim(\W_1 \cap \W_2) = \dim(\W_1)                                                 \\
		\iff & \begin{dcases}
			       \exists \beta_1 \subseteq \W_1 \cap \W_2 \\
			       \exists \beta_2 \subseteq \W_1
		       \end{dcases} : \begin{dcases}
			                      \beta_1 \text{ is a basis for } \W_1 \cap \W_2 \text{ over } \F \\
			                      \beta_2 \text{ is a basis for } \W_1 \text{ over } \F           \\
			                      \#(\beta_1) = \#(\beta_2)
		                      \end{dcases} &  & \by{1.6.8}    \\
		\iff & \W_1 \cap \W_2 = \W_1                                              &  & \by{1.11} \\
		\iff & \W_1 \subseteq \W_2.
	\end{align*}
\end{proof}

\begin{ex}\label{ex:1.6.23}
	Let \(\seq{v}{1,2,,k}, v\) be vectors in a vector space \(\V\) over \(\F\), and define \(W_1 = \spn{\set{\seq{v}{1,2,,k}}}\), and \(\W_2 = \spn{\set{\seq{v}{1,2,,k}, v}}\).
	\begin{enumerate}
		\item Find necessary and sufficient conditions on \(v\) such that \(\dim(\W_1) = \dim(\W_2)\).
		\item State and prove a relationship involving \(\dim(\W_1)\) and \(\dim(\W_2)\) in the case that \(\dim(\W_1) \neq \dim(\W_2)\).
	\end{enumerate}
\end{ex}

\begin{proof}[\pf{ex:1.6.23}(a)]
	We have
	\begin{align*}
		     & \begin{dcases}
			       \W_1 \subseteq \W_2 \\
			       \dim(\W_1) = \dim(\W_2)
		       \end{dcases}                                     &  & \by{ex:1.4.13}        \\
		\iff & \W_1 = \W_2                                                 &  & \by{1.11}  \\
		\iff & v \in \spn{\set{\seq{v}{1,2,,k}}}                           &  & \by{1.4.3} \\
		\iff & v \cup \set{\seq{v}{1,2,,k}} \text{ is linearly dependent}. &  & \by{1.7}
	\end{align*}
\end{proof}

\begin{proof}[\pf{ex:1.6.23}(b)]
	By \cref{1.11} we have
	\[
		\begin{dcases}
			\W_1 \subseteq \W_2 \\
			\W_1 \neq \W_2
		\end{dcases} \implies \dim(\W_1) \neq \dim(\W_2).
	\]
\end{proof}

\begin{ex}\label{ex:1.6.24}
	Let \(f(x)\) be a polynomial of degree n in \(\ps[n]{\R}\).
	Prove that for any \(g(x) \in \ps[n]{\R}\) there exist scalars \(\seq{c}{0,1,,n}\) such that
	\[
		g(x) = c_0 f(x) + c_1 f'(x) + c_2 f''(x) + \cdots + c_n f^{(n)}(x),
	\]
	where \(f^{(n)}(x)\) denotes the \(n\)th derivative of \(f(x)\).
\end{ex}

\begin{proof}[\pf{ex:1.6.24}]
	We denote \(f^{(0)} = f\).
	Since \(f^{(i)}(x)\) has degree \(n - i\) for all \(i = 0, 1, \dots, n\), we know that the set
	\[
		\beta = \set{f^{(i)} : i = 0, 1, \dots, n}
	\]
	is linearly independent.
	Since \(\#(\beta) = n + 1\), by \cref{1.6.15}(b) we know that \(\beta\) is a basis for \(\ps[n]{\R}\) over \(\R\), thus there exist \(\seq{c}{0,1,,n}\) such that \(g = \sum_{i = 0}^n c_i f^{(i)}\).
\end{proof}

\begin{ex}\label{ex:1.6.25}
	If \(\V\) and \(\W\) are vector spaces over \(\F\) of dimensions \(m\) and \(n\), determine the dimension of \(\V \times \W\) (See \cref{ex:1.2.21}).
\end{ex}

\begin{proof}[\pf{ex:1.6.25}]
	Let \(\beta_v = \set{\seq{v}{1,2,,m}}, \beta_w = \set{\seq{w}{1,2,n}}\) be a basis for \(\V, \W\) over \(\F\), respectively.
	Let \(\zv_{\V}, \zv_{\W}\) be the zero vectors of \(\V, \W\), respectively.
	Then we claim the set
	\[
		\beta = \pa{\beta_v \times \set{\zv_{\W}}} \cup \pa{\set{\zv_{\V}} \times \beta_w}
	\]
	is a basis for \(\V \times \W\) over \(\F\).
	Clearly \(\beta \subseteq \V \times \W\).
	Since
	\begin{align*}
		         & \forall \seq{a}{1,2,,m + n} \in \F, \sum_{i = 1}^m a_i (v_i, \zv_{\W}) + \sum_{i = 1}^n a_{m + i} (\zv_{\V}, w_i)                     \\
		         & = (\sum_{i = 1}^m a_i v_i, \sum_{i = 1}^n a_{m + i} w_i) = (\zv_{\V}, \zv_{\W})                                   &  & \by{ex:1.2.21} \\
		\implies & \begin{dcases}
			           \sum_{i = 1}^m a_i v_i = \zv_{\V} \\
			           \sum_{i = 1}^n a_{m + i} w_i = \zv_{\W}
		           \end{dcases}                                                                                                \\
		\implies & \seq[=]{a}{1,2,,m + n} = 0,
	\end{align*}
	by \cref{1.5.3} we know that \(\beta\) is linearly independent.
	Since
	\begin{align*}
		         & \forall (v, w) \in \V \times \W, \exists \seq{a}{1,2,,m + n} \in \F :                               \\
		         & (v, w) = (\sum_{i = 1}^m a_i v_i, \sum_{i = 1}^n a_{m + i} w_i)                                     \\
		         & = \sum_{i = 1}^m a_i (v_i, \zv_{\W}) + \sum_{i = 1}^n a_{m + i} (\zv_{\V}, w_i) &  & \by{ex:1.2.21} \\
		\implies & \forall (v, w) \in \V \times \W, (v, w) \in \spn{\beta}                                             \\
		\implies & \V \times \W = \spn{\beta},                                                     &  & \by{1.5}
	\end{align*}
	by \cref{1.6.1} we know that \(\beta\) is a basis for \(\V \times \W\) over \(\F\) and \(\dim(\V \times \W) = m + n\).
\end{proof}

\begin{ex}\label{ex:1.6.26}
	For a fixed \(a \in \R\), determine the dimension of the subspace of \(\ps[n]{\R}\) defined by \(\set{f \in \ps[n]{\R} : f(a) = 0}\).
\end{ex}

\begin{proof}[\pf{ex:1.6.26}]
	Since the set
	\[
		\beta = \set{x - a, x^2 - a^2, \dots, x^n - a^n} \subseteq \set{f \in \ps[n]{\R} : f(a) = 0} \subseteq \ps[n]{\R}
	\]
	is linearly independent (by \cref{ex:1.5.5}) and
	\begin{align*}
		         & \forall g \in \set{f \in \ps[n]{\R} : f(a) = 0}, \exists \seq{c}{1,2,,n} \in \R :                 \\
		         & \forall x \in \R, g(x) = c_1 (x - a) + c_2 (x^2 - a^2) + \cdots + c_n (x^n - a^n)                 \\
		\implies & \forall g \in \set{f \in \ps[n]{\R} : f(a) = 0}, g \in \spn{\beta}                &  & \by{1.4.3} \\
		\implies & \set{f \in \ps[n]{\R} : f(a) = 0} = \spn{\beta},                                  &  & \by{1.5}
	\end{align*}
	by \cref{1.6.1} we know that \(\beta\) is a basis for \(\set{f \in \ps[n]{\R} : f(a) = 0}\) over \(\R\).
	Thus by \cref{1.6.8} we have \(\dim(\set{f \in \ps[n]{\R} : f(a) = 0}) = \#(\beta) = n\).
\end{proof}

\begin{ex}\label{ex:1.6.27}
	Let \(\W_1\) and \(\W_2\) be the subspaces of \(\ps{\F}\) over \(\F\) defined in \cref{ex:1.3.25}.
	Determine the dimensions of the subspaces \(\W_1 \cap \ps[n]{\F}\) and \(\W_2 \cap \ps[n]{\F}\).
\end{ex}

\begin{proof}[\pf{ex:1.6.27}]
	First suppose that \(n\) is odd.
	We define
	\begin{align*}
		\beta_1 & = \set{1, x^2, x^4 \dots, x^{n - 1}} \\
		\beta_2 & = \set{x, x^3, x^5 \dots, x^n}.
	\end{align*}
	Then we have
	\begin{align*}
		         & \begin{dcases}
			           \beta_1 \text{ is linearly independent} \\
			           \beta_2 \text{ is linearly independent} \\
			           \W_1 = \spn{\beta_1}                    \\
			           \W_2 = \spn{\beta_2}
		           \end{dcases}               &  & \by{ex:1.3.25}        \\
		\implies & \begin{dcases}
			           \beta_1 \text{ is a basis for } \W_1 \text{ over } \F \\
			           \beta_2 \text{ is a basis for } \W_2 \text{ over } \F
		           \end{dcases} &  & \by{1.6.1} \\
		\implies & \begin{dcases}
			           \dim(\W_1) = \#(\beta_1) = \frac{n - 1}{2} \\
			           \dim(\W_2) = \#(\beta_2) = \frac{n - 1}{2}
		           \end{dcases}.            &  & \by{1.6.8}
	\end{align*}
	Using similar arguments as above we can show that when \(n\) is even we have
	\[
		\begin{dcases}
			\dim(\W_1) = \#(\beta_1) = \frac{n}{2} + 1 \\
			\dim(\W_2) = \#(\beta_2) = \frac{n}{2}
		\end{dcases}.
	\]
\end{proof}

\begin{ex}\label{ex:1.6.28}
	Let \(\V\) be a finite-dimensional vector space over \(\C\) with dimension \(n\).
	Prove that if \(\V\) is now regarded as a vector space over \(\R\), then \(\dim(\V) = 2n\).
\end{ex}

\begin{proof}[\pf{ex:1.6.28}]
	Let \(\beta = \set{\seq{v}{1,2,,n}}\) be a basis for \(\V\) over \(\C\).
	Since
	\begin{align*}
		         & \spn{\beta} = \V                                                           &  & \by{1.6.1}                  \\
		\implies & \forall v \in \V, \exists \seq{a}{1,2,,n} \in \C :                                                          \\
		         & v = \sum_{j = 1}^n a_j v_j = \sum_{j = 1}^n \pa{\Re(a_j) + i \Im(a_j)} v_j                                  \\
		         & = \sum_{j = 1}^n \Re(a_j) (1 v_j) + \sum_{j = 1}^n \Im(a_j) (i v_j)        &  & \by{1.4.3}                  \\
		\implies & \forall v \in \V, \exists \seq{b}{1,2,,2n} \in \R :                                                         \\
		         & v = \sum_{j = 1}^n b_j v_j + \sum_{j = 1}^n b_{n + j} (i v_j)              &  & (\Re(a_j), \Im(a_j) \in \R) \\
		\implies & \spn{\set{\seq{v}{1,2,,n}, \seq{iv}{1,2,,n}}} = \V                         &  & \by{1.4.3}
	\end{align*}
	and
	\begin{align*}
		         & \forall \seq{b}{1,2,,2n} \in \R, \sum_{j = 1}^n b_j v_j + \sum_{j = 1}^n b_{n + j} (i v_j) = \zv                                            \\
		\implies & \sum_{j = 1}^n (b_j + i b_{n + j}) v_j = \zv                                                                                                \\
		\implies & b_1 + i b_{n + 1} = b_2 + i b_{n + 2} = \cdots = b_n + i b_{2n} = 0                              &  & \by{1.5.3}                            \\
		\implies & \seq[=]{b}{1,2,,2n} = 0,                                                                         &  & (\set{\seq{b}{1,2,,2n}} \subseteq \R)
	\end{align*}
	by \cref{1.6.1} we know that \(\set{\seq{v}{1,2,,n}, \seq{iv}{1,2,,n}}\) is a basis for \(\V\) over \(\R\) with dimension \(2n\).
\end{proof}

\begin{ex}\label{ex:1.6.29}
	\quad
	\begin{enumerate}
		\item Prove that if \(\W_1\) and \(\W_2\) are finite-dimensional subspaces of a vector space \(\V\) over \(\F\), then the subspace \(\W_1 + \W_2\) is finite-dimensional, and \(\dim(\W_1 + \W_2) = \dim(\W_1) + \dim(\W_2) - \dim(\W_1 \cap \W_2)\).
		\item Let \(\W_1\) and \(\W_2\) be finite-dimensional subspaces of a vector space \(\V\) over \(\F\), and let \(\V = \W_1 + \W_2\).
		      Deduce that \(\V\) is the direct sum of \(\W_1\) and \(\W_2\) iff \(\dim(\V) = \dim(\W_1) + \dim(\W_2)\).
	\end{enumerate}
\end{ex}

\begin{proof}[\pf{ex:1.6.29}(a)]
	Note that \(\V\) can be infinite-dimensional, but the result does not change as long as \(\W_1, \W_2\) are finite-dimensional.
	If \(\W_1 = \W_2\), then we have
	\begin{align*}
		         & \W_1 + \W_1 = \W_1                                                  &  & \by{1.3.10} \\
		\implies & \dim(\W_1) = \dim(\W_1 + \W_1) = \dim(\W_1 \cap \W_1)               &  & \by{1.6.8}  \\
		\implies & \dim(\W_1 + \W_1) = \dim(\W_1) + \dim(\W_1) - \dim(\W_1 \cap \W_1).
	\end{align*}
	So suppose that \(\W_1 \neq \W_2\).

	Since \(\W_1\) is finite-dimensional and \(\W_1 \cap \W_2 \subseteq \W_1\), by \cref{1.4,1.11} we know that \(\W_1 \cap \W_2\) is finite-dimensional.
	Thus by \cref{1.6.8} we can define \(\beta\) as a basis for \(\W_1 \cap \W_2\) over \(\F\).
	Note that \(\#(\beta) \geq 0\).

	By \cref{1.6.15}(c) we can extend \(\beta\) to \(\beta_1\) and \(\beta_2\) which are bases for \(\W_1\) and \(\W_2\) over \(\F\), respectively.
	Note that \(\#(\beta_1), \#(\beta_2) \geq 0\).
	We claim that \((\beta_1 \setminus \beta) \cap \spn{\beta_2} = \varnothing\).
	Suppose for sake of contradiction that
	\[
		\exists v \in \beta_1 \setminus \beta : v \in \spn{\beta_2}.
	\]
	Then we have
	\begin{align*}
		         & \begin{dcases}
			           v \in \beta_1 \setminus \beta \subseteq \beta_1 \subseteq \W_1 \\
			           v \in \spn{\beta_2} = \W_2
		           \end{dcases} &  & \by{1.6.1}                    \\
		\implies & v \in \W_1 \cap \W_2 = \spn{\beta}.                               &  & \by{1.6.1}
	\end{align*}
	Since \(\beta_1 \setminus \beta \subseteq \beta_1\), by \cref{1.5.6} \(\beta_1 \setminus \beta\) is linearly independent.
	But
	\begin{align*}
		         & v \in \beta_1 \setminus \beta                  \\
		\implies & v \notin \spn{\beta}          &  & \by{1.5.3},
	\end{align*}
	a contradiction.
	Thus \((\beta_1 \setminus \beta) \cap \spn{\beta_2} = \varnothing\).

	Now we claim that \(\beta_1 \cup \beta_2\) is a basis for \(\W_1 + \W_2\) over \(\F\).
	Since
	\begin{align*}
		         & \forall v \in \W_1 + \W_2, \exists (v_1, v_2) \in \W_1 \times \W_2 :                        \\
		         & v = v_1 + v_2 \in \spn{\beta_1 \cup \beta_2}                         &  & \by{1.3.10,1.5.3} \\
		\implies & \W_1 + \W_2 \subseteq \spn{\beta_1 \cup \beta_2}                                            \\
		\implies & \W_1 + \W_2 = \spn{\beta_1 \cup \beta_2}                             &  & \by{1.5}
	\end{align*}
	and
	\begin{align*}
		         & \begin{dcases}
			           \beta_1 \cup \beta_2 = (\beta_1 \setminus \beta) \cup \beta_2 \\
			           (\beta_1 \setminus \beta) \cap \spn{\beta_2} = \varnothing
		           \end{dcases} &  & \text{(from the claim above)}                    \\
		\implies & \beta_1 \cup \beta_2 \text{ is linearly independent},            &  & \by{1.5.3}
	\end{align*}
	by \cref{1.6.1} we know that \(\beta_1 \cup \beta_2\) is a basis for \(\W_1 + \W_2\) over \(\F\).
	Since \(\beta_1\) and \(\beta_2\) are finite, we know that \(\beta_1 \cup \beta_2\) is also finite, thus by \cref{1.6.8} \(\W_1 + \W_2\) is finite-dimensional, and we have
	\begin{align*}
		         & \#(\beta_1 \cup \beta_2) = \#(\beta_1) + \#(\beta_2) - \#(\beta_1 \cap \beta_2) \\
		\implies & \#(\beta_1 \cup \beta_2) = \#(\beta_1) + \#(\beta_2) - \#(\beta)                \\
		\implies & \dim(\W_1 + \W_2) = \dim(\W_1) + \dim(\W_2) - \dim(\W_1 \cap \W_2).
	\end{align*}
\end{proof}

\begin{proof}[\pf{ex:1.6.29}(b)]
	We have
	\begin{align*}
		     & \V = \W_1 \oplus \W_2                                                          \\
		\iff & \begin{dcases}
			       \V = \W_1 + \W_2 \\
			       \W_1 \cap \W_2 = \varnothing
		       \end{dcases}                            &  & \by{1.3.11}                       \\
		\iff & \begin{dcases}
			       \V = \W_1 + \W_2 \\
			       \dim(\W_1 \cap \W_2) = 0
		       \end{dcases}                                &  & \by{1.6.2}                    \\
		\iff & \dim(\V) = \dim(\W_1 + \W_2) = \dim(\W_1) + \dim(\W_2). &  & \by{ex:1.6.29}[a]
	\end{align*}
\end{proof}

\setcounter{ex}{30}
\begin{ex}\label{ex:1.6.31}
	Let \(\W_1\) and \(\W_2\) be subspaces of a vector space \(\V\) over \(\F\) having dimensions \(m\) and \(n\), respectively, where \(m \geq n\).
	\begin{enumerate}
		\item Prove that \(\dim(\W_1 \cap \W_2) \leq n\).
		\item Prove that \(\dim(\W_1 + \W_2) \leq m + n\).
	\end{enumerate}
\end{ex}

\begin{proof}[\pf{ex:1.6.31}(a)]
	We have
	\begin{align*}
		         & \W_1 \cap \W_2 \subseteq \W_2                            \\
		\implies & \dim(\W_1 \cap \W_2) \leq \dim(\W_2) = n. &  & \by{1.11}
	\end{align*}
\end{proof}

\begin{proof}[\pf{ex:1.6.31}(b)]
	We have
	\begin{align*}
		\dim(\W_1 + \W_2) & = \dim(\W_1) + \dim(\W_2) - \dim(\W_1 \cap \W_2) &  & \by{ex:1.6.29}[a] \\
		                  & = m + n - \dim(\W_1 \cap \W_2)                                          \\
		                  & \leq m + n.
	\end{align*}
\end{proof}

\setcounter{ex}{32}
\begin{ex}\label{ex:1.6.33}
	\quad
	\begin{enumerate}
		\item Let \(\W_1\) and \(\W_2\) be subspaces of a vector space \(\V\) over \(\F\) such that \(\V = \W_1 \oplus \W_2\).
		      If \(\beta_1\) and \(\beta_2\) are bases for \(\W_1\) and \(\W_2\) over \(\F\), respectively, show that \(\beta_1 \cap \beta_2 = \varnothing\) and \(\beta_1 \cup \beta_2\) is a basis for \(\V\) over \(\F\).
		\item Conversely, let \(\beta_1\) and \(\beta_2\) be disjoint bases for subspaces \(\W_1\) and \(\W_2\) over \(\F\), respectively, of a vector space \(\V\) over \(\F\).
		      Prove that if \(\beta_1 \cup \beta_2\) is a basis for \(\V\) over \(\F\), then \(\V = \W_1 \oplus \W_2\).
	\end{enumerate}
\end{ex}

\begin{proof}[\pf{ex:1.6.33}(a)]
	First observe that
	\begin{align*}
		         & \V = \W_1 \oplus \W_2                                                                                \\
		\implies & \V = \W_1 + \W_2                                                                    &  & \by{1.3.11} \\
		\implies & \forall v \in \V, \exists (x, y) \in \W_1 \times \W_2 : v = x + y                                    \\
		\implies & \forall v \in \V, \exists (x, y) \in \spn{\beta_1} \times \spn{\beta_2} : v = x + y &  & \by{1.6.1}  \\
		\implies & \forall v \in \V, v \in \spn{\beta_1 \cup \beta_2}                                  &  & \by{1.2.1}  \\
		\implies & \V = \spn{\beta_1 \cup \beta_2}.                                                    &  & \by{1.5}
	\end{align*}
	Since
	\begin{align*}
		         & V = \W_1 \oplus \W_2                                                               \\
		\implies & \W_1 \cap \W_2 = \set{\zv}                                  &  & \by{1.3.11}       \\
		\implies & \beta_1 \cap \beta_2 = \varnothing                                                 \\
		\implies & \dim(\V) = \dim(\W_1 \oplus \W_2) = \dim(\W_1) + \dim(\W_2) &  & \by{ex:1.6.29}[b] \\
		         & = \#(\beta_1) + \#(\beta_2) = \#(\beta_1 \cup \beta_2),
	\end{align*}
	by \cref{1.6.15}(a) we know that \(\beta_1 \cup \beta_2\) is a basis for \(\V\) over \(\F\).
\end{proof}

\begin{proof}[\pf{ex:1.6.33}(b)]
	Since
	\begin{align*}
		         & \begin{dcases}
			           \beta_1 \cap \beta_2 = \varnothing \\
			           \spn{\beta_1 \cup \beta_2} = \V
		           \end{dcases}                                                                      \\
		\implies & \forall v \in \V, v \in \spn{\beta_1 \cup \beta_2}                                                      \\
		\implies & \forall v \in \V, \exists (x, y) \in \spn{\beta_1} \times \spn{\beta_2} : v = x + y &  & \by{1.2.1}     \\
		\implies & \forall v \in \V, v \in \W_1 + \W_2                                                 &  & \by{1.3.10}    \\
		\implies & \V \subseteq \W_1 + \W_2                                                            &  & \by{1.5}       \\
		\implies & \V = \W_1 + \W_2                                                                    &  & \by{ex:1.3.23}
	\end{align*}
	and
	\begin{align*}
		         & \begin{dcases}
			           \beta_1 \cap \beta_2 = \varnothing \\
			           \beta_1 \cup \beta_2 \text{ is a basis for } \V \text{ over } \F
		           \end{dcases}                  \\
		\implies & \begin{dcases}
			           \spn{\beta_1} \cap \beta_2 = \varnothing \\
			           \spn{\beta_2} \cap \beta_1 = \varnothing
		           \end{dcases}                      &  & \by{1.7}                                  \\
		\implies & \spn{\beta_1} \cap \spn{\beta_2} = \set{\zv}                     &  & \by{1.5.3} \\
		\implies & \W_1 \cap \W_2 = \set{\zv},
	\end{align*}
	by \cref{1.3.11} we know that \(\V = \W_1 \oplus \W_2\).
\end{proof}

\begin{ex}\label{ex:1.6.34}
	\quad
	\begin{enumerate}
		\item Prove that if \(\W_1\) is any subspace of a finite-dimensional vector space \(\V\) over \(\F\), then there exists a subspace \(\W_2\) of \(\V\) over \(\F\) such that \(\V = \W_1 \oplus \W_2\).
		\item Let \(\V = \R^2\) and \(\W_1 = \set{(a_1, 0) : a_1 \in \R}\).
		      Give examples of two different subspaces \(\W_2\) and \(\W_2'\) such that \(\V = \W_1 \oplus \W_2\) and \(\V = \W_1 \oplus \W_2'\).
	\end{enumerate}
\end{ex}

\begin{proof}[\pf{ex:1.6.34}(a)]
	Let \(\beta_1\) be a basis for \(\W_1\) over \(\F\).
	By \cref{1.6.19} we know that \(\beta_1\) can extend to a basis \(\beta\) for \(\V\) over \(\F\).
	Let \(\beta_2 = \beta \setminus \beta_1\) and let \(\W_2 = \spn{\beta_2}\).
	Since \(\beta_1 \cap \beta_2 = \varnothing\) and \(\spn{\beta_1 \cup \beta_2} = \V\), by \cref{ex:1.6.33}(b) we have \(\V = \W_1 \oplus \W_2\).
\end{proof}

\begin{proof}[\pf{ex:1.6.34}(b)]
	Let
	\begin{align*}
		\W_2  & = \set{(0, a_2) : a_2 \in \R}    \\
		\W_2' & = \set{(a_2, a_2) : a_2 \in \R}.
	\end{align*}
	Clearly we have
	\begin{align*}
		\W_1 \cap \W_2  & = \set{(0, 0)}  \\
		\W_1 \cap \W_2' & = \set{(0, 0)}.
	\end{align*}
	Since
	\begin{align*}
		         & \forall x \in \R^2, x = (x_1, x_2) = (x_1, 0) + (0, x_2)                             \\
		\implies & \forall x \in \R^2, \exists (y, z) \in \W_1 \times \W_2 : x = y + z                  \\
		\implies & \forall x \in \R^2, x \in \W_1 + \W_2                               &  & \by{1.3.10} \\
		\implies & \R^2 = \W_1 + \W_2                                                  &  & \by{1.5}
	\end{align*}
	and
	\begin{align*}
		         & \forall x \in \R^2, x = (x_1, x_2) = (x_1 - x_2, 0) + (x_2, x_2)                      \\
		\implies & \forall x \in \R^2, \exists (y, z) \in \W_1 \times \W_2' : x = y + z                  \\
		\implies & \forall x \in \R^2, x \in \W_1 + \W_2'                               &  & \by{1.3.10} \\
		\implies & \R^2 = \W_1 + \W_2',                                                 &  & \by{1.5}
	\end{align*}
	by \cref{1.3.11} we know that \(\V = \W_1 \oplus \W_2 = \W_1 \oplus \W_2'\).
\end{proof}

\begin{ex}\label{ex:1.6.35}
	Let \(\W\) be a subspace of a finite-dimensional vector space \(\V\) over \(\F\), and consider the basis \(\set{\seq{u}{1,2,,k}}\) for \(\W\) over \(\F\).
	Let \(\set{\seq{u}{1,2,,k,k + 1,,n}}\) be an extension of this basis to a basis for \(\V\) over \(\F\).
	\begin{enumerate}
		\item Prove that \(\set{u_{k + 1} + \W, u_{k + 2} + \W, \dots , u_n + \W}\) is a basis for \(\V / \W\) over \(\F\).
		\item Derive a formula relating \(\dim(\V)\), \(\dim(\W)\), and \(\dim(\V / \W)\).
	\end{enumerate}
\end{ex}

\begin{proof}[\pf{ex:1.6.35}(a)]
	Let \(\beta = \set{u_{k + 1} + \W, u_{k + 2} + \W, \dots , u_n + \W}\).
	First we show that \(\spn{\beta} = \V / \W\).
	\begin{align*}
		         & \forall v + \W \in \V / \W, \begin{dcases}
			                                       v \in \V \\
			                                       v + \W = \set{v + w : w \in \W}
		                                       \end{dcases}                   &  & \by{ex:1.3.31}             \\
		\implies & \forall v + \W \in \V / \W, \exists \seq{a}{1,2,,n} \in \F :                               \\
		         & v + \W = \pa{\sum_{i = 1}^n a_i u_i} + \W                    &  & \by{1.4.3}               \\
		         & = \sum_{i = 1}^n a_i (u_i + \W)                              &  & \by{ex:1.3.31}           \\
		         & = \sum_{i = k + 1}^n a_i (u_i + \W)                          &  & (\seq{u}{1,2,,k} \in \W) \\
		\implies & \spn{\beta} = \V / \W.                                       &  & \by{1.4.3}.
	\end{align*}
	Now we show that \(\beta\) is linearly independent.
	\begin{align*}
		         & \forall \seq{a}{k + 1,,n} \in \F, \sum_{i = k + 1}^n a_i (u_i + \W) = \zv + \W                                    \\
		\implies & \pa{\sum_{i = k + 1}^n a_i u_i} + \W = \W                                      &  & \by{ex:1.3.31}                \\
		\implies & \sum_{i = k + 1}^n a_i u_i \in \W                                              &  & \by{ex:1.3.31}                \\
		\implies & \seq[=]{a}{k + 1,,n} = 0.                                                      &  & (\seq{u}{k + 1,,n} \notin \W)
	\end{align*}
	From all proofs above we conclude by \cref{1.6.1} that \(\beta\) is a basis for \(\V / \W\) over \(\F\).
\end{proof}

\begin{proof}[\pf{ex:1.6.35}(b)]
	From \cref{ex:1.6.35}(a) we see that \(\dim(\V / \W) = \dim(\V) - \dim(\W)\).
\end{proof}

\chapter{Complex Numbers}\label{ch:d}

\begin{defn}\label{d.0.1}
  A complex number is an expression of the form \(z = a + bi\), where \(a\) and \(b\) are real numbers called the \textbf{real part} and the \textbf{imaginary part} of \(z\), respectively.

  The \textbf{sum} and \textbf{product} of two complex numbers \(z = a + bi\) and \(w = c + di\) (where \(a, b, c\), and \(d\) are real numbers) are defined, respectively, as follows:
  \[
    z + w = (a + bi) + (c + di) = (a + c) + (b + d) i
  \]
  and
  \[
    zw = (a + bi)(c + di) = (ac - bd) + (ad + bc) i.
  \]
\end{defn}

\begin{defn}\label{d.0.2}
  Any real number \(c\) may be regarded as a complex number by identifying \(c\) with the complex number \(c + 0i\).
  Observe that this correspondence preserves sums and products;
  that is,
  \[
    (c + 0i) + (d + 0i) = (c + d) + 0i \quad \text{and} \quad (c + 0i)(d + 0i) = cd + 0i.
  \]
  Any complex number of the form \(bi = 0 + bi\), where \(b\) is a nonzero real number, is called \textbf{imaginary}.
  The product of two imaginary numbers is real since
  \[
    (bi)(di) = (0 + bi)(0 + di) = (0 - bd) + (0 \cdot d + b \cdot 0) i = -bd.
  \]
  In particular, for \(i = 0 + 1i\), we have \(i \cdot i = -1\).
\end{defn}

\begin{note}
  The observation that \(i^2 = i \cdot i = -1\) provides an easy way to remember the definition of multiplication of complex numbers:
  simply multiply two complex numbers as you would any two algebraic expressions, and replace \(i^2\) by \(-1\).
\end{note}

\begin{defn}\label{d.0.3}
  The real number \(0\), regarded as a complex number, is an additive identity element for the complex numbers since
  \[
    (a + bi) + 0 = (a + bi) + (0 + 0i) = (a + 0) + (b + 0) i = a + bi.
  \]
  Likewise the real number \(1\), regarded as a complex number, is a multiplicative identity element for the set of complex numbers since
  \[
    (a + bi) \cdot 1 = (a + bi)(1 + 0i) = (a \cdot 1 - b \cdot 0) + (a \cdot 0 + b \cdot 1) i = a + bi.
  \]
  Every complex number \(a + bi\) has an additive inverse, namely \((-a) + (-b)i\).
  But also each complex number except \(0\) has a multiplicative inverse.
  In fact,
  \[
    (a + bi)^{-1} = \dfrac{a}{a^2 + b^2} - \dfrac{b}{a^2 + b^2} i.
  \]
\end{defn}

\begin{thm}\label{d.1}
  The set of complex numbers with the operations of addition and multiplication previously defined is a field.
\end{thm}

\begin{proof}[\pf{d.1}]
  Let \(x, y, z \in \C\).
  By \cref{d.0.1} we know that \(x = a + bi, y = c + di, z = e + fi\) for some \(a, b, c, d, e, f \in \R\).
  By \cref{d.0.1} we know that \(x + y \in \C\) and \(xy \in \C\).
  Thus by \cref{c.0.1} we only need to show that \ref{f1} -- \ref{f5} hold.
  \begin{description}
    \item[For \ref{f1}:]
      We have
      \begin{align*}
        x + y & = (a + c) + (b + d) i &  & \by{d.0.1}          \\
              & = (c + a) + (d + b) i &  & (a, b, c, d \in \R) \\
              & = y + x               &  & \by{d.0.1}
      \end{align*}
      and
      \begin{align*}
        xy & = (ac - bd) + (ad + bc) i &  & \by{d.0.1}          \\
           & = (ca - db) + (da + cb) i &  & (a, b, c, d \in \R) \\
           & = yx.                     &  & \by{d.0.1}
      \end{align*}
    \item[For \ref{f2}:]
      We have
      \begin{align*}
        (x + y) + z & = ((a + c) + (b + d)i) + z        &  & \by{d.0.1}                \\
                    & = ((a + c) + e) + ((b + d) + f) i &  & \by{d.0.1}                \\
                    & = (a + (c + e)) + (b + (d + f)) i &  & (a, b, c, d, e, f \in \R) \\
                    & = x + ((c + e) + (d + f)i)        &  & \by{d.0.1}                \\
                    & = x + (y + z)                     &  & \by{d.0.1}
      \end{align*}
      and
      \begin{align*}
        (x \cdot y) \cdot z & = ((ac - bd) + (ad + bc) i) \cdot z   &  & \by{d.0.1}                \\
                            & = ((ac - bd) e - (ad + bc) f)                                        \\
                            & \quad + ((ac - bd) f + (ad + bc) e) i &  & \by{d.0.1}                \\
                            & = (a (ce - df) - b (cf + de))                                        \\
                            & \quad + (a (cf + de) + b(ce - df)) i  &  & (a, b, c, d, e, f \in \R) \\
                            & = x \cdot ((ce - df) + (cf + de) i)   &  & \by{d.0.1}                \\
                            & = x \cdot (y \cdot z).                &  & \by{d.0.1}
      \end{align*}
    \item[For \ref{f3}:]
      See \cref{d.0.3}.
    \item[For \ref{f4}:]
      See \cref{d.0.3}.
    \item[For \ref{f5}:]
      We have
      \begin{align*}
        x \cdot (y + z) & = x \cdot ((c + e) + (d + f) i)               &  & \by{d.0.1}                        \\
                        & = (a (c + e) - b (d + f))                                                            \\
                        & \quad + (a (d + f) + b (c + e)) i             &  & \by{d.0.1}                        \\
                        & = (ac + ae - bd - bf) + (ad + af + bc + be) i &  & (a, b, c, d, e, f \in \R)         \\
                        & = (ac - bd) + (ad + bc) i                                                            \\
                        & \quad + (ae - bf) + (af + be) i               &  & \text{(by \ref{f1} and \ref{f2})} \\
                        & = xy + xz.                                    &  & \by{d.0.1}
      \end{align*}
      From all cases above we conclude by \cref{c.0.1} that \(\C\) is a field.
  \end{description}
\end{proof}

\begin{defn}\label{d.0.4}
  The \textbf{(complex) conjugate} of a complex number \(a + bi\) is the complex number \(a - bi\).
  We denote the conjugate of the complex number \(z\) by \(\conj{z}\).
\end{defn}

\begin{thm}\label{d.2}
  Let \(z\) and \(w\) be complex numbers.
  Then the following statements are true.
  \begin{enumerate}
    \item \(\conj{\conj{z}} = z\).
    \item \(\conj{z + w} = \conj{z} + \conj{w}\).
    \item \(\conj{zw} = \conj{z} \cdot \conj{w}\).
    \item \(\conj{\dfrac{z}{w}} = \dfrac{\conj{z}}{\conj{w}}\) if \(w \neq 0\).
    \item \(z \in \R\) iff \(\conj{z} = z\).
  \end{enumerate}
\end{thm}

\begin{proof}[\pf{d.2}(a)]
  We have
  \begin{align*}
    \conj{\conj{z}} & = \conj{\Re(z) - \Im(z) i} &  & \by{d.0.4} \\
                    & = \Re(z) + \Im(z) i        &  & \by{d.0.4} \\
                    & = z.
  \end{align*}
\end{proof}

\begin{proof}[\pf{d.2}(b)]
  We have
  \begin{align*}
    \conj{z + w} & = \conj{\Re(z) + \Re(w) + (\Im(z) + \Im(w)) i} &  & \by{d.0.1} \\
                 & = \Re(z) + \Re(w) - (\Im(z) + \Im(w)) i        &  & \by{d.0.4} \\
                 & = \Re(z) - \Im(z) i + \Re(w) - \Im(w) i        &  & \by{d.1}   \\
                 & = \conj{z} + \conj{w}.                         &  & \by{d.0.4}
  \end{align*}
\end{proof}

\begin{proof}[\pf{d.2}(c)]
  We have
  \begin{align*}
    \conj{zw} & = \conj{(\Re(z) \Re(w) - \Im(z) \Im(w)) + (\Re(z) \Im(w) + \Im(z) \Re(w)) i} &  & \by{d.0.1} \\
              & = (\Re(z) \Re(w) - \Im(z) \Im(w)) - (\Re(z) \Im(w) + \Im(z) \Re(w)) i        &  & \by{d.0.4} \\
              & = (\Re(z) - \Im(z) i) (\Re(w) - \Im(w) i)                                    &  & \by{d.0.1} \\
              & = \conj{z} \cdot \conj{w}.                                                   &  & \by{d.0.4}
  \end{align*}
\end{proof}

\begin{proof}[\pf{d.2}(d)]
  We have
  \begin{align*}
    \conj{\dfrac{z}{w}} & = \conj{z \cdot w^{-1}}                                                                                          &  & \by{d.0.3}  \\
                        & = \conj{z} \cdot \conj{w^{-1}}                                                                                   &  & \by{d.2}[c] \\
                        & = \conj{z} \cdot \pa{\conj{\dfrac{\Re(w)}{(\Re(w))^2 + (\Im(w))^2} - \dfrac{\Im(w)}{(\Re(w))^2 + (\Im(w))^2} i}} &  & \by{d.0.3}  \\
                        & = \conj{z} \cdot \pa{\dfrac{\Re(w)}{(\Re(w))^2 + (\Im(w))^2} + \dfrac{\Im(w)}{(\Re(w))^2 + (\Im(w))^2} i}        &  & \by{d.0.4}  \\
                        & = \conj{z} \cdot (\Re(w) - \Im(w))^{-1}                                                                          &  & \by{d.0.3}  \\
                        & = \conj{z} \cdot \conj{w}^{-1}                                                                                   &  & \by{d.0.4}  \\
                        & = \dfrac{\conj{z}}{\conj{w}}.                                                                                    &  & \by{d.0.3}
  \end{align*}
\end{proof}

\begin{proof}[\pf{d.2}(e)]
  We have
  \begin{align*}
         & z \in \R                                              \\
    \iff & \Im(z) = 0                            &  & \by{d.0.2} \\
    \iff & \Re(z) + \Im(z) i = \Re(z) - \Im(z) i                 \\
    \iff & \conj{z} = z.                         &  & \by{d.0.4}
  \end{align*}
\end{proof}

\begin{defn}\label{d.0.5}
  Let \(z = a + bi\), where \(a, b \in \R\).
  The \textbf{absolute value} (or \textbf{modulus}) of \(z\) is the real number \(\sqrt{a^2 + b^2}\).
  We denote the absolute value of \(z\) by \(\abs{z}\).

  Observe that \(z \conj{z} = \abs{z}^2\).
  The fact that the product of a complex number and its conjugate is real provides an easy method for determining the quotient of two complex numbers;
  for if \(c + di \neq 0\), then
  \[
    \dfrac{a + bi}{c + di} = \dfrac{a + bi}{c + di} \cdot \dfrac{c - di}{c - di} = \dfrac{(ac + bd) + (bc - ad) i}{c^2 + d^2} = \dfrac{ac + bd}{c^2 + d^2} + \dfrac{bc - ad}{c^2 + d^2} i.
  \]
\end{defn}

\begin{thm}\label{d.3}
  Let \(z\) and \(w\) denote any two complex numbers.
  Then the following statements are true.
  \begin{enumerate}
    \item \(\abs{zw} = \abs{z} \cdot \abs{w}\).
    \item \(\abs{\dfrac{z}{w}} = \dfrac{\abs{z}}{\abs{w}}\) if \(w \neq 0\).
    \item \(\abs{z + w} \leq \abs{z} + \abs{w}\).
    \item \(\abs{z} - \abs{w} \leq \abs{z + w}\).
  \end{enumerate}
\end{thm}

\begin{proof}[\pf{d.3}(a)]
  By \cref{d.2} we have
  \[
    \abs{zw}^2 = (zw) \conj{(zw)} = (zw) (\conj{z} \cdot \conj{w}) = (z \conj{z}) (w \conj{w}) = \abs{z}^2 \abs{w}^2
  \]
  and thus \(\abs{zw} = \abs{z} \abs{w}\).
\end{proof}

\begin{proof}[\pf{d.3}(b)]
  We have
  \begin{align*}
    \abs{z} & = \abs{\dfrac{z}{w} w}       &  & \by{d.0.3}  \\
            & = \abs{\dfrac{z}{w}} \abs{w} &  & \by{d.3}[a]
  \end{align*}
  and thus
  \[
    \dfrac{\abs{z}}{\abs{w}} = \abs{\dfrac{z}{w}}.
  \]
\end{proof}

\begin{proof}[\pf{d.3}(c)]
  For any complex number \(x = a + bi\), where \(a, b \in \R\), observe that
  \[
    x + \conj{x} = (a + bi) + (a - bi) = 2a \leq 2 \sqrt{a^2 + b^2} = 2 \abs{x}.
  \]
  Thus \(x + \conj{x}\) is real and satisfies the inequality \(x + \conj{x} \leq 2 \abs{x}\).
  Taking \(x = w \conj{z}\), we have, by \cref{d.2} and \cref{d.3}(a),
  \[
    w \conj{z} + \conj{w} z \leq 2 \abs{w \conj{z}} = 2 \abs{w} \abs{\conj{z}} = 2 \abs{z} \abs{w}.
  \]
  Using \cref{d.2} again gives
  \begin{align*}
    \abs{z + w}^2 & = (z + w) \conj{(z + w)} = (z + w) (\conj{z} + \conj{w}) = z \conj{z} + w \conj{z} + z \conj{w} + w \conj{w} \\
                  & \leq \abs{z}^2 + 2 \abs{z} \abs{w} + \abs{w}^2 = (\abs{z} + \abs{w})^2.
  \end{align*}
  By taking square roots, we obtain \cref{d.3}(c).
\end{proof}

\begin{proof}[\pf{d.3}(d)]
  From \cref{d.3}(a) and (c), it follows that
  \[
    \abs{z} = \abs{(z + w) - w} \leq \abs{z + w} + \abs{-w} = \abs{z + w} + \abs{w}.
  \]
  So
  \[
    \abs{z} - \abs{w} \leq \abs{z + w},
  \]
  proving \cref{d.3}(d).
\end{proof}

\begin{defn}\label{d.0.6}
  It is interesting as well as useful that complex numbers have both a geometric and an algebraic representation.
  Suppose that \(z = a + bi\), where \(a\) and \(b\) are real numbers.
  We may represent \(z\) as a vector in the complex plane.
  Notice that, as in \(\R^2\), there are two axes, the \textbf{real axis} and the \textbf{imaginary axis}.
  The real and imaginary parts of \(z\) are the first and second coordinates, and the absolute value of \(z\) gives the length of the vector \(z\).
  It is clear that addition of complex numbers may be represented as in \(\R^2\) using the parallelogram law.

  In \cref{sec:2.7}, we introduce Euler's formula.
  The special case \(e^{i \theta} = \cos(\theta) + i \sin(\theta)\) is of particular interest.
  \(e^{i \theta}\) is the unit vector that makes an angle \(\theta\) with the positive real axis.
  We see that any nonzero complex number \(z\) may be depicted as a multiple of a unit vector, namely, \(z = \abs{z} e^{i \phi}\), where \(\phi\) is the angle that the vector \(z\) makes with the positive real axis.
  Thus multiplication, as well as addition, has a simple geometric interpretation:
  If \(z = \abs{z} e^{i \theta}\) and \(w = \abs{w} e^{i \omega}\) are two nonzero complex numbers, then from the properties established in \cref{sec:2.7} and \cref{d.3}, we have
  \[
    zw = \abs{z} e^{i \theta} \cdot \abs{w} e^{i \omega} = \abs{zw} e^{i (\theta + \omega)}.
  \]
  So \(zw\) is the vector whose length is the product of the lengths of \(z\) and \(w\), and makes the angle \(\theta + \omega\) with the positive real axis.
\end{defn}

\begin{thm}[The Fundamental Theorem of Algebra]\label{d.4}
  Suppose that \(p(z) = a_n z^n + a_{n - 1} z^{n - 1} + \cdots + a_1 z + a_0\) is a polynomial in \(\ps{\C}\) of degree \(n \geq 1\).
  Then \(p(z)\) has a zero.
\end{thm}

\begin{proof}[\pf{d.4}]
  We want to find \(z_0\) in \(\C\) such that \(p(z_0) = 0\).
  Let \(m\) be the greatest lower bound of \(\set{\abs{p(z)} : z \in \C}\).
  For \(\abs{z} = s > 0\), we have
  \begin{align*}
    \abs{p(z)} & = \abs{a_n z^n + a_{n - 1} z^{n - 1} + \cdots + a_0}                                               \\
               & \geq \abs{a_n} \abs{z^n} - \abs{a_{n - 1}} \abs{z^{n - 1}} - \cdots - \abs{a_0} &  & \by{d.3}[a,d] \\
               & = \abs{a_n} s^n - \abs{a_{n - 1}} s^{n - 1} - \cdots - \abs{a_0}                                   \\
               & = s^{n} \pa{\abs{a_n} - \abs{a_{n - 1}} s^{-1} - \cdots - \abs{a_0} s^{-n}}.
  \end{align*}
  Because the last expression approaches infinity as \(s\) approaches infinity, the set \(\set{\abs{p(z)} : (z \in \C) \land (\abs{p(z)} \leq m + 1)}\) is non-empty and closed.
  Since \(p\) is continuous, we know that the set \(D = \set{z \in \C : \abs{p(z)} \leq m + 1}\) is also closed.
  It follows that \(m\) is the greatest lower bound of \(\set{\abs{p(z)} : z \in D}\).
  Because \(D \subseteq \C\) is closed, \(\C\) is compact and \(\abs{p}\) is continuous, we know that \(\abs{p}\) is bounded.
  Since the codomain of \(\abs{p}\) is \(\R\), by maximum principle there exists \(z_0\) in \(D\) such that \(\abs{p(z_0)} = m\).
  We want to show that \(m = 0\).
  We argue by contradiction.

  Assume that \(m \neq 0\).
  Let \(q(z) = \dfrac{p(z + z_0)}{p(z_0)}\).
  Then \(q(z)\) is a polynomial of degree \(n\), \(q(0) = 1\), and \(\abs{q(z)} \geq 1\) for all \(z \in \C\)
  (this is true since \(\abs{p(z + z_0)} \geq \abs{p(z_0)}\)).
  So we may write
  \[
    q(z) = 1 + b_k z^k + b_{k + 1} z^{k + 1} + \cdots + b_n z^n,
  \]
  where \(k \in \N\), \(1 \leq k \leq n\) and \(b_k \neq 0\).
  Because \(-\dfrac{\abs{b_k}}{b_k}\) has modulus one (\(\abs{-\dfrac{\abs{b_k}}{b_k}} = 1\)), we may pick a real number \(\theta\) such that \(e^{i k \theta} = - \dfrac{\abs{b_k}}{b_k}\), or \(e^{i k \theta} b_k = -\abs{b_k}\).
  For any \(r > 0\), we have
  \begin{align*}
    q(r e^{i \theta}) & = 1 + b_k r^k e^{i k \theta} + b_{k + 1} r^{k + 1} e^{i (k + 1) \theta} + \cdots + b_n r^n e^{i n \theta} \\
                      & = 1 - \abs{b_k} r^k + b_{k + 1} r^{k + 1} e^{i (k + 1) \theta} + \cdots + b_n r^n e^{i n \theta}.
  \end{align*}
  Choose \(r\) small enough so that \(1 - \abs{b_k} r^k > 0\).
  Then
  \begin{align*}
    \abs{q(r e^{i \theta})} & \leq \abs{1 - \abs{b_k} r^k} + \abs{b_{k + 1} r^{k + 1} e^{i (k + 1) \theta}} + \cdots + \abs{b_n r^n e^{i n \theta}} &  & \by{d.3}[c] \\
                            & = 1 - \abs{b_k} r^k + \abs{b_{k + 1}} r^{k + 1} + \cdots + \abs{b_n} r^n                                              &  & \by{d.3}[a] \\
                            & = 1 - r^k \pa{\abs{b_k} - \abs{b_{k + 1}} r - \cdots - \abs{b_n} r^{n - k}}.
  \end{align*}
  Now choose \(r\) even smaller, if necessary, so that the expression within the brackets is positive.
  We obtain that \(\abs{q(r e^{i \theta})} < 1\).
  But this is a contradiction.
\end{proof}

\begin{cor}\label{d.0.7}
  If \(p(z) = a_n z^n + a_{n - 1} z^{n - 1} + \cdots + a_1 z + a_0\) is a polynomial of degree \(n \geq 1\) with complex coefficients, then there exist complex numbers \(\seq{c}{1,,n}\) (not necessarily distinct) such that
  \[
    p(z) = a_n (z - c_1)(z - c_2) \cdots (z - c_n).
  \]
\end{cor}

\begin{proof}[\pf{d.0.7}]
  We use induction on \(n\).
  For \(n = 1\), let \(p \in \ps{\C}\) be the function
  \[
    \forall z \in \C, p(z) = a_1 z + a_0
  \]
  where \(\seq{a}{0,1} \in \C\) and \(a_1 \neq 0\).
  Then we have
  \[
    p(z) = a_1 z + a_0 = a_1 (z - (-a_0 / a_1))
  \]
  and thus the base case holds.
  Suppose inductively that \cref{d.0.7} is true for some \(n \geq 1\).
  Then we need to show that \cref{d.0.7} is true for \(n + 1\).
  Let \(p \in \ps{\C}\) be the function
  \[
    \forall z \in \C, p(z) = a_{n + 1} z^{n + 1} + a_n z^n + \cdots + a_1 z + a_0
  \]
  where \(\seq{a}{0,,n+1} \in \C\) and \(a_{n + 1} \neq 0\).
  By \cref{d.4} we know that there exists a \(c_{n + 1} \in \C\) such that \(p(c_{n + 1}) = 0\).
  By \cref{e.0.3} there exists a polynomial \(q_1 \in \ps{\C}\) such that
  \[
    \forall z \in \C, p(z) = q_1(z) (z - c_{n + 1}).
  \]
  Since \(a_{n + 1} \neq 0\), we can rewrite the above equation as
  \[
    \forall z \in \C, p(z) = a_{n + 1} (z - c_{n + 1}) q_2(z)
  \]
  where \(q_2 = \dfrac{1}{a_{n + 1}} q_1\).
  Since \(p\) has degree \(n + 1\), we know that \(q_2\) has degree \(n\).
  Then there exist \(\seq{b}{0,,n-1} \in \C\) such that
  \[
    \forall z \in \C, q_2(z) = b_0 + b_1 z + \cdots + b_n z^n.
  \]
  Note that we must have \(b_n = 1\), otherwise the \((n + 1)\)th coefficient of \(p\) is \(a_n b_n\) instead of \(a_n\).
  By induction hypothesis \(q_2\) can be written as
  \[
    \forall z \in \C, q_2(z) = (z - c_1) (z - c_2) \cdots (z - c_n)
  \]
  where \(\seq{c}{1,,n} \in \C\).
  Then we have
  \begin{align*}
    \forall z \in \C, p(z) & = a_{n + 1} (z - c_{n + 1}) q_2(z)                               \\
                           & = a_{n + 1} (z - c_1) (z - c_2) \cdots (z - c_n) (z - c_{n + 1})
  \end{align*}
  and this closes the induction.
\end{proof}

\begin{defn}\label{d.0.8}
  A field is called \textbf{algebraically closed} if it has the property that every polynomial of positive degree with coefficients from that field factors as a product of polynomials of degree \(1\).
  Thus \cref{d.0.7} asserts that the field of complex numbers is algebraically closed.
\end{defn}

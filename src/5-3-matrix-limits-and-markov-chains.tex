\section{Matrix Limits and Markov Chains}\label{sec:5.3}

\begin{defn}\label{5.3.1}
  Let \(L, \seq{A}{1,2,}\) be \(n \times p\) matrices having complex entries.
  The sequence \(\seq{A}{1,2,}\) is said to \textbf{converge} to the \(n \times p\) matrix \(L\), called the \textbf{limit} of the sequence, if
  \[
    \lim_{m \to \infty} (A_m)_{i j} = L_{i j}
  \]
  for all \(i \in \set{1, \dots, n}\) and \(j \in \set{1, \dots, p}\).
  To designate that \(L\) is the limit of the sequence, we write
  \[
    \lim_{m \to \infty} A_m = L.
  \]
\end{defn}

\begin{thm}\label{5.12}
  Let \(\seq{A}{1,2,}\) be a sequence of \(n \times p\) matrices with complex entries that converges to the matrix \(L\).
  Then for any \(P \in \ms{r}{n}{\C}\) and \(Q \in \ms{p}{s}{\C}\),
  \[
    \lim_{m \to \infty} P A_m = PL \quad \text{and} \quad \lim_{m \to \infty} A_m Q = LQ.
  \]
\end{thm}

\begin{proof}[\pf{5.12}]
  For any \(i \in \set{1, \dots, r}\) and \(j \in \set{1, \dots, p}\),
  \begin{align*}
    \lim_{m \to \infty} (P A_m)_{i j} & = \lim_{m \to \infty} \sum_{k = 1}^n P_{i k} (A_m)_{k j}      &  & \text{(by \cref{2.3.1})} \\
                                      & = \sum_{k = 1}^n P_{i k} \pa{\lim_{m \to \infty} (A_m)_{k j}}                               \\
                                      & = \sum_{k = 1}^n P_{i k} L_{k j}                              &  & \text{(by \cref{5.3.1})} \\
                                      & = (PL)_{i j}.                                                 &  & \text{(by \cref{2.3.1})}
  \end{align*}
  Hence \(\lim_{m \to \infty} P A_m = PL\).
  The proof that \(\lim_{m \to \infty} A_m Q = LQ\) is similar.
\end{proof}

\begin{cor}\label{5.3.2}
  Let \(A \in \ms{n}{n}{\C}\) be such that \(\lim_{m \to \infty} A^m = L\).
  Then for any invertible matrix \(Q \in \ms{n}{n}{\C}\),
  \[
    \lim_{m \to \infty} (Q A Q^{-1})^m = Q L Q^{-1}.
  \]
\end{cor}

\begin{proof}[\pf{5.3.2}]
  Since
  \[
    (Q A Q^{-1})^m = (Q A Q^{-1}) (Q A Q^{-1}) \cdots (Q A Q^{-1}) = Q A^m Q^{-1},
  \]
  we have
  \[
    \lim_{m \to \infty} (Q A Q^{-1})^m = \lim_{m \to \infty} Q A^m Q^{-1} = Q \pa{\lim_{m \to \infty} A^m} Q^{-1} = Q L Q^{-1}
  \]
  by applying \cref{5.12} twice.
\end{proof}

\begin{defn}\label{5.3.3}
  In the discussion that follows, we frequently encounter the set
  \[
    S = \set{\lambda \in \C : \abs{\lambda} < 1 \text{ or } \lambda = 1}.
  \]
  Geometrically, this set consists of the complex number \(1\) and the interior of the unit disk (the disk of radius \(1\) centered at the origin).
  This set is of interest because if \(\lambda\) is a complex number, then \(\lim_{m \to \infty} \lambda^m\) exists iff \(\lambda \in S\).
  This fact, which is obviously true if \(\lambda\) is real, can be shown to be true for complex numbers also.
\end{defn}

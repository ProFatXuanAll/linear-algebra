% We use chapter structure.
\documentclass[12pt,oneside]{book}

%==============================================================================
% Preamble.
%==============================================================================

% Correctly showing characters outside ASCII.
\usepackage[T1]{fontenc}
% File is written and read with utf8 encoding.
\usepackage[utf8]{inputenc}
% Set paging layout.
\usepackage[margin=1.2in]{geometry}
% Including `amsfonts'.  Must be loaded before `mathtools'.
\usepackage{amssymb}
% Including `amsmath' and fixing bugs for `amsmath'.
\usepackage{mathtools}
% Must be loaded after `amsmath' and `mathtools'.
\usepackage{amsthm}
% Automatically adjust character spacing at margins.
\usepackage{microtype}
% Provide further utilities and fix bugs for `enumerate', `itemize' and
% `description'.
\usepackage{enumitem}
% Provide better quoting environment.
\usepackage{dirtytalk}
% Automatically add hyperlinks to labels/refs.  Must be loaded after all
% packages above and before `cleveref'.  Recommend to use with `natbib' when
% you need bibtex.
\usepackage{hyperref}

\hypersetup{         % This macro come with `hyperref'.
    colorlinks=true, % Color hyperlinks.
    linkcolor=blue,  % Color local hyperlinks with blue.
    urlcolor=cyan,   % Color url links with cyan.
}

% Must be loaded after `hyperref'.  We always capitalize each cross-references'
% type name.  See `cleveref' for details.
\usepackage[capitalize]{cleveref}

%------------------------------------------------------------------------------
% Define environments.
%------------------------------------------------------------------------------

% Text inside the body of theorem-like environments are set to Roman font.
% theorem-like environments share their counters, counters follow section and
% reset in every sections.  Exercises has their owned counter.  Notes do not
% use counter.  See `amsthm' for details.
\theoremstyle{definition}
\newtheorem{ax}{Axiom}[section]
\newtheorem{cor}[ax]{Corollary}
\newtheorem{defn}[ax]{Definition}
\newtheorem{eg}[ax]{Example}
\newtheorem{ex}{Exercise}[section]
\newtheorem{lem}[ax]{Lemma}
\newtheorem*{note}{Note}
\newtheorem{prop}[ax]{Proposition}
\newtheorem{rem}[ax]{Remark}
\newtheorem{thm}[ax]{Theorem}

% In `enumerate' enviroments, items' label are alphabets and surrounded by
% parentheses.  See `enumitem' for details.
\renewcommand{\labelenumi}{\textnormal{(}\alph{enumi}\textnormal{)}}

% Formatting equations tag appearence.  See `mathtools' for details.
\renewcommand{\theequation}{\thechapter.\thesection.\arabic{equation}}
\numberwithin{equation}{section}

%------------------------------------------------------------------------------
% Define operators and symbols.
%------------------------------------------------------------------------------

% Define common operators with paired of delimiters.  Always use star versions
% of these operator to automatically adjust height.  See `mathtools' for
% details.

% Absolute value.
\DeclarePairedDelimiter{\absTmp}{\lvert}{\rvert}
\newcommand{\abs}[1]{\absTmp*{#1}}
% Ceiling.
\DeclarePairedDelimiter{\ceilTmp}{\lceil}{\rceil}
\newcommand{\ceil}[1]{\ceilTmp*{#1}}
% Floor.
\DeclarePairedDelimiter{\floorTmp}{\lfloor}{\rfloor}
\newcommand{\floor}[1]{\floorTmp*{#1}}
% Parenthese.
\DeclarePairedDelimiter{\pTmp}{\lparen}{\rparen}
\newcommand{\p}[1]{\pTmp*{#1}}
% Bracket.
\DeclarePairedDelimiter{\bkTmp}{\lbrack}{\rbrack}
\newcommand{\bk}[1]{\bkTmp*{#1}}
% Brace.
\DeclarePairedDelimiter{\BTmp}{\lbrace}{\rbrace}
\newcommand{\B}[1]{\BTmp*{#1}}
% Set.
\newcommand{\set}[1]{\B{#1}}

% Define common symbols.  See `amsmath' section 9.2 for details.

% Fields.
\newcommand{\field}[1]{\mathbf{#1}}
% General field.
\newcommand{\F}{\field{F}}
% Complex number field.
\newcommand{\C}{\field{C}}
% Natural number field.
\newcommand{\N}{\field{N}}
% Rational number field.
\newcommand{\Q}{\field{Q}}
% Real number field.
\newcommand{\R}{\field{R}}
% Integer number field.
\newcommand{\Z}{\field{Z}}

% Vector spaces.
\newcommand{\vs}[1]{\mathsf{#1}}
% General vector space.
\newcommand{\V}{\vs{V}}
\newcommand{\W}{\vs{W}}
% Metric spaces.
\newcommand{\ms}[3]{{\vs{M}_{{#1} \times #2}({#3})}}
% General metric space.
\newcommand{\MS}{{\ms{m}{n}{\F}}}
% Function spaces
\newcommand{\fs}[2]{{\mathcal{F}\p{#1, #2}}}
% General function space.
\newcommand{\FS}{{\fs{S}{\F}}}
% Polynomial spaces.
\newcommand{\ps}[1]{{\vs{P}\p{#1}}}

% n-tuple.
\newcommand{\tp}[2]{{\p{{#1}_{1}, {#1}_{2}, \dots, {#1}_{#2}}}}

% 0 vector.
\newcommand{\zv}{\mathit{0}}
% 0 metric.
\newcommand{\zm}{\mathit{O}}

% Formatting exercises section.
\newcommand{\exercisesection}{
    \begin{center}
        \textbf{EXERCISES}
    \end{center}
}

%==============================================================================
% Document.
%==============================================================================

\begin{document}

%------------------------------------------------------------------------------
% Front matters.
%------------------------------------------------------------------------------

\frontmatter

% Author informations.
\title{Linear Algebra}
\author{ProFatXuanAll}
\maketitle

% Table of contents.
\tableofcontents

%------------------------------------------------------------------------------
% Main matters.
%------------------------------------------------------------------------------

\mainmatter

% All chapters are in separated files.  We include them here.
\chapter{Vector Spaces}\label{ch:1}

% All sections are in separated files.  We include them here.
\section{Introduction}\label{sec:1.1}

\begin{note}
    Experiments show that if two like quantities act together, their effect is predictable.
    In this case, the vectors used to represent these quantities can be combined to form a resultant vector that represents the combined effects of the original quantities.
    This resultant vector is called the \emph{sum} of the original vectors, and the rule for their combination is called the \emph{parallelogram law}.
\end{note}

\begin{axiom}[Parallelogram Law for Vector Addition]\label{ax:1.1.1}
    The sum of two vectors \(x\) and \(y\) that act at the same point \(P\) is the vector beginning at \(P\) that is represented by the diagonal of parallelogram having \(x\) and \(y\) as adjacent sides.
\end{axiom}

\begin{note}
    Since a vector beginning at the origin is completely determined by its endpoint, we sometimes refer to \emph{the point \(x\)} rather than \emph{the endpoint of the vector \(x\)} if \(x\) is a vector emanating from the origin.
\end{note}

\begin{note}
    Besides the operation of vector addition, there is another natural operation that can be performed on vectors
    --- the length of a vector may be magnified or contracted.
    This operation, called \emph{scalar multiplication}, consists of multiplying the vector by a real number.
    If the vector \(x\) is represented by an arrow, then for any real number \(t\), the vector \(tx\) is represented by an arrow in the same direction if \(t \geq 0\) and in the opposite direction if \(t < 0\).
    The length of the arrow \(tx\) is \(\abs*{t}\) times the length of the arrow \(x\).
    Two nonzero vectors \(x\) and \(y\) are called \textbf{parallel} if \(y = tx\) for some nonzero real number \(t\).
    (Thus nonzero vectors having the same or opposite directions are parallel.)
\end{note}
\section{Vector Spaces}\label{sec:1.2}

\begin{defn}\label{1.2.1}
    A \textbf{vector space} (or \textbf{linear space}) \(\V\) over a field \(\F\) consists of a set on which two operations (called \textbf{addition} and \textbf{scalar multiplication}, respectively) are defined so that for each pair of elements \(x\), \(y\) in \(\V\) there is a unique element \(x + y\) in \(\V\), and for each element \(a\) in \(\F\) and each element \(x\) in \(\V\) there is a unique element \(ax\) in \(\V\), such that the following conditions hold.
    \begin{enumerate}[label=(VS \arabic*), ref=VS \arabic*]
        \item\label{vs1}
        For all \(x, y\) in \(\V\), \(x + y = y + x\)
        (commutativity of addition).
        \item\label{vs2}
        For all \(x, y, z\) in \(\V\), \(\p{x + y} + z = x + \p{y + z}\)
        (associativity of addition).
        \item\label{vs3}
        There exists an element in \(\V\) denoted by \(0\) such that \(x + 0 = x\) for each \(x\) in \(\V\).
        \item\label{vs4}
        For each element \(x\) in \(\V\) there exists an element \(y\) in \(\V\) such that \(x + y = 0\).
        \item\label{vs5}
        For each element \(x\) in \(\V\), \(1x = x\).
        \item\label{vs6}
        For each pair of elements \(a, b\) in \(\F\) and each element \(x\) in \(\V\), \(\p{ab} x = a \p{bx}\).
        \item\label{vs7}
        For each element \(a\) in \(\F\) and each pair of elements \(x, y\) in \(\V\), \(a \p{x + y} = ax + ay\).
        \item\label{vs8}
        For each pair of elements \(a, b\) in \(\F\) and each element \(x\) in \(\V\), \(\p{a + b} x = ax + bx\).
    \end{enumerate}
    The elements \(x + y\) and \(ax\) are called the \textbf{sum} of \(x\) and \(y\) and the \textbf{product} of \(a\) and \(x\), respectively.
\end{defn}

\begin{defn}\label{1.2.2}
    The elements of the field \(\F\) are called \textbf{scalars} and the elements of the vector space \(\V\) are called \textbf{vectors}.
\end{defn}

\begin{note}
    A vector space is frequently discussed in the text without explicitly mentioning its field of scalars.
    The reader is cautioned to remember, however, that \emph{every vector space is regarded as a vector space over a given field, which is denoted by \(\F\)}.
    Occasionally we restrict our attention to the fields of real and complex numbers, which are denoted \(\R\) and \(\C\), respectively.
\end{note}

\begin{note}
    \ref{vs2} permits us to unambiguously define the addition of any finite number of vectors
    (without the use of parentheses).
\end{note}

\begin{defn}\label{1.2.3}
    An object of the form \(\tp{a}{n}\), where the entries \(a_{1}, a_{2}, \dots, a_{n}\) are elements of a field \(\F\), is called an \textbf{\(n\)-tuple} with entries from \(\F\).
    The elements \(a_{1}, a_{2}, \dots, a_{n}\) are called the \textbf{entries} or \textbf{components} of the \(n\)-tuple.
    Two \(n\)-tuples \(\tp{a}{n}\) and \(\tp{b}{n}\) with entries from a field \(\F\) are called \textbf{equal} if \(a_i = b_i\) for \(i = 1, 2, \dots, n\).
\end{defn}

\begin{eg}\label{1.2.4}
    The set of all \(n\)-tuples with entries from a field \(\F\) is denoted by \(\vs{F}^{n}\).
    This set is a vector space over \(\F\) with the operations of coordinatewise addition and scalar multiplication;
    that is, if \(u = \tp{a}{n} \in \vs{F}^{n}\), \(v = \tp{b}{n} \in \vs{F}^{n}\), and \(c \in \F\), then
    \[
        u + v = (a_{1} + b_{1}, a_{2} + b_{2}, \dots, a_{n} + b_{n}) \quad \text{ and } \quad cu = \tp{ca}{n}.
    \]
\end{eg}

\begin{proof}
    Clearly we have
    \[
        \forall u, v \in \vs{F}^{n}, u + v \in \vs{F}^{n}
    \]
    and
    \[
        \begin{dcases}
            \forall u \in \vs{F}^{n} \\
            \forall c \in \F
        \end{dcases}, cu \in \vs{F}^{n}.
    \]

    Let \(I_{n} = \set{i \in \N : 1 \leq i \leq n}\).
    First we show that addition and scalar multiplication in \(\vs{F}^{n}\) over \(\F\) are unique.
    Suppose that \(u, u', v, v' \in \vs{F}^{n}\) such that \(u = u'\) and \(v = v'\).
    Then we have
    \begin{align*}
                 & \p{u = u'} \land \p{v = v'}                                                                                 \\
        \implies & \forall i \in I_{n}, \begin{dcases}
            u_{i} = u_{i}' \\
            v_{i} = v_{i}'
        \end{dcases}       &  & \text{(this is the definition of \(\vs{F}^{n}\))} \\
        \implies & \forall i \in I_{n}, u_{i} + v_{i} = u_{i}' + v_{i}' &  & \text{(\(\F\) is a field)}                        \\
        \implies & u + v = u' + v'                                      &  & \text{(this is the definition of \(\vs{F}^{n}\))}
    \end{align*}
    and thus the addition in \(\vs{F}^{n}\) over \(\F\) is unique.
    Now suppose that \(u, u' \in \vs{F}^{n}\) and \(c, c' \in \F\) such that \(u = u'\) and \(c = c'\).
    Then we have
    \begin{align*}
                 & \p{u = u'} \land \p{c = c'}                                                                     \\
        \implies & \begin{dcases}
            \forall i \in I_{n}, u_{i} = u_{i}' \\
            c = c'
        \end{dcases}                &  & \text{(this is the definition of \(\vs{F}^{n}\))} \\
        \implies & \forall i \in I_{n}, c u_{i} = c' u_{i}' &  & \text{(\(\F\) is a field)}                        \\
        \implies & cu = c' u'                               &  & \text{(this is the definition of \(\vs{F}^{n}\))}
    \end{align*}
    and thus the scalar multiplication in \(\vs{F}^{n}\) over \(\F\) is unique.

    Now we show that \ref{vs1}--\ref{vs8} holds for \cref{1.2.4}.
    \begin{description}
        \item[For \ref{vs1}:]
            For all \(x, y \in \vs{F}^{n}\), we have
            \begin{align*}
                x + y & = \p{x_{1} + y_{1}, x_{2} + y_{2}, \dots, x_{n} + y_{n}} &  & \text{(by \cref{1.2.4})}   \\
                      & = \p{y_{1} + x_{1}, y_{2} + x_{2}, \dots, y_{n} + x_{n}} &  & \text{(\(\F\) is a field)} \\
                      & = y + x.                                                 &  & \text{(by \cref{1.2.4})}
            \end{align*}
        \item[For \ref{vs2}:]
            For all \(x, y, z \in \vs{F}^{n}\), we have
            \begin{align*}
                 & \p{x + y} + z                                                                                                                \\
                 & = \p{x_{1} + y_{1}, x_{2} + y_{2}, \dots, x_{n} + y_{n}} + z                                 &  & \text{(by \cref{1.2.4})}   \\
                 & = \p{\p{x_{1} + y_{1}} + z_{1}, \p{x_{2} + y_{2}} + z_{2}, \dots, \p{x_{n} + y_{n}} + z_{n}} &  & \text{(by \cref{1.2.4})}   \\
                 & = \p{x_{1} + \p{y_{1} + z_{1}}, x_{2} + \p{y_{2} + z_{2}}, \dots, x_{n} + \p{y_{n} + z_{n}}} &  & \text{(\(\F\) is a field)} \\
                 & = x + \p{y_{1} + z_{1}, y_{2} + z_{2}, \dots, y_{n} + z_{n}}                                 &  & \text{(by \cref{1.2.4})}   \\
                 & = x + \p{y + z}.                                                                             &  & \text{(by \cref{1.2.4})}
            \end{align*}
        \item[For \ref{vs3}:]
            Since \(\F\) is a field, we know that \(0 \in \F\) and thus \(\p{0, \dots, 0} \in \vs{F}^{n}\).
            Then for all \(x \in \vs{F}^{n}\), we have
            \begin{align*}
                x + \p{0, \dots, 0} & = \p{x_{1} + 0, x_{2} + 0, \dots, x_{n} + 0} &  & \text{(by \cref{1.2.4})}   \\
                                    & = \tp{x}{n}                                  &  & \text{(\(\F\) is a field)} \\
                                    & = x.
            \end{align*}
            We denote \(\zv = \p{0, \dots, 0}\).
        \item[For \ref{vs4}:]
            For all \(x \in \vs{F}^{n}\), we have
            \begin{align*}
                         & \forall i \in I_{n}, x_{i} \in \F                                                                                 \\
                \implies & \forall i \in I_{n}, \exists y_{i} \in \F : x_{i} + y_{i} = 0                  &  & \text{(\(\F\) is a field)}    \\
                \implies & \exists y_{1}, y_{2}, \dots, y_{n} \in \F :                                                                       \\
                         & \p{x_{1} + y_{1}, x_{2} + y_{2}, \dots, x_{n} + y_{n}} = \p{0, \dots, 0} = \zv                                    \\
                \implies & \exists y \in \vs{F}^{n} : x + y = \zv.                                        &  & \text{(from the proof above)}
            \end{align*}
        \item[For \ref{vs5}:]
            Since \(\F\) is a field, we know that \(1 \in \F\).
            Then for all \(x \in \vs{F}^{n}\), we have
            \begin{align*}
                1x & = \p{1 x_{1}, 1 x_{2}, \dots, 1 x_{n}} &  & \text{(by \cref{1.2.4})}   \\
                   & = \tp{x}{n}                            &  & \text{(\(\F\) is a field)} \\
                   & = x.
            \end{align*}
        \item[For \ref{vs6}:]
            For all \(a, b \in \F\) and \(x \in \vs{F}^{n}\), we have
            \begin{align*}
                \p{ab} x & = \p{\p{ab} x_{1}, \p{ab} x_{2}, \dots, \p{ab} x_{n}}    &  & \text{(by \cref{1.2.4})}   \\
                         & = \p{a \p{b x_{1}}, a \p{b x_{2}}, \dots, a \p{b x_{n}}} &  & \text{(\(\F\) is a field)} \\
                         & = a \p{b x_{1}, b x_{2}, \dots, b x_{n}}                 &  & \text{(by \cref{1.2.4})}   \\
                         & = a \p{bx}.                                              &  & \text{(by \cref{1.2.4})}
            \end{align*}
        \item[For \ref{vs7}:]
            For all \(a \in \F\) and \(x, y \in \vs{F}^{n}\), we have
            \begin{align*}
                a \p{x + y} & = a \p{x_{1} + y_{1}, x_{2} + y_{2}, \dots, x_{n} + y_{n}}                    &  & \text{(by \cref{1.2.4})}   \\
                            & = \p{a \p{x_{1} + y_{1}}, a \p{x_{2} + y_{2}}, \dots, a \p{x_{n} + y_{n}}}    &  & \text{(by \cref{1.2.4})}   \\
                            & = \p{a x_{1} + a y_{1}, a x_{2} + a y_{2}, \dots, a x_{n} + a y_{n}}          &  & \text{(\(\F\) is a field)} \\
                            & = \p{a x_{1}, a x_{2}, \dots, a x_{n}} + \p{a y_{1}, a y_{2}, \dots, a y_{n}} &  & \text{(by \cref{1.2.4})}   \\
                            & = a x + a y.                                                                  &  & \text{(by \cref{1.2.4})}
            \end{align*}
        \item[For \ref{vs8}:]
            For all \(a, b \in \F\) and \(x \in \vs{F}^{n}\), we have
            \begin{align*}
                \p{a + b} x & = \p{\p{a + b} x_{1}, \p{a + b} x_{2}, \dots, \p{a + b} x_{n}}                &  & \text{(by \cref{1.2.4})}   \\
                            & = \p{a x_{1} + b x_{1}, a x_{2} + b x_{2}, \dots, a x_{n} + b x_{n}}          &  & \text{(\(\F\) is a field)} \\
                            & = \p{a x_{1}, a x_{2}, \dots, a x_{n}} + \p{b x_{1}, b x_{2}, \dots, b x_{n}} &  & \text{(by \cref{1.2.4})}   \\
                            & = a x + b x.                                                                  &  & \text{(by \cref{1.2.4})}
            \end{align*}
    \end{description}
    From all proofs above we conclude by \cref{1.2.1} that \cref{1.2.4} is indeed a vector space.
\end{proof}

\section{Subspaces}\label{sec:1.3}

\begin{defn}\label{1.3.1}
  A subset \(\W\) of a vector space \(\V\) over a field \(\F\) is called a \textbf{subspace} of \(\V\) if \(\W\) is a vector space over \(\F\) with the operations of addition and scalar multiplication defined on \(\V\).
\end{defn}

\begin{eg}\label{1.3.2}
  In any vector space \(\V\), note that \(\V\) and \(\set{\zv}\) are subspaces.
  The latter is called the \textbf{zero subspace} of \(\V\).
\end{eg}

\begin{proof}
  Since \(\V \subseteq \V\) and \(\V\) is a vector space over \(\F\) with the operations of addition and scalar multiplication defined on \(\V\), by \cref{1.3.1} we know that \(\V\) is a subspace of \(\V\).
  Since \(\zv \in \V\) (by \ref{vs3}), we know that \(\set{\zv} \subseteq \V\).
  Thus by \cref{ex:1.2.11} and \cref{1.3.1} \(\set{\zv}\) is a subspace of \(\V\).
\end{proof}
\section{Linear Combinations and Systems of Linear Equations}\label{sec:1.4}

\begin{defn}\label{1.4.1}
  Let \(\V\) be a vector space over \(\F\) and \(S\) a nonempty subset of \(\V\).
  A vector \(v \in \V\) is called a \textbf{linear combination} of vectors of \(S\) if there exist a finite number of vectors \(\seq{u}{1,2,,n}\) in \(S\) and scalars \(\seq{a}{1,2,,n}\) in \(\F\) such that \(v = \seq[+]{a,u}{1,2,,n}\).
  In this case we also say that \(v\) is a linear combination of \(\seq{u}{1,2,,n}\) and call \(\seq{a}{1,2,,n}\) the \textbf{coefficients} of the linear combination.
\end{defn}

\begin{eg}\label{1.4.2}
  Observe that in any vector space \(\V\), \(0v = \zv\) for each \(v \in \V\).
  Thus the zero vector is a linear combination of any nonempty subset of \(\V\).
\end{eg}

\begin{defn}\label{1.4.3}
  Let \(S\) be a nonempty subset of a vector space \(\V\) over \(\F\).
  The \textbf{span} of \(S\), denoted \(\spn{S}\), is the set consisting of all linear combinations of the vectors in \(S\).
  For convenience, we define \(\spn{\varnothing} = \set{\zv}\).
\end{defn}

\begin{thm}\label{1.5}
  The span of any subset \(S\) of a vector space \(\V\) over \(\F\) is a subspace of \(\V\) over \(\F\).
  Moreover, any subspace of \(\V\) over \(\F\) that contains \(S\) must also contain the span of \(S\).
\end{thm}

\begin{proof}
  This result is immediate if \(S = \varnothing\) because \(\spn{\varnothing} = \set{\zv}\), which is a subspace that is contained in any subspace of \(\V\) over \(\F\)
  (see \cref{1.3.2}).

  If \(S \neq \varnothing\), then \(S\) contains a vector \(z\).
  So \(0z = \zv\) is in \(\spn{S}\).
  Let \(x, y \in \spn{S}\).
  Then there exist vectors \(\seq{u}{1,2,,m}\), \(\seq{v}{1,2,,n}\) in \(S\) and scalars \(\seq{a}{1,2,,m}\), \(\seq{b}{1,2,,n}\) in \(\F\) such that
  \[
    x = \seq[+]{a,u}{1,2,,m} \quad \text{and} \quad y = \seq[+]{b,v}{1,2,,n}
  \]
  Then
  \[
    x + y = \seq[+]{a,u}{1,2,,m} + \seq[+]{b,v}{1,2,,n}
  \]
  and, for any scalar \(c \in \F\),
  \[
    cx = (ca_1) u_1 + (ca_2) u_2 + \cdots + (ca_m) u_m
  \]
  are clearly linear combinations of the vectors in \(S\);
  so \(x + y\) and \(cx\) are in \(\spn{S}\).
  Thus \(\spn{S}\) is a subspace of \(\V\) over \(\F\) (by \cref{1.3}).

  Now let \(\W\) denote any subspace of \(\V\) over \(\F\) that contains \(S\).
  If \(w \in \spn{S}\), then \(w\) has the form \(w = \seq[+]{c,w}{1,2,,k}\) for some vectors \(\seq{w}{1,2,,k}\) in \(S\) and some scalars \(\seq{c}{1,2,,k}\).
  Since \(S \subseteq \W\), we have \(\seq{w}{1,2,,k} \in \W\).
  Therefore \(w = \seq[+]{c,w}{1,2,,k}\) is in \(\W\) by \cref{ex:1.3.20} of \cref{sec:1.3}.
  Because \(w\), an arbitrary vector in \(\spn{S}\), belongs to \(\W\), it follows that \(\spn{S} \subseteq \W\).
\end{proof}

\begin{defn}\label{1.4.4}
  A subset \(S\) of a vector space \(\V\) over \(\F\) \textbf{generates} (or \textbf{spans}) \(\V\) if \(\spn{S} = \V\).
  In this case, we also say that the vectors of \(S\) generate (or span) \(\V\).
\end{defn}

\begin{note}
  Usually there are many different subsets that generate a subspace \(\W\).
  It is natural to seek a subset of \(\W\) that generates \(\W\) and is as small as possible.
\end{note}

\section{Linear Dependence and Linear Independence}\label{sec:1.5}

\begin{note}
  Suppose that \(\V\) is a vector space over an infinite field and that \(\W\) is a subspace of \(\V\).
  Unless \(\W\) is the zero subspace, \(\W\) is an infinite set.
  It is desirable to find a ``small'' finite subset \(S\) that generates \(\W\) because we can then describe each vector in \(\W\) as a linear combination of the finite number of vectors in \(S\).
  Indeed, the smaller that \(S\) is, the fewer computations that are required to represent vectors in \(\W\).

  Checking that some vector in \(S\) is a linear combination of the other vectors in \(S\) could require that we solve several different systems of linear equations before we determine which, if any, vectors in \(S\) is a linear combination of the others.

  Because some vector in \(S\) is a linear combination of the others, the zero vector can be expressed as a linear combination of the vectors in \(S\) using coefficients that are not all zero.
  The converse of this statement is also true:
  If the zero vector can be written as a linear combination of the vectors in \(S\) in which not all the coefficients are zero, then some vector in \(S\) is a linear combination of the others.
  Thus, rather than asking whether some vector in \(S\) is a linear combination of the other vectors in \(S\), it is more efficient to ask whether the zero vector can be expressed as a linear combination of the vectors in \(S\) with coefficients that are not all zero.
\end{note}

\begin{defn}\label{1.5.1}
  A subset \(S\) of a vector space \(\V\) over \(\F\) is called \textbf{linearly dependent} if there exist a finite number of distinct vectors \(\seq{u}{1,,n}\) in \(S\) and scalars \(\seq{a}{1,,n}\) in \(\F\), not all zero, such that
  \[
    \seq[+]{a,u}{1,,n} = \zv.
  \]
  In this case we also say that the vectors of \(S\) are linearly dependent.
\end{defn}

\begin{defn}\label{1.5.2}
  For any vectors \(\seq{u}{1,,n}\), we have \(\seq[+]{a,u}{1,,n} = \zv\) if \(\seq[=]{a}{1,,n} = 0\).
  We call this the \textbf{trivial representation} of \(\zv\) as a linear combination of \(\seq{u}{1,,n}\).
  Thus, for a set to be linearly dependent, there must exist a nontrivial representation of \(\zv\) as a linear combination of vectors in the set.
  Consequently, any subset of a vector space that contains the zero vector is linearly dependent, because \(\zv = 1 \cdot \zv\) is a nontrivial representation of \(\zv\) as a linear combination of vectors in the set.
\end{defn}

\begin{defn}\label{1.5.3}
  A subset \(S\) of a vector space over \(\F\) that is not linearly dependent is called \textbf{linearly independent}.
  As before, we also say that the vectors of \(S\) are linearly independent.
\end{defn}

\begin{eg}\label{1.5.4}
  The following facts about linearly independent sets are true in any vector space.
  \begin{enumerate}
    \item The empty set is linearly independent, for linearly dependent sets must be nonempty.
    \item A set consisting of a single nonzero vector is linearly independent.
          For if \(\set{u}\) is linearly dependent, then \(au = \zv\) for some nonzero scalar \(a\).
          Thus
          \[
            u = a^{-1} (au) = a^{-1} \zv = \zv.
          \]
    \item A set is linearly independent iff the only representations of \(\zv\) as linear combinations of its vectors are trivial representations.
  \end{enumerate}
\end{eg}

\begin{eg}\label{1.5.5}
  For \(k = 0, 1, \dots, n\) let \(p_k(x) = x^k + x^{k + 1} + \cdots + x^n\).
  The set
  \[
    \set{p_0(x), p_1(x), \dots, p_n(x)}
  \]
  is linearly independent in \(\ps[n]{\F}\).
  For if
  \[
    a_0 p_0(x) + a_1 p_1(x) + \cdots + a_n p_n(x) = \zv
  \]
  for some scalars \(\seq{a}{0,1,,n}\), then
  \[
    a_0 + (a_0 + a_1) x + (a_0 + a_1 + a_2) x^2 + \cdots + (a_0 + a_1 + \cdots + a_n) x^n = \zv.
  \]
  By equating the coefficients of \(x^k\) on both sides of this equation for \(k = 0, 1, \dots, n\), we obtain
  \[
    \begin{matrix*}[l]
      & \seq[+]{a}{0}      & = 0 \\
      & \seq[+]{a}{0,1}    & = 0 \\
      & \seq[+]{a}{0,1,2}  & = 0 \\
      & \vdots & \\
      & \seq[+]{a}{0,1,2,,n} & = 0
    \end{matrix*}
  \]
  Clearly the only solution to this system of linear equations is \(\seq[=]{a}{0,1,,n} = 0\).
\end{eg}

\begin{thm}\label{1.6}
  Let \(\V\) be a vector space over \(\F\), and let \(S_1 \subseteq S_2 \subseteq \V\).
  If \(S_1\) is linearly dependent, then \(S_2\) is linearly dependent.
\end{thm}

\begin{proof}[\pf{1.6}]
  We have
  \begin{align*}
             & \begin{dcases}
                 S_1 \subseteq S_2 \\
                 S_1 \text{ is linearly dependent}
               \end{dcases}                               \\
    \implies & \begin{dcases}
                 \exists \seq{u}{1,,n} \in S_1 \subseteq S_2 \\
                 \exists \seq{a}{1,,n} \in \F
               \end{dcases} :                    \\
             & \begin{dcases}
                 \seq[+]{a,u}{1,,n} = \zv \\
                 \lnot (\seq[=]{a}{1,,n} = 0)
               \end{dcases}                   &  & \by{1.5.1}                 \\
    \implies & S_2 \text{ is linearly dependent}.             &  & \by{1.5.1}
  \end{align*}
\end{proof}

\begin{cor}\label{1.5.6}
  Let \(\V\) be a vector space, and let \(S_1 \subseteq S_2 \subseteq \V\).
  If \(S_2\) is linearly independent, then \(S_1\) is linearly independent.
\end{cor}

\begin{proof}[\pf{1.5.6}]
  Suppose for sake of contradiction that \(S_1\) is linearly dependent.
  But by \cref{1.6} \(S_1 \subseteq S_2\) implies \(S_2\) is linearly dependent, which contradicts to the fact that \(S_2\) is linearly independent.
  Thus \(S_1\) is linearly independent.
\end{proof}

\begin{note}
  Earlier in this section, we noted that the issue of whether \(S\) is the smallest generating set for its span is related to the question of whether some vector in \(S\) is a linear combination of the other vectors in \(S\).
  Thus the issue of whether \(S\) is the smallest generating set for its span is related to the question of whether \(S\) is linearly dependent.

  More generally, suppose that \(S\) is any linearly dependent set containing two or more vectors.
  Then some vector \(v \in S\) can be written as a linear combination of the other vectors in \(S\), and the subset obtained by removing \(v\) from \(S\) has the same span as \(S\).
  It follows that \emph{if no proper subset of \(S\) generates the span of \(S\), then \(S\) must be linearly independent.}
\end{note}

\begin{thm}\label{1.7}
  Let \(S\) be a linearly independent subset of a vector space \(\V\) over \(\F\), and let \(v\) be a vector in \(\V\) that is not in \(S\).
  Then \(S \cup \set{v}\) is linearly dependent iff \(v \in \spn{S}\).
\end{thm}

\begin{proof}[\pf{1.7}]
  By \cref{1.5.2} we see that when \(v = \zv\) the statement holds.
  So suppose that \(v \neq \zv\).

  If \(S \cup \set{v}\) is linearly dependent, then there are vectors \\
  \(\seq{u}{1,,n}\) in \(S\) such that \(a_0 v + \seq[+]{a,u}{1,,n} = \zv\) for some nonzero scalars \(\seq{a}{0,1,2,,n}\) in \(\F\).
  We claim that \(a_0 \neq 0\).
  So suppose for sake of contradiction that \(a_0 = 0\).
  Because \(S\) is linearly independent, we know that
  \begin{align*}
             & a_0 v + \seq[+]{a,u}{1,,n}                            \\
             & = 0v + \seq[+]{a,u}{1,,n}                             \\
             & = \zv + \seq[+]{a,u}{1,,n} &  & \by{1.2}[a]           \\
             & = \seq[+]{a,u}{1,,n}       &  & \text{(by \ref{vs3})} \\
             & = \zv                                                 \\
    \implies & \seq[=]{a}{1,,n} = 0.      &  & \by{1.5.3}
  \end{align*}
  But this means \(\seq[=]{a}{0,1,2,,n} = 0\), a contradiction.
  Thus we must have \(a_0 \neq 0\), and so
  \[
    v = a_0^{-1} (-\seq[-]{a,u}{1,,n}) = -(a_0^{-1} a_1) u_1 - (a_0^{-1} a_2) u_2 - \cdots - (a_0^{-1} a_n) u_n.
  \]
  Since \(v\) is a linear combination of \(\seq{u}{1,,n}\), which are in \(S\), we have \(v \in \spn{S}\).

  Conversely, let \(v \in \spn{S}\).
  Then there exist vectors \(\seq{v}{1,,m}\) in \(S\) and scalars \(\seq{b}{1,,m}\) such that \(v = \seq[+]{b,v}{1,,m}\).
  Hence
  \[
    \zv = \seq[+]{b,v}{1,,m} + (-1)v.
  \]
  Since \(v \neq v_i\) for \(i = 1, 2, \dots, m\), the coefficient of \(v\) in this linear combination is nonzero, and so the set \(\set{\seq{v}{1,,m}, v}\) is linearly dependent.
  Therefore \(S \cup \set{v}\) is linearly dependent by \cref{1.6}.
\end{proof}

\exercisesection

\setcounter{ex}{3}
\begin{ex}\label{ex:1.5.4}
  In \(\vs{F}^n\), let \(e_j\) denote the vector whose \(j\)th coordinate is \(1\) and whose other coordinates are \(0\).
  Prove that \(\set{\seq{e}{1,,n}}\) is linearly independent.
\end{ex}

\begin{proof}[\pf{ex:1.5.4}]
  Let \(\seq{a}{1,,n} \in \F\).
  Since
  \begin{align*}
             & \seq[+]{a,e}{1,,n} = \zv                          \\
    \implies & \begin{dcases}
                 a_1 1 + a_2 0 + \cdots + a_n 0 = 0 \\
                 a_1 0 + a_2 1 + \cdots + a_n 0 = 0 \\
                 \vdots                             \\
                 a_1 0 + a_2 0 + \cdots + a_n 1 = 0
               \end{dcases}             &  & \by{1.2.4}          \\
    \implies & \forall i \in \set{1, \dots, n}, a_i 1 = a_i = 0,
  \end{align*}
  by \cref{1.5.3} we know that \(\set{\seq{e}{1,,n}}\) is linearly independent.
\end{proof}

\begin{ex}\label{ex:1.5.5}
  Show that the set \(\set{1, x, x^2, \dots, x^n}\) is linearly independent in \(\ps[n]{\F}\).
\end{ex}

\begin{proof}[\pf{ex:1.5.5}]
  Let \(\seq{a}{0,1,2,,n} \in \F\).
  Since
  \begin{align*}
             & a_0 + a_1 x^1 + a_2 x^2 + \cdots + a_n x^n = \zv = 0 + 0x + 0x^2 + \cdots + 0x^n \\
    \implies & \seq[=]{a}{0,1,2,,n} = 0,
  \end{align*}
  by \cref{1.5.3} we know that \(\set{1, x, x^2, \dots, x^n}\) is linearly independent.
\end{proof}

\begin{ex}\label{ex:1.5.6}
  In \(\ms\), let \(E^{i j}\) denote the matrix whose only nonzero entry is \(1\) in the \(i\)th row and \(j\)th column.
  Prove that \(\set{E^{i j} : 1 \leq i \leq m, 1 \leq j \leq n}\) is linearly independent.
\end{ex}

\begin{proof}[\pf{ex:1.5.6}]
  Let \(a_{i j} \in \F\) where \(1 \leq i \leq m\) and \(1 \leq j \leq n\).
  Since
  \begin{align*}
             & \sum_{i = 1}^m \sum_{j = 1}^{n} a_{i j} E^{i j} = \zm                 \\
    \implies & \begin{pmatrix}
                 a_{1 1} 1 & a_{1 2} 1 & \cdots & a_{1 n} 1 \\
                 a_{2 1} 1 & a_{2 2} 1 & \cdots & a_{2 n} 1 \\
                 \vdots    & \vdots    & \ddots & \vdots    \\
                 a_{m 1} 1 & a_{m 2} 1 & \cdots & a_{m n} 1
               \end{pmatrix}                            \\
             & = \begin{pmatrix}
                   a_{1 1} & a_{1 2} & \cdots & a_{1 n} \\
                   a_{2 1} & a_{2 2} & \cdots & a_{2 n} \\
                   \vdots  & \vdots  & \ddots & \vdots  \\
                   a_{m 1} & a_{m 2} & \cdots & a_{m n}
                 \end{pmatrix} = \zm                                \\
    \implies & a_{i j} = 0,                                          &  & \by{1.2.8}
  \end{align*}
  by \cref{1.5.3} we know that \(\set{E^{i j} : 1 \leq i \leq m, 1 \leq j \leq n}\) is linearly independent.
\end{proof}

\setcounter{ex}{7}
\begin{ex}\label{ex:1.5.8}
  Let \(S = \set{(1, 1, 0), (1, 0, 1), (0, 1, 1)}\) be a subset of the vector space \(\vs{F}^3\) over \(\F\).
  \begin{enumerate}
    \item Prove that if \(\F = \R\), then \(S\) is linearly independent.
    \item Prove that if \(\F\) has characteristic \(2\), then \(S\) is linearly dependent.
  \end{enumerate}
\end{ex}

\begin{proof}[\pf{ex:1.5.8}(a)]
  Let \(\seq{a}{1,2,3} \in \R\).
  Since
  \begin{align*}
             & a_1 (1, 1, 0) + a_2 (1, 0, 1) + a_3 (0, 1, 1) = (0, 0, 0) \\
    \implies & \begin{dcases}
                 a_1 1 + a_2 1 + a_3 0 = 0 \\
                 a_1 1 + a_2 0 + a_3 1 = 0 \\
                 a_1 0 + a_2 1 + a_3 1 = 0
               \end{dcases}                              &  & \by{1.2.4} \\
    \implies & \seq[=]{a}{1,2,3} = 0,
  \end{align*}
  by \cref{1.5.3} we know that \(S\) is linearly independent.
\end{proof}

\begin{proof}[\pf{ex:1.5.8}(b)]
  Observe that
  \begin{align*}
    (1, 1, 0) + (1, 0, 1) + (0, 1, 1) & = (1 + 1, 1 + 0, 0 + 1) + (0, 1, 1) \\
                                      & = (0, 1, 1) + (0, 1, 1)             \\
                                      & = (0 + 0, 1 + 1, 1 + 1)             \\
                                      & = (0, 0, 0).
  \end{align*}
  Thus by \cref{1.5.1} \(S\) is linearly dependent.
\end{proof}

\begin{ex}\label{ex:1.5.9}
  Let \(u\) and \(v\) be distinct vectors in a vector space \(\V\) over \(\F\).
  Show that \(\set{u, v}\) is linearly dependent iff \(u\) or \(v\) is a multiple of the other.
\end{ex}

\begin{proof}[\pf{ex:1.5.9}]
  We have
  \begin{align*}
         & \set{u, v} \text{ is linearly dependent}                        \\
    \iff & \exists a, b \in \F : \begin{dcases}
                                   au + bv = \zv \\
                                   \lnot (a = b = 0)
                                 \end{dcases}         &  & \by{1.5.1}      \\
    \iff & \exists a, b \in \F : \begin{dcases}
                                   au = -bv \\
                                   \lnot (a = b = 0)
                                 \end{dcases}                          \\
    \iff & \exists a, b \in \F : \begin{dcases}
                                   u = -\dfrac{b}{a} v & \text{if } a \neq 0 \\
                                   v = -\dfrac{a}{b} u & \text{if } b \neq 0
                                 \end{dcases} \\
    \iff & \exists c \in \F : u = cv.
  \end{align*}
\end{proof}

\setcounter{ex}{10}
\begin{ex}\label{ex:1.5.11}
  Let \(S = \set{\seq{u}{1,,n}}\) be a linearly independent subset of a vector space \(\V\) over the field \(\Z_2\).
  How many vectors are there in \(\spn{S}\)?
  Justify your answer.
\end{ex}

\begin{proof}[\pf{ex:1.5.11}]
  We have
  \[
    \forall v \in \spn{S}, \exists \seq{a}{1,,n} \in \Z_2 : \seq[+]{a,u}{1,,n} = v.
  \]
  Since \(a_i \in \Z_2\) for all \(i \in \set{1, \dots, n}\), \(a_i\) has only two choices, which are \(0\) or \(1\).
  Since there are \(n\) variables with \(2\) choices for each, there are \(2^n\) vectors in \(\spn{S}\).
\end{proof}

\setcounter{ex}{12}
\begin{ex}\label{ex:1.5.13}
  Let \(\V\) be a vector space over a field of characteristic not equal to two.
  \begin{enumerate}
    \item Let \(u\) and \(v\) be distinct vectors in \(\V\).
          Prove that \(\set{u,v}\) is linearly independent iff \(\set{u + v, u - v}\) is linearly independent.
    \item Let \(u, v\), and \(w\) be distinct vectors in \(\V\).
          Prove that \(\set{u, v, w}\) is linearly independent iff \(\set{u + v, u + w, v + w}\) is linearly independent.
  \end{enumerate}
\end{ex}

\begin{proof}[\pf{ex:1.5.13}(a)]
  We have
  \begin{align*}
         & \set{u, v} \text{ is linearly independent}                          \\
    \iff & \forall a, b \in \F,                                                \\
         & [au + bv = \zv \implies a = b = 0]                  &  & \by{1.5.3} \\
    \iff & \forall a, b \in \F,                                                \\
         & [a(u + v) + b(u - v) = (a + b) u + (a - b) v                        \\
         & = \zv \implies a + b = a - b = 0]                                   \\
    \iff & \set{u + v, u - v} \text{ is linearly independent}. &  & \by{1.5.3}
  \end{align*}
\end{proof}

\begin{proof}[\pf{ex:1.5.13}(b)]
  We have
  \begin{align*}
         & \set{u, v, w} \text{ is linearly independent}                              \\
    \iff & \forall a, b, c \in \F,                                                    \\
         & [au + bv + cw = \zv \implies a = b = c = 0]                &  & \by{1.5.3} \\
    \iff & \forall a, b \in \F,                                                       \\
         & [a(u + v) + b(u + w) + c(v + w)                                            \\
         & = (a + b) u + (a + c) v + (b + c) w                                        \\
         & = \zv \implies a + b = a + c = b + c = 0]                                  \\
    \iff & \set{u + v, u + w, v + w} \text{ is linearly independent}. &  & \by{1.5.3}
  \end{align*}
\end{proof}

\begin{ex}\label{ex:1.5.14}
  Prove that a set \(S\) is linearly dependent iff \(S = \set{\zv}\) or there exist distinct vectors \(v, \seq{u}{1,,n}\) in \(S\) such that \(v\) is a linear combination of \(\seq{u}{1,,n}\).
\end{ex}

\begin{proof}[\pf{ex:1.5.14}]
  By \cref{1.5.4}(a) we know that \(\varnothing\) is linearly independent, so \(S\) has at least one element.
  First suppose that \(S\) has only one element.
  Then by \cref{1.5.4}(b) we know that \(S\) is linearly dependent iff \(S = \set{\zv}\).

  Now suppose that \(S\) has more than one element.
  Then we have
  \begin{align*}
         & S \text{ is linearly dependent}                                                      \\
    \iff & \begin{dcases}
             \exists v \in S                               \\
             \exists \seq{u}{1,,n} \in S \setminus \set{v} \\
             \exists a_0 \in \F \setminus \set{0}          \\
             \exists \seq{a}{1,,n} \in \F
           \end{dcases} :                                        \\
         & a_0 v + \seq[+]{a,u}{1,,n} = \zv                                     &  & \by{1.5.1} \\
    \iff & \begin{dcases}
             \exists v \in S                               \\
             \exists \seq{u}{1,,n} \in S \setminus \set{v} \\
             \exists a_0 \in \F \setminus \set{0}          \\
             \exists \seq{a}{1,,n} \in \F
           \end{dcases} :                                        \\
         & v = -a_0^{-1} a_1 u_1 - a_0^{-1} a_2 u_2 - \cdots - a_0^{-1} a_n u_n                 \\
    \iff & \exists v, \seq{u}{1,,n} \in S :                                                     \\
         & v \text{ is a linear combination of } \seq{u}{1,,n}.                 &  & \by{1.4.1}
  \end{align*}
  Combine all proofs above we conclude that \(S\) is linearly dependent iff \(S = \set{\zv}\) or there exist distinct vectors \(v, \seq{u}{1,,n}\) in \(S\) such that \(v\) is a linear combination of \(\seq{u}{1,,n}\).
\end{proof}

\begin{ex}\label{ex:1.5.15}
  Let \(S = \set{\seq{u}{1,,n}}\) be a finite set of vectors.
  Prove that \(S\) is linearly dependent iff \(u_1 = \zv\) or \(u_{k + 1} \in \spn{\set{\seq{u}{1,,k}}}\) for some \(k\) (\(1 \leq k < n\)).
\end{ex}

\begin{proof}[\pf{ex:1.5.15}]
  By \cref{1.5.1} and \cref{1.5.2} we know that if \(u_1 = \zv\) or \(u_{k + 1} \in \spn{\set{\seq{u}{1,,k}}}\) for some \(k\) (\(1 \leq k < n\)), then \(S\) is linearly dependent.
  So we only need to show the converse is also true.

  Let \(S = \set{\seq{u}{1,,n}}\) be linearly dependent.
  Suppose for sake of contradiction that \(u_1 \neq \zv\) and
  \[
    \forall k \in \set{1, \dots, n - 1}, u_{k + 1} \notin \spn{\set{\seq{u}{1,,k}}}.
  \]
  By \cref{1.4.2} we know that \(u_k \neq \zv\) for all \(k \in \set{1, \dots, n}\).
  But then we have
  \begin{align*}
             & u_n \notin \spn{\set{\seq{u}{1,,n - 1}}}                                \\
    \implies & \forall \seq{a}{1,,n - 1} \in \F,                                       \\
             & \seq[+]{a,u}{1,,n - 1} \neq u_n                         &  & \by{1.4.3} \\
    \implies & \begin{dcases}
                 \forall \seq{a}{1,,n - 1} \in \F \\
                 \forall a_n \in \F \setminus \set{0}
               \end{dcases},                                     \\
             & \seq[+]{a,u}{1,,n - 1} \neq a_n u_n                                     \\
    \implies & \begin{dcases}
                 \forall \seq{a}{1,,n - 1} \in \F \\
                 \forall a_n \in \F \setminus \set{0}
               \end{dcases},                                     \\
             & \seq[+]{a,u}{1,,n} \neq \zv                                             \\
    \implies & \forall \seq{a}{1,,n} \in \F, [\seq[+]{a,u}{1,,n} = \zv                 \\
             & \iff \seq[=]{a}{1,,n} = 0]                                              \\
    \implies & S \text{ is linearly independent},                      &  & \by{1.5.3}
  \end{align*}
  a contradiction.
  Thus the converse must be true.
\end{proof}

\begin{ex}\label{ex:1.5.16}
  Prove that a set \(S\) of vectors is linearly independent iff each finite subset of \(S\) is linearly independent.
\end{ex}

\begin{proof}[\pf{ex:1.5.16}]
  First suppose that \(S\) is linearly independent.
  Then by \cref{1.5.6} we know that every finite subset of \(S\) is linearly independent.

  Now suppose that every finite subset of \(S\) is linearly independent.
  Suppose for sake of contradiction that \(S\) is linearly dependent.
  Then by \cref{ex:1.5.14} we know that \(S = \set{\zv}\) or there exist \(v, \seq{u}{1,,n} \in S\) such that \(v\) is a linear combination of \(\seq{u}{1,,n}\).
  Clearly \(S \neq \set{\zv}\) since \(\set{\zv}\) is a finite subset of \(S\) but \(\set{\zv}\) is linearly dependent.
  So we must have \(v\) as a linear combination of \(\seq{u}{1,,n}\).
  Since the set \(\set{v,\seq{u}{1,,n}}\) is finite, by hypothese it is linearly independent.
  But this contradicts to the fact that \(v\) is a linear combination of \(\seq{u}{1,,n}\).
  Thus \(S\) is linearly independent.
\end{proof}

\begin{ex}\label{ex:1.5.17}
  Let \(M \in \ms[n][n][\F]\) be a square upper triangular matrix (as defined in \cref{ex:1.3.12}) with nonzero diagonal entries.
  Prove that the columns of \(M\) are linearly independent.
\end{ex}

\begin{proof}[\pf{ex:1.5.17}]
  The columns of \(M\) are vectors in \(\vs{F}^n\), thus we can write the \(i\)th column (\(1 \leq i \leq n\)) of \(M\) as
  \[
    \begin{pmatrix}
      M_{1 i} \\
      M_{2 i} \\
      \vdots  \\
      M_{n i}
    \end{pmatrix} = M_{1 i} \cdot e_1 + M_{2 i} \cdot e_2 + \cdots + M_{n i} \cdot e_n = \sum_{j = 1}^n M_{j i} \cdot e_j,
  \]
  where \(\seq{e}{1,,n}\) are defined as in \cref{ex:1.5.4}.
  We now show that the set of column vectors of \(M\) is linearly independent.
  Let \(\seq{a}{1,,n} \in \F\).
  Since
  \begin{align*}
             & \sum_{i = 1}^n \br{a_i \cdot \pa{\sum_{j = 1}^n M_{j i} \cdot e_j}} = \zv                                                        \\
    \implies & \sum_{i = 1}^n \pa{\sum_{j = 1}^n a_i \cdot M_{j i} \cdot e_j} = \zv                                                             \\
    \implies & \sum_{j = 1}^n \pa{\sum_{i = 1}^n a_i \cdot M_{j i} \cdot e_j} = \zv      &  & \text{(by \ref{vs1} and \ref{vs2})}               \\
    \implies & \sum_{j = 1}^n \pa{\sum_{i = 1}^n a_i \cdot M_{j i}} \cdot e_j = \zv                                                             \\
    \implies & \forall j \in \set{1, \dots, n}, \sum_{i = 1}^n a_i \cdot M_{j i} = 0     &  & \by{ex:1.5.4}                                     \\
    \implies & \forall j \in \set{1, \dots, n}, \sum_{i = j}^n a_i \cdot M_{j i} = 0     &  & \by{ex:1.3.12}                                    \\
    \implies & \seq[=]{a}{1,,n} = 0,                                                     &  & (\forall j \in \set{1, \dots, n}, M_{j j} \neq 0)
  \end{align*}
  by \cref{1.5.3} we know that the column vectors of \(M\) is linearly independent.
\end{proof}

\begin{ex}\label{ex:1.5.18}
  Let \(S\) be a set of nonzero polynomials in \(\ps{\F}\) such that no two have the same degree.
  Prove that \(S\) is linearly independent.
\end{ex}

\begin{proof}[\pf{ex:1.5.18}]
  Suppose for sake of contradiction that \(S\) is linearly dependent.
  Then by \cref{ex:1.5.14} we know that
  \[
    \begin{dcases}
      \exists g, \seq{f}{1,,n} \in S \\
      \exists \seq{a}{1,,n} \in \F
    \end{dcases} : \begin{dcases}
      g = \seq[+]{a,f}{1,,n} \\
      \lnot (\seq[=]{a}{1,,n} = 0)
    \end{dcases}.
  \]
  Let \(\deg(h)\) be the degree of any function \(h \in \ps{\F}\) and let \(m = \deg(g)\).
  Then by \cref{1.2.11} we have
  \[
    \exists \seq{c}{0,1,,m} \in \F : \begin{dcases}
      g(x) = \sum_{j = 0}^m c_j x^j \\
      c_m \neq 0
    \end{dcases}.
  \]
  By hypothese we know that
  \[
    \forall i \in \set{1, \dots, n}, \deg(f_i) \neq m.
  \]
  But by \cref{ex:1.5.5} we have
  \[
    c_m x^m = \sum_{i = 1}^n a_i \cdot (0x^m) = 0 \implies c_m = 0,
  \]
  a contradiction.
  Thus \(S\) is linearly independent.
\end{proof}

\begin{ex}\label{ex:1.5.19}
  Prove that if \(\set{\seq{A}{1,,k}}\) is a linearly independent subset of \(\ms[n][n][\F]\), then \(\set{\tp{A}_1, \tp{A}_2, \dots, \tp{A}_k}\) is also linearly independent.
\end{ex}

\begin{proof}[\pf{ex:1.5.19}]
  Let \(i, j \in \set{1, \dots, n}\).
  Then we have
  \begin{align*}
             & \set{\seq{A}{1,,k}} \text{ is linearly independent}                                                         \\
    \implies & \br{\forall \seq{c}{1,,k} \in \F, \sum_{p = 1}^k c_p A_p = \zm \implies c_p = 0}            &  & \by{1.5.3} \\
    \implies & \br{\forall \seq{c}{1,,k} \in \F, \sum_{p = 1}^k c_p (A_p)_{i j} = 0 \implies c_p = 0}      &  & \by{1.2.9} \\
    \implies & \br{\forall \seq{c}{1,,k} \in \F, \sum_{p = 1}^k c_p \tp{(A_p)}_{j i} = 0 \implies c_p = 0} &  & \by{1.3.3} \\
    \implies & \br{\forall \seq{c}{1,,k} \in \F, \sum_{p = 1}^k c_p \tp{(A_p)} = \zm \implies c_p = 0}     &  & \by{1.2.9} \\
    \implies & \set{\tp{A}_1, \tp{A}_2, \dots, \tp{A}_k} \text{ is linearly independent}.                  &  & \by{1.5.3}
  \end{align*}
\end{proof}

\begin{ex}\label{ex:1.5.20}
  Let \(f, g \in \fs(\R, \R)\) be the functions defined by \(f(t) = e^{rt}\) and \(g(t) = e^{st}\), where \(r \neq s\).
  Prove that \(f\) and \(g\) are linearly independent in \(\fs(\R, \R)\).
\end{ex}

\begin{proof}[\pf{ex:1.5.20}]
  Suppose for sake of contradiction that \(f, g\) are linearly dependent.
  Then by \cref{1.5.1} we have
  \[
    \exists a, b \in \R : \begin{dcases}
      af + bg = \zv \\
      \lnot (a = b = 0)
    \end{dcases}.
  \]
  But this means
  \begin{align*}
             & \forall t \in \R, ae^{rt} + be^{st} = 0                                                         \\
    \implies & \forall t \in \R, a = -b e^{(s - r) t}                                                          \\
    \implies & a = b = 0,                              &  & \text{(\(t \mapsto e^{(s - r) t}\) is one-to-one)}
  \end{align*}
  a contradiction.
  Thus \(f, g\) are linearly independent.
\end{proof}

\section{Bases and Dimension}\label{sec:1.6}

\begin{defn}\label{1.6.1}
  A \textbf{basis} \(\beta\) for a vector space \(\V\) over \(\F\) is a linearly independent subset of \(\V\) that generates \(\V\).
  If \(\beta\) is a basis for \(\V\), we also say that the vectors of \(\beta\) form a basis for \(\V\).
\end{defn}

\begin{eg}\label{1.6.2}
  Recalling that \(\spn{\varnothing} = \set{\zv}\) and \(\varnothing\) is linearly independent, we see that \(\varnothing\) is a basis for the zero vector space.
\end{eg}

\begin{eg}\label{1.6.3}
  In \(\vs{F}^n\), let \(e_1 = (1, 0, 0, \dots, 0)\), \(e_2 = (0, 1, 0, \dots, 0)\), \dots, \(e_n = (0, 0, \dots, 0, 1)\);
  \(\set{\seq{e}{1,2,,n}}\) is readily seen to be a basis for \(\vs{F}^n\) and is called the \textbf{standard basis} for \(\vs{F}^n\).
\end{eg}

\begin{proof}[\pf{1.6.3}]
  By \cref{ex:1.5.4} we know that \(\set{\seq{e}{1,2,,n}}\) is linearly independent.
  By \cref{ex:1.4.7} we know that \(\vs{F}^n = \spn{\set{\seq{e}{1,2,,n}}}\).
  Thus by \cref{1.6.1} \(\set{\seq{e}{1,2,,n}}\) is a basis for \(\vs{F}^n\) over \(\F\).
\end{proof}

\begin{eg}\label{1.6.4}
  In \(\MS\), let \(E^{i j}\) denote the matrix whose only nonzero entry is a \(1\) in the \(i\)th row and \(j\)th column.
  Then \(\set{E^{i j} : 1 \leq i \leq m, 1 \leq j \leq n}\) is a basis for \(\MS\).
\end{eg}

\begin{proof}[\pf{1.6.4}]
  By \cref{ex:1.5.6} we know that \(\set{E^{i j} : 1 \leq i \leq m, 1 \leq j \leq n}\) is linearly independent.
  Since
  \begin{align*}
             & \forall A \in \MS, A = \begin{pmatrix}
      A_{1 1} & A_{1 2} & \cdots & A_{1 n} \\
      A_{2 1} & A_{2 2} & \cdots & A_{2 n} \\
      \vdots  & \vdots  & \ddots & \vdots  \\
      A_{m 1} & A_{m 2} & \cdots & A_{m n}
    \end{pmatrix}                                                              \\
             & = \sum_{i = 1}^m \sum_{j = 1}^n A_{i j} E^{i j}                                 &  & \text{(by \cref{1.2.9})} \\
    \implies & \forall A \in \MS, A \in \spn{\set{E^{i j} : 1 \leq i \leq m, 1 \leq j \leq n}} &  & \text{(by \cref{1.4.3})} \\
    \implies & \MS = \spn{\set{E^{i j} : 1 \leq i \leq m, 1 \leq j \leq n}},                   &  & \text{(by \cref{1.5})}
  \end{align*}
  by \cref{1.6.1} we know that \(\set{E^{i j} : 1 \leq i \leq m, 1 \leq j \leq n}\) is a basis for \(\MS\) over \(\F\).
\end{proof}

\begin{eg}\label{1.6.5}
  In \(\ps[n]{\F}\) the set \(\set{1, x, x^2, \dots, x^n}\) is a basis.
  We call this basis the \textbf{standard basis} for \(\ps[n]{\F}\).
\end{eg}

\begin{proof}[\pf{1.6.5}]
  By \cref{ex:1.5.5} we know that \(\set{1, x, x^2, \dots, x^n}\) is linearly independent.
  By \cref{ex:1.4.8} we know that \(\ps[n]{\F} = \spn{\set{1, x, x^2, \dots, x^n}}\).
  Thus by \cref{1.6.1} \(\set{1, x, x^2, \dots, x^n}\) is a basis for \(\ps[n]{\F}\) over \(\F\).
\end{proof}

\begin{eg}\label{1.6.6}
  In \(\ps{\F}\) the set \(\set{1, x, x^2, \dots}\) is a basis.
\end{eg}

\begin{proof}[\pf{1.6.6}]
  Suppose for sake of contradiction that \(\set{1, x, x^2, \dots}\) is not a basis for \(\ps{\F}\).
  Then by \cref{1.6.1} we can split into two cases:
  \begin{itemize}
    \item If \(\set{1, x, x^2, \dots}\) is linearly dependent, then by \cref{ex:1.5.14} we have
          \[
            \begin{dcases}
              \exists x^n \in \set{1, x, x^2, \dots} \\
              \exists \set{\seq{a}{0,1,2,}} \subseteq \F
            \end{dcases} : x^n = \sum_{i \in \N : i \neq n} a_i x^i.
          \]
          By setting \(a_n = -1\) we have
          \begin{align*}
                     & \sum_{i \in \N} a_i x^i = \zv = \sum_{i \in \N} 0x^i                                \\
            \implies & \seq[=]{a}{0,1,2,} = 0.                              &  & \text{(by \cref{1.2.11})}
          \end{align*}
          But this means \(a_n = 0\), a contradiction.
    \item If \(\ps{\F} \neq \spn{\set{1, x, x^2, \dots}}\), then by \cref{1.4.3} we have
          \[
            \exists f \in \ps{\F} : \forall \set{\seq{a}{0,1,2,}} \subseteq \F, f(x) \neq \sum_{i \in \N} a_i x^i.
          \]
          Let \(m\) be the degree of \(f\).
          Then by \cref{1.2.11} we have
          \[
            \exists \seq{c}{0,1,,m} \in \F : f(x) = c_0 + c_1 x + \cdots + c_m x^m.
          \]
          But by setting
          \[
            \begin{dcases}
              a_i = c_i & \text{if } i \leq m \\
              a_i = 0   & \text{if } i > m
            \end{dcases}
          \]
          we have \(f(x) = \sum_{i \in \N} a_i x^i\), a contradiction.
  \end{itemize}
  From all cases above we derived contradictions.
  Thus \(\set{1, x, x^2, \dots}\) is a basis for \(\ps{\F}\).
\end{proof}

\begin{note}
  Observe that \cref{1.6.6} shows that a basis need not be finite.
  In fact, later in \cref{sec:1.6} it is shown that no basis for \(\ps{\F}\) can be finite.
  Hence not every vector space has a finite basis.
\end{note}

\begin{thm}\label{1.8}
  Let \(\V\) be a vector space over \(\F\) and \(\beta = \set{\seq{u}{1,2,,n}}\) be a subset of \(\V\).
  Then \(\beta\) is a basis for \(\V\) if and only if each \(v \in \V\) can be uniquely expressed as a linear combination of vectors of \(\beta\), that is, can be expressed in the form
  \[
    v = \seq[+]{a,u}{1,2,,n}
  \]
  for unique scalars \(\seq{a}{1,2,,n} \in \F\).
\end{thm}

\begin{proof}[\pf{1.8}]
  First suppose that \(\beta\) be a basis for \(\V\).
  If \(v \in V\), then \(v \in \spn{\beta}\) because \(\spn{\beta} = \V\).
  Thus \(v\) is a linear combination of the vectors of \(\beta\).
  Suppose that
  \[
    v = \seq[+]{a,u}{1,2,,n} \quad \text{and} \quad v = \seq[+]{b,u}{1,2,,n}
  \]
  are two such representations of \(v\).
  Subtracting the second equation from the first gives
  \[
    \zv = (a_1 - b_1) u_1 + (a_2 - b_2) u_2 + \cdots + (a_n - b_n) u_n.
  \]
  Since \(\beta\) is linearly independent, it follows that \(a_1 - b_1 = a_2 - b_2 = \cdots = a_n - b_n = 0\).
  Hence \(a_1 = b_1, a_2 = b_2, \dots, a_n = b_n\), and so \(v\) is uniquely expressible as a linear combination of the vectors of \(\beta\).

  Now suppose that each \(v \in \V\) can be uniquely expressed as a linear combination of vectors of \(\beta\).
  By \cref{1.4.3} and \cref{1.5} this means \(\V = \spn{\beta}\).
  Thus to show that \(\beta\) is a basis for \(\V\), by \cref{1.6.1} we need to show that \(\beta\) is linearly independent.
  This is true since
  \begin{align*}
             & \zv \in \V                                             &  & \text{(by \ref{vs3})}    \\
    \implies & \exists! \seq{a}{1,2,,n} \in \F :                                                    \\
             & \zv = \seq[+]{a,u}{1,2,,n} = \seq[+]{0u}{1,2,,n}       &  & \text{(by hypothesis)}   \\
    \implies & \seq[=]{a}{1,2,,n} = 0                                                               \\
    \implies & \set{\seq{u}{1,2,,n}} \text{ is linearly independent}. &  & \text{(by \cref{1.5.3})}
  \end{align*}
\end{proof}

\begin{note}
  \cref{1.8} shows that if the vectors \(\seq{u}{1,2,,n}\) form a basis for a vector space \(\V\), then every vector in \(\V\) can be uniquely expressed in the form
  \[
    v = \seq[+]{a,u}{1,2,,n}
  \]
  for appropriately chosen scalars \(\seq{a}{1,2,,n}\).
  Thus \(v\) determines a unique \(n\)-tuple of scalars \(\tuple{a}{1,2,,n}\) and, conversely, each \(n\)-tuple of scalars determines a unique vector \(v \in \V\) by using the entries of the \(n\)-tuple as the coefficients of a linear combination of \(\seq{u}{1,2,,n}\).
  This fact suggests that \(\V\) is like the vector space \(\vs{F}^n\), where \(n\) is the number of vectors in the basis for \(\V\).
  We see in \cref{sec:2.4} that this is indeed the case.
\end{note}

\begin{thm}\label{1.9}
  If a vector space \(\V\) over \(\F\) is generated by a finite set \(S\), then some subset of \(S\) is a basis for \(\V\).
  Hence \(\V\) has a finite basis.
\end{thm}

\begin{proof}[\pf{1.9}]
  If \(S = \varnothing\) or \(S = \set{\zv}\), then \(\V = \set{\zv}\) and \(\varnothing\) is a subset of \(S\) that is a basis for \(\V\).
  Otherwise \(S\) contains a nonzero vector \(u_1\).
  By \cref{1.5.4}(b), \(\set{u_1}\) is a linearly independent set.
  Continue, if possible, choosing vectors \(\seq{u}{2,,k}\) in \(S\) such that \(\set{\seq{u}{1,2,,k}}\) is linearly independent.
  Since \(S\) is a finite set, we must eventually reach a stage at which \(\beta = \set{\seq{u}{1,2,,k}}\) is a linearly independent subset of \(S\), but adjoining to \(\beta\) any vector in \(S \setminus \beta\) produces a linearly dependent set.
  We claim that \(\beta\) is a basis for \(\V\).
  Because \(\beta\) is linearly independent by construction, it suffices to show that \(\beta\) spans \(\V\).
  By \cref{1.5} we need to show that \(S \subseteq \spn{\beta}\).
  Let \(v \in S\).
  If \(v \in \beta\), then clearly \(v \in \spn{\beta}\).
  Otherwise, if \(v \notin \beta\), then the preceding construction shows that \(\beta \cup \set{v}\) is linearly dependent.
  So \(v \in \spn{\beta}\) by \cref{1.7}.
  Thus \(S \subseteq \spn{\beta}\).
\end{proof}

\begin{note}
  Because of the method by which the basis \(\beta\) was obtained in the proof of \cref{1.9}, this theorem is often remembered as saying that \emph{a finite spanning set for \(\V\) can be reduced to a basis for \(\V\).}
\end{note}

\begin{thm}[Replacement Theorem]\label{1.10}
  Let \(\V\) be a vector space over \(\F\) that is generated by a set \(G\) containing exactly \(n\) vectors, and let \(L\) be a linearly independent subset of \(\V\) containing exactly \(m\) vectors.
  Then \(m \leq n\) and there exists a subset \(H\) of \(G\) containing exactly \(n - m\) vectors such that \(L \cup H\) generates \(\V\).
\end{thm}

\begin{proof}[\pf{1.10}]
  The proof is by mathematical induction on \(m\).
  The induction begins with \(m = 0\);
  for in this case \(L = \varnothing\), and so taking \(H = G\) gives the desired result.

  Now suppose that the theorem is true for some integer \(m \geq 0\).
  We prove that the theorem is true for \(m + 1\).
  Let \(L = \set{\seq{v}{1,2,,m + 1}}\) be a linearly independent subset of \(\V\) consisting of \(m + 1\) vectors.
  By \cref{1.5.6} \(\set{\seq{v}{1,2,,m}} \subseteq L\) is linearly independent, and so we may apply the induction hypothesis to conclude that \(m \leq n\) and that there is a subset \(\set{\seq{u}{1,2,,n - m}}\) of \(G\) such that \(\set{\seq{v}{1,2,,m}} \cup \set{\seq{u}{1,2,,n - m}}\) generates \(\V\).
  Thus there exist scalars \(\seq{a}{1,2,,m}, \seq{b}{1,2,,n - m} \in \F\) such that
  \begin{equation}\label{eq:1.6.1}
    \seq[+]{a,v}{1,2,,m} + \seq[+]{b,u}{1,2,,n - m} = v_{m + 1}
  \end{equation}
  Note that \(n - m > 0\), otherwise \(n - m = 0\) implies \(v_{m + 1}\) is a linear combination of \(\seq{v}{1,2,,m}\), which by \cref{1.7} contradicts the assumption that \(L\) is linearly independent.
  Hence \(n > m\);
  that is, \(n \geq m + 1\).
  Moreover, some \(b_i\), say \(b_1\), is nonzero, for otherwise we obtain the same contradiction.
  Solving \cref{eq:1.6.1} for \(u_1\) gives
  \begin{multline*}
    u_1 = (-b_1^{-1} a_1) v_1 + (-b_1^{-1} a_2) v_2 + \cdots + (-b_1^{-1} a_m) v_m + (b_1^{-1}) v_{m + 1} \\
    + (-b_1^{-1} b_2) u_2 + \cdots + (-b_1^{-1} b_{n - m}) u_{n - m}.
  \end{multline*}
  Let \(H = \set{\seq{u}{2,,n - m}}\).
  Then \(u_1 \in \spn{L \cup H}\), and because \(\seq{v}{1,2,,m}\), \\
  \(\seq{u}{2,,n - m}\) are clearly in \(\spn{L \cup H}\), it follows that
  \[
    \set{\seq{v}{1,2,,m}, \seq{u}{1,2,,n - m}} \subseteq \spn{L \cup H}.
  \]
  Because \(\set{\seq{v}{1,2,,m}, \seq{u}{1,2,n - m}}\) generates \(\V\), \cref{1.5} implies that \\
  \(\spn{L \cup H} = \V\).
  Since \(H\) is a subset of \(G\) that contains \((n - m) - 1 = n - (m + 1)\) vectors, the theorem is true for \(m + 1\).
  This completes the induction.
\end{proof}

\begin{cor}\label{1.6.7}
  Let \(\V\) be a vector space over \(\F\) having a finite basis.
  Then every basis for \(\V\) contains the same number of vectors.
\end{cor}

\begin{proof}[\pf{1.6.7}]
  Suppose that \(\beta\) is a finite basis for \(\V\) that contains exactly \(n\) vectors, and let \(\gamma\) be any other basis for \(\V\).
  If \(\gamma\) contains more than \(n\) vectors, then we can select a subset \(S\) of \(\gamma\) containing exactly \(n + 1\) vectors.
  Since \(S\) is linearly independent and \(\beta\) generates \(\V\), the replacement theorem (\cref{1.10}) implies that \(n + 1 \leq n\), a contradiction.
  Therefore \(\gamma\) is finite, and the number \(m\) of vectors in \(\gamma\) satisfies \(m \leq n\).
  Reversing the roles of \(\beta\) and \(\gamma\) and arguing as above, we obtain \(n \leq m\).
  Hence \(m = n\).
\end{proof}

\begin{note}
  If a vector space has a finite basis, \cref{1.6.7} asserts that the number of vectors in \emph{any} basis for \(\V\) is an intrinsic property of \(\V\).
\end{note}

\begin{defn}\label{1.6.8}
  A vector space is called \textbf{finite-dimensional} if it has a basis consisting of a finite number of vectors.
  The unique number of vectors in each basis for \(\V\) is called the \textbf{dimension} of \(\V\) and is denoted by \(\dim(\V)\).
  A vector space that is not finite-dimensional is called \textbf{infinite-dimensional}.
\end{defn}

\begin{eg}\label{1.6.9}
  The vector space \(\set{\zv}\) has dimension zero.
\end{eg}

\begin{proof}[\pf{1.6.9}]
  By \cref{1.6.2} and \cref{1.6.7} we are done.
\end{proof}

\begin{eg}\label{1.6.10}
  The vector space \(\vs{F}^n\) has dimension \(n\).
\end{eg}

\begin{proof}[\pf{1.6.10}]
  By \cref{1.6.3} and \cref{1.6.7} we are done.
\end{proof}

\begin{eg}\label{1.6.11}
  The vector space \(\MS\) has dimension \(mn\).
\end{eg}

\begin{proof}[\pf{1.6.11}]
  By \cref{1.6.4} and \cref{1.6.7} we are done.
\end{proof}

\begin{eg}\label{1.6.12}
  The vector space \(\ps[n]{\F}\) has dimension \(n + 1\).
\end{eg}

\begin{proof}[\pf{1.6.12}]
  By \cref{1.6.5} and \cref{1.6.7} we are done.
\end{proof}

\begin{eg}\label{1.6.13}
  Over the field of complex numbers, the vector space of complex numbers has dimension \(1\).
  (A basis is \(\set{1}\).)
\end{eg}

\begin{proof}[\pf{1.6.13}]
  We have
  \begin{align*}
             & \forall c \in \C, c = c \cdot 1                               \\
    \implies & \spn{\set{1}} = \C              &  & \text{(by \cref{1.5})}   \\
    \implies & \#\pa{\set{1}} = 1 = \dim(\C).  &  & \text{(by \cref{1.6.8})}
  \end{align*}
\end{proof}

\begin{eg}\label{1.6.14}
  Over the field of real numbers, the vector space of complex numbers has dimension \(2\).
  (A basis is \(\set{1, i}\).)
\end{eg}

\begin{proof}[\pf{1.6.14}]
  We have
  \begin{align*}
             & \forall c \in \C, c = \Re(c) + i \Im(c) = \Re(c) \cdot 1 + \Im(c) \cdot i &  & (\Re(c), \Im(c) \in \R)  \\
    \implies & \spn{\set{1, i}} = \C                                                     &  & \text{(by \cref{1.5})}   \\
    \implies & \#\pa{\set{1, i}} = 2 = \dim(\C).                                         &  & \text{(by \cref{1.6.8})}
  \end{align*}
\end{proof}

\begin{note}
  From \cref{1.6.13} and \cref{1.6.14} we see that the dimension of a vector space depends on its field of scalars.
\end{note}

\begin{note}
  In the terminology of dimension, the first conclusion in the replacement theorem states that if \(\V\) is a finite-dimensional vector space over \(\F\), then no linearly independent subset of \(\V\) can contain more than \(\dim(\V)\) vectors.
  From this fact it follows that the vector space \(\ps{\F}\) over \(\F\) is infinite-dimensional because it has an infinite linearly independent set, namely \(\set{1, x, x^2, \dots}\).
  This set is, in fact, a basis for \(\ps{\F}\).
  Yet nothing that we have proved in this section guarantees an infinite-dimensional vector space must have a basis.
  In \cref{sec:1.7} it is shown, however, that \emph{every vector space has a basis}.
\end{note}

\begin{cor}\label{1.6.15}
  Let \(\V\) be a vector space over \(\F\) with dimension \(n\).
  \begin{enumerate}
    \item Any finite generating set for \(\V\) contains at least \(n\) vectors, and a generating set for \(\V\) that contains exactly \(n\) vectors is a basis for \(\V\).
    \item Any linearly independent subset of \(\V\) that contains exactly \(n\) vectors is a basis for \(\V\).
    \item Every linearly independent subset of \(\V\) can be extended to a basis for \(\V\).
  \end{enumerate}
\end{cor}

\begin{proof}[\pf{1.6.15}]
  Let \(\beta\) be a basis for \(\V\).
  \begin{enumerate}
    \item Let \(G\) be a finite generating set for \(\V\).
          By \cref{1.9} some subset \(H\) of \(G\) is a basis for \(\V\).
          \cref{1.6.7} implies that \(H\) contains exactly \(n\) vectors.
          Since a subset of \(G\) contains \(n\) vectors, \(G\) must contain at least \(n\) vectors.
          Moreover, if \(G\) contains exactly \(n\) vectors, then we must have \(H = G\), so that \(G\) is a basis for \(\V\).
    \item Let \(L\) be a linearly independent subset of \(\V\) containing exactly \(n\) vectors.
          It follows from the replacement theorem that there is a subset \(H\) of \(\beta\) containing \(n - n = 0\) vectors such that \(L \cup H\) generates \(\V\).
          Thus \(H = \varnothing\), and \(L\) generates \(\V\).
          Since \(L\) is also linearly independent, \(L\) is a basis for \(\V\).
    \item If \(L\) is a linearly independent subset of \(\V\) containing \(m\) vectors, then the replacement theorem asserts that there is a subset \(H\) of \(\beta\) containing exactly \(n - m\) vectors such that \(L \cup H\) generates \(\V\).
          Now \(L \cup H\) contains at most \(n\) vectors;
          therefore (a) implies that \(L \cup H\) contains exactly \(n\) vectors and that \(L \cup H\) is a basis for \(\V\).
  \end{enumerate}
\end{proof}

\begin{eg}\label{1.6.16}
  For \(k = 0, 1, \dots, n\), let \(p_k(x) = x^k + x^{k + 1} + \cdots + x^n\).
  It follows from \cref{1.5.5} and \cref{1.6.15}(b) that
  \[
    \set{p_0(x), p_1(x), \dots, p_n(x)}
  \]
  is a basis for \(\ps[n]{\F}\).
\end{eg}

\begin{thm}\label{1.11}
  Let \(\W\) be a subspace of a finite-dimensional vector space \(\V\) over \(\F\).
  Then \(\W\) is finite-dimensional and \(\dim(\W) \leq \dim(\V)\).
  Moreover, if \(\dim(\W) = \dim(\V)\), then \(\V = \W\).
\end{thm}

\begin{proof}[\pf{1.11}]
  Let \(\dim(\V) = n\).
  If \(\W = \set{\zv}\), then \(\W\) is finite-dimensional and \(\dim(\W) = 0 \leq n\).
  Otherwise, \(\W\) contains a nonzero vector \(x_1\);
  so \(\set{x_1}\) is a linearly independent set.
  Continue choosing vectors, \(\seq{x}{1,2,,k}\) in \(\W\) such that \(\set{\seq{x}{1,2,,k}}\) is linearly independent.
  Since no linearly independent subset of \(\V\) can contain more than \(n\) vectors, this process must stop at a stage where \(k \leq n\) and \(\set{\seq{x}{1,2,,k}}\) is linearly independent but adjoining any other vector from \(\W\) produces a linearly dependent set.
  \cref{1.7} implies that \(\set{\seq{x}{1,2,,k}}\) generates \(\W\), and hence it is a basis for \(\W\).
  Therefore \(\dim(\W) = k \leq n\).

  If \(\dim(\W) = n\), then a basis for \(\W\) is a linearly independent subset of \(\V\) containing \(n\) vectors.
  But \cref{1.6.15}(b) implies that this basis for \(\W\) is also a basis for \(\V\);
  so \(\W = \V\).
\end{proof}

\begin{eg}\label{1.6.17}
  The set of diagonal \(n \times n\) matrices is a subspace \(\W\) of \(\ms{n}{n}{\F}\)
  (see \cref{1.3.8}).
  A basis for \(\W\) is
  \[
    \set{E^{1 1}, E^{2 2}, \dots, E^{n n}},
  \]
  where \(E^{i j}\) is the matrix in which the only nonzero entry is a \(1\) in the \(i\)th row and \(j\)th column.
  Thus \(\dim(\W) = n\).
\end{eg}

\begin{proof}[\pf{1.6.17}]
  By \cref{ex:1.5.6} we know that \(\set{E^{1 1}, E^{2 2}, \dots, E^{n n}}\) is linearly independent.
  Since \(\W = \spn{\set{E^{1 1}, E^{2 2}, \dots, E^{n n}}}\), by \cref{1.6.15}(a) we know that \(\dim(\W) \leq n\).
  By \cref{1.6.15}(c) we also know that \(\dim(\W) \geq n\).
  Thus we have \(\dim(\W) = n\).
\end{proof}

\begin{eg}\label{1.6.18}
  The set of symmetric \(n \times n\) matrices is a subspace \(\W\) of \(\ms{n}{n}{\F}\) over \(\F\).
  A basis for \(\W\) is
  \[
    \set{A^{i j} : 1 \leq i \leq j \leq n}
  \]
  where \(A^{i j}\) is the \(n \times n\) matrix having \(1\) in the \(i\)th row and \(j\)th column, \(1\) in the \(j\)th row and \(i\)th column, and \(0\) elsewhere.
  It follows that
  \[
    \dim(\W) = n + (n - 1) + \cdots + 1 = \frac{1}{2} n(n + 1).
  \]
\end{eg}

\begin{proof}[\pf{1.6.18}]
  By \cref{ex:1.5.6} we see that each \(A^{i j}\) can only express as \(E^{i j} + E^{j i}\).
  Thus \(\set{A^{i j} : 1 \leq i \leq j \leq n}\) is linearly independent and by \cref{1.6.15}(c)we have  \(\dim(\W) \geq \#\pa{\set{A^{i j} : 1 \leq i \leq j \leq n}}\).
  Since
  \begin{align*}
             & \forall M \in \W, M_{i j} = M_{j i} = M_{i j} \cdot 1                                                 \\
    \implies & \forall M \in \W, M = \sum_{i = 1}^n \sum_{j = i}^n M_{i j} A^{i j} &  & \text{(by \cref{1.2.9})}     \\
    \implies & \W = \spn{\set{A^{i j} : 1 \leq i \leq j \leq n}}                   &  & \text{(by \cref{1.5})}       \\
    \implies & \dim(\W) \leq \#\pa{\set{A^{i j} : 1 \leq i \leq j \leq n}},        &  & \text{(by \cref{1.6.15}(a))}
  \end{align*}
  we have \(\dim(\W) = \#\pa{\set{A^{i j} : 1 \leq i \leq j \leq n}} = \frac{1}{2} n(n + 1)\).
\end{proof}

\begin{cor}\label{1.6.19}
  If \(\W\) is a subspace of a finite-dimensional vector space \(\V\) over \(\F\), then any basis for \(\W\) can be extended to a basis for \(\V\).
\end{cor}

\begin{proof}[\pf{1.6.19}]
  Let \(S\) be a basis for \(\W\).
  Because \(S\) is a linearly independent subset of \(\V\), \cref{1.6.15}(c) guarantees that \(S\) can be extended to a basis for \(\V\).
\end{proof}

\begin{defn}[The Lagrange Interpolation Formula]\label{1.6.20}
  Let \(\seq{c}{0,1,,n}\) be distinct scalars in an infinite field \(\F\).
  The polynomials \(f_0(x), f_1(x), \dots, f_n(x)\) defined by
  \[
    f_i(x) = \frac{(x - c_0) \cdots (x - c_{i - 1}) (x - c_{i + 1}) \cdots (x - c_n)}{(c_i - c_0) \cdots (c_i - c_{i - 1}) (c_i - c_{i + 1}) \cdots (c_i - c_n)} = \prod_{\substack{k = 0 \\ k \neq i}}^n \frac{x - c_k}{c_i - c_k}
  \]
  are called the \textbf{Lagrange polynomials} (associated with \(\seq{c}{0,1,,n}\)).
  Note that each \(f_i(x)\) is a polynomial of degree \(n\) and hence is in \(\ps[n]{\F}\).
  By regarding \(f_i(x)\) as a polynomial function \(f_i : \F \to \F\), we see that
  \begin{equation}\label{eq:1.6.2}
    f_i(c_j) = \begin{dcases}
      0 & \text{if } i \neq j \\
      1 & \text{if } i = j
    \end{dcases}.
  \end{equation}

  This property of Lagrange polynomials can be used to show that \(\beta = \set{\seq{f}{0,1,,n}}\) is a linearly independent subset of \(\ps[n]{\F}\).
  Suppose that
  \[
    \sum_{i = 0}^n a_i f_i = \zv \quad \text{for some scalars } \seq{a}{0,1,,n},
  \]
  where \(\zv\) denotes the zero function.
  Then
  \[
    \sum_{i = 0}^n a_i f_i(c_j) = 0 \quad \text{for } j = 0, 1, \dots, n.
  \]
  But also
  \[
    \sum_{i = 0}^n a_i f_i(c_j) = a_j
  \]
  by \cref{eq:1.6.2}.
  Hence \(a_j = 0\) for \(j = 0, 1, \dots, n\);
  so \(\beta\) is linearly independent.
  Since the dimension of \(\ps[n]{\F}\) is \(n + 1\), it follows from \cref{1.6.15} that \(\beta\) is a basis for \(\ps[n]{\F}\).

  Because \(\beta\) is a basis for \(\ps[n]{\F}\), every polynomial function \(g\) in \(\ps[n]{\F}\) is a linear combination of polynomial functions of \(\beta\), say,
  \[
    g = \sum_{i = 0}^n b_i f_i.
  \]
  It follows that
  \[
    g(c_j) = \sum_{i = 0}^n b_i f_i(c_j) = b_j;
  \]
  so
  \[
    g = \sum_{i = 0}^n g(c_i) f_i
  \]
  is the unique representation of \(g\) as a linear combination of elements of \(\beta\).
  This representation is called the \textbf{Lagrange interpolation formula}.
  Notice that the preceding argument shows that if \(\seq{b}{0,1,,n}\) are any \(n + 1\) scalars in \(\F\) (not necessarily distinct), then the polynomial function
  \[
    g = \sum_{i = 0}^n b_i f_i
  \]
  is the unique polynomial in \(\ps[n]{\F}\) such that \(g(c_j) = b_j\).
  Thus we have found the unique polynomial of degree not exceeding \(n\) that has specified values \(b_j\) at given points \(c_j\) in its domain (\(j = 0, 1, \dots, n\)).

  An important consequence of the Lagrange interpolation formula is the following result:
  If \(f \in \ps[n]{\F}\) and \(f(c_i) = 0\) for \(n + 1\) distinct scalars \(\seq{c}{0,1,,n}\) in \(\F\), then \(f\) is the zero function.
\end{defn}

\exercisesection

\setcounter{ex}{10}
\begin{ex}\label{ex:1.6.11}
  Let \(u\) and \(v\) be distinct vectors of a vector space \(\V\) over \(\F\).
  Show that if \(\set{u, v}\) is a basis for \(\V\) and \(a\) and \(b\) are nonzero scalars, then both \(\set{u + v, au}\) and \(\set{au, bv}\) are also bases for \(\V\).
\end{ex}

\begin{proof}[\pf{ex:1.6.11}]
  Let \(c_1, c_2 \in \F\).
  Since
  \begin{align*}
             & c_1 (u + v) + c_2 (au) = \zv                                \\
    \implies & (c_1 + c_2 a) u + c_1 v = \zv &  & \text{(by \cref{1.2.1})} \\
    \implies & \begin{dcases}
      c_1 + c_2 a = 0 \\
      c_1 = 0
    \end{dcases}    &  & \text{(by \cref{1.5.3})} \\
    \implies & \begin{dcases}
      c_2 a = 0 \\
      c_1 = 0
    \end{dcases}                                  \\
    \implies & c_1 = c_2 = 0                 &  & (a \neq 0)
  \end{align*}
  and
  \begin{align*}
             & c_1 (au) + c_2 (bv) = \zv                                 \\
    \implies & (c_1 a) u + (c_2 b) v = \zv &  & \text{(by \cref{1.2.1})} \\
    \implies & \begin{dcases}
      c_1 a = 0 \\
      c_2 b = 0
    \end{dcases}  &  & \text{(by \cref{1.5.3})} \\
    \implies & c_1 = c_2 = 0,              &  & (a \neq 0 \neq b)
  \end{align*}
  by \cref{1.5.3} we know that \(\set{u + v, au}\) and \(\set{au, bv}\) are linearly independent.
  Since
  \[
    \#(\set{u + v, au}) = \#(\set{au, bv}) = 2 = \#(\set{u, v}),
  \]
  by \cref{1.6.15}(a) we know that \(\set{u + v, au}\) and \(\set{au, bv}\) are basis for \(\V\).
\end{proof}

\begin{ex}\label{ex:1.6.12}
  Let \(u, v\), and \(w\) be distinct vectors of a vector space \(\V\) over \(\F\).
  Show that if \(\set{u, v, w}\) is a basis for \(\V\), then \(\set{u + v + w, v + w, w}\) is also a basis for \(\V\).
\end{ex}

\begin{proof}[\pf{ex:1.6.12}]
  Let \(a, b, c \in \F\).
  Since
  \begin{align*}
             & a(u + v + w) + b(v + w) + cw = \zv                               \\
    \implies & au + (a + b)v + (a + b + c)w = \zv &  & \text{(by \cref{1.2.1})} \\
    \implies & \begin{dcases}
      a = 0     \\
      a + b = 0 \\
      a + b + c = 0
    \end{dcases}         &  & \text{(by \cref{1.5.3})} \\
    \implies & a = b = c = 0,
  \end{align*}
  by \cref{1.5.3} we know that \(\set{u + v + w, v + w, w}\) is linearly independent.
  Since
  \[
    \#(\set{u + v + w, v + w, w}) = 3 = \#(\set{u, v, w}),
  \]
  by \cref{1.6.15}(a) we know that \(\set{u + v + w, v + w, w}\) is a basis for \(\V\).
\end{proof}

\setcounter{ex}{14}
\begin{ex}\label{ex:1.6.15}
  The set of all \(n \times n\) matrices having trace equal to zero is a subspace \(\W\) of \(\ms{n}{n}{\F}\) (see \cref{1.3.9}).
  Find a basis for \(\W\).
  What is the dimension of \(\W\)?
\end{ex}

\begin{proof}[\pf{ex:1.6.15}]
  Let \(E^{i j} \in \ms{n}{n}{\F}\) be matrix defined as in \cref{ex:1.5.6} and let \(\beta\) be the set
  \[
    \beta = \set{E^{i j} : i, j \in \set{1, \dots, n}, i \neq j} \cup \set{E^{i i} - E^{1 1} : 2 \leq i \leq n}.
  \]
  Observe that
  \begin{align*}
             & \forall A \in \W, \begin{dcases}
      \tr{A} = 0 \\
      A = \sum_{i = 1}^n \sum_{j = 1}^n A_{i j} E^{i j}
    \end{dcases}                   &  & \text{(by \cref{ex:1.5.6})} \\
    \implies & \forall A \in \W, \begin{dcases}
      A_{1 1} + A_{2 2} + \cdots + A_{n n} = 0 \\
      A = \sum_{i = 1}^n \sum_{j = 1}^n A_{i j} E^{i j}
    \end{dcases}                   &  & \text{(by \cref{1.3.9})}    \\
    \implies & \forall A \in \W, \begin{dcases}
      A_{2 2} + \cdots + A_{n n} = -A_{1 1}        \\
      A = \pa{\sum_{i = 1}^n \sum_{\substack{j = 1 \\ j \neq i}}^n A_{i j} E^{i j}} + \pa{\sum_{i = 1}^n A_{i i} E^{i i}}
    \end{dcases}                                                    \\
    \implies & \forall A \in \W, A = \pa{\sum_{i = 1}^n \sum_{\substack{j = 1                                  \\ j \neq i}}^n A_{i j} E^{i j}} + \pa{\sum_{i = 2}^n A_{i i} (E^{i i} - E^{1 1})} \\
    \implies & \forall A \in \W, A \in \spn{\beta}                            &  & \text{(by \cref{1.4.3})}    \\
    \implies & \W \subseteq \spn{\beta}.
  \end{align*}
  Since
  \[
    \tr{E^{i j}} = 0 \quad \forall i, j \in \set{1, \dots, n} \text{ and } i \neq j
  \]
  and
  \[
    \tr{E^{i i} - E^{1 1}} = 1 + (-1) = 0 \quad \forall i \in \set{2, \dots, n},
  \]
  we know that \(\beta \subseteq \W\).
  Thus by \cref{1.5} we have \(\W = \spn{\beta}\).
  By \cref{ex:1.5.6} we know that \(\beta\) is linearly independent, thus by \cref{1.6.1} \(\beta\) is a basis for \(\W\), and by \cref{1.6.8} we know that \(\dim(\W) = n^2 - 1\).
\end{proof}

\begin{ex}\label{ex:1.6.16}
  The set of all upper triangular \(n \times n\) matrices is a subspace \(\W\) of \(\ms{n}{n}{\F}\) (see \cref{ex:1.3.12}).
  Find a basis for \(\W\).
  What is the dimension of \(\W\)?
\end{ex}

\begin{proof}[\pf{ex:1.6.16}]
  Let \(E^{i j} \in \ms{n}{n}{\F}\) be matrix defined as in \cref{ex:1.5.6} and let \(\beta\) be the set
  \[
    \beta = \set{E^{i j} : i, j \in \set{1, \dots, n}, i \leq j}.
  \]
  Clearly \(\beta \subseteq \W\).
  Since
  \begin{align*}
             & \forall A \in \W, A = \sum_{i = 1}^n \sum_{j = 1}^n A_{i j} E^{i j} = \sum_{i = 1}^n \sum_{j = i}^n A_{i j} E^{i j} &  & \text{(by \cref{ex:1.3.12})} \\
    \implies & \forall A \in \W, A \in \spn{\beta}                                                                                 &  & \text{(by \cref{1.4.3})}     \\
    \implies & \W = \spn{\beta}                                                                                                    &  & \text{(by \cref{1.5})}
  \end{align*}
  and \(\beta\) is linearly independent (by \cref{ex:1.5.6}), by \cref{1.6.1} we know that \(\beta\) is a basis for \(\W\).
  Thus by \cref{1.6.8} we have \(\dim(\W) = \frac{1}{2} n(n + 1)\).
\end{proof}

\begin{ex}\label{ex:1.6.17}
  The set of all skew-symmetric \(n \times n\) matrices is a subspace \(\W\) of \(\ms{n}{n}{\F}\) (see \cref{ex:1.3.28}).
  Find a basis for \(\W\).
  What is the dimension of \(\W\)?
\end{ex}

\begin{proof}[\pf{ex:1.6.17}]
  Let \(E^{i j} \in \ms{n}{n}{\F}\) be matrix defined as in \cref{ex:1.5.6} and let \(\beta\) be the set
  \[
    \beta = \set{E^{i j} - E^{j i} : i, j \in \set{1, \dots, n}, i < j}.
  \]
  By \cref{ex:1.3.28} we know that \(\beta \subseteq \W\).
  Since
  \begin{align*}
             & \forall A \in \W, \tp{A} = -A                                                  &  & \text{(by \cref{ex:1.3.28})} \\
    \implies & \forall A \in \W, A_{j i} = -A_{i j} \text{ where } i, j \in \set{1, \dots, n} &  & \text{(by \cref{1.3.3})}     \\
    \implies & \forall A \in \W, \begin{dcases}
      A_{i j} = -A_{j i} & \text{if } i \neq j \\
      A_{i j} = 0        & \text{if } i = j
    \end{dcases}                                                                     \\
    \implies & \forall A \in \W, A = \sum_{i = 1}^n \sum_{j = 1}^n A_{i j} E^{i j}                                              \\
             & = \sum_{i = 1}^n \sum_{j = 1}^{i - 1} (A_{i j} E^{i j} + A_{j i} E^{j i})                                        \\
             & = \sum_{i = 1}^n \sum_{j = 1}^{i - 1} (A_{i j} E^{i j} - A_{i j} E^{j i})                                        \\
             & = \sum_{i = 1}^n \sum_{j = 1}^{i - 1} A_{i j} (E^{i j} - E^{j i})                                                \\
    \implies & \forall A \in \W, A \in \spn{\beta}                                            &  & \text{(by \cref{1.4.3})}     \\
    \implies & \W = \spn{\beta}                                                               &  & \text{(by \cref{1.5})}
  \end{align*}
  and \(\beta\) is linearly independent (by \cref{ex:1.5.6}), by \cref{1.6.1} we know that \(\beta\) is a basis for \(\W\).
  Thus by \cref{1.6.8} we have \(\dim(\W) = \frac{1}{2} n(n - 1)\).
\end{proof}

\begin{ex}\label{ex:1.6.18}
  Find a basis for the vector space in \cref{1.2.13}.
  Justify your answer.
\end{ex}

\begin{proof}[\pf{ex:1.6.18}]
  Let \(\V\) be the vector space in \cref{1.2.13} over field \(\F\).
  We claim that the set
  \[
    \beta = \set{\set{e_n^i} : e_n^i = 0 \text{ when } n \neq i ; e_n^i = 1 \text{ when } n = i}
  \]
  is a basis for \(\V\).
  Clearly \(\beta \subseteq \V\).
  Since
  \begin{align*}
             & \forall \set{a_n} \in \V, \set{a_n} = \sum_{i \in \N} a_n \set{e_n^i} &  & \text{(by \cref{1.2.13})} \\
    \implies & \V = \spn{\beta}                                                      &  & \text{(by \cref{1.5})}
  \end{align*}
  and \(\beta\) is linearly independent (for obvious reason), by \cref{1.6.1} we know that \(\beta\) is a basis for \(\V\).
\end{proof}

\setcounter{ex}{19}
\begin{ex}\label{ex:1.6.20}
  Let \(\V\) be a vector space over \(\F\) having dimension \(n\), and let \(S\) be a subset of \(\V\) that generates \(\V\).
  \begin{enumerate}
    \item Prove that there is a subset of \(S\) that is a basis for \(\V\).
          (Be careful not to assume that \(S\) is finite.)
    \item Prove that \(S\) contains at least \(n\) vectors.
  \end{enumerate}
\end{ex}

\begin{proof}[\pf{ex:1.6.20}(a)]
  Note that the hypothesis of \cref{ex:1.6.20}(a) is different from \cref{1.9} that in \cref{1.9} \(S\) is assumed to be finite.
  Suppose for sake of contradiction that such subset does not exist.
  By \cref{1.6.1} this means every subset of \(S\) must either be linearly dependent or cannot generate \(\V\).
  Since \(S\) is a subset of \(S\) and \(\spn{S} = \V\), we know that \(S\) must be linearly dependent.

  Now we let \(\beta_0\) be a finite, linearly independent subset of \(S\).
  From previous paragraph we know that \(\beta_0 \neq S\) and \(\V \neq \spn{\beta_0}\).
  By \cref{1.10} we know that \(\#(\beta_0) \leq n\).
  Then there exists a \(v_1 \in S\) such that \(v_1 \cup \beta_0\) is linearly independent.
  If such \(v_1\) does not exist, then we would have \(S \subseteq \spn{\beta_0}\), which implies \(\spn{\beta_0} = \V\), a contradiction.
  So such \(v_1\) exists and we let \(\beta_1 = \beta_0 \cup v_1\).
  Again we must have \(\beta_1 \neq S\), \(\V \neq \spn{\beta_1}\) and \(\#(\beta_1) \leq n\).
  Using the same argument as above there must exist a \(v_2 \in S\) such that \(v_2 \cup \beta_1\) is linearly independent.
  Continue this definition we can define \(\beta_n = \set{\seq{v}{1,2,,n}}\).
  But by \cref{1.6.15}(b) we know that \(\beta_n\) must be a basis for \(\V\), a contradiction.
  Thus there must exists a subset of \(S\) which is a basis for \(\V\).
\end{proof}

\begin{proof}[\pf{ex:1.6.20}(b)]
  From \cref{ex:1.6.20}(a) we know that there exists a subset of \(S\) which is a basis for \(\V\).
  Thus by \cref{1.6.15}(a) \(S\) must has at least \(n\) vectors.
\end{proof}

\begin{ex}\label{ex:1.6.21}
  Prove that a vector space is infinite-dimensional if and only if it contains an infinite linearly independent subset.
\end{ex}

\begin{proof}[\pf{ex:1.6.21}]
  Let \(\V\) be a vector spaces over \(\F\).
  Then
  \begin{align*}
         & \V \text{ is infinite-dimensional}                                                    \\
    \iff & \V \text{ has an infinite basis}                        &  & \text{(by \cref{1.6.8})} \\
    \iff & \V \text{ has an infinite linearly independent subset}. &  & \text{(by \cref{1.6.1})}
  \end{align*}
\end{proof}

\begin{ex}\label{ex:1.6.22}
  Let \(\W_1\) and \(\W_2\) be subspaces of a finite-dimensional vector space \(\V\) over \(\F\).
  Determine necessary and sufficient conditions on \(\W_1\) and \(\W_2\) so that \(\dim(\W_1 \cap \W_2) = \dim(\W_1)\).
\end{ex}

\begin{proof}[\pf{ex:1.6.22}]
  We have
  \begin{align*}
         & \dim(\W_1 \cap \W_2) = \dim(\W_1)                                                     \\
    \iff & \begin{dcases}
      \exists \beta_1 \subseteq \W_1 \cap \W_2 \\
      \exists \beta_2 \subseteq \W_1
    \end{dcases} : \begin{dcases}
      \beta_1 \text{ is a basis for } \W_1 \cap \W_2 \\
      \beta_2 \text{ is a basis for } \W_1           \\
      \#(\beta_1) = \#(\beta_2)
    \end{dcases} &  & \text{(by \cref{1.6.8})} \\
    \iff & \W_1 \cap \W_2 = \W_1                                   &  & \text{(by \cref{1.11})}  \\
    \iff & \W_1 \subseteq \W_2.
  \end{align*}
\end{proof}

\begin{ex}\label{ex:1.6.23}
  Let \(\seq{v}{1,2,,k}, v\) be vectors in a vector space \(\V\), and define \(W_1 = \spn{\set{\seq{v}{1,2,,k}}}\), and \(\W_2 = \spn{\set{\seq{v}{1,2,,k}, v}}\).
  \begin{enumerate}
    \item Find necessary and sufficient conditions on \(v\) such that \(\dim(\W_1) = \dim(\W_2)\).
    \item State and prove a relationship involving \(\dim(\W_1)\) and \(\dim(\W_2)\) in the case that \(\dim(\W_1) \neq \dim(\W_2)\).
  \end{enumerate}
\end{ex}

\begin{proof}[\pf{ex:1.6.23}(a)]
  We have
  \begin{align*}
         & \begin{dcases}
      \W_1 \subseteq \W_2 \\
      \dim(\W_1) = \dim(\W_2)
    \end{dcases}                                  &  & \text{(by \cref{ex:1.4.13})} \\
    \iff & \W_1 = \W_2                                                 &  & \text{(by \cref{1.11})}      \\
    \iff & v \in \spn{\set{\seq{v}{1,2,,k}}}                           &  & \text{(by \cref{1.4.3})}     \\
    \iff & v \cup \set{\seq{v}{1,2,,k}} \text{ is linearly dependent}. &  & \text{(by \cref{1.7})}
  \end{align*}
\end{proof}

\begin{proof}[\pf{ex:1.6.23}(b)]
  By \cref{1.11} we have
  \[
    \begin{dcases}
      \W_1 \subseteq \W_2 \\
      \W_1 \neq \W_2
    \end{dcases} \implies \dim(\W_1) \neq \dim(\W_2).
  \]
\end{proof}

\begin{ex}\label{ex:1.6.24}
  Let \(f(x)\) be a polynomial of degree n in \(\ps[n]{\R}\).
  Prove that for any \(g(x) \in \ps[n]{\R}\) there exist scalars \(\seq{c}{0,1,,n}\) such that
  \[
    g(x) = c_0 f(x) + c_1 f'(x) + c_2 f''(x) + \cdots + c_n f^{(n)}(x),
  \]
  where \(f^{(n)}(x)\) denotes the \(n\)th derivative of \(f(x)\).
\end{ex}

\begin{proof}[\pf{ex:1.6.24}]
  We denote \(f^{(0)} = f\).
  Since \(f^{(i)}(x)\) has degree \(n - i\) for all \(i = 0, 1, \dots, n\), we know that the set
  \[
    \beta = \set{f^{(i)} : i = 0, 1, \dots, n}
  \]
  is linearly independent.
  Since \(\#(\beta) = n + 1\), by \cref{1.6.15}(b) we know that \(\beta\) is a basis for \(\ps[n]{\R}\), thus there exist \(\seq{c}{0,1,,n}\) such that \(g = \sum_{i = 0}^n c_i f^{(i)}\).
\end{proof}

\begin{ex}\label{ex:1.6.25}
  If \(\V\) and \(\W\) are vector spaces over \(\F\) of dimensions \(m\) and \(n\), determine the dimension of \(\V \times \W\) (see \cref{ex:1.2.21}).
\end{ex}

\begin{proof}[\pf{ex:1.6.25}]
  Let \(\beta_v = \set{\seq{v}{1,2,,m}}, \beta_w = \set{\seq{w}{1,2,n}}\) be a basis for \(\V, \W\), respectively.
  Let \(\zv_v, \zv_w\) be the zero vectors of \(\V, \W\), respectively.
  Then we claim the set
  \[
    \beta = \pa{\beta_v \times \set{\zv_w}} \cup \pa{\set{\zv_v} \times \beta_w}
  \]
  is a basis for \(\V \times \W\).
  Clearly \(\beta \subseteq \V \times \W\).
  Since
  \begin{align*}
             & \forall \seq{a}{1,2,,m + n} \in \F, \sum_{i = 1}^m a_i (v_i, \zv_w) + \sum_{i = 1}^n a_{m + i} (\zv_v, w_i)                                   \\
             & = (\sum_{i = 1}^m a_i v_i, \sum_{i = 1}^n a_{m + i} w_i) = (\zv_v, \zv_w)                                   &  & \text{(by \cref{ex:1.2.21})} \\
    \implies & \begin{dcases}
      \sum_{i = 1}^m a_i v_i = \zv_v \\
      \sum_{i = 1}^n a_{m + i} w_i = \zv_w
    \end{dcases}                                                                                                                    \\
    \implies & \seq[=]{a}{1,2,,m + n} = 0,
  \end{align*}
  by \cref{1.5.3} we know that \(\beta\) is linearly independent.
  Since
  \begin{align*}
             & \forall (v, w) \in \V \times \W, \exists \seq{a}{1,2,,m + n} \in \F :                                       \\
             & (v, w) = (\sum_{i = 1}^m a_i v_i, \sum_{i = 1}^n a_{m + i} w_i)                                             \\
             & = \sum_{i = 1}^m a_i (v_i, \zv_w) + \sum_{i = 1}^n a_{m + i} (\zv_v, w_i) &  & \text{(by \cref{ex:1.2.21})} \\
    \implies & \forall (v, w) \in \V \times \W, (v, w) \in \spn{\beta}                                                     \\
    \implies & \V \times \W = \spn{\beta},                                               &  & \text{(by \cref{1.5})}
  \end{align*}
  by \cref{1.6.1} we know that \(\beta\) is a basis for \(\V \times \W\) and \(\dim(\V \times \W) = m + n\).
\end{proof}

\begin{ex}\label{ex:1.6.26}
  For a fixed \(a \in \R\), determine the dimension of the subspace of \(\ps[n]{\R}\) defined by \(\set{f \in \ps[n]{\R} : f(a) = 0}\).
\end{ex}

\begin{proof}[\pf{ex:1.6.26}]
  Since the set
  \[
    \beta = \set{x - a, x^2 - a^2, \dots, x^n - a^n} \subseteq \set{f \in \ps[n]{\R} : f(a) = 0} \subseteq \ps[n]{\R}
  \]
  is linearly independent (by \cref{ex:1.5.5}) and
  \begin{align*}
             & \forall g \in \set{f \in \ps[n]{\R} : f(a) = 0}, \exists \seq{c}{1,2,,n} \in \R :                               \\
             & \forall x \in \R, g(x) = c_1 (x - a) + c_2 (x^2 - a^2) + \cdots + c_n (x^n - a^n)                               \\
    \implies & \forall g \in \set{f \in \ps[n]{\R} : f(a) = 0}, g \in \spn{\beta}                &  & \text{(by \cref{1.4.3})} \\
    \implies & \set{f \in \ps[n]{\R} : f(a) = 0} = \spn{\beta},                                  &  & \text{(by \cref{1.5})}
  \end{align*}
  by \cref{1.6.1} we know that \(\beta\) is a basis for \(\set{f \in \ps[n]{\R} : f(a) = 0}\).
  Thus by \cref{1.6.8} we have \(\dim(\set{f \in \ps[n]{\R} : f(a) = 0}) = \#(\beta) = n\).
\end{proof}

\begin{ex}\label{ex:1.6.27}
  Let \(\W_1\) and \(\W_2\) be the subspaces of \(\ps{\F}\) defined in \cref{ex:1.3.25}.
  Determine the dimensions of the subspaces \(\W_1 \cap \ps[n]{\F}\) and \(\W_2 \cap \ps[n]{\F}\).
\end{ex}

\begin{proof}[\pf{ex:1.6.27}]
  First suppose that \(n\) is odd.
  We define
  \begin{align*}
    \beta_1 & = \set{1, x^2, x^4 \dots, x^{n - 1}} \\
    \beta_2 & = \set{x, x^3, x^5 \dots, x^n}.
  \end{align*}
  Then we have
  \begin{align*}
             & \begin{dcases}
      \beta_1 \text{ is linearly independent} \\
      \beta_2 \text{ is linearly independent} \\
      \W_1 = \spn{\beta_1}                    \\
      \W_2 = \spn{\beta_2}
    \end{dcases}  &  & \text{(by \cref{ex:1.3.25})} \\
    \implies & \begin{dcases}
      \beta_1 \text{ is a basis for } \W_1 \\
      \beta_2 \text{ is a basis for } \W_2
    \end{dcases}  &  & \text{(by \cref{1.6.1})}     \\
    \implies & \begin{dcases}
      \dim(\W_1) = \#(\beta_1) = \frac{n - 1}{2} \\
      \dim(\W_2) = \#(\beta_2) = \frac{n - 1}{2}
    \end{dcases}. &  & \text{(by \cref{1.6.8})}
  \end{align*}
  Using similar arguments as above we can show that when \(n\) is even we have
  \[
    \begin{dcases}
      \dim(\W_1) = \#(\beta_1) = \frac{n}{2} + 1 \\
      \dim(\W_2) = \#(\beta_2) = \frac{n}{2}
    \end{dcases}.
  \]
\end{proof}

\begin{ex}\label{ex:1.6.28}
  Let \(\V\) be a finite-dimensional vector space over \(\C\) with dimension \(n\).
  Prove that if \(\V\) is now regarded as a vector space over \(\R\), then \(\dim(\V) = 2n\).
\end{ex}

\begin{proof}[\pf{ex:1.6.28}]
  Let \(\beta = \set{\seq{v}{1,2,,n}}\) be a basis for \(\V\) over \(\C\).
  Since
  \begin{align*}
             & \spn{\beta} = \V                                                           &  & \text{(by \cref{1.6.1})}    \\
    \implies & \forall v \in \V, \exists \seq{a}{1,2,,n} \in \C :                                                          \\
             & v = \sum_{j = 1}^n a_j v_j = \sum_{j = 1}^n \pa{\Re(a_j) + i \Im(a_j)} v_j                                  \\
             & = \sum_{j = 1}^n \Re(a_j) (1 v_j) + \sum_{j = 1}^n \Im(a_j) (i v_j)        &  & \text{(by \cref{1.4.3})}    \\
    \implies & \forall v \in \V, \exists \seq{b}{1,2,,2n} \in \R :                                                         \\
             & v = \sum_{j = 1}^n b_j v_j + \sum_{j = 1}^n b_{n + j} (i v_j)              &  & (\Re(a_j), \Im(a_j) \in \R) \\
    \implies & \spn{\set{\seq{v}{1,2,,n}, \seq{iv}{1,2,,n}}} = \V                         &  & \text{(by \cref{1.4.3})}
  \end{align*}
  and
  \begin{align*}
             & \forall \seq{b}{1,2,,2n} \in \R, \sum_{j = 1}^n b_j v_j + \sum_{j = 1}^n b_{n + j} (i v_j) = \zv                                            \\
    \implies & \sum_{j = 1}^n (b_j + i b_{n + j}) v_j = \zv                                                                                                \\
    \implies & b_1 + i b_{n + 1} = b_2 + i b_{n + 2} = \cdots = b_n + i b_{2n} = 0                              &  & \text{(by \cref{1.5.3})}              \\
    \implies & \seq[=]{b}{1,2,,2n} = 0,                                                                         &  & (\set{\seq{b}{1,2,,2n}} \subseteq \R)
  \end{align*}
  by \cref{1.6.1} we know that \(\set{\seq{v}{1,2,,n}, \seq{iv}{1,2,,n}}\) is a basis for \(\V\) over \(\R\) with dimension \(2n\).
\end{proof}

\begin{ex}\label{ex:1.6.29}
  \quad
  \begin{enumerate}
    \item Prove that if \(\W_1\) and \(\W_2\) are finite-dimensional subspaces of a vector space \(\V\) over \(\F\), then the subspace \(\W_1 + \W_2\) is finite-dimensional, and \(\dim(\W_1 + \W_2) = \dim(\W_1) + \dim(\W_2) - \dim(\W_1 \cap \W_2)\).
    \item Let \(\W_1\) and \(\W_2\) be finite-dimensional subspaces of a vector space \(\V\) over \(\F\), and let \(\V = \W_1 + \W_2\).
          Deduce that \(\V\) is the direct sum of \(\W_1\) and \(\W_2\) if and only if \(\dim(\V) = \dim(\W_1) + \dim(\W_2)\).
  \end{enumerate}
\end{ex}

\begin{proof}[\pf{ex:1.6.29}(a)]
  If \(\dim(\W_1) = \dim(\W_2) = 0\), then we have
  \[
    \dim(\W_1 + \W_2) = \dim(\set{\zv}) = 0 = 0 + 0 - 0 = \dim(\W_1) + \dim(\W_2) - \dim(\W_1 \cap \W_2).
  \]
  So suppose that \(\dim(\W_1) > 0\) or \(\dim(\W_2) > 0\).

  By \cref{1.4} we know that \(\W_1 \cap \W_2\) is a subspace of \(\W_1\) and \(\W_2\) over \(\F\).
  Since \(\W_1\) and \(\W_2\) are finite-dimensional, by \cref{1.11} we know that \(\W_1 \cap \W_2\) is finite-dimensional.

  Let \(\beta = \set{\seq{u}{1,2,,k}}\) be a basis for \(\W_1 \cap \W_2\) over \(\F\).
  Note that \(k \geq 0\) and when \(k = 0\) we have \(\beta = \varnothing\).
  By \cref{1.6.15}(c) we can extend \(\beta\) to another set \(\beta_1 = \set{\seq{u}{1,2,,k}, \seq{v}{1,2,,m}}\) which is basis for \(\W_1\) over \(\F\).
  Using similar argument we can extend \(\beta\) to \(\beta_2 = \set{\seq{u}{1,2,,k}, \seq{w}{1,2,,p}}\) which is a basis for \(\W_2\) over \(\F\).
  Note that \(m, p \geq 0\) and \(\max(m, p) \geq 1\).
  Since
  \begin{align*}
             & \begin{dcases}
      \forall v \in \beta_1, v + \zv \in \W_1 + \W_2 \\
      \forall w \in \beta_2, \zv + w \in \W_1 + \W_2
    \end{dcases}                 &  & \text{(by \cref{1.3.10})} \\
    \implies & \beta_1 \cup \beta_2 \subseteq \W_1 + \W_2
  \end{align*}
  and
  \begin{align*}
             & \forall x \in \W_1 + \W_2, \exists (y, z) \in \W_1 \times \W_2 : x = y + z                                                          &  & \text{(by \cref{1.3.10})} \\
    \implies & \forall x \in \W_1 + \W_2, \begin{dcases}
      \exists \seq{a}{1,2,,k + m} \in \F \\
      \exists \seq{b}{1,2,,k + p} \in \F
    \end{dcases} :                                                                                                            \\
             & x = \pa{\sum_{i = 1}^k a_i u_i + \sum_{i = k + 1}^{k + m} a_i v_i} + \pa{\sum_{i = 1}^k b_i u_i + \sum_{i = k + 1}^{k + p} b_i w_i} &  & \text{(by \cref{1.6.1})}  \\
             & = \sum_{i = 1}^k (a_i + b_i) u_i + \sum_{i = k + 1}^{k + m} a_i v_i + \sum_{i = k + 1}^{k + p} b_i w_i                              &  & \text{(by \cref{1.2.1})}  \\
    \implies & \forall x \in \W_1 + \W_2, \exists \seq{a}{1,2,,k + m + p} \in \F :                                                                                                \\
             & x = \sum_{i = 1}^k a_i u_i + \sum_{i = k + 1}^{k + m} a_i v_i + \sum_{i = k + m + 1}^{k + m + p} a_i w_i                                                           \\
    \implies & \forall x \in \W_1 + \W_2, x \in \spn{\beta_1 \cup \beta_2}                                                                         &  & \text{(by \cref{1.4.3})}  \\
    \implies & \W_1 + \W_2 = \spn{\beta_1 \cup \beta_2},                                                                                           &  & \text{(by \cref{1.5})}
  \end{align*}
  by \cref{1.9} we know that \(\dim(\W_1 + \W_2) \leq \#(\beta_1 \cup \beta_2)\) and there exist some \(S \subseteq \beta_1 \cup \beta_2\) such that \(S\) is a basis for \(\W_1 + \W_2\) over \(\F\).
  We now claim that \(\beta_1 \cup \beta_2\) is a basis for \(\W_1 + \W_2\) over \(\F\).
  By \cref{1.6.1} we need to show that \(\beta_1 \cup \beta_2\) is linearly independent.
  Since
  \begin{align*}
             & \beta_1 \cap \beta_2 = \beta                                     \\
    \implies & \begin{dcases}
      \spn{\beta_1 \setminus \beta} \cap \beta_2 = \varnothing \\
      \spn{\beta_2 \setminus \beta} \cap \beta_1 = \varnothing
    \end{dcases}   &  & (\spn{\beta} = \W_1 \cap \W_2) \\
    \implies & \begin{dcases}
      \spn{\beta_1} \cap \beta_2 = \varnothing \\
      \spn{\beta_2} \cap \beta_1 = \varnothing
    \end{dcases},
  \end{align*}
  we know that
  \begin{align*}
             & \forall \seq{a}{1,2,,k + m + p} \in \F,                                                                                                  \\
             & \sum_{i = 1}^k a_i u_i + \sum_{i = k + 1}^{k + m} a_i v_i + \sum_{i = k + m + 1}^{k + m + p} a_i w_i = \zv                               \\
    \implies & \sum_{i = 1}^k a_i u_i + \sum_{i = k + 1}^{k + m} a_i v_i = \sum_{i = k + m + 1}^{k + m + p} -a_i w_i      &  & \text{(by \cref{1.2.1})} \\
    \implies & \seq[=]{a}{1,2,,k + m + p} = 0.                                                                            &  & \text{(by \cref{1.5.3})}
  \end{align*}
  Thus by \cref{1.5.3} \(\beta_1 \cup \beta_2\) is linearly independent and by \cref{1.6.1} \(\beta_1 \cup \beta_2\) is a basis for \(\W_1 + \W_2\).
  Then we have
  \[
    \dim(\W_1 + \W_2) = k + m + p = (k + m) + (k + p) - k = \dim(\W_1) + \dim(\W_2) - \dim(\W_1 \cap \W_2).
  \]
\end{proof}

\begin{proof}[\pf{ex:1.6.29}(b)]
  We have
  \begin{align*}
         & \V = \W_1 \oplus \W_2                                                                        \\
    \iff & \begin{dcases}
      \V = \W_1 + \W_2 \\
      \W_1 \cap \W_2 = \varnothing
    \end{dcases}                             &  & \text{(by \cref{1.3.11})}       \\
    \iff & \begin{dcases}
      \V = \W_1 + \W_2 \\
      \dim(\W_1 \cap \W_2) = 0
    \end{dcases}                             &  & \text{(by \cref{1.6.2})}        \\
    \iff & \dim(\V) = \dim(\W_1 + \W_2) = \dim(\W_1) + \dim(\W_2). &  & \text{(by \cref{ex:1.6.29}(a))}
  \end{align*}
\end{proof}

\setcounter{ex}{30}
\begin{ex}\label{ex:1.6.31}
  Let \(\W_1\) and \(\W_2\) be subspaces of a vector space \(\V\) over \(\F\) having dimensions \(m\) and \(n\), respectively, where \(m \geq n\).
  \begin{enumerate}
    \item Prove that \(\dim(\W_1 \cap \W_2) \leq n\).
    \item Prove that \(\dim(\W_1 + \W_2) \leq m + n\).
  \end{enumerate}
\end{ex}

\begin{proof}[\pf{ex:1.6.31}(a)]
  We have
  \begin{align*}
             & \W_1 \cap \W_2 \subseteq \W_2                                          \\
    \implies & \dim(\W_1 \cap \W_2) \leq \dim(\W_2) = n. &  & \text{(by \cref{1.11})}
  \end{align*}
\end{proof}

\begin{proof}[\pf{ex:1.6.31}(b)]
  We have
  \begin{align*}
    \dim(\W_1 + \W_2) & = \dim(\W_1) + \dim(\W_2) - \dim(\W_1 \cap \W_2) &  & \text{(by \cref{ex:1.6.29}(a))} \\
                      & = m + n - \dim(\W_1 \cap \W_2)                                                        \\
                      & \leq m + n.
  \end{align*}
\end{proof}

\setcounter{ex}{32}
\begin{ex}\label{ex:1.6.33}
  \quad
  \begin{enumerate}
    \item Let \(\W_1\) and \(\W_2\) be subspaces of a vector space \(\V\) over \(\F\) such that \(\V = \W_1 \oplus \W_2\).
          If \(\beta_1\) and \(\beta_2\) are bases for \(\W_1\) and \(\W_2\) over \(\F\), respectively, show that \(\beta_1 \cap \beta_2 = \varnothing\) and \(\beta_1 \cup \beta_2\) is a basis for \(\V\) over \(\F\).
    \item Conversely, let \(\beta_1\) and \(\beta_2\) be disjoint bases for subspaces \(\W_1\) and \(\W_2\), respectively, of a vector space \(\V\) over \(\F\).
          Prove that if \(\beta_1 \cup \beta_2\) is a basis for \(\V\) over \(\F\), then \(\V = \W_1 \oplus \W_2\).
  \end{enumerate}
\end{ex}

\begin{proof}[\pf{ex:1.6.33}(a)]
  First observe that
  \begin{align*}
             & \V = \W_1 \oplus \W_2                                                                                              \\
    \implies & \V = \W_1 + \W_2                                                                    &  & \text{(by \cref{1.3.11})} \\
    \implies & \forall v \in \V, \exists (x, y) \in \W_1 \times \W_2 : v = x + y                                                  \\
    \implies & \forall v \in \V, \exists (x, y) \in \spn{\beta_1} \times \spn{\beta_2} : v = x + y &  & \text{(by \cref{1.6.1})}  \\
    \implies & \forall v \in \V, v \in \spn{\beta_1 \cup \beta_2}                                  &  & \text{(by \cref{1.2.1})}  \\
    \implies & \V = \spn{\beta_1 \cup \beta_2}.                                                    &  & \text{(by \cref{1.5})}
  \end{align*}
  Since
  \begin{align*}
             & V = \W_1 \oplus \W_2                                                                             \\
    \implies & \W_1 \cap \W_2 = \set{\zv}                                  &  & \text{(by \cref{1.3.11})}       \\
    \implies & \beta_1 \cap \beta_2 = \varnothing                                                               \\
    \implies & \dim(\V) = \dim(\W_1 \oplus \W_2) = \dim(\W_1) + \dim(\W_2) &  & \text{(by \cref{ex:1.6.29}(b))} \\
             & = \#(\beta_1) + \#(\beta_2) = \#(\beta_1 \cup \beta_2),
  \end{align*}
  by \cref{1.6.15}(a) we know that \(\beta_1 \cup \beta_2\) is a basis for \(\V\) over \(\F\).
\end{proof}

\begin{proof}[\pf{ex:1.6.33}(b)]
  Since
  \begin{align*}
             & \begin{dcases}
      \beta_1 \cap \beta_2 = \varnothing \\
      \spn{\beta_1 \cup \beta_2} = \V
    \end{dcases}                                                                                           \\
    \implies & \forall v \in \V, v \in \spn{\beta_1 \cup \beta_2}                                                                    \\
    \implies & \forall v \in \V, \exists (x, y) \in \spn{\beta_1} \times \spn{\beta_2} : v = x + y &  & \text{(by \cref{1.2.1})}     \\
    \implies & \forall v \in \V, v \in \W_1 + \W_2                                                 &  & \text{(by \cref{1.3.10})}    \\
    \implies & \V \subseteq \W_1 + \W_2                                                            &  & \text{(by \cref{1.5})}       \\
    \implies & \V = \W_1 + \W_2                                                                    &  & \text{(by \cref{ex:1.3.23})}
  \end{align*}
  and
  \begin{align*}
             & \begin{dcases}
      \beta_1 \cap \beta_2 = \varnothing \\
      \beta_1 \cup \beta_2 \text{ is a basis for } \V \text{ over } \F
    \end{dcases}                                                \\
    \implies & \begin{dcases}
      \spn{\beta_1} \cap \beta_2 = \varnothing \\
      \spn{\beta_2} \cap \beta_1 = \varnothing
    \end{dcases}                  &  & \text{(by \cref{1.7})}   \\
    \implies & \spn{\beta_1} \cap \spn{\beta_2} = \set{\zv} &  & \text{(by \cref{1.5.3})} \\
    \implies & \W_1 \cap \W_2 = \set{\zv},
  \end{align*}
  by \cref{1.3.11} we know that \(\V = \W_1 \oplus \W_2\).
\end{proof}



%------------------------------------------------------------------------------
% Back matters.
%------------------------------------------------------------------------------

\backmatter

\end{document}

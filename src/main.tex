% We use chapter structure.
\documentclass[12pt,oneside]{book}

%==============================================================================
% Preamble.
%==============================================================================

% Correctly showing characters outside ASCII.
\usepackage[T1]{fontenc}
% File is written and read with utf8 encoding.
\usepackage[utf8]{inputenc}
% Set paging layout.
\usepackage[margin=1.2in]{geometry}
% Including `amsfonts'.  Must be loaded before `mathtools'.
\usepackage{amssymb}
% Including `amsmath' and fixing bugs for `amsmath'.
\usepackage{mathtools}
% Must be loaded after `amsmath' and `mathtools'.
\usepackage{amsthm}
% Define page header and footer layout.
\usepackage{fancyhdr}
% LaTex 3 new command tools.
\usepackage{xparse}

% Use `fancyhdr' page style.  All page style settings must be called only after
% this command.  See `fancyhdr' for details.
\pagestyle{fancy}

% Make header height enough to fit chapter / section titles.
\setlength{\headheight}{15pt}
% Change chapter and section marks' formatting.  Note that I use `\markright'
% to make sure header use chapter information when section information is not
% available (this is need for appendix).
\renewcommand{\chaptermark}[1]{\markright{\textsf{\small Chap. \thechapter \quad #1}}}
\renewcommand{\sectionmark}[1]{\markright{\textsf{\small Sec. \thesection \quad #1}}}
% Cleanup page header settings.
\fancyhead{}
% Put page number on the left of page header.
\fancyhead[L]{\textbf{\textsf{\small \thepage}}}
% Put chapter / section information on the right of page header.
\fancyhead[R]{\rightmark}
% Cleanup page footer settings.
\fancyfoot{}

% Reset `plain' page style which is used for the first page of each chapter.
% I set this to make page number style consistent.
\fancypagestyle{plain}{%
  \fancyhf{}% clear all fields
  \fancyfoot[C]{\textbf{\textsf{\small \thepage}}}%
  \renewcommand{\headrulewidth}{0pt}%
}

% Automatically adjust character spacing at margins.
\usepackage{microtype}
% Provide further utilities and fix bugs for `enumerate', `itemize' and
% `description'.
\usepackage{enumitem}
% Provide better quoting environment.
\usepackage{dirtytalk}
% Parsing list inside `\newcommand'.
\usepackage{listofitems}
% Nice looking if-then-else structure with comparison functionality.
\usepackage{ifthen}
% Automatically add hyperlinks to labels/refs.  Must be loaded after all
% packages above and before `cleveref'.  Recommend to use with `natbib' when
% you need bibtex.
\usepackage{hyperref}

\hypersetup{       % This macro come with `hyperref'.
	colorlinks=true, % Color hyperlinks.
	linkcolor=blue,  % Color local hyperlinks with blue.
	urlcolor=cyan,   % Color url links with cyan.
}

% Must be loaded after `hyperref'.  We always capitalize each cross-references'
% type name.  See `cleveref' for details.
\usepackage[capitalize]{cleveref}

% Allow page break in the middle of multi-line equations.
\allowdisplaybreaks

%------------------------------------------------------------------------------
% Define environments.
%------------------------------------------------------------------------------

% Text inside the body of theorem-like environments are set to Roman font.
% theorem-like environments share their counters, counters follow section and
% reset in every sections (except for theorems, theorems counters are reset for
% each chapter).  Theorems and exercises has their owned counter.  Notes do not
% use counter.  See `amsthm' for details.
\theoremstyle{definition}
\newtheorem{ax}{Ax.}[section]
\newtheorem{cor}[ax]{Cor.}
\newtheorem{defn}[ax]{Def.}
\newtheorem{eg}[ax]{E.g.}
\newtheorem{ex}{Ex.}[section]
\newtheorem{lem}[ax]{Lem.}
\newtheorem{prop}[ax]{Prop.}
\newtheorem{thm}{Thm.}[chapter]
\newtheorem*{note}{Note}

% Define plural form for theorem-like environments.  This is needed when we
% call `cref' with multiple references.  See `cleveref' for details.
\crefname{ax}{Ax.}{Ax.}
\crefname{cor}{Cor.}{Cor.}
\crefname{defn}{Def.}{Def.}
\crefname{eg}{E.g.}{E.g.}
\crefname{ex}{Ex.}{Ex.}
\crefname{lem}{Lem.}{Lem.}
\crefname{note}{Note}{Note}
\crefname{prop}{Prop.}{Prop.}
\crefname{section}{Sec.}{Sec.}
\crefname{chapter}{Ch.}{Ch.}
\crefname{thm}{Thm.}{Thm.}

% Proof environments reference text.
\NewDocumentCommand{\pf}{m}{%
	Proof of \cref{#1}%
}
% Proof statements reference text.
\NewDocumentCommand{\byOptionalArgumentProcess}{m}{(#1)}
\NewDocumentCommand{\by}{m >{\SplitList{,}} o}{%
	\IfNoValueTF{#2}{%
		\text{(by \cref{#1})}%
	}{%
		\text{(by \cref{#1}\ProcessList{#2}{\byOptionalArgumentProcess})}%
	}%
}

% In `enumerate' enviroments, items' label are alphabets and surrounded by
% parentheses.  See `enumitem' for details.
\renewcommand{\labelenumi}{\textnormal{(}\alph{enumi}\textnormal{)}}

% Formatting equations tag appearence.  See `mathtools' for details.
\renewcommand{\theequation}{\thechapter.\thesection.\arabic{equation}}
\numberwithin{equation}{section}

%------------------------------------------------------------------------------
% Define operators and symbols.
%------------------------------------------------------------------------------

% Define common operators with paired of delimiters.  Always use star versions
% of these operator to automatically adjust height.  See `mathtools' for
% details.

% Absolute value.
\DeclarePairedDelimiter{\absTmp}{\lvert}{\rvert}
\NewDocumentCommand{\abs}{m}{\absTmp*{#1}}
% Ceiling.
\DeclarePairedDelimiter{\ceilTmp}{\lceil}{\rceil}
\NewDocumentCommand{\ceil}{m}{\ceilTmp*{#1}}
% Floor.
\DeclarePairedDelimiter{\floorTmp}{\lfloor}{\rfloor}
\NewDocumentCommand{\floor}{m}{\floorTmp*{#1}}
% Evaluate.
\DeclarePairedDelimiter{\evalTmp}{.}{\rvert}
\NewDocumentCommand{\eval}{m}{\evalTmp*{#1}}
% Inner product.
\DeclarePairedDelimiter{\innTmp}{\langle}{\rangle}
\NewDocumentCommand{\inn}{m}{\innTmp*{#1}}
% Norm.
\DeclarePairedDelimiter{\normTmp}{\lVert}{\rVert}
\NewDocumentCommand{\norm}{m}{\normTmp*{#1}}
% Parenthese.
\DeclarePairedDelimiter{\paTmp}{\lparen}{\rparen}
\NewDocumentCommand{\pa}{m}{\paTmp*{#1}}
% Bracket.
\DeclarePairedDelimiter{\brTmp}{\lbrack}{\rbrack}
\NewDocumentCommand{\br}{m}{\brTmp*{#1}}
% Brace.
\DeclarePairedDelimiter{\BTmp}{\lbrace}{\rbrace}
\NewDocumentCommand{\B}{m}{\BTmp*{#1}}
% Set.
\NewDocumentCommand{\set}{m}{\B{#1}}

% Define common symbols.  See `amsmath' section 9.2 for details.

% Fields.
\NewDocumentCommand{\field}{m}{\mathit{#1}}
% General field.
\NewDocumentCommand{\F}{}{\field{F}}
% Complex number field.
\NewDocumentCommand{\C}{}{\mathbb{C}}
% Natural number field.
\NewDocumentCommand{\N}{}{\mathbb{N}}
% Rational number field.
\NewDocumentCommand{\Q}{}{\mathbb{Q}}
% Real number field.
\NewDocumentCommand{\R}{}{\mathbb{R}}
% Integer number field.
\NewDocumentCommand{\Z}{}{\mathbb{Z}}

% Complex conjugate.
\NewDocumentCommand{\conj}{m}{\overline{#1}}

% Vector space.
\NewDocumentCommand{\vs}{m}{\mathsf{#1}}
% General vector space.
\NewDocumentCommand{\V}{}{\vs{V}}
\NewDocumentCommand{\W}{}{\vs{W}}
% Collection of vector space.
\NewDocumentCommand{\cvs}{}{\mathcal{C}}

% Metric space with shape #1 x #2 over field #3.
\NewDocumentCommand{\ms}{O{m}O{n}O{\F}}{\vs{M}_{{#1} \times {#2}}\pa{#3}}

% Function space.
\NewDocumentCommand{\fs}{}{\mathcal{F}}
% General function space.
\NewDocumentCommand{\FS}{}{\fs(S, \F)}

% Continuous function space with continuous #1-th derivative.
\NewDocumentCommand{\cfs}{o}{%
	\IfNoValueTF{#1}{%
		\vs{C}%
	}{%
		\vs{C}^{#1}%
	}%
}

% Polynomial spaces of degree #1 over field #2.
% #1 (optional) is the degree of polynomial.
% #2 is field.
%
% For example:
% If we use \(\ps{\F}\), we will get
%
%    \vs{P}(\F)
%
% If we use \(\ps[n]{\F}\), we will get
%
%    \vs{P}_{n}(\F)
%
\NewDocumentCommand{\ps}{om}{%
	\IfNoValueTF{#1}{%
		\vs{P}\pa{#2}%
	}{%
		\vs{P}_{#1}\pa{#2}%
	}%
}

% Derivative operation.
\NewDocumentCommand{\Dop}{}{\mathsf{D}}

% Sequence.
% #1 is a join operator (defaults to comma).
% #2 is a comma-separated list of sequence symbols.
% #3 is a comma-separated list of sequence index.
%
% For example:
% If we use \(\seq{a}{1,2,3}\), we will get
%
%    a_{1}, a_{2}, a_{3}
%
% If we use \(\seq[+]{a,b}{1,2,3}\), we will get
%
%    a_{1} b_{1} + a_{2} b_{2} + a_{3} b_{3}
%
% If we use \(\seq{a}{1}\), we will get
%
%    a_{1}
%
% If we use \(\seq{a}{1,,n}\), we will get
%
%    a_{1}, a_{2}, \dots, a_{n}
%
\NewDocumentCommand{\seq}{O{,} mm}{%
	\setsepchar{,}%                              List items are separated by comma.
	\readlist\SeqSymbols{#2}%                    Define macro `\SeqSymbols' using #1.
	\readlist\SeqIndices{#3}%                    Define macro `\SeqIndices' using #2.
	\foreachitem\SeqIndex\in\SeqIndices{%        Loop over indices.
		\ifthenelse{%                              Output join operator if `\SeqIndex' is not the first index.
			\equal{\SeqIndexcnt}{1}
		}{%
		}{%
			#1
		}%
		%--------------------------------------------------------------------------
		\ifthenelse{%                              Output dots if `\SeqIndex' is empty.
			\equal{\SeqIndex}{}%
		}%
		{%
			\ifthenelse{%                            If #1 is `,' use `\dots', otherwise use `\cdots'.
				\equal{#1}{,}%
			}%
			{\dots}%
			{\cdots}%
		}%
		{%
			\foreachitem\SeqSymbol\in\SeqSymbols{%   Loop over symbols.
				{\SeqSymbol}_{\SeqIndex}%              Output symbols with index.
			}%
		}%
	}%
}

% Tuple.
\NewDocumentCommand{\tuple}{mm}{\pa{\seq{#1}{#2}}}

% 0 vector.
\NewDocumentCommand{\zv}{}{\mathit{0}}
% 0 metric.
\NewDocumentCommand{\zm}{}{\mathit{O}}

% Transpose of matrix #1.
\NewDocumentCommand{\tp}{m}{{#1}^{t}}

% Trace of matrix #1.
\NewDocumentCommand{\tr}{o}{%
	\IfNoValueTF{#1}{%
		\operatorname{tr}%
	}{%
		\operatorname{tr}\pa{#1}%
	}%
}

% Span of a set.
% We cannot use `\span' since it is a tex primitive.
\NewDocumentCommand{\spn}{m}{\operatorname{span}\pa{#1}}

% Linear transformation.
\NewDocumentCommand{\lt}{m}{\mathsf{#1}}
% Linear transformation T.
\NewDocumentCommand{\T}{}{\lt{T}}
% Linear transformation U.
\NewDocumentCommand{\U}{}{\lt{U}}
% Identity transformation I.
\NewDocumentCommand{\IT}{o}{%
	\IfNoValueTF{#1}{%
		\lt{I}%
	}{%
		\lt{I}_{#1}%
	}%
}
% Zero transformation T_0.
\NewDocumentCommand{\zT}{}{\T_0}
% Left-multiplication transformation L_A.
\RenewDocumentCommand{\L}{}{\lt{L}}

% Null space N(T).
\NewDocumentCommand{\ns}{m}{\mathsf{N}\pa{#1}}

% Range R(T).
\NewDocumentCommand{\rg}{m}{\mathsf{R}\pa{#1}}

% Nullity nullity(T).
\NewDocumentCommand{\nt}{m}{\operatorname{nullity}\pa{#1}}

% Rank rank(T).
\NewDocumentCommand{\rk}{m}{\operatorname{rank}\pa{#1}}

% Vector spaces of linear transformations.
% For example:
% 	\ls(\V) or \ls(\V, \W)
\NewDocumentCommand{\ls}{}{\mathcal{L}}

% Formatting exercises section.
\NewDocumentCommand{\exercisesection}{}{
	\begin{center}
		\textbf{EXERCISES}
	\end{center}
}

%==============================================================================
% Document.
%==============================================================================

\begin{document}

%------------------------------------------------------------------------------
% Front matters.
%------------------------------------------------------------------------------

\frontmatter

% Author informations.
\title{Linear Algebra}
\author{ProFatXuanAll}
\maketitle

% Table of contents.
\tableofcontents

% Notations explanation.
\chapter*{Notations}

Almost all statements (defintions, examples, theorems, etc.) in this book are wrapped inside environments.
The following table provide the name of each environment and their meanings.

\begin{table}[h]
  \centering
  \begin{tabular}{|c|c|}
    \hline
    Environment         & Meaning     \\
    \hline
    \namecref{1.1.1}    & Axiom       \\
    \hline
    \namecref{ch:1}     & Chapter     \\
    \hline
    \namecref{1.2.14}   & Corollary   \\
    \hline
    \namecref{1.2.1}    & Definition  \\
    \hline
    \namecref{1.2.4}    & Example     \\
    \hline
    \namecref{ex:1.2.8} & Exercise    \\
    \hline
    \namecref{2.4.5}    & Lemma       \\
    \hline
    Note                & Note        \\
    \hline
    \namecref{2.1.2}    & Proposition \\
    \hline
    \namecref{sec:1.1}  & Section     \\
    \hline
    \namecref{1.1}      & Theorem     \\
    \hline
  \end{tabular}
\end{table}

\addcontentsline{toc}{chapter}{Notations}

%------------------------------------------------------------------------------
% Main matters.
%------------------------------------------------------------------------------

\mainmatter

% All chapters are in separated files.  We include them here.
\chapter{Vector Spaces}\label{ch:1}

% All sections are in separated files.  We include them here.
\section{Introduction}\label{sec 1.1}

\begin{note}
    Experiments show that if two like quantities act together, their effect is predictable.
    In this case, the vectors used to represent these quantities can be combined to form a resultant vector that represents the combined effects of the original quantities.
    This resultant vector is called the \emph{sum} of the original vectors, and the rule for their combination is called the \emph{parallelogram law}.
\end{note}

\begin{axiom}[Parallelogram Law for Vector Addition]\label{ac 1.1.1}
    The sum of two vectors $x$ and $y$ that act at the same point $P$ is the vector beginning at $P$ that is represented by the diagonal of parallelogram having $x$ and $y$ as adjacent sides.
\end{axiom}

\begin{note}
    Since a vector beginning at the origin is completely determined by its endpoint, we sometimes refer to \emph{the point $x$} rather than \emph{the endpoint of the vector $x$} if $x$ is a vector emanating from the origin.
\end{note}

\begin{note}
    Besides the operation of vector addition, there is another natural operation that can be performed on vectors
    --- the length of a vector may be magnified or contracted.
    This operation, called \emph{scalar multiplication}, consists of multiplying the vector by a real number.
    If the vector $x$ is represented by an arrow, then for any real number $t$, the vector $tx$ is represented by an arrow in the same direction if $t \geq 0$ and in the opposite direction if $t < 0$.
    The length of the arrow $tx$ is $\abs*{t}$ times the length of the arrow $x$.
    Two nonzero vectors $x$ and $y$ are called \textbf{parallel} if $y = tx$ for some nonzero real number $t$.
    (Thus nonzero vectors having the same or opposite directions are parallel.)
\end{note}
\section{Vector Spaces}\label{sec:1.2}

\begin{defn}\label{1.2.1}
    A \textbf{vector space} (or \textbf{linear space}) \(\V\) over a field \(\F\) consists of a set on which two operations (called \textbf{addition} and \textbf{scalar multiplication}, respectively) are defined so that for each pair of elements \(x\), \(y\) in \(\V\) there is a unique element \(x + y\) in \(\V\), and for each element \(a\) in \(\F\) and each element \(x\) in \(\V\) there is a unique element \(ax\) in \(\V\), such that the following conditions hold.
    \begin{enumerate}[label=(VS \arabic*), ref=VS \arabic*]
        \item\label{VS1}
        For all \(x, y\) in \(\V\), \(x + y = y + x\)
        (commutativity of addition).
        \item\label{VS2}
        For all \(x, y, z\) in \(\V\), \(\p{x + y} + z = x + \p{y + z}\)
        (associativity of addition).
        \item\label{VS3}
        There exists an element in \(\V\) denoted by \(0\) such that \(x + 0 = x\) for each \(x\) in \(\V\).
        \item\label{VS4}
        For each element \(x\) in \(\V\) there exists an element \(y\) in \(\V\) such that \(x + y = 0\).
        \item\label{VS5}
        For each element \(x\) in \(\V\), \(1x = x\).
        \item\label{VS6}
        For each pair of elements \(a, b\) in \(\F\) and each element \(x\) in \(\V\), \(\p{ab} x = a \p{bx}\).
        \item\label{VS7}
        For each element \(a\) in \(\F\) and each pair of elements \(x, y\) in \(\V\), \(a \p{x + y} = ax + ay\).
        \item\label{VS8}
        For each pair of elements \(a, b\) in \(\F\) and each element \(x\) in \(\V\), \(\p{a + b} x = ax + bx\).
    \end{enumerate}
    The elements \(x + y\) and \(ax\) are called the \textbf{sum} of \(x\) and \(y\) and the \textbf{product} of \(a\) and \(x\), respectively.
\end{defn}

\begin{defn}\label{1.2.2}
    The elements of the field \(\F\) are called \textbf{scalars} and the elements of the vector space \(\V\) are called \textbf{vectors}.
\end{defn}

\begin{note}
    A vector space is frequently discussed in the text without explicitly mentioning its field of scalars.
    The reader is cautioned to remember, however, that \emph{every vector space is regarded as a vector space over a given field, which is denoted by \(\F\)}.
    Occasionally we restrict our attention to the fields of real and complex numbers, which are denoted \(\R\) and \(\C\), respectively.
\end{note}

\begin{note}
    \ref{VS2} permits us to unambiguously define the addition of any finite number of vectors
    (without the use of parentheses).
\end{note}

\begin{defn}\label{1.2.3}
    An object of the form \(\tp{a}{n}\), where the entries \(a_{1}, a_{2}, \dots, a_{n}\) are elements of a field \(\F\), is called an \textbf{\(n\)-tuple} with entries from \(\F\).
    The elements \(a_{1}, a_{2}, \dots, a_{n}\) are called the \textbf{entries} or \textbf{components} of the \(n\)-tuple.
    Two \(n\)-tuples \(\tp{a}{n}\) and \(\tp{b}{n}\) with entries from a field \(\F\) are called \textbf{equal} if \(a_i = b_i\) for \(i = 1, 2, \dots, n\).
\end{defn}

\begin{eg}\label{1.2.4}
    The set of all \(n\)-tuples with entries from a field \(\F\) is denoted by \(\vs{F}^n\).
    This set is a vector space over \(\F\) with the operations of coordinatewise addition and scalar multiplication;
    that is, if \(u = \tp{a}{n} \in \vs{F}^n\), \(v = \tp{b}{n} \in \vs{F}^n\), and \(c \in \F\), then
    \[
        u + v = (a_{1} + b_{1}, a_{2} + b_{2}, \dots, a_{n} + b_{n}) \quad \text{ and } \quad cu = \tp{ca}{n}.
    \]
\end{eg}


\section{Subspaces}\label{sec:1.3}

\begin{defn}\label{1.3.1}
  A subset \(\W\) of a vector space \(\V\) over a field \(\F\) is called a \textbf{subspace} of \(\V\) over \(\F\) if \(\W\) is a vector space over \(\F\) with the operations of addition and scalar multiplication defined on \(\V\).
\end{defn}

\begin{eg}\label{1.3.2}
  In any vector space \(\V\) over \(\F\), note that \(\V\) and \(\set{\zv}\) are subspaces.
  The latter is called the \textbf{zero subspace} of \(\V\) over \(\F\).
\end{eg}

\begin{proof}
  Since \(\V \subseteq \V\) and \(\V\) is a vector space over \(\F\) with the operations of addition and scalar multiplication defined on \(\V\), by \cref{1.3.1} we know that \(\V\) is a subspace of \(\V\) over \(\F\).
  Since \(\zv \in \V\) (by \ref{vs3}), we know that \(\set{\zv} \subseteq \V\).
  Thus by \cref{ex:1.2.11} and \cref{1.3.1} \(\set{\zv}\) is a subspace of \(\V\) over \(\F\).
\end{proof}

\begin{thm}\label{1.3}
  Let \(\V\) be a vector space over \(\F\) and \(\W\) a subset of \(\V\).
  Then \(\W\) is a subspace of \(\V\) over \(\F\) if and only if the following three conditions hold for the operations defined in \(\V\).
  \begin{enumerate}
    \item \(\zv \in \W\).
    \item (\(\W\) is \textbf{closed under addition}.)
          \(x + y \in \W\) whenever \(x \in \W\) and \(y \in \W\).
    \item (\(\W\) is \textbf{closed under scalar multiplication}.)
          \(cx \in \W\) whenever \(c \in \F\) and \(x \in \W\).
  \end{enumerate}
\end{thm}

\begin{proof}
  If \(\W\) is a subspace of \(\V\) over \(\F\), then \(\W\) is a vector space over \(\F\) with the operations of addition and scalar multiplication defined on \(\V\).
  Hence conditions (b) and (c) hold, and there exists a vector \(\zv' \in \W\) such that \(x + \zv' = x\) for each \(x \in \W\).
  But also \(x + \zv = x\), and thus \(\zv' = \zv\) by \cref{1.1}.
  So condition (a) holds.

  Since properties \ref{vs1}, \ref{vs2}, \ref{vs5}, \ref{vs6}, \ref{vs7}, and \ref{vs8} hold for all vectors in the vector space, these properties automatically hold for the vectors in any subset.
  Thus if conditions (a), (b), and (c) hold, then \(\W\) is a subspace of \(\V\) over \(\F\) if the additive inverse of each vector in \(\W\) lies in \(\W\).
  But if \(x \in W\), then \(\p{-1} x \in \W\) by condition (c), and \(-x = \p{-1} x\) by \cref{1.2}.
  Hence \(\W\) is a subspace of \(\V\) over \(\F\).
\end{proof}

\begin{defn}\label{1.3.3}
  The \textbf{transpose} \(\tp{A}\) of an \(m \times n\) matrix \(A\) is the \(n \times m\) matrix obtained from \(A\) by interchanging the rows with the columns;
  that is, \(\p{\tp{A}}_{i j} = A_{j i}\).
\end{defn}

\begin{defn}\label{1.3.4}
  A \textbf{symmetric matrix} is a matrix \(A\) such that \(\tp{A} = A\).
  Clearly, a symmetric matrix must be square.
\end{defn}

\begin{eg}\label{1.3.5}
  The set \(\W\) of all symmetric matrices in \(\ms{n}{n}{\F}\) is a subspace of \(\ms{n}{n}{\F}\) over \(\F\).
\end{eg}

\begin{proof}
  Observe that
  \begin{itemize}
    \item The zero matrix is equal to its transpose and hence belongs to \(\W\).
    \item If \(A \in \W\) and \(B \in \W\), then \(\tp{A} = A\) and \(\tp{B} = B\).
          Thus by \cref{ex:1.3.3} \(\tp{\p{A + B}} = \tp{A} + \tp{B} = A + B\), so that \(A + B \in \W\).
    \item If \(A \in \W\), then \(\tp{A} = A\).
          So for any \(a \in \F\), we have \(\tp{\p{aA}} = a\tp{A} = aA\) by \cref{ex:1.3.3}
          Thus \(aA \in W\).
  \end{itemize}
  Thus by \cref{1.3} \(\W\) is a subspace of \(\ms{n}{n}{\F}\) over \(\F\).
\end{proof}

\begin{eg}\label{1.3.6}
  Let \(n\) be a nonnegative integer, and let \(\ps[n]{\F}\) consist of all polynomials in \(\ps{\F}\) having degree less than or equal to \(n\).
  Then \(\ps[n]{\F}\) is a subspace of \(\ps{\F}\) over \(\F\).
\end{eg}

\begin{proof}
  Since the zero polynomial has degree \(-1\), it is in \(\ps[n]{\F}\).
  Moreover, the sum of two polynomials with degrees less than or equal to \(n\) is another polynomial of degree less than or equal to \(n\), and the product of a scalar and a polynomial of degree less than or equal to \(n\) is a polynomial of degree less than or equal to \(n\).
  So \(\ps[n]{\F}\) is closed under addition and scalar multiplication.
  It therefore follows from \cref{1.3} that \(\ps[n]{\F}\) is a subspace of \(\ps{\F}\) over \(\F\).
\end{proof}

\begin{eg}\label{1.3.7}
  Let \(\cfs{\R}\) denote the set of all continuous real-valued functions defined on \(\R\).
  Clearly \(\cfs{\R}\) is a subset of the vector space \(\fs{\R}{\R}\) defined in \cref{1.2.10} of \cref{sec:1.2}.
  We claim that \(\cfs{\R}\) is a subspace of \(\fs{\R}{\R}\) over \(\R\).
\end{eg}

\begin{proof}
  First note that the zero of \(\fs{\R}{\R}\) is the constant function defined by \(f\p{t} = 0\) for all \(t \in \R\).
  Since constant functions are continuous, we have \(f \in \cfs{\R}\).
  Moreover, the sum of two continuous functions is continuous, and the product of a real number and a continuous function is continuous.
  So \(\cfs{\R}\) is closed under addition and scalar multiplication and hence is a subspace of \(\fs{\R}{\R}\) over \(\R\) by \cref{1.3}.
\end{proof}

\begin{eg}\label{1.3.8}
  An \(n \times n\) matrix \(M\) is called a \textbf{diagonal matrix} if \(M_{i j} = 0\) whenever \(i \neq j\), that is, if all its nondiagonal entries are zero.
  Then the set of diagonal matrices is a subspace of \(\ms{n}{n}{\F}\) over \(\F\).
\end{eg}

\begin{proof}
  Clearly the zero matrix is a diagonal matrix because all of its entries are \(0\).
  Moreover, if \(A\) and \(B\) are diagonal \(n \times n\) matrices, then whenever \(i \neq j\),
  \[
    \p{A + B}_{i j} = A_{i j} + B_{i j} = 0 + 0 = 0 \quad \text{and} \quad \p{cA}_{i j} = cA_{i j} = c0 = 0
  \]
  for any scalar \(c \in \F\).
  Hence \(A + B\) and \(cA\) are diagonal matrices for any scalar \(c \in \F\).
  Therefore the set of diagonal matrices is a subspace of \(\ms{n}{n}{\F}\) over \(\F\) by \cref{1.3}.
\end{proof}

\begin{eg}\label{1.3.9}
  The \textbf{trace} of an \(n \times n\) matrix \(M\), denoted \(\tr{M}\), is the sum of the diagonal entries of \(M\);
  that is,
  \[
    \tr{M} = M_{1 1} + M_{2 2} + \dots + M_{n n}.
  \]
  The set \(\W\) of \(n \times n\) matrices having trace equal to \(0\) is a subspace of \(\ms{n}{n}{\F}\) over \(\F\).
\end{eg}

\begin{proof}
  Clearly we have \(\W \subseteq \ms{n}{n}{\F}\), \(\tr{\zm} = 0\) and \(\zm \in \W\).
  Moreover, if \(A, B \in \W\), then
  \begin{align*}
    \tr{A + B} & = \p{A + B}_{1 1} + \p{A + B}_{2 2} + \dots + \p{A + B}_{n n}               &  & \text{(by \cref{1.3.9})}  \\
               & = A_{1 1} + B_{1 1} + A_{2 2} + B_{2 2} + \dots + A_{n n} + B_{n n}         &  & \text{(by \cref{1.2.9})}  \\
               & = A_{1 1} + A_{2 2} + \dots + A_{n n} + B_{1 1} + B_{2 2} + \dots + B_{n n} &  & (A_{i i}, B_{i i} \in \F) \\
               & = 0                                                                         &  & (A, B \in \W)
  \end{align*}
  and
  \begin{align*}
    \tr{cA} & = \p{cA}_{1 1} + \p{cA}_{2 2} + \dots + \p{cA}_{n n} &  & \text{(by \cref{1.3.9})} \\
            & = cA_{1 1} + cA_{2 2} + \dots + cA_{n n}             &  & \text{(by \cref{1.2.9})} \\
            & = c\p{A_{1 1} + A_{2 2} + A_{n n}}                   &  & (c, A_{i i} \in \F)      \\
            & = c0                                                 &  & (A \in \W)               \\
            & = 0                                                  &  & (c \in \F)
  \end{align*}
  for any scalar \(c \in \F\).
  Therefore \(\W\) is a subspace of \(\ms{n}{n}{\F}\) over \(\F\) by \cref{1.3}.
\end{proof}

\begin{thm}\label{1.4}
  Any intersection of subspaces of a vector space \(\V\) is a subspace of \(\V\).
\end{thm}

\begin{proof}
  Let \(\cvs\) be a collection of subspaces of \(\V\) over \(\F\), and let \(\W\) denote the intersection of the subspaces in \(\cvs\).
  Since every subspace contains the zero vector, \(\zv \in \W\).
  Let \(a \in \F\) and \(x, y \in \W\).
  Then \(x\) and \(y\) are contained in each subspace in \(\cvs\).
  Because each subspace in \(\cvs\) is closed under addition and scalar multiplication, it follows that \(x + y\) and \(ax\) are contained in each subspace in \(\cvs\).
  Hence \(x + y\) and \(ax\) are also contained in \(\W\), so that \(\W\) is a subspace of \(\V\) over \(\F\) by \cref{1.3}.
\end{proof}

\exercisesection

\setcounter{ex}{2}
\begin{ex}\label{ex:1.3.3}
  Prove that \(\tp{\p{aA + bB}} = a\tp{A} + b\tp{B}\) for any \(A, B \in \ms{m}{n}{\F}\) and any \(a, b \in \F\).
\end{ex}

\begin{proof}
  Let \(i = 1, \dots, n\) and \(j = 1, \dots, m\).
  Since
  \begin{align*}
    \p{a\tp{A} + b\tp{B}}_{i j} & = a\p{\tp{A}}_{i j} + b\p{\tp{B}}_{i j} &  & \text{(by \cref{1.2.9})} \\
                                & = aA_{j i} + bB_{j i}                   &  & \text{(by \cref{1.3.3})} \\
                                & = \p{aA}_{j i} + \p{bB}_{j i}           &  & \text{(by \cref{1.2.9})} \\
                                & = \p{aA + bB}_{j i}                     &  & \text{(by \cref{1.2.9})} \\
                                & = \tp{\p{aA + bB}}_{i j},               &  & \text{(by \cref{1.3.3})}
  \end{align*}
  by \cref{1.2.7} we know that \(\tp{\p{aA + bB}} = a\tp{A} + b\tp{B}\).
\end{proof}

\begin{ex}\label{ex:1.3.4}
  Prove that \(\tp{\p{\tp{A}}} = A\) for each \(A \in \MS\).
\end{ex}

\begin{proof}
  Let \(i = 1, \dots, m\) and \(j = 1, \dots, n\).
  Since
  \begin{align*}
    A_{i j} & = \p{\tp{A}}_{j i}           &  & \text{(by \cref{1.3.3})} \\
            & = \p{\tp{\p{\tp{A}}}}_{i j}, &  & \text{(by \cref{1.3.3})}
  \end{align*}
  by \cref{1.2.8} we know that \(\tp{\p{\tp{A}}} = A\).
\end{proof}

\begin{ex}\label{ex:1.3.5}
  Prove that \(A + \tp{A}\) is symmetric for any square matrix \(A \in \ms{n}{n}{\F}\).
\end{ex}

\begin{proof}
  Let \(i, j = 1, \dots, n\).
  Since
  \begin{align*}
    \p{\tp{\p{A + \tp{A}}}}_{i j} & = \p{A + \tp{A}}_{j i}       &  & \text{(by \cref{1.3.3})}  \\
                                  & = A_{j i} + \p{\tp{A}}_{j i} &  & \text{(by \cref{1.2.9})}  \\
                                  & = A_{j i} + A_{i j}          &  & \text{(by \cref{1.3.3})}  \\
                                  & = A_{i j} + A_{j i}          &  & (A_{i j}, A_{j i} \in \F) \\
                                  & = A_{i j} + \p{\tp{A}}_{i j} &  & \text{(by \cref{1.3.3})}  \\
                                  & = \p{A + \tp{A}}_{i j},      &  & \text{(by \cref{1.2.9})}
  \end{align*}
  by \cref{1.2.8} we know that \(\tp{\p{A + \tp{A}}} = A + \tp{A}\).
  Thus by \cref{1.3.4} \(A + \tp{A}\) is symmetric.
\end{proof}

\begin{ex}\label{ex:1.3.6}
  Prove that \(\tr{aA + bB} = a\tr{A} + b\tr{B}\) for any \(A, B \in \ms{n}{n}{\F}\).
\end{ex}

\begin{proof}
  We have
  \begin{align*}
     & \tr{aA + bB}                                                                                                         \\
     & = (aA + bB)_{1 1} + (aA + bB)_{2 2} + \dots + (aA + bB)_{n n}                       &  & \text{(by \cref{1.3.9})}    \\
     & = aA_{1 1} + bB_{1 1} + aA_{2 2} + bB_{2 2} + \dots + aA_{n n} + bB_{n n}           &  & \text{(by \cref{1.2.9})}    \\
     & = a\p{A_{1 1} + A_{2 2} + \dots + A_{n n}} + b\p{B_{1 1} + B_{2 2} + \dots B_{n n}} &  & (aA_{i i}, bB_{i i} \in \F) \\
     & = a\tr{A} + b\tr{B}.                                                                &  & \text{(by \cref{1.3.9})}
  \end{align*}
\end{proof}

\begin{ex}\label{ex:1.3.7}
  Prove that diagonal matrices are symmetric matrices.
\end{ex}

\begin{proof}
  Let \(A \in \ms{n}{n}{\F}\) be a diagonal matrices and let \(i, j = 1, \dots, n\).
  Then we have
  \begin{align*}
             & \begin{dcases}
      A_{i j} = A_{j i} = 0 & \text{if } i \neq j \\
      A_{i i} = A_{i i}     & \text{otherwise}
    \end{dcases}           &  & \text{(by \cref{1.3.8})} \\
    \implies & A_{i j} = A_{j i} = \p{\tp{A}}_{i j} &  & \text{(by \cref{1.3.3})} \\
    \implies & A = \tp{A}                           &  & \text{(by \cref{1.2.8})}
  \end{align*}
  and thus by \cref{1.3.4} diagonal matrices are symmetric.
\end{proof}
\chapter{Linear Transformations and Matrices}\label{ch:2}

% All sections are in separated files.  We include them here.
\section{Linear Transformations, Null Spaces and Ranges}\label{sec:2.1}

\begin{defn}\label{2.1.1}
  Let \(\V\) and \(\W\) be vector spaces over \(\F\).
  We call a function \(\T : \V \to \W\) a \textbf{linear transformation from \(\V\) to \(\W\)} if, for all \(x, y \in \V\) and \(c \in \F\), we have
  \begin{enumerate}
    \item \(\T(x + y) = \T(x) + \T(y)\) and
    \item \(\T(cx) = c\T(x)\).
  \end{enumerate}
  We often simply call \(\T\) \textbf{linear}.
\end{defn}

\begin{note}
  If the underlying field \(\F\) is the field of rational numbers, then \cref{2.1.1}(a) implies \cref{2.1.1}(b) (see \cref{ex:2.1.37}), but, in general \cref{2.1.1}(a)(b) are logically independent.
\end{note}

\begin{prop}\label{2.1.2}
  Let \(\V\) and \(\W\) be vector spaces over \(\F\) and let \(\T : \V \to \W\) be a function.
  \begin{enumerate}
    \item If \(\T\) is linear, then \(\T(\zv_v) = \zv_w\).
    \item \(\T\) is linear if and only if \(\T(cx + y) = c\T(x) + \T(y)\) for all \(x, y \in \V\) and \(c \in \F\).
    \item If \(\T\) is linear, then \(\T(x - y) = \T(x) - \T(y)\) for all \(x, y \in \V\).
    \item \(\T\) is linear if and only if, for \(\seq{x}{1,2,,n} \in \V\) and \(\seq{a}{1,2,,n} \in F\), we have
          \[
            \T\pa{\sum_{i = 1}^n a_i x_i} = \sum_{i = 1}^n a_i \T(x_i).
          \]
  \end{enumerate}
\end{prop}

\begin{proof}[\pf{2.1.2}(a)]
  We have
  \begin{align*}
             & \T \text{ is linear}                                                       \\
    \implies & \T(\zv_v) + \T(\zv_v) = \T(\zv_v + \zv_v) &  & \text{(by \cref{2.1.1}(a))} \\
             & = \T(\zv_v) = \T(\zv_v) + \zv_w           &  & \text{(by \ref{vs3})}       \\
    \implies & \T(\zv_v) = \zv_w.                        &  & \text{(by \cref{1.1})}
  \end{align*}
\end{proof}

\begin{proof}[\pf{2.1.2}(b)]
  We have
  \begin{align*}
             & \T \text{ is linear}                                                                                  \\
    \implies & \begin{dcases}
      \forall x, y \in \V \\
      \forall c \in \F
    \end{dcases}, \begin{dcases}
      \T(x + y) = \T(x) + \T(y) \\
      \T(cx) = c\T(x)
    \end{dcases}                    &  & \text{(by \cref{2.1.1})} \\
    \implies & \begin{dcases}
      \forall x, y \in \V \\
      \forall c \in \F
    \end{dcases}, \T(cx + y) = \T(cx) + \T(y) = c\T(x) + \T(y) &  & \text{(by \cref{2.1.1})}
  \end{align*}
  and
  \begin{align*}
             & \begin{dcases}
      \forall x, y \in \V \\
      \forall c \in \F
    \end{dcases}, \T(cx + y) = c\T(x) + \T(y)                                  \\
    \implies & \begin{dcases}
      \forall x, y \in \V \\
      \forall c \in \F
    \end{dcases}, \begin{dcases}
      \T(x + y) = \T(x) + \T(y)       & \text{if } c = 0     \\
      \T(cx + \zv_v) = c\T(x) + \zv_w & \text{if } y = \zv_v
    \end{dcases}  &  & \text{(by \cref{2.1.2}(a))} \\
    \implies & \begin{dcases}
      \forall x, y \in \V \\
      \forall c \in \F
    \end{dcases}, \begin{dcases}
      \T(x + y) = \T(x) + \T(y) & \text{if } c = 0     \\
      \T(cx) = c\T(x)           & \text{if } y = \zv_v
    \end{dcases} &  & \text{(by \ref{vs3})}       \\
    \implies & \T \text{ is linear}.                                  &  & \text{(by \cref{2.1.1})}
  \end{align*}
  Thus
  \[
    \T \text{ is linear} \iff \begin{dcases}
      \forall x, y \in \V \\
      \forall c \in \F
    \end{dcases}, \T(cx + y) = c\T(x) + \T(y).
  \]
\end{proof}

\begin{proof}[\pf{2.1.2}(c)]
  For all \(x, y \in \V\), we have
  \begin{align*}
    \T(x - y) & = \T(x + (-1)y)      &  & \text{(by \cref{1.2}(b))}   \\
              & = \T(x) + \T((-1)y)  &  & \text{(by \cref{2.1.1}(a))} \\
              & = \T(x) + (-1) \T(y) &  & \text{(by \cref{2.1.1}(b))} \\
              & = \T(x) - \T(y).     &  & \text{(by \cref{1.2}(b))}
  \end{align*}
\end{proof}

\begin{proof}[\pf{2.1.2}(d)]
  We have
  \begin{align*}
             & \T \text{ is linear}                                                                                                  \\
    \implies & \begin{dcases}
      \forall \seq{x}{1,2,,n} \in \V \\
      \forall \seq{a}{1,2,,n} \in \F
    \end{dcases},                                                                                           \\
             & \T\pa{\sum_{i = 1}^n a_i x_i} = \sum_{i = 1}^n \T(a_i x_i) = \sum_{i = 1}^n a_i \T(x_i) &  & \text{(by \cref{2.1.1})}
  \end{align*}
  and
  \begin{align*}
             & \begin{dcases}
      \forall \seq{x}{1,2,,n} \in \V \\
      \forall \seq{a}{1,2,,n} \in \F
    \end{dcases},                                                                 \\
             & \T\pa{\sum_{i = 1}^n a_i x_i} = \sum_{i = 1}^n a_i \T(x_i) &  & \text{(by \cref{2.1.1})}    \\
    \implies & \begin{dcases}
      \forall x, y \in \V \\
      \forall c \in \F
    \end{dcases},                                                                 \\
             & \T(cx + 1y) = c\T(x) + 1\T(y) = c\T(x) + \T(y)             &  & \text{(by \ref{vs5})}       \\
    \implies & \T \text{ is linear}.                                      &  & \text{(by \cref{2.1.2}(b))}
  \end{align*}
  Thus
  \[
    \T \text{ is linear} \iff \begin{dcases}
      \forall \seq{x}{1,2,,n} \in \V \\
      \forall \seq{a}{1,2,,n} \in \F
    \end{dcases}, \T\pa{\sum_{i = 1}^n a_i x_i} = \sum_{i = 1}^n a_i \T(x_i).
  \]
\end{proof}

\begin{note}
  We generally use \cref{2.1.2}(b) to prove that a given transformation is linear.
\end{note}

\begin{eg}\label{2.1.3}
  For any angle \(\theta\), define \(\T_{\theta} : \R^2 \to \R^2\) by the rule: \(\T_{\theta}(a_1, a_2)\) is the vector obtained by rotating \((a_1, a_2)\) counterclockwise by \(\theta\) if \((a_1, a_2) \neq (0, 0)\), and \(\T_{\theta}(0, 0) = (0, 0)\).
  Then \(\T_{\theta} : \R^2 \to \R^2\) is a linear transformation that is called the \textbf{rotation by \(\theta\)}.

  We determine an explicit formula for \(\T_{\theta}\).
  Fix a nonzero vector \((a_1, a_2) \in \R^2\).
  Let \(\alpha\) be the angle that \((a_1, a_2)\) makes with the positive \(x\)-axis, and let \(r = \sqrt{a_1^2 +a_2^2}\).
  Then \(a_1 = r \cos(\alpha)\) and \(a_2 = r \sin(\alpha)\).
  Also, \(\T_{\theta}(a_1, a_2)\) has length \(r\) and makes an angle \(\alpha + \theta\) with the positive \(x\)-axis.
  It follows that
  \begin{align*}
    \T_{\theta}(a_1, a_2) & = (r \cos(\alpha + \theta), r \sin(\alpha + \theta))                                                                     \\
                          & = (r \cos(\alpha) \cos(\theta) - r \sin(\alpha) \sin(\theta), r \cos(\alpha) \sin(\theta) + r \sin(\alpha) \cos(\theta)) \\
                          & = (a_1 \cos(\theta) - a_2 \sin(\theta), a_1 \sin(\theta) + a_2 \cos(\theta)).
  \end{align*}
  Finally, observe that this same formula is valid for \((a_1 ,a_2) = (0, 0)\).
  It is now easy to show that \(\T_{\theta}\) is linear.
\end{eg}

\begin{proof}[\pf{2.1.3}]
  For all \(x, y \in \R^2\) and \(c \in \R\), we have
  \begin{align*}
    \T_{\theta}(cx + y) & = \T_{\theta}(cx_1 + y_1, cx_2 + y_2)                                              &  & \text{(by \cref{1.2.4})} \\
                        & = ((cx_1 + y_1) \cos(\theta) - (cx_2 + y_2)\sin(\theta),                                                         \\
                        & \quad (cx_1 + y_1) \sin(\theta) + (cx_2 + y_2) \cos(\theta))                       &  & \text{(by \cref{2.1.3})} \\
                        & = c (x_1 \cos(\theta) - x_2 \sin(\theta), x_1 \sin(\theta) + x_2 \cos(\theta))     &  & \text{(by \cref{1.2.1})} \\
                        & \quad + (y_1 \cos(\theta) - y_2 \sin(\theta), y_1 \sin(\theta) + y_2 \cos(\theta))                               \\
                        & = c\T(x_1, x_2) + \T(y_1, y_2)                                                     &  & \text{(by \cref{2.1.3})} \\
                        & = c\T_{\theta}(x) + \T_{\theta}(y).                                                &  & \text{(by \cref{1.2.4})}
  \end{align*}
  Thus by \cref{2.1.2}(b) \(\T_{\theta}\) is linear.
\end{proof}

\begin{eg}\label{2.1.4}
  Define \(\T : \R^2 \to \R^2\) by \(\T(a_1, a_2) = (a_1, -a_2)\).
  \(\T\) is called the \textbf{reflection about the \(x\)-axis} and \(\T\) is linear.
\end{eg}

\begin{proof}[\pf{2.1.4}]
  For all \(x, y \in \R^2\) and \(c \in \R\), we have
  \begin{align*}
    \T(cx + y) & = \T(cx_1 + y_1, cx_2 + y_2)   &  & \text{(by \cref{1.2.4})} \\
               & = (cx_1 + y_1, -cx_2 - y_2)    &  & \text{(by \cref{2.1.4})} \\
               & = c(x_1, -x_2) + (y_1, -y_2)   &  & \text{(by \cref{1.2.1})} \\
               & = c\T(x_1, x_2) + \T(y_1, y_2) &  & \text{(by \cref{2.1.4})} \\
               & = c\T(x) + \T(y).              &  & \text{(by \cref{1.2.4})}
  \end{align*}
  Thus by \cref{2.1.2}(b) \(\T\) is linear.
\end{proof}

\begin{eg}\label{2.1.5}
  Define \(\T : \R^2 \to \R^2\) by \(\T(a_1, a_2) = (a_1, 0)\).
  \(\T\) is called the \textbf{projection on the \(x\)-axis} and \(\T\) is linear.
\end{eg}

\begin{proof}[\pf{2.1.5}]
  For all \(x, y \in \R^2\) and \(c \in \R\), we have
  \begin{align*}
    \T(cx + y) & = \T(cx_1 + y_1, cx_2 + y_2)   &  & \text{(by \cref{1.2.4})} \\
               & = (cx_1 + y_1, 0)              &  & \text{(by \cref{2.1.5})} \\
               & = c(x_1, 0) + (y_1, 0)         &  & \text{(by \cref{1.2.1})} \\
               & = c\T(x_1, x_2) + \T(y_1, y_2) &  & \text{(by \cref{2.1.5})} \\
               & = c\T(x) + \T(y).              &  & \text{(by \cref{1.2.4})}
  \end{align*}
  Thus by \cref{2.1.2}(b) \(\T\) is linear.
\end{proof}

\begin{eg}\label{2.1.6}
  Define \(\T : \ms{m}{n}{\F} \to \ms{n}{m}{\F}\) by \(\T(A) = \tp{A}\), where \(\tp{A}\) is the transpose of \(A\), defined in \cref{1.3.3}.
  Then \(\T\) is linear.
\end{eg}

\begin{proof}[\pf{2.1.6}]
  By \cref{ex:1.3.3} and \cref{2.1.2}(b) we see that \(\T\) is linear.
\end{proof}

\begin{eg}\label{2.1.7}
  Define \(\T : \ps[n]{\R} \to \ps[n - 1]{\R}\) by \(\T(f(x)) = f'(x)\), where \(f'(x)\) denotes the derivative of \(f(x)\).
  Then \(\T\) is linear.
\end{eg}

\begin{proof}[\pf{2.1.7}]
  Let \(g(x), h(x) \in \ps[n]{\R}\) and \(a \in \R\).
  Now
  \[
    \T(ag(x) + h(x)) = (ag(x) + h(x))' = ag'(x) + h'(x) = a\T(g(x)) + \T(h(x)).
  \]
  So by \cref{2.1.2}(b) \(\T\) is linear.
\end{proof}

\begin{eg}\label{2.1.8}
  Let \(\V = \cfs{\R}\), the vector space of continuous real-valued functions on \(\R\).
  Let \(a, b \in \R\), \(a < b\).
  Define \(\T : \V \to \R\) by
  \[
    \T(f) = \int_a^b f(t) \; dt
  \]
  for all \(f \in \V\).
  Then \(\T\) is linear.
\end{eg}

\begin{proof}[\pf{2.1.8}]
  Let \(g, h \in \cfs{\R}\) and \(a \in \R\).
  Now
  \begin{align*}
    \T(cg + h) & = \int_a^b (cg + h)(t) \; dt                  &  & \text{(by \cref{2.1.8})}  \\
               & = \int_a^b cg(t) + h(t) \; dt                 &  & \text{(by \cref{1.2.10})} \\
               & = c \int_a^b g(t) \; dt + \int_a^b h(t) \; dt                                \\
               & = c \T(g) + \T(h).                            &  & \text{(by \cref{2.1.8})}
  \end{align*}
  So by \cref{2.1.2}(b) \(\T\) is linear.
\end{proof}

\exercisesection

\begin{ex}\label{ex:2.1.37}
  A function \(\T : \V \to \W\) between vector spaces \(\V\) and \(\W\) over \(\F\) is called \textbf{additive} if \(\T(x + y) = \T(x) + \T(y)\) for all \(x, y \in \V\).
  Prove that if \(\V\) and \(\W\) are vector spaces over the field of rational numbers \(\Q\), then any additive function from \(\V\) into \(\W\) is a linear transformation.
\end{ex}

\section{The Matrix Representation of a Linear Transformation}\label{sec:2.2}

\begin{note}
  In \cref{sec:2.2}, we embark on one of the most useful approaches to the analysis of a linear transformation on a finite-dimensional vector space:
  the representation of a linear transformation by a matrix.
  In fact, we develop a one-to-one correspondence between matrices and linear transformations that allows us to utilize properties of one to study properties of the other.
\end{note}

\begin{defn}\label{2.2.1}
  Let \(\V\) be a finite-dimensional vector space over \(\F\).
  An \textbf{ordered basis} for \(\V\) over \(\F\) is a basis for \(\V\) over \(\F\) endowed with a specific order;
  that is, an ordered basis for \(\V\) over \(\F\) is a finite sequence of linearly independent vectors in \(\V\) that generates \(\V\).
\end{defn}

\begin{defn}\label{2.2.2}
  For the vector space \(\vs{F}^n\), we call \(\set{\seq{e}{1,2,,n}}\) the \textbf{standard ordered basis} for \(\vs{F}^n\) over \(\F\).
  Similarly, for the vector space \(\ps[n]{\F}\), we call \(\set{1, x, \dots, x^n}\) the \textbf{standard ordered basis} for \(\ps[n]{\F}\) over \(\F\).
\end{defn}

\begin{defn}\label{2.2.3}
  Let \(\beta = \set{\seq{u}{1,2,,n}}\) be an ordered basis for a finite-dimensional vector space \(\V\) over \(\F\).
  For \(x \in \V\), let \(\seq{a}{1,2,,n} \in \F\) be the unique scalars such that
  \[
    x = \sum_{i = 1}^n a_i u_i.
  \]
  We define the \textbf{coordinate vector of \(x\) relative to \(\beta\)}, denoted \([x]_{\beta}\), by
  \[
    [x]_{\beta} = \begin{pmatrix}
      a_1    \\
      a_2    \\
      \vdots \\
      a_n
    \end{pmatrix}.
  \]
\end{defn}

\begin{defn}\label{2.2.4}
  Suppose that \(\V\) and \(\W\) are finite-dimensional vector spaces over \(\F\) with ordered bases \(\beta = \set{\seq{v}{1,2,,n}}\) and \(\gamma = \set{\seq{w}{1,2,,m}}\) over \(\F\), respectively.
  Let \(\T : \V \to \W\) be linear.
  Then for each \(j\), \(1 \leq j \leq n\), there exist unique scalars \(a_{i j} \in \F\), \(1 \leq i \leq m\), such that
  \[
    \T(v_j) = \sum_{i = 1}^m a_{i j} w_i \quad \text{for } 1 \leq j \leq n.
  \]
  We call the \(m \times n\) matrix \(A\) defined by \(A_{i j} = a_{i j}\) the \textbf{matrix representation of \(\T\) in the ordered bases \(\beta\) and \(\gamma\)} and write \(A = [\T]_{\beta}^{\gamma}\).
  If \(\V = \W\) and \(\beta = \gamma\), then we write \(A = [\T]_{\beta}\).
\end{defn}

\begin{note}
  Notice that the \(j\)th column of \(A\) is simply \([\T(v_j)]_{\gamma}\).
  Also observe that if \(\U : \V \to \W\) is a linear transformation such that \([\U]_{\beta}^{\gamma} = [\T]_{\beta}^{\gamma}\), then \(\U = \T\) by \cref{2.1.13}.
\end{note}

\begin{defn}\label{2.2.5}
  Let \(\T, \U : \V \to \W\) be arbitrary functions, where \(\V\) and \(\W\) are vector spaces over \(\F\), and let \(a \in \F\).
  We define \(\T + \U : \V \to \W\) by \((\T + \U)(x) = \T(x) + \U(x)\) for all \(x \in \V\), and \(a \T: \V \to \W\) by \((a \T)(x) = a \T(x)\) for all \(x \in \V\).
\end{defn}

\begin{thm}\label{2.7}
  Let \(\V\) and \(\W\) be vector spaces over a field \(\F\), and let \(\T, \U : \V \to \W\) be linear.
  \begin{enumerate}
    \item For all \(a \in \F\), \(a \T + \U\) is linear.
    \item Using the operations of addition and scalar multiplication in \cref{2.2.5}, the collection of all linear transformations from \(\V\) to \(\W\) is a vector space over \(\F\).
  \end{enumerate}
\end{thm}

\begin{proof}[\pf{2.7}(a)]
  Let \(x, y \in \V\) and \(c \in \F\).
  Then
  \begin{align*}
    (a \T + \U)(cx + y) & = a \T(cx + y) + \U(cx + y)            &  & \text{(by \cref{2.2.5})}    \\
                        & = a(\T(cx + y)) + c \U(x) + \U(y)      &  & \text{(by \cref{2.1.2}(b))} \\
                        & = a(c \T(x) + \T(y)) + c \U(x) + \U(y) &  & \text{(by \cref{2.1.2}(b))} \\
                        & = ac \T(x) + c \U(x) + a \T(y) + \U(y) &  & \text{(by \cref{1.2.1})}    \\
                        & = c (a \T + \U)(x) + (a \T + \U)(y).   &  & \text{(by \cref{2.2.5})}
  \end{align*}
  So \(a \T + \U\) is linear.
\end{proof}

\begin{proof}[\pf{2.7}(b)]
  Let \(\ls(\V, \W)\) be the set of all linear transformation from \(\V\) to \(\W\).
  First we show that \ref{vs1}--\ref{vs8} is true.
  Let \(f, g, h \in \ls(\V, \W)\) and let \(a, b \in \F\).
  Now we split into eight cases:
  \begin{description}
    \item[For \ref{vs1}:] We have
      \begin{align*}
        \forall x \in \V, (f + g)(x) & = f(x) + g(x) &  & \text{(by \cref{2.2.5})} \\
                                     & = g(x) + f(x) &  & \text{(by \ref{vs1})}    \\
                                     & = (g + f)(x)  &  & \text{(by \cref{2.2.5})}
      \end{align*}
      and thus \(f + g = g + f\).
    \item[For \ref{vs2}:] We have
      \begin{align*}
        \forall x \in \V, ((f + g) + h)(x) & = (f + g)(x) + h(x)    &  & \text{(by \cref{2.2.5})} \\
                                           & = (f(x) + g(x)) + h(x) &  & \text{(by \cref{2.2.5})} \\
                                           & = f(x) + (g(x) + h(x)) &  & \text{(by \ref{vs2})}    \\
                                           & = f(x) + (g + h)(x)    &  & \text{(by \cref{2.2.5})} \\
                                           & = (f + (g + h))(x)     &  & \text{(by \cref{2.2.5})}
      \end{align*}
      and thus \((f + g) + h = f + (g + h)\).
    \item[For \ref{vs3}:] We have
      \begin{align*}
        \forall x \in \V, (f + \zT)(x) & = f(x) + \zT(x)   &  & \text{(by \cref{2.2.5})} \\
                                       & = f(x) + \zv_{\W} &  & \text{(by \cref{2.1.9})} \\
                                       & = f(x)            &  & \text{(by \ref{vs3})}
      \end{align*}
      and thus \(f + \zT = f\).
    \item[For \ref{vs4}:] We have
      \begin{align*}
        \forall x \in \V, (f + ((-1)f))(x) & = f(x) + ((-1)f)(x) &  & \text{(by \cref{2.2.5})} \\
                                           & = f(x) + (-1)f(x)   &  & \text{(by \cref{2.2.5})} \\
                                           & = \zv_{\W}          &  & \text{(by \ref{vs4})}    \\
                                           & = \zT(x)            &  & \text{(by \cref{2.1.9})}
      \end{align*}
      and thus \(f + (-1)f = \zT\).
    \item[For \ref{vs5}:] We have
      \begin{align*}
        \forall x \in \V, (1f)(x) & = 1f(x) &  & \text{(by \cref{2.2.5})} \\
                                  & = f(x)  &  & \text{(by \ref{vs5})}
      \end{align*}
      and thus \(1f = f\).
    \item[For \ref{vs6}:] We have
      \begin{align*}
        \forall x \in \V, ((ab)f)(x) & = (ab) f(x)   &  & \text{(by \cref{2.2.5})} \\
                                     & = a (bf(x))   &  & \text{(by \ref{vs6})}    \\
                                     & = a ((bf)(x)) &  & \text{(by \cref{2.2.5})} \\
                                     & = (a (bf))(x) &  & \text{(by \cref{2.2.5})}
      \end{align*}
      and thus \((ab)f = a (bf)\).
    \item[For \ref{vs7}:] We have
      \begin{align*}
        \forall x \in \V, (a(f + g))(x) & = a((f + g)(x))     &  & \text{(by \cref{2.2.5})} \\
                                        & = a(f(x) + g(x))    &  & \text{(by \cref{2.2.5})} \\
                                        & = af(x) + ag(x)     &  & \text{(by \ref{vs7})}    \\
                                        & = (af)(x) + (ag)(x) &  & \text{(by \cref{2.2.5})} \\
                                        & = (af + ag)(x)      &  & \text{(by \cref{2.2.5})}
      \end{align*}
      and thus \(a(f + g) = af + ag\).
    \item[For \ref{vs8}:] We have
      \begin{align*}
        \forall x \in \V, ((a + b)f)(x) & = (a + b) f(x)    &  & \text{(by \cref{2.2.5})} \\
                                        & = af(x) + bf(x)   &  & \text{(by \ref{vs8})}    \\
                                        & = af(x) + (bf)(x) &  & \text{(by \cref{2.2.5})} \\
                                        & = (af + bf)(x)    &  & \text{(by \cref{2.2.5})}
      \end{align*}
      and thus \((a + b)f = af + bf\).
  \end{description}
  From the proofs above we see that \ref{vs1}--\ref{vs8} is true.

  By \cref{2.7}(a) and the proofs above we see that
  \[
    \forall \T, \U \in \ls(\V, \W), \T + \U = 1 \T + \U \in \ls(\V, \W)
  \]
  and
  \[
    \forall (c, \T) \in \F \times \ls(\V, \W), c \T = c \T + \zT \in \ls(\V, \W).
  \]
  Thus by \cref{1.2.1} \(\ls(\V, \W)\) is a vector space over \(\F\).
\end{proof}

\begin{defn}\label{2.2.6}
  Let \(\V\) and \(\W\) be vector spaces over \(\F\).
  We denote the vector space of all linear transformations from \(\V\) into \(\W\) by \(\ls(\V, \W)\).
  In the case that \(\V = \W\), we write \(\ls(\V)\) instead of \(\ls(\V, \W)\).
\end{defn}

\begin{thm}\label{2.8}
  Let \(\V\) and \(\W\) be finite-dimensional vector spaces over \(\F\) with ordered bases \(\beta\) and \(\gamma\) over \(\F\), respectively, and let \(\T, \U : \V \to \W\) be linear transformations.
  Then
  \begin{enumerate}
    \item \([\T + \U]_{\beta}^{\gamma} = [\T]_{\beta}^{\gamma} + [\U]_{\beta}^{\gamma}\) and
    \item \([a \T]_{\beta}^{\gamma} = a [\T]_{\beta}^{\gamma}\) for all scalars \(a \in \F\).
  \end{enumerate}
\end{thm}

\begin{proof}[\pf{2.8}(a)]
  Let \(\beta = \set{\seq{v}{1,2,,n}}\) and \(\gamma = \set{\seq{w}{1,2,,m}}\).
  There exist unique scalars \(a_{i j}\) and \(b_{i j}\) (\(1 \leq i \leq m\), \(1 \leq j \leq n\)) such that
  \[
    \T(v_j) = \sum_{i = 1}^m a_{i j} w_i \quad \text{and} \quad \U(v_j) = \sum_{i = 1}^m b_{i j} w_i \quad \text{for } 1 \leq j \leq n.
  \]
  Hence
  \[
    (\T + \U)(v_j) = \sum_{i = 1}^m (a_{i j} + b_{i j}) w_i.
  \]
  Thus
  \[
    ([\T + \U]_{\beta}^{\gamma})_{i j} = a_{i j} + b_{i j} = ([\T]_{\beta}^{\gamma} + [\U]_{\beta}^{\gamma})_{i j}.
  \]
\end{proof}

\begin{proof}[\pf{2.8}(b)]
  Let \(\beta = \set{\seq{v}{1,2,,n}}\).
  There exist unique scalars \(c_{i j}\) (\(1 \leq i \leq m\), \(1 \leq j \leq n\)) such that
  \[
    \T(v_j) = \sum_{i = 1}^m c_{i j} w_i \quad \text{for } 1 \leq j \leq n.
  \]
  Hence
  \[
    (a \T)(v_j) = \sum_{i = 1}^m (a c_{i j}) w_i.
  \]
  Thus
  \[
    ([a \T]_{\beta}^{\gamma})_{i j} = a c_{i j} = a ([\T]_{\beta}^{\gamma})_{i j}.
  \]
\end{proof}

\exercisesection

\setcounter{ex}{7}
\begin{ex}\label{ex:2.2.8}
  Let \(\V\) be an \(n\)-dimensional vector space over \(\F\) with an ordered basis \(\beta\) over \(\F\).
  Define \(\T : \V \to \vs{F}^n\) by \(\T(x) = [x]_{\beta}\).
  Prove that \(\T\) is linear.
\end{ex}

\begin{proof}[\pf{ex:2.2.8}]
  Let \(\beta = \set{\seq{v}{1,2,,n}}\), let \(x, y \in \V\) and let \(c \in \F\).
  Since
  \begin{align*}
             & \exists \seq{a}{1,2,,n}, \seq{b}{1,2,,n} \in \F : \begin{dcases}
                                                                   x = \sum_{i = 1}^n a_i v_i \\
                                                                   y = \sum_{i = 1}^n b_i v_i
                                                                 \end{dcases} &  & \text{(by \cref{1.8})}       \\
    \implies & cx + y = \sum_{i = 1}^n (ca_i + b_i) v_i                         &  & \text{(by \cref{1.2.1})}   \\
    \implies & \T(cx + y) = [cx + y]_{\beta} = \begin{pmatrix}
                                                 ca_1 + b_1 \\
                                                 \vdots     \\
                                                 ca_n + b_n
                                               \end{pmatrix}                  &  & \text{(by \cref{2.2.3})}     \\
             & = c \begin{pmatrix}
                     a_1    \\
                     \vdots \\
                     a_n
                   \end{pmatrix} + \begin{pmatrix}
                                     b_1    \\
                                     \vdots \\
                                     b_n
                                   \end{pmatrix}                                  &  & \text{(by \cref{1.2.9})} \\
             & = c [x]_{\beta} + [y]_{\beta} = c \T(x) + \T(y),                 &  & \text{(by \cref{2.2.3})}
  \end{align*}
  by \cref{2.1.2}(b) we know that \(\T\) is linear.
\end{proof}

\begin{ex}\label{ex:2.2.9}
  Let \(\V\) be the vector space of complex numbers \(\C\) over the field \(\R\).
  Define \(\T : \V \to \V\) by \(\T(z) = \overline{z}\), where \(\overline{z}\) is the complex conjugate of \(z\).
  Prove that \(\T\) is linear, and compute \([\T]_{\beta}\), where \(\beta = \set{1, i}\).
  (Recall by \cref{ex:2.1.38} that \(\T\) is not linear if \(\V\) is regarded as a vector space over the field \(\C\).)
\end{ex}

\begin{proof}[\pf{ex:2.2.9}]
  Let \(x, y \in \C\) and let \(c \in \R\).
  Since
  \begin{align*}
    \T(cx + y) & = \overline{cx + y}                              \\
               & = \overline{cx} + \overline{y}                   \\
               & = \overline{c} \cdot \overline{x} + \overline{y} \\
               & = c \overline{x} + \overline{y}                  \\
               & = c \T(x) + \T(y),
  \end{align*}
  by \cref{2.1.2}(b) we know that \(\T\) is linear.
  Since
  \begin{align*}
    \T(1) & = 1 = 1 \cdot 1 + 0 \cdot i,     \\
    \T(i) & = -i = 0 \cdot 1 + (-1) \cdot i,
  \end{align*}
  by \cref{2.2.4} we know that
  \[
    [\T]_{\beta} = \begin{pmatrix}
      1 & 0  \\
      0 & -1
    \end{pmatrix}.
  \]
\end{proof}

\begin{ex}\label{ex:2.2.10}
  Let \(\V\) be a vector space over \(\F\) with the ordered basis \(\beta = \set{\seq{v}{1,2,,n}}\) over \(\F\).
  Define \(v_0 = 0\).
  By \cref{2.6} there exists a linear transformation \(\T : \V \to \V\) such that \(\T(v_j) = v_j + v_{j - 1}\) for \(j = 1, 2, \dots, n\).
  Compute \([\T]_{\beta}\).
\end{ex}

\begin{proof}[\pf{ex:2.2.10}]
  By \cref{2.2.4,ex:1.5.6} we have
  \[
    [\T]_{\beta} = \begin{pmatrix}
      1      & 1      & 0      & \cdots & 0      & 0      & 0      \\
      0      & 1      & 1      & \cdots & 0      & 0      & 0      \\
      0      & 0      & 1      & \cdots & 0      & 0      & 0      \\
      \vdots & \vdots & \vdots & \ddots & \vdots & \vdots & \vdots \\
      0      & 0      & 0      & \cdots & 1      & 1      & 0      \\
      0      & 0      & 0      & \cdots & 0      & 1      & 1      \\
      0      & 0      & 0      & \cdots & 0      & 0      & 1
    \end{pmatrix} = \sum_{i = 1}^n E^{i i} + \sum_{i = 2}^n E^{(i - 1) i}.
  \]
\end{proof}

\begin{ex}\label{ex:2.2.11}
  Let \(\V\) be an \(n\)-dimensional vector space over \(\F\), and let \(\T : \V \to \V\) be a linear transformation.
  Suppose that \(\W\) is a \(\T\)-invariant subspace of \(\V\) over \(\F\) (see \cref{2.1.15}) having dimension \(k\).
  Show that there is a basis \(\beta\) for \(\V\) over \(\F\) such that \([\T]_{\beta}\) has the form
  \[
    \begin{pmatrix}
      A   & B \\
      \zm & C
    \end{pmatrix},
  \]
  where \(A\) is a \(k \times k\) matrix and \(\zm\) is the \((n - k) \times k\) zero matrix.
\end{ex}

\begin{proof}[\pf{ex:2.2.11}]
  Let \(\beta_{\W} = \set{\seq{v}{1,2,,k}}\) be a basis for \(\W\) over \(\F\).
  By \cref{1.6.19} we can extend \(\beta_{\W}\) to \(\beta = \set{\seq{v}{1,2,,n}}\) such that \(\beta\) is a basis for \(\V\) over \(\F\).
  Since \(\W\) is \(\T\)-invariant, we know that
  \begin{align*}
             & \T(\W) \subseteq \W                                                                                               &  & \text{(by \cref{2.1.15})} \\
    \implies & \forall v_j \in \beta_{\W}, \T(v_j) \in \W = \spn{\beta_{\W}}                                                     &  & \text{(by \cref{1.6.1})}  \\
    \implies & \forall v_j \in \beta_{\W}, \exists a_{1 j}, a_{2 j}, \dots, a_{k j} \in \F :                                                                    \\
             & \T(v_j) = \sum_{i = 1}^k a_{i j} v_i = \sum_{i = 1}^k a_{i j} v_i + \sum_{i = k + 1}^n 0 v_i                      &  & \text{(by \cref{1.2}(a))} \\
    \implies & \forall v_j \in \beta_{\W}, \exists a_{1 j}, a_{2 j}, \dots, a_{k j} \in \F : [\T(v_j)]_{\beta} = \begin{pmatrix}
                                                                                                                   a_{1 j} \\
                                                                                                                   \vdots  \\
                                                                                                                   a_{k j} \\
                                                                                                                   0       \\
                                                                                                                   \vdots  \\
                                                                                                                   0
                                                                                                                 \end{pmatrix}. &  & \text{(by \cref{2.2.3})}
  \end{align*}
  By setting
  \begin{align*}
    A & = \begin{pmatrix}
            a_{1 1} & \cdots & a_{1 k} \\
            \vdots  & \ddots & \vdots  \\
            a_{k 1} & \cdots & a_{k k}
          \end{pmatrix} \in \ms{k}{k}{\F}                                                                    \\
    B & = \begin{pmatrix}
            ([\T(v_{k + 1})]_{\beta})_1 & \cdots & ([\T(v_n)]_{\beta})_1 \\
            \vdots                      & \ddots & \vdots                \\
            ([\T(v_{k + 1})]_{\beta})_k & \cdots & ([\T(v_n)]_{\beta})_k
          \end{pmatrix} \in \ms{k}{(n - k)}{\F}             \\
    C & = \begin{pmatrix}
            ([\T(v_{k + 1})]_{\beta})_{k + 1} & \cdots & ([\T(v_n)]_{\beta})_{k + 1} \\
            \vdots                            & \ddots & \vdots                      \\
            ([\T(v_{k + 1})]_{\beta})_n       & \cdots & ([\T(v_n)]_{\beta})_n
          \end{pmatrix} \in \ms{(n - k)}{(n - k)}{\F}
  \end{align*}
  we have
  \begin{align*}
    [\T]_{\beta} & = \begin{pmatrix}
                       ([\T(v_1)]_{\beta})_1 & \cdots & ([\T(v_n)]_{\beta})_1   \\
                       \vdots                & \ddots & \vdots                  \\
                       ([\T(v_1)]_{\beta})_n & \cdots & ([\T(v_n)]_{\beta})_{n}
                     \end{pmatrix}                                        &  & \text{(by \cref{2.2.4})}                                        \\
                 & = \begin{pmatrix}
                       a_{1 1} & \cdots & a_{1 k} & ([\T(v_{k + 1})]_{\beta})_1       & \cdots & ([\T(v_n)]_{\beta})_1       \\
                       \vdots  & \ddots & \vdots  & \vdots                            & \ddots & \vdots                      \\
                       a_{k 1} & \cdots & a_{k k} & ([\T(v_{k + 1})]_{\beta})_k       & \cdots & ([\T(v_n)]_{\beta})_k       \\
                       0       & \cdots & 0       & ([\T(v_{k + 1})]_{\beta})_{k + 1} & \cdots & ([\T(v_n)]_{\beta})_{k + 1} \\
                       \vdots  & \ddots & \vdots  & \vdots                            & \ddots & \vdots                      \\
                       0       & \cdots & 0       & ([\T(v_{k + 1})]_{\beta})_{n}     & \cdots & ([\T(v_n)]_{\beta})_{n}
                     \end{pmatrix} \\
                 & = \begin{pmatrix}
                       A   & B \\
                       \zm & C
                     \end{pmatrix}.
  \end{align*}
\end{proof}

\begin{ex}\label{ex:2.2.12}
  Let \(\V\) be a finite-dimensional vector space over \(\F\) and \(\T\) be the projection on \(\W\) along \(\W'\), where \(\W\) and \(\W'\) are subspaces of \(\V\) over \(\F\).
  (See \cref{2.1.14}.)
  Find an ordered basis \(\beta\) for \(\V\) over \(\F\) such that \([\T]_{\beta}\) is a diagonal matrix.
\end{ex}

\begin{proof}[\pf{ex:2.2.12}]
  Let \(\beta_{\W} = \set{\seq{v}{1,2,,k}}\) be a basis for \(\W\) over \(\F\).
  By \cref{1.6.19} we can extend \(\beta_{\W}\) to \(\beta = \set{\seq{v}{1,2,,n}}\) such that \(\beta\) is a basis for \(\V\) over \(\F\).
  Since
  \begin{align*}
             & \begin{dcases}
                 \forall v_j \in \beta_{\W}, v_j = v_j + \zv \in \W + \W' \\
                 \forall v_j \in \beta \setminus \beta_{\W}, v_j = \zv + v_j \in \W + \W'
               \end{dcases}                         &  & \text{(by \cref{1.3.10})}                         \\
    \implies & \begin{dcases}
                 \forall v_j \in \beta_{\W}, \T(v_j) = v_j \\
                 \forall v_j \in \beta \setminus \beta_{\W}, \T(v_j) = \zv
               \end{dcases}                                        &  & \text{(by \cref{2.1.14})}              \\
    \implies & \begin{dcases}
                 \forall v_j \in \beta_{\W}, [\T(v_j)]_{\beta} = e_j \in \vs{F}^n \\
                 \forall v_j \in \beta \setminus \beta_{\W}, [\T(v_j)]_{\beta} = \zv \in \vs{F}^n
               \end{dcases} &  & \text{(by \cref{2.2.2})} \\
    \implies & [\T]_{\beta} = \begin{pmatrix}
                                e_1 & \cdots & e_k & \zv & \cdots & \zv
                              \end{pmatrix}                                          \\
             & = \begin{pmatrix}
                   1      & 0      & \cdots & 0      & 0      & \cdots & 0      \\
                   0      & 1      & \cdots & 0      & 0      & \cdots & 0      \\
                   \vdots & \vdots & \ddots & \vdots & 0      & \cdots & 0      \\
                   0      & 0      & \cdots & 1      & 0      & \cdots & 0      \\
                   0      & 0      & \cdots & 0      & 0      & \cdots & 0      \\
                   \vdots & \vdots & \ddots & \vdots & \vdots & \ddots & \vdots \\
                   0      & 0      & \cdots & 0      & 0      & \cdots & 0
                 \end{pmatrix},
  \end{align*}
  by \cref{1.3.8} we know that \([\T]_{\beta}\) is a diagonal matrix.
\end{proof}

\section{Composition of Linear Transformations and Matrix Multiplication}\label{sec:2.3}

\begin{note}
  We use the more convenient notation of \(\U \T\) rather than \(\U \circ \T\) for the composite of linear transformations \(\U\) and \(\T\).
\end{note}

\begin{thm}\label{2.9}
  Let \(\V\), \(\W\), and \(\vs{Z}\) be vector spaces over the same field \(\F\), and let \(\T : \V \to \W\) and \(\U : \W \to \vs{Z}\) be linear.
  Then \(\U \T : \V \to \vs{Z}\) is linear.
\end{thm}

\begin{proof}[\pf{2.9}]
  Let \(x, y \in \V\) and \(a \in \F\).
  Then
  \begin{align*}
    \U \T(ax + y) & = \U(\T(ax + y))                                            \\
                  & = \U(a \T(x) + \T(y))      &  & \text{(by \cref{2.1.2}(b))} \\
                  & = a \U(\T(x)) + \U(\T(y))  &  & \text{(by \cref{2.1.2}(b))} \\
                  & = a (\U \T)(x) + \U \T(y).
  \end{align*}
\end{proof}

\begin{thm}\label{2.10}
  Let \(\V, \W, \vs{X}, \vs{Y}\) be vector spaces over \(\F\).
  Let \(\lt{S}, \lt{S}_1, \lt{S}_2 \in \ls(\vs{X}, \vs{Y})\), let \(\T \in \ls(\W, \vs{X})\) and let \(\U, \U_1, \U_2 \in \ls(\V, \W)\).
  Then
  \begin{enumerate}
    \item \(\T(\U_1 + \U_2) = \T \U_1 + \T \U_2\) and \((\lt{S}_1 + \lt{S}_2) \T = \lt{S}_1 \T + \lt{S}_2 \T\).
    \item \(\lt{S} (\T \U) = (\lt{S} \T) \U\).
    \item \(\T \IT[\W] = \IT[\vs{X}] \T = \T\).
    \item \(a(\T \U) = (a \T) \U = \T (a \U)\) for all scalars \(a \in \F\).
  \end{enumerate}
\end{thm}

\begin{proof}[\pf{2.10}(a)]
  For all \(x \in \V\), we have
  \begin{align*}
    (\T(\U_1 + \U_2))(x) & = \T((\U_1 + \U_2)(x))                                      \\
                         & = \T(\U_1(x) + \U_2(x))       &  & \text{(by \cref{2.2.5})} \\
                         & = \T(\U_1(x)) + \T(\U_2(x))   &  & \text{(by \cref{2.1.1})} \\
                         & = (\T \U_1)(x) + (\T \U_2)(x)                               \\
                         & = (\T \U_1 + \T \U_2)(x).     &  & \text{(by \cref{2.2.5})}
  \end{align*}
  Thus \(\T(\U_1 + \U_2) = \T \U_1 + \T \U_2\).
  For all \(x \in \W\), we have
  \begin{align*}
    ((\vs{S}_1 + \vs{S}_2) \T)(x) & = (\vs{S}_1 + \vs{S}_2)(\T(x))                                      \\
                                  & = \vs{S}_1(\T(x)) + \vs{S}_2(\T(x))   &  & \text{(by \cref{2.2.5})} \\
                                  & = (\vs{S}_1 \T)(x) + (\vs{S}_2 \T)(x)                               \\
                                  & = (\vs{S}_1 \T + \vs{S}_2 \T)(x).     &  & \text{(by \cref{2.2.5})}
  \end{align*}
  Thus \((\vs{S}_1 + \vs{S}_2) \T = \vs{S}_1 \T + \vs{S}_2 \T\).
\end{proof}

\begin{proof}[\pf{2.10}(b)]
  For all \(x \in \V\), we have
  \begin{align*}
    (\lt{S} (\T \U))(x) & = \lt{S}((\T \U)(x))  \\
                        & = \lt{S}(\T(\U(x)))   \\
                        & = (\lt{S} \T)(\U(x))  \\
                        & = ((\lt{S} \T) \U)(x)
  \end{align*}
  and thus \(\lt{S} (\T \U) = (\lt{S} \T) \U\).
\end{proof}

\begin{proof}[\pf{2.10}(c)]
  For all \(x \in \V\), we have
  \begin{align*}
    (\T \IT[\W])(x) & = \T(\IT[\W](x))                                    \\
                    & = \T(x)               &  & \text{(by \cref{2.1.9})} \\
                    & = \IT[\vs{X}](\T(x))  &  & \text{(by \cref{2.1.9})} \\
                    & = (\IT[\vs{X}] \T)(x)
  \end{align*}
  and thus \(\T \IT[\W] = \T = \IT[\vs{X}] \T\).
\end{proof}

\begin{proof}[\pf{2.10}(d)]
  For all \(x \in \V\), we have
  \begin{align*}
    (a (\T \U))(x) & = a (\T \U)(x)                                 \\
                   & = a \T(\U(x))                                  \\
                   & = (a \T)(\U(x))  &  & \text{(by \cref{2.2.5})} \\
                   & = ((a \T) \U)(x)                               \\
                   & = \T(a \U(x))    &  & \text{(by \cref{2.1.1})} \\
                   & = \T((a \U)(x))                                \\
                   & = (\T (a \U))(x)
  \end{align*}
  and thus \(a (\T \U) = (a \T) \U = \T (a \U)\).
\end{proof}

\begin{defn}\label{2.3.1}
  Let \(A \in \MS\) matrix and \(B \in \ms{n}{p}{\F}\).
  We define the \textbf{product} of \(A\) and \(B\), denoted \(AB\), to be the \(m \times p\) matrix such that
  \[
    (AB)_{i j} = \sum_{k = 1}^n A_{i k} B_{k j} \quad \text{for } 1 \leq i \leq m, 1 \leq j \leq p.
  \]
  Note that \((AB)_{i j}\) is the sum of products of corresponding entries from the \(i\)th row of \(A\) and the \(j\)th column of \(B\).
\end{defn}

\begin{note}
  The reader should observe that in order for the product \(AB\) to be defined, there are restrictions regarding the relative sizes of \(A\) and \(B\).
  The following mnemonic device is helpful:
  ``\((m \times n) \cdot (n \times p) = (m \times p)\)'';
  that is, in order for the product \(AB\) to be defined, the two ``inner'' dimensions must be equal, and the two ``outer'' dimensions yield the size of the product.
\end{note}

\begin{note}
  As in the case with composition of functions, we have that matrix multiplication is not commutative. Consider the following two products:
  \[
    \begin{pmatrix}
      1 & 1 \\
      0 & 0
    \end{pmatrix} \begin{pmatrix}
      0 & 1 \\
      1 & 0
    \end{pmatrix} = \begin{pmatrix}
      1 & 1 \\
      0 & 0
    \end{pmatrix} \quad \text{and} \quad \begin{pmatrix}
      0 & 1 \\
      1 & 0
    \end{pmatrix} \begin{pmatrix}
      1 & 1 \\
      0 & 0
    \end{pmatrix} = \begin{pmatrix}
      0 & 0 \\
      1 & 1
    \end{pmatrix}.
  \]
  Hence we see that even if both of the matrix products \(AB\) and \(BA\) are defined, it need not be true that \(AB = BA\).
\end{note}

\begin{eg}\label{2.3.2}
  If \(A \in \MS\) and \(B \in \ms{n}{p}{\F}\), then \(\tp{(AB)} = \tp{B} \tp{A}\).
\end{eg}

\begin{proof}[\pf{2.3.2}]
  Since
  \[
    \tp{(AB)}_{i j} = (AB)_{j i} = \sum_{k = 1}^n A_{j k} B_{k i}
  \]
  and
  \[
    (\tp{B} \tp{A})_{i j} = \sum_{k = 1}^n \tp{B}_{i k} \tp{A}_{k j} = \sum_{k = 1}^n B_{k i} A_{j k},
  \]
  we are finished.
  Therefore the transpose of a product is the product of the transposes in the \emph{opposite order}.
\end{proof}

\begin{thm}\label{2.11}
  Let \(\V\), \(\W\), and \(\vs{Z}\) be finite-dimensional vector spaces over \(\F\) with ordered bases \(\alpha\), \(\beta\), and \(\gamma\) over \(\F\), respectively.
  Let \(\T : \V \to \W\) and \(\U : \W \to \vs{Z}\) be linear transformations.
  Then
  \[
    [\U \T]_{\alpha}^{\gamma} = [\U]_{\beta}^{\gamma} [\T]_{\alpha}^{\beta}.
  \]
\end{thm}

\begin{proof}[\pf{2.11}]
  Define
  \begin{align*}
    \alpha & = \set{\seq{v}{1,2,,n}}; \\
    \beta  & = \set{\seq{w}{1,2,,m}}; \\
    \gamma & = \set{\seq{z}{1,2,,p}}.
  \end{align*}
  For \(1 \leq j \leq n\), we have
  \begin{align*}
    (\U \T)(v_j) & = \sum_{i = 1}^p ([\U \T]_{\alpha}^{\gamma})_{i j} \cdot z_i                                                      &  & \text{(by \cref{2.2.4})}    \\
                 & = \U(\T(v_j))                                                                                                                                      \\
                 & = \U\pa{\sum_{k = 1}^m ([\T]_{\alpha}^{\beta})_{k j} \cdot w_k}                                                   &  & \text{(by \cref{2.2.4})}    \\
                 & = \sum_{k = 1}^m ([\T]_{\alpha}^{\beta})_{k j} \cdot \U(w_k)                                                      &  & \text{(by \cref{2.1.2}(d))} \\
                 & = \sum_{k = 1}^m ([\T]_{\alpha}^{\beta})_{k j} \cdot \pa{\sum_{i = 1}^p ([\U]_{\beta}^{\gamma})_{i k} \cdot z_i}  &  & \text{(by \cref{2.2.4})}    \\
                 & = \sum_{k = 1}^m \sum_{i = 1}^p \pa{([\T]_{\alpha}^{\beta})_{k j}  \cdot ([\U]_{\beta}^{\gamma})_{i k} \cdot z_i} &  & \text{(by \cref{1.2.1})}    \\
                 & = \sum_{k = 1}^m \sum_{i = 1}^p \pa{([\U]_{\beta}^{\gamma})_{i k} \cdot ([\T]_{\alpha}^{\beta})_{k j} \cdot z_i}  &  & \text{(by \cref{1.2.1})}    \\
                 & = \sum_{i = 1}^p \sum_{k = 1}^m \pa{([\U]_{\beta}^{\gamma})_{i k} \cdot ([\T]_{\alpha}^{\beta})_{k j} \cdot z_i}  &  & \text{(by \cref{1.2.1})}    \\
                 & = \sum_{i = 1}^p \pa{\sum_{k = 1}^m ([\U]_{\beta}^{\gamma})_{i k} \cdot ([\T]_{\alpha}^{\beta})_{k j}} \cdot z_i. &  & \text{(by \cref{1.2.1})}
  \end{align*}
  Thus by \cref{2.3.1} we see that \([\U \T]_{\alpha}^{\gamma} = [\U]_{\beta}^{\gamma} [\T]_{\alpha}^{\beta}\).
\end{proof}

\begin{cor}\label{2.3.3}
  Let \(\V\) be a finite-dimensional vector space over \(\F\) with an ordered basis \(\beta\).
  Let \(\T, \U \in \ls(\V)\).
  Then \([\U \T]_{\beta} = [\U]_{\beta} [\T]_{\beta}\).
\end{cor}

\begin{proof}[\pf{2.3.3}]
  We have
  \begin{align*}
    [\U \T]_{\beta} & = [\U \T]_{\beta}^{\beta}                   &  & \text{(by \cref{2.2.4})} \\
                    & = [\U]_{\beta}^{\beta} [\T]_{\beta}^{\beta} &  & \text{(by \cref{2.11})}  \\
                    & = [\U]_{\beta} [\T]_{\beta}.                &  & \text{(by \cref{2.2.4})}
  \end{align*}
\end{proof}

\begin{defn}\label{2.3.4}
  We define the \textbf{Kronecker delta} \(\delta_{i j}\) by \(\delta_{i j} = 1\) if \(i = j\) and \(\delta_{i j} = 0\) if \(i \neq j\).
  The \(n \times n\) \textbf{identity matrix} \(I_n\) is defined by \((I_n)_{i j} = \delta_{i j}\).
\end{defn}

\begin{thm}\label{2.12}
  Let \(A \in \MS\), let \(B, C \in \ms{n}{p}{\F}\) and let \(D, E \in \ms{q}{m}{\F}\).
  Then
  \begin{enumerate}
    \item \(A (B + C) = AB + AC\) and \((D + E) A = DA + EA\).
    \item \(a (AB) = (aA) B = A (aB)\) for any \(a \in \F\).
    \item \(I_m A = A = A I_n\).
    \item If \(\V\) is an \(n\)-dimensional vector space over \(\F\) with an ordered basis \(\beta\), then \([\IT[\V]]_{\beta} = I_n\).
  \end{enumerate}
\end{thm}

\begin{proof}[\pf{2.12}(a)]
  We have
  \begin{align*}
    (A (B + C))_{i j} & = \sum_{k = 1}^n A_{i k} (B + C)_{k j}                            &  & \text{(by \cref{2.3.1})}           \\
                      & = \sum_{k = 1}^n A_{i k} (B_{k j} + C_{k j})                      &  & \text{(by \cref{1.2.9})}           \\
                      & = \sum_{k = 1}^n (A_{i k} B_{k j} + A_{i k} C_{k j})              &  & (A_{i k}, B_{k j}, C_{k j} \in \F) \\
                      & = \sum_{k = 1}^n A_{i k} B_{k j} + \sum_{k = 1}^n A_{i k} C_{k j} &  & (A_{i k}, B_{k j}, C_{k j} \in \F) \\
                      & = (AB)_{i j} + (AC)_{i j}                                         &  & \text{(by \cref{2.3.1})}           \\
                      & = (AB + AC)_{i j}                                                 &  & \text{(by \cref{1.2.9})}
  \end{align*}
  and
  \begin{align*}
    ((D + E) A)_{i j} & = \sum_{k = 1}^m (D + E)_{i k} A_{k j}                            &  & \text{(by \cref{2.3.1})}           \\
                      & = \sum_{k = 1}^m (D_{i k} + E_{i k}) A_{k j}                      &  & \text{(by \cref{1.2.9})}           \\
                      & = \sum_{k = 1}^m (D_{i k} A_{k j} + E_{i k} A_{k j})              &  & (A_{k j}, D_{i k}, E_{i k} \in \F) \\
                      & = \sum_{k = 1}^m D_{i k} A_{k j} + \sum_{k = 1}^m E_{i k} A_{k j} &  & (A_{k j}, D_{i k}, E_{i k} \in \F) \\
                      & = (DA)_{i j} + (EA)_{i j}                                         &  & \text{(by \cref{2.3.1})}           \\
                      & = (DA + EA)_{i j}.                                                &  & \text{(by \cref{1.2.9})}
  \end{align*}
  Thus by \cref{1.2.8} \(A (B + C) = AB + AC\) and \((D + E) A = DA + EA\).
\end{proof}

\begin{proof}[\pf{2.12}(b)]
  We have
  \begin{align*}
    (a (AB))_{i j} & = a (AB)_{i j}                          &  & \text{(by \cref{1.2.9})} \\
                   & = a \pa{\sum_{k = 1}^n A_{i k} B_{k j}} &  & \text{(by \cref{2.3.1})} \\
                   & = \sum_{k = 1}^n (a A_{i k}) B_{k j}    &  & \text{(by \cref{1.2.9})} \\
                   & = \sum_{k = 1}^n (a A)_{i k} B_{k j}    &  & \text{(by \cref{1.2.9})} \\
                   & = ((aA) B)_{i j}                        &  & \text{(by \cref{2.3.1})} \\
                   & = \sum_{k = 1}^n A_{i k} (a B_{k j})    &  & \text{(by \cref{1.2.9})} \\
                   & = \sum_{k = 1}^n A_{i k} (a B)_{k j}    &  & \text{(by \cref{1.2.9})} \\
                   & = (A (aB))_{i j}                        &  & \text{(by \cref{2.3.1})}
  \end{align*}
  thus by \cref{1.2.8} \(a (AB) = (aA) B = A (aB)\).
\end{proof}

\begin{proof}[\pf{2.12}(c)]
  We have
  \begin{align*}
    (I_m A)_{i j} & = \sum_{k = 1}^m (I_m)_{i k} A_{k j}  &  & \text{(by \cref{2.3.1})} \\
                  & = \sum_{k = 1}^m \delta_{i k} A_{k j} &  & \text{(by \cref{2.3.4})} \\
                  & = A_{i j}                             &  & \text{(by \cref{2.3.4})} \\
                  & = \sum_{k = 1}^n A_{i k} \delta_{k j} &  & \text{(by \cref{2.3.4})} \\
                  & = \sum_{k = 1}^n A_{i k} (I_n)_{k j}  &  & \text{(by \cref{2.3.4})} \\
                  & = (A I_n)_{i j}                       &  & \text{(by \cref{2.3.1})}
  \end{align*}
  thus by \cref{1.2.8} \(I_m A = A = A I_n\).
\end{proof}

\begin{proof}[\pf{2.12}(d)]
  Let \(\beta = \set{\seq{v}{1,2,,n}}\).
  For all \(j \in \set{1, 2, \dots, n}\), we have
  \begin{align*}
             & \IT[\V](v_j) = v_j                                                                          &  & \text{(by \cref{2.1.9})} \\
             & = \sum_{i = 1}^n ([\IT[\V]]_{\beta})_{i j} v_i                                              &  & \text{(by \cref{2.2.4})} \\
    \implies & \forall i \in \set{1, 2, \dots, n}, ([\IT[\V]]_{\beta})_{i j} = \delta_{i j} = (I_n)_{i j}. &  & \text{(by \cref{2.3.4})}
  \end{align*}
  Thus by \cref{1.2.8} \([\IT[\V]]_{\beta} = I_n\).
\end{proof}

\begin{note}
  \cref{2.12} provides analogs of (a), (c), and (d) of \cref{2.10}.
  \cref{2.10}(b) has its analog in \cref{2.16}.
  Observe also that part (c) of the \cref{2.12} illustrates that the identity matrix acts as a multiplicative identity in \(\ms{n}{n}{\F}\).
  When the context is clear, we sometimes omit the subscript \(n\) from \(I_n\).
\end{note}

\begin{cor}\label{2.3.5}
  Let
  \begin{align*}
    A               & \in \MS;           \\
    \seq{B}{1,2,,k} & \in \ms{n}{p}{\F}; \\
    \seq{C}{1,2,,k} & \in \ms{q}{m}{\F}; \\
    \seq{a}{1,2,,k} & \in \F.
  \end{align*}
  Then
  \begin{align*}
    A \pa{\sum_{i = 1}^k a_i B_i} & = \sum_{i = 1}^k a_i A B_i, \\
    \pa{\sum_{i = 1}^k a_i C_i} A & = \sum_{i = 1}^k a_i C_i A.
  \end{align*}
\end{cor}

\begin{proof}[\pf{2.3.5}]
  We have
  \begin{align*}
    A \pa{\sum_{i = 1}^k a_i B_i} & = \sum_{i = 1}^k (A a_i B_i)  &  & \text{(by \cref{2.12}(a))} \\
                                  & = \sum_{i = 1}^k (a_i A B_i), &  & \text{(by \cref{2.12}(b))} \\
    \pa{\sum_{i = 1}^k a_i C_i} A & = \sum_{i = 1}^k (a_i C_i A). &  & \text{(by \cref{2.12}(a))}
  \end{align*}
\end{proof}

\begin{defn}\label{2.3.6}
  For an \(A \in \ms{n}{n}{\F}\), we define \(A^1 = A\), \(A^2 = AA\), \(A^3 = A^2 A\), and, in general, \(A^k = A^{k - 1} A\) for \(k = 2, 3, \dots\).
  We define \(A^0 = I_n\).
\end{defn}

\begin{eg}\label{2.3.7}
  If
  \[
    A = \begin{pmatrix}
      0 & 0 \\
      1 & 0
    \end{pmatrix},
  \]
  then \(A^2 = \zm\) (the zero matrix) even though \(A \neq \zm\).
  Thus the cancellation property for multiplication in fields is not valid for matrices.
  To see why, assume that the cancellation law is valid.
  Then, from \(A \cdot A = A^2 = \zm = A \cdot \zm\), we would conclude that \(A = \zm\), which is false.
\end{eg}

\begin{thm}\label{2.13}
  Let \(A \in \MS\) matrix and \(B \in \ms{n}{p}{\F}\).
  For each \(j\) (\(1 \leq j \leq p\)) let \(u_j\) and \(v_j\) denote the \(j\)th columns of \(AB\) and \(B\), respectively.
  Then
  \begin{enumerate}
    \item \(u_j = A v_j\).
    \item \(v_j = B e_j\), where \(e_j\) is the \(j\)th standard vector of \(\vs{F}^p\).
  \end{enumerate}
\end{thm}

\begin{proof}[\pf{2.13}(a)]
  We have
  \begin{align*}
    u_j & = \begin{pmatrix}
              (AB)_{1 j} \\
              (AB)_{2 j} \\
              \vdots     \\
              (AB)_{m j}
            \end{pmatrix}                 &  & \text{(by \cref{2.2.4})}   \\
        & = \begin{pmatrix}
              \sum_{k = 1}^n A_{1 k} B_{k j} \\
              \sum_{k = 1}^n A_{2 k} B_{k j} \\
              \vdots                         \\
              \sum_{k = 1}^n A_{m k} B_{k j}
            \end{pmatrix} &  & \text{(by \cref{2.3.1})}                   \\
        & = A \begin{pmatrix}
                B_{1 j} \\
                B_{2 j} \\
                \vdots  \\
                B_{n j}
              \end{pmatrix}               &  & \text{(by \cref{2.3.1})}   \\
        & = A v_j.                          &  & \text{(by \cref{2.2.4})}
  \end{align*}
\end{proof}

\begin{proof}[\pf{2.13}(b)]
  We have
  \begin{align*}
    v_j & = \begin{pmatrix}
              B_{1 j} \\
              B_{2 j} \\
              \vdots  \\
              B_{m j}
            \end{pmatrix}                     &  & \text{(by \cref{2.2.4})}   \\
        & = \begin{pmatrix}
              (B I_p)_{1 j} \\
              (B I_p)_{2 j} \\
              \vdots        \\
              (B I_p)_{m j}
            \end{pmatrix}                     &  & \text{(by \cref{2.12}(c))} \\
        & = \begin{pmatrix}
              \sum_{k = 1}^p B_{1 k} (I_p)_{k j} \\
              \sum_{k = 1}^p B_{2 k} (I_p)_{k j} \\
              \vdots                             \\
              \sum_{k = 1}^p B_{m k} (I_p)_{k j}
            \end{pmatrix} &  & \text{(by \cref{2.3.1})}                       \\
        & = B \begin{pmatrix}
                (I_p)_{1 j} \\
                (I_p)_{2 j} \\
                \vdots      \\
                (I_p)_{n j}
              \end{pmatrix}                   &  & \text{(by \cref{2.3.1})}   \\
        & = B e_j.                              &  & \text{(by \cref{2.3.4})}
  \end{align*}
\end{proof}

\begin{note}
  It follows (see \cref{ex:2.3.14}) from \cref{2.13} that column \(j\) of \(AB\) is a linear combination of the columns of \(A\) with the coefficients in the linear combination being the entries of column \(j\) of \(B\).
  An analogous result holds for rows;
  that is, row \(i\) of \(AB\) is a linear combination of the rows of \(B\) with the coefficients in the linear combination being the entries of row \(i\) of \(A\).
\end{note}

\begin{thm}\label{2.14}
  Let \(\V\) and \(\W\) be finite-dimensional vector spaces over \(\F\) having ordered bases \(\beta\) and \(\gamma\) over \(\F\), respectively, and let \(\T : \V \to \W\) be linear.
  Then, for each \(u \in \V\), we have
  \[
    [\T(u)]_{\gamma} = [\T]_{\beta}^{\gamma} [u]_{\beta}.
  \]
\end{thm}

\begin{proof}[\pf{2.14}]
  Fix \(u \in \V\), and define the linear transformations \(f : \F \to \V\) by \(f(a) = au\) and \(g : \F \to \W\) by \(g(a) = a \T(u)\) for all \(a \in \F\).
  Let \(\alpha = \set{1}\) be the standard ordered basis for \(\F\).
  Notice that \(g = \T f\).
  Identifying column vectors as matrices and using \cref{2.11}, we obtain
  \[
    [\T(u)]_{\gamma} = [g(1)]_{\gamma} = [g]_{\alpha}^{\gamma} = [\T f]_{\alpha}^{\gamma} = [\T]_{\beta}^{\gamma} [f]_{\alpha}^{\beta} = [\T]_{\beta}^{\gamma} [f(1)]_{\beta} = [\T]_{\beta}^{\gamma} [u]_{\beta}.
  \]
\end{proof}

\begin{defn}\label{2.3.8}
  Let \(A \in \ms{m}{n}{\F}\).
  We denote by \(\L_A\) the mapping \(\L_A : \vs{F}^n \to \vs{F}^m\) defined by \(\L_A(x) = Ax\) (the matrix product of \(A\) and \(x\)) for each column vector \(x \in \vs{F}^n\).
  We call \(\L_A\) a \textbf{left-multiplication transformation}.
\end{defn}

\begin{thm}\label{2.15}
  Let \(A \in \MS\).
  Then the left-multiplication transformation \(\L_A : \vs{F}^n \to \vs{F}^m\) is linear.
  Furthermore, if \(B \in \MS\) and \(\beta\) and \(\gamma\) are the standard ordered bases for \(\vs{F}^n\) and \(\vs{F}^m\) over \(\F\), respectively, then we have the following properties.
  \begin{enumerate}
    \item \([\L_A]_{\beta}^{\gamma} = A\).
    \item \(\L_A = \L_B\) iff \(A = B\).
    \item \(\L_{A + B} = \L_A + \L_B\) and \(\L_{aA} = a \L_A\) for all \(a \in \F\).
    \item If \(\T : \vs{F}^n \to \vs{F}^m\) is linear, then there exists a unique \(C \in \MS\) such that \(\T = \L_C\).
          In fact, \(C = [\T]_{\beta}^{\gamma}\).
    \item If \(E \in \ms{n}{p}{\F}\), then \(\L_{AE} = \L_A \L_E\).
    \item If \(m = n\), then \(\L_{I_n} = \IT[\vs{F}^n]\).
  \end{enumerate}
\end{thm}

\begin{proof}[\pf{2.15}(a)]
  The fact that \(\L_A\) is linear follows immediately from \cref{2.12}(a)(b).
  By \cref{2.2.4} the \(j\)th column of \([\L_A]_{\beta}^{\gamma}\) is equal to \(\L_A(e_j)\).
  However by \cref{2.3.8} we have \(\L_A(e_j) = A e_j\), which is also the \(j\)th column of \(A\) by \cref{2.13}(b).
  So \([\L_A]_{\beta}^{\gamma} = A\).
\end{proof}

\begin{proof}[\pf{2.15}(b)]
  If \(\L_A = \L_B\), then we may use \cref{2.15}(a) to write \(A = [\L_A]_{\beta}^{\gamma} = [\L_B]_{\beta}^{\gamma} = B\).
  Hence \(A = B\).
  The proof of the converse is trivial.
\end{proof}

\begin{proof}[\pf{2.15}(c)]
  For all \(x \in \vs{F}^n\), we have
  \begin{align*}
    \L_{A + B}(x) & = (A + B) x         &  & \text{(by \cref{2.3.8})}   \\
                  & = Ax + Bx           &  & \text{(by \cref{2.12}(a))} \\
                  & = \L_A(x) + \L_B(x) &  & \text{(by \cref{2.3.8})}   \\
                  & = (\L_A + \L_B)(x)  &  & \text{(by \cref{2.2.5})}
  \end{align*}
  and
  \begin{align*}
    \L_{aA}(x) & = (aA) x       &  & \text{(by \cref{2.3.8})}   \\
               & = a (Ax)       &  & \text{(by \cref{2.12}(b))} \\
               & = a \L_A(x)    &  & \text{(by \cref{2.3.8})}   \\
               & = (a \L_A)(x). &  & \text{(by \cref{2.2.5})}
  \end{align*}
  Thus \(\L_{A + B} = \L_A + \L_B\) and \(\L_{aA} = a \L_A\).
\end{proof}

\begin{proof}[\pf{2.15}(d)]
  Let \(C = [\T]_{\beta}^{\gamma}\).
  By \cref{2.14}, we have \([\T(x)]_{\gamma} = [\T]_{\beta}^{\gamma} [x]_{\beta}\), or \(\T(x) = Cx = \L_C(x)\) for all \(x \in \vs{F}^n\).
  So \(\T = \L_C\).
  The uniqueness of \(C\) follows from \cref{2.15}(b).
\end{proof}

\begin{proof}[\pf{2.15}(e)]
  For any \(j\) (\(1 \leq j \leq p\)), we may apply \cref{2.13} several times to note that \((AE) e_j\) is the \(j\)th column of \(AE\) and that the \(j\)th column of \(AE\) is also equal to \(A (E e_j)\).
  So \((AE) e_j = A (Ee_j)\).
  Thus
  \[
    \L_{AE}(e_j) = (AE) e_j = A (E e_j) = \\L_A(E e_j) = \L_A(L_E(e_j)).
  \]
  Hence \(\L_{AE} = \L_A \L_E\) by \cref{2.1.13}.
\end{proof}

\begin{proof}[\pf{2.15}(f)]
  For all \(x \in \vs{F}^n\), we have
  \begin{align*}
    \L_{I_n}(x) & = I_n x              &  & \text{(by \cref{2.3.8})}   \\
                & = x                  &  & \text{(by \cref{2.12}(c))} \\
                & = \IT_{\vs{F}^n}(x). &  & \text{(by \cref{2.1.9})}
  \end{align*}
  Thus \(\L_{I_n} = \IT_{\vs{F}^n}\).
\end{proof}

\begin{thm}\label{2.16}
  Let \(A\), \(B\), and \(C\) be matrices such that \(A (BC)\) is defined.
  Then \((AB) C\) is also defined and \(A (BC) = (AB) C\);
  that is, matrix multiplication is associative.
\end{thm}

\begin{proof}[\pf{2.16}]
  Since \(A (BC)\) is defined, by \cref{2.3.1} we can let \(A \in \MS\) and \(BC \in \ms{n}{p}{\F}\) such that \(A (BC) \in \ms{m}{p}{\F}\).
  Since \(BC\) is also defined, by \cref{2.3.1} again we can let \(B \in \ms{n}{k}{\F}\) and \(C \in \ms{k}{p}{\F}\).
  Then we have
  \begin{align*}
             & \begin{dcases}
                 A \in \MS           \\
                 B \in \ms{n}{k}{\F} \\
                 C \in \ms{k}{p}{\F}
               \end{dcases}                                    \\
    \implies & \begin{dcases}
                 AB \in \ms{m}{k}{\F} \\
                 C \in \ms{k}{p}{\F}
               \end{dcases}  &  & \text{(by \cref{2.3.1})}            \\
    \implies & (AB) C \in \ms{m}{p}{\F} &  & \text{(by \cref{2.3.1})}
  \end{align*}
  and thus \((AB) C\) is defined.

  Using \cref{2.15}(e) and the associativity of functional composition, we have
  \[
    \L_{A (BC)} = \L_A \L_{BC} = \L_A (\L_B \L_C) = (\L_A \L_B) \L_C = \L_{AB} \L_C = \L_{(AB) C}.
  \]
  So by \cref{2.15}(b), it follows that \(A (BC) = (AB) C\).
\end{proof}

\begin{defn}\label{2.3.9}
  An \textbf{incidence matrix} is a square matrix in which all the entries are either zero or one and, for convenience, all the diagonal entries are zero.
  If we have a relationship on a set of \(n\) objects that we denote by \(1, 2, \dots, n\), then we define the associated incidence matrix \(A\) by \(A_{i j} = 1\) if \(i\) is related to \(j\), and \(A_{i j} = 0\) otherwise.
\end{defn}

\begin{eg}\label{2.3.10}
  A maximal collection of three or more people with the property that any two can send to each other is called a \textbf{clique}.
  The problem of determining cliques is difficult, but there is a simple method for determining if someone belongs to a clique.
  If we define a new matrix \(B\) by \(B_{i j} = 1\) if \(i\) and \(j\) can send to each other, and \(B_{i j} = 0\) otherwise, then person \(i\) belongs to a clique if and only if \((B^3)_{i i} > 0\).
\end{eg}

\begin{eg}\label{2.3.11}
  A relation among a group of people is called a \textbf{dominance relation} if the associated incidence matrix \(A\) has the property that for all distinct pairs \(i\) and \(j\), \(A_{i j} = 1\) if and only if \(A_{j i} = 0\), that is, given any two people, exactly one of them dominates the other.
  Since \(A\) is an incidence matrix, \(A_{i i} = 0\) for all \(i\).
  For such a relation, it can be shown that the matrix \(A + A^2\) has a row [column] in which each entry is positive except for the diagonal entry.
  In other words, there is at least one person who dominates [is dominated by] all others in one or two stages.
  In fact, it can be shown that any person who dominates [is dominated by] the greatest number of people in the first stage has this property.
\end{eg}

\exercisesection

\setcounter{ex}{9}
\begin{ex}\label{ex:2.3.10}
  Let \(A \in \ms{n}{n}{\F}\).
  Prove that \(A\) is a diagonal matrix iff \(A_{i j} = \delta_{i j} A_{i j}\) for all \(i\) and \(j\).
\end{ex}

\begin{proof}[\pf{ex:2.3.10}]
  We have
  \begin{align*}
         & A \text{ is diagonal matrix}                                                                                              \\
    \iff & A_{i j} = 0 \text{ for all } i, j \in \set{1, 2, \dots, n} \text{ and } i \neq j            &  & \text{(by \cref{1.3.8})} \\
    \iff & A_{i j} = \delta_{i j} \text{ for all } i, j \in \set{1, 2, \dots, n} \text{ and } i \neq j &  & \text{(by \cref{2.3.4})} \\
    \iff & A_{i j} = \delta_{i j} A_{i j} \text{ for all } i, j \in \set{1, 2, \dots, n}.              &  & \text{(by \cref{2.3.4})}
  \end{align*}
\end{proof}

\begin{ex}\label{ex:2.3.11}
  Let \(\V\) be a vector space over \(\F\), and let \(\T : \V \to \V\) be linear.
  Prove that \(\T^2 = \zT\) iff \(\rg{\T} \subseteq \ns{\T}\).
\end{ex}

\begin{proof}[\pf{ex:2.3.11}]
  We have
  \begin{align*}
         & \T^2 = \zT                                                                \\
    \iff & \forall x \in \V, \T(\T(x)) = \zT(x) = \zv &  & \text{(by \cref{2.1.9})}  \\
    \iff & \forall x \in \V, \T(x) \in \ns{\T}        &  & \text{(by \cref{2.1.10})} \\
    \iff & \rg{\T} \subseteq \ns{\T}.                 &  & \text{(by \cref{2.1.10})}
  \end{align*}
\end{proof}

\begin{ex}\label{ex:2.3.12}
  Let \(\V\), \(\W\), and \(\vs{Z}\) be vector spaces over \(\F\), and let \(\T \in \ls(\V, \W)\) and \(\U \in \ls(\W, \vs{Z})\).
  \begin{enumerate}
    \item Prove that if \(\U \T\) is one-to-one, then \(\T\) is one-to-one.
          Must \(\U\) also be one-to-one?
    \item Prove that if \(\U \T\) is onto, then \(\U\) is onto.
          Must \(\T\) also be onto?
    \item Prove that if \(\U\) and \(\T\) are one-to-one and onto, then \(\U \T\) is also.
  \end{enumerate}
\end{ex}

\begin{proof}[\pf{ex:2.3.12}(a)]
  Let \(x, y \in \V\) such that \(x \neq y\).
  Then we have
  \begin{align*}
             & (\U \T)(x) \neq (\U \T)(y) &  & \text{(\(\U \T\) is one-to-one)}            \\
    \implies & \U(\T(x)) \neq \U(\T(y))                                                    \\
    \implies & \T(x) \neq \T(y)           &  & \text{(this is the definition of function)} \\
    \implies & \T \text{ is one-to-one}.
  \end{align*}
  From the proof above we see that is doesn't matter whether \(\U\) is one-to-one or not.
\end{proof}

\begin{proof}[\pf{ex:2.3.12}(b)]
  Let \(z \in \vs{Z}\).
  Then we have
  \begin{align*}
             & \exists x \in \V : (\U \T)(x) = z                &  & \text{(\(\U \T\) is onto)} \\
    \implies & \exists (x, y) \in \V \times \W : \begin{dcases}
                                                   \T(x) = y \\
                                                   \U(\T(x)) = \U(y) = z
                                                 \end{dcases} &  & \text{(\(\U \T\) is onto)}   \\
    \implies & \U \text{ is onto}.
  \end{align*}
  From the proof above we see that is doesn't matter whether \(\T\) is onto or not.
\end{proof}

\begin{proof}[\pf{ex:2.3.12}(c)]
  First we show that \(\U \T\) is one-to-one.
  Let \(x, y \in \V\) such that \(x \neq y\).
  Then we have
  \begin{align*}
             & \T(x) \neq \T(y)             &  & \text{(\(\T\) is one-to-one)} \\
    \implies & \U(\T(x)) \neq \U(\T(y))     &  & \text{(\(\U\) is one-to-one)} \\
    \implies & \U \T \text{ is one-to-one}.
  \end{align*}

  Now we show that \(\U \T\) is onto.
  This is true since
  \begin{align*}
             & \begin{dcases}
                 \forall y \in \W, \exists x \in \V : \T(x) = y \\
                 \forall z \in \vs{Z}, \exists y \in \W : \U(y) = z
               \end{dcases}     &  & \text{(\(\U, \T\) are onto)}     \\
    \implies & \forall z \in \vs{Z}, \exists x \in \V : \U(\T(x)) = z \\
    \implies & \U \T \text{ is onto}.
  \end{align*}
\end{proof}

\begin{ex}\label{ex:2.3.13}
  Let \(A, B \in \ms{n}{n}{\F}\).
  Prove that \(\tr[AB] = \tr[BA]\) and \(\tr[A] = \tr[\tp{A}]\).
\end{ex}

\begin{proof}[\pf{ex:2.3.13}]
  We have
  \begin{align*}
    \tr[AB] & = \sum_{i = 1}^n (AB)_{i i}                          &  & \text{(by \cref{1.3.9})} \\
            & = \sum_{i = 1}^n \pa{\sum_{k = 1}^n A_{i k} B_{k i}} &  & \text{(by \cref{2.3.1})} \\
            & = \sum_{i = 1}^n \pa{\sum_{k = 1}^n B_{k i} A_{i k}} &  & \text{(by \cref{1.2.1})} \\
            & = \sum_{k = 1}^n \pa{\sum_{i = 1}^n B_{k i} A_{i k}} &  & \text{(by \cref{1.2.1})} \\
            & = \sum_{k = 1}^n (BA)_{k k}                          &  & \text{(by \cref{2.3.1})} \\
            & = \tr[BA]                                            &  & \text{(by \cref{1.3.9})}
  \end{align*}
  and
  \begin{align*}
    \tr[A] & = \sum_{i = 1}^n A_{i i}        &  & \text{(by \cref{1.3.9})} \\
           & = \sum_{i = 1}^n (\tp{A})_{i i} &  & \text{(by \cref{1.3.3})} \\
           & = \tr[\tp{A}].                  &  & \text{(by \cref{1.3.9})}
  \end{align*}
\end{proof}

\begin{ex}\label{ex:2.3.14}
  Assume the notation in \cref{2.13}.
  \begin{enumerate}
    \item Suppose that \(z\) is a (column) vector in \(\vs{F}^p\).
          Use \cref{2.13}(b) to prove that \(Bz\) is a linear combination of the columns of \(B\).
          In particular, if \(z = \tp{\tuple{a}{1,2,,p}}\), then show that
          \[
            Bz = \sum_{j = 1}^p a_j v_j.
          \]
    \item Extend (a) to prove that column \(j\) of \(AB\) is a linear combination of the columns of \(A\) with the coefficients in the linear combination being the entries of column \(j\) of \(B\).
    \item For any row vector \(w \in \vs{F}^m\), prove that \(wA\) is a linear combination of the rows of \(A\) with the coefficients in the linear combination being the coordinates of \(w\).
    \item Prove the analogous result to (b) about rows:
          Row \(i\) of \(AB\) is a linear combination of the rows of \(B\) with the coefficients in the linear combination being the entries of row \(i\) of \(A\).
  \end{enumerate}
\end{ex}

\begin{proof}[\pf{ex:2.3.14}(a)]
  We have
  \begin{align*}
    Bz & = \begin{pmatrix}
             (Bz)_{1 1} \\
             (Bz)_{2 1} \\
             \vdots     \\
             (Bz)_{n 1}
           \end{pmatrix} = \begin{pmatrix}
                             \sum_{k = 1}^p B_{1 k} z_{k 1} \\
                             \sum_{k = 1}^p B_{2 k} z_{k 1} \\
                             \vdots                         \\
                             \sum_{k = 1}^p B_{n k} z_{k 1}
                           \end{pmatrix}                      &  & \text{(by \cref{2.3.1})}                   \\
       & = \begin{pmatrix}
             \sum_{k = 1}^p B_{1 k} z_k \\
             \sum_{k = 1}^p B_{2 k} z_k \\
             \vdots                     \\
             \sum_{k = 1}^p B_{n k} z_k
           \end{pmatrix} = \begin{pmatrix}
                             \sum_{k = 1}^p B_{1 k} a_k \\
                             \sum_{k = 1}^p B_{2 k} a_k \\
                             \vdots                     \\
                             \sum_{k = 1}^p B_{n k} a_k
                           \end{pmatrix}                                                         \\
       & = \sum_{k = 1}^p \begin{pmatrix}
                            B_{1 k} a_k \\
                            B_{2 k} a_k \\
                            \vdots      \\
                            B_{n k} a_k
                          \end{pmatrix} = \sum_{k = 1}^p \pa{a_k \begin{pmatrix}
                                                                     B_{1 k} \\
                                                                     B_{2 k} \\
                                                                     \vdots  \\
                                                                     B_{n k}
                                                                   \end{pmatrix}} &  & \text{(by \cref{1.2.9})} \\
       & = \sum_{k = 1}^p a_k v_k.                              &  & \text{(by \cref{2.13})}
  \end{align*}
\end{proof}

\begin{proof}[\pf{ex:2.3.14}(b)]
  We have
  \begin{align*}
    u_j & = A v_j = A \begin{pmatrix}
                        B_{1 j} \\
                        B_{2 j} \\
                        \vdots  \\
                        B_{n j}
                      \end{pmatrix}              &  & \text{(by \cref{2.13}(a))}       \\
        & = \sum_{k = 1}^n B_{k j} \begin{pmatrix}
                                     A_{1 k} \\
                                     A_{2 k} \\
                                     \vdots  \\
                                     A_{n k}
                                   \end{pmatrix}. &  & \text{(by \cref{ex:2.3.14}(a))}
  \end{align*}
\end{proof}

\begin{proof}[\pf{ex:2.3.14}(c)]
  We have
  \begin{align*}
    wA & = \begin{pmatrix}
             (wA)_1 & (wA)_2 & \cdots & (wA)_n
           \end{pmatrix}                                                                         &  & \text{(by \cref{2.3.1})}                    \\
       & = \begin{pmatrix}
             (wA)_{1 1} & (wA)_{1 2} & \cdots & (wA)_{1 n}
           \end{pmatrix}                                                             &  & \text{(by \cref{2.3.1})}                                \\
       & = \begin{pmatrix}
             \sum_{k = 1}^m w_{1 k} A_{k 1} & \sum_{k = 1}^m w_{1 k} A_{k 2} & \cdots & \sum_{k = 1}^m w_{1 k} A_{k n}
           \end{pmatrix} &  & \text{(by \cref{2.3.1})}                              \\
       & = \sum_{k = 1}^m \begin{pmatrix}
                            w_{1 k} A_{k 1} & w_{1 k} A_{k 2} & \cdots & w_{1 k} A_{k n}
                          \end{pmatrix}                                              &  & \text{(by \cref{1.2.9})}                                \\
       & = \sum_{k = 1}^m \begin{pmatrix}
                            w_k A_{k 1} & w_k A_{k 2} & \cdots & w_k A_{k n}
                          \end{pmatrix}                                                          &  & \text{(by \cref{2.3.1})}                    \\
       & = \sum_{k = 1}^m \pa{w_k \begin{pmatrix}
                                      A_{k 1} & A_{k 2} & \cdots & A_{k n}
                                    \end{pmatrix}}.                                                                   &  & \text{(by \cref{1.2.9})} \\
  \end{align*}
\end{proof}

\begin{proof}[\pf{ex:2.3.14}(d)]
  We have
  \begin{align*}
     & \begin{pmatrix}
         (AB)_{i 1} & (AB)_{i 2} & \cdots & (AB)_{i p}
       \end{pmatrix}                                                                                            \\
     & = \begin{pmatrix}
           \sum_{k = 1}^n A_{i k} B_{k 1} & \sum_{k = 1}^n A_{i k} B_{k 2} & \cdots & \sum_{k = 1}^n A_{i k} B_{k p}
         \end{pmatrix} &  & \text{(by \cref{2.3.1})}                              \\
     & = \sum_{k = 1}^n \begin{pmatrix}
                          A_{i k} B_{k 1} & A_{i k} B_{k 2} & \cdots & A_{i k} B_{k p}
                        \end{pmatrix}                                              &  & \text{(by \cref{1.2.9})}                                \\
     & = \sum_{k = 1}^n \pa{A_{i k} \begin{pmatrix}
                                        B_{k 1} & B_{k 2} & \cdots & B_{k p}
                                      \end{pmatrix}}.                                                               &  & \text{(by \cref{1.2.9})}
  \end{align*}
\end{proof}

\begin{ex}\label{ex:2.3.15}
  Let \(M\) and \(A\) be matrices for which the product matrix \(MA\) is defined.
  If the \(j\)th column of \(A\) is a linear combination of a set of columns of \(A\), prove that the \(j\)th column of \(MA\) is a linear combination of the corresponding columns of \(MA\) with the same corresponding coefficients.
\end{ex}

\begin{proof}[\pf{ex:2.3.15}]
  Let \(M \in \MS\) and let \(A \in \ms{n}{p}{\F}\).
  For all \(i \in \set{1, 2, \dots, p}\) we define \(v_i\) to be the \(i\)th column of \(A\).
  By hypothesis we know that
  \[
    \exists \seq{c}{1,2,,p} \in \F : v_j = \sum_{i = 1}^p c_i v_i.
  \]
  Then we have
  \begin{align*}
    \begin{pmatrix}
      (MA)_{1 j} \\
      (MA)_{2 j} \\
      \vdots     \\
      (MA)_{p j}
    \end{pmatrix} & = M v_j                              &  & \text{(by \cref{2.13}(a))}   \\
                    & = M \pa{\sum_{i = 1}^p c_i v_i}                                      \\
                    & = \sum_{i = 1}^p M (c_i v_i)         &  & \text{(by \cref{2.12}(a))} \\
                    & = \sum_{i = 1}^p c_i (M v_i)         &  & \text{(by \cref{2.12}(b))} \\
                    & = \sum_{i = 1}^p c_i \begin{pmatrix}
                                             (MA)_{1 i} \\
                                             (MA)_{2 i} \\
                                             \vdots     \\
                                             (MA)_{p i}
                                           \end{pmatrix}. &  & \text{(by \cref{2.13}(b))}
  \end{align*}
\end{proof}

\begin{ex}\label{ex:2.3.16}
  Let \(V\) be a finite-dimensional vector space over \(\F\), and let \(\T : \V \to \V\) be linear.
  \begin{enumerate}
    \item If \(\rk{\T} = \rk{\T^2}\), prove that \(\rg{\T} \cap \ns{\T} = \set{\zv}\).
          Deduce that \(\V = \rg{\T} \oplus \ns{\T}\).
    \item Prove that \(\V = \rg{\T^k} \oplus \ns{\T^k}\) for some positive integer \(k\).
  \end{enumerate}
\end{ex}

\begin{proof}[\pf{ex:2.3.16}(a)]
  First observe that
  \begin{align*}
             & \begin{dcases}
                 \T(\V) \subseteq \V \\
                 \rk{\T^2} = \rk{\T}
               \end{dcases}                                                                                             \\
    \implies & \begin{dcases}
                 \rg{\T^2} = \T^{2}(\V) = \T(\T(\V)) \subseteq \T(\V) = \rg{\T} \\
                 \rk{\T^2} = \rk{\T}
               \end{dcases} &  & \text{(by \cref{2.1.10})}                                 \\
    \implies & \rg{\T^2} = \rg{\T}.                                                               &  & \text{(by \cref{1.11})}
  \end{align*}
  Let \(\beta = \set{\seq{v}{1,2,,n}}\) be a basis for \(\rg{\T}\) over \(\F\).
  Since \(\rg{\T^2} = \rg{\T}\), by \cref{2.2} we know that \(\T(\beta)\) is a basis for \(\rg{\T}\) over \(\F\).
  Let \(x \in \rg{\T} \cap \ns{\T}\).
  Since \(x \in \rg{\T}\), by \cref{1.6.1} we know that
  \[
    \exists \seq{a}{1,2,,n} \in \F : x = \sum_{i = 1}^n a_i v_i.
  \]
  Since \(x \in \ns{\T}\), we have
  \begin{align*}
             & \T(x) = \zv                                                   &  & \text{(by \cref{2.1.10})}   \\
    \implies & \T(\sum_{i = 1}^n a_i v_i) = \sum_{i = 1}^n a_i \T(v_i) = \zv &  & \text{(by \cref{2.1.2}(d))} \\
    \implies & \seq[=]{a}{1,2,,n} = 0                                        &  & \text{(by \cref{1.5.3})}    \\
    \implies & x = \zv.                                                      &  & \text{(by \cref{1.2}(a))}
  \end{align*}
  Thus \(\rg{\T} \cap \ns{\T} = \zv\).
  Since
  \begin{align*}
     & \dim(\rg{\T} + \ns{\T})                                                                        \\
     & = \dim(\rg{\T}) + \dim(\ns{\T}) - \dim(\rg{\T} \cap \ns{\T}) &  & \text{(by \cref{ex:1.6.29})} \\
     & = \dim(\rg{\T}) + \dim(\ns{\T})                              &  & \text{(by \cref{1.6.9})}     \\
     & = \rk{\T} + \nt{\T} = \dim(\V)                               &  & \text{(by \cref{2.3})}
  \end{align*}
  and by \cref{ex:1.3.23}(a) \(\rg{\T} + \ns{\T}\) is a subspace of \(\V\) over \(\F\), by \cref{1.11} we know that \(\V = \rg{\T} + \ns{\T}\).
  Thus by \cref{1.3.11} we know that \(\V = \rg{\T} \oplus \ns{\T}\).
\end{proof}

\begin{proof}[\pf{ex:2.3.16}(b)]
  First observe that
  \begin{align*}
             & \forall k \in \Z^+, \rg{\T^{k + 1}} = \T^{k + 1}(\V) \subseteq \T^k(\V) = \rg{\T^k} &  & \text{(by \cref{2.1.10})}   \\
    \implies & \forall k \in \Z^+, 0 \leq \rk{\T^{k + 1}} \leq \rk{\T^k} \leq \dim(\V).            &  & \text{(by \cref{1.11,2.3})}
  \end{align*}
  Since \(\V\) is finite-dimensional, we know that there must exists a \(k \in \Z^+\) such that \(\rk{\T^{k + 1}} = \rk{\T^k}\).
  By \cref{ex:2.3.16}(a) we see that \(\rg{\T^k} \cap \ns{\T^k} = \set{\zv}\) and \(\V = \rg{\T^k} \oplus \ns{\T^k}\).
\end{proof}

\section{Invertibility and Isomorphisms}\label{sec:2.4}

\begin{defn}\label{2.4.1}
  Let \(\V\) and \(\W\) be vector spaces over \(\F\), and let \(\T : \V \to \W\) be linear.
  A function \(\U : \W \to \V\) is said to be an \textbf{inverse} of \(\T\) if \(\T \U = \IT[\W]\) and \(\U \T = \IT[\V]\).
  If \(\T\) has an inverse, then \(\T\) is said to be \textbf{invertible}.
  If \(\T\) is invertible, then the inverse of \(\T\) is unique and is denoted by \(\T^{-1}\).
  The following facts hold for invertible functions T and U.
  \begin{itemize}
    \item \((\T \U)^{-1} = \U^{-1} \T^{-1}\).
    \item \((\T^{-1})^{-1} = \T\);
          in particular, \(\T^{-1}\) is invertible.
  \end{itemize}
  We often use the fact that a function is invertible iff it is both one-to-one and onto.
  We can therefore restate \cref{2.5} as follows.
  \begin{itemize}
    \item Let \(\T : \V \to \W\) be a linear transformation, where \(\V\) and \(\W\) are finite-dimensional spaces over \(\F\) of equal dimension.
          Then \(\T\) is invertible if and only if \(\rk{\T} = \dim(\V)\).
  \end{itemize}
\end{defn}

\begin{thm}\label{2.17}
  Let \(\V\) and \(\W\) be vector spaces over \(\F\), and let \(\T : \V \to \W\) be linear and invertible.
  Then \(\T^{-1} : \W \to \V\) is linear.
\end{thm}

\begin{proof}[\pf{2.17}]
  Let \(y_1, y_2 \in \W\) and \(c \in \F\).
  Since \(\T\) is onto and one-to-one, there exist unique vectors \(x_1\) and \(x_2\) such that \(\T(x_1) = y_1\) and \(\T(x_2) = y_2\).
  Thus \(x_1 = \T^{-1}(y_1)\) and \(x_2 = \T^{-1}(y_2)\);
  so
  \begin{align*}
    \T^{-1}(c y_1 + y_2) & = \T^{-1}(c \T(x_1) + \T(x_2))   &  & \text{(by \cref{2.4.1})}    \\
                         & = \T^{-1}(\T(c x_1 + x_2))       &  & \text{(by \cref{2.1.2}(b))} \\
                         & = c x_1 + x_2                    &  & \text{(by \cref{2.4.1})}    \\
                         & = c \T^{-1}(x_1) + \T^{-1}(x_2). &  & \text{(by \cref{2.4.1})}
  \end{align*}
\end{proof}

\begin{cor}\label{2.4.2}
  If \(\T\) is a linear transformation between vector spaces of equal (finite) dimension, then the conditions of being invertible, one-to-one, and onto are all equivalent.
\end{cor}

\begin{proof}[\pf{2.4.2}]
  By \cref{2.5} we see that this is true.
\end{proof}

\begin{defn}\label{2.4.3}
  Let \(A \in \ms{n}{n}{\F}\).
  Then \(A\) is \textbf{invertible} if there exists an \(B \in \ms{n}{n}{\F}\) such that \(AB = BA = I_n\).
\end{defn}

\begin{cor}\label{2.4.4}
  If \(A\) is invertible, then the matrix \(B\) such that \(AB = BA = I\) is unique.
  The matrix \(B\) is called the \textbf{inverse} of \(A\) and is denoted by \(A^{-1}\).
\end{cor}

\begin{proof}[\pf{2.4.4}]
  If \(C\) were another such matrix, then
  \[
    C = CI = C(AB) = (CA)B = IB = B.
  \]
  Thus \(B\) is unique.
\end{proof}

\begin{lem}\label{2.4.5}
  Let \(\T\) be an invertible linear transformation from \(\V\) to \(\W\).
  Then \(\V\) is finite-dimensional iff \(\W\) is finite-dimensional.
  In this case, \(\dim(\V) = \dim(\W)\).
\end{lem}

\begin{proof}[\pf{2.4.5}]
  Suppose that \(\V\) is finite-dimensional.
  Let \(\beta = \set{\seq{x}{1,,n}}\) be a basis for \(\V\) over \(\F\).
  By \cref{2.2} \(\T(\beta)\) spans \(\rg{\T} = \W\);
  hence \(\W\) is finite-dimensional by \cref{1.9}.
  Conversely, if \(\W\) is finite-dimensional, then so is \(\V\) by a similar argument, using \(T^{-1}\).

  Now suppose that \(\V\) and \(\W\) are finite-dimensional.
  Because \(\T\) is one-to-one and onto, we have
  \[
    \nt{\T} = 0 \quad \text{and} \quad \rk{\T} = \dim(\rg{\T}) = \dim(\W).
  \]
  So by the dimension theorem (\cref{2.3}), it follows that \(\dim(\V) = \dim(\W)\).
\end{proof}

\begin{thm}\label{2.18}
  Let \(\V\) and \(\W\) be finite-dimensional vector spaces over \(\F\) with ordered bases \(\beta\) and \(\gamma\) over \(\F\), respectively.
  Let \(\T : \V \to \W\) be linear.
  Then \(\T\) is invertible iff \([\T]_{\beta}^{\gamma}\) is invertible.
  Furthermore, \([\T^{-1}]_{\gamma}^{\beta} = ([\T]_{\beta}^{\gamma})^{-1}\).
\end{thm}

\begin{proof}[\pf{2.18}]
  Suppose that \(\T\) is invertible.
  By \cref{2.4.5}, we have \(\dim(\V) = \dim(\W)\).
  Let \(n = \dim(\V)\).
  So \([\T]_{\beta}^{\gamma} \in \ms{n}{n}{\F}\).
  Now \(\T^{-1} : \W \to \V\) satisfies \(\T \T^{-1} = \IT[\W]\) and \(\T^{-1} \T = \IT[\V]\).
  Thus
  \[
    I_n = [\IT[\V]]_{\beta} = [\T^{-1} \T]_{\beta} = [\T^{-1}]_{\gamma}^{\beta} [\T]_{\beta}^{\gamma}.
  \]
  Similarly, \([\T]_{\beta}^{\gamma} [\T^{-1}]_{\gamma}^{\beta} = I_n\).
  So \([\T]_{\beta}^{\gamma}\) is invertible and \(\pa{[\T]_{\beta}^{\gamma}}^{-1} = [\T^{-1}]_{\gamma}^{\beta}\).

  Now suppose that \(A = [\T]_{\beta}^{\gamma}\) is invertible.
  Then there exists an \(B \in \ms{n}{n}{\F}\) such that \(AB = BA = I_n\).
  By \cref{2.6} there exists \(U \in \ls(\W, \V)\) such that
  \[
    \U(w_j) = \sum_{i = 1}^n B_{i j} v_i \quad \text{for } j \in \set{1, \dots, n},
  \]
  where \(\gamma = \set{\seq{w}{1,,n}}\) and \(\beta = \set{\seq{v}{1,,n}}\).
  It follows that \([\U]_{\gamma}^{\beta} = B\).
  To show that \(\U = \T^{-1}\), observe that
  \[
    [\U \T]_{\beta} = [\U]_{\gamma}^{\beta} [\T]_{\beta}^{\gamma} = BA = I_n = [\IT[\V]]_{\beta}
  \]
  by \cref{2.11}.
  So \(\U \T = \IT[\V]\), and similarly, \(\T \U = \IT[\W]\).
\end{proof}

\section{The Change of Coordinate Matrix}\label{sec:2.5}

\section{Dual Spaces}\label{sec:2.6}

\begin{defn}\label{2.6.1}
  In this section, we are concerned exclusively with linear transformations from a vector space \(\V\) into its field of scalars \(\F\), which is itself a vector space of dimension \(1\) over \(\F\).
  Such a linear transformation is called a \textbf{linear functional} on \(\V\).
\end{defn}

\begin{eg}\label{2.6.2}
  Let \(\V\) be the vector space of continuous real-valued functions on the interval \([0, 2\pi]\).
  Fix a function \(g \in \V\).
  The function \(h : \V \to \R\) defined by
  \[
    h(x) = \frac{1}{2\pi} \int_{0}^{2\pi} x(t) g(t) \; dt
  \]
  is a linear functional on \(\V\).
  In the cases that \(g(t)\) equals \(\sin(nt)\) or \(\cos(nt)\), \(h(x)\) is often called the \textbf{\(n\)th Fourier coefficient of \(x\)}.
\end{eg}

\begin{proof}[\pf{2.6.2}]
  Let \(f_1, f_2 \in \V\) and let \(c \in \R\).
  Then we have
  \begin{align*}
    h(cf_1 + f_2) & = \frac{1}{2\pi} \int_{0}^{2\pi} (cf_1 + f_2)(t) g(t) \; dt                                           &  & \text{(by \cref{2.6.2})} \\
                  & = \frac{1}{2\pi} \int_{0}^{2\pi} cf_1(t) g(t) + f_2(t) g(t) \; dt                                                                   \\
                  & = \frac{c}{2\pi} \int_{0}^{2\pi} f_1(t) g(t) \; dt + \frac{1}{2\pi} \int_{0}^{2\pi} f_2(t) g(t) \; dt                               \\
                  & = c h(f_1) + h(f_2)                                                                                   &  & \text{(by \cref{2.6.2})}
  \end{align*}
  and thus by \cref{2.1.2}(b) \(h \in \ls(\V, \F)\).
\end{proof}

\begin{eg}\label{2.6.3}
  The trace function \(\tr : \ms{n}{n}{\F} \to \F\) is a linear functional.
\end{eg}

\begin{proof}[\pf{2.6.3}]
  By \cref{ex:1.3.6} we see that this is true.
\end{proof}

\begin{eg}\label{2.6.4}
  Let \(\V\) be a finite-dimensional vector space over \(\F\), and let \(\beta = \set{\seq{x}{1,,n}}\) be an ordered basis for \(\V\) over \(\F\).
  For each \(i \in \set{1, \dots, n}\), define \(f_i(x) = a_i\), where
  \[
    [x]_{\beta} = \begin{pmatrix}
      a_1    \\
      \vdots \\
      a_n
    \end{pmatrix}
  \]
  is the coordinate vector of \(x\) relative to \(\beta\).
  Then \(f_i\) is a linear functional on \(\V\) called the \textbf{\(i\)th coordinate function with respect to the basis \(\beta\)}.
  Note that \(f_i(x_j) = \delta_{i j}\), where \(\delta_{i j}\) is the Kronecker delta.
  These linear functionals play an important role in the theory of dual spaces (see \cref{2.24}).
\end{eg}

\begin{proof}[\pf{2.6.4}]
  Let \(a, b \in \V\) and let \(c \in \F\).
  By \cref{1.8} there exist some \(\seq{a}{1,,n}, \seq{b}{1,,n} \in \F\) such that
  \[
    a = \sum_{j = 1}^n a_j x_j \quad \text{and} \quad b = \sum_{j = 1}^n b_j x_j.
  \]
  Then we have
  \begin{align*}
    f_i(ca + b) & = f_i\pa{c \pa{\sum_{j = 1}^n a_j x_j} + \sum_{j = 1}^n b_j x_j}                               \\
                & = f_i\pa{\sum_{j = 1}^n (c a_j + b_j) x_j}                       &  & \text{(by \cref{1.2.1})} \\
                & = c a_i + b_i                                                    &  & \text{(by \cref{2.6.4})} \\
                & = c f_i(a) + f_i(b)                                              &  & \text{(by \cref{2.6.4})}
  \end{align*}
  and thus by \cref{2.1.2}(b) \(f_i \in \ls(\V, \F)\).
\end{proof}

\begin{defn}\label{2.6.5}
  For a vector space \(\V\) over \(\F\), we define the \textbf{dual space} of \(\V\) to be the vector space \(\ls(\V, \F)\), denoted by \(\V^*\).
  Thus \(\V^*\) is the vector space consisting of all linear functionals on \(\V\) with the operations of addition and scalar multiplication as defined in \cref{sec:2.2}.
  Note that if \(\V\) is finite-dimensional, then by the \cref{2.4.10}
  \[
    \dim(\V^*) = \dim(\ls(\V, \F)) = \dim(\V) \cdot \dim(\F) = \dim(\V).
  \]
  Hence by \cref{2.19} \(\V\) and \(\V^*\) are isomorphic.
  We also define the \textbf{double dual} \(\V^{**}\) of \(\V\) to be the dual of \(\V^*\).
  In \cref{2.26}, we show, in fact, that there is a natural identification of \(\V\) and \(\V^{**}\) in the case that \(\V\) is finite-dimensional.
\end{defn}

\begin{thm}\label{2.24}
  Suppose that \(\V\) is a finite-dimensional vector space over \(\F\) with the ordered basis \(\beta = \set{\seq{x}{1,,n}}\).
  Let \(f_i\) (\(1 \leq i \leq n\)) be the \(i\)th coordinate function with respect to \(\beta\) as defined in \cref{2.6.4}, and let \(\beta^* = \set{\seq{f}{1,,n}}\).
  Then \(\beta^*\) is an ordered basis for \(\V^*\), and, for any \(f \in \V^*\), we have
  \[
    f = \sum_{i = 1}^n f(x_i) f_i.
  \]
\end{thm}

\begin{proof}[\pf{2.24}]
  Let \(f \in \V^*\).
  Since \(\dim(\V^*) = n\), we need only show that
  \[
    f = \sum_{i = 1}^n f(x_i) f_i,
  \]
  from which it follows that \(\beta^*\) generates \(\V^*\), and hence is a basis by \cref{1.6.15}(a).
  Let
  \[
    g = \sum_{i = 1}^n f(x_i) f_i.
  \]
  For \(1 \leq j \leq n\), we have
  \begin{align*}
    g(x_j) & = \pa{\sum_{i = 1}^n f(x_i) f_i}(x_j)                               \\
           & = \sum_{i = 1}^n f(x_i) f_i(x_j)      &  & \text{(by \cref{2.2.5})} \\
           & = \sum_{i = 1}^n f(x_i) \delta_{i j}  &  & \text{(by \cref{2.6.4})} \\
           & = f(x_j).
  \end{align*}
  Therefore \(f = g\) by \cref{2.1.13}.
\end{proof}

\begin{defn}\label{2.6.6}
  Using the notation of \cref{2.24}, we call the ordered basis \(\beta^* = \set{\seq{f}{1,,n}}\) of \(\V^*\) over \(\F\) that satisfies \(f_i(x_j) = \delta_{i j}\) (\(1 \leq i, j \leq n\)) the \textbf{dual basis} of \(\beta\).
\end{defn}

\begin{thm}\label{2.26}

\end{thm}

\section{Homogeneous Linear Differential Equations with Constant Coefficients}\label{sec:2.7}


\chapter{Elementary Matrix Operations and Systems of Linear Equations}\label{ch:3}

\chapter{Determinants}\label{ch:4}

% All sections are in separated files.  We include them here.
\section{Determinants of Order \textrm{2}}\label{sec:4.1}

\begin{defn}\label{4.1.1}
  If
  \[
    A = \begin{pmatrix}
      a & b \\
      c & d
    \end{pmatrix}
  \]
  is a \(2 \times 2\) matrix with entries from a field \(\F\), then we define the \textbf{determinant} of \(A\), denoted \(\det(A)\) or \(\abs{A}\), to be the scalar \(ad - bc\).
\end{defn}

\begin{note}
  There exist \(A, B \in \ms{2}{2}{\F}\) such that \(\det(A + B) \neq \det(A) + \det(B)\), the function \(\det : \ms{2}{2}{\F} \to \F\) is \emph{not} a linear transformation.
\end{note}

\begin{thm}\label{4.1}
  The function \(\det : \ms{2}{2}{\F} \to \F\) is a linear function of each row of a \(2 \times 2\) matrix when the other row is held fixed.
  That is, if \(u, v, w \in \vs{F}^2\) and \(k \in \F\), then
  \[
    \det\begin{pmatrix}
      u + kv \\
      w
    \end{pmatrix} = \det\begin{pmatrix}
      u \\
      w
    \end{pmatrix} + k \det\begin{pmatrix}
      v \\
      w
    \end{pmatrix}
  \]
  and
  \[
    \det\begin{pmatrix}
      w \\
      u + kv
    \end{pmatrix} = \det\begin{pmatrix}
      w \\
      u
    \end{pmatrix} + k \det\begin{pmatrix}
      w \\
      v
    \end{pmatrix}.
  \]
\end{thm}

\begin{proof}[\pf{4.1}]
  Let \(u = \tuple{a}{1,2}, v = \tuple{b}{1,2}, w = \tuple{c}{1,2} \in \vs{F}^2\) and let \(k \in \F\).
  Then
  \begin{align*}
    \det\begin{pmatrix}
          u \\
          w
        \end{pmatrix} + k \det\begin{pmatrix}
                                v \\
                                w
                              \end{pmatrix} & = \det\begin{pmatrix}
                                                      a_1 & a_2 \\
                                                      c_1 & c_2
                                                    \end{pmatrix} + k \det\begin{pmatrix}
                                                                            b_1 & b_2 \\
                                                                            c_1 & c_2
                                                                          \end{pmatrix}  \\
                                          & = (a_1 c_2 - a_2 c_1) + k (b_1 c_2 - b_2 c_1) \\
                                          & = (a_1 + k b_1) c_2 - (a_2 + k b_2) c_1       \\
                                          & = \det\begin{pmatrix}
                                                    a_1 + k b_1 & a_2 + k b_2 \\
                                                    c_1         & c_2
                                                  \end{pmatrix}               \\
                                          & = \det\begin{pmatrix}
                                                    u + kv \\
                                                    w
                                                  \end{pmatrix}.
  \end{align*}
  A similar calculation shows that
  \[
    \det\begin{pmatrix}
      w \\
      u
    \end{pmatrix} + k \det\begin{pmatrix}
      w \\
      v
    \end{pmatrix} = \det\begin{pmatrix}
      w \\
      u + kv
    \end{pmatrix}.
  \]
\end{proof}

\begin{thm}\label{4.2}
  Let \(A \in \ms{2}{2}{\F}\).
  Then the determinant of \(A\) is nonzero iff \(A\) is invertible.
  Moreover, if \(A\) is invertible, then
  \[
    A^{-1} = \frac{1}{\det(A)} \begin{pmatrix}
      A_{2 2}  & -A_{1 2} \\
      -A_{2 1} & A_{1 1}
    \end{pmatrix}.
  \]
\end{thm}

\begin{proof}[\pf{4.2}]
  If \(\det(A) \neq 0\), then we can define a matrix
  \[
    M = \frac{1}{\det(A)} \begin{pmatrix}
      A_{2 2}  & -A_{1 2} \\
      -A_{2 1} & A_{1 1}
    \end{pmatrix}.
  \]
  A straightforward calculation shows that \(AM = MA = I\), and so \(A\) is invertible and \(M = A^{-1}\).

  Conversely, suppose that \(A\) is invertible.
  \cref{3.2.2} shows that the rank of
  \[
    A = \begin{pmatrix}
      A_{1 1} & A_{1 2} \\
      A_{2 1} & A_{2 2}
    \end{pmatrix}
  \]
  must be \(2\).
  Hence \(A_{1 1} \neq 0\) or \(A_{2 1} \neq 0\).
  If \(A_{1 1} \neq 0\), add \(-A_{2 1} / A_{1 1}\) times row \(1\) of \(A\) to row \(2\) to obtain the matrix
  \[
    \begin{pmatrix}
      A_{1 1} & A_{1 2}                                   \\
      0       & A_{2 2} - \frac{A_{1 2} A_{2 1}}{A_{1 1}}
    \end{pmatrix}.
  \]
  Because elementary row operations are rank-preserving by \cref{3.2.3}, it follows that
  \[
    A_{2 2} - \frac{A_{1 2} A_{2 1}}{A_{1 1}} \neq 0.
  \]
  Therefore \(\det(A) = A_{1 1} A_{2 2} - A_{1 2} A_{2 1} \neq 0\).
  On the other hand, if \(A_{2 1} \neq 0\), we see that \(\det(A) \neq 0\) by adding \(-A_{1 1} / A_{2 1}\) times row \(2\) of \(A\) to row \(1\) and applying a similar argument.
  Thus, in either case, \(\det(A) \neq 0\).
\end{proof}

\begin{defn}\label{4.1.2}
  By the \textbf{angle} between two vectors in \(\R^2\), we mean the angle with measure \(\theta\) (\(0 \leq \theta < \pi\)) that is formed by the vectors having the same magnitude and direction as the given vectors but emanating from the origin.

  If \(\beta = \set{u, v}\) is an ordered basis for \(\R^2\), we define the \textbf{orientation} of \(\beta\) to be the real number
  \[
    \mathbf{O}\begin{pmatrix}
      u \\
      v
    \end{pmatrix} = \frac{\det\begin{pmatrix}
        u \\
        v
      \end{pmatrix}}{\abs{\det\begin{pmatrix}
          u \\
          v
        \end{pmatrix}}}.
  \]
  (The denominator of this fraction is nonzero by \cref{4.2}.)
  Clearly
  \[
    \mathbf{O}\begin{pmatrix}
      u \\
      v
    \end{pmatrix} = \pm 1.
  \]
  Notice that
  \[
    \mathbf{O}\begin{pmatrix}
      e_1 \\
      e_2
    \end{pmatrix} = 1 \quad \text{and} \quad \mathbf{O}\begin{pmatrix}
      e_1 \\
      -e_2
    \end{pmatrix} = -1.
  \]

  Recall that a coordinate system \(\set{u, v}\) is called \textbf{right-handed} if \(u\) can be rotated in a counterclockwise direction through an angle \(\theta\) (\(0 < \theta < \pi\)) to coincide with \(v\).
  Otherwise \(\set{u, v}\) is called a \textbf{left-handed} system.
  In general (see \cref{ex:4.1.12}),
  \[
    \mathbf{O}\begin{pmatrix}
      u \\
      v
    \end{pmatrix} = 1
  \]
  iff the ordered basis \(\set{u, v}\) forms a right-handed coordinate system.
  For convenience, we also define
  \[
    \mathbf{O}\begin{pmatrix}
      u \\
      v
    \end{pmatrix} = 1
  \]
  if \(\set{u, v}\) is linearly dependent.
\end{defn}

\begin{defn}\label{4.1.3}
  Any ordered set \(\set{u, v}\) in \(\R^2\) determines a parallelogram in the following manner.
  Regarding \(u\) and \(v\) as arrows emanating from the origin of \(\R^2\), we call the parallelogram having \(u\) and \(v\) as adjacent sides the \textbf{parallelogram determined by \(u\) and \(v\)}.
  Observe that if the set \(\set{u, v}\) is linearly dependent (i.e., if \(u\) and \(v\) are parallel), then the ``parallelogram'' determined by \(u\) and \(v\) is actually a line segment, which we consider to be a degenerate parallelogram having area zero.
\end{defn}

\begin{prop}\label{4.1.4}
  Let \(\set{u, v} \subseteq \R^2\).
  If we define
  \[
    \mathbf{A}\begin{pmatrix}
      u \\
      v
    \end{pmatrix}
  \]
  as the area of the parallelogram determined by \(u\) and \(v\), then
  \[
    \mathbf{A}\begin{pmatrix}
      u \\
      v
    \end{pmatrix} = \mathbf{O}\begin{pmatrix}
      u \\
      v
    \end{pmatrix} \cdot \det\begin{pmatrix}
      u \\
      v
    \end{pmatrix} = \abs{\det\begin{pmatrix}
        u \\
        v
      \end{pmatrix}}.
  \]
\end{prop}

\begin{proof}[\pf{4.1.4}]
  First, since
  \[
    \mathbf{O}\begin{pmatrix}
      u \\
      v
    \end{pmatrix} = \pm 1,
  \]
  we may multiply both sides of the desired equation by
  \[
    \mathbf{O}\begin{pmatrix}
      u \\
      v
    \end{pmatrix}
  \]
  to obtain the equivalent form
  \[
    \mathbf{O}\begin{pmatrix}
      u \\
      v
    \end{pmatrix} \cdot \mathbf{A}\begin{pmatrix}
      u \\
      v
    \end{pmatrix} = \det\begin{pmatrix}
      u \\
      v
    \end{pmatrix}.
  \]
  We establish this equation by verifying that the three conditions of \cref{ex:4.1.11} are satisfied by the function
  \[
    \delta\begin{pmatrix}
      u \\
      v
    \end{pmatrix} = \mathbf{O}\begin{pmatrix}
      u \\
      v
    \end{pmatrix} \cdot \mathbf{A}\begin{pmatrix}
      u \\
      v
    \end{pmatrix}.
  \]
  \begin{enumerate}
    \item We begin by showing that for any real number \(c\)
          \[
            \delta\begin{pmatrix}
              u \\
              cv
            \end{pmatrix} = c \cdot \delta\begin{pmatrix}
              u \\
              v
            \end{pmatrix}.
          \]
          Observe that this equation is valid if \(c = 0\) because
          \[
            \delta\begin{pmatrix}
              u \\
              cv
            \end{pmatrix} = \mathbf{O}\begin{pmatrix}
              u \\
              \zv
            \end{pmatrix} \cdot \mathbf{A}\begin{pmatrix}
              u \\
              \zv
            \end{pmatrix} = 1 \cdot 0 = 0.
          \]
          So assume that \(c \neq 0\).
          Regarding \(cv\) as the base of the parallelogram determined by \(u\) and \(cv\), we see that
          \[
            \mathbf{A}\begin{pmatrix}
              u \\
              cv
            \end{pmatrix} = \text{base } \times \text{ altitude} = \abs{c} (\text{length of } v) (\text{altitude}) = \abs{c} \cdot \mathbf{A}\begin{pmatrix}
              u \\
              v
            \end{pmatrix},
          \]
          since the altitude of the parallelogram determined by \(u\) and \(cv\) is the same as that in the parallelogram determined by \(u\) and \(v\).
          Hence
          \begin{align*}
            \delta\begin{pmatrix}
                    u \\
                    cv
                  \end{pmatrix} & = \mathbf{O}\begin{pmatrix}
                                                u \\
                                                cv
                                              \end{pmatrix} \cdot \mathbf{A}\begin{pmatrix}
                                                                              u \\
                                                                              cv
                                                                            \end{pmatrix}                                                            \\
                                  & = \pa{\frac{c}{\abs{c}} \mathbf{O}\begin{pmatrix}
                                                                          u \\
                                                                          v
                                                                        \end{pmatrix}} \pa{\abs{c} \mathbf{A}\begin{pmatrix}
                                                                                                               u \\
                                                                                                               v
                                                                                                             \end{pmatrix}} &  & \text{(by \cref{4.1})} \\
                                  & = c \cdot \mathbf{O}\begin{pmatrix}
                                                          u \\
                                                          v
                                                        \end{pmatrix} \cdot \mathbf{A}\begin{pmatrix}
                                                                                        u \\
                                                                                        v
                                                                                      \end{pmatrix}                                                  \\
                                  & = c \cdot \delta\begin{pmatrix}
                                                      u \\
                                                      v
                                                    \end{pmatrix}.
          \end{align*}
          A similar argument shows that
          \[
            \delta\begin{pmatrix}
              cu \\
              v
            \end{pmatrix} = c \cdot \delta\begin{pmatrix}
              u \\
              v
            \end{pmatrix}.
          \]

          We next prove that
          \[
            \delta\begin{pmatrix}
              u \\
              au + bw
            \end{pmatrix} = b \cdot \delta\begin{pmatrix}
              u \\
              w
            \end{pmatrix}
          \]
          for any \(u, w \in \R^2\) and any real numbers \(a\) and \(b\).
          Because the parallelograms determined by \(u\) and \(w\) and by \(u\) and \(u + w\) have a common base \(u\) and the same altitude, it follows that
          \[
            \mathbf{A}\begin{pmatrix}
              u \\
              w
            \end{pmatrix} = \mathbf{A}\begin{pmatrix}
              u \\
              u + w
            \end{pmatrix}.
          \]
          If \(a = 0\), then
          \[
            \delta\begin{pmatrix}
              u \\
              au + bw
            \end{pmatrix} = \delta\begin{pmatrix}
              u \\
              bw
            \end{pmatrix} = b \cdot \delta\begin{pmatrix}
              u \\
              w
            \end{pmatrix}
          \]
          by the first paragraph of (a).
          Otherwise, if \(a \neq 0\), then
          \[
            \delta\begin{pmatrix}
              u \\
              au + bw
            \end{pmatrix} = a \cdot \delta\begin{pmatrix}
              u \\
              u + \frac{b}{a} w
            \end{pmatrix} = a \cdot \delta\begin{pmatrix}
              u \\
              \frac{b}{a} w
            \end{pmatrix} = b \cdot \delta\begin{pmatrix}
              u \\
              w
            \end{pmatrix}.
          \]
          So the desired conclusion is obtained in either case.

          We are now able to show that
          \[
            \delta\begin{pmatrix}
              u \\
              v_1 + v_2
            \end{pmatrix} = \delta\begin{pmatrix}
              u \\
              v_1
            \end{pmatrix} + \delta\begin{pmatrix}
              u \\
              v_2
            \end{pmatrix}
          \]
          for all \(u, v_1, v_2 \in \R^2\).
          Since the result is immediate if \(u = 0\), we assume that \(u \neq 0\).
          Choose any vector \(w \in \R^2\) such that \(\set{u, w}\) is linearly independent.
          Then for any vectors \(v_1, v_2 \in \R^2\) there exist scalars \(a_i\) and \(b_i\) such that \(v_i = a_i u + b_i w\) (\(i \in \set{1, 2}\)).
          Thus
          \begin{align*}
            \delta\begin{pmatrix}
                    u \\
                    v_1 + v_2
                  \end{pmatrix} & = \delta\begin{pmatrix}
                                            u \\
                                            (a_1 + a_2) u + (b_1 + b_2) w
                                          \end{pmatrix}           \\
                                  & = (b_1 + b_2) \delta\begin{pmatrix}
                                                          u \\
                                                          w
                                                        \end{pmatrix}           \\
                                  & = \delta\begin{pmatrix}
                                              u \\
                                              a_1 u + b_1 w
                                            \end{pmatrix} + \delta\begin{pmatrix}
                                                                    u \\
                                                                    a_2 u + b_2 w
                                                                  \end{pmatrix} \\
                                  & = \delta\begin{pmatrix}
                                              u \\
                                              v_1
                                            \end{pmatrix} + \delta\begin{pmatrix}
                                                                    u \\
                                                                    v_2
                                                                  \end{pmatrix}.
          \end{align*}
          A similar argument shows that
          \[
            \delta\begin{pmatrix}
              u_1 + u_2 \\
              v
            \end{pmatrix} = \delta\begin{pmatrix}
              u_1 \\
              v
            \end{pmatrix} + \delta\begin{pmatrix}
              u_2 \\
              v
            \end{pmatrix}
          \]
          for all \(u_1, u_2, v \in \R^2\).
    \item Since
          \[
            \mathbf{A}\begin{pmatrix}
              u \\
              u
            \end{pmatrix} = 0,
          \]
          it follows that
          \[
            \delta\begin{pmatrix}
              u \\
              u
            \end{pmatrix} = \mathbf{O}\begin{pmatrix}
              u \\
              u
            \end{pmatrix} \cdot \mathbf{A}\begin{pmatrix}
              u \\
              u
            \end{pmatrix} = 0
          \]
          for any \(u \in \R^2\).
    \item Because the parallelogram determined by \(e_1\) and \(e_2\) is the unit square,
          \[
            \delta\begin{pmatrix}
              e_1 \\
              e_2
            \end{pmatrix} = \mathbf{O}\begin{pmatrix}
              e_1 \\
              e_2
            \end{pmatrix} \cdot \mathbf{A}\begin{pmatrix}
              e_1 \\
              e_2
            \end{pmatrix} = 1 \cdot 1 = 1.
          \]
          Therefore \(\delta\) satisfies the three conditions of \cref{ex:4.1.11}, and hence \(\delta = \det\).
          So the area of the parallelogram determined by \(u\) and \(v\) equals
          \[
            \mathbf{O}\begin{pmatrix}
              u \\
              v
            \end{pmatrix} \cdot \det\begin{pmatrix}
              u \\
              v
            \end{pmatrix}.
          \]
  \end{enumerate}
\end{proof}

\exercisesection

\setcounter{ex}{4}
\begin{ex}\label{ex:4.1.5}
  Prove that if \(B\) is the matrix obtained by interchanging the rows of \(A \in \ms{2}{2}{\F}\), then \(\det(B) = -\det(A)\).
\end{ex}

\begin{proof}[\pf{ex:4.1.5}]
  We have
  \begin{align*}
    \det(B) & = \det\begin{pmatrix}
                      A_{2 1} & A_{2 2} \\
                      A_{1 1} & A_{1 2}
                    \end{pmatrix}                                              \\
            & = A_{2 1} A_{1 2} - A_{2 2} A_{1 1}    &  & \text{(by \cref{4.1.1})} \\
            & = -(A_{1 1} A_{2 2} - A_{1 2} A_{2 1})                               \\
            & = -\det(A).                            &  & \text{(by \cref{4.1.1})}
  \end{align*}
\end{proof}

\begin{ex}\label{ex:4.1.6}
  Prove that if the two columns of \(A \in \ms{2}{2}{\F}\) are identical, then \(\det(A) = 0\).
\end{ex}

\begin{proof}[\pf{ex:4.1.6}]
  We have
  \begin{align*}
             & (A_{1 1}, A_{2 1}) = (A_{1 2}, A_{2 2})                                                                            \\
    \implies & \det(A) = A_{1 1} A_{2 2} - A_{1 2} A_{2 1} = A_{1 1} A_{2 2} - A_{1 1} A_{2 2} = 0. &  & \text{(by \cref{4.1.1})}
  \end{align*}
\end{proof}

\begin{ex}\label{ex:4.1.7}
  Prove that \(\det(\tp{A}) = \det(A)\) for any \(A \in \ms{2}{2}{\F}\).
\end{ex}

\begin{proof}[\pf{ex:4.1.7}]
  We have
  \begin{align*}
    \det(\tp{A}) & = \det\begin{pmatrix}
                           A_{1 1} & A_{2 1} \\
                           A_{1 2} & A_{2 2}
                         \end{pmatrix}               &  & \text{(by \cref{1.3.3})}   \\
                 & = A_{1 1} A_{2 2} - A_{2 1} A_{1 2} &  & \text{(by \cref{4.1.1})} \\
                 & = A_{1 1} A_{2 2} - A_{1 2} A_{2 1}                               \\
                 & = \det(A).                          &  & \text{(by \cref{4.1.1})}
  \end{align*}
\end{proof}

\begin{ex}\label{ex:4.1.8}
  Prove that if \(A \in \ms{2}{2}{\F}\) is upper triangular, then \(\det(A)\) equals the product of the diagonal entries of \(A\).
\end{ex}

\begin{proof}[\pf{ex:4.1.8}]
  We have
  \begin{align*}
             & A_{2 1} = 0                                                    &  & \text{(by \cref{ex:1.3.12})} \\
    \implies & \det(A) = A_{1 1} A_{2 2} - A_{1 2} A_{2 1} = A_{1 1} A_{2 2}. &  & \text{(by \cref{4.1.1})}
  \end{align*}
\end{proof}

\begin{ex}\label{ex:4.1.9}
  Prove that \(\det(AB) = \det(A) \cdot \det(B)\) for any \(A, B \in \ms{2}{2}{\F}\).
\end{ex}

\begin{proof}[\pf{ex:4.1.9}]
  We have
  \begin{align*}
     & \det(AB)                                                                                                                                      \\
     & = \det\begin{pmatrix}
               A_{1 1} B_{1 1} + A_{1 2} B_{2 1} & A_{1 1} B_{1 2} + A_{1 2} B_{2 2} \\
               A_{2 1} B_{1 1} + A_{2 2} B_{2 1} & A_{2 1} B_{1 2} + A_{2 2} B_{2 2}
             \end{pmatrix}                                        &  & \text{(by \cref{2.3.1})}                                                      \\
     & = (A_{1 1} B_{1 1} + A_{1 2} B_{2 1}) (A_{2 1} B_{1 2} + A_{2 2} B_{2 2})                                                                     \\
     & \quad - (A_{1 1} B_{1 2} + A_{1 2} B_{2 2}) (A_{2 1} B_{1 1} + A_{2 2} B_{2 1})                                 &  & \text{(by \cref{4.1.1})} \\
     & = (A_{1 1} A_{2 2}) (B_{1 1} B_{2 2} - B_{1 2} B_{2 1}) - (A_{1 2} A_{2 1}) (B_{1 1} B_{2 2} - B_{1 2} B_{2 1})                               \\
     & = (A_{1 1} A_{2 2} - A_{1 2} A_{2 1}) (B_{1 1} B_{2 2} - B_{1 2} B_{2 1})                                                                     \\
     & = \det(A) \det(B).                                                                                              &  & \text{(by \cref{4.1.1})}
  \end{align*}
\end{proof}

\begin{ex}\label{ex:4.1.10}
  The \textbf{classical adjoint} of a matrix \(A \in \ms{2}{2}{\F}\) is the matrix
  \[
    C = \begin{pmatrix}
      A_{2 2}  & -A_{1 2} \\
      -A_{2 1} & A_{1 1}
    \end{pmatrix}.
  \]
  Prove that
  \begin{enumerate}
    \item \(CA = AC = \det(A) I\).
    \item \(\det(C) = \det(A)\).
    \item The classical adjoint of \(\tp{A}\) is \(\tp{C}\).
    \item If \(A\) is invertible, then \(A^{-1} = (\det(A))^{-1} C\).
  \end{enumerate}
\end{ex}

\begin{proof}[\pf{ex:4.1.10}(a)]
  We have
  \begin{align*}
    CA & = \begin{pmatrix}
             C_{1 1} A_{1 1} + C_{1 2} A_{2 1} & C_{1 1} A_{1 2} + C_{1 2} A_{2 2} \\
             C_{2 1} A_{1 1} + C_{2 2} A_{2 1} & C_{2 1} A_{1 2} + C_{2 2} A_{2 2}
           \end{pmatrix}   &  & \text{(by \cref{2.3.1})}                                  \\
       & = \begin{pmatrix}
             A_{2 2} A_{1 1} - A_{1 2} A_{2 1}  & A_{2 2} A_{1 2} - A_{1 2} A_{2 2}  \\
             -A_{2 1} A_{1 1} + A_{1 1} A_{2 1} & -A_{2 1} A_{1 2} + A_{1 1} A_{2 2}
           \end{pmatrix} &  & \text{(by \cref{ex:4.1.10})}                                \\
       & = \begin{pmatrix}
             \det(A) & 0       \\
             0       & \det(A)
           \end{pmatrix}                                                       &  & \text{(by \cref{4.1.1})}      \\
       & = \det(A) I                                                                &  & \text{(by \cref{1.2.9})}
  \end{align*}
  and
  \begin{align*}
    AC & = \begin{pmatrix}
             A_{1 1} C_{1 1} + A_{1 2} C_{2 1} & A_{1 1} C_{1 2} + A_{1 2} C_{2 2} \\
             A_{2 1} C_{1 1} + A_{2 2} C_{2 1} & A_{2 1} C_{1 2} + A_{2 2} C_{2 2}
           \end{pmatrix}  &  & \text{(by \cref{2.3.1})}                                 \\
       & = \begin{pmatrix}
             A_{1 1} A_{2 2} - A_{1 2} A_{2 1} & -A_{1 1} A_{1 2} + A_{1 2} A_{1 1} \\
             A_{2 1} A_{2 2} - A_{2 2} A_{2 1} & -A_{2 1} A_{1 2} + A_{2 2} A_{1 1}
           \end{pmatrix} &  & \text{(by \cref{ex:4.1.10})}                                \\
       & = \begin{pmatrix}
             \det(A) & 0       \\
             0       & \det(A)
           \end{pmatrix}                                                      &  & \text{(by \cref{4.1.1})}      \\
       & = \det(A) I.                                                              &  & \text{(by \cref{1.2.9})}
  \end{align*}
\end{proof}

\begin{proof}[\pf{ex:4.1.10}(b)]
  We have
  \begin{align*}
    \det(C) & = A_{2 2} A_{1 1} - (-A_{1 2}) (-A_{2 1}) &  & \text{(by \cref{4.1.1})} \\
            & = A_{1 1} A_{2 2} - A_{1 2} A_{2 1}                                     \\
            & = \det(A).                                &  & \text{(by \cref{4.1.1})}
  \end{align*}
\end{proof}

\begin{proof}[\pf{ex:4.1.10}(c)]
  We have
  \begin{align*}
     & \text{the classical adjoint of } \tp{A}                                   \\
     & = \begin{pmatrix}
           (\tp{A})_{2 2}  & -(\tp{A})_{1 2} \\
           -(\tp{A})_{2 1} & (\tp{A})_{1 1}
         \end{pmatrix}    &  & \text{(by \cref{ex:4.1.10})}                      \\
     & = \begin{pmatrix}
           A_{2 2}  & -A_{2 1} \\
           -A_{1 2} & A_{1 1}
         \end{pmatrix}                  &  & \text{(by \cref{1.3.3})}            \\
     & = \tp{C}.                               &  & \text{(by \cref{ex:4.1.10})}
  \end{align*}
\end{proof}

\begin{proof}[\pf{ex:4.1.10}(d)]
  We have
  \begin{align*}
    \frac{1}{\det(A)} C & = \frac{1}{\det(A)} \begin{pmatrix}
                                                A_{2 2}  & -A_{1 2} \\
                                                -A_{2 1} & A_{1 1}
                                              \end{pmatrix} &  & \text{(by \cref{ex:4.1.10})} \\
                        & = A^{-1}.                           &  & \text{(by \cref{4.2})}
  \end{align*}
\end{proof}

\begin{ex}\label{ex:4.1.11}
  Let \(\delta : \ms{2}{2}{\F} \to \F\) be a function with the following three properties.
  \begin{enumerate}
    \item \(\delta\) is a linear function of each row of the matrix when the other row is held fixed.
    \item If the two rows of \(A \in \ms{2}{2}{\F}\) are identical, then \(\delta(A) = 0\).
    \item \(\delta(I_2) = 1\).
  \end{enumerate}
  Prove that \(\delta(A) = \det(A)\) for all \(A \in \ms{2}{2}{\F}\).
\end{ex}

\begin{proof}[\pf{ex:4.1.11}]
  Let \(A \in \ms{n}{n}{\F}\), let \(e_1 = (1, 0)\) and let \(e_2 = (0, 1)\).
  Since
  \begin{align*}
    0 & = \delta\begin{pmatrix}
                  e_1 + e_2 \\
                  e_1 + e_2
                \end{pmatrix}               &  & \text{(by \cref{ex:4.1.11}(b))}                                                     \\
      & = \delta\begin{pmatrix}
                  e_1 \\
                  e_1 + e_2
                \end{pmatrix} + \delta\begin{pmatrix}
                                        e_2 \\
                                        e_1 + e_2
                                      \end{pmatrix} &  & \text{(by \cref{ex:4.1.11}(a))}                                             \\
      & = \delta\begin{pmatrix}
                  e_1 \\
                  e_1
                \end{pmatrix} + \delta\begin{pmatrix}
                                        e_1 \\
                                        e_2
                                      \end{pmatrix} + \delta\begin{pmatrix}
                                                              e_2 \\
                                                              e_1
                                                            \end{pmatrix} + \delta\begin{pmatrix}
                                                                                    e_2 \\
                                                                                    e_2
                                                                                  \end{pmatrix} &  & \text{(by \cref{ex:4.1.11}(a))} \\
      & = 0 + 1 + \delta\begin{pmatrix}
                          e_2 \\
                          e_1
                        \end{pmatrix} + 0,       &  & \text{(by \cref{ex:4.1.11}(b)(c))}
  \end{align*}
  we know that
  \[
    \delta\begin{pmatrix}
      e_2 \\
      e_1
    \end{pmatrix} = -1
  \]
  and thus
  \begin{align*}
    \delta(A) & = \delta\begin{pmatrix}
                          A_{1 1} & A_{1 2} \\
                          A_{2 1} & A_{2 2}
                        \end{pmatrix}                                                                                                   \\
              & = \delta\begin{pmatrix}
                          A_{1 1} e_1 + A_{1 2} e_2 \\
                          A_{2 1} e_1 + A_{2 2} e_2
                        \end{pmatrix}                                                                                           \\
              & = A_{1 1} \delta\begin{pmatrix}
                                  e_1 \\
                                  A_{2 1} e_1 + A_{2 2} e_2
                                \end{pmatrix} + A_{1 2} \delta\begin{pmatrix}
                                                                e_2 \\
                                                                A_{2 1} e_1 + A_{2 2} e_2
                                                              \end{pmatrix}             &  & \text{(by \cref{ex:4.1.11}(a))}                \\
              & = A_{1 1} \pa{A_{2 1} \delta\begin{pmatrix}
                                                e_1 \\
                                                e_1
                                              \end{pmatrix} + A_{2 2} \delta\begin{pmatrix}
                                                                              e_1 \\
                                                                              e_2
                                                                            \end{pmatrix}}             &  & \text{(by \cref{ex:4.1.11}(a))}   \\
              & \quad + A_{1 2} \pa{A_{2 1} \delta\begin{pmatrix}
                                                      e_2 \\
                                                      e_1
                                                    \end{pmatrix} + A_{2 2} \delta\begin{pmatrix}
                                                                                    e_2 \\
                                                                                    e_2
                                                                                  \end{pmatrix}}         &  & \text{(by \cref{ex:4.1.11}(a))} \\
              & = A_{1 1} A_{2 2} \delta\begin{pmatrix}
                                          e_1 \\
                                          e_2
                                        \end{pmatrix} + A_{1 2} A_{2 1} \delta\begin{pmatrix}
                                                                                e_2 \\
                                                                                e_1
                                                                              \end{pmatrix}     &  & \text{(by \cref{ex:4.1.11}(b))}        \\
              & = A_{1 1} A_{2 2} + A_{1 2} A_{2 1} \delta\begin{pmatrix}
                                                            e_2 \\
                                                            e_1
                                                          \end{pmatrix} &  & \text{(by \cref{ex:4.1.11}(c))}                                \\
              & = A_{1 1} A_{2 2} - A_{1 2} A_{2 1}                       &  & \text{(from the proof above)}                                \\
              & = \det(A).                                                &  & \text{(by \cref{4.1.1})}
  \end{align*}
  Since \(A\) is arbitrary, we conclude that \(\delta = \det\).
\end{proof}

\begin{ex}\label{ex:4.1.12}
  Let \(\set{u, v}\) be an ordered basis for \(\R^2\).
  Prove that
  \[
    \mathbf{O}\begin{pmatrix}
      u \\
      v
    \end{pmatrix} = 1
  \]
  iff \(\set{u, v}\) forms a right-handed coordinate system.
\end{ex}

\begin{proof}[\pf{ex:4.1.12}]
  First suppose that \(\set{u, v}\) forms a right-handed coordinate system.
  By \cref{4.1.2} there exists a \(\theta \in (0, \pi)\) such that by rotating \(u\) with angle \(\theta\) and scaling \(u\) with some \(t > 0\) we get \(v\).
  Note that \(\theta \neq 0\) since \(\set{u, v}\) is a basis for \(\R^2\) over \(\R\).
  By \cref{2.1.3} this means
  \[
    v = t (u_1 \cos(\theta) - u_2 \sin(\theta), u_1 \sin(\theta) + u_2 \cos(\theta)).
  \]
  Thus we have
  \begin{align*}
    \det\begin{pmatrix}
          u \\
          v
        \end{pmatrix} & = \det\begin{pmatrix}
                                u_1                                     & u_2                                     \\
                                t u_1 \cos(\theta) - t u_2 \sin(\theta) & t u_1 \sin(\theta) + t u_2 \cos(\theta)
                              \end{pmatrix}                              \\
                        & = t \det\begin{pmatrix}
                                    u_1                                 & u_2                                 \\
                                    u_1 \cos(\theta) - u_2 \sin(\theta) & u_1 \sin(\theta) + u_2 \cos(\theta)
                                  \end{pmatrix}         &  & \text{(by \cref{4.1})}                                  \\
                        & = t (u_1^2 \sin(\theta) + u_2^2 \sin(\theta))                                        &  & \text{(by \cref{4.1.1})} \\
                        & > 0.                                                                                 &  & (\theta \in (0, \pi))
  \end{align*}
  By \cref{4.1.2} we have
  \[
    \mathbf{O}\begin{pmatrix}
      u \\
      v
    \end{pmatrix} = \frac{\det\begin{pmatrix}
        u \\
        v
      \end{pmatrix}}{\abs{\det\begin{pmatrix}
          u \\
          v
        \end{pmatrix}}} = \frac{\det\begin{pmatrix}
        u \\
        v
      \end{pmatrix}}{\det\begin{pmatrix}
        u \\
        v
      \end{pmatrix}} = 1.
  \]

  Now suppose that
  \[
    \mathbf{O}\begin{pmatrix}
      u \\
      v
    \end{pmatrix} = 1.
  \]
  Since \(\set{u, v}\) is a basis for \(\R^2\), by rotating \(u\) with angle \(\theta > 0\) and scaling \(u\) with \(t > 0\) we get \(v\).
  By \cref{2.1.3} this means
  \[
    v = t (u_1 \cos(\theta) - u_2 \sin(\theta), u_1 \sin(\theta) + u_2 \cos(\theta)).
  \]
  Then we have
  \begin{align*}
             & \mathbf{O}\begin{pmatrix}
                           u \\
                           v
                         \end{pmatrix} = 1                                                        \\
    \implies & \det\begin{pmatrix}
                     u \\
                     v
                   \end{pmatrix} > 0                             &  & \text{(by \cref{4.1.2})}    \\
    \implies & u_1 v_2 - u_2 v_1 > 0                           &  & \text{(by \cref{4.1.1})}      \\
    \implies & t u_1^2 \sin(\theta) + t u_2^2 \sin(\theta) > 0 &  & \text{(from the proof above)} \\
    \implies & \sin(\theta) > 0                                &  & (t > 0)                       \\
    \implies & \theta \in (0, \pi)
  \end{align*}
  and thus by \cref{4.1.2} \(\set{u, v}\) forms a right-handed coordinate system.
\end{proof}

\section{Determinants of Order \textit{n}}\label{sec:4.2}

\begin{defn}\label{4.2.1}
  Given \(A \in \ms{n}{n}{\F}\), for \(n \geq 2\), denote the \((n - 1) \times (n - 1)\) matrix obtained from \(A\) by deleting row \(i\) and column \(j\) by \(\tilde{A}_{i j}\).
\end{defn}

\begin{defn}\label{4.2.2}
  Let \(A \in \ms{n}{n}{\F}\).
  If \(n = 1\), so that \(A = (A_{1 1})\), we define \(\det(A) = A_{1 1}\).
  For \(n \geq 2\), we define \(\det(A)\) recursively as
  \[
    \det(A) = \sum_{j = 1}^n (-1)^{1 + j} A_{1 j} \cdot \det(\tilde{A}_{1 j}).
  \]
  The scalar \(\det(A)\) is called the \textbf{determinant} of \(A\) and is also denoted by \(\abs{A}\).
  The scalar
  \[
    (-1)^{i + j} \det(\tilde{A}_{i j})
  \]
  is called the \textbf{cofactor} of the entry of \(A\) in row \(i\), column \(j\).

  Letting
  \[
    c_{i j} = (-1)^{i + j} \det(\tilde{A}_{i j})
  \]
  denote the cofactor of the row \(i\), column \(j\) entry of \(A\), we can express the formula for the determinant of \(A\) as
  \[
    \det(A) = A_{1 1} c_{1 1} + A_{1 2} c_{1 2} + \cdots + A_{1 n} c_{1 n}.
  \]
  Thus the determinant of \(A\) equals the sum of the products of each entry in row \(1\) of \(A\) multiplied by its cofactor.
  This formula is called \textbf{cofactor expansion along the first row of \(A\)}.
  Note that, for \(2 \times 2\) matrices, this definition of the determinant of \(A\) agrees with the one given in \cref{4.1.1} because
  \[
    \det(A) = A_{1 1} (-1)^{1 + 1} \det(\tilde{A}_{1 1}) + A_{1 2} (-1)^{1 + 2} \det(\tilde{A}_{1 2}) = A_{1 1} A_{2 2} - A_{1 2} A_{2 1}.
  \]
\end{defn}

\begin{eg}\label{4.2.3}
  The determinant of the identity matrix \(I_n\) is \(1\).
\end{eg}

\begin{proof}[\pf{4.2.3}]
  We prove this assertion by mathematical induction on \(n\).
  The result is clearly true for the \(I_1\).
  Assume that the determinant of \(I_n\) is \(1\) for some \(n \geq 1\).
  Using cofactor expansion along the first row of \(I_{n + 1}\), we obtain
  \begin{align*}
     & \det(I_{n + 1})                                                                                                                             \\
     & = (-1)^2 (1) \cdot \det(\tilde{(I_{n + 1})}_{1 1}) + (-1)^3 (0) \cdot \det(\tilde{(I_{n + 1})}_{1 2})                                       \\
     & \quad + \cdots  + (-1)^{n + 2} (0) \cdot \det(\tilde{(I_{n + 1})}_{1 (n + 1)})                        &  & \text{(by \cref{4.2.2})}         \\
     & = 1(1) + 0 + \cdots + 0                                                                               &  & \text{(by induction hypothesis)} \\
     & = 1
  \end{align*}
  because \(\tilde{(I_{n + 1})}_{1 1} = I_n\).
  This shows that the determinant of \(I_{n + 1}\) is \(1\), and so the determinant of any identity matrix is \(1\) by the principle of mathematical induction.
\end{proof}

\begin{thm}\label{4.3}
  The determinant of an \(n \times n\) matrix is a linear function of each row when the remaining rows are held fixed.
  That is, for \(1 \leq r \leq n\), we have
  \[
    \det\begin{pmatrix}
      a_1       \\
      \vdots    \\
      a_{r - 1} \\
      u + kv    \\
      a_{r + 1} \\
      \vdots    \\
      a_n
    \end{pmatrix} = \det\begin{pmatrix}
      a_1       \\
      \vdots    \\
      a_{r - 1} \\
      u         \\
      a_{r + 1} \\
      \vdots    \\
      a_n
    \end{pmatrix} + k \det\begin{pmatrix}
      a_1       \\
      \vdots    \\
      a_{r - 1} \\
      v         \\
      a_{r + 1} \\
      \vdots    \\
      a_n
    \end{pmatrix}
  \]
  whenever \(k \in \F\) and \(u, v\), and each \(a_i\) are row vectors in \(\vs{F}^n\).
\end{thm}

\begin{proof}[\pf{4.3}]
  The proof is by mathematical induction on \(n\).
  The result is immediate if \(n = 1\).
  Assume that for some integer \(n \geq 1\) the determinant of any \(n \times n\) matrix is a linear function of each row when the remaining rows are held fixed.
  Let \(A \in \ms{(n + 1)}{(n + 1)}{\F}\) with rows \(\seq{a}{1,,n+1}\), and suppose that for some \(r \in \set{1, \dots, n + 1}\), we have \(a_r = u + kv\) for some \(u, v \in \vs{F}^{n + 1}\) and some \(k \in \F\).
  Let \(u = \tuple{b}{1,,n+1}\) and \(v = \tuple{c}{1,,n+1}\), and let \(B\) and \(C\) be the matrices obtained from \(A\) by replacing row \(r\) of \(A\) by \(u\) and \(v\), respectively.
  We must prove that \(\det(A) = \det(B) + k \det(C)\).
  If \(r = 1\), we have
  \begin{align*}
    \det(A) & = \sum_{j = 1}^{n + 1} (-1)^{1 + j} A_{1 j} \det(\tilde{A}_{1 j})                                                                     &  & \text{(by \cref{4.2.2})} \\
            & = \sum_{j = 1}^{n + 1} (-1)^{1 + j} (u + kv)_j \det(\tilde{A}_{1 j})                                                                                                \\
            & = \sum_{j = 1}^{n + 1} (-1)^{1 + j} (b_j + k c_j) \det(\tilde{A}_{1 j})                                                                                             \\
            & = \sum_{j = 1}^{n + 1} (-1)^{1 + j} b_j \det(\tilde{A}_{1 j}) + k \sum_{j = 1}^{n + 1} (-1)^{1 + j} c_j \det(\tilde{A}_{1 j})                                       \\
            & = \sum_{j = 1}^{n + 1} (-1)^{1 + j} B_{1 j} \det(\tilde{A}_{1 j}) + k \sum_{j = 1}^{n + 1} (-1)^{1 + j} C_{1 j} \det(\tilde{A}_{1 j})                               \\
            & = \det(B) + k \det(C).                                                                                                                &  & \text{(by \cref{4.2.2})}
  \end{align*}
  For \(r > 1\) and \(j \in \set{1, \dots, n + 1}\), the rows of \(\tilde{A}_{1 j}, \tilde{B}_{1 j}, \tilde{C}_{1 j}\) are the same except for row \(r - 1\).
  Moreover, row \(r - 1\) of \(\tilde{A}_{1 j}\) is
  \[
    (b_1 + k c_1, \dots, b_{j - 1} + k c_{j - 1}, b_{j + 1} + k c_{j + 1}, \dots, b_{n + 1} + k c_{n + 1}),
  \]
  which is the sum of row \(r - 1\) of \(\tilde{B}_{1 j}\) and \(k\) times row \(r - 1\) of \(\tilde{C}_{1 j}\).
  Since \(\tilde{B}_{1 j}\) and \(\tilde{C}_{1 j}\) are \(n \times n\) matrices, we have
  \[
    \det(\tilde{A}_{1 j}) = \det(\tilde{B}_{1 j}) + k \det(\tilde{C}_{1 j})
  \]
  by the induction hypothesis.
  Thus since \(A_{1 j} = B_{1 j} = C_{1 j}\), we have
  \begin{align*}
    \det(A) & = \sum_{j = 1}^{n + 1} (-1)^{1 + j} A_{1 j} \cdot \det(\tilde{A}_{1 j})                                                                           \\
            & = \sum_{j = 1}^{n + 1} (-1)^{1 + j} A_{1 j} \cdot \pa{\det(\tilde{B}_{1 j}) + k \det(\tilde{C}_{1 j})}                                            \\
            & = \sum_{j = 1}^{n + 1} (-1)^{1 + j} A_{1 j} \cdot \det(\tilde{B}_{1 j}) + k \sum_{j = 1}^{n + 1} (-1)^{1 + j} A_{1 j} \cdot \det(\tilde{C}_{1 j}) \\
            & = \det(B) + k \det(C).
  \end{align*}
  This shows that the theorem is true for \((n + 1) \times (n + 1)\) matrices, and so the theorem is true for all square matrices by mathematical induction.
\end{proof}


\chapter{Diagonalization}\label{ch:5}

% All sections are in separated files.  We include them here.

\chapter{Inner Product Spaces}\label{ch:6}

% All sections are in separated files.  We include them here.
\section{Inner Products and Norms}\label{sec:6.1}

\begin{defn}\label{6.1.1}
  Let \(\V\) be a vector space over \(\F\).
  An \textbf{inner product} on \(\V\) over \(\F\) is a function that assigns, to every ordered pair of vectors \(x\) and \(y\) in \(\V\), a scalar in \(\F\), denoted \(\inn{x, y}\), such that for all \(x\), \(y\), and \(z\) in \(\V\) and all \(c\) in \(\F\), the following hold:
  \begin{enumerate}
    \item \(\inn{x + z, y} = \inn{x, y} + \inn{z, y}\).
    \item \(\inn{cx, y} = c \inn{x, y}\).
    \item \(\conj{\inn{x, y}} = \inn{y, x}\), where the bar denotes complex conjugation.
    \item \(\inn{x, x} > 0\) if \(x \neq \zv\).
  \end{enumerate}
\end{defn}

\begin{note}
  Note that \cref{6.1.1}(c) reduces to \(\inn{x, y} = \inn{y, x}\) if \(\F = \R\).
  \cref{6.1.1}(a)(b) simply require that the inner product be linear in the first component.
  It is easily shown that if \(\seq{a}{1,,n} \in \F\) and \(y, \seq{v}{1,,n} \in \V\), then
  \[
    \inn{\sum_{i = 1}^n a_i v_i, y} = \sum_{i = 1}^n a_i \inn{v_i, y}.
  \]
\end{note}

\begin{eg}\label{6.1.2}
  For \(x = \tuple{a}{1,,n}\) and \(y = \tuple{b}{1,,n}\) in \(\vs{F}^n\), define
  \[
    \inn{x, y} = \sum_{i = 1}^n a_i \conj{b_i}.
  \]
  The inner product defined above is called the \textbf{standard inner product} on \(\vs{F}^n\).
  When \(\F = \R\) the conjugations are not needed, and in early courses this standard inner product is usually called the \emph{dot product} and is denoted by \(x \cdot y\) instead of \(\inn{x, y}\).
\end{eg}

\begin{proof}[\pf{6.1.2}]
  Let \(x, y, z \in \vs{F}^n\) and let \(c \in \F\).
  Since
  \begin{align*}
    \inn{x + y, z}    & = \sum_{i = 1}^n (x + y)_i \conj{z_i}                           &  & \text{(by \cref{6.1.2})}        \\
                      & = \sum_{i = 1}^n x_i \conj{z_i} + \sum_{i = 1}^n y_i \conj{z_i} &  & \text{(by \cref{c.0.1})}        \\
                      & = \inn{x, z} + \inn{y, z}                                       &  & \text{(by \cref{6.1.2})}        \\
    \inn{cx, y}       & = \sum_{i = 1}^n (cx)_i \conj{y_i}                              &  & \text{(by \cref{6.1.2})}        \\
                      & = c \sum_{i = 1}^n x_i \conj{y_i}                               &  & \text{(by \cref{c.0.1})}        \\
                      & = c \inn{x, y}                                                  &  & \text{(by \cref{6.1.2})}        \\
    \conj{\inn{x, y}} & = \conj{\sum_{i = 1}^n x_i \conj{y_i}}                          &  & \text{(by \cref{6.1.2})}        \\
                      & = \sum_{i = 1}^n \conj{x_i} y_i                                 &  & \text{(by \cref{d.2}(a)(b)(c))} \\
                      & = \sum_{i = 1}^n y_i \conj{x_i}                                 &  & \text{(by \cref{c.0.1})}        \\
                      & = \inn{y, x}                                                    &  & \text{(by \cref{6.1.2})}
  \end{align*}
  and
  \begin{align*}
             & x \neq \zv                                                                                       \\
    \implies & \exists i \in \set{1, \dots, n} : x_i \neq 0                                                     \\
    \implies & \exists i \in \set{1, \dots, n} : \abs{x_i}^2 = x_i \conj{x_i} > 0 &  & \text{(by \cref{d.0.5})} \\
    \implies & \inn{x, x} = \sum_{i = 1}^n x_i \conj{x_i} > 0,                    &  & \text{(by \cref{6.1.2})}
  \end{align*}
  by \cref{6.1.1} we know that \(\inn{\cdot, \cdot}\) is an inner product on \(\vs{F}^n\) over \(\F\).
\end{proof}

\begin{eg}\label{6.1.3}
  If \(\inn{x, y}\) is any inner product on a vector space \(\V\) over \(\F\) and \(r > 0\), we may define another inner product by the rule \(\inn{x, y}' = r \inn{x, y}\).
  If \(r \leq 0\), then \cref{6.1.1}(d) would not hold.
\end{eg}

\begin{proof}[\pf{6.1.3}]
  Let \(x, y, z \in \V\), let \(c \in \F\) and let \(r \in \R^+\).
  Since
  \begin{align*}
    \inn{x + y, z}'    & = r\inn{x + y, z}             &  & \text{(by \cref{6.1.3})}    \\
                       & = r (\inn{x, z} + \inn{y, z}) &  & \text{(by \cref{6.1.1}(a))} \\
                       & = r \inn{x, z} + r \inn{y, z} &  & \text{(by \cref{c.0.1})}    \\
                       & = \inn{x, z}' + \inn{y, z}'   &  & \text{(by \cref{6.1.3})}    \\
    \inn{cx, y}'       & = r \inn{cx, y}               &  & \text{(by \cref{6.1.3})}    \\
                       & = rc \inn{x, y}               &  & \text{(by \cref{6.1.1}(b))} \\
                       & = cr \inn{x, y}               &  & \text{(by \cref{c.0.1})}    \\
                       & = c \inn{x, y}'               &  & \text{(by \cref{6.1.3})}    \\
    \conj{\inn{x, y}'} & = \conj{r \inn{x, y}}         &  & \text{(by \cref{6.1.3})}    \\
                       & = \conj{r} \conj{\inn{x, y}}  &  & \text{(by \cref{d.2}(c))}   \\
                       & = \conj{r} \inn{y, x}         &  & \text{(by \cref{6.1.1}(c))} \\
                       & = r \inn{y, x}                &  & (r \in \R^+)                \\
                       & = \inn{y, x}'                 &  & \text{(by \cref{6.1.3})}
  \end{align*}
  and
  \begin{align*}
             & \begin{dcases}
                 x \neq \zv \\
                 r > 0
               \end{dcases}                                                    \\
    \implies & \inn{x, x}' = r \inn{x, x} > 0, &  & \text{(by \cref{6.1.1}(d))}
  \end{align*}
  by \cref{6.1.1} we see that \(\inn{\cdot, \cdot}'\) is an inner product on \(\V\) over \(\F\).
\end{proof}

\exercisesection

\begin{ex}\label{ex:6.1.15}

\end{ex}

\section{The Gram-Schmidt Orthogonalization Process and Orthogonal Complements}\label{sec:6.2}

\begin{defn}\label{6.2.1}
  Let \(\V\) be an inner product space over \(\F\).
  A subset of \(\V\) is an \textbf{orthonormal basis} for \(\V\) over \(\F\) if it is an ordered basis that is orthonormal.
\end{defn}

\begin{eg}\label{6.2.2}
  The standard ordered basis for \(\vs{F}^n\) over \(\F\) is an orthonormal basis for \(\vs{F}^n\) over \(\F\).
\end{eg}

\begin{thm}\label{6.3}
  Let \(\V\) be an inner product space over \(\F\) and \(S = \set{\seq{v}{1,,k}}\) be an orthogonal subset of \(\V\) consisting of nonzero vectors.
  If \(y \in \spn{S}\), then
  \[
    y = \sum_{i = 1}^k \frac{\inn{y, v_i}}{\norm{v_i}^2} v_i.
  \]
\end{thm}

\begin{proof}[\pf{6.3}]
  Write \(y = \sum_{i = 1}^k a_i v_i\), where \(\seq{a}{1,,k} \in \F\).
  Then, for \(j \in \set{1, \dots, k}\), we have
  \begin{align*}
    \inn{y, v_j} & = \inn{\sum_{i = 1}^k a_i v_i, v_j}                                     \\
                 & = \sum_{i = 1}^k a_i \inn{v_i, v_j} &  & \text{(by \cref{6.1.1}(a)(b))} \\
                 & = a_j \inn{v_j, v_j}                &  & \text{(by \cref{6.1.12})}      \\
                 & = a_j \norm{v_j}^2.                 &  & \text{(by \cref{6.1.9})}
  \end{align*}
  So \(a_j = \frac{\inn{y, v_j}}{\norm{v_j}^2}\), and the result follows.
\end{proof}

\begin{cor}\label{6.2.3}
  If, in addition to the hypotheses of \cref{6.3}, \(S\) is orthonormal and \(y \in \spn{S}\), then
  \[
    y = \sum_{i = 1}^k \inn{y, v_i} v_i.
  \]
\end{cor}

\begin{proof}[\pf{6.2.3}]
  We have
  \begin{align*}
    y & = \sum_{i = 1}^k \frac{\inn{y, v_i}}{\norm{v_i}^2} v_i &  & \text{(by \cref{6.3})}    \\
      & = \sum_{i = 1}^k \inn{y, v_i} v_i.                     &  & \text{(by \cref{6.1.12})}
  \end{align*}
\end{proof}

\begin{note}
  If \(\V\) possesses a finite orthonormal basis, then \cref{6.2.3} allows us to compute the coefficients in a linear combination very easily.
\end{note}

\begin{cor}\label{6.2.4}
  Let \(\V\) be an inner product space over \(\F\), and let \(S\) be an orthogonal subset of \(\V\) consisting of nonzero vectors.
  Then \(S\) is linearly independent.
\end{cor}

\begin{proof}[\pf{6.2.4}]
  Suppose that \(\seq{v}{1,,k} \in \S\) and
  \[
    \sum_{i = 1}^k a_i v_i = \zv.
  \]
  As in the proof of \cref{6.3} with \(y = \zv\), we have \(a_j = \inn{\zv, v_j} / \norm{v_j}^2 = 0\) for all \(j \in \set{1, \dots, k}\).
  So \(S\) is linearly independent.
\end{proof}

\begin{note}
  \cref{6.2.4} tells us that the vector space \(\vs{H}\) in \cref{6.1.13} contains an infinite linearly independent set, and hence \(\vs{H}\) is not a finite-dimensional vector space.
\end{note}

\begin{thm}\label{6.4}
  Let \(\V\) be an inner product space over \(\F\) and \(S = \set{\seq{w}{1,,n}}\) be a linearly independent subset of \(\V\).
  Define \(S' = \set{\seq{v}{1,,n}}\), where \(v_1 = w_1\) and
  \begin{equation}\label{eq:6.2.1}
    v_k = w_k - \sum_{j = 1}^{k - 1} \frac{\inn{w_k, v_j}}{\norm{v_j}^2} v_j \quad \text{for } k \in \set{2, \dots, n}.
  \end{equation}
  Then \(S'\) is an orthogonal set of nonzero vectors such that \(\spn{S'} = \spn{S}\).
  The construction of \(\set{\seq{v}{1,,n}}\) is called the \textbf{Gram--Schmidt process}.
\end{thm}

\begin{proof}[\pf{6.4}]
  The proof is by mathematical induction on \(n\), the number of vectors in \(S\).
  For \(k \in \set{1, \dots, n}\), let \(S_k = \set{\seq{w}{1,,k}}\).
  If \(n = 1\), then the theorem is proved by taking \(S_1' = S_1\);
  i.e., \(v_1 = w_1 \neq \zv\).
  Assume then that the set \(S_n' = \set{\seq{v}{1,,n}}\) with the desired properties has been constructed by the repeated use of \cref{eq:6.2.1}.
  We show that the set \(S_{n + 1}' = \set{\seq{v}{1,,n+1}}\) also has the desired properties, where \(v_{n + 1}\) is obtained from \(S_n'\) by \cref{eq:6.2.1}.
  If \(v_{n + 1} = \zv\), then \cref{eq:6.2.1} implies that \(w_{n + 1} \in \spn{S_n'} = \spn{S_n}\), which contradicts the assumption that \(S_{n + 1}\) is linearly independent.
  For \(i \in \set{1, \dots, n}\), it follows from \cref{eq:6.2.1} that
  \begin{align*}
    \inn{v_{n + 1}, v_i} & = \inn{w_{n + 1}, v_i} - \sum_{j = 1}^n \frac{\inn{w_{n + 1}, v_j}}{\norm{v_j}^2} \inn{v_j, v_i} &  & \text{(by \cref{6.1.1}(a)(b))}   \\
                         & = \inn{w_{n + 1}, v_i} - \frac{\inn{w_{n + 1}, v_i}}{\norm{v_i}^2} \norm{v_i}^2                  &  & \text{(by induction hypotheses)} \\
                         & = 0,
  \end{align*}
  since \(\inn{v_j, v_i} = 0\) if \(i \neq j\) by the induction assumption that \(S_n'\) is orthogonal.
  Hence \(S_{n + 1}'\) is an orthogonal set of nonzero vectors.
  Now, by \cref{eq:6.2.1}, we have that \(\spn{S_{n + 1}'} \subseteq \spn{S_{n + 1}}\).
  But by \cref{6.2.4}, \(S_{n + 1}'\) is linearly independent;
  so \(\dim(\spn{S_{n + 1}'}) = \dim(\spn{S_{n + 1}}) = n + 1\).
  Therefore by \cref{1.11} we have \(\spn{S_{n + 1}'} = \spn{S_{n + 1}}\) and this closes the induction.
\end{proof}

\begin{defn}\label{6.2.5}
  If we continue applying the Gram--Schmidt orthogonalization process to the basis \(\set{1, x, x^2, \dots}\) for \(\ps{\R}\), we obtain an orthogonal basis whose elements are called the \emph{Legendre polynomials}.
\end{defn}

\begin{thm}\label{6.5}
  Let \(\V\) be a nonzero finite-dimensional inner product space over \(\F\).
  Then \(\V\) has an orthonormal basis \(\beta\).
  Furthermore, if \(\beta = \set{\seq{v}{1,,n}}\) and \(x \in \V\), then
  \[
    x = \sum_{i = 1}^n \inn{x, v_i} v_i.
  \]
\end{thm}

\begin{proof}[\pf{6.5}]
  Let \(\beta_0\) be an ordered basis for \(\V\) over \(\F\).
  Apply \cref{6.4} to obtain an orthogonal set \(\beta'\) of nonzero vectors with \(\spn{\beta'} = \spn{\beta_0} = \V\).
  By normalizing each vector in \(\beta'\), we obtain an orthonormal set \(\beta\) that generates \(\V\).
  By \cref{6.2.4}, \(\beta\) is linearly independent;
  therefore \(\beta\) is an orthonormal basis for \(\V\) over \(\F\).
  The remainder of the theorem follows from \cref{6.2.3}.
\end{proof}

\begin{cor}\label{6.2.6}
  Let \(\V\) be a finite-dimensional inner product space over \(\F\) with an orthonormal basis \(\beta = \set{\seq{v}{1,,n}}\).
  Let \(\T \in \ls(\V)\), and let \(A = [\T]_{\beta}\).
  Then for any \(i, j \in \set{1, \dots, n}\), \(A_{i j} = \inn{\T(v_j), v_i}\).
\end{cor}

\begin{proof}[\pf{6.2.6}]
  From \cref{6.5}, we have
  \[
    \T(v_j) = \sum_{i = 1}^n \inn{\T(v_j), v_i} v_i.
  \]
  Hence by \cref{2.2.4} \(A_{i j} = \inn{\T(v_j), v_i}\).
\end{proof}

\begin{defn}\label{6.2.7}
  Let \(\beta\) be an orthonormal subset (possibly infinite) of an inner product space \(\V\) over \(\F\), and let \(x \in \V\).
  We define the \textbf{Fourier coefficients} of \(x\) relative to \(\beta\) to be the scalars \(\inn{x, y}\), where \(y \in \beta\).

  In the first half of the 19th century, the French mathematician Jean Baptiste Fourier was associated with the study of the scalars
  \[
    \int_0^{2\pi} f(t) \sin(nt) \; dt \quad \text{and} \quad \int_0^{2\pi} f(t) \cos(nt) \; dt,
  \]
  or more generally,
  \[
    c_n = \frac{1}{2\pi} \int_0^{2\pi} f(t) e^{-int} \; dt,
  \]
  for a function \(f\).
  In the context of \cref{6.1.8}, we see that \(c_n = \inn{f, f_n}\), where \(f_n(t) = e^{int}\);
  that is, \(c_n\) is the \(n\)th Fourier coefficient for a continuous function \(f \in \V\) relative to \(S\).
  These coefficients are the ``classical'' Fourier coefficients of a function, and the literature concerning the behavior of these coefficients is extensive.
\end{defn}

\begin{eg}\label{6.2.8}
  Let \(S = \set{e^{int} : n \text{ is an integer}}\).
  In \cref{6.1.13}, \(S\) was shown to be an orthonormal set in \(\vs{H}\).
  We compute the Fourier coefficients of \(f(t) = t\) relative to \(S\).
  Using integration by parts, we have, for \(n \neq 0\),
  \begin{align*}
    \inn{f, f_n} & = \frac{1}{2\pi} \int_0^{2\pi} t \conj{e^{int}} \; dt                                         \\
                 & = \frac{1}{2\pi} \int_0^{2\pi} t e^{-int} \; dt                                               \\
                 & = \frac{1}{2\pi} \pa{\eval{\frac{-1}{in} t e^{-int}}_0^{2\pi} - \int_0^{2\pi} e^{-int} \; dt} \\
                 & = \frac{1}{2\pi} \pa{\frac{-2\pi}{in} - \frac{-1}{in} \eval{e^{-int}}_0^{2\pi}}               \\
                 & = \frac{-1}{in},
  \end{align*}
  and, for \(n = 0\),
  \[
    \inn{f, 1} = \frac{1}{2\pi} \int_0^{2\pi} t(1) \; dt = \pi.
  \]
  As a result of these computations, and using \cref{ex:6.2.16}, we obtain an upper bound for the sum of a special infinite series as follows:
  \begin{align*}
    \norm{f}^2 & \geq \sum_{n = -k}^{-1} \abs{\inn{f, f_n}}^2 + \abs{\inn{f, 1}}^2 + \sum_{i = 1}^k \abs{\inn{f, f_n}}^2 \\
               & = \sum_{n = -k}^1 \frac{1}{n^2} + \pi^2 + \sum_{n = 1}^k \frac{1}{n^2}                                  \\
               & = 2 \sum_{n = 1}^k \frac{1}{n^2} + \pi^2
  \end{align*}
  for every \(k \in \Z^+\).
  Now, using the fact that \(\norm{f}^2 = \frac{4}{3} \pi^2\), we obtain
  \[
    \frac{4}{3} \pi^2 \geq 2 \sum_{n = 1}^k \frac{1}{n^2} + \pi^2,
  \]
  or
  \[
    \frac{\pi^2}{6} \geq \sum_{n = 1}^k \frac{1}{n^2}.
  \]
  Because this inequality holds for all \(k \in \Z^+\), we may let \(k \to \infty\) to obtain
  \[
    \frac{\pi^2}{6} \geq \sum_{n = 1}^\infty \frac{1}{n^2}.
  \]
  Additional results may be produced by replacing \(f\) by other functions.
\end{eg}

\begin{defn}\label{6.2.9}
  Let \(S\) be a nonempty subset of an inner product space \(\V\) over \(\F\).
  We define \(S^{\perp}\) (read ``\(S\) perp'') to be the set of all vectors in \(\V\) that are orthogonal to every vector in \(S\);
  that is, \(S^{\perp} = \set{x \in \V : \inn{x, y} = 0 \text{ for all } y \in S}\).
  The set \(S^{\perp}\) is called the \textbf{orthogonal complement} of \(S\).
\end{defn}

\begin{prop}\label{6.2.10}
  Let \(S\) be a nonempty subset of an inner product space \(\V\) over \(\F\).
  Then \(S^{\perp}\) is a subspace of \(\V\) over \(\F\).
\end{prop}

\begin{proof}[\pf{6.2.10}]
  Let \(x, y \in S^{\perp}\) and let \(c \in \F\).
  Since
  \begin{align*}
             & \forall z \in S, \inn{x, z} = \inn{y, z} = 0 &  & \text{(by \cref{6.2.9})}       \\
    \implies & \forall z \in S, \inn{cx + y, z} = 0         &  & \text{(by \cref{6.1.1}(a)(b))} \\
    \implies & cx + y \in S^{\perp}                         &  & \text{(by \cref{6.2.9})}
  \end{align*}
  and
  \begin{align*}
             & \forall z \in S, \inn{\zv, z} = 0 &  & \text{(by \cref{6.1}(c))} \\
    \implies & \zv \in S^{\perp},                &  & \text{(by \cref{6.2.9})}
  \end{align*}
  by \cref{ex:1.3.18} we see that \(S^{\perp}\) is a subspace of \(\V\) over \(\F\).
\end{proof}

\begin{eg}\label{6.2.11}
  \(\set{\zv}^{\perp} = \V\) and \(\V^{\perp} = \set{\zv}\) for any inner product space \(\V\) over \(\F\).
\end{eg}

\begin{proof}[\pf{6.2.11}]
  We have
  \begin{align*}
             & \forall x \in \V, \inn{x, \zv} = 0        &  & \text{(by \cref{6.1}(c))} \\
    \implies & \forall x \in \V, x \in \set{\zv}^{\perp} &  & \text{(by \cref{6.2.9})}  \\
    \implies & \V = \set{\zv}^{\perp}                    &  & \text{(by \cref{6.2.10})}
  \end{align*}
  and
  \begin{align*}
             & \forall x \in \V^{\perp}, \inn{x, x} = 0 &  & \text{(by \cref{6.2.9})}  \\
    \implies & \forall x \in \V^{\perp}, x = \zv        &  & \text{(by \cref{6.1}(d))} \\
    \implies & \V^{\perp} = \set{\zv}.
  \end{align*}
\end{proof}

\begin{thm}\label{6.6}
  Let \(\W\) be a finite-dimensional subspace of an inner product space \(\V\) over \(\F\), and let \(y \in \V\).
  Then there exist unique vectors \(u \in \W\) and \(z \in \W^{\perp}\) such that \(y = u + z\).
  Furthermore, if \(\set{\seq{v}{1,,k}}\) is an orthonormal basis for \(\W\), then
  \[
    u = \sum_{i = 1}^k \inn{y, v_i} v_i.
  \]
\end{thm}

\begin{proof}[\pf{6.6}]
  Let \(\set{\seq{v}{1,,k}}\) be an orthonormal basis for \(\W\) over \(\F\), let \(u\) be as defined in the preceding equation, and let \(z = y - u\).
  Clearly \(u \in \W\) and \(y = u + z\).

  To show that \(z \in \W^{\perp}\), it suffices to show, by \cref{ex:6.2.7}, that \(z\) is orthogonal to each \(v_j\).
  For any \(j \in \set{1, \dots, k}\), we have
  \begin{align*}
    \inn{z, v_j} & = \inn{y - \sum_{i = 1}^k \inn{y, v_i} v_i, v_j}                                                \\
                 & = \inn{y, v_j} - \sum_{i = 1}^k \inn{y, v_i} \inn{v_i, v_j} &  & \text{(by \cref{6.1.1}(a)(b))} \\
                 & = \inn{y, v_j} - \inn{y, v_j}                               &  & \text{(by \cref{6.1.12})}      \\
                 & = 0.
  \end{align*}
  To show uniqueness of \(u\) and \(z\), suppose that \(y = u + z = u' + z'\), where \(u' \in \W\) and \(z' \in \W^{\perp}\).
  Then \(u - u' = z' - z \in \W \cap \W^{\perp} = \set{\zv}\) (by \cref{6.2.11}).
  Therefore, \(u = u'\) and \(z = z'\).
\end{proof}

\begin{cor}\label{6.2.12}
  In the notation of \cref{6.6}, the vector \(u\) is the unique vector in \(\W\) that is ``closest'' to \(y\);
  that is, for any \(x \in \W\), \(\norm{y - x} \geq \norm{y - u}\), and this inequality is an equality iff \(x = u\).
  The vector \(u\) is called the \textbf{orthogonal projection} of \(y\) on \(W\).
\end{cor}

\begin{proof}[\pf{6.2.12}]
  As in \cref{6.6}, we have that \(y = u + z\), where \(z \in \W^{\perp}\).
  Let \(x \in \W\).
  Then \(u - x\) is orthogonal to \(z\), so we have
  \begin{align*}
    \norm{y - x}^2 & = \norm{u + z - x}^2                                          \\
                   & = \norm{(u - x) + z}^2                                        \\
                   & = \norm{u - x} + \norm{z}^2 &  & \text{(by \cref{ex:6.1.10})} \\
                   & \geq \norm{z}^2             &  & \text{(by \cref{6.2}(b))}    \\
                   & = \norm{y - u}^2.
  \end{align*}
  Now suppose that \(\norm{y - x} = \norm{y - u}\).
  Then the inequality above becomes an equality, and therefore \(\norm{u - x}^2 + \norm{z}^2 = \norm{z}^2\).
  It follows that \(\norm{u - x} = 0\), and hence \(x = u\).
  The proof of the converse is obvious.
\end{proof}

\begin{thm}\label{6.7}
  Suppose that \(S = \set{\seq{v}{1,,k}}\) is an orthonormal set in an \(n\)-dimensional inner product space \(\V\) over \(\F\).
  Then
  \begin{enumerate}
    \item \(S\) can be extended to an orthonormal basis \(\set{\seq{v}{1,,k,k+1,,n}}\) for \(\V\) over \(\F\).
    \item If \(\W = \spn{S}\), then \(S_1 = \set{\seq{v}{k+1,,n}}\) is an orthonormal basis for \(\W^{\perp}\) over \(\F\) (using the preceding notation).
    \item If \(\W\) is any subspace of \(\V\), then \(\dim(\V) = \dim(\W) + \dim(\W^{\perp})\).
  \end{enumerate}
\end{thm}

\begin{proof}[\pf{6.7}(a)]
  By \cref{1.6.15}(c), \(S\) can be extended to an ordered basis \(S' = \set{\seq{v}{1,,k}, \seq{w}{k+1,,n}}\) for \(\V\) over \(\F\).
  Now apply the Gram--Schmidt process (\cref{6.4}) to \(S'\).
  The first \(k\) vectors resulting from this process are the vectors in \(S\) by \cref{ex:6.2.8}, and this new set spans \(\V\).
  Normalizing the last \(n - k\) vectors of this set produces an orthonormal set that spans \(\V\).
  The result now follows.
\end{proof}

\begin{proof}[\pf{6.7}(b)]
  Because \(S_1\) is a subset of a basis, it is linearly independent.
  Since \(S_1\) is clearly a subset of \(\W^{\perp}\), we need only show that it spans \(\W^{\perp}\).
  Note that, for any \(x \in \V\), we have
  \[
    x = \sum_{i = 1}^n \inn{x, v_i} v_i.
  \]
  If \(x \in \W^{\perp}\), then \(\inn{x, v_i} = 0\) for \(i \in \set{1, \dots, k}\).
  Therefore,
  \[
    x = \sum_{i = k + 1}^n \inn{x, v_i} v_i \in \spn{S_1}.
  \]
\end{proof}

\begin{proof}[\pf{6.7}(c)]
  Let \(\W\) be a subspace of \(\V\) over \(\F\).
  It is a finite-dimensional inner product space because \(\V\) is, and so it has an orthonormal basis \(\set{\seq{v}{1,,k}}\).
  By \cref{6.7}(a)(b), we have
  \[
    \dim(\V) = n = k + (n - k) = \dim(\W) + \dim(\W^{\perp}).
  \]
\end{proof}

\exercisesection

\setcounter{ex}{5}
\begin{ex}\label{ex:6.2.6}
  Let \(\V\) be an inner product space over \(\F\), and let \(\W\) be a finite-dimensional subspace of \(\V\) over \(\F\).
  If \(x \notin \W\), prove that there exists \(y \in \V\) such that \(y \in \W^{\perp}\), but \(\inn{x, y} \neq 0\).
\end{ex}

\begin{proof}[\pf{ex:6.2.6}]
  Since \(x \in \V\), by \cref{6.6} there exist a \((u, v) \in \W \times \W^{\perp}\) such that \(x = u + v\).
  Then we have
  \begin{align*}
             & \begin{dcases}
                 x \notin \W \\
                 \zv \in \W
               \end{dcases} &  & \text{(by \cref{1.3}(c))} \\
    \implies & x \neq \zv                                  \\
    \implies & v \neq \zv     &  & (x \notin \W)
  \end{align*}
  and
  \begin{align*}
    \inn{x, v} & = \inn{u + v, v}                                           \\
               & = \inn{u, v} + \inn{v, v} &  & \text{(by \cref{6.1.1}(a))} \\
               & = \inn{v, v}              &  & \text{(by \cref{6.2.9})}    \\
               & > 0.                      &  & \text{(by \cref{6.1.1}(d))}
  \end{align*}
  By setting \(y = v\) we are done.
\end{proof}

\begin{ex}\label{ex:6.2.7}
  Let \(\beta\) be a basis for a subspace \(\W\) of an inner product space \(\V\) over \(\F\), and let \(z \in \V\).
  Prove that \(z \in \W^{\perp}\) iff \(\inn{z, v} = 0\) for every \(v \in \beta\).
\end{ex}

\begin{proof}[\pf{ex:6.2.7}]
  We have
  \begin{align*}
             & z \in \W^{\perp}                                                  \\
    \implies & \forall w \in \W, \inn{z, w} = 0    &  & \text{(by \cref{6.2.9})} \\
    \implies & \forall v \in \beta, \inn{z, v} = 0 &  & (\beta \subseteq \W)
  \end{align*}
  and
  \begin{align*}
             & \forall v \in \beta, \inn{z, v} = 0                                                                                         \\
    \implies & \forall w \in \W, \begin{dcases}
                                   \exists \seq{v}{1,,k} \in \beta \\
                                   \exists \seq{a}{1,,k} \in \F
                                 \end{dcases} :                                                                           \\
             & \inn{z, w} = \inn{z, \sum_{i = 1}^k a_i v_i} = \sum_{i = 1}^k \conj{a_i} \inn{z, v_i} = 0 &  & \text{(by \cref{6.1}(a)(b))} \\
    \implies & z \in \W^{\perp}.                                                                         &  & \text{(by \cref{6.2.9})}
  \end{align*}
\end{proof}

\begin{ex}\label{ex:6.2.8}
  Prove that if \(\set{\seq{w}{1,,n}}\) is an orthogonal set of nonzero vectors, then the vectors \(\seq{v}{1,,n}\) derived from the Gram--Schmidt process satisfy \(v_i = w_i\) for \(i \in \set{1, \dots, n}\).
\end{ex}

\begin{proof}[\pf{ex:6.2.8}]
  We use induction on \(n\).
  For \(n = 1\) we see by \cref{6.4} that \(v_1 = w_1\).
  Thus the base case holds.
  Suppose inductively that the statement is true for some \(n \geq 1\).
  We need to show that for \(n + 1\) the statement is still true.
  Let \(\set{\seq{w}{1,,n+1}}\) be an orthogonal set of nonzero vectors and let \(\set{\seq{v}{1,,n+1}}\) derived from the Gram--Schmidt process.
  By induction hypothesis we see that \(\set{\seq{v}{1,,n+1}} = \set{\seq{w}{1,,n}, v_{n + 1}}\).
  Thus we have
  \begin{align*}
    v_{n + 1} & = w_{n + 1} + \sum_{i = 1}^n \frac{\inn{w_{n + 1}, v_i}}{\norm{v_i}^2} v_i &  & \text{(by \cref{6.4})}           \\
              & = w_{n + 1} + \sum_{i = 1}^n \frac{\inn{w_{n + 1}, w_i}}{\norm{w_i}^2} w_i &  & \text{(by induction hypothesis)} \\
              & = w_{n + 1}                                                                &  & \text{(by \cref{6.1.12})}
  \end{align*}
  and this closes the induction.
\end{proof}

\begin{ex}\label{ex:6.2.16}

\end{ex}

\section{The Adjoint of a Linear Operator}\label{sec:6.3}

\begin{thm}\label{6.8}
  Let \(\V\) be a finite-dimensional inner product space over \(\F\), and let \(g \in \ls(\V, \F)\) be a linear transformation.
  Then there exists a unique vector \(y \in \V\) such that \(g(x) = \inn{x, y}\) for all \(x \in \V\).
\end{thm}

\begin{proof}[\pf{6.8}]
  Let \(\beta = \set{\seq{v}{1,,n}}\) be an orthonormal basis for \(\V\) over \(\F\), and let
  \[
    y = \sum_{i = 1}^n \conj{g(v_i)} v_i.
  \]
  Define \(h : \V \to \F\) by \(h(x) = \inn{x, y}\), which is clearly linear (by \cref{6.1.1}(a)(b)).
  Furthermore, for \(j \in \set{1, \dots, n}\) we have
  \begin{align*}
    h(v_j) & = \inn{v_j, y}                                                                  \\
           & = \inn{v_j, \sum_{i = 1}^n \conj{g(v_i)} v_i}                                   \\
           & = \sum_{i = 1}^n g(v_i) \inn{v_j, v_i}        &  & \text{(by \cref{6.1}(a)(b))} \\
           & = \sum_{i = 1}^n g(v_i) \delta_{j i}          &  & \text{(by \cref{6.1.12})}    \\
           & = g(v_j).
  \end{align*}
  Since \(g\) and \(h\) both agree on \(\beta\), we have that \(g = h\) by \cref{2.1.13}.
  To show that \(y\) is unique, suppose that \(g(x) = \inn{x, y'}\) for all \(x \in \V\).
  Then \(\inn{x, y} = \inn{x, y'}\) for all \(x \in \V\);
  so by \cref{6.1}(e), we have \(y = y'\).
\end{proof}

\begin{thm}\label{6.9}
  Let \(\V\) be a finite-dimensional inner product space over \(\F\), and let \(\T \in \ls(\V)\).
  Then there exists a unique function \(\T^* : \V \to \V\) such that \(\inn{\T(x), y} = \inn{x, \T^*(y)}\) for all \(x, y \in \V\).
  Furthermore, \(\T^*\) is linear.
  \(\T^*\) is called the \textbf{adjoint} of the operator \(\T\).
  The symbol \(\T^*\) is read ``\(\T\) star.''
\end{thm}

\begin{proof}[\pf{6.9}]
  Let \(y \in \V\).
  Define \(g : \V \to \F\) by \(g(x) = \inn{\T(x), y}\) for all \(x \in \V\).
  We first show that \(g\) is linear.
  Let \(x_1, x_2 \in \V\) and \(c \in \F\).
  Then
  \begin{align*}
    g(c x_1 + x_2) & = \inn{\T(c x_1 + x_2), y}                                                  \\
                   & = \inn{c \T(x_1) + \T(x_2), y}          &  & \text{(by \cref{2.1.2}(b))}    \\
                   & = c \inn{\T(x_1), y} + \inn{\T(x_2), y} &  & \text{(by \cref{6.1.1}(a)(b))} \\
                   & = c g(x_1) + g(x_2).
  \end{align*}
  Hence by \cref{2.1.2}(b) \(g \in \ls(\V, \F)\).

  We now apply \cref{6.8} to obtain a unique vector \(y' \in \V\) such that \(g(x) = \inn{x, y'}\);
  that is, \(\inn{\T(x), y} = \inn{x, y'}\) for all \(x \in \V\).
  Defining \(\T^* : \V \to \V\) by \(\T^*(y) = y'\), we have \(\inn{\T(x), y} = \inn{x, \T^*(y)}\).

  To show that \(\T^*\) is linear, let \(y_1, y_2 \in \V\) and \(c \in \F\).
  Then for any \(x \in \V\),
  we have
  \begin{align*}
    \inn{x, \T^*(c y_1 + y_2)} & = \inn{\T(x), c y_1 + y_2}                                                           \\
                               & = \conj{c} \inn{\T(x), y_1} + \inn{\T(x), y_2}     &  & \text{(by \cref{6.1}(a)(b))} \\
                               & = \conj{c} \inn{x, \T^*(y_1)} + \inn{x, \T^*(y_2)}                                   \\
                               & = \inn{x, c \T^*(y_1) + \T^*(y_2)}.                &  & \text{(by \cref{6.1}(a)(b))}
  \end{align*}
  Since \(x\) is arbitrary, \(\T^*(c y_1 + y_2) = c \T^*(y_1) + \T^*(y_2)\) by \cref{6.1}(e).

  Finally, we need to show that \(\T^*\) is unique. Suppose that \(\U \in \ls(\V)\) and that it satisfies \(\inn{\T(x), y} = \inn{x, \U(y)}\) for all \(x, y \in \V\).
  Then \(\inn{x, \T^*(y)} = \inn{x, \U(y)}\) for all \(x, y \in \V\), so \(\T^* = \U\).
\end{proof}

\begin{note}
  Thus by \cref{6.9} \(\T^*\) is the unique operator on \(\V\) satisfying \(\inn{\T(x), y} = \inn{x, \T^*(y)}\) for all \(x, y \in \V\).
  Note that we also have
  \begin{align*}
    \inn{x, \T(y)} & = \conj{\inn{\T(y), x}}   &  & \text{(by \cref{6.1.1}(c))} \\
                   & = \conj{\inn{y, \T^*(x)}} &  & \text{(by \cref{6.9})}      \\
                   & = \inn{\T^*(x), y};       &  & \text{(by \cref{6.1.1}(c))}
  \end{align*}
  so \(\inn{x, \T(y)} = \inn{\T^*(x), y}\) for all \(x, y \in \V\).
  We may view these equations symbolically as adding a \(*\) to \(\T\) when shifting its position inside the inner product symbol.
\end{note}

\begin{note}
  For an infinite-dimensional inner product space, the adjoint of a linear operator \(\T\) may be defined to be the function \(\T^*\) such that \(\inn{\T(x), y} = \inn{x, \T^*(y)}\) for all \(x, y \in \V\), provided it exists.
  Although the uniqueness and linearity of \(\T^*\) follow as before, the existence of the adjoint is not guaranteed (see \cref{ex:6.3.24}).
  The reader should observe the necessity of the hypothesis of finite-dimensionality in the proof of \cref{6.8}.
  Many of the theorems we prove about adjoints, nevertheless, do not depend on \(\V\) being finite-dimensional.
  \emph{Thus, unless stated otherwise, for the remainder of this chapter we adopt the convention that a reference to the adjoint of a linear operator on an infinite-dimensional inner product space assumes its existence}.
\end{note}

\begin{thm}\label{6.10}
  Let \(\V\) be a finite-dimensional inner product space over \(\F\), and let \(\beta\) be an orthonormal basis for \(\V\) over \(\F\).
  If \(\T \in \ls(\V)\), then
  \[
    [\T^*]_{\beta} = [\T]_{\beta}^*.
  \]
\end{thm}

\begin{proof}[\pf{6.10}]
  Let \(A = [\T]_{\beta}\), \(B = [\T^*]_{\beta}\), and \(\beta = \set{\seq{v}{1,,n}}\).
  Then we have
  \begin{align*}
    \forall i, j \in \set{1, \dots, n}, B_{i j} & = \inn{\T^*(v_j), v_i}        &  & \text{(by \cref{6.2.6})}    \\
                                                & = \conj{\inn{v_i, \T^*(v_j)}} &  & \text{(by \cref{6.1.1}(c))} \\
                                                & = \conj{\inn{\T(v_i), v_j}}   &  & \text{(by \cref{6.9})}      \\
                                                & = \conj{A_{j i}}              &  & \text{(by \cref{6.2.6})}    \\
                                                & = (A^*)_{i j}.                &  & \text{(by \cref{6.1.5})}
  \end{align*}
  Hence \(B = A^*\).
\end{proof}

\begin{cor}\label{6.3.1}
  Let \(A \in \ms{n}{n}{\F}\).
  Then \(\L_{A^*} = (\L_A)^*\).
\end{cor}

\begin{proof}[\pf{6.3.1}]
  If \(\beta\) is the standard ordered basis for \(\vs{F}^n\), then, by \cref{2.15}(a), we have \([\L_A]_{\beta} = A\).
  Hence \([(\L_A)^*]_{\beta} = [\L_A]_{\beta}^* = A^* = [\L_{A^*}]_{\beta}\), and so \((\L_A)^* = \L_{A^*}\).
\end{proof}

\begin{thm}\label{6.11}
  Let \(\V\) be an inner product space over \(\F\) and let \(\T, \U \in \ls(\V)\).
  Then
  \begin{enumerate}
    \item \((\T + \U)^* = \T^* + \U^*\);
    \item \((c \T)^* = \conj{c} \T^*\) for any \(c \in \F\);
    \item \((\T \U)^* = \U^* \T^*\);
    \item \(\T^{**} = \T\);
    \item \(\IT[\V]^* = \IT[\V]\).
  \end{enumerate}
\end{thm}

\begin{proof}[\pf{6.11}(a)]
  Since
  \begin{align*}
    \forall x, y \in \V, \inn{x, (\T + \U)^*(y)} & = \inn{(\T + \U)(x), y}               &  & \text{(by \cref{6.9})}      \\
                                                 & = \inn{\T(x) + \U(x), y}              &  & \text{(by \cref{2.2.5})}    \\
                                                 & = \inn{\T(x), y} + \inn{\U(x), y}     &  & \text{(by \cref{6.1.1}(a))} \\
                                                 & = \inn{x, \T^*(y)} + \inn{x, \U^*(y)} &  & \text{(by \cref{6.9})}      \\
                                                 & = \inn{x, \T^*(y) + \U^*(y)}          &  & \text{(by \cref{6.1}(a))}   \\
                                                 & = \inn{x, (\T^* + \U^*)(y)},          &  & \text{(by \cref{2.2.5})}
  \end{align*}
  by \cref{6.1}(e) we know that \(\T^* + \U^* = (\T + \U)^*\).
\end{proof}

\begin{proof}[\pf{6.11}(b)]
  Since
  \begin{align*}
    \forall x, y \in \V, \inn{x, (c \T)^*(y)} & = \inn{(c \T)(x), y}           &  & \text{(by \cref{6.9})}      \\
                                              & = \inn{c \T(x), y}             &  & \text{(by \cref{2.2.5})}    \\
                                              & = c \inn{\T(x), y}             &  & \text{(by \cref{6.1.1}(b))} \\
                                              & = c \inn{x, \T^*(y)}           &  & \text{(by \cref{6.9})}      \\
                                              & = \inn{x, \conj{c} \T^*(y)}    &  & \text{(by \cref{6.1}(b))}   \\
                                              & = \inn{x, (\conj{c} \T^*)(y)}, &  & \text{(by \cref{2.2.5})}
  \end{align*}
  by \cref{6.1}(e) we know that \((c \T)^* = \conj{c} \T^*\).
\end{proof}

\begin{proof}[\pf{6.11}(c)]
  Since
  \begin{align*}
    \forall x, y \in \V, \inn{x, (\T \U)^*(y)} & = \inn{(\T \U)(x), y}      &  & \text{(by \cref{6.9})} \\
                                               & = \inn{\T(\U(x)), y}                                   \\
                                               & = \inn{\U(x), \T^*(y)}     &  & \text{(by \cref{6.9})} \\
                                               & = \inn{x, \U^*(\T^*(y))}   &  & \text{(by \cref{6.9})} \\
                                               & = \inn{x, (\U^* \T^*)(y)},
  \end{align*}
  by \cref{6.1}(e) we know that \((\T \U)^* = \U^* \T^*\).
\end{proof}

\begin{proof}[\pf{6.11}(d)]
  Since
  \begin{align*}
    \forall x, y \in \V, \inn{x, \T(y)} & = \inn{\T^*(x), y}     &  & \text{(by \cref{6.9})} \\
                                        & = \inn{x, \T^{**}(y)}, &  & \text{(by \cref{6.9})}
  \end{align*}
  by \cref{6.1}(e) we know that \(\T^{**} = \T\).
\end{proof}

\begin{proof}[\pf{6.11}(e)]
  Since
  \begin{align*}
    \forall x, y \in \V, \inn{x, \IT[\V](y)} & = \inn{x, y}             &  & \text{(by \cref{2.1.9})} \\
                                             & = \inn{\IT[\V](x), y}    &  & \text{(by \cref{2.1.9})} \\
                                             & = \inn{x, \IT[\V]^*(y)}, &  & \text{(by \cref{6.9})}
  \end{align*}
  by \cref{6.1}(e) we know that \(\IT[\V]^* = \IT[\V]\).
\end{proof}

\begin{note}
  The same proof works for \cref{6.11} in the infinite-dimensional case, provided that the existence of \(\T^*\) and \(\U^*\) is assumed.
\end{note}

\begin{cor}\label{6.3.2}
  Let \(A, B \in \ms{n}{n}{\F}\).
  Then
  \begin{enumerate}
    \item \((A + B)^* = A^* + B^*\);
    \item \((cA)^* = \conj{c} A^*\) for all \(c \in \F\).
    \item \((AB)^* = B^* A^*\);
    \item \(A^{**} = A\);
    \item \(I_n^* = I_n\).
  \end{enumerate}
\end{cor}

\begin{proof}[\pf{6.3.2}(a)]
  Since
  \begin{align*}
    \L_{(A + B)^*} & = (\L_{A + B})^*      &  & \text{(by \cref{6.3.1})}   \\
                   & = (\L_A + \L_B)^*     &  & \text{(by \cref{2.15}(c))} \\
                   & = (\L_A)^* + (\L_B)^* &  & \text{(by \cref{6.11}(a))} \\
                   & = \L_{A^*} + \L_{B^*} &  & \text{(by \cref{6.3.1})}   \\
                   & = \L_{A^* + B^*},     &  & \text{(by \cref{2.15}(c))}
  \end{align*}
  by \cref{2.15}(b) we have \((A + B)^* = A^* + B^*\).
\end{proof}

\begin{proof}[\pf{6.3.2}(b)]
  Since
  \begin{align*}
    \L_{(cA)^*} & = (\L_{cA})^*        &  & \text{(by \cref{6.3.1})}   \\
                & = (c \L_A)^*         &  & \text{(by \cref{2.15}(c))} \\
                & = \conj{c} (\L_A)^*  &  & \text{(by \cref{6.11}(b))} \\
                & = \conj{c} \L_{A^*}  &  & \text{(by \cref{6.3.1})}   \\
                & = \L_{\conj{c} A^*}, &  & \text{(by \cref{2.15}(c))}
  \end{align*}
  by \cref{2.15}(b) we have \((cA)^* = \conj{c} A^*\).
\end{proof}

\begin{proof}[\pf{6.3.2}(c)]
  Since
  \begin{align*}
    \L_{(AB)^*} & = (\L_{AB})^*       &  & \text{(by \cref{6.3.1})}   \\
                & = (\L_A \L_B)^*     &  & \text{(by \cref{2.15}(e))} \\
                & = (\L_B)^* (\L_A)^* &  & \text{(by \cref{6.11}(c))} \\
                & = \L_{B^*} \L_{A^*} &  & \text{(by \cref{6.3.1})}   \\
                & = \L_{B^* A^*},     &  & \text{(by \cref{2.15}(e))}
  \end{align*}
  by \cref{2.15}(b) we have \((AB)^* = B^* A^*\).
\end{proof}

\begin{proof}[\pf{6.3.2}(d)]
  Since
  \begin{align*}
    \L_A & = (\L_A)^{**}  &  & \text{(by \cref{6.11}(d))} \\
         & = (\L_{A^*})^* &  & \text{(by \cref{6.3.1})}   \\
         & = \L_{A^{**}}, &  & \text{(by \cref{6.3.1})}
  \end{align*}
  by \cref{2.15}(b) we have \(A = A^{**}\).
\end{proof}

\begin{proof}[\pf{6.3.2}(e)]
  Since
  \begin{align*}
    \L_{I_n} & = \IT[\vs{F}^n]   &  & \text{(by \cref{2.15}(f))} \\
             & = \IT[\vs{F}^n]^* &  & \text{(by \cref{6.11}(e))} \\
             & = (\L_{I_n})^*    &  & \text{(by \cref{2.15}(f))} \\
             & = \L_{I_n^*},     &  & \text{(by \cref{6.3.1})}
  \end{align*}
  by \cref{2.15}(b) we have \(I_n = I_n^*\).
\end{proof}

\exercisesection

\begin{ex}\label{ex:6.3.24}

\end{ex}

\section{Normal and Self-Adjoint Operators}\label{sec:6.4}

\begin{lem}\label{6.4.1}
  Let \(\T\) be a linear operator on a finite-dimensional inner product space \(\V\) over \(\F\).
  If \(\T\) has an eigenvector, then so does \(\T^*\).
\end{lem}

\begin{proof}[\pf{6.4.1}]
  Suppose that \(v\) is an eigenvector of \(\T\) with corresponding eigenvalue \(\lambda\).
  Then for any \(x \in \V\),
  \begin{align*}
    0 & = \inn{\zv, x}                                 &  & \text{(by \cref{6.1}(c))}        \\
      & = \inn{(\T - \lambda \IT[\V])(v), x}           &  & \text{(by \cref{5.2.4})}         \\
      & = \inn{v, (\T - \lambda \IT[\V])^*(x)}         &  & \text{(by \cref{6.9})}           \\
      & = \inn{v, (\T^* - \conj{\lambda} \IT[\V])(x)}, &  & \text{(by \cref{6.11}(a)(b)(e))}
  \end{align*}
  and hence \(v\) is orthogonal to the range of \(\T^* - \conj{\lambda} \IT[\V]\).
  So \(\T^* - \conj{\lambda} \IT[\V]\) is not onto (\(v \notin \rg{\T^* - \conj{\lambda} \IT[\V]}\)) and hence is not one-to-one (by \cref{2.5}).
  Thus \(\T^* - \conj{\lambda} \IT[\V]\) has a nonzero null space, and any nonzero vector in this null space is an eigenvector of \(\T^*\) with corresponding eigenvalue \(\conj{\lambda}\).
\end{proof}

\begin{thm}[Schur's theorem]\label{6.14}
  Let \(\T\) be a linear operator on a finite-dimensional inner product space \(\V\) over \(\F\).
  Suppose that the characteristic polynomial of \(\T\) splits.
  Then there exists an orthonormal basis \(\beta\) for \(\V\) over \(\F\) such that the matrix \([\T]_{\beta}\) is upper triangular.
\end{thm}

\begin{proof}[\pf{6.14}]
  The proof is by mathematical induction on the dimension \(n\) of \(\V\).
  The result is immediate if \(n = 1\).
  So suppose that the result is true for linear operators on \(n\)-dimensional inner product spaces whose characteristic polynomials split.
  By \cref{6.4.1}, we can assume that \(\T^*\) has a unit eigenvector \(z\).
  Suppose that \(\T^*(z) = \lambda z\) and that \(\W = \spn{\set{z}}\).
  We show that \(\W^{\perp}\) is \(\T\)-invariant.
  If \(y \in \W^{\perp}\) and \(x = cz \in \W\), then
  \begin{align*}
    \inn{\T(y), x} & = \inn{\T(y), cz}                                              \\
                   & = \inn{y, \T^*(cz)}           &  & \text{(by \cref{6.9})}      \\
                   & = \inn{y, c \T^*(z)}          &  & \text{(by \cref{2.1.1}(b))} \\
                   & = \inn{y, c \lambda z}        &  & \text{(by \cref{5.1.2})}    \\
                   & = \conj{c \lambda} \inn{y, z} &  & \text{(by \cref{6.1}(b))}   \\
                   & = \conj{c \lambda} (0)        &  & \text{(by \cref{6.2.9})}    \\
                   & = 0.
  \end{align*}
  So \(\T(y) \in \W^{\perp}\).
  It is easy to show (see \cref{5.21}, or as a consequence of \cref{ex:4.4.6}) that the characteristic polynomial of \(\T_{\W^{\perp}}\) divides the characteristic polynomial of \(\T\) and hence splits.
  By \cref{6.7}(c), \(\dim(\W^{\perp}) = n\), so we may apply the induction hypothesis to \(\T_{\W^{\perp}}\) and obtain an orthonormal basis \(\gamma\) for \(\W^{\perp}\) over \(\F\) such that \([\T_{\W^{\perp}}]_{\gamma}\) is upper triangular.
  Clearly, \(\beta = \gamma \cup \set{z}\) is an orthonormal basis for \(\V\) over \(\F\) (by \cref{1.6.15,6.2.4}) such that \([\T]_{\beta}\) is upper triangular.
\end{proof}

\begin{cor}\label{6.4.2}
  Let \(\V\) be a finite-dimensional inner product space over \(\F\) and let \(\T \in \ls(\V)\).
  Suppose that there exists an orthonormal basis of eigenvectors of \(\T\).
  Then \(\T \T^* = \T^* \T\).
\end{cor}

\begin{proof}[\pf{6.4.2}]
  If such an orthonormal basis \(\beta\) exists, then \([\T]_{\beta}\) is a diagonal matrix, and hence \([\T^*]_{\beta} = [\T]_{\beta}^*\) is also a diagonal matrix.
  Because diagonal matrices commute, we conclude that \(\T\) and \(\T^*\) commute.
\end{proof}

\begin{defn}\label{6.4.3}
  Let \(\V\) be an inner product space over \(\F\), and let \(\T \in \ls(\V)\).
  We say that \(\T\) is \textbf{normal} if \(\T \T^* = \T^* \T\).
  An \(n \times n\) real or complex matrix \(A\) is \textbf{normal} if \(A A^* = A^* A\).
\end{defn}

\begin{cor}\label{6.4.4}
  Let \(\V\) be an inner product space over \(\F\), and let \(\T \in \ls(\V)\).
  \(\T\) is normal iff \([\T]_{\beta}\) is normal, where \(\beta\) is an orthonormal basis.
\end{cor}

\begin{proof}[\pf{6.4.4}]
  We have
  \begin{align*}
         & \T \text{ is normal}                                                                    \\
    \iff & \T \T^* = \T^* \T                                         &  & \text{(by \cref{6.4.3})} \\
    \iff & [\T]_{\beta} [\T]_{\beta}^* = [\T]_{\beta}^* [\T]_{\beta} &  & \text{(by \cref{6.10})}  \\
    \iff & [\T]_{\beta} \text{ is normal}.                           &  & \text{(by \cref{6.4.3})}
  \end{align*}
\end{proof}

\begin{eg}\label{6.4.5}
  Let \(\T : \R^2 \to \R^2\) be rotation by \(\theta\), where \(0 < \theta < \pi\).
  The matrix representation of \(\T\) in the standard ordered basis is given by
  \[
    A = \begin{pmatrix}
      \cos\theta & -\sin\theta \\
      \sin\theta & \cos\theta
    \end{pmatrix}.
  \]
  Note that \(A A^* = I_2 = A^* A\);
  so \(A\), and hence \(\T\), is normal.
\end{eg}

\begin{eg}\label{6.4.6}
  Suppose that \(A\) is a real skew-symmetric matrix;
  that is, \(\tp{A} = -A\).
  Then \(A\) is normal because both \(A \tp{A}\) and \(\tp{A} A\) are equal to \(-A^2\).
\end{eg}

\begin{note}
  Clearly, the operator \(\T\) in \cref{6.4.5} does not even possess one eigenvector.
  So in the case of a real inner product space, we see that normality is not sufficient to guarantee an orthonormal basis of eigenvectors.
  All is not lost, however.
  We show that normality suffices if \(\V\) is a complex inner product space.
\end{note}

\begin{thm}\label{6.15}
  Let \(\V\) be an inner product space over \(\F\), and let \(\T\) be a normal operator on \(\V\).
  Then the following statements are true.
  \begin{enumerate}
    \item \(\norm{\T(x)} = \norm{\T^*(x)}\) for all \(x \in \V\).
    \item \(\T - c \IT[\V]\) is normal for every \(c \in \F\).
    \item If \(x\) s an eigenvector of \(\T\), then \(x\) is also an eigenvector of \(\T^*\).
          In fact, if \(\T(x) = \lambda x\), then \(\T^*(x) = \conj{\lambda} x\).
    \item If \(\lambda_1\) and \(\lambda_2\) are distinct eigenvalues of \(\T\) with corresponding eigenvectors \(x_1\) and \(x_2\), then \(x_1\) and \(x_2\) are orthogonal.
  \end{enumerate}
\end{thm}

\begin{proof}[\pf{6.15}(a)]
  For any \(x \in \V\), we have
  \begin{align*}
    \norm{\T(x)}^2 & = \inn{\T(x), \T(x)}     &  & \text{(by \cref{6.1.9})} \\
                   & = \inn{\T^* \T(x), x}    &  & \text{(by \cref{6.9})}   \\
                   & = \inn{\T \T^*(x), x}    &  & \text{(by \cref{6.4.3})} \\
                   & = \inn{\T^*(x), \T^*(x)} &  & \text{(by \cref{6.9})}   \\
                   & = \norm{\T^*(x)}^2.      &  & \text{(by \cref{6.1.9})}
  \end{align*}
\end{proof}

\begin{proof}[\pf{6.15}(b)]
  We have
  \begin{align*}
    (\T - c \IT[\V]) (\T - c \IT[\V])^* & = (\T - c \IT[\V]) (\T^* - \conj{c} \IT[\V])                                  &  & \text{(by \cref{6.11}(a)(b)(e))} \\
                                        & = \T \T^* - \conj{c} \T \IT[\V] - c \IT[\V] \T^* + c \conj{c} \IT[\V] \IT[\V] &  & \text{(by \cref{2.10})}          \\
                                        & = \T^* \T - \conj{c} \IT[\V] \T - c \T^* \IT[\V] + c \conj{c} \IT[\V] \IT[\V] &  & \text{(by \cref{6.4.3})}         \\
                                        & = (\T^* - \conj{c} \IT[\V]) (\T - c \IT[\V])                                  &  & \text{(by \cref{2.10})}          \\
                                        & = (\T - c \IT[\V])^* (\T - c \IT[\V])                                         &  & \text{(by \cref{6.11}(a)(b)(e))}
  \end{align*}
  and thus by \cref{6.4.3} \(\T - c \IT[\V]\) is normal for all \(c \in \F\).
\end{proof}

\begin{proof}[\pf{6.15}(c)]
  Suppose that \(\T(x) = \lambda x\) for some \(x \in \V\).
  Let \(\U = \T - \lambda \IT[\V]\).
  Then \(\U(x) = \zv\), and \(\U\) is normal by (b).
  Thus (a) implies that
  \begin{align*}
    0 & = \norm{\U(x)}                              &  & \text{(by \cref{6.2}(b))}        \\
      & = \norm{\U^*(x)}                            &  & \text{(by \cref{6.15}(a))}       \\
      & = \norm{(\T^* - \conj{\lambda} \IT[\V])(x)} &  & \text{(by \cref{6.11}(a)(b)(e))} \\
      & = \norm{\T^*(x) - \conj{\lambda} x}.        &  & \text{(by \cref{2.2.5})}
  \end{align*}
  Hence by \cref{6.2}(b) \(\T^*(x) = \conj{\lambda} x\).
  So \(x\) is an eigenvector of \(\T^*\).
\end{proof}

\begin{proof}[\pf{6.15}(d)]
  Let \(\lambda_1\) and \(\lambda_2\) be distinct eigenvalues of \(\T\) with corresponding eigenvectors \(x_1\) and \(x_2\).
  Then, using (c), we have
  \begin{align*}
    \lambda_1 \inn{x_1, x_2} & = \inn{\lambda_1 x_1, x_2}        &  & \text{(by \cref{6.1.1}(b))} \\
                             & = \inn{\T(x_1), x_2}              &  & \text{(by \cref{5.1.2})}    \\
                             & = \inn{x_1, \T^*(x_2)}            &  & \text{(by \cref{6.9})}      \\
                             & = \inn{x_1, \conj{\lambda_2} x_2} &  & \text{(by \cref{6.15}(c))}  \\
                             & = \lambda_2 \inn{x_1, x_2}.       &  & \text{(by \cref{6.1}(b))}
  \end{align*}
  Since \(\lambda_1 \neq \lambda_2\), we conclude that \(\inn{x_1, x_2} = 0\).
\end{proof}

\begin{thm}\label{6.16}
  Let \(\T\) be a linear operator on a finite-dimensional complex inner product space \(\V\).
  Then \(\T\) is normal iff there exists an orthonormal basis for \(\V\) consisting of eigenvectors of \(\T\).
\end{thm}

\begin{proof}[\pf{6.16}]
  Suppose that \(\T\) is normal.
  By the fundamental theorem of algebra (\cref{d.4}), the characteristic polynomial of \(\T\) splits.
  So we may apply Schur's theorem (\cref{6.14}) to obtain an orthonormal basis \(\beta = \set{\seq{v}{1,,n}}\) for \(\V\) over \(\F\) such that \([\T]_{\beta} = A\) is upper triangular.
  We know that \(v_1\) is an eigenvector of \(\T\) because \(A\) is upper triangular.
  Assume that \(\seq{v}{1,,k-1}\) are eigenvectors of \(\T\).
  We claim that \(v_k\) is also an eigenvector of \(\T\).
  It then follows by mathematical induction on \(k\) that all of the \(v_i\)'s are eigenvectors of \(\T\).
  Consider any \(j < k\), and let \(\lambda_j\) denote the eigenvalue of \(\T\) corresponding to \(v_j\).
  By \cref{6.15}(c), \(\T^*(v_j) = \conj{\lambda_j} v_j\).
  Since \(A\) is upper triangular, by \cref{2.2.4} we have
  \[
    \T(v_k) = A_{1 k} v_1 + \cdots + A_{j k} v_j + \cdots + A_{k k} v_k.
  \]
  Furthermore,
  \begin{align*}
    A_{j k} & = \inn{\T(v_k), v_j}              &  & \text{(by \cref{6.2.6})}   \\
            & = \inn{v_k, \T^*(v_j)}            &  & \text{(by \cref{6.9})}     \\
            & = \inn{v_k, \conj{\lambda_j} v_j} &  & \text{(by \cref{6.15}(c))} \\
            & = \lambda_j \inn{v_k, v_j}        &  & \text{(by \cref{6.1}(b))}  \\
            & = 0.                              &  & \text{(by \cref{6.1.12})}
  \end{align*}
  It follows that \(\T(v_k) = A_{k k} v_k\), and hence \(v_k\) is an eigenvector of \(\T\).
  So by induction, all the vectors in \(\beta\) are eigenvectors of \(\T\).

  The converse was already proved by \cref{6.4.2}.
\end{proof}

\begin{note}
  Interestingly, as \cref{6.4.7} shows, \cref{6.16} does not extend to infinite-dimensional complex inner product spaces.
\end{note}

\begin{eg}\label{6.4.7}
  Consider the inner product space \(\vs{H}\) with the orthonormal set \(S\) from \cref{6.1.13}.
  Let \(\V = \spn{S}\), and let \(\T, \U \in \ls(\V)\) defined by \(\T(f) = f_1 f\) and \(\U(f) = f_{-1} f\).
  Then
  \[
    \T(f_n) = f_{n + 1} \quad \text{and} \quad \U(f_n) = f_{n - 1}
  \]
  for all integers \(n\).
  Thus
  \begin{align*}
    \inn{\T(f_m), f_n} & = \inn{f_{m + 1}, f_n}                                \\
                       & = \delta_{(m + 1), n}  &  & \text{(by \cref{6.1.13})} \\
                       & = \delta_{m, (n - 1)}                                 \\
                       & = \inn{f_m, f_{n - 1}} &  & \text{(by \cref{6.1.13})} \\
                       & = \inn{f_m, \U(f_n)}.
  \end{align*}
  It follows that \(\U = \T^*\).
  Furthermore, \(\T \T^* = \IT[\V] = \T^* \T\);
  so \(\T\) is normal.

  We show that \(\T\) has no eigenvectors.
  Suppose that \(f\) is an eigenvector of \(\T\), say, \(\T(f) = \lambda f\) for some \(\lambda\).
  Since \(\V\) equals the span of \(S\), we may write
  \[
    f = \sum_{i = n}^m a_i f_i, \quad \text{where } a_m \neq 0.
  \]
  Hence
  \[
    \sum_{i = n}^m a_i f_{i + 1} = \T(f) = \lambda f = \sum_{i = n}^m \lambda a_i f_i.
  \]
  Since \(a_m \neq 0\), we can write \(f_{m + 1}\) as a linear combination of \(\seq{f}{n,n+1,,m}\).
  But this is a contradiction because \(S\) is linearly independent.
\end{eg}

\begin{note}
  \cref{6.4.5} illustrates that normality is not sufficient to guarantee the existence of an orthonormal basis of eigenvectors for real inner product spaces.
  For real inner product spaces, we must replace normality by the stronger condition that \(\T = \T^*\) in order to guarantee such a basis.
\end{note}

\begin{defn}\label{6.4.8}
  Let \(\T\) be a linear operator on an inner product space \(\V\) over \(\F\).
  We say that \(\T\) is \textbf{self-adjoint} (\textbf{Hermitian}) if \(\T = \T^*\).
  An \(n \times n\) real or complex matrix \(A\) is \textbf{self-adjoint} (\textbf{Hermitian}) if \(A = A^*\).
\end{defn}

\begin{cor}\label{6.4.9}
  If \(\beta\) is an orthonormal basis, then \(\T\) is self-adjoint iff \([\T]_{\beta}\) is self-adjoint.
  For real matrices, this condition reduces to the requirement that \(A\) be symmetric.
\end{cor}

\begin{proof}[\pf{6.4.9}]
  We have
  \begin{align*}
         & \T \text{ is self-adjoint}                                                   \\
    \iff & \T = \T^*                                      &  & \text{(by \cref{6.4.8})} \\
    \iff & [\T]_{\beta} = [\T^*]_{\beta} = [\T]_{\beta}^* &  & \text{(by \cref{6.10})}  \\
    \iff & [\T]_{\beta} \text{ is self-adjoint}.          &  & \text{(by \cref{6.4.8})}
  \end{align*}
\end{proof}

\begin{lem}\label{6.4.10}
  Let \(\T\) be a self-adjoint operator on a finite-dimensional inner product space \(\V\) over \(\F\).
  Then
  \begin{enumerate}
    \item Every eigenvalue of \(\T\) is real.
    \item Suppose that \(\V\) is a real inner product space.
          Then the characteristic polynomial of \(\T\) splits.
  \end{enumerate}
\end{lem}

\begin{proof}[\pf{6.4.10}(a)]
  Suppose that \(\T(x) = \lambda x\) for \(x \neq \zv\).
  Because a self-adjoint operator is also normal, we have
  \begin{align*}
    \lambda x & = \T(x)             &  & \text{(by \cref{5.1.2})}   \\
              & = \T^*(x)           &  & \text{(by \cref{6.4.8})}   \\
              & = \conj{\lambda} x. &  & \text{(by \cref{6.15}(c))}
  \end{align*}
  So \(\lambda = \conj{\lambda}\);
  that is, \(\lambda\) is real.
\end{proof}

\begin{proof}[\pf{6.4.10}(b)]
  Let \(n = \dim(\V)\), \(\beta\) be an orthonormal basis for \(\V\) over \(\R\), and \(A = [\T]_{\beta}\).
  Then \(A\) is self-adjoint by \cref{6.4.9}.
  Let \(\T_A\) be the linear operator on \(\C^n\) defined by \(\T_A(x) = Ax\) for all \(x \in \C^n\).
  Note that \(\T_A\) is self-adjoint because \([\T_A]_{\gamma} = A\), where \(\gamma\) is the standard ordered (orthonormal) basis for \(\C^n\) over \(\C\) (see \cref{2.15}(a)).
  So, by (a), the eigenvalues of \(\T_A\) are real.
  By the fundamental theorem of algebra (\cref{d.4}), the characteristic polynomial of \(\T_A\) splits into factors of the form \(t - \lambda\).
  Since each \(\lambda\) is real, the characteristic polynomial splits over \(\R\).
  But \(\T_A\) has the same characteristic polynomial as \(A\), which has the same characteristic polynomial as \(\T\) (see \cref{5.1.6}).
  Therefore the characteristic polynomial of \(\T\) splits.
\end{proof}

\begin{thm}\label{6.17}
  Let \(\T\) be a linear operator on a finite-dimensional real inner product space \(\V\).
  Then \(\T\) is self-adjoint iff there exists an orthonormal basis \(\beta\) for \(\V\) over \(\R\) consisting of eigenvectors of \(\T\).
\end{thm}

\begin{proof}[\pf{6.17}]
  Suppose that \(\T\) is self-adjoint.
  By \cref{6.4.10}(b), we may apply Schur's theorem (\cref{6.14}) to obtain an orthonormal basis \(\beta\) for \(\V\) over \(\R\) such that the matrix \(A = [\T]_{\beta}\) is upper triangular.
  But
  \begin{align*}
    A^* & = [\T]_{\beta}^*                               \\
        & = [\T^*]_{\beta} &  & \text{(by \cref{6.10})}  \\
        & = [\T]_{\beta}   &  & \text{(by \cref{6.4.8})} \\
        & = A.
  \end{align*}
  So \(A\) and \(A^*\) are both upper triangular, and therefore \(A\) is a diagonal matrix.
  Thus \(\beta\) must consist of eigenvectors of \(\T\).

  Now suppose that there exists an orthonormal basis \(\beta\) for \(\V\) over \(\R\) consisting of eigenvectors of \(\T\).
  By \cref{5.1.1} we know that \([\T]_{\beta}\) is a diagonal matrix.
  Since \(\V\) is over \(\R\), we have
  \begin{align*}
    [\T^*]_{\beta} & = [\T]_{\beta}^*      &  & \text{(by \cref{6.10})}                        \\
                   & = \tp{([\T]_{\beta})} &  & \text{(\(\V\) is over \(\R\))}                 \\
                   & = [\T]_{\beta}.       &  & \text{(\([\T]_{\beta}\) is a diagonal matrix)}
  \end{align*}
  Thus by \cref{6.4.8} \([\T]_{\beta}\) is self-adjoint and by \cref{6.4.9} \(\T\) is self-adjoint.
\end{proof}

\begin{note}
  \cref{6.17} is used extensively in many areas of mathematics and statistics.
  We restate this theorem in matrix form in \cref{sec:6.5}.
\end{note}

\section{Unitary and Orthogonal Operators and Their Matrices}\label{sec:6.5}

\begin{note}
  In \cref{sec:6.5}, we study those linear operators \(\T\) on an inner product space \(\V\) over \(\F\) such that \(\T \T^* = \T^* \T = \IT[\V]\).
  We will see that these are precisely the linear operators that ``preserve length'' in the sense that \(\norm{\T(x)} = \norm{x}\) for all \(x \in \V\).
  As another characterization, we prove that, on a finite-dimensional complex inner product space, these are the normal operators whose eigenvalues all have absolute value \(1\).

  In past chapters, we were interested in studying those functions that preserve the structure of the underlying space.
  In particular, linear operators preserve the operations of vector addition and scalar multiplication, and isomorphisms preserve all the vector space structure.
  It is now natural to consider those linear operators \(\T\) on an inner product space that preserve length.
  We will see that this condition guarantees, in fact, that \(\T\) preserves the inner product.
\end{note}

\begin{defn}\label{6.5.1}
  Let \(\T\) be a linear operator on a finite-dimensional inner product space \(\V\) over \(\F\).
  If \(\norm{\T(x)} = \norm{x}\) for all \(x \in \V\), we call \(\T\) a \textbf{unitary operator} if \(\F = \C\) and an \textbf{orthogonal operator} if \(\F = \R\).

  in the infinite-dimensional case, an operator satisfying the preceding norm requirement is generally called an \textbf{isometry}.
  If, in addition, the operator is onto (the condition guarantees one-to-one, see \cref{ex:6.1.17}), then the operator is called a \textbf{unitary} or \textbf{orthogonal operator}.
\end{defn}

\begin{note}
  Clearly, any rotation or reflection in \(\R^2\) preserves length and hence is an orthogonal operator.
  We study these operators in much more detail in \cref{sec:6.11}.
\end{note}

\begin{eg}\label{6.5.2}
  Let \(h \in \vs{H}\) (see \cref{6.1.8}) satisfy \(\abs{h(x)} = 1\) for all \(x \in [0, 2 \pi]\).
  Define \(\T \in \ls(\vs{H})\) by \(\T(f) = hf\).
  Then
  \begin{align*}
    \norm{\T(f)}^2 & = \norm{hf}^2                                                                     \\
                   & = \inn{hf, hf}                                                    &  & \by{6.1.9} \\
                   & = \frac{1}{2 \pi} \int_0^{2 \pi} h(t) f(t) \conj{h(t) f(t)} \; dt &  & \by{6.1.8} \\
                   & = \inn{f, f}                                                      &  & \by{6.1.8} \\
                   & = \norm{f}^2                                                      &  & \by{6.1.9}
  \end{align*}
  since \(\abs{h(t)}^2 = 1\) for all \(t \in [0, 2 \pi]\).
  So \(\T\) is a unitary operator.
\end{eg}

\begin{lem}\label{6.5.3}
  Let \(\U\) be a self-adjoint operator on a finite-dimensional inner product space \(\V\) over \(\F\).
  If \(\inn{x, \U(x)} = 0\) for all \(x \in \V\), then \(\U = \zT\).
\end{lem}

\begin{proof}[\pf{6.5.3}]
  By either \cref{6.16} or \cref{6.17}, we may choose an orthonormal basis \(\beta\) for \(\V\) over \(\F\) consisting of eigenvectors of \(\U\).
  If \(x \in \beta\), then \(\U(x) = \lambda x\) for some \(\lambda\).
  Thus
  \begin{align*}
    0 & = \inn{x, \U(x)}                              \\
      & = \inn{x, \lambda x}         &  & \by{5.1.2}  \\
      & = \conj{\lambda} \inn{x, x}; &  & \by{6.1}[b]
  \end{align*}
  so \(\lambda = 0\).
  Hence \(\U(x) = \zv\) for all \(x \in \beta\), and thus \(\U = \zT\).
\end{proof}

\begin{note}
  Compare \cref{6.5.3} to \cref{ex:6.4.11}(b).
\end{note}

\begin{thm}\label{6.18}
  Let \(\T\) be a linear operator on a finite-dimensional inner product space \(\V\) over \(\F\).
  Then the following statements are equivalent.
  \begin{enumerate}
    \item \(\T \T^* = \T^* \T = \IT[\V]\).
    \item \(\inn{\T(x), \T(y)} = \inn{x, y}\) for all \(x, y \in \V\).
    \item If \(\beta\) is an orthonormal basis for \(\V\) over \(\F\), then \(\T(\beta)\) is an orthonormal basis for \(\V\) over \(\F\).
    \item There exists an orthonormal basis \(\beta\) for \(\V\) over \(\F\) such that \(\T(\beta)\) is an orthonormal basis for \(\V\) over \(\F\).
    \item \(\norm{\T(x)} = \norm{x}\) for all \(x \in \V\).
  \end{enumerate}
  Thus all the conditions above are equivalent to the definition of a unitary or orthogonal operator.
  From (a), it follows that unitary or orthogonal operators are normal.
\end{thm}

\begin{proof}[\pf{6.18}]
  We prove first that (a) implies (b).
  Let \(x, y \in \V\).
  Then \(\inn{x, y} = \inn{\T^* \T(x), y} = \inn{\T(x), \T(y)}\).

  Second, we prove that (b) implies (c).
  Let \(\beta = \set{\seq{v}{1,,n}}\) be an orthonormal basis for \(\V\) over \(\F\);
  so \(\T(\beta) = \set{\T(v_1), \dots, \T(v_n)}\).
  It follows that \(\inn{\T(v_i), \T(v_j)} = \inn{v_i, v_j} = \delta_{i j}\).
  Therefore \(\T(\beta)\) is an orthonormal basis for \(\V\) over \(\F\).

  That (c) implies (d) is obvious.

  Next we prove that (d) implies (e).
  Let \(x \in \V\), and let \(\beta = \set{\seq{v}{1,,n}}\).
  Now
  \[
    x = \sum_{i = 1}^n a_i v_i
  \]
  for some scalars \(a_i \in \F\), and so
  \begin{align*}
    \norm{x}^2 & = \inn{\sum_{i = 1}^n a_i v_i, \sum_{j = 1}^n a_j v_j}         &  & \by{6.1.9}      \\
               & = \sum_{i = 1}^n  a_i \inn{v_i, \sum_{j = 1}^n \conj{a_j} v_j} &  & \by{6.1.1}[a,b] \\
               & = \sum_{i = 1}^n \sum_{j = 1}^n a_i \conj{a_j} \inn{v_i, v_j}  &  & \by{6.1}[a,b]   \\
               & = \sum_{i = 1}^n \sum_{j = 1}^n a_i \conj{a_j} \delta_{i j}    &  & \by{6.1.12}     \\
               & = \sum_{i = 1}^n a_i \conj{a_i}                                &  & \by{2.3.4}      \\
               & = \sum_{i = 1}^n \abs{a_i}^2                                   &  & \by{d.0.5}
  \end{align*}
  since \(\beta\) is orthonormal.

  Applying the same manipulations to
  \[
    \T(x) = \sum_{i = 1}^n a_i \T(v_i)
  \]
  and using the fact that \(\T(\beta)\) is also orthonormal, we obtain
  \[
    \norm{\T(x)}^2 = \sum_{i = 1}^n \abs{a_i}^2.
  \]
  Hence \(\norm{\T(x)} = \norm{x}\).

  Finally, we prove that (e) implies (a).
  For any \(x \in \V\), we have
  \begin{align*}
    \inn{x, x} & = \norm{x}^2           &  & \by{6.1.9} \\
               & = \norm{\T(x)}^2                       \\
               & = \inn{\T(x), \T(x)}   &  & \by{6.1.9} \\
               & = \inn{x, \T^* \T(x)}. &  & \by{6.9}
  \end{align*}
  So \(\inn{x, (\IT[\V] - \T^* \T)(x)} = 0\) for all \(x \in \V\).
  Let \(\U = \IT[\V] - \T^* \T\);
  then \(\U\) is self-adjoint (by \cref{6.11}) and \(\inn{x, \U(x)} = 0\) for all \(x \in \V\).
  Hence, by \cref{6.5.3}, we have \(\zT = \U = \IT[\V] - \T^* \T\), and therefore \(\T^* \T = \IT[\V]\).
  Since \(\V\) is finite-dimensional, we may use \cref{ex:2.4.10} to conclude that \(\T \T^* = \IT[\V]\).
\end{proof}

\begin{cor}\label{6.5.4}
  Let \(\T\) be a linear operator on a finite-dimensional real inner product space \(\V\).
  Then \(\V\) has an orthonormal basis of eigenvectors of \(\T\) with corresponding eigenvalues of absolute value \(1\) iff \(\T\) is both self-adjoint and orthogonal.
\end{cor}

\begin{proof}[\pf{6.5.4}]
  Suppose that \(\V\) has an orthonormal basis \(\set{\seq{v}{1,,n}}\) over \(\R\) such that \(\T(v_i) = \lambda_i v_i\) and \(\abs{\lambda_i} = 1\) for all \(i \in \set{1, \dots, n}\).
  By \cref{6.17}, \(\T\) is self-adjoint.
  Thus
  \begin{align*}
    (\T \T^*)(v_i) & = (\T \T)(v_i)            &  & \by{6.4.8}                                          \\
                   & = \T(\lambda_i v_i)       &  & \by{5.1.2}                                          \\
                   & = \lambda_i \lambda_i v_i &  & \by{5.1.2}                                          \\
                   & = \lambda_i^2 v_i                                                                  \\
                   & = v_i                     &  & (\abs{\lambda_i} = 1 \text{ and } \lambda_i \in \R)
  \end{align*}
  for each \(i \in \set{1, \dots, n}\).
  So \(\T \T^* = \IT[\V]\), and again by \cref{ex:2.4.10}, \(\T\) is orthogonal by \cref{6.18}(a).

  If \(\T\) is self-adjoint, then, by \cref{6.17}, we have that \(\V\) possesses an orthonormal basis \(\set{\seq{v}{1,,n}}\) over \(\R\) such that \(\T(v_i) = \lambda_i v_i\) for all \(i \in \set{1, \dots, n}\).
  If \(\T\) is also orthogonal, we have
  \begin{align*}
    \abs{\lambda_i} \cdot \norm{v_i} & = \norm{\lambda_i v_i} &  & \by{6.2}[a] \\
                                     & = \norm{\T(v_i)}       &  & \by{5.1.2}  \\
                                     & = \norm{v_i};          &  & \by{6.5.1}
  \end{align*}
  so \(\abs{\lambda_i} = 1\) for every \(i \in \set{1, \dots, n}\).
\end{proof}

\begin{cor}\label{6.5.5}
  Let \(\T\) be a linear operator on a finite-dimensional complex inner product space \(\V\).
  Then \(\V\) has an orthonormal basis of eigenvectors of \(\T\) with corresponding eigenvalues of absolute value \(1\) iff \(\T\) is unitary.
\end{cor}

\begin{proof}[\pf{6.5.5}]
  Suppose that \(\V\) has an orthonormal basis \(\set{\seq{v}{1,,n}}\) over \(\C\) such that \(\T(v_i) = \lambda_i v_i\) and \(\abs{\lambda_i} = 1\) for all \(i \in \set{1, \dots, n}\).
  By \cref{6.16}, \(\T\) is normal.
  Thus
  \begin{align*}
    (\T^* \T)(v_i) & = (\T \T^*)(v_i)                 &  & \by{6.4.3}            \\
                   & = \T(\conj{\lambda_i} v_i)       &  & \by{6.15}[c]          \\
                   & = \conj{\lambda_i} \lambda_i v_i &  & \by{5.1.2}            \\
                   & = \abs{\lambda_i}^2 v_i          &  & \by{d.0.5}            \\
                   & = v_i                            &  & (\abs{\lambda_i} = 1)
  \end{align*}
  for each \(i \in \set{1, \dots, n}\).
  So \(\T^* \T = \T \T^* = \IT[\V]\) by \cref{2.1.13}.
  Thus by \cref{6.18}(a) \(\T\) is unitary.

  If \(\T\) is unitary, then by \cref{6.18}(a)(e) we know that \(\T\) is normal.
  By \cref{6.16}, we have that \(\V\) possesses an orthonormal basis \(\set{\seq{v}{1,,n}}\) over \(\C\) such that \(\T(v_i) = \lambda_i v_i\) for all \(i \in \set{1, \dots, n}\).
  Then we have
  \begin{align*}
    \abs{\lambda_i} \cdot \norm{v_i} & = \norm{\lambda_i v_i} &  & \by{6.2}[a] \\
                                     & = \norm{\T(v_i)}       &  & \by{5.1.2}  \\
                                     & = \norm{v_i};          &  & \by{6.5.1}
  \end{align*}
  so \(\abs{\lambda_i} = 1\) for every \(i \in \set{1, \dots, n}\).
\end{proof}

\section{Orthogonal Projections and the Spectral Theorem}\label{sec:6.6}

\begin{note}
  In this section, we rely heavily on \cref{6.16,6.17} to develop an elegant representation of a normal (if \(\F = \C\)) or a self-adjoint (if \(\F = \R\)) operator \(\T\) on a finite-dimensional inner product space.
  We prove that \(\T\) can be written in the form \(\seq[+]{\lambda,\T}{1,,k}\), where \(\seq{\lambda}{1,,k} \in \F\) are the distinct eigenvalues of \(\T\) and \(\seq{\T}{1,,k}\) are \emph{orthogonal projections}.

  Recall from \cref{2.1.14} that if \(\V = \W_1 \oplus \W_2\), then a linear operator \(\T\) on \(\V\) is the \textbf{projection on \(\W_1\) along \(\W_2\)} if, whenever \(x = x_1 + x_2\), with \(x_1 \in \W_1\) and \(x_2 \in \W_2\), we have \(\T(x) = x_1\).
  By \cref{ex:2.1.26}(a)(b), we have
  \[
    \rg{\T} = \W_1 = \set{x \in \V : \T(x) = x} \quad \text{and} \quad \ns{\T} = \W_2.
  \]
  So \(\V = \rg{\T} \oplus \ns{\T}\).
  Thus there is no ambiguity if we refer to \(\T\) as a ``projection on \(\W_1\)'' or simply as a ``projection.''
  In fact, it can be shown (see \cref{ex:2.3.17}) that \(\T\) is a projection iff \(\T = \T^2\).
  Because \(\V = \W_1 \oplus \W_2 = \W_1 \oplus \W_3\) does \emph{not} imply that \(\W_2 = \W_3\), we see that \(\W_1\) does not uniquely determine \(\T\).
  For an \emph{orthogonal} projection \(\T\), however, \(\T\) is uniquely determined by its range (see \cref{6.6.2}).
\end{note}

\begin{defn}\label{6.6.1}
  Let \(\V\) be an inner product space over \(\F\), and let \(\T \in \ls(\V)\) be a projection.
  We say that \(\T\) is an \textbf{orthogonal projection} if \(\rg{\T}^{\perp} = \ns{\T}\) and \(\ns{\T}^{\perp} = \rg{\T}\).
\end{defn}

\begin{note}
  By \cref{ex:6.2.13}(c), if \(\V\) is finite-dimensional, we need only assume that one of the preceding conditions holds.
  For example, if \(\rg{\T}^{\perp} = \ns{\T}\), then \(\rg{\T} = \rg{\T}^{\perp \perp} = \ns{\T}^{\perp}\).
\end{note}

\begin{prop}\label{6.6.2}
  Assume that \(\W\) is a finite-dimensional subspace of an inner product space \(\V\) over \(\F\).
  In the notation of \cref{6.6}, we can define a function \(\T : \V \to \V\) by \(\T(y) = u\).
  Then \(\T\) is the unique orthogonal projection on \(\W\).
  We call \(\T\) the \textbf{orthogonal projection} of \(\V\) on \(\W\).
\end{prop}

\begin{proof}[\pf{6.6.2}]
  For convenience, for each \(v \in \V\), we define \((v_1, v_2) \in \W \times \W^{\perp}\) such that \(v = v_1 + v_2\) (such definition is well-defined thanks to \cref{6.6}).

  First we show that \(\T\) is a orthogonal projection of \(\V\) on \(\W\).
  By \cref{2.1.14}, \cref{6.6} and \cref{ex:6.2.13}(d) we see that \(\T\) is a projection on \(\W\) along \(\W^{\perp}\).
  Thus by \cref{ex:2.1.26}(b) we have \(\W = \rg{\T}\) and \(\W^{\perp} = \ns{\T}\).
  Since
  \begin{align*}
         & v \in \rg{\T}^{\perp}                                                       \\
    \iff & \forall y \in \rg{\T}, \inn{v, y} = 0                    &  & \by{6.2.9}    \\
    \iff & \forall x \in \V, 0 = \inn{v, \T(x)} = \inn{v, x_1}      &  & \by{2.1.10}   \\
         & = \inn{v_1 + v_2, x_1} = \inn{v_1, x_1} + \inn{v_2, x_1} &  & \by{6.1.1}[a] \\
         & = \inn{v_1, x_1} = \inn{v_1, x_1} + \inn{v_1, x_2}       &  & \by{6.2.9}    \\
         & = \inn{v_1, x_1 + x_2} = \inn{v_1, x} = \inn{\T(v), x}   &  & \by{6.1.1}[a] \\
    \iff & \T(v) = \zv                                              &  & \by{6.1}[c,d] \\
    \iff & v \in \ns{\T}
  \end{align*}
  and
  \begin{align*}
         & v \in \ns{\T}^{\perp}                                                           \\
    \iff & \forall x \in \ns{\T}, \inn{v, x} = 0                    &  & \by{6.2.9}        \\
    \iff & \forall x \in \W^{\perp}, \inn{v, x} = 0                 &  & \by{ex:2.1.26}[b] \\
    \iff & (v = \zv) \lor (v \notin \W^{\perp} \setminus \set{\zv}) &  & \by{6.1}[c,d]     \\
    \iff & v \in \W                                                                        \\
    \iff & v \in \rg{\T},                                           &  & \by{ex:2.1.26}[b]
  \end{align*}
  we know that \(\rg{\T}^{\perp} = \ns{\T}\) and \(\ns{\T}^{\perp} = \rg{\T}\).
  Thus by \cref{6.6.1} \(\T\) is an orthogonal projection of \(\V\) on \(\W\).

  Now we show that \(\T\) is unique.
  For if \(\T\) and \(\U\) are orthogonal projections on \(\W\), then \(\rg{\T} = \W = \rg{\U}\).
  Hence \(\ns{\T} = \rg{\T}^{\perp} = \rg{\U}^{\perp} = \ns{\U}\), and since every projection is uniquely determined by its range and null space, we have \(\T = \U\).
\end{proof}

\begin{note}
  If \(\T\) is the orthogonal projection of \(\V\) on \(\W\), then \(\T(v)\) is the ``best approximation in \(\W\) to \(v\)'';
  that is, if \(w \in \W\), then \(\norm{w - v} \geq \norm{\T(v) - v}\).
  In fact, this approximation property characterizes \(\T\).
  These results follow immediately from \cref{6.2.12}.
\end{note}

\exercisesection

\begin{ex}\label{ex:6.6.9}

\end{ex}

\begin{ex}\label{ex:6.6.10}

\end{ex}

\section{The Singular Value Decomposition and the Pseudoinverse}\label{sec:6.7}

\begin{thm}[Singular Value Theorem for Linear Transformations]\label{6.26}
  Let \(\V\) and \(\W\) be finite-dimensional inner product spaces over \(\F\), and let \(\T \in \ls(\V, \W)\) be of rank \(r\).
  Then there exist orthonormal bases \(\set{\seq{v}{1,,n}}\) for \(\V\) over \(\F\) and \(\set{\seq{u}{1,,m}}\) for \(\W\) over \(\F\) and positive scalars \(\seq[\geq]{\sigma}{1,,r}\) such that
  \begin{equation}\label{eq:6.7.1}
    \T(v_i) = \begin{dcases}
      \sigma_i u_i & \text{if } i \in \set{1, \dots, r}     \\
      \zv          & \text{if } i \in \set{r + 1, \dots, n}
    \end{dcases}.
  \end{equation}
  Conversely, suppose that the preceding conditions are satisfied.
  Then for \(i \in \set{1, \dots, n}\), \(v_i\) is an eigenvector of \(\T^* \T\) with corresponding eigenvalue \(\sigma_i^2\) if \(i \in \set{1, \dots, r}\) and \(0\) if \(i \in \set{r + 1, \dots, n}\).
  Therefore the scalars \(\seq{\sigma}{1,,r}\) are uniquely determined by \(\T\).
\end{thm}

\begin{proof}[\pf{6.26}]
  Let \(\inn{\cdot, \cdot}_{\V}\) and \(\inn{\cdot, \cdot}_{\W}\) be inner products on \(\V\) and \(\W\), respectively.
  For each \(\T \in \ls(\V, \W)\), we denote the adjoint of \(\T\) with respect to \(\inn{\cdot, \cdot}_{\V}\) and \(\inn{\cdot, \cdot}_{\W}\) as \(\T^*\), i.e.,
  \[
    \forall (x, y) \in \V \times \W, \inn{\T(x), y}_{\W} = \inn{x, \T^*(y)}_{\V}.
  \]
  The existence and uniqueness of adjoint operator has been proved by \cref{ex:6.3.15}.

  We first establish the existence of the bases and scalars.
  By \cref{ex:6.4.18} and \cref{ex:6.3.15}(d), \(\T^* \T\) is a positive semidefinite linear operator of rank \(r\) on \(\V\);
  hence there is an orthonormal basis \(\set{\seq{v}{1,,n}}\) for \(\V\) over \(\F\) with respect to \(\inn{\cdot, \cdot}_{\V}\) consisting of eigenvectors of \(\T^* \T\) with corresponding eigenvalues \(\lambda_i\), where \(\seq[\geq]{\lambda}{1,,r} > 0\), and \(\lambda_i = 0\) for \(i \in \set{r + 1, \dots, n}\).
  For \(i \in \set{1, \dots, r}\), define \(\sigma_i = \sqrt{\lambda_i}\) and \(u_i = \frac{1}{\sigma_i} \T(v_i)\) (see \cref{ex:6.4.17}(a)).
  We show that \(\set{\seq{u}{1,,r}}\) is an orthonormal subset of \(\W\) with respect to \(\inn{\cdot, \cdot}_{\W}\).
  Suppose \(i, j \in \set{1, \dots, r}\).
  Then
  \begin{align*}
    \inn{u_i, u_j}_{\W} & = \inn{\frac{1}{\sigma_i} \T(v_i), \frac{1}{\sigma_j} \T(v_j)}_{\W}                        \\
                        & = \frac{1}{\sigma_i} \inn{\T(v_i), \frac{1}{\sigma_j} \T(v_j)}_{\W} &  & \by{6.1.1}[b]     \\
                        & = \frac{1}{\sigma_i \sigma_j} \inn{\T(v_i), \T(v_j)}_{\W}           &  & \by{6.1}[b]       \\
                        & = \frac{1}{\sigma_i \sigma_j} \inn{\T^*(\T(v_i)), v_j}_{\V}         &  & \by{ex:6.3.15}[a] \\
                        & = \frac{1}{\sigma_i \sigma_j} \inn{\lambda_i v_i, v_j}_{\V}         &  & \by{5.1.2}        \\
                        & = \frac{\sigma_i^2}{\sigma_i \sigma_j} \inn{v_i, v_j}_{\V}          &  & \by{6.1.1}[b]     \\
                        & = \frac{\sigma_i^2}{\sigma_i \sigma_j} \delta_{i j}                 &  & \by{6.1.12}       \\
                        & = \delta_{i j},                                                     &  & \by{2.3.4}
  \end{align*}
  and hence \(\set{\seq{u}{1,,r}}\) is orthonormal.
  By \cref{6.7}(a), this set extends to an orthonormal basis \(\set{\seq{u}{1,,r,,m}}\) for \(\W\) over \(\F\).
  Clearly \(\T(v_i) = \sigma_i u_i\) if \(i \in \set{1, \dots, r}\).
  If \(i \in \set{r + 1, \dots, n}\), then \(\T^* \T(v_i) = \zv\), and so \(\T(v_i) = \zv\) by \cref{ex:6.3.15}(d).

  To establish uniqueness, suppose that \(\set{\seq{v}{1,,n}}\), \(\set{\seq{u}{1,,m}}\), and \(\seq[\geq]{\sigma}{1,,r} > 0\) satisfy the properties stated in the first part of the theorem.
  Then for \(i \in \set{1, \dots, m}\) and \(j \in \set{1, \dots, n}\),
  \begin{align*}
    \inn{\T^*(u_i), v_j}_{\V} & = \inn{u_i, \T(v_j)}_{\W}                                           &  & \by{ex:6.3.15}[a] \\
                              & = \begin{dcases}
                                    \inn{u_i, \sigma_j u_j} & \text{if } j \in \set{1, \dots, r}     \\
                                    \inn{u_i, \zv}          & \text{if } j \in \set{r + 1, \dots, n}
                                  \end{dcases}                         \\
                              & = \begin{dcases}
                                    \sigma_i & \text{if } i = j \leq r \\
                                    0        & \text{otherwise}
                                  \end{dcases},                               &  & \by{6.1}[b,c]
  \end{align*}
  and hence by \cref{6.5}, for any \(i \in \set{1, \dots, m}\),
  \begin{equation}\label{eq:6.7.2}
    \T^*(u_i) = \sum_{j = 1}^n \inn{\T^*(u_i), v_j} v_j = \begin{dcases}
      \sigma_i v_i & \text{if } i = j \leq r \\
      \zv          & \text{otherwise}
    \end{dcases}.
  \end{equation}
  So for \(i \in \set{1, \dots, r}\),
  \[
    \T^* \T(v_i) = \T^*(\sigma_i u_i) = \sigma_i \T^*(u_i) = \sigma_i^2 v_i
  \]
  and \(\T^* \T(v_i) = \T^*(\zv) = \zv\) for \(i \in \set{r + 1, \dots, n}\).
  Therefore each \(v_i\) is an eigenvector of \(\T^* \T\) with corresponding eigenvalue \(\sigma_i^2\) if \(i \in \set{1, \dots, r}\) and \(0\) if \(i \in \set{r + 1, \dots, n}\).
\end{proof}

\begin{defn}\label{6.7.1}
  The unique scalars \(\seq{\sigma}{1,,r}\) in \cref{6.26} are called the \textbf{singular values} of \(\T\).
  If \(r\) is less than both \(m\) and \(n\), then the term singular value is extended to include \(\seq[=]{\sigma}{r+1,,k} = 0\), where \(k\) is the minimum of \(m\) and \(n\).
\end{defn}

\begin{note}
  Although the singular values of a linear transformation \(\T\) are uniquely determined by \(\T\), the orthonormal bases given in the statement of \cref{6.26} are not uniquely determined because there is more than one orthonormal basis of eigenvectors of \(\T^* \T\).

  In view of \cref{eq:6.7.2}, the singular values of a linear transformation \(\T \in \ls(\V, \W)\) and its adjoint \(\T^* \in \ls(\W, \V)\) are identical.
  Furthermore, the orthonormal bases for \(\V\) and \(\W\) given in \cref{6.26} are simply reversed for \(\T^*\).
\end{note}

\begin{defn}\label{6.7.2}
  Let \(A \in \ms\).
  We define the \textbf{singular values} of \(A\) to be the singular values of the linear transformation \(\L_A\).
\end{defn}

\begin{thm}[Singular Value Decomposition Theorem for Matrices]\label{6.27}
  Let \(A \in \ms\) be of rank \(r\) with the positive singular values \(\seq[\geq]{\sigma}{1,,r}\), and let \(\Sigma \in \ms\) defined by
  \[
    \Sigma_{i j} = \begin{dcases}
      \sigma_i & \text{if } i = j \leq r \\
      0        & \text{otherwise}
    \end{dcases}.
  \]
  Then there exists unitary matrices \(U \in \ms[m][m][\F]\) and \(\V \in \ms[n][n][\F]\) such that
  \[
    A = U \Sigma V^*.
  \]
\end{thm}

\begin{proof}[\pf{6.27}]
  Let \(\T = \L_A \in \ls(\vs{F}^n, \vs{F}^m)\).
  By \cref{6.26}, there exist orthonormal bases \(\beta = \set{\seq{v}{1,,n}}\) for \(\vs{F}^n\) over \(\F\) and \(\gamma = \set{\seq{u}{1,,m}}\) for \(\vs{F}^m\) over \(\F\) such that \(\T(v_i) = \sigma_i u_i\) for \(i \in \set{1, \dots, r}\) and \(\T(v_i) = \zv\) for \(i \in \set{r + 1, \dots, n}\).
  Let \(U \in \ms[m][m][\F]\) whose \(j\)th column is \(u_j\) for all \(j \in \set{1, \dots, m}\), and let \(V \in \ms[n][n][\F]\) whose \(j\)th column is \(v_j\) for all \(j \in \set{1, \dots, n}\).
  Note that both \(U\) and \(V\) are unitary matrices.

  By \cref{2.13}(a), the \(j\)th column of \(AV\) is \(A v_j = \sigma_j u_j\).
  Observe that the \(j\)th column of \(\Sigma\) is \(\sigma_j e_j\), where \(e_j\) is the \(j\)th standard vector of \(\vs{F}^m\).
  So by \cref{2.13}(a)(b), the \(j\)th column of \(U \Sigma\) is given by
  \[
    U(\sigma_j e_j) = \sigma_j (U e_j) = \sigma_j u_j.
  \]
  It follows that \(AV\) and \(U \Sigma\) are \(m \times n\) matrices whose corresponding columns are equal, and hence \(AV = U \Sigma\).
  Therefore \(A = A V V^* = U \Sigma V^*\).
\end{proof}

\begin{defn}\label{6.7.3}
  Let \(A \in \ms\) be of rank \(r\) with positive singular values \(\seq[\geq]{\sigma}{1,,r}\).
  A factorization \(A = U \Sigma V^*\) where \(U\) and \(V\) are unitary matrices and \(\Sigma \in \ms\) is defined as in \cref{6.27} is called a \textbf{singular value decomposition} of \(A\).
\end{defn}

\begin{note}
  In the proof of \cref{6.27}, the columns of \(V\) are the vectors in \(\beta\), and the columns of \(U\) are the vectors in \(\gamma\).
  Furthermore, the nonzero singular values of \(A\) are the same as those of \(\L_A\);
  hence they are the square roots of the nonzero eigenvalues of \(A^* A\) or of \(A A^*\).
  (See \cref{ex:6.7.9}.)
\end{note}

\begin{note}
  A singular value decomposition of a matrix can be used to factor a square matrix in a manner analogous to the factoring of a complex number as the product of a complex number of length \(1\) and a nonnegative number.
  In the case of matrices, the complex number of length \(1\) is replaced by a unitary matrix, and the nonnegative number is replaced by a positive semidefinite matrix.
\end{note}

\begin{thm}[Polar Decomposition]\label{6.28}
  For any square matrix \(A\), there exists a unitary matrix \(W\) and a positive semidefinite matrix \(P\) such that
  \[
    A = WP.
  \]
  Furthermore, if \(A\) is invertible, then the representation is unique.
\end{thm}

\begin{proof}[\pf{6.28}]
  By \cref{6.27}, there exist unitary matrices \(U\) and \(V\) and a diagonal matrix \(\Sigma\) with nonnegative diagonal entries such that \(A = U \Sigma V^*\).
  So
  \begin{align*}
    A & = U \Sigma \V^* \by{6.27}                 \\
      & = U I \Sigma \V^*         &  & \by{2.3.4} \\
      & = U V^* V \Sigma V^*      &  & \by{6.5.9} \\
      & = WP,
  \end{align*}
  where \(W = U V^*\) and \(P = V \Sigma V^*\).
  Since \(W\) is the product of unitary matrices, \(W\) is unitary (\cref{ex:6.5.3}), and since \(\Sigma\) is positive semidefinite (\cref{ex:6.4.17}(a)) and \(P\) is unitarily equivalent to \(\Sigma\), \(P\) is positive semidefinite by \cref{ex:6.5.14}.

  Now suppose that \(A\) is invertible and factors as the products
  \[
    A = WP = ZQ,
  \]
  where \(W\) and \(Z\) are unitary and \(P\) and \(Q\) are positive semidefinite.
  Since \(A\) is invertible, it follows that \(P\) and \(Q\) are positive definite and invertible \cref{ex:6.4.19}(c), and therefore \(Z^* W = Q P^{-1}\).
  Thus \(Q P^{-1}\) is unitary (\cref{ex:6.5.3}), and so
  \begin{align*}
    I & = (Q P^{-1})^* (Q P^{-1}) &  & \by{6.5.9}        \\
      & = (P^{-1})^* Q^* Q P^{-1} &  & \by{6.3.2}[c]     \\
      & = (P^{-1})^* Q^2 P^{-1}   &  & \by{6.4.11}       \\
      & = P^{-1} Q^2 P^{-1}.      &  & \by{ex:6.4.19}[c]
  \end{align*}
  Hence \(P^2 = Q^2\).
  Since both \(P\) and \(Q\) are positive definite, it follows that \(P = Q\) by \cref{ex:6.4.17}(d).
  Therefore \(W = Z\), and consequently the factorization is unique.
\end{proof}

\begin{defn}\label{6.7.4}
  The factorization of a square matrix \(A\) as \(WP\) where \(W\) is unitary and \(P\) is positive semidefinite, is called a \textbf{polar decomposition} of \(A\).
\end{defn}

\begin{prop}\label{6.7.5}
  Let \(\V\) and \(\W\) be finite-dimensional inner product spaces over the same field \(\F\), and let \(\T \in \ls(\V, \W)\).
  It is desirable to have a linear transformation from \(\W\) to \(\V\) that captures some of the essence of an inverse of \(\T\) even if \(\T\) is not invertible.
  A simple approach to this problem is to focus on the ``part'' of \(\T\) that is invertible, namely, the restriction of \(\T\) to \(\ns{\T}^{\perp}\).
  Let \(\lt{L} : \ns{\T}^{\perp} \to \rg{\T}\) be the linear transformation defined by \(\lt{L}(x) = \T(x)\) for all \(x \in \ns{\T}^{\perp}\).
  Then \(\lt{L}\) is invertible, and we can use the inverse of \(\lt{L}\) to construct a linear transformation from \(\W\) to \(\V\) that salvages some of the benefits of an inverse of \(\T\).
\end{prop}

\begin{proof}[\pf{6.7.5}]
  Let \(y \in \rg{\T}\).
  Then there exists an \(x \in \V\) such that \(\T(x) = y\).
  By \cref{ex:6.2.13}(d) we have \(\V = \ns{\T} \oplus \ns{\T}^{\perp}\), thus by \cref{6.6} there exists an unique tuple \(\tuple{x}{1,2} \in \ns{\T} \times \ns{\T}^{\perp}\) such that \(x = x_1 + x_2\).
  Then we have
  \begin{align*}
    y & = \T(x)                                 \\
      & = \T(x_1 + x_2)      &  & \by{6.6}      \\
      & = \T(x_1) + \T(x_2)  &  & \by{2.1.2}[a] \\
      & = \zv_{\W} + \T(x_2) &  & \by{2.1.10}   \\
      & = \T(x_2).
  \end{align*}
  Thus \(\lt{L}\) is onto.
  Since
  \begin{align*}
    \dim(\ns{\T}^{\perp}) & = \dim(\V) - \dim(\ns{\T}) &  & \by{6.7}[c] \\
                          & = \dim(\V) - \nt{\T}       &  & \by{2.1.12} \\
                          & = \rk{\T},                 &  & \by{2.3}
  \end{align*}
  by \cref{2.5} we know that \(\lt{L}\) is one-to-one, and thus invertible.
\end{proof}

\begin{defn}\label{6.7.6}
  Let \(\V\) and \(\W\) be finite-dimensional inner product spaces over the same field \(\F\), and let \(\T \in \ls(\V, \W)\).
  Let \(\lt{L} \in \ls(\ns{\T}^{\perp}, \rg{\T})\) defined by \(\lt{L}(x) = \T(x)\) for all \(x \in \ns{\T}^{\perp}\).
  The \textbf{pseudoinverse} (or Moore-Penrose generalized inverse) of \(\T\), denoted by \(\T^{\dag}\), is defined as the unique linear transformation from \(\W\) to \(\V\) such that
  \[
    \T^{\dag}(y) = \begin{dcases}
      \lt{L}^{-1}(y) & \text{for } y \in \rg{\T}         \\
      \zv            & \text{for } y \in \rg{\T}^{\perp}
    \end{dcases}.
  \]

  The pseudoinverse of a linear transformation \(\T\) on a finite-dimensional inner product space exists even if \(\T\) is not invertible (see \cref{6.7.5}).
  Furthermore, if \(\T\) is invertible, then \(\T^{\dag} = \T^{-1}\) because \(\ns{\T}^{\perp} = \V\) (see \cref{2.4}), and \(\lt{L}\) (as just defined) coincides with \(\T\).
\end{defn}

\begin{eg}\label{6.7.7}
  Consider the zero transformation \(\zT \in \ls(\V, \W)\) between two finite-dimensional inner product spaces \(\V\) and \(\W\) over \(\F\).
  Then \(\rg{\zT} = \set{\zv}\), and therefore \(\zT^{\dag}\) is the zero transformation from \(\W\) to \(\V\).
\end{eg}

\begin{prop}\label{6.7.8}
  We can use the singular value theorem to describe the pseudoinverse of a linear transformation.
  Suppose that \(\V\) and \(\W\) are finite-dimensional vector spaces and \(\T \in \ls(\V, \W)\) is of rank \(r\).
  Let \(\set{\seq{v}{1,,n}}\) and \(\set{\seq{u}{1,,m}}\) be orthonormal bases for \(\V\) and \(\W\) over \(\F\), respectively, and let \(\seq[\leq]{\sigma}{1,,r}\) be the nonzero singular values of \(\T\) satisfying \cref{eq:6.7.1} in \cref{6.26}.
  Then \(\set{\seq{v}{1,,r}}\) is a basis for \(\ns{\T}^{\perp}\) over \(\F\), \(\set{\seq{v}{r+1,,n}}\) is a basis for \(\ns{\T}\) over \(\F\), \(\set{\seq{u}{1,,r}}\) is a basis for \(\rg{\T}\) over \(\F\), and \(\set{\seq{u}{r+1,,m}}\) is a basis for \(\rg{\T}^{\perp}\).
  Let \(\lt{L}\) be the restriction of \(\T\) to \(\ns{\T}^{\perp}\), as in the definition of pseudoinverse (\cref{6.7.6}).
  Then \(\lt{L}^{-1}(u_i) = \frac{1}{\sigma_i} v_i\) for \(i \in \set{1, \dots, r}\).
  Therefore
  \begin{equation}\label{eq:6.7.3}
    \T^{\dag}(u_i) = \begin{dcases}
      \frac{1}{\sigma_i} v_i & \text{if } i \in \set{1, \dots, r}     \\
      \zv                    & \text{if } i \in \set{r + 1, \dots, m}
    \end{dcases}.
  \end{equation}
\end{prop}

\begin{proof}[\pf{6.7.8}]
  By \cref{6.7}(c) and \cref{ex:6.2.13} we see that \(\set{\seq{v}{1,,r}}\) is a basis for \(\ns{\T}^{\perp}\) over \(\F\), \(\set{\seq{v}{r+1,,n}}\) is a basis for \(\ns{\T}\) over \(\F\), \(\set{\seq{u}{1,,r}}\) is a basis for \(\rg{\T}\) over \(\F\), and \(\set{\seq{u}{r+1,,m}}\) is a basis for \(\rg{\T}^{\perp}\).
  Then we have
  \begin{align*}
             & \forall x \in \ns{\T}^{\perp}, \lt{L}(x) = \T(x)                           &  & \by{6.7.6}                        \\
    \implies & \forall i \in \set{1, \dots, r}, \lt{L}(v_i) = \T(v_i) = \sigma_i u_i      &  & \by{6.26}                         \\
    \implies & \forall i \in \set{1, \dots, r}, \lt{L}^{-1}(u_i) = \frac{1}{\sigma_i} v_i &  & \by{2.17}                         \\
    \implies & \forall i \in \set{1, \dots, m}, \T^{\dag}(u_i) = \begin{dcases}
                                                                   \frac{1}{\sigma_i} v_i & \text{if } i \in \set{1, \dots, r}     \\
                                                                   \zv                    & \text{if } i \in \set{r + 1, \dots, m}
                                                                 \end{dcases}.         &  & \by{6.7.6}
  \end{align*}
\end{proof}

\exercisesection

\begin{ex}\label{ex:6.7.9}

\end{ex}

\section{Bilinear and Quadratic Forms}\label{sec:6.8}
\section{Einstein's Special Theory of Relativity}\label{sec:6.9}

\begin{note}
  As a consequence of physical experiments performed in the latter half of the nineteenth century (most notably the Michelson--Morley experiment of 1887), physicists concluded that \emph{the results obtained in measuring the speed of light are independent of the velocity of the instrument used to measure the speed of light}.

  This revelation led to a new way of relating coordinate systems used to locate events in space--time.
  The result was Albert Einstein's \emph{special theory of relativity}.
  In this section, we develop via a linear algebra viewpoint the essence of Einstein's theory.
\end{note}

\begin{defn}\label{6.9.1}
  The basic problem is to compare two different inertial (nonaccelerating) coordinate systems \(S\) and \(S'\) in three-space (\(\R^3\)) that are in motion relative to each other under the assumption that the speed of light is the same when measured in either system.
  We assume that \(S'\) moves at a constant velocity in relation to \(S\) as measured from \(S\).
  To simplify matters, let us suppose that the following conditions hold:
  \begin{enumerate}
    \item The corresponding axes of \(S\) and \(S'\) (\(x\) and \(x'\), \(y\) and \(y'\), \(z\) and \(z'\)) are parallel, and the origin of \(S'\) moves in the positive direction of the \(x\)-axis of \(S\) at a constant velocity \(v > 0\) relative to \(S\).
    \item Two clocks \(C\) and \(C'\) are placed in space---the first stationary relative to the coordinate system \(S\) and the second stationary relative to the coordinate system \(S'\).
          These clocks are designed to give real numbers in units of seconds as readings.
          The clocks are calibrated so that at the instant the origins of \(S\) and \(S'\) coincide, both clocks give the reading zero.
    \item The unit of length is the \textbf{light second} (the distance light travels in \(1\) second), and the unit of time is the second.
          Note that, with respect to these units, the speed of light is \(1\) light second per second.
  \end{enumerate}

  Given any event (something whose position and time of occurrence can be described), we may assign a set of \emph{space--time coordinates} to it.
  For example, if \(p\) is an event that occurs at position
  \[
    \begin{pmatrix}
      x \\
      y \\
      z
    \end{pmatrix}
  \]
  relative to \(S\) and at time \(t\) as read on clock \(C\), we can assign to \(p\) the set of coordinates
  \[
    \begin{pmatrix}
      x \\
      y \\
      z \\
      t
    \end{pmatrix}.
  \]
  This ordered \(4\)-tuple is called the \textbf{space--time coordinates} of \(p\) relative to \(S\) and \(C\).
  Likewise, \(p\) has a set of space--time coordinates
  \[
    \begin{pmatrix}
      x' \\
      y' \\
      z' \\
      t'
    \end{pmatrix}
  \]
  relative to \(S'\) and \(C'\).
\end{defn}

\begin{ax}[Axioms of the Special Theory of Relativity]\label{6.9.2}
  Einstein made certain assumptions about \(\T_v : \R^4 \to \R^4\) that led to his special theory of relativity.
  We formulate an equivalent set of assumptions.
  \begin{enumerate}[label=(R \arabic*)]
    \item The speed of any light beam, when measured in either coordinate system using a clock stationary relative to that coordinate system, is \(1\).
    \item The mapping \(\T_v : \R^4 \to \R^4\) is an isomorphism.
    \item If
          \[
            \T_v\begin{pmatrix}
              x \\
              y \\
              z \\
              t
            \end{pmatrix} = \begin{pmatrix}
              x' \\
              y' \\
              z' \\
              t'
            \end{pmatrix},
          \]
          then \(y' = y\) and \(z' = z\).
    \item If
          \[
            \T_v\begin{pmatrix}
              x   \\
              y_1 \\
              z_1 \\
              t
            \end{pmatrix} = \begin{pmatrix}
              x' \\
              y' \\
              z' \\
              t'
            \end{pmatrix} \quad \text{and} \quad \T_v\begin{pmatrix}
              x   \\
              y_2 \\
              z_2 \\
              t
            \end{pmatrix} = \begin{pmatrix}
              x'' \\
              y'' \\
              z'' \\
              t''
            \end{pmatrix},
          \]
          then \(x'' = x'\) and \(t'' = t'\).
    \item The origin of \(S\) moves in the negative direction of the \(x'\)-axis of \(S'\) at the constant velocity \(-v < 0\) as measured from \(S'\).
  \end{enumerate}
  Axioms (R 3) and (R 4) tell us that for \(p \in \R^4\), the second and third coordinates of \(\T_v(p)\) are unchanged and the first and fourth coordinates of \(\T_v(p)\) are independent of the second and third coordinates of \(p\).

  As we will see, these five axioms completely characterize \(\T_v\).
  The operator \(\T_v\) is called the \textbf{Lorentz transformation} in direction \(x\).
  We intend to compute \(\T_v\) and use it to study the curious phenomenon of time contraction.
\end{ax}

\begin{thm}\label{6.39}
  On \(\R^4\), the following statements are true.
  \begin{enumerate}
    \item \(\T_v(e_i) = e_i\) for \(i \in \set{2, 3}\).
    \item \(\spn{\set{\seq{e}{2,3}}}\) is \(\T_v\)-invariant.
    \item \(\spn{\set{\seq{e}{1,4}}}\) is \(\T_v\)-invariant.
    \item Both \(\spn{\set{\seq{e}{2,3}}}\) and \(\spn{\set{\seq{e}{1,4}}}\) are \(\T_v^*\)-invariant.
    \item \(\T_v^*(e_i) = e_i\) for \(i \in \set{2, 3}\).
  \end{enumerate}
\end{thm}

\begin{proof}[\pf{6.39}(a)]
  By \cref{6.9.2} (R 2),
  \[
    \T_v\begin{pmatrix}
      0 \\
      0 \\
      0 \\
      0
    \end{pmatrix} = \begin{pmatrix}
      0 \\
      0 \\
      0 \\
      0
    \end{pmatrix},
  \]
  and hence, by \cref{6.9.2} (R 4), the first and fourth coordinates of
  \[
    \T_v\begin{pmatrix}
      0 \\
      a \\
      b \\
      0
    \end{pmatrix}
  \]
  are both zero for any \(a, b \in \R\).
  Thus, by \cref{6.9.2} (R 3),
  \[
    \T_v\begin{pmatrix}
      0 \\
      1 \\
      0 \\
      0
    \end{pmatrix} = \begin{pmatrix}
      0 \\
      1 \\
      0 \\
      0
    \end{pmatrix} \quad \text{and} \quad \T_v\begin{pmatrix}
      0 \\
      0 \\
      1 \\
      0
    \end{pmatrix} = \begin{pmatrix}
      0 \\
      0 \\
      1 \\
      0
    \end{pmatrix}.
  \]
\end{proof}

\begin{proof}[\pf{6.39}(b)]
  Since
  \begin{align*}
    \T_v(\spn{\set{\seq{e}{2,3}}}) & = \spn{\set{\T_v(e_2), \T_v(e_3)}} &  & \by{2.2}     \\
                                   & = \spn{\set{\seq{e}{2,3}}},        &  & \by{6.39}[a]
  \end{align*}
  by \cref{5.4.1} we see that \(\spn{\set{\seq{e}{2,3}}}\) is \(\T\)-invariant.
\end{proof}

\begin{proof}[\pf{6.39}(c)]
  By \cref{6.9.2} (R 2) we have
  \[
    \T_v(e_1) = \begin{pmatrix}
      a \\
      0 \\
      0 \\
      b
    \end{pmatrix} = a e_1 + b e_4 \quad \text{and} \quad \T_v(e_4) = \begin{pmatrix}
      c \\
      0 \\
      0 \\
      d
    \end{pmatrix} = c e_1 + d e_4
  \]
  for some \(a, b, c, d \in \R\).
  Thus by \cref{5.4.1} we see that \(\spn{\set{\seq{e}{1,4}}}\) is \(\T\)-invariant.
\end{proof}

\begin{proof}[\pf{6.39}(d)]
  Since \(\set{\seq{e}{1,2,3,4}}\) is orthonormal with respect to the standard inner product on \(\R^4\) over \(\R\), by \cref{6.7}(b) we see that
  \[
    \spn{\set{\seq{e}{1,4}}}^{\perp} = \spn{\set{\seq{e}{2,3}}} \quad \text{and} \quad \spn{\set{\seq{e}{2,3}}}^{\perp} = \spn{\set{\seq{e}{1,4}}}.
  \]
  Thus by \cref{ex:6.4.7}(b) we know that \(\spn{\set{\seq{e}{1,4}}}\) and \(\spn{\set{\seq{e}{2,3}}}\) are \(\T_v^*\) invariant.
\end{proof}

\begin{proof}[\pf{6.39}(e)]
  For any \(j \in \set{1, 3, 4}\),
  \begin{align*}
    \inn{\T_v^*(e_2), e_j} & = \inn{e_2, \T_v(e_j)} &  & \by{6.9}       \\
                           & = 0;                   &  & \by{6.39}[a,c]
  \end{align*}
  for \(j = 2\),
  \begin{align*}
    \inn{\T_v^*(e_2), e_2} & = \inn{e_2, \T_v(e_2)} &  & \by{6.9}     \\
                           & = \inn{e_2, e_2}       &  & \by{6.39}[a] \\
                           & = 1.                   &  & \by{6.1.2}
  \end{align*}
  We conclude by \cref{6.2.6} that \(\T_v^*(e_2)\) is a multiple of \(e_2\) (i.e., that \(\T_v^*(e_2) = k e_2\) for some \(k \in \R\)).
  Thus,
  \begin{align*}
    1 & = \inn{e_2, e_2}         &  & \by{6.1.2}    \\
      & = \inn{e_2, \T_v(e_2)}   &  & \by{6.39}[a]  \\
      & = \inn{\T_v^*(e_2), e_2} &  & \by{6.9}      \\
      & = \inn{k e_2, e_2}                          \\
      & = k,                     &  & \by{6.1.1}[b]
  \end{align*}
  and hence \(\T_v^*(e_2) = e_2\).
  Similarly, \(\T_v^*(e_3) = e_3\).
\end{proof}

\begin{thm}\label{6.40}
  Suppose that, at the instant the origins of \(S\) and \(S'\) coincide, a light flash is emitted from their common origin.
  The event of the light flash when measured either relative to \(S\) and \(C\) or relative to \(S'\) and \(C'\) has space--time
  coordinates
  \[
    \begin{pmatrix}
      0 \\
      0 \\
      0 \\
      0
    \end{pmatrix}.
  \]
  Let \(P\) be the set of all events whose space--time coordinates
  \[
    \begin{pmatrix}
      x \\
      y \\
      z \\
      t
    \end{pmatrix}
  \]
  relative to \(S\) and \(C\) are such that the flash is observable from the point with coordinates
  \[
    \begin{pmatrix}
      x \\
      y \\
      z
    \end{pmatrix}
  \]
  (as measured relative to \(S\)) at the time \(t\) (as measured on \(C\)).
  Let us characterize \(P\) in terms of \(x, y, z\), and \(t\).
  Since the speed of light is \(1\), at any time \(t \geq 0\) the light flash is observable from any point whose distance to the origin of \(S\) (as measured on \(S\)) is \(t \cdot 1 = t\).
  These are precisely the points that lie on the sphere of radius \(t\) with center at the origin.
  The coordinates (relative to \(S\)) of such points satisfy the equation \(x^2 + y^2 + z^2 - t^2 = 0\).
  Hence an event lies in \(P\) iff its space--time coordinates
  \[
    \begin{pmatrix}
      x \\
      y \\
      z \\
      t
    \end{pmatrix} \quad (t \geq 0)
  \]
  relative to \(S\) and \(C\) satisfy the equation \(x^2 + y^2 + z^2 - t^2 = 0\).
  By virtue of \cref{6.9.2} (R 1), we can characterize \(P\) in terms of the space--time coordinates relative to \(S'\) and \(C'\) similarly:
  An event lies in \(P\) iff, relative to \(S'\) and \(C'\), its space--time coordinates
  \[
    \begin{pmatrix}
      x' \\
      y' \\
      z' \\
      t'
    \end{pmatrix} \quad (t' \geq 0)
  \]
  satisfy the equation \((x')^2 + (y')^2 + (z')^2 - (t')^2 = 0\).
  Let
  \[
    A = \begin{pmatrix}
      1 & 0 & 0 & 0  \\
      0 & 1 & 0 & 0  \\
      0 & 0 & 1 & 0  \\
      0 & 0 & 0 & -1
    \end{pmatrix}.
  \]
  If \(\inn{\L_A(w), w} = 0\) for some \(w \in \R^4\), then
  \[
    \inn{\T_v^* \L_A \T_v(w), w} = 0.
  \]
\end{thm}

\begin{proof}[\pf{6.40}]
  Let
  \[
    w = \begin{pmatrix}
      x \\
      y \\
      z \\
      t
    \end{pmatrix} \in \R^4,
  \]
  and suppose that \(\inn{\L_A(w), w} = 0\).
  \begin{description}
    \item[Case 1.]
      \(t \geq 0\).
      Since \(\inn{\L_A(w), w} = x^2 + y^2 + z^2 - t^2\), the vector \(w\) gives the coordinates of an event in \(P\) relative to \(S\) and \(C\).
      Because
      \[
        \begin{pmatrix}
          x \\
          y \\
          z \\
          t
        \end{pmatrix} \quad \text{and} \quad \begin{pmatrix}
          x' \\
          y' \\
          z' \\
          t'
        \end{pmatrix}
      \]
      are the space--time coordinates of the same event relative to \(S'\) and \(C'\), \cref{6.9.2} (R 1) yields
      \[
        (x')^2 + (y')^2 + (z')^2 - (t')^2 = 0.
      \]
      Thus \(\inn{\T_v^* \L_A \T_v(w), w} = \inn{\L_A \T_v(w), \T_v(w)} = (x')^2 + (y')^2 + (z')^2 - (t')^2 = 0\), and the conclusion follows.
    \item[Case 2.]
      \(t < 0\).
      The proof follows by applying Case 1 to \(-w\), i.e.,
      \begin{align*}
        0 & = \inn{\T_v^* \L_A \T_v(-w), -w} &  & \text{(by Case 1)} \\
          & = \inn{-\T_v^* \L_A \T_v(w), -w} &  & \by{2.1.1}[b]      \\
          & = -\inn{\T_v^* \L_A \T_v(w), -w} &  & \by{6.1.1}[b]      \\
          & = \inn{\T_v^* \L_A \T_v(w), w}.  &  & \by{6.1}[b]
      \end{align*}
  \end{description}
\end{proof}

\begin{thm}\label{6.41}
  We now proceed to deduce information about \(\T_v\).
  Let
  \[
    w_1 = \begin{pmatrix}
      1 \\
      0 \\
      0 \\
      1
    \end{pmatrix} \quad \text{and} \quad w_2 = \begin{pmatrix}
      1 \\
      0 \\
      0 \\
      -1
    \end{pmatrix}.
  \]
  By \cref{ex:6.9.3}, \(\set{\seq{w}{1,2}}\) is an orthogonal basis for \(\spn{\set{\seq{e}{1,4}}}\) over \(\R\), and \(\spn{\set{\seq{e}{1,4}}}\) is \(\T_v^* \L_A \T_v\)-invariant.
  Then there exist nonzero scalars \(a, b \in \R\) such that
  \begin{enumerate}
    \item \(\T_v^* \L_A \T_v(w_1) = a w_2\).
    \item \(\T_v^* \L_A \T_v(w_2) = b w_1\).
  \end{enumerate}
\end{thm}

\begin{proof}[\pf{6.41}]
  Because \(\inn{\L_A(w_1), w_1} = 0\), \(\inn{\T_v^* \L_A \T_v(w_1), w_1} = 0\) by \cref{6.40}.
  Thus \(\T_v^* \L_A \T_v(w_1)\) is orthogonal to \(w_1\).
  Since \(\spn{\set{\seq{e}{1,4}}} = \spn{\set{\seq{w}{1,2}}}\) is \(\T_v^* \L_A \T_v\)-invariant, \(\T_v^* \L_A \T_v(w_1)\) must lie in this set.
  But \(\set{\seq{w}{1,2}}\) is an orthogonal basis for this subspace, and so \(\T_v^* \L_A \T_v(w_1)\) must be a multiple of \(w_2\).
  Thus \(\T_v^* \L_A \T_v (w_1) = a w_2\) for some scalar \(a \in \R\).
  Since \(\T_v\) and \(A\) are invertible, so is \(\T_v^* \L_A \T_v\) (\cref{ex:6.3.8}).
  Thus \(a \neq 0\), proving (a).

  The proof of (b) is similar to (a).
\end{proof}

\begin{cor}\label{6.9.3}
  Let \(B_v = [\T_v]_{\beta}\), where \(\beta\) is the standard ordered basis for \(\R^4\) over \(\R\).
  Then
  \begin{enumerate}
    \item \(B_v^* A B_v = A\).
    \item \(\T_v^* \L_A \T_v = \L_A\).
  \end{enumerate}
\end{cor}

\begin{proof}[\pf{6.9.3}]
  By \cref{6.41} we know that there exist \(a, b \in \R \setminus \set{0}\) such that
  \begin{align*}
    \T_v^* \L_A \T_v(e_1 + e_4) & = \T_v^* \L_A \T_v(e_1) + \T_v^* \L_A \T_v(e_4) &  & \by{2.1.1}[a] \\
                                & = a e_1 - a e_4;                                &  & \by{6.41}     \\
    \T_v^* \L_A \T_v(e_1 - e_4) & = \T_v^* \L_A \T_v(e_1) - \T_v^* \L_A \T_v(e_4) &  & \by{2.1.2}[c] \\
                                & = b e_1 + b e_4.                                &  & \by{6.41}
  \end{align*}
  Thus we have
  \begin{align*}
    \T_v^* \L_A \T_v(e_1) & = \dfrac{a + b}{2} e_1 - \dfrac{a - b}{2} e_4; \\
    \T_v^* \L_A \T_v(e_4) & = \dfrac{a - b}{2} e_1 - \dfrac{a + b}{2} e_4.
  \end{align*}
  Since
  \begin{align*}
    \forall i \in \set{2, 3}, \T_v^* \L_A \T_v(e_i) & = \T_v^* \L_A(e_i) &  & \by{6.39}[a] \\
                                                    & = \T_v^*(e_i)      &  & \by{2.3.8}   \\
                                                    & = e_i,             &  & \by{6.39}[e]
  \end{align*}
  we have
  \begin{align*}
    [\T_v^* \L_A \T_v]_{\beta} & = [\T_v^*]_{\beta} [\L_A]_{\beta} [\T_v]_{\beta} &  & \by{2.3.3}   \\
                               & = [\T_v]_{\beta}^* [\L_A]_{\beta} [\T_v]_{\beta} &  & \by{6.10}    \\
                               & = [\T_v]_{\beta}^* A [\T_v]_{\beta}              &  & \by{2.15}[a] \\
                               & = B_v^* A B_v                                                      \\
                               & = \begin{pmatrix}
                                     \dfrac{a + b}{2}  & 0 & 0 & \dfrac{a - b}{2}  \\
                                     0                 & 1 & 0 & 0                 \\
                                     0                 & 0 & 1 & 0                 \\
                                     -\dfrac{a - b}{2} & 0 & 0 & -\dfrac{a + b}{2}
                                   \end{pmatrix}. &  & \by{2.2.4}
  \end{align*}
  Then we have
  \begin{align*}
             & (B_v^* A B_v)^* = B_v^* A^* B_v = B_v^* A B_v   &  & \by{6.3.2}[c,d] \\
    \implies & \dfrac{a - b}{2} = -\dfrac{a - b}{2}            &  & \by{6.1.5}      \\
    \implies & \dfrac{a - b}{2} = 0                                                 \\
    \implies & B_v^* A B_v = \begin{pmatrix}
                               \dfrac{a + b}{2} & 0 & 0 & 0                 \\
                               0                & 1 & 0 & 0                 \\
                               0                & 0 & 1 & 0                 \\
                               0                & 0 & 0 & -\dfrac{a + b}{2}
                             \end{pmatrix}.
  \end{align*}
  But
  \begin{align*}
    \inn{\L_A(e_2 + e_4), e_2 + e_4} & = \inn{\L_A(e_2) + \L_A(e_4), e_2 + e_4} &  & \by{2.1.1}[a] \\
                                     & = \inn{e_2 - e_4, e_2 + e_4}             &  & \by{6.39}[a]  \\
                                     & = 0                                      &  & \by{6.1.2}
  \end{align*}
  implies
  \begin{align*}
             & \inn{\T_v^* \L_A \T_v(e_2 + e_4), e_2 + e_4} = 0 &  & \by{6.40}       \\
    \implies & \inn{\begin{pmatrix}
                        0 \\
                        1 \\
                        0 \\
                        -\dfrac{a + b}{2}
                      \end{pmatrix}, e_2 + e_4} = 0                    &  & \by{2.3.1} \\
    \implies & 1 - \dfrac{a + b}{2} = 0                         &  & \by{6.1.2}      \\
    \implies & \dfrac{a + b}{2} = 1                                                  \\
    \implies & B_v^* A B_v = \begin{pmatrix}
                               1 & 0 & 0 & 0  \\
                               0 & 1 & 0 & 0  \\
                               0 & 0 & 1 & 0  \\
                               0 & 0 & 0 & -1
                             \end{pmatrix} = A.
  \end{align*}
  Thus by \cref{2.3.3} we have \(\T_v^* \L_A \T_v = \L_A\).
\end{proof}

\begin{thm}\label{6.42}
  Let \(\beta\) be the standard ordered basis for \(\R^4\).
  Then
  \[
    [\T_v]_{\beta} = B_v = \begin{pmatrix}
      \dfrac{1}{\sqrt{1 - v^2}}  & 0 & 0 & \dfrac{-v}{\sqrt{1 - v^2}} \\
      0                          & 1 & 0 & 0                          \\
      0                          & 0 & 1 & 0                          \\
      \dfrac{-v}{\sqrt{1 - v^2}} & 0 & 0 & \dfrac{1}{\sqrt{1 - v^2}}
    \end{pmatrix}.
  \]
\end{thm}

\begin{proof}[\pf{6.42}]
  Consider the situation \(1\) second after the origins of \(S\) and \(S'\) have coincided as measured by the clock \(C\).
  Since the origin of \(S'\) is moving along the \(x\)-axis at a velocity \(v\) as measured in \(S\), its space--time coordinates relative to \(S\) and \(C\) are
  \[
    \begin{pmatrix}
      v \\
      0 \\
      0 \\
      1
    \end{pmatrix}.
  \]
  Similarly, the space--time coordinates for the origin of \(S'\) relative to \(S'\) and \(C'\) must be
  \[
    \begin{pmatrix}
      0 \\
      0 \\
      0 \\
      t'
    \end{pmatrix}
  \]
  for some \(t' > 0\).
  Thus we have
  \[
    \T_v\begin{pmatrix}
      v \\
      0 \\
      0 \\
      1
    \end{pmatrix} = \begin{pmatrix}
      0 \\
      0 \\
      0 \\
      t'
    \end{pmatrix} \quad \text{for some } t' > 0.
  \]
  By \cref{6.9.3},
  \begin{align*}
    \inn{\T_v^* \L_A \T_v\begin{pmatrix}
                             v \\
                             0 \\
                             0 \\
                             1
                           \end{pmatrix}, \begin{pmatrix}
                                            v \\
                                            0 \\
                                            0 \\
                                            1
                                          \end{pmatrix}} & = \inn{\L_A\begin{pmatrix}
                                                                        v \\
                                                                        0 \\
                                                                        0 \\
                                                                        1
                                                                      \end{pmatrix}, \begin{pmatrix}
                                                                                       v \\
                                                                                       0 \\
                                                                                       0 \\
                                                                                       1
                                                                                     \end{pmatrix}} &  & \by{6.9.3}[b] \\
                                         & = \inn{\begin{pmatrix}
                                                      v \\
                                                      0 \\
                                                      0 \\
                                                      -1
                                                    \end{pmatrix}, \begin{pmatrix}
                                                                     v \\
                                                                     0 \\
                                                                     0 \\
                                                                     1
                                                                   \end{pmatrix}} &  & \by{2.3.1}                      \\
                                         & = v^2 - 1.                     &  & \by{6.1.2}
  \end{align*}
  But also
  \begin{align*}
    \inn{\T_v^* \L_A \T_v\begin{pmatrix}
                             v \\
                             0 \\
                             0 \\
                             1
                           \end{pmatrix}, \begin{pmatrix}
                                            v \\
                                            0 \\
                                            0 \\
                                            1
                                          \end{pmatrix}} & = \inn{\L_A \T_v\begin{pmatrix}
                                                                             v \\
                                                                             0 \\
                                                                             0 \\
                                                                             1
                                                                           \end{pmatrix}, \T_v\begin{pmatrix}
                                                                                                v \\
                                                                                                0 \\
                                                                                                0 \\
                                                                                                1
                                                                                              \end{pmatrix}} &  & \by{6.9} \\
                                         & = \inn{\L_A\begin{pmatrix}
                                                          0 \\
                                                          0 \\
                                                          0 \\
                                                          t'
                                                        \end{pmatrix}, \begin{pmatrix}
                                                                         0 \\
                                                                         0 \\
                                                                         0 \\
                                                                         t'
                                                                       \end{pmatrix}}                                      \\
                                         & = \inn{\begin{pmatrix}
                                                      0 \\
                                                      0 \\
                                                      0 \\
                                                      -t'
                                                    \end{pmatrix}, \begin{pmatrix}
                                                                     0 \\
                                                                     0 \\
                                                                     0 \\
                                                                     t'
                                                                   \end{pmatrix}}     &  & \by{2.3.1}                      \\
                                         & = -(t')^2.                         &  & \by{6.1.2}
  \end{align*}
  Combining equations above, we conclude that \(v^2 - 1 = -(t')^2\), or
  \[
    t' = \sqrt{1 - v^2}.
  \]
  Thus we obtain
  \[
    \T_v\begin{pmatrix}
      v \\
      0 \\
      0 \\
      1
    \end{pmatrix} = \begin{pmatrix}
      0 \\
      0 \\
      0 \\
      \sqrt{1 - v^2}
    \end{pmatrix}.
  \]

  Next recall that the origin of \(S\) moves in the negative direction of the \(x'\)-axis of \(S'\) at the constant velocity \(-v < 0\) as measured from \(S'\).
  (This fact is \cref{6.9.2} (R 5))
  Consequently, \(1\) second after the origins of \(S\) and \(S'\) have coincided as measured on clock \(C\), there exists a time \(t'' > 0\) as measured on clock \(C'\) such that
  \[
    \T_v\begin{pmatrix}
      0 \\
      0 \\
      0 \\
      1
    \end{pmatrix} = \begin{pmatrix}
      -v t'' \\
      0      \\
      0      \\
      t''
    \end{pmatrix}.
  \]
  Since
  \begin{align*}
    \inn{\T_v^* \L_A \T_v\begin{pmatrix}
                             0 \\
                             0 \\
                             0 \\
                             1
                           \end{pmatrix}, \begin{pmatrix}
                                            0 \\
                                            0 \\
                                            0 \\
                                            1
                                          \end{pmatrix}} & = \inn{\L_A\begin{pmatrix}
                                                                        0 \\
                                                                        0 \\
                                                                        0 \\
                                                                        1
                                                                      \end{pmatrix}, \begin{pmatrix}
                                                                                       0 \\
                                                                                       0 \\
                                                                                       0 \\
                                                                                       1
                                                                                     \end{pmatrix}} &  & \by{6.9.3}[b] \\
                                         & = \inn{\begin{pmatrix}
                                                      0 \\
                                                      0 \\
                                                      0 \\
                                                      -1
                                                    \end{pmatrix}, \begin{pmatrix}
                                                                     0 \\
                                                                     0 \\
                                                                     0 \\
                                                                     1
                                                                   \end{pmatrix}} &  & \by{2.3.1}                      \\
                                         & = -1                           &  & \by{6.1.2}
  \end{align*}
  and
  \begin{align*}
    \inn{\T_v^* \L_A \T_v\begin{pmatrix}
                             0 \\
                             0 \\
                             0 \\
                             1
                           \end{pmatrix}, \begin{pmatrix}
                                            0 \\
                                            0 \\
                                            0 \\
                                            1
                                          \end{pmatrix}} & = \inn{\L_A \T_v\begin{pmatrix}
                                                                             0 \\
                                                                             0 \\
                                                                             0 \\
                                                                             1
                                                                           \end{pmatrix}, \T_v\begin{pmatrix}
                                                                                                0 \\
                                                                                                0 \\
                                                                                                0 \\
                                                                                                1
                                                                                              \end{pmatrix}} &  & \by{6.9} \\
                                         & = \inn{\L_A\begin{pmatrix}
                                                          -v t'' \\
                                                          0      \\
                                                          0      \\
                                                          t''
                                                        \end{pmatrix}, \begin{pmatrix}
                                                                         -v t'' \\
                                                                         0      \\
                                                                         0      \\
                                                                         t''
                                                                       \end{pmatrix}}     &  & \by{6.9.2}[R 5]             \\
                                         & = \inn{\begin{pmatrix}
                                                      -v t'' \\
                                                      0      \\
                                                      0      \\
                                                      -t''
                                                    \end{pmatrix}, \begin{pmatrix}
                                                                     -v t'' \\
                                                                     0      \\
                                                                     0      \\
                                                                     t''
                                                                   \end{pmatrix}}     &  & \by{2.3.1}                      \\
                                         & =  v^2 (t'')^2 - (t'')^2,          &  & \by{6.1.2}
  \end{align*}
  it follows that
  \begin{align*}
             & -1 = v^2 (t'')^2 - (t'')^2 = (v^2 - 1) (t'')^2 \\
    \implies & \dfrac{1}{1 - v^2} = (t'')^2                   \\
    \implies & t'' = \dfrac{1}{\sqrt{1 - v^2}};
  \end{align*}
  hence
  \[
    \T_v\begin{pmatrix}
      0 \\
      0 \\
      0 \\
      1
    \end{pmatrix} = \begin{pmatrix}
      \dfrac{-v}{\sqrt{1 - v^2}} \\
      0                          \\
      0                          \\
      \dfrac{1}{\sqrt{1 - v^2}}
    \end{pmatrix}.
  \]

  Since
  \begin{align*}
    \T_v(e_1) & = \dfrac{1}{v} \T_v(v e_1)                                &  & \by{2.1.1}[b]    \\
              & = \dfrac{1}{v} \T_v(v e_1 + e_4 - e_4)                                          \\
              & = \dfrac{1}{v} \T_v(v e_1 + e_4) - \dfrac{1}{v} \T_v(e_4) &  & \by{2.1.2}[c]    \\
              & = \dfrac{1}{v} \begin{pmatrix}
                                 0 \\
                                 0 \\
                                 0 \\
                                 \sqrt{1 - v^2}
                               \end{pmatrix} - \dfrac{1}{v} \begin{pmatrix}
                                                              \dfrac{-v}{\sqrt{1 - v^2}} \\
                                                              0                          \\
                                                              0                          \\
                                                              \dfrac{1}{\sqrt{1 - v^2}}
                                                            \end{pmatrix} \\
              & = \begin{pmatrix}
                    \dfrac{1}{\sqrt{1 - v^2}} \\
                    0                         \\
                    0                         \\
                    \dfrac{-v}{\sqrt{1 - v^2}}
                  \end{pmatrix};                                            \\
    \T_v(e_2) & = e_2;                                                    &  & \by{6.39}[a]     \\
    \T_v(e_3) & = e_3;                                                    &  & \by{6.39}[a]     \\
    \T_v(e_4) & = \begin{pmatrix}
                    \dfrac{-v}{\sqrt{1 - v^2}} \\
                    0                          \\
                    0                          \\
                    \dfrac{1}{\sqrt{1 - v^2}}
                  \end{pmatrix},
  \end{align*}
  by \cref{2.2.4} we conclude that
  \[
    [\T_v]_{\beta} = \begin{pmatrix}
      \dfrac{1}{\sqrt{1 - v^2}}  & 0 & 0 & \dfrac{-v}{\sqrt{1 - v^2}} \\
      0                          & 1 & 0 & 0                          \\
      0                          & 0 & 1 & 0                          \\
      \dfrac{-v}{\sqrt{1 - v^2}} & 0 & 0 & \dfrac{1}{\sqrt{1 - v^2}}
    \end{pmatrix}.
  \]
\end{proof}

\begin{cor}\label{6.9.4}
  A most curious and paradoxical conclusion follows if we accept Einstein's theory.
  Suppose that an astronaut leaves our solar system in a space vehicle traveling at a fixed velocity \(v\) as measured relative to our solar system.
  It follows from Einstein's theory that, at the end of time \(t\) as measured on Earth, the time that passes on the space vehicle is only \(t \sqrt{1 - v^2}\).
\end{cor}

\begin{proof}[\pf{6.9.4}]
  To establish this result, consider the coordinate systems \(S\) and \(S'\) and clocks \(C\) and \(C'\) that we have been studying.
  Suppose that the origin of \(S'\) coincides with the space vehicle and the origin of \(S\) coincides with a point in the solar system (stationary relative to the sun) so that the origins of \(S\) and \(S'\) coincide and clocks \(C\) and \(C'\) read zero at the moment the astronaut embarks on the trip.

  As viewed from \(S\), the space--time coordinates of the vehicle at any time \(t > 0\) as measured by \(C\) are
  \[
    \begin{pmatrix}
      vt \\
      0  \\
      0  \\
      t
    \end{pmatrix},
  \]
  whereas, as viewed from \(S'\), the space--time coordinates of the vehicle at any time \(t' > 0\) as measured by \(C'\) are
  \[
    \begin{pmatrix}
      0 \\
      0 \\
      0 \\
      t'
    \end{pmatrix}.
  \]
  But if two sets of space--time coordinates
  \[
    \begin{pmatrix}
      vt \\
      0  \\
      0  \\
      t
    \end{pmatrix} \quad \text{and} \quad \begin{pmatrix}
      0 \\
      0 \\
      0 \\
      t'
    \end{pmatrix}
  \]
  are to describe the same event, it must follow that
  \[
    \T_v\begin{pmatrix}
      vt \\
      0  \\
      0  \\
      t
    \end{pmatrix} = \begin{pmatrix}
      0 \\
      0 \\
      0 \\
      t'
    \end{pmatrix}.
  \]
  Thus
  \begin{align*}
    [\T_v]_{\beta} \begin{pmatrix}
                     vt \\
                     0  \\
                     0  \\
                     t
                   \end{pmatrix} & = \begin{pmatrix}
                                       \dfrac{1}{\sqrt{1 - v^2}}  & 0 & 0 & \dfrac{-v}{\sqrt{1 - v^2}} \\
                                       0                          & 1 & 0 & 0                          \\
                                       0                          & 0 & 1 & 0                          \\
                                       \dfrac{-v}{\sqrt{1 - v^2}} & 0 & 0 & \dfrac{1}{\sqrt{1 - v^2}}
                                     \end{pmatrix} \begin{pmatrix}
                                                     vt \\
                                                     0  \\
                                                     0  \\
                                                     t
                                                   \end{pmatrix} &  & \by{6.42} \\
                                   & = \begin{pmatrix}
                                         0 \\
                                         0 \\
                                         0 \\
                                         t'
                                       \end{pmatrix}.
  \end{align*}
  From the preceding equation, we obtain \(\dfrac{-v^2 t}{\sqrt{1 - v^2}} + \dfrac{t}{\sqrt{1 - v^2}} = t'\), or
  \[
    t' = t \sqrt{1 - v^2}.
  \]
  This is the desired result.
\end{proof}

\begin{note}
  A dramatic consequence of time contraction is that distances are contracted along the line of motion (see \cref{ex:6.9.9}).
\end{note}

\begin{cor}\label{6.9.5}
  Let us make one additional point.
  Suppose that we consider units of distance and time more commonly used than the light second and second, such as the mile and hour, or the kilometer and second.
  Let \(c\) denote the speed of light relative to our chosen units of distance.
  It is easily seen that if an object travels at a velocity \(v\) relative to a set of units, then it is traveling at a velocity \(v / c\) in units of light seconds per second.
  Thus, for an arbitrary set of units of distance and time, we have
  \[
    t' = t \sqrt{1 - \dfrac{v^2}{c^2}}.
  \]
\end{cor}

\begin{proof}[\pf{6.9.5}]
  Let the distance and time units of \(v\) be \(d_v\) and \(s_v\).
  Let the unit of light second and second be \(d_c\) and \(s_c\).
  So the object is traveling at velocity \(v \, d_v\) unit per \(s_v\) time and light is traveling at velocity \(1 \, d_c\) per \(s_c\) time.
  If light is traveling at velocity \(c \, d_v\) unit per \(s_v\) time, then the object must be traveling at velocity \(v / c \, d_c\) per \(s_c\) time since
  \[
    \dfrac{1}{c} = \dfrac{\dfrac{v}{c}}{v}.
  \]
  Thus by \cref{6.9.4} we have
  \[
    t' = t \sqrt{1 - \dfrac{v^2}{c^2}}.
  \]
\end{proof}

\exercisesection

\setcounter{ex}{2}
\begin{ex}\label{ex:6.9.3}
  For
  \[
    w_1 = \begin{pmatrix}
      1 \\
      0 \\
      0 \\
      1
    \end{pmatrix} \quad \text{and} \quad w_2 = \begin{pmatrix}
      1 \\
      0 \\
      0 \\
      -1
    \end{pmatrix},
  \]
  show that
  \begin{enumerate}
    \item \(\set{\seq{w}{1,2}}\) is an orthogonal basis for \(\spn{\set{\seq{e}{1,4}}}\) over \(\R\);
    \item \(\spn{\set{\seq{e}{1,4}}}\) is \(\T_v^* \L_A \T_v\)-invariant, where \(\L_A\) is defined in \cref{6.40}.
  \end{enumerate}
\end{ex}

\begin{proof}[\pf{ex:6.9.3}(a)]
  By \cref{6.1.2} \(\set{\seq{w}{1,2}}\) is orthogonal with respect to the standard inner product on \(\R^4\) over \(\R\).
  Since
  \begin{align*}
    w_1                            & = e_1 + e_4; \\
    w_2                            & = e_1 - e_4; \\
    \dim(\spn{\set{\seq{e}{1,4}}}) & = 2,
  \end{align*}
  by \cref{1.6.15}(b) and \cref{6.2.4} we know that \(\set{\seq{w}{1,2}}\) is an orthogonal basis for \(\spn{\set{\seq{e}{1,4}}}\) over \(\R\).
\end{proof}

\begin{proof}[\pf{ex:6.9.3}(b)]
  Since
  \begin{align*}
    (\T_v^* \L_A \T_v)(\spn{\set{\seq{e}{1,4}}}) & = \T_v^*(\L_A(\T_v(\spn{\set{\seq{e}{1,4}}})))                         \\
                                                 & \subseteq \T_v^*(\L_A(\spn{\set{\seq{e}{1,4}}})) &  & \by{6.39}[c]     \\
                                                 & = \T_v^*(\L_A(\spn{\set{\seq{w}{1,2}}}))         &  & \by{ex:6.9.3}[a] \\
                                                 & = \T_v^*(\spn{\set{\seq{w}{1,2}}})               &  & \by{2.3.1}       \\
                                                 & = \T_v^*(\spn{\set{\seq{e}{1,4}}})               &  & \by{ex:6.9.3}[a] \\
                                                 & \subseteq \spn{\set{\seq{e}{1,4}}},              &  & \by{6.39}[d]
  \end{align*}
  by \cref{5.4.1} we know that \(\spn{\set{\seq{e}{1,4}}}\) is \(\T_v^* \L_A \T_v\)-invariant.
\end{proof}

\setcounter{ex}{5}
\begin{ex}\label{ex:6.9.6}
  Consider three coordinate systems \(S\), \(S'\), and \(S''\) with the corresponding axes (\(x, x', x''\); \(y, y', y''\); and \(z, z', z''\)) parallel and such that the \(x\)-, \(x'\)-, and \(x''\)-axes coincide.
  Suppose that \(S'\) is moving past \(S\) at a velocity \(v_1 > 0\) (as measured on \(S\)), \(S''\) is moving past \(S'\) at a velocity \(v_2 > 0\) (as measured on \(S'\)), and \(S''\) is moving past \(S\) at a velocity \(v_3 > 0\) (as measured on \(S\)), and that there are three clocks \(C\), \(C'\), and \(C''\) such that \(C\) is stationary relative to \(S\), \(C'\) is stationary relative to \(S'\), and \(C''\) is stationary relative to \(S''\).
  Suppose that when measured on any of the three clocks, all the origins of \(S\), \(S'\), and \(S''\) coincide at time \(0\).
  Assuming that \(\T_{v_3} = \T_{v_2} \T_{v_1}\) (i.e., \(B_{v_3} = B_{v_2} B_{v_1}\)), prove that
  \[
    v_3 = \dfrac{v_1 + v_2}{1 + v_1 v_2}.
  \]
  Note that substituting \(v_2 = 1\) in this equation yields \(v_3 = 1\).
  This tells us that the speed of light as measured in \(S\) or \(S'\) is the same.
  Why would we be surprised if this were not the case?
\end{ex}

\begin{proof}[\pf{ex:6.9.6}]
  Let \(\beta\) be the standard ordered basis for \(\R^4\) over \(\R\).
  Then we have
  \begin{align*}
             & \T_{v_3} = \T_{v_2} \T_{v_1}                                                                                                                                           \\
    \implies & B_{v_3} = [\T_{v_3}]_{\beta} = [\T_{v_2} \T_{v_1}]_{\beta}                                                                                             &  & \by{6.9.3} \\
             & = [\T_{v_2}]_{\beta} [\T_{v_1}]_{\beta} = B_{v_2} B_{v_1}                                                                                              &  & \by{2.3.3} \\
    \implies & \begin{pmatrix}
                 \dfrac{1}{\sqrt{1 - v_3^2}}    & 0 & 0 & \dfrac{-v_3}{\sqrt{1 - v_3^2}} \\
                 0                              & 1 & 0 & 0                              \\
                 0                              & 0 & 1 & 0                              \\
                 \dfrac{-v_3}{\sqrt{1 - v_3^2}} & 0 & 0 & \dfrac{1}{\sqrt{1 - v_3^2}}
               \end{pmatrix}                                                               &  & \by{6.42}                                                                             \\
             & = \begin{pmatrix}
                   \dfrac{1}{\sqrt{1 - v_2^2}}    & 0 & 0 & \dfrac{-v_2}{\sqrt{1 - v_2^2}} \\
                   0                              & 1 & 0 & 0                              \\
                   0                              & 0 & 1 & 0                              \\
                   \dfrac{-v_2}{\sqrt{1 - v_2^2}} & 0 & 0 & \dfrac{1}{\sqrt{1 - v_2^2}}
                 \end{pmatrix} \begin{pmatrix}
                                 \dfrac{1}{\sqrt{1 - v_1^2}}    & 0 & 0 & \dfrac{-v_1}{\sqrt{1 - v_1^2}} \\
                                 0                              & 1 & 0 & 0                              \\
                                 0                              & 0 & 1 & 0                              \\
                                 \dfrac{-v_1}{\sqrt{1 - v_1^2}} & 0 & 0 & \dfrac{1}{\sqrt{1 - v_1^2}}
                               \end{pmatrix}                                                               &  & \by{6.42}                                                             \\
             & = \begin{pmatrix}
                   \dfrac{1 + v_1 v_2}{\sqrt{1 - v_2^2} \sqrt{1 - v_1^2}} & 0 & 0 & -\dfrac{v_1 + v_2}{\sqrt{1 - v_2^2} \sqrt{1 - v_1^2}}  \\
                   0                                                      & 1 & 0 & 0                                                      \\
                   0                                                      & 0 & 1 & 0                                                      \\
                   -\dfrac{v_1 + v_2}{\sqrt{1 - v_2^2} \sqrt{1 - v_1^2}}  & 0 & 0 & \dfrac{1 + v_1 v_2}{\sqrt{1 - v_2^2} \sqrt{1 - v_1^2}}
                 \end{pmatrix} &  & \by{2.3.1}                  \\
    \implies & \begin{dcases}
                 \dfrac{1}{\sqrt{1 - v_3^2}} = \dfrac{1 + v_1 v_2}{\sqrt{1 - v_2^2} \sqrt{1 - v_1^2}} \\
                 \dfrac{v_3}{\sqrt{1 - v_3^2}} = \dfrac{v_1 + v_2}{\sqrt{1 - v_2^2} \sqrt{1 - v_1^2}}
               \end{dcases}                                           &  & \by{1.2.8}                                                              \\
    \implies & v_3 = \dfrac{v_3}{1} = \dfrac{v_1 + v_2}{1 + v_1 v_2}.
  \end{align*}
  If \(v_1 = v_2 = 1\) but \(v_3 \neq 1\), then \cref{6.9.2} (R 1) does not hold.
\end{proof}

\begin{ex}\label{ex:6.9.7}
  Compute \((B_v)^{-1}\).
  Show \((B_v)^{-1} = B_{(-v)}\).
  Conclude that if \(S'\) moves at a negative velocity \(v\) relative to \(S\), then \([\T_v]_{\beta} = B_v\), where \(B_v\) is of the form given in \cref{6.42}.
\end{ex}

\begin{proof}[\pf{ex:6.9.7}]
  First observe that
  \begin{align*}
             & \begin{pmatrix}
                 \dfrac{1}{\sqrt{1 - v^2}} & 0 & 0 & \dfrac{v}{\sqrt{1 - v^2}} \\
                 0                         & 1 & 0 & 0                         \\
                 0                         & 0 & 1 & 0                         \\
                 \dfrac{v}{\sqrt{1 - v^2}} & 0 & 0 & \dfrac{1}{\sqrt{1 - v^2}}
               \end{pmatrix} \begin{pmatrix}
                               \dfrac{1}{\sqrt{1 - v^2}}  & 0 & 0 & \dfrac{-v}{\sqrt{1 - v^2}} \\
                               0                          & 1 & 0 & 0                          \\
                               0                          & 0 & 1 & 0                          \\
                               \dfrac{-v}{\sqrt{1 - v^2}} & 0 & 0 & \dfrac{1}{\sqrt{1 - v^2}}
                             \end{pmatrix}            \\
             & = I_4                                                                                      &  & \by{2.3.1} \\
    \implies & (B_v)^{-1} = \begin{pmatrix}
                              \dfrac{1}{\sqrt{1 - v^2}} & 0 & 0 & \dfrac{v}{\sqrt{1 - v^2}} \\
                              0                         & 1 & 0 & 0                         \\
                              0                         & 0 & 1 & 0                         \\
                              \dfrac{v}{\sqrt{1 - v^2}} & 0 & 0 & \dfrac{1}{\sqrt{1 - v^2}}
                            \end{pmatrix}         &  & \by{ex:2.4.10}[b]               \\
             & = \begin{pmatrix}
                   \dfrac{1}{\sqrt{1 - (-v)^2}}     & 0 & 0 & \dfrac{-(-v)}{\sqrt{1 - (-v)^2}} \\
                   0                                & 1 & 0 & 0                                \\
                   0                                & 0 & 1 & 0                                \\
                   \dfrac{-(-v)}{\sqrt{1 - (-v)^2}} & 0 & 0 & \dfrac{1}{\sqrt{1 - (-v)^2}}
                 \end{pmatrix} = B_{(-v)}. &  & \by{6.42}
  \end{align*}
  If \(S'\) moves at a negative velocity \(v\) relative to \(S\), then \(S\) moves at a positive velocity \(-v\) relative to \(S'\).
  Thus
  \begin{align*}
    [\T_v]_{\beta} & = [\T_{-v}^{-1}]_{\beta}                                    \\
                   & = [\T_{-v}]_{\beta}^{-1} &  & \by{2.4.6}                    \\
                   & = B_{(-v)}^{-1}          &  & \by{6.42}                     \\
                   & = B_v.                   &  & \text{(from the proof above)}
  \end{align*}
\end{proof}

\begin{ex}\label{ex:6.9.8}
  Suppose that an astronaut left Earth in the year \(2000\) and traveled to a star \(99\) light years away from Earth at \(99\%\) of the speed of light and that upon reaching the star immediately turned around and returned to Earth at the same speed.
  Assuming Einstein's special theory of relativity, show that if the astronaut was \(20\) years old at the time of departure, then he or she would return to Earth at age \(48.2\) in the year \(2200\).
  Explain the use of \cref{ex:6.9.7} in solving this problem.
\end{ex}

\begin{proof}[\pf{ex:6.9.8}]
  The time (measure on Earth) traveling from Earth to the star is \(\dfrac{99}{0.99} = 100\) years.
  The time (measure on the space ship) traveling from Earth to the star is \(100 \sqrt{1 - 0.99^2} \approx 14.1\) years by \cref{6.9.4}.
  Using \cref{ex:6.9.7}, we get the same result when traveling back from the star to Earth.
  Thus the total time measure on Earth is \(100 \times 2 = 200\) years and the total time measure on space ship is \(14.1 \times 2 = 28.2\) years.
  If the astronaut was \(20\) years old at the time of departure (year \(2000\)), then he or she would return to Earth at age \(20 + 28.2 = 48.2\) in the year \(2000 + 200 = 2200\).
\end{proof}

\begin{ex}\label{ex:6.9.9}
  Recall the moving space vehicle considered in the study of time contraction.
  Suppose that the vehicle is moving toward a fixed star located on the \(x\)-axis of \(S\) at a distance \(b\) units from the origin of \(S\).
  If the space vehicle moves toward the star at velocity \(v\), Earthlings (who remain ``almost'' stationary relative to \(S\)) compute the time it takes for the vehicle to reach the star as \(t = b / v\).
  Due to the phenomenon of time contraction, the astronaut perceives a time span of \(t' = t \sqrt{1 - v^2} = (b / v) \sqrt{1 - v^2}\).
  A paradox appears in that the astronaut perceives a time span inconsistent with a distance of \(b\) and a velocity of \(v\).
  The paradox is resolved by observing that the distance from the solar system to the star as measured by the astronaut is less than \(b\).

  Assuming that the coordinate systems \(S\) and \(S'\) and clocks \(C\) and \(C'\) are as in the discussion of time contraction, prove the following results.
  \begin{enumerate}
    \item At time \(t\) (as measured on \(C\)), the space--time coordinates of star relative to \(S\) and \(C\) are
          \[
            \begin{pmatrix}
              b \\
              0 \\
              0 \\
              t
            \end{pmatrix}.
          \]
    \item At time \(t\) (as measured on \(C\)), the space--time coordinates of the star relative to \(S'\) and \(C'\) are
          \[
            \begin{pmatrix}
              \dfrac{b - vt}{\sqrt{1 - v^2}} \\
              0                              \\
              0                              \\
              \dfrac{t - bv}{\sqrt{1 - v^2}}
            \end{pmatrix}.
          \]
    \item For
          \[
            x' = \dfrac{b - tv}{\sqrt{1 - v^2}} \quad \text{and} \quad t' = \dfrac{t - bv}{\sqrt{1 - v^2}},
          \]
          we have \(x' = b \sqrt{1 - v^2} - t' v\).
          This result may be interpreted to mean that at time \(t'\) as measured by the astronaut, the distance from the astronaut to the star as measured by the astronaut is
          \[
            b \sqrt{1 - v^2} - t' v.
          \]
    \item Conclude from the preceding equation that
          \begin{enumerate}[label=(\arabic*)]
            \item the speed of the space vehicle relative to the star, as measured by the astronaut, is \(v\);
            \item the distance from Earth to the star, as measured by the astronaut, is \(b \sqrt{1 - v^2}\).
          \end{enumerate}
          Thus distances along the line of motion of the space vehicle appear to be contracted by a factor of \(\sqrt{1 - v^2}\).
  \end{enumerate}
\end{ex}

\begin{proof}[\pf{ex:6.9.9}(a)]
  We have
  \[
    \begin{pmatrix}
      vt \\
      0  \\
      0  \\
      t
    \end{pmatrix} = \begin{pmatrix}
      \dfrac{b}{t} t \\
      0              \\
      0              \\
      t
    \end{pmatrix} = \begin{pmatrix}
      b \\
      0 \\
      0 \\
      t
    \end{pmatrix}.
  \]
\end{proof}

\begin{proof}[\pf{ex:6.9.9}(b)]
  We have
  \begin{align*}
    \br{\T_v\begin{pmatrix}
                b \\
                0 \\
                0 \\
                t
              \end{pmatrix}}_{\beta} & = [\T_v]_{\beta} \begin{pmatrix}
                                                        b \\
                                                        0 \\
                                                        0 \\
                                                        t
                                                      \end{pmatrix}                                                     &  & \by{2.14} \\
                            & = \begin{pmatrix}
                                  \dfrac{1}{\sqrt{1 - v^2}}  & 0 & 0 & \dfrac{-v}{\sqrt{1 - v^2}} \\
                                  0                          & 1 & 0 & 0                          \\
                                  0                          & 0 & 1 & 0                          \\
                                  \dfrac{-v}{\sqrt{1 - v^2}} & 0 & 0 & \dfrac{1}{\sqrt{1 - v^2}}
                                \end{pmatrix} \begin{pmatrix}
                                                b \\
                                                0 \\
                                                0 \\
                                                t
                                              \end{pmatrix} &  & \by{6.42}                      \\
                            & = \begin{pmatrix}
                                  \dfrac{b - vt}{\sqrt{1 - v^2}} \\
                                  0                              \\
                                  0                              \\
                                  \dfrac{t - bv}{\sqrt{1 - v^2}}
                                \end{pmatrix}.                                           &  & \by{2.3.1}
  \end{align*}
\end{proof}

\begin{proof}[\pf{ex:6.9.9}(c)]
  We have
  \begin{align*}
    b \sqrt{1 - v^2} - t' v & = b \sqrt{1 - v^2} - \dfrac{t - bv}{\sqrt{1 - v^2}} v \\
                            & = \dfrac{b - b v^2 - tv + b v^2}{\sqrt{1 - v^2}}      \\
                            & = \dfrac{b - tv}{\sqrt{1 - v^2}}                      \\
                            & = x'.
  \end{align*}
\end{proof}

\begin{proof}[\pf{ex:6.9.9}(d)]
  For (1), we have
  \begin{align*}
    \dfrac{\eval{x'}_{t'} - \eval{x'}_0}{t' - 0} & = \dfrac{b \sqrt{1 - v^2} - t' v - b \sqrt{1 - v^2}}{t'} &  & \by{ex:6.9.9}[c] \\
                                                 & = -v.
  \end{align*}
  Thus the speed of the space vehicle measured by the astronaut is \(\abs{-v} = v\).

  For (2), we have
  \begin{align*}
    v t' & = v t \sqrt{1 - v^2}            &  & \by{6.9.4} \\
         & = \dfrac{b}{t} t \sqrt{1 - v^2}                 \\
         & = b \sqrt{1 - v^2}.
  \end{align*}
  Thus the distances from Earch to the star, as measured by the astronaut, is \(b \sqrt{1 - v^2}\).
\end{proof}

\section{Conditioning and the Rayleigh Quotient}\label{sec:6.10}

\begin{defn}\label{6.10.1}
  In \cref{sec:3.4}, we studied specific techniques that allow us to solve systems of linear equations in the form \(Ax = b\), where \(A \in \ms\) and \(b \in \ms[m][1][\F]\).
  Such systems often arise in applications to the real world.
  The coefficients in the system are frequently obtained from experimental data, and, in many cases, both \(m\) and \(n\) are so large that a computer must be used in the calculation of the solution.
  Thus two types of errors must be considered.
  First, experimental errors arise in the collection of data since no instruments can provide completely accurate measurements.
  Second, computers introduce roundoff errors.
  One might intuitively feel that small relative changes in the coefficients of the system cause small relative errors in the solution.
  A system that has this property is called \textbf{well-conditioned};
  otherwise, the system is called \textbf{ill-conditioned}.
\end{defn}

\begin{note}
  We now consider several examples of these types of errors, concentrating primarily on changes in \(b\) rather than on changes in the entries of \(A\).
  In addition, we assume that \(A\) is a square, complex (or real), invertible matrix since this is the case most frequently encountered in applications.

  Of course, we are really interested in \emph{relative changes} since a change in the solution of, say, \(10\), is considered large if the original solution is of the order \(10^{-2}\), but small if the original solution is of the order \(10^6\).
\end{note}

\begin{defn}\label{6.10.2}
  We use the notation \(\delta b\) to denote the vector \(b' - b\), where \(b\) is the vector in the original system and \(b'\) is the vector in the modified system.
  We now define the \textbf{relative change} in \(b\) to be the scalar \(\dfrac{\norm{\delta b}}{\norm{b}}\), where \(\norm{\cdot}\) denotes the standard norm on \(\C^n\) (or \(\R^n\));
  that is, \(\norm{b} = \sqrt{\inn{b, b}}\).
  Most of what follows, however, is true for any norm.
  Similar definitions hold for the \textbf{relative change} in \(x\).
\end{defn}

\begin{note}
  If the lines defined by the two equations are nearly coincident, then a small change in either line could greatly alter the point of intersection, that is, the solution to the system.
\end{note}

\begin{defn}\label{6.10.3}
  Let \(A\) be a complex (or real) \(n \times n\) matrix.
  Define the \textbf{(Euclidean) norm} of \(A\) by
  \[
    \norm{A} = \max_{x \neq \zv} \dfrac{\norm{Ax}}{\norm{x}},
  \]
  where \(x \in \C^n\) or \(x \in \R^n\).
\end{defn}

\begin{note}
  Intuitively, \(\norm{A}\) represents the maximum \emph{magnification} of a vector by the matrix \(A\).
  The question of whether or not this maximum exists, as well as the problem of how to compute it, can be answered by the use of the so-called \emph{Rayleigh quotient}.
\end{note}

\begin{defn}\label{6.10.4}
  Let \(B\) be an \(n \times n\) self-adjoint matrix.
  The \textbf{Rayleigh quotient} for \(x \neq \zv\) is defined to be the scalar \(R(x) = \dfrac{\inn{Bx, x}}{\norm{x}^2}\).
\end{defn}

\begin{thm}\label{6.43}
  For a self-adjoint matrix \(B \in \ms[n][n][\F]\), we have that \(\max_{x \neq \zv} R(x)\) is the largest eigenvalue of \(B\) and \(\min_{x \neq \zv} R(x)\) is the smallest eigenvalue of \(B\).
\end{thm}

\begin{proof}[\pf{6.10.4}]
  By \cref{6.19,6.20}, we may choose an orthonormal basis \(\set{\seq{v}{1,,n}}\) of eigenvectors of \(B\) over \(\F\) such that \(B v_i = \lambda_i v_i\) (\(i \in \set{1, \dots, n}\)), where \(\seq[\geq]{\lambda}{1,,n}\).
  (Recall that by \cref{6.4.10}(a), the eigenvalues of \(B\) are real.)
  Now, for \(x \in \vs{F}^n\), there exist scalars \(\seq{a}{1,,n} \in \F\) such that
  \[
    x = \sum_{i = 1}^n a_i v_i.
  \]
  Hence
  \begin{align*}
    R(x) & = \dfrac{\inn{Bx, x}}{\norm{x}^2}                                                           &  & \by{6.10.4}     \\
         & = \dfrac{\inn{\sum_{i = 1}^n a_i \lambda_i v_i, \sum_{j = 1}^n a_j v_j}}{\norm{x}^2}        &  & \by{5.1.2}      \\
         & = \dfrac{\sum_{i = 1}^n a_i \lambda_i \inn{v_i, \sum_{j = 1}^n a_j v_j}}{\norm{x}^2}        &  & \by{6.1.1}[a,b] \\
         & = \dfrac{\sum_{i = 1}^n \sum_{j = 1}^n a_i \conj{a_j} \lambda_i \inn{v_i, v_j}}{\norm{x}^2} &  & \by{6.1}[a,b]   \\
         & = \dfrac{\sum_{i = 1}^n a_i \conj{a_i} \lambda_i}{\norm{x}^2}                               &  & \by{6.1.12}     \\
         & = \dfrac{\sum_{i = 1}^n \lambda_i \abs{a_i}^2}{\norm{x}^2}                                  &  & \by{d.0.5}      \\
         & \leq \dfrac{\lambda_1 \sum_{i = 1}^n \abs{a_i}^2}{\norm{x}^2}                               &  & \by{6.2}[b]     \\
         & = \dfrac{\lambda_1 \norm{x}^2}{\norm{x}^2}                                                  &  & \by{ex:6.1.12}  \\
         & = \lambda_1.
  \end{align*}
  It is easy to see that \(R(v_1) = \lambda_1\), so we have demonstrated the first half of the theorem.
  The second half is proved similarly, i.e.,
  \begin{align*}
    R(x) & = \dfrac{\sum_{i = 1}^n \lambda_i \abs{a_i}^2}{\norm{x}^2}    &  & \text{(from the proof above)} \\
         & \geq \dfrac{\lambda_n \sum_{i = 1}^n \abs{a_i}^2}{\norm{x}^2} &  & \by{6.2}[b]                   \\
         & = \dfrac{\lambda_n \norm{x}^2}{\norm{x}^2}                    &  & \by{ex:6.1.12}                \\
         & = \lambda_n.
  \end{align*}
\end{proof}

\begin{cor}\label{6.10.5}
  For any square matrix \(A\), \(\norm{A}\) is finite and, in fact, equals \(\sqrt{\lambda}\), where \(\lambda\) is the largest eigenvalue of \(A^* A\).
\end{cor}

\begin{proof}[\pf{6.10.5}]
  Let \(B\) be the self-adjoint matrix \(A^* A\) (\cref{ex:6.4.18}(a)), and let \(\lambda\) be the largest eigenvalue of \(B\) (\cref{ex:6.4.17}(a)).
  Since, for \(x \neq \zv\),
  \begin{align*}
    0 & \leq \dfrac{\norm{Ax}^2}{\norm{x}^2}   &  & \by{6.2}[b] \\
      & = \dfrac{\inn{Ax, Ax}}{\norm{x}^2}     &  & \by{6.1.9}  \\
      & = \dfrac{\inn{A^* A x, x}}{\norm{x}^2} &  & \by{6.3.4}  \\
      & = \dfrac{\inn{Bx, x}}{\norm{x}^2}                       \\
      & = R(x),                                &  & \by{6.10.4}
  \end{align*}
  it follows from \cref{6.43} that \(\norm{A}^2 = \lambda\).
\end{proof}

\begin{note}
  Observe that the proof of \cref{6.10.5} shows that all the eigenvalues of \(A^* A\) are nonnegative
  (positive semidefinite by \cref{ex:6.4.17}(a)).
\end{note}

\begin{lem}\label{6.10.6}
  For any square matrix \(A\), \(\lambda\) is an eigenvalue of \(A^* A\) iff \(\lambda\) is an eigenvalue of \(A A^*\).
\end{lem}

\begin{proof}[\pf{6.10.6}]
  Let \(\lambda\) be an eigenvalue of \(A^* A\).
  If \(\lambda = 0\), then \(A^* A\) is not invertible.
  Hence \(A\) and \(A^*\) are not invertible (\cref{6.3.6}), so that \(\lambda\) is also an eigenvalue of \(A A^*\).
  The proof of the converse is similar.

  Now suppose that \(\lambda \neq 0\) and there exists \(x \neq \zv\) such that \(A^* A x = \lambda x\).
  Apply \(A\) to both sides to obtain
  \begin{align*}
    A A^* A x & = A (\lambda x) &  & \by{5.1.2}   \\
              & = \lambda (Ax). &  & \by{2.12}[b]
  \end{align*}
  Note that \(Ax \neq \zv\), otherwise we would have \(A^* A x = A^* \zv = \zv \neq \lambda x\).
  So we have that \(\lambda\) is an eigenvalue of \(A A^*\) and \(Ax\) is an eigenvector of \(A A^*\).

  Finally suppose that \(\lambda \neq 0\) and \(A A^* x = \lambda x\) for some \(x \neq \zv\).
  Apply \(A^*\) to both sides to obtain
  \begin{align*}
    A^* A A^* x & = A^* (\lambda x) &  & \by{5.1.2}   \\
                & = \lambda A^* x.  &  & \by{2.12}[b]
  \end{align*}
  Note that \(A^* x \neq \zv\), otherwise we would have \(A A^* x = A \zv = \zv \neq \lambda x\).
  So we have that \(\lambda\) is an eigenvalue of \(A^* A\) and \(A^* x\) is an eigenvector of \(A^* A\).
\end{proof}

\begin{cor}\label{6.10.7}
  Let \(A\) be an invertible matrix.
  Then \(\norm{A^{-1}} = 1 / \sqrt{\lambda}\), where \(\lambda\) is the smallest eigenvalue of \(A^* A\).
\end{cor}

\begin{proof}[\pf{6.10.7}]
  Recall that \(\lambda\) is an eigenvalue of an invertible matrix iff \(\lambda^{-1}\) is an eigenvalue of its inverse (\cref{ex:5.1.8}(b)).

  Now let \(\seq[\geq]{\lambda}{1,,n}\) be the eigenvalues of \(A^* A\), which by \cref{6.10.6} are the eigenvalues of \(A A^*\).
  Then \(\norm{A^{-1}}^2\) equals the largest eigenvalue of \((A A^*)^{-1}\) which equals \(1 / \lambda_n\).
  This is true since
  \begin{align*}
    \pa{A^{-1}}^* A^{-1} & = \pa{A^*}^{-1} A^{-1} &  & \by{ex:6.3.8} \\
                         & = (A A^*)^{-1}.        &  & \by{ex:2.4.4}
  \end{align*}
\end{proof}

\begin{note}
  For many applications, it is only the largest and smallest eigenvalues that are of interest.
  For example, in the case of vibration problems, the smallest eigenvalue represents the lowest frequency at which vibrations can occur.
\end{note}

\begin{cor}\label{6.10.8}
  Now that we know \(\norm{A}\) exists for every square matrix \(A\) (\cref{6.10.5}), we can make use of the inequality \(\norm{Ax} \leq \norm{A} \cdot \norm{x}\), which holds for every \(x\).
\end{cor}

\begin{proof}[\pf{6.10.8}]
  Suppose that \(x = \zv\).
  Then we have
  \begin{align*}
             & \zv = A \zv                                                    &  & \by{2.1.2}[a] \\
    \implies & 0 = \norm{\zv} = \norm{A \zv} = \norm{A} \cdot \norm{\zv} = 0. &  & \by{6.2}[b]
  \end{align*}

  Now suppose that \(x \neq \zv\).
  Then we have
  \begin{align*}
             & \norm{A} = \max_{y \neq \zv} \dfrac{\norm{Ay}}{\norm{y}} \geq \dfrac{\norm{Ax}}{\norm{x}} &  & \by{6.10.3} \\
    \implies & \norm{Ax} \leq \norm{A} \cdot \norm{x}.
  \end{align*}
\end{proof}

\begin{cor}\label{6.10.9}
  Assume in what follows that \(A\) is invertible, \(b \neq \zv\), and \(Ax = b\).
  For a given \(\delta b\), let \(\delta x\) be the vector that satisfies \(A(x + \delta x) = b + \delta b\).
  Prove that \(A(\delta x) = \delta b\), \(\delta x = A^{-1} (\delta b)\), and
  \[
    \dfrac{1}{\norm{A} \cdot \norm{A^{-1}}} \cdot \dfrac{\norm{\delta b}}{\norm{b}} \leq \dfrac{\norm{\delta x}}{\norm{x}} \leq \norm{A} \cdot \norm{A^{-1}} \cdot \dfrac{\norm{\delta b}}{\norm{b}}.
  \]
\end{cor}

\begin{proof}[\pf{6.10.9}]
  Since
  \begin{align*}
             & A (x + \delta x) = b + \delta b                 \\
    \implies & b + A (\delta x) = b + \delta b &  & (Ax = b)   \\
    \implies & A (\delta x) = \delta b         &  & \by{1.1}   \\
    \implies & (\delta x) = A^{-1} (\delta b), &  & \by{2.4.4}
  \end{align*}
  by \cref{6.10.8} we have
  \begin{align*}
    \norm{b}        & = \norm{Ax} \leq \norm{A} \cdot \norm{x};                            \\
    \norm{\delta b} & = \norm{A (\delta x)} \leq \norm{A} \cdot \norm{\delta x};           \\
    \norm{x}        & = \norm{A^{-1} b} \leq \norm{A^{-1}} \cdot \norm{b};                 \\
    \norm{\delta x} & = \norm{A^{-1} (\delta b)} \leq \norm{A^{-1}} \cdot \norm{\delta b}.
  \end{align*}
  Since \(b \neq \zv\), by \cref{2.4} we have \(x \neq \zv\).
  Thus
  \begin{align*}
    \dfrac{\norm{\delta x}}{\norm{x}} & \leq \norm{A} \cdot \dfrac{\norm{\delta x}}{\norm{b}}                     &  & (\norm{b} \leq \norm{A} \cdot \norm{x})                    \\
                                      & \leq \norm{A} \cdot \norm{A^{-1}} \cdot \dfrac{\norm{\delta b}}{\norm{b}} &  & (\norm{\delta x} \leq \norm{A^{-1}} \cdot \norm{\delta b})
  \end{align*}
  and
  \begin{align*}
    \dfrac{\norm{\delta x}}{\norm{x}} & \geq \dfrac{1}{\norm{A^{-1}}} \cdot \dfrac{\norm{\delta x}}{\norm{b}}                 &  & (\norm{x} \leq \norm{A^{-1}} \cdot \norm{b})          \\
                                      & \geq \dfrac{1}{\norm{A} \cdot \norm{A^{-1}}} \cdot \dfrac{\norm{\delta b}}{\norm{b}}. &  & (\norm{\delta b} \leq \norm{A} \cdot \norm{\delta x})
  \end{align*}
\end{proof}

\begin{defn}\label{6.10.10}
  The number \(\norm{A} \cdot \norm{A^{-1}}\) is called the \textbf{condition number} of \(A\) and is denoted \(\cond{A}\).
  It should be noted that the definition of \(\cond{A}\) depends on how the norm of \(A\) is defined.
  There are many reasonable ways of defining the norm of a matrix.
  In fact, the only property needed to establish the inequalities above is that \(\norm{Ax} \leq \norm{A} \cdot \norm{x}\) for all \(x\).
  We summarize these results in \cref{6.44}.
\end{defn}

\begin{thm}\label{6.44}
  For the system \(Ax = b\), where \(A\) is invertible and \(b \neq \zv\), the following statements are true.
  \begin{enumerate}
    \item For any norm \(\norm{\cdot}\), we have
          \[
            \dfrac{1}{\cond{A}} \dfrac{\norm{\delta b}}{\norm{b}} \leq \dfrac{\norm{\delta x}}{\norm{x}} \leq \cond{A} \dfrac{\norm{\delta b}}{\norm{b}}.
          \]
    \item If \(\norm{\cdot}\) is the Euclidean norm, then \(\cond{A} = \sqrt{\dfrac{\lambda_1}{\lambda_n}}\), where \(\lambda_1\) and \(\lambda_n\) are the largest and smallest eigenvalues, respectively, of \(A^* A\).
    \item \(\cond{A} \geq 1\).
  \end{enumerate}
\end{thm}

\begin{proof}[\pf{6.44}(a)]
  See \cref{6.10.9,6.10.10}.
\end{proof}

\begin{proof}[\pf{6.44}(b)]
  See \cref{6.10.5,6.10.7,6.10.10}.
\end{proof}

\begin{proof}[\pf{6.44}(c)]
  This follows from \cref{6.44}(b).
\end{proof}

\begin{note}
  In \cref{ex:6.10.11} we show that \(\cond{A} = 1\) iff \(A\) is a scalar multiple of a unitary or orthogonal matrix.
  Moreover, it can be shown with some work that equality can be obtained in \cref{6.44}(a) by an appropriate choice of \(b\) and \(\delta b\).

  We can see immediately from \cref{6.44}(a) that if \(\cond{A}\) is close to \(1\), then a small relative error in \(b\) forces a small relative error in \(x\).
  If \(\cond{A}\) is large, however, then the relative error in \(x\) may be small even though the relative error in \(b\) is large, or the relative error in \(x\) may be large even though the relative error in \(b\) is small!
  In short, \(\cond{A}\) merely indicates the \emph{potential} for large relative errors.

  We have so far considered only errors in the vector \(b\).
  If there is an error \(\delta A\) in the coefficient matrix of the system \(Ax = b\), the situation is more complicated.
  For example, \(A + \delta A\) may fail to be invertible.
  But under the appropriate assumptions, it can be shown that a bound for the relative error in \(x\) can be given in terms of \(\cond{A}\).
  For example, Charles Cullen (Charles G. Cullen, An Introduction to Numerical Linear Algebra, PWS Publishing Co., Boston 1994, p. 60) shows that if \(A + \delta A\) is invertible, then
  \[
    \dfrac{\norm{\delta x}}{\norm{x + \delta x}} \leq \cond{A} \dfrac{\norm{\delta A}}{\norm{A}}.
  \]

  It should be mentioned that, in practice, one never computes \(\cond{A}\) from its definition, for it would be an unnecessary waste of time to compute \(A^{-1}\) merely to determine its norm.
  In fact, if a computer is used to find \(A^{-1}\), the computed inverse of \(A\) in all likelihood only approximates \(A^{-1}\), and the error in the computed inverse is affected by the size of \(\cond{A}\).
  So we are caught in a vicious circle!
  There are, however, some situations in which a usable approximation of \(\cond{A}\) can be found.
  Thus, in most cases, the estimate of the relative error in \(x\) is based on an estimate of \(\cond{A}\).
\end{note}

\exercisesection

\setcounter{ex}{2}
\begin{ex}\label{ex:6.10.3}
  Prove that if \(B\) is a symmetric real matrix, then \(\norm{B}\) is the eigenvalue of \(B\) with the largest absolute value.
\end{ex}

\begin{proof}[\pf{ex:6.10.3}]
  Since \(B\) is a symmetric real matrix, by \cref{6.4.8} we see that \(B^* = \tp{B} = B\).
  Thus by \cref{6.10.5} \(\norm{B}\) is the largest eigenvalue of \(B^* B = B^2\), which equals to the square of the eigenvalue of \(B\) with the largest absolute value.
\end{proof}

\setcounter{ex}{6}
\begin{ex}\label{ex:6.10.7}
  Let \(B\) be a symmetric matrix.
  Prove that \(\min_{x \neq \zv} R(x)\) equals the smallest eigenvalue of \(B\).
\end{ex}

\begin{proof}[\pf{ex:6.10.7}]
  See \cref{6.43}.
\end{proof}

\begin{ex}\label{ex:6.10.8}
  Prove that if \(\lambda\) is an eigenvalue of \(A A^*\), then \(\lambda\) is an eigenvalue of \(A^* A\).
  This completes the proof of \cref{6.10.6}.
\end{ex}

\begin{proof}[\pf{ex:6.10.8}]
  See \cref{6.10.6}.
\end{proof}

\begin{ex}\label{ex:6.10.9}
  Prove that if \(A\) is an invertible matrix and \(Ax = b\), then
  \[
    \dfrac{1}{\norm{A} \cdot \norm{A^{-1}}} \dfrac{\norm{\delta b}}{\norm{b}} \leq \dfrac{\norm{\delta x}}{\norm{x}}.
  \]
\end{ex}

\begin{proof}[\pf{ex:6.10.9}]
  See \cref{6.10.9}.
\end{proof}

\begin{ex}\label{ex:6.10.10}
  Prove the left inequality of \cref{6.44}(a).
\end{ex}

\begin{proof}[\pf{ex:6.10.10}]
  See \cref{6.10.9}.
\end{proof}

\begin{ex}\label{ex:6.10.11}
  Let \(A \in \ms[n][n][\F]\) be invertible.
  Prove that \(\cond{A} = 1\) iff \(A\) is a scalar multiple of an unitary or orthogonal matrix.
\end{ex}

\begin{proof}[\pf{ex:6.10.11}]
  Let \(\beta\) be the standard ordered basis for \(\vs{F}^n\) over \(\F\), and let \(\lambda_1, \lambda_n \in \F\) be the largest and the smallest eigenvalues, respectively, of \(A^* A\) (\cref{ex:6.4.18}(a)).

  First suppose that \(\cond{A} = 1\).
  Then we have
  \begin{align*}
             & 1 = \cond{A} = \sqrt{\dfrac{\lambda_1}{\lambda_n}}     &  & \by{6.44}[b]      \\
    \implies & \lambda_1 = \lambda_n                                                         \\
    \implies & \text{all eigenvalues of } A^* A \text{ are the same}. &  & \by{ex:6.4.17}[a]
  \end{align*}
  By \cref{6.19,6.20} there exists an unitary (orthogonal) matrix \(Q \in \ms[n][n][\F]\) such that \(Q^* A^* A Q = \lambda_1 I_n\).
  Then we have
  \begin{align*}
             & Q^* A^* A Q = \lambda_1 I_n                                                                                                       \\
    \implies & A^* A = Q (\lambda_1 I_n) Q^*                                                                              &  & \by{ex:6.1.23}[c] \\
             & = \lambda_1 Q I_n Q^*                                                                                      &  & \by{2.12}[b]      \\
             & = \lambda_1 I_n                                                                                            &  & \by{ex:6.1.23}[c] \\
    \implies & I_n = \dfrac{1}{\lambda_1} A^* A = \pa{\dfrac{1}{\sqrt{\lambda_1}} A}^* \pa{\dfrac{1}{\sqrt{\lambda_1}} A} &  & \by{6.3.2}[b]     \\
    \implies & \dfrac{1}{\sqrt{\lambda_1}} A \text{ is unitary (orthogonal)}                                              &  & \by{6.5.9}        \\
    \implies & A \text{ is a multiple of an unitary or orthogonal matrix}.
  \end{align*}

  Now suppose that \(A = \lambda Q\) where \(\lambda \in \F \setminus \set{0}\) and \(Q \in \ms[n][n][\F]\) is unitary (orthogonal).
  Then we have
  \begin{align*}
    A^* A & = \pa{\conj{\lambda} Q^*} (\lambda Q) &  & \by{6.3.2}[b]     \\
          & = \abs{\lambda}^2 I_n                 &  & \by{ex:6.1.23}[c]
  \end{align*}
  and thus \(\abs{\lambda}^2\) is the only eigenvalue of \(A^* A\).
  By \cref{6.10.5,6.10.7} this means \(\norm{A} = \sqrt{\abs{\lambda}^2}\) and \(\norm{A^{-1}} = 1 / \sqrt{\abs{\lambda}^2}\).
  By \cref{6.10.10} this means \(\cond{A} = 1\).
\end{proof}

\begin{ex}\label{ex:6.10.12}
  \begin{enumerate}
    \item Let \(A\) and \(B\) be square matrices that are unitarily equivalent.
          Prove that \(\norm{A} = \norm{B}\).
    \item Let \(\T\) be a linear operator on a finite-dimensional inner product space \(\V\) over \(\F\).
          Define
          \[
            \norm{\T} = \max_{x \neq \zv} \dfrac{\norm{\T(x)}}{\norm{x}}.
          \]
          Prove that \(\norm{\T} = \norm{[\T]_{\beta}}\), where \(\beta\) is any orthonormal basis for \(\V\) over \(\F\).
    \item Let \(\V\) be an infinite-dimensional inner product space with an orthonormal basis \(\set{\seq{v}{1,2,}}\).
          Let \(\T \in \ls(\V)\) such that \(\T(v_k) = k v_k\).
          Prove that \(\norm{\T}\) (defined in (b)) does not exist.
  \end{enumerate}
\end{ex}

\begin{proof}[\pf{ex:6.10.12}(a)]
  By \cref{6.5.13} there exists an unitary matrix \(Q\) such that \(B = Q^* A Q\).
  Then we have
  \begin{align*}
    \norm{B} & = \norm{Q^* A Q}                                                        \\
             & = \max_{x \neq \zv} \dfrac{\norm{Q^* A Q x}}{\norm{x}} &  & \by{6.10.3} \\
             & = \max_{x \neq \zv} \dfrac{\norm{A Q x}}{\norm{Q x}}   &  & \by{6.5.1}  \\
             & = \max_{y \neq \zv} \dfrac{\norm{A y}}{\norm{y}}       &  & \by{2.5}    \\
             & = \norm{A}.                                            &  & \by{6.10.3}
  \end{align*}
\end{proof}

\begin{proof}[\pf{ex:6.10.12}(b)]
  Let \(\dim(\V) = n\).
  Let \(\inn{\cdot, \cdot}\) be an inner product on \(\V\) over \(\F\) such that \(\norm{\cdot}^2 = \inn{\cdot, \cdot}\).
  Let \(\inn{\cdot, \cdot}'\) be the standard inner product on \(\vs{F}^n\) over \(\F\) and let \((\norm{\cdot}')^2 = \inn{\cdot, \cdot}'\).
  Let \(\beta = \set{\seq{v}{1,,n}}\) be an orthonormal basis for \(\V\) over \(\F\) with respect to \(\inn{\cdot, \cdot}\).
  Let \(\phi_{\beta} \in \ls(\V)\) be defined as in \cref{2.4.11}.
  Since
  \begin{align*}
    \forall x \in \V, \norm{x}^2 & = \norm{\sum_{i = 1}^n \inn{x, v_i} v_i}^2         &  & \by{6.5}       \\
                                 & = \sum_{i = 1}^k \abs{\inn{x, v_i}}^2 \norm{v_i}^2 &  & \by{ex:6.1.12} \\
                                 & = \sum_{i = 1}^k \abs{\inn{x, v_i}}^2              &  & \by{6.1.12}    \\
                                 & = \sum_{i = 1}^k \abs{(\phi_{\beta}(x))_i}^2       &  & \by{2.4.11}    \\
                                 & = \inn{\phi_{\beta}(x), \phi_{\beta}(x)}'          &  & \by{6.1.2}     \\
                                 & = (\norm{\phi_{\beta}(x)}')^2,                     &  & \by{6.1.9}
  \end{align*}
  we have
  \begin{align*}
    \norm{\T} & = \max_{x \in \V, x \neq \zv} \dfrac{\norm{\T(x)}}{\norm{x}}                                        &  & \by{6.10.3}                   \\
              & = \max_{x \in \V, x \neq \zv} \dfrac{\norm{\phi_{\beta}(\T(x))}'}{\norm{\phi_{\beta}(x)}'}          &  & \text{(from the proof above)} \\
              & = \max_{x \in \V, x \neq \zv} \dfrac{\norm{[\T]_{\beta} \phi_{\beta}(x)}'}{\norm{\phi_{\beta}(x)}'} &  & \by{2.4.12}                   \\
              & = \max_{x \in \vs{F}^n, x \neq \zv} \dfrac{\norm{[\T]_{\beta} x}'}{\norm{x}'}                                                          \\
              & = \norm{[\T]_{\beta}}.                                                                              &  & \by{6.10.3}
  \end{align*}
\end{proof}

\begin{proof}[\pf{ex:6.10.12}(c)]
  Since
  \begin{align*}
             & \forall k \in \Z^+, \T(v_k) = k v_k                                                                                                                                \\
    \implies & \forall k \in \Z^+, \dfrac{\norm{\T(v_k)}}{\norm{v_k}} = \dfrac{\norm{k v_k}}{\norm{v_k}} = \dfrac{\abs{k} \norm{v_k}}{\norm{v_k}} = \abs{k} = k, &  & \by{6.2}[a]
  \end{align*}
  we know that \(\max_{x \neq \zv} \dfrac{\norm{\T(x)}}{\norm{x}} = \infty\), thus \(\norm{\T}\) is not well-defined.
\end{proof}

\begin{ex}\label{ex:6.10.13}
  Let \(A \in \ms[n][n][\F]\) be of rank \(r\) with the nonzero singular values \(\seq[\geq]{\sigma}{1,,r}\).
  Prove each of the following results.
  \begin{enumerate}
    \item \(\norm{A} = \sigma_1\).
    \item \(\norm{A^{\dag}} = \dfrac{1}{\sigma_r}\).
    \item If \(A\) is invertible (and hence \(r = n\)), then \(\cond{A} = \dfrac{\sigma_1}{\sigma_n}\).
  \end{enumerate}
\end{ex}

\begin{proof}[\pf{ex:6.10.13}(a)]
  By \cref{6.26} we know that \(\sigma_1^2\) is the largest eigenvalue of \(A^* A\).
  Thus by \cref{6.10.5} we have \(\norm{A} = \sigma_1\).
\end{proof}

\begin{proof}[\pf{ex:6.10.13}(b)]
  By \cref{6.29} we know that the nonzero singular values of \(A^{\dag}\) are \(\dfrac{1}{\sigma_r} \geq \dots \geq \dfrac{1}{\sigma_1}\).
  Thus by \cref{ex:6.10.13}(a) we know that \(\norm{A^{\dag}} = \dfrac{1}{\sigma_r}\).
\end{proof}

\begin{proof}[\pf{ex:6.10.13}(c)]
  By \cref{6.26} we know that \(\sigma_n^2\) is the smallest eigenvalue of \(A^* A\).
  Thus by \cref{6.10.7} we have \(\norm{A^{-1}} = \dfrac{1}{\sigma_n}\) and by \cref{6.10.10} \(\cond{A} = \dfrac{\sigma_1}{\sigma_n}\).
\end{proof}

\section{The Geometry of Orthogonal Operators}\label{sec:6.11}

\begin{note}
  By \cref{6.22}, any rigid motion on a finite-dimensional real inner product space is the composite of an orthogonal operator and a translation.
  Thus, to understand the geometry of rigid motions thoroughly, we must analyze the structure of orthogonal operators.
  Such is the aim of this section.
  We show that any orthogonal operator on a finite-dimensional real inner product space is the composite of rotations and reflections.
\end{note}

\begin{defn}\label{6.11.1}
  Let \(\T\) be a linear operator on a finite-dimensional real inner product space \(\V\).
  The operator \(\T\) is called a \textbf{rotation} if \(\T\) is the identity on \(\V\) or if there exists a two-dimensional subspace \(\W\) of \(\V\) over \(\R\), an orthonormal basis \(\beta = \set{\seq{x}{1,2}}\) for \(\W\) over \(\R\), and a real number \(\theta\) such that
  \[
    \T(x_1) = \cos(\theta) x_1 + \sin(\theta) x_2, \quad \T(x_2) = -\sin(\theta) x_1 + \cos(\theta) x_2,
  \]
  and \(\T(y) = y\) for all \(y \in \W^{\perp}\).
  In this context, \(\T\) is called a \textbf{rotation of \(\W\) about \(\W^{\perp}\)}.
  The subspace \(\W^{\perp}\) is called the \textbf{axis of rotation}.
\end{defn}

\begin{defn}\label{6.11.2}
  Let \(\T\) be a linear operator on a finite-dimensional real inner product space \(\V\).
  The operator \(\T\) is called a \textbf{reflection} if there exists a one-dimensional subspace \(\W\) of \(\V\) over \(\R\) such that \(\T(x) = -x\) for all \(x \in \W\) and \(\T(y) = y\) for all \(y \in \W^{\perp}\).
  In this context, \(\T\) is called a \textbf{reflection of \(\V\) about \(\W^{\perp}\)}.
\end{defn}

\begin{note}
  It should be noted that rotations and reflections (or composites of these) are orthogonal operators (see \cref{ex:6.11.2}).
  The principal aim of this section is to establish that the converse is also true, that is, any orthogonal operator on a finite-dimensional real inner product space is the composite of rotations and reflections.
\end{note}

\begin{eg}\label{6.11.3}
  A Characterization of Orthogonal Operators on a One-Dimensional Real Inner Product Space.

  Let \(\T\) be an orthogonal operator on a one-dimensional real inner product space \(\V\).
  Choose any nonzero vector \(x\) in \(\V\).
  Then \(\V = \spn{\set{x}}\), and so \(\T(x) = \lambda x\) for some \(\lambda \in \R\).
  Since \(\T\) is orthogonal and \(\lambda\) is an eigenvalue of \(\T\), \(\lambda = \pm 1\) (see \cref{6.2}(a) and \cref{6.5.1}).
  If \(\lambda = 1\), then \(\T\) is the identity on \(\V\), and hence \(\T\) is a rotation (\cref{6.11.1}).
  If \(\lambda = -1\), then \(\T(x) = -x\) for all \(x \in \V\);
  so \(\T\) is a reflection of \(\V\) about \(\V^{\perp} = \set{\zv}\) (\cref{6.2.11,6.11.2}).
  Thus \(\T\) is either a rotation or a reflection.
  Note that in the first case, \(\det(\T) = 1\), and in the second case, \(\det(\T) = -1\) (\cref{ex:5.1.7}).
\end{eg}

\begin{cor}\label{6.11.4}
  Let \(\V\) be an \(n\)-dimensional vector space over \(\R\).
  Let \(\T \in \ls(\V)\) be a rotation.
  Then there exists an orthonormal basis \(\gamma\) for \(\V\) over \(\R\) and a \(\theta \in \R\) such that
  \[
    [\T]_{\gamma} = \begin{pmatrix}
      \cos(\theta) & -\sin(\theta) & \zm       \\
      \sin(\theta) & \cos(\theta)  & \zm       \\
      \zm          & \zm           & I_{n - 2}
    \end{pmatrix}.
  \]
\end{cor}

\begin{proof}[\pf{6.11.4}]
  Following the notation of \cref{6.11.1}, let \(\gamma = \set{\seq{x}{1,2}, \seq{v}{1,,n-2}}\) be an orthonormal basis for \(\V\) over \(\R\), where \(\set{\seq{v}{1,,n-2}}\) is an orthonormal basis for \(\W^{\perp}\).
  By \cref{6.5} such \(\gamma\) must exist.
  The proof follows from \cref{2.2.4,6.11.1}.
\end{proof}

\begin{thm}\label{6.45}
  Let \(\T\) be an orthogonal operator on a two-dimensional real inner product space \(\V\).
  Then \(\T\) is either a rotation or a reflection.
  Furthermore, \(\T\) is a rotation iff \(\det(\T) = 1\), and \(\T\) is a reflection iff \(\det(\T) = -1\).
\end{thm}

\begin{proof}[\pf{6.45}]
  Let \(\inn{\cdot, \cdot}\) be an inner product on \(\V\) over \(\R\) and let \(\inn{\cdot, \cdot}'\) be the standard inner product on \(\R^2\) over \(\R\).
  Let \(\norm{\cdot}\) be a norm on \(\V\) over \(\R\) such that \(\norm{\cdot}^2 = \inn{\cdot, \cdot}\).
  Let \(\norm{\cdot}'\) be a norm on \(\R^2\) over \(\R\) such that \((\norm{\cdot}')^2 = \inn{\cdot, \cdot}'\).
  Let \(\T \in \ls(\V)\) be orthogonal with respect to \(\norm{\cdot}\) and let \(\beta = \set{\seq{v}{1,2}}\) be an orthonormal basis with respect to \(\inn{\cdot, \cdot}\) for \(\V\) over \(\R\).
  Let \(\gamma\) be the standard ordered basis for \(\R^2\) over \(\R\).
  Define \(\phi_{\beta} \in \ls(\V, \R^2)\) as in \cref{2.4.11}.

  Because \(\T\) is an orthogonal operator, \(\T(\beta)\) is an orthonormal basis for \(\V\) over \(\R\) by \cref{6.18}(c).
  Since
  \begin{align*}
    1 & = \norm{v_1}^2                                      &  & \by{6.1.12}    \\
      & = \norm{\T(v_1)}^2                                  &  & \by{6.5.1}     \\
      & = \inn{\T(v_1), \T(v_1)}                            &  & \by{6.1.9}     \\
      & = \inn{\phi_{\beta} \T(v_1), \phi_{\beta} \T(v_1)}' &  & \by{ex:6.2.15} \\
      & = (\norm{\phi_{\beta} \T(v_1)}')^2,                 &  & \by{6.1.9}
  \end{align*}
  we know that \(\phi_{\beta} \T(v_1)\) is a unit vector and there is an unique angle \(\theta \in [0, 2 \pi)\) such that \(\phi_{\beta} \T(v_1) = (\cos(\theta), \sin(\theta))\).
  Similarly \(\phi_{\beta} \T(v_2)\) is a unit vector and is orthogonal to \(\phi_{\beta} \T(v_1)\) (\cref{ex:6.2.15}), there are only two possible choices for \(\phi_{\beta} \T(v_2)\).
  Either
  \[
    \phi_{\beta} \T(v_2) = (-\sin(\theta), \cos(\theta)) \quad \text{or} \quad \phi_{\beta} \T(v_2) = (\sin(\theta), -\cos(\theta)).
  \]
  First, suppose that \(\phi_{\beta} \T(v_2) = (-\sin(\theta), \cos(\theta))\).
  Then
  \begin{align*}
    \begin{pmatrix}
      \cos(\theta) & -\sin(\theta) \\
      \sin(\theta) & \cos(\theta)
    \end{pmatrix} & = [\phi_{\beta} \T]_{\beta}^{\gamma}           &  & \by{2.2.4}                   \\
                                    & = [\phi_{\beta}]_{\beta}^{\gamma} [\T]_{\beta} &  & \by{2.11}  \\
                                    & = I_2 [\T]_{\beta}                             &  & \by{2.2.4} \\
                                    & = [\T]_{\beta}.                                &  & \by{2.3.8}
  \end{align*}
  It follows from \cref{6.11.1} that \(\T\) is a rotation of \(\V\) about \(\V^{\perp} = \set{\zv}\) (\cref{6.2.11}).
  Also
  \begin{align*}
    \det(\T) & = \det([\T]_{\beta})                       &  & \by{ex:5.1.7} \\
             & = (\cos(\theta))^2 + (\sin(\theta))^2 = 1. &  & \by{4.1.1}
  \end{align*}
  Now suppose that \(\phi_{\beta} \T(v_2) = (\sin(\theta), -\cos(\theta))\).
  Then
  \[
    \begin{pmatrix}
      \cos(\theta) & \sin(\theta)  \\
      \sin(\theta) & -\cos(\theta)
    \end{pmatrix} = [\phi_{\beta} \T]_{\beta}^{\gamma} = [\T]_{\beta}.
  \]
  It follows from \cref{6.11.2} that \(\T\) is a reflection on \(\V\) about
  \[
    \W^{\perp} = \spn{\set{\sin(\theta) v_1 + (1 - \cos(\theta)) v_2}}
  \]
  where
  \[
    \W = \spn{\set{(\cos(\theta) - 1) v_1 + \sin(\theta) v_2}}.
  \]
  Furthermore,
  \begin{align*}
    \det(\T) & = \det([\T]_{\beta})                          &  & \by{ex:5.1.7} \\
             & = - (\cos(\theta))^2 - (\sin(\theta))^2 = -1. &  & \by{4.1.1}
  \end{align*}
\end{proof}

\begin{cor}\label{6.11.5}
  Let \(\V\) be a two-dimensional real inner product space.
  The composite of a reflection and a rotation on \(\V\) is a reflection on \(\V\).
\end{cor}

\begin{proof}[\pf{6.11.5}]
  If \(\T_1\) is a reflection on \(\V\) and \(\T_2\) is a rotation on \(\V\), then by \cref{6.45}, \(\det(\T_1) = 1\) and \(\det(\T_2) = -1\).
  Let \(\T = \T_2 \T_1\) be the composite.
  Since \(\T_2\) and \(\T_1\) are orthogonal, so is \(\T\) (\cref{ex:6.5.3}).
  Moreover,
  \begin{align*}
    \det(\T) & = \det(\T_2) \cdot \det(\T_1) &  & \by{ex:5.1.7}[d] \\
             & = -1.                         &  & \by{6.45}
  \end{align*}
  Thus, by \cref{6.45}, \(\T\) is a reflection.
  The proof for \(\T_1 \T_2\) is similar, i.e.,
  \begin{align*}
    \det(\T_1 \T_2) & = \det(\T_1) \cdot \det(\T_2) &  & \by{ex:5.1.7}[d] \\
                    & = -1.
  \end{align*}
\end{proof}

\begin{lem}\label{6.11.6}
  If \(\T\) is a linear operator on a nonzero finite-dimensional real vector space \(\V\), then there exists a \(\T\)-invariant subspace \(\W\) of \(\V\) over \(\R\) such that \(1 \leq \dim(\W) \leq 2\).
\end{lem}

\begin{proof}[\pf{6.11.6}]
  Fix an ordered basis \(\beta = \set{\seq{y}{1,,n}}\) for \(\V\) over \(\R\), and let \(A = [\T]_{\beta}\).
  Let \(\phi_{\beta} \in \ls(\V, \R^n)\) be the isomorphism defined by \cref{2.4.11}.
  By \cref{2.4.12} \(\L_A \phi_{\beta} = \phi_{\beta} \T\).
  As a consequence, it suffices to show that there exists an \(\L_A\)-invariant subspace \(\vs{Z}\) of \(\R^n\) such that \(1 \leq \dim(\vs{Z}) \leq 2\).
  If we then define \(\W = \phi_{\beta}^{-1}(\vs{Z})\), it follows that \(\W\) satisfies the conclusions of the lemma (see \cref{ex:6.11.13}).

  The matrix \(A\) can be considered as an \(n \times n\) matrix over \(\C\) and, as such, can be used to define a linear operator \(\U\) on \(\C^n\) by \(\U(v) = Av\).
  Since \(\U\) is a linear operator on a finite-dimensional vector space over \(\C\), it has an eigenvalue \(\lambda \in \C\) (\cref{5.2,d.4}).
  Let \(x \in \C^n\) be an eigenvector corresponding to \(\lambda\).
  We may write \(\lambda = \lambda_1 + i \lambda_2\), where \(\lambda_1\) and \(\lambda_2\) are real, and
  \[
    x = \begin{pmatrix}
      a_1 + i b_1 \\
      \vdots      \\
      a_n + i b_n
    \end{pmatrix},
  \]
  where the \(a_i\)'s and \(b_i\)'s are real.
  Thus, setting
  \[
    x_1 = \begin{pmatrix}
      a_1    \\
      \vdots \\
      a_n
    \end{pmatrix} \quad \text{and} \quad x_2 = \begin{pmatrix}
      b_1    \\
      \vdots \\
      b_n
    \end{pmatrix},
  \]
  we have \(x = x_1 + i x_2\), where \(x_1\) and \(x_2\) have real entries.
  Note that at least one of \(x_1\) or \(x_2\) is nonzero since \(x \neq \zv\).
  Hence
  \begin{align*}
    \U(x) & = \lambda x                                                            \\
          & = (\lambda_1 + i \lambda_2) (x_1 + i x_2)                              \\
          & = (\lambda_1 x_1 - \lambda_2 x_2) + i (\lambda_1 x_2 + \lambda_2 x_1).
  \end{align*}
  Similarly,
  \[
    \U(x) = A (x_1 + i x_2) = A x_1 + i A x_2.
  \]
  Comparing the real and imaginary parts of these two expressions for \(\U(x)\), we conclude that
  \[
    A x_1 = \lambda_1 x_1 - \lambda_2 x_2 \quad \text{and} \quad A x_2 = \lambda_1 x_2 + \lambda_2 x_1.
  \]
  Finally, let \(\vs{Z} = \spn{\set{\seq{x}{1,2}}}\), the span being taken as a subspace of \(\R^n\) over \(\R\).
  Since \(x_1 \neq \zv\) or \(x_2 \neq \zv\), \(\vs{Z}\) is a nonzero subspace.
  Thus \(1 \leq \dim(\vs{Z}) \leq 2\), and the preceding pair of equations shows that \(\vs{Z}\) is \(\L_A\)-invariant.
\end{proof}

\begin{thm}\label{6.46}
  Let \(\T\) be an orthogonal operator on a nonzero finite-dimensional real inner product space \(\V\).
  Then there exists a collection of pairwise orthogonal \(\T\)-invariant subspaces \(\set{\seq{\W}{1,,m}}\) of \(\V\) such that
  \begin{enumerate}
    \item \(1 \leq \dim(\W_i) \leq 2\) for \(i \in \set{1, \dots, m}\).
    \item \(\V = \seq[\oplus]{\W}{1,,m}\).
  \end{enumerate}
\end{thm}

\begin{proof}[\pf{6.46}]
  The proof is by mathematical induction on \(\dim(\V)\).
  If \(\dim(\V) = 1\), the result is obvious.
  So assume that the result is true whenever \(\dim(\V) = n\) for some fixed integer \(n \geq 1\).

  Suppose \(\dim(\V) = n + 1\).
  By \cref{6.11.6}, there exists a \(\T\)-invariant subspace \(\W_1\) of \(\V\) such that \(1 \leq \dim(\W_1) \leq 2\).
  If \(\W_1 = \V\), the result is established.
  Otherwise, \(\W_1^{\perp} \neq \set{\zv}\) (\cref{6.2.11}).
  By \cref{ex:6.11.14}, \(\W_1^{\perp}\) is \(\T\)-invariant and the restriction of \(\T\) to \(\W_1^{\perp}\) is orthogonal.
  Since \(\dim(\W_1^{\perp}) \leq n\), we may apply the induction hypothesis to \(\T_{\W_1^{\perp}}\) and conclude that there exists a collection of pairwise orthogonal \(\T\)-invariant subspaces \(\set{\seq{\W}{2,,m}}\) of \(\W_1^{\perp}\) such that \(1 \leq \dim(\W_i) \leq 2\) for \(i \in \set{2, \dots, m}\) and \(\W_1^{\perp} = \seq[\oplus]{\W}{2,,m}\).
  Thus \(\set{\seq{\W}{1,,m}}\) is pairwise orthogonal, and by \cref{ex:6.2.13}(d),
  \[
    \V = \W_1 \oplus \W_1^{\perp} = \seq[\oplus]{\W}{1,,m}.
  \]
\end{proof}

\begin{note}
  Applying \cref{6.11.3,6.45} in the context of \cref{6.46}, we conclude that the restriction of \(\T\) to \(\W_i\) is either a rotation or a reflection for each \(i \in \set{1, \dots, m}\).
  Thus, in some sense, \(\T\) is composed of rotations and reflections.
  Unfortunately, very little can be said about the uniqueness of the decomposition of \(\V\) in \cref{6.46}.
  For example, the \(\W_i\)'s, the number \(m\) of \(\W_i\)'s, and the number of \(\W_i\)'s for which \(\T_{\W_i}\) is a reflection are not unique.
  Although the number of \(\W_i\)'s for which \(\T_{\W_i}\) is a reflection is not unique, whether this number is even or odd is an intrinsic property of \(\T\).
  Moreover, we can always decompose \(\V\) so that \(\T_{\W_i}\) is a reflection for at most one \(\W_i\).
  These facts are established in \cref{6.47}.
\end{note}

\begin{thm}\label{6.47}
  Let \(\T, \V, \seq{\W}{1,,m}\) be as in \cref{6.46}.
  \begin{enumerate}
    \item The number of \(\W_i\)'s for which \(\T_{\W_i}\) is a reflection is even or odd according to whether \(\det(\T) = 1\) or \(\det(\T) = -1\).
    \item It is always possible to decompose \(\V\) as in \cref{6.46} so that the number of \(\W_i\)'s for which \(\T_{\W_i}\) is a reflection is zero or one according to whether \(\det(\T) = 1\) or \(\det(\T) = -1\).
          Furthermore, if \(\T_{\W_i}\) is a reflection, then \(\dim(\W_i) = 1\).
  \end{enumerate}
\end{thm}

\begin{proof}[\pf{6.47}(a)]
  Let \(r\) denote the number of \(\W_i\)'s in the decomposition for which \(\T_{\W_i}\) is a reflection.
  Then we have
  \begin{align*}
    \det(\T) & = \det(\T_{\W_1}) \cdots \det(\T_{\W_m}) &  & \by{ex:6.11.15}  \\
             & = 1^{m - r} \cdot (-1)^r                 &  & \by{6.11.3,6.45} \\
             & = (-1)^r.
  \end{align*}
\end{proof}

\begin{proof}[\pf{6.47}(b)]
  Let \(E = \set{x \in \V : \T(x) = -x}\);
  then \(E\) is a \(\T\)-invariant subspace of \(\V\) over \(\R\).
  If \(\W = E^{\perp}\), then \(\W\) is \(\T\)-invariant (\cref{ex:6.11.14}(b)).
  So by applying \cref{6.46} to \(\T_{\W}\), we obtain a collection of pairwise orthogonal \(\T\)-invariant subspaces \(\set{\seq{\W}{1,,k}}\) of \(\W\) such that \(\W = \seq[\oplus]{\W}{1,,k}\) and for \(i \in \set{1, \dots, k}\), the dimension of each \(\W_i\) is either \(1\) or \(2\).
  Observe that, for each \(i \in \set{1, \dots, k}\), \(\T_{\W_i}\) is a rotation.
  For otherwise, if \(\T_{\W_i}\) is a reflection, there exists a nonzero \(x \in \W_i\) for which \(\T(x) = -x\).
  But then, \(x \in \W_i \cap E \subseteq E^{\perp} \cap E = \set{\zv}\), a contradiction.
  If \(E = \set{\zv}\), the result follows.
  Otherwise, choose an orthonormal basis \(\beta\) for \(E\) over \(\R\) containing \(p\) vectors (\(p > 0\)).
  It is possible to decompose \(\beta\) into a pairwise disjoint union \(\beta = \seq[\cup]{\beta}{1,,r}\) such that each \(\beta_i\) contains exactly two vectors for \(i \in \set{1, \dots, r - 1}\), and \(\beta_r\) contains two vectors if \(p\) is even and one vector if \(p\) is odd.
  For each \(i \in \set{1, \dots, r}\), let \(\W_{k + i} = \spn{\beta_i}\).
  Then, clearly, \(\set{\seq{\W}{1,,k,,k+r}}\) is pairwise orthogonal, and
  \[
    \V = \seq[\oplus]{\W}{1,,k,,k+r}.
  \]
  Moreover, if any \(\beta_i\) contains two vectors, then
  \begin{align*}
    \det(\T_{\W_{k + i}}) & = \det([\T_{\W_{k + i}}]_{\beta_i}) &  & \by{ex:5.1.7} \\
                          & = \det\begin{pmatrix}
                                    -1 & 0  \\
                                    0  & -1
                                  \end{pmatrix}               &  & (\T(x) = -x)    \\
                          & = 1.                                &  & \by{4.1.1}
  \end{align*}
  So \(\T_{\W_{k + i}}\) is a rotation, and hence \(\T_{\W_j}\) is a rotation for \(j \in \set{1, \dots, k + r - 1}\).
  If \(\beta_r\) consists of one vector, then \(\dim(\W_{k + r}) = 1\) and
  \begin{align*}
    \det(\T_{\W_{k + r}}) & = \det([\T_{\W_{k + r}}]_{\beta_r}) &  & \by{ex:5.1.7} \\
                          & = \det(-1)                          &  & (\T(x) = -x)  \\
                          & = -1.                               &  & \by{4.2.2}
  \end{align*}
  Thus \(\T_{\W_{k + r}}\) is a reflection by \cref{6.46}, and we conclude that the decomposition
  \[
    \V = \seq[\oplus]{\W}{1,,k,,k+r}
  \]
  satisfies the condition of (b).
\end{proof}

\begin{cor}\label{6.11.7}
  Let \(\T\) be an orthogonal operator on a finite-dimensional real inner product space \(\V\).
  Then there exists a collection \(\set{\seq{\T}{1,,m}}\) of orthogonal operators on \(\V\) such that the following statements are true.
  \begin{enumerate}
    \item For each \(i \in \set{1, \dots, m}\), \(\T_i\) is either a reflection or a rotation.
    \item For at most one \(i \in \set{1, \dots, m}\), \(\T_i\) is a reflection.
    \item \(\T_i \T_j = \T_j \T_i\) for all \(i, j \in \set{1, \dots, m}\).
    \item \(\T = \seq[]{\T}{1,,m}\).
    \item \[
            \det(\T) = \begin{dcases}
              1  & \text{if } \T_i \text{ is a rotation for each } i \in \set{1, \dots, m} \\
              -1 & \text{otherwise}.
            \end{dcases}
          \]
  \end{enumerate}
\end{cor}

\begin{proof}[\pf{6.11.7}]
  As in the proof of \cref{6.47}(b), we can write
  \[
    \V = \seq[\oplus]{\W}{1,,m},
  \]
  where \(\T_{\W_i}\) is a rotation for \(i \in \set{1, \dots, m - 1}\).
  For each \(i \in \set{1, \dots, m}\), define \(\T_i : \V \to \V\) by
  \[
    \T_i(\seq[+]{x}{1,,m}) = \seq[+]{x}{1,,i-1} + \T(x_i) + \seq[+]{x}{i+1,,m},
  \]
  where \(x_j \in \W_j\) for all \(j \in \set{1, \dots, m}\).

  First we show that \(\T_i\) is an orthogonal operator on \(\V\) for all \(i \in \set{1, \dots, m}\).
  Clearly \(\T_i \in \ls(\V)\).
  Since
  \begin{align*}
             & \forall x \in \V, \exists \tuple{x}{1,,m} \in \seq[\times]{\W}{1,,m} : x = \seq[+]{x}{1,,m}                  &  & \by{5.2.7}     \\
    \implies & \forall x \in \V, \norm{\T_i(x)}^2                                                                                               \\
             & = \norm{\seq[+]{x}{1,,i-1} + \T(x_i) + \seq[+]{x}{i+1,,m}}^2                                                                     \\
             & = \norm{x_1}^2 + \cdots + \norm{x_{i - 1}}^2 + \norm{\T(x_i)}^2 + \norm{x_{i + 1}}^2 + \cdots + \norm{x_m}^2 &  & \by{ex:6.1.10} \\
             & = \norm{x_1}^2 + \cdots + \norm{x_{i - 1}}^2 + \norm{x_i}^2 + \norm{x_{i + 1}}^2 + \cdots + \norm{x_m}^2     &  & \by{6.5.1}     \\
             & = \norm{\seq[+]{x}{1,,m}}^2                                                                                  &  & \by{ex:6.1.10} \\
             & = \norm{x}^2                                                                                                 &  & \by{5.2.7}     \\
    \implies & \forall x \in \V, \norm{\T(x)} = \norm{x},                                                                   &  & \by{6.2}[b]
  \end{align*}
  by \cref{6.5.1} we know that \(\T_i\) is orthogonal for all \(i \in \set{1, \dots, m}\).

  Next we claim that \(\T_i\) is a rotation or a reflection according to whether \(\T_{\W_i}\) is a rotation or a reflection.
  We split into two cases:
  \begin{itemize}
    \item If \(\T_{\W_i}\) is a rotation, then by \cref{6.11.1} there exists an orthonormal basis \(\beta = \set{\seq{v}{1,2}}\) for \(\W_i\) over \(\R\), and a \(\theta \in \R\) such that
          \begin{align*}
            \T_{\W_i}(v_1) & = \cos(\theta) v_1 + \sin(\theta) v_2;  \\
            \T_{\W_i}(v_2) & = -\sin(\theta) v_1 + \cos(\theta) v_2.
          \end{align*}
          Note that \(\dim(\W_i) = 2\) and thus \(\W_i^{\perp} = \set{\zv}\) when we only consider vectors in \(\W_i\) itself.
          Since \(\seq{\W}{1,,k}\) are pairwise orthogonal, we see that
          \[
            \W_i^{\perp} = \pa{\sum_{j = 1}^m \W_j} \setminus \W_i \quad \text{and} \quad \forall x \in \pa{\sum_{j = 1}^m \W_j} \setminus \W_i, \T_i(x) = x.
          \]
          Thus by \cref{6.11.1} \(\T_i\) is rotation when \(\T_{\W_i}\) is a rotation.
    \item If \(\T_{\W_i}\) is a reflection, then by \cref{6.47}(b) we have \(\dim(\W_i) = 1\).
          By \cref{6.11.2} we see that \(\T_{\W_i}(x) = -x\) for all \(x \in \W_i\).
          Note that \(\dim(\W_i) = 1\) and thus \(\W_i^{\perp} = \set{\zv}\) when we only consider vectors in \(\W_i\) itself.
          Since \(\seq{\W}{1,,k}\) are pairwise orthogonal, we see that
          \[
            \W_i^{\perp} = \pa{\sum_{j = 1}^m \W_j} \setminus \W_i \quad \text{and} \quad \forall x \in \pa{\sum_{j = 1}^m \W_j} \setminus \W_i, \T_i(x) = x.
          \]
          Thus by \cref{6.11.2} \(\T_i\) is reflection when \(\T_{\W_i}\) is a reflection.
  \end{itemize}
  This establishes (a).

  Next we proof (b).
  Since there is at most one \(\T_{\W_i}\) is a reflection by \cref{6.47}(b), we know that at most one \(\T_i\) is a reflection.
  This establish (b).

  Next we proof (c).
  Thus case for \(i = j\) is trivial.
  So we proof the case \(i \neq j\).
  Without the loss of generality, suppose that \(i < j\).
  Since
  \begin{align*}
    \forall x \in \V, (\T_i \T_j)(x) & = \T_i(\T_j(x))                                                                   \\
                                     & = \T_i(\T_j(\seq[+]{x}{1,,m}))                                 &  & \by{5.2.7}    \\
                                     & = \T_i(\T_j(x_1)) + \cdots + \T_i(\T_j(x_i))                                      \\
                                     & \quad + \cdots \T_i(\T_j(x_j)) + \cdots + \T_i(\T_j(x_m))      &  & \by{2.1.1}[a] \\
                                     & = x_1 + \cdots + \T_i(x_i) + \cdots + \T_j(x_j) + \cdots + x_m                    \\
                                     & = \T_j(\T_i(x_1)) + \cdots + \T_j(\T_i(x_i))                                      \\
                                     & \quad + \cdots \T_j(\T_i(x_j)) + \cdots + \T_j(\T_i(x_m))      &  & \by{2.1.1}[a] \\
                                     & = \T_j(\T_i(\seq[+]{x}{1,,m}))                                 &  & \by{2.1.1}[a] \\
                                     & = \T_j(\T_i(x))                                                &  & \by{5.2.7}    \\
                                     & = (\T_j \T_i)(x),
  \end{align*}
  we see that \(\T_i \T_j = \T_j \T_i\).
  This establish (c).

  Next we proof (d).
  Since
  \begin{align*}
    \forall x \in \V, (\seq[]{\T}{1,,m})(x) & = (\seq[]{\T}{1,,m-1})(\T_m(x))                                                  \\
                                            & = (\seq[]{\T}{1,,m-1})\pa{\sum_{i = 1}^{m - 1} x_i + \T(x_m)} &  & \by{5.2.7}    \\
                                            & = (\seq[]{\T}{1,,m-1})\pa{\sum_{i = 1}^{m - 1} x_i} + \T(x_m) &  & \by{2.1.1}[a] \\
                                            & = \sum_{i = 1}^m \T(x_i)                                                         \\
                                            & = \T\pa{\sum_{i = 1}^m x_i}                                   &  & \by{2.1.1}[a] \\
                                            & = \T(x),                                                      &  & \by{5.2.7}
  \end{align*}
  we see that \(\T = \seq[]{\T}{1,,m}\).
  This establish (d).

  Finally we proof (e).
  By \cref{ex:5.1.7}(d) and \cref{6.11.7}(d) we have \(\det(\T) = \det(\T_1) \cdots \det(\T_m)\).
  The rest follows from \cref{6.47}(b).
\end{proof}

\begin{eg}\label{6.11.8}
  Orthogonal Operators on a Three-Dimensional Real Inner Product Space.

  Let \(\T\) be an orthogonal operator on a three-dimensional real inner product space \(\V\).
  We show that \(\T\) can be decomposed into the composite of a rotation and at most one reflection.
  Let
  \[
    \V = \seq[\oplus]{\W}{1,,m}
  \]
  be a decomposition as in \cref{6.47}(b).
  Clearly, \(m = 2\) or \(m = 3\).

  If \(m = 2\), then \(\V = \W_1 \oplus \W_2\).
  Without loss of generality, suppose that \(\dim(\W_1) = 1\) and \(\dim(\W_2) = 2\).
  Thus \(\T_{\W_1}\) is a reflection or the identity on \(\W_1\), and \(\T_{\W_2}\) is a rotation.
  Defining \(\T_1\) and \(\T_2\) as in the proof of \cref{6.11.7}, we have that \(\T = \T_1 \T_2\) is the composite of a rotation and at most one reflection.
  (Note that if \(\T_{\W_1}\) is not a reflection, then \(\T_1\) is the identity on \(\V\) and \(\T = \T_2\).)

  If \(m = 3\), then \(\V = \W_1 \oplus \W_2 \oplus \W_3\) and \(\dim(\W_i) = 1\) for all \(i \in \set{1, 2, 3}\).
  For each \(i \in \set{1, 2, 3}\), let \(\T_i\) be as in the proof of \cref{6.11.7}.
  If \(\T_{\W_i}\) is not a reflection, then \(\T_i\) is the identity on \(\W_i\).
  Otherwise, \(\T_i\) is a reflection.
  Since \(\T_{\W_i}\) is a reflection for at most one \(i \in \set{1, 2, 3}\), we conclude that \(\T\) is either a single reflection or the identity (a rotation).
\end{eg}

\exercisesection

\setcounter{ex}{1}
\begin{ex}\label{ex:6.11.2}
  Prove that rotations, reflections, and composites of rotations and reflections are orthogonal operators.
\end{ex}

\begin{proof}[\pf{ex:6.11.2}]
  Let \(\T\) be a linear operator on a \(n\)-dimensional real inner product space \(\V\).

  First we show that rotations are orthogonal operators.
  Suppose that \(\T\) is a rotation.
  By \cref{6.11.1} there exists a two-dimensional subspace \(\W\) of \(\V\) over \(\R\), an orthonormal basis \(\beta = \set{\seq{x}{1,2}}\) for \(\W\) over \(\R\), and a \(\theta \in \R\) such that
  \begin{align*}
    \T(x_1)                         & = \cos(\theta) x_1 + \sin(\theta) x_2;  \\
    \T(x_2)                         & = -\sin(\theta) x_1 + \cos(\theta) x_2; \\
    \forall y \in \W^{\perp}, \T(y) & = y.
  \end{align*}
  Let \(\gamma = \set{\seq{v}{1,,n-2}}\) be an orthonormal basis for \(\W^{\perp}\) over \(\R\).
  By \cref{6.2.4} we know that \(\beta \cup \gamma\) is an orthonormal basis for \(\V\) over \(\R\).
  Since
  \begin{align*}
    \inn{\T(x_1), \T(x_1)} & = \inn{\cos(\theta) x_1 + \sin(\theta) x_2, \cos(\theta) x_1 + \sin(\theta) x_2}        &  & \by{6.11.1}     \\
                           & = \cos(\theta) \inn{x_1, \cos(\theta) x_1 + \sin(\theta) x_2}                           &  & \by{6.1.1}[a,b] \\
                           & \quad + \sin(\theta) \inn{x_2, \cos(\theta) x_1 + \sin(\theta) x_2}                     &  & \by{6.1.1}[a,b] \\
                           & = \cos(\theta) \inn{x_1, \cos(\theta) x_1} + \sin(\theta) \inn{x_2, \sin(\theta) x_2}   &  & \by{6.1.12}     \\
                           & = (\cos(\theta))^2 \inn{x_1, x_1} + (\sin(\theta))^2 \inn{x_2, x_2}                     &  & \by{6.1.1}[b]   \\
                           & = (\cos(\theta))^2 + (\sin(\theta))^2                                                   &  & \by{6.1.12}     \\
                           & = 1;                                                                                                         \\
    \inn{\T(x_1), \T(x_2)} & = \inn{\cos(\theta) x_1 + \sin(\theta) x_2, -\sin(\theta) x_1 + \cos(\theta) x_2}       &  & \by{6.11.1}     \\
                           & = \cos(\theta) \inn{x_1, -\sin(\theta) x_1 + \cos(\theta) x_2}                          &  & \by{6.1.1}[a,b] \\
                           & \quad + \sin(\theta) \inn{x_2, -\sin(\theta) x_1 + \cos(\theta) x_2}                    &  & \by{6.1.1}[a,b] \\
                           & = \cos(\theta) \inn{x_1, -\sin(\theta) x_1} + \sin(\theta) \inn{x_2, \cos(\theta) x_2}  &  & \by{6.1.12}     \\
                           & = -\sin(\theta) \cos(\theta) \inn{x_1, x_1} + \sin(\theta) \cos(\theta) \inn{x_2, x_2}  &  & \by{6.1}[b]     \\
                           & = -\sin(\theta) \cos(\theta) + \sin(\theta) \cos(\theta)                                &  & \by{6.1.12}     \\
                           & = 0;                                                                                                         \\
    \inn{\T(x_2), \T(x_2)} & = \inn{-\sin(\theta) x_1 + \cos(\theta) x_2, -\sin(\theta) x_1 + \cos(\theta) x_2}      &  & \by{6.11.1}     \\
                           & = -\sin(\theta) \inn{x_1, -\sin(\theta) x_1 + \cos(\theta) x_2}                         &  & \by{6.1.1}[a,b] \\
                           & \quad + \cos(\theta) \inn{x_2, -\sin(\theta) x_1 + \cos(\theta) x_2}                    &  & \by{6.1.1}[a,b] \\
                           & = -\sin(\theta) \inn{x_1, -\sin(\theta) x_1} + \cos(\theta) \inn{x_2, \cos(\theta) x_2} &  & \by{6.1.12}     \\
                           & = (\sin(\theta))^2 \inn{x_1, x_1} + (\cos(\theta))^2 \inn{x_2, x_2}                     &  & \by{6.1.1}[b]   \\
                           & = (\sin(\theta))^2 + (\cos(\theta))^2                                                   &  & \by{6.1.12}     \\
                           & = 1,
  \end{align*}
  we know that \(\T(\beta) \subseteq \W\) is orthonormal.
  Since \(\T(\gamma) = \gamma\), we know that \(\T(\beta) \cup \T(\gamma)\) is an orthonormal basis for \(\V\) over \(\R\).
  Thus by \cref{6.18}(c)(e) we know that \(\T\) is an orthogonal operator.

  Next we show that reflections are orthogonal operators.
  Suppose that \(\T\) is a reflection.
  By \cref{6.11.2} there exists a one-dimensional subspace \(\W\) of \(\V\) over \(\R\) such that
  \begin{align*}
     & \forall x \in \W, \T(x) = -x;        \\
     & \forall x \in \W^{\perp}, \T(x) = x.
  \end{align*}
  Let \(y \in \V\).
  By \cref{6.6} there exists an unique tuple \(\tuple{y}{1,2} \in \W \times \W^{\perp}\) such that \(y = y_1 + y_2\).
  Since
  \begin{align*}
    \norm{\T(y)}^2 & = \norm{\T(y_1 + y_2)}^2       &  & \by{6.6}       \\
                   & = \norm{\T(y_1) + \T(y_2)}^2   &  & \by{2.1.1}[a]  \\
                   & = \norm{-y_1 + y_2}^2          &  & \by{6.11.2}    \\
                   & = \norm{-y_1}^2 + \norm{y_2}^2 &  & \by{ex:6.1.10} \\
                   & = \norm{y_1}^2 + \norm{y_2}^2  &  & \by{6.2}[a]    \\
                   & = \norm{y_1 + y_2}^2           &  & \by{ex:6.1.10} \\
                   & = \norm{y}^2,                  &  & \by{6.6}
  \end{align*}
  by \cref{6.2}(b) we have \(\norm{\T(y)} = \norm{y}\).
  Thus by \cref{6.5.1} we know that \(\T\) is an orthogonal operator.

  Finally we show that composites of rotations and reflections are orthogonal operators.
  From the proof above we see that rotations and reflections are orthogonal operators.
  Thus we only need to prove that the composites of orthogonal operators are orthogonal operators.
  This is done by \cref{ex:6.5.3}.
\end{proof}

\setcounter{ex}{3}
\begin{ex}\label{ex:6.11.4}
  For any real number \(\phi\), let
  \[
    A = \begin{pmatrix}
      \cos(\phi) & \sin(\phi)  \\
      \sin(\phi) & -\cos(\phi)
    \end{pmatrix}.
  \]
  \begin{enumerate}
    \item Prove that \(\L_A\) is a reflection.
    \item Find the axis in \(\R^2\) about which \(\L_A\) reflects.
  \end{enumerate}
\end{ex}

\begin{proof}[\pf{ex:6.11.4}(a)]
  Since
  \begin{align*}
    \det(\L_A) & = \det(A)                          &  & \by{ex:5.1.7} \\
               & = -(\cos(\phi))^2 - (\sin(\phi))^2 &  & \by{4.1.1}    \\
               & = -1,
  \end{align*}
  by \cref{6.45} we know that \(\L_A\) is a reflection.
\end{proof}

\begin{proof}[\pf{ex:6.11.4}(b)]
  Since
  \begin{align*}
             & A \begin{pmatrix}
                   x \\
                   y
                 \end{pmatrix} = \begin{pmatrix}
                                   x \\
                                   y
                                 \end{pmatrix}                                 &  & \by{6.11.2} \\
    \implies & \begin{pmatrix}
                 \cos(\phi) x + \sin(\phi) y \\
                 \sin(\phi) x - \cos(\phi) y
               \end{pmatrix} = \begin{pmatrix}
                                 x \\
                                 y
                               \end{pmatrix}                                 &  & \by{2.3.1}    \\
    \implies & \begin{dcases}
                 \sin(\phi) y = (1 - \cos(\phi)) x \\
                 \sin(\phi) x = (1 + \cos(\phi)) y
               \end{dcases}                                                \\
    \implies & (\sin(\phi), 1 - \cos(\phi)) \in \set{x \in \R^2 : Ax = x},                      \\
    \implies & \set{x \in \R^2 : Ax = x} = \spn{(\sin(\phi), 1 - \cos(\phi))}, &  & \by{6.11.2}
  \end{align*}
  by \cref{6.11.2} we know that \(\L_A\) is a reflection of \(\R^2\) about \(\spn{(\sin(\phi), 1 - \cos(\phi))}\).
\end{proof}

\begin{ex}\label{ex:6.11.5}
  For any real number \(\phi\), define \(\T_{\phi} = \L_A\), where
  \[
    A = \begin{pmatrix}
      \cos(\phi) & -\sin(\phi) \\
      \sin(\phi) & \cos(\phi)
    \end{pmatrix}.
  \]
  \begin{enumerate}
    \item Prove that any rotation on \(\R^2\) is of the form \(\T_{\phi}\) for some \(\phi\).
    \item Prove that \(\T_{\phi} \T_{\psi} = \T_{(\phi + \psi)}\) for any \(\phi, \psi \in \R\).
    \item Deduce that any two rotations on \(\R^2\) commute.
  \end{enumerate}
\end{ex}

\begin{proof}[\pf{ex:6.11.5}(a)]
  See \cref{6.11.4}.
\end{proof}

\begin{proof}[\pf{ex:6.11.5}(b)]
  Since
  \begin{align*}
     & \begin{pmatrix}
         \cos(\phi) & -\sin(\phi) \\
         \sin(\phi) & \cos(\phi)
       \end{pmatrix} \begin{pmatrix}
                       \cos(\psi) & -\sin(\psi) \\
                       \sin(\psi) & \cos(\psi)
                     \end{pmatrix}                                                            \\
     & = \begin{pmatrix}
           \cos(\phi) \cos(\psi) - \sin(\phi) \sin(\psi) & - \cos(\phi) \sin(\psi) - \sin(\phi) \cos(\psi) \\
           \sin(\phi) \cos(\psi) + \cos(\phi) \sin(\psi) & - \sin(\phi) \sin(\psi) + \cos(\phi) \cos(\psi)
         \end{pmatrix} &  & \by{2.3.1} \\
     & = \begin{pmatrix}
           \cos(\phi + \psi) & -\sin(\phi + \psi) \\
           \sin(\phi + \psi) & \cos(\phi + \psi)
         \end{pmatrix},
  \end{align*}
  by \cref{2.15}(e) we know that \(\T_{\phi} \T_{\psi} = \T_{\phi + \psi}\).
\end{proof}

\begin{proof}[\pf{ex:6.11.5}(c)]
  We have
  \begin{align*}
    \T_{\phi} \T_{\psi} & = \T_{\phi + \psi}    &  & \by{ex:6.11.5}[b] \\
                        & = \T_{\psi + \phi}                           \\
                        & = \T_{\psi} \T_{\phi} &  & \by{ex:6.11.5}[b]
  \end{align*}
  and thus any two rotations on \(\R^2\) commute.
\end{proof}

\begin{ex}\label{ex:6.11.6}
  Prove that the composite of any two rotations on \(\R^3\) is a rotation on \(\R^3\).
\end{ex}

\begin{proof}[\pf{ex:6.11.6}]

\end{proof}

\begin{ex}\label{ex:6.11.7}
  Given real numbers \(\phi\) and \(\psi\), define matrices
  \[
    A = \begin{pmatrix}
      1 & 0          & 0           \\
      0 & \cos(\phi) & -\sin(\phi) \\
      0 & \sin(\phi) & \cos(\phi)
    \end{pmatrix} \quad \text{and} \quad B = \begin{pmatrix}
      \cos(\psi) & -\sin(\psi) & 0 \\
      \sin(\psi) & \cos(\psi)  & 0 \\
      0          & 0           & 1
    \end{pmatrix}.
  \]
  \begin{enumerate}
    \item Prove that \(\L_{A}\) and \(\L_{B}\) are rotations.
    \item Prove that \(\L_{AB}\) is a rotation.
    \item Find the axis of rotation for \(\L_{AB}\).
  \end{enumerate}
\end{ex}

\begin{proof}[\pf{ex:6.11.7}(a)]

\end{proof}

\begin{proof}[\pf{ex:6.11.7}(b)]

\end{proof}

\begin{proof}[\pf{ex:6.11.7}(c)]

\end{proof}

\begin{ex}\label{ex:6.11.8}
  Prove \cref{6.45} using the hints preceding the statement of the theorem.
\end{ex}

\begin{proof}[\pf{ex:6.11.8}]
  See \cref{6.45}.
\end{proof}

\begin{ex}\label{ex:6.11.9}
  Prove that no orthogonal operator can be both a rotation and a reflection.
\end{ex}

\begin{proof}[\pf{ex:6.11.9}]

\end{proof}

\begin{ex}\label{ex:6.11.10}
  Prove that if \(\V\) is a two- or three-dimensional real inner product space, then the composite of two reflections on \(\V\) is a rotation of \(\V\).
\end{ex}

\begin{proof}[\pf{ex:6.11.10}]

\end{proof}

\begin{ex}\label{ex:6.11.11}
  Give an example of an orthogonal operator that is neither a reflection nor a rotation.
\end{ex}

\begin{proof}[\pf{ex:6.11.11}]

\end{proof}

\begin{ex}\label{ex:6.11.12}
  Let \(\V\) be a finite-dimensional real inner product space.
  Define \(\T : \V \to \V\) by \(\T(x) = -x\).
  Prove that \(\T\) is a product of rotations iff \(\dim(\V)\) is even.
\end{ex}

\begin{proof}[\pf{ex:6.11.12}]

\end{proof}

\begin{ex}\label{ex:6.11.13}
  Complete the proof of \cref{6.11.6} by showing that \(\W = \phi_{\beta}^{-1}(\vs{Z})\) satisfies the required conditions.
\end{ex}

\begin{proof}[\pf{ex:6.11.13}]
  By \cref{6.11.6} we need to show that \(\W\) is a subspace of \(\V\) over \(\R\), \(\W\) is \(\T\)-invariant and \(1 \leq \dim(\W) \leq 2\).

  First we show that \(\W\) is a subspace of \(\V\) over \(\R\).
  Since \(\vs{Z}\) is a subspace of \(\R^n\) over \(\R\) and \(\phi_{\beta}^{-1} \in \ls(\R^n, \V)\), by \cref{2.1} we know that \(\W = \phi_{\beta}^{-1}(\vs{Z})\) is a subspace of \(\V\) over \(\R\).

  Next we show that \(\W\) is \(\T\)-invariant.
  Let \(x \in \W\).
  Then there exists a \(y \in \vs{Z}\) such that \(\phi_{\beta}^{-1}(y) = x\) or \(\phi_{\beta}(x) = y\).
  By \cref{6.11.6} we know that \(\vs{Z}\) is \(\L_A\)-invariant.
  Thus by \cref{5.4.1} \(\L_A(y) \in \vs{Z}\).
  By \cref{2.21} this means \((\phi_{\beta}^{-1} \L_A)(y) \in \W\).
  Thus we have
  \begin{align*}
    \T(x) & = (\phi_{\beta}^{-1} \phi_{\beta} \T)(x)   &  & \by{2.21}   \\
          & = (\phi_{\beta}^{-1} \L_A \phi_{\beta})(x) &  & \by{2.4.12} \\
          & = (\phi_{\beta}^{-1} \L_A)(y)                               \\
          & \in \W.
  \end{align*}
  By \cref{5.4.1} \(\W\) is \(\T\)-invariant.

  Finally we show that \(1 \leq \dim(\W) \leq 2\).
  Since \(1 \leq \dim(\vs{Z}) \leq 2\) and \(\W = \phi_{\beta}^{-1}(\vs{Z})\), by \cref{2.19} we have \(1 \leq \dim(\W) \leq 2\).
\end{proof}

\begin{ex}\label{ex:6.11.14}
  Let \(\T\) be an orthogonal (unitary) operator on a finite-dimensional real (complex) inner product space \(\V\).
  If \(\W\) is a \(\T\)-invariant subspace of \(\V\) over \(\R\), prove the following results.
  \begin{enumerate}
    \item \(\T_{\W}\) is an orthogonal (unitary) operator on \(\W\).
    \item \(\W^{\perp}\) is a \(\T\)-invariant subspace of \(\V\).
    \item \(\T_{\W^{\perp}}\) is an orthogonal (unitary) operator on \(\W\).
  \end{enumerate}
\end{ex}

\begin{proof}[\pf{ex:6.11.14}(a)]
  We have
  \begin{align*}
    \forall x \in \W, \norm{\T_{\W}(x)} & = \norm{\T(x)} &  & \by{b.0.4} \\
                                        & = \norm{x}     &  & \by{6.5.1}
  \end{align*}
  and thus by \cref{6.5.1} \(\T_{\W}\) is orthogonal (unitary).
\end{proof}

\begin{proof}[\pf{ex:6.11.14}(b)]
  See \cref{ex:6.5.15}.
\end{proof}

\begin{proof}[\pf{ex:6.11.14}(c)]
  This follows from \cref{ex:6.11.14}(a)(b).
\end{proof}

\begin{ex}\label{ex:6.11.15}
  Let \(\T\) be a linear operator on a finite-dimensional vector space \(\V\) over \(\F\), where \(\V\) is a direct sum of \(\T\)-invariant subspaces, say, \(\V = \seq[\oplus]{\W}{1,,k}\).
  Prove that \(\det(\T) = \det(\T_{\W_1}) \cdots \det(\T_{\W_k})\).
\end{ex}

\begin{proof}[\pf{ex:6.11.15}]
  By \cref{5.25} we have
  \[
    [\T]_{\beta} = \begin{pmatrix}
      [\T_{\W_1}]_{\beta_1} & \zm                   & \cdots & \zm                   \\
      \zm                   & [\T_{\W_2}]_{\beta_2} & \cdots & \zm                   \\
      \zm                   & \zm                   & \cdots & [\T_{\W_k}]_{\beta_k}
    \end{pmatrix},
  \]
  where \(\beta\) is an ordered basis for \(\V\) over \(\F\) and \(\beta_i = \beta \cap \W_i\) for all \(i \in \set{1, \dots, k}\).
  Thus we have
  \begin{align*}
    \det(\T) & = \det([\T]_{\beta})                                             &  & \by{ex:5.1.7}  \\
             & = \det([\T_{\W_1}]_{\beta_1}) \cdots \det([\T_{\W_k}]_{\beta_k}) &  & \by{ex:4.3.21} \\
             & = \det(\T_{\W_1}) \cdots \det(\T_{\W_k}).                        &  & \by{ex:5.1.7}
  \end{align*}
\end{proof}


\chapter{Canonical Forms}\label{ch:7}

\begin{note}
  As we learned in \cref{ch:5}, the advantage of a diagonalizable linear operator lies in the simplicity of its description.
  Such an operator has a diagonal matrix representation, or, equivalently, there is an ordered basis for the underlying vector space consisting of eigenvectors of the operator.
  However, not every linear operator is diagonalizable, even if its characteristic polynomial splits.

  It is the purpose of this chapter to consider alternative matrix representations for nondiagonalizable operators.
  These representations are called \emph{canonical forms}.
  There are different kinds of canonical forms, and their advantages and disadvantages depend on how they are applied.
  The choice of a canonical form is determined by the appropriate choice of an ordered basis.
  Naturally, the canonical forms of a linear operator are not diagonal matrices if the linear operator is not diagonalizable.

  In this chapter, we treat two common canonical forms.
  The first of these, the \emph{Jordan canonical form}, requires that the characteristic polynomial of the operator splits.
  This form is always available if the underlying field is algebraically closed, that is, if every polynomial with coefficients from the field splits.
  For example, the field of complex numbers is algebraically closed by the fundamental theorem of algebra (see \cref{d.4}).
  The first two sections deal with this form.
  The \emph{rational canonical form}, treated in \cref{sec:7.4}, does not require such a factorization.
\end{note}

% All sections are in separated files.  We include them here.
\section{The Jordan Canonical Form I}\label{sec:7.1}

\begin{defn}\label{7.1.1}
  Let \(\T\) be a linear operator on a finite-dimensional vector space \(\V\) over \(\F\), and suppose that the characteristic polynomial of \(\T\) splits.
  Recall from \cref{5.9} that the diagonalizability of \(\T\) depends on whether the union of ordered bases for the distinct eigenspaces of \(\T\) is an ordered basis for \(\V\) over \(\F\).
  So a lack of diagonalizability means that at least one eigenspace of \(\T\) is too ``small.''

  In this section, we extend the definition of eigenspace to \emph{generalized eigenspace}.
  From these subspaces, we select ordered bases whose union is an ordered basis \(\beta\) for \(\V\) over \(\F\) such that
  \[
    [\T]_{\beta} = \begin{pmatrix}
      A_1    & \zm    & \cdots & \zm    \\
      \zm    & A_2    & \cdots & \zm    \\
      \vdots & \vdots &        & \vdots \\
      \zm    & \zm    & \cdots & A_k
    \end{pmatrix},
  \]
  where each \(\zm\) is a zero matrix, and each \(A_i\) is a square matrix of the form \((\lambda)\) or
  \[
    \begin{pmatrix}
      \lambda & 1       & \zv    & \cdots & \zv     & \zv     \\
      \zv     & \lambda & 1      & \cdots & \zv     & \zv     \\
      \vdots  & \vdots  & \vdots &        & \vdots  & \vdots  \\
      \zv     & \zv     & \zv    & \cdots & \lambda & 1       \\
      \zv     & \zv     & \zv    & \cdots & \zv     & \lambda
    \end{pmatrix}
  \]
  for some eigenvalue \(\lambda\) of \(\T\).
  Such a matrix \(A_i\) is called a \textbf{Jordan block} corresponding to \(\lambda\), and the matrix \([\T]_{\beta}\) is called a \textbf{Jordan canonical form} of \(\T\).
  We also say that the ordered basis \(\beta\) is a \textbf{Jordan canonical basis} for \(\T\).
  Observe that each Jordan block \(A_i\) is ``almost'' a diagonal matrix
  ---
  in fact, \([\T]_{\beta}\) is a diagonal matrix iff each \(A_i\) is of the form \((\lambda)\).
\end{defn}

\begin{note}
  In \cref{sec:7.1,sec:7.2}, we prove that every linear operator whose characteristic polynomial splits has a Jordan canonical form that is unique up to the order of the Jordan blocks.
  Nevertheless, it is not the case that the Jordan canonical form is completely determined by the characteristic polynomial of the operator.
\end{note}

\begin{defn}\label{7.1.2}
  Let \(\T\) be a linear operator on a vector space \(\V\) over \(\F\), and let \(\lambda \in \F\).
  Because of the structure of each Jordan block in a Jordan canonical form, we can generalize these observations:
  If \(v\) lies in a Jordan canonical basis for a linear operator \(\T\) and is associated with a Jordan block with diagonal entry \(\lambda\), then \((\T - \lambda \IT[\V])^p(v) = \zv\) for sufficiently large \(p\).
  Eigenvectors satisfy this condition for \(p = 1\).
  A nonzero vector \(x\) in \(\V\) is called a \textbf{generalized eigenvector of \(\T\) corresponding to \(\lambda\)} if \((\T - \lambda \IT[\V])^p(x) = \zv\) for some positive integer \(p\).
\end{defn}

\begin{cor}\label{7.1.3}
  Let \(\T\) be a linear operator on a vector space \(\V\) over \(\F\), and let \(\lambda \in \F\).
  If \(x\) is a generalized eigenvector of \(\T\) corresponding to \(\lambda\), and \(p\) is the smallest positive integer for which \((\T - \lambda \IT[\V])^p(x) = \zv\), then \((\T - \lambda \IT[\V])^{p - 1}(x)\) is an eigenvector of \(\T\) corresponding to \(\lambda\).
  Therefore \(\lambda\) is an eigenvalue of \(\T\).
\end{cor}

\begin{proof}[\pf{7.1.3}]
  Observe that
  \begin{align*}
             & (\T - \lambda \IT[\V])^p(x) = \zv                                   \\
    \implies & (\T - \lambda \IT[\V])\pa{(\T - \lambda \IT[\V])^{p - 1}(x)} = \zv.
  \end{align*}
  Since \(p\) is the smallest positive integer for which \((\T - \lambda \IT[\V])^p(x) = \zv\), we know that \((\T - \lambda \IT[\V])^{p - 1}(x) \neq \zv\).
  Thus by \cref{5.4} \((\T - \lambda \IT[\V])^{p - 1}(x)\) is an eigenvector of \(\T\) and therefore \(\lambda\) is an eigenvalue of \(\T\).
\end{proof}

\begin{defn}\label{7.1.4}
  Let \(\T\) be a linear operator on a vector space \(\V\) over \(\F\), and let \(\lambda\) be an eigenvalue of \(\T\).
  The \textbf{generalized eigenspace of \(\T\) corresponding to \(\lambda\)}, denoted \(\vs{K}_{\lambda}\), is the subset of \(\V\) defined by
  \[
    \vs{K}_{\lambda} = \set{x \in \V : (\T - \lambda \IT[\V])^p(x) = \zv \text{ for some } p \in \Z^+}.
  \]
  Note that \(\vs{K}_{\lambda}\) consists of the zero vector (\cref{2.1.2}(a)) and all generalized eigenvectors corresponding to \(\lambda\) (\cref{7.1.2}).
\end{defn}

\begin{thm}\label{7.1}
  Let \(\T\) be a linear operator on a vector space \(\V\) over \(\F\), and let \(\lambda \in \F\) be an eigenvalue of \(\T\).
  Then
  \begin{enumerate}
    \item \(\vs{K}_{\lambda}\) is a \(\T\)-invariant subspace of \(\V\) containing \(\vs{E}_{\lambda}\)
          (the eigenspace of \(\T\) corresponding to \(\lambda\)).
    \item For any scalar \(\mu \neq \lambda\), the restriction of \(\T - \mu \IT[\V]\) to \(\vs{K}_{\lambda}\) is one-to-one.
  \end{enumerate}
\end{thm}

\begin{proof}[\pf{7.1}(a)]
  Clearly, \(\zv \in \vs{K}_{\lambda}\).
  Suppose that \(x\) and \(y\) are in \(\vs{K}_{\lambda}\).
  Then by \cref{7.1.4} there exist positive integers \(p\) and \(q\) such that
  \[
    (\T - \lambda \IT[\V])^p(x) = (\T - \lambda \IT[\V])^q(y) = \zv.
  \]
  Therefore
  \begin{align*}
    (\T - \lambda \IT[\V])^{p + q}(x + y) & = (\T - \lambda \IT[\V])^{p + q}(x) + (\T - \lambda \IT[\V])^{p + q}(y) &  & \by{2.9}      \\
                                          & = (\T - \lambda \IT[\V])^q(\zv) + (\T - \lambda \IT[\V])^p(\zv)         &  & \by{7.1.4}    \\
                                          & = \zv,                                                                  &  & \by{2.1.2}[a]
  \end{align*}
  and hence \(x + y \in \vs{K}_{\lambda}\).
  Similarly,
  \begin{align*}
    \forall c \in \F, (\T - \lambda \IT[\V])^p(cx) & = c (\T - \lambda \IT[\V])^p(x) &  & \by{2.9}    \\
                                                   & = c \zv                         &  & \by{7.1.4}  \\
                                                   & = \zv                           &  & \by{1.2}[c]
  \end{align*}
  and hence \(cx \in \vs{K}_{\lambda}\).

  To show that \(\vs{K}_{\lambda}\) is \(\T\)-invariant, consider any \(x \in \vs{K}_{\lambda}\).
  Choose a positive integer \(p\) such that \((\T - \lambda \IT[\V])^p(x) = \zv\).
  Then
  \begin{align*}
    (\T - \lambda \IT[\V])^p \T(x) & = \T (\T - \lambda \IT[\V])^p(x) &  & \by{ex:5.4.4} \\
                                   & = \T(\zv)                        &  & \by{7.1.4}    \\
                                   & = \zv.                           &  & \by{2.1.2}[a]
  \end{align*}
  Therefore \(\T(x) \in \vs{K}_{\lambda}\).

  Finally, it is a simple observation that \(\vs{E}_{\lambda}\) is contained in \(\vs{K}_{\lambda}\).
\end{proof}

\begin{proof}[\pf{7.1}(b)]
  Let \(x \in \vs{K}_{\lambda}\) and \((\T - \mu \IT[\V])(x) = \zv\).
  By way of contradiction, suppose that \(x \neq \zv\).
  Let \(p\) be the smallest positive integer for which \((\T - \lambda \IT[\V])^p(x) = \zv\), and let \(y = (\T - \lambda \IT[\V])^{p - 1}(x)\).
  Then
  \[
    (\T - \lambda \IT[\V])(y) = (\T - \lambda \IT[\V])^p(x) = \zv,
  \]
  and hence \(y \in \vs{E}_{\lambda}\) (\cref{7.1.3}).
  Furthermore,
  \begin{align*}
    (\T - \mu \IT[\V])(y) & = (\T - \mu \IT[\V]) (\T - \lambda \IT[\V])^{p - 1}(x)  &  & \by{7.1.3}    \\
                          & = (\T - \lambda \IT[\V])^{p - 1} (\T - \mu \IT[\V]) (x) &  & \by{ex:5.4.4} \\
                          & = \zv,                                                  &  & \by{2.1.2}[a]
  \end{align*}
  so that \(y \in \vs{E}_{\mu}\).
  But by \cref{5.8} \(\vs{E}_{\lambda} \cap \vs{E}_{\mu} = \set{\zv}\), and thus \(y = \zv\), contrary to the hypothesis of \(p\).
  So \(x = \zv\), and the restriction of \(\T - \mu \IT[\V]\) to \(\vs{K}_{\lambda}\) is one-to-one.
\end{proof}

\begin{thm}\label{7.2}
  Let \(\T\) be a linear operator on a finite-dimensional vector space \(\V\) over \(\F\) such that the characteristic polynomial of \(\T\) splits.
  Suppose that \(\lambda\) is an eigenvalue of \(\T\) with multiplicity \(m\).
  Then
  \begin{enumerate}
    \item \(\dim(\vs{K}_{\lambda}) \leq m\).
    \item \(\vs{K}_{\lambda} = \ns{(\T - \lambda \IT[\V])^m}\).
  \end{enumerate}
\end{thm}

\begin{proof}[\pf{7.2}(a)]
  Let \(h\) be the characteristic polynomial of \(\T_{\vs{K}_{\lambda}}\).
  By \cref{5.21}, \(h\) divides the characteristic polynomial of \(\T\), and by \cref{7.1}(b), \(\lambda\) is the only eigenvalue of \(\T_{\vs{K}_{\lambda}}\).
  Hence \(h(t) = (-1)^d (t - \lambda)^d\), where \(d = \dim(\vs{K}_{\lambda})\) (\cref{2.5}), and \(d \leq m\).
\end{proof}

\begin{proof}[\pf{7.2}(b)]
  Clearly \(\ns{(\T - \lambda \IT[\V])^m} \subseteq \vs{K}_{\lambda}\).
  Now let \(h\) and \(d\) be as in (a).
  Then \(h(\T_{\vs{K}_{\lambda}})\) is identically zero by the Cayley--Hamilton theorem (\cref{5.23});
  therefore \((\T - \lambda \IT[\V])^d(x) = \zv\) for all \(x \in \vs{K}_{\lambda}\).
  Since \(d \leq m\), we have \(\vs{K}_{\lambda} \subseteq \ns{(\T - \lambda \IT[\V])^m}\).
\end{proof}

\begin{thm}\label{7.3}
  Let \(\T\) be a linear operator on a finite-dimensional vector space \(\V\) over \(\F\) such that the characteristic polynomial of \(\T\) splits, and let \(\seq{\lambda}{1,,k} \in \F\) be the distinct eigenvalues of \(\T\).
  Then, for every \(x \in \V\), there exist vectors \(v_i \in \vs{K}_{\lambda_i}\), \(i \in \set{1, \dots, k}\), such that
  \[
    x = \seq[+]{v}{1,,k}.
  \]
\end{thm}

\begin{proof}[\pf{7.3}]
  The proof is by mathematical induction on the number \(k\) of distinct eigenvalues of \(\T\).
  First suppose that \(k = 1\), and let \(m\) be the multiplicity of \(\lambda_1\).
  Then \((\lambda_1 - t)^m\) is the characteristic polynomial of \(\T\), and hence \((\lambda_1 \IT[\V] - \T)^m = \zT\) by the Cayley--Hamilton theorem (\cref{5.23}).
  Thus
  \begin{align*}
    \V & = \ns{\zT}                        &  & \by{2.3}    \\
       & = \ns{(\T - \lambda_1 \IT[\V])^m} &  & \by{5.23}   \\
       & = \vs{K}_{\lambda_1},             &  & \by{7.2}(b)
  \end{align*}
  and the result follows.

  Now suppose that for some integer \(k \geq 1\), the result is established whenever \(\T\) has fewer than \(k + 1\) distinct eigenvalues, and suppose that \(\T\) has \(k + 1\) distinct eigenvalues.
  Let \(m\) be the multiplicity of \(\lambda_{k + 1}\), and let \(f\) be the characteristic polynomial of \(\T\).
  Then \(f(t) = (t - \lambda_{k + 1})^m g(t)\) for some polynomial \(g\) not divisible by \((t - \lambda_{k + 1})\).
  Let \(\W = \rg{(\T - \lambda_{k + 1} \IT[\V])^m}\).
  Then \(\W\) is \(\T\)-invariant since
  \begin{align*}
             & \forall y \in \rg{(\T - \lambda_{k + 1} \IT[\V])^m}, \exists x \in \V : (\T - \lambda_{k + 1} \IT[\V])^m(x) = y \\
    \implies & \forall y \in \rg{(\T - \lambda_{k + 1} \IT[\V])^m}, \T(y) = \T ((\T - \lambda_{k + 1} \IT[\V])^m(x))           \\
             & = ((\T - \lambda_{k + 1} \IT[\V])^m \T)(x) = ((\T - \lambda_{k + 1} \IT[\V])^m)(\T(x))                          \\
             & \in \rg{(\T - \lambda_{k + 1} \IT[\V])^m}                                                                       \\
    \implies & \T(\rg{(\T - \lambda_{k + 1} \IT[\V])^m}) \subseteq \rg{(\T - \lambda_{k + 1} \IT[\V])^m}.
  \end{align*}
  We claim that \((\T - \lambda_{k + 1} \IT[\V])^m\) maps \(\vs{K}_{\lambda_i}\) onto itself for \(i \in \set{1, \dots, k}\).
  Suppose that \(i \in \set{1, \dots, k}\).
  First we show that \((\T - \lambda_{k + 1} \IT[\V])^m\) maps \(\vs{K}_{\lambda_i}\) into itself.
  This is true since
  \begin{align*}
             & \forall x \in \vs{K}_{\lambda_i}, \exists p \in \Z^+ : (\T - \lambda_i \IT[\V])^p(x) = \zv        &  & \by{7.1.4}    \\
    \implies & \forall x \in \vs{K}_{\lambda_i}, (\T - \lambda_i \IT[\V])^p((\T - \lambda_{k + 1} \IT[\V])^m(x))                    \\
             & = (\T - \lambda_{k + 1} \IT[\V])^m((\T - \lambda_i \IT[\V])^p(x))                                 &  & \by{ex:5.4.4} \\
             & = \zv                                                                                             &  & \by{2.1.2}[a] \\
    \implies & \forall x \in \vs{K}_{\lambda_i}, (\T - \lambda_{k + 1} \IT[\V])^m(x) \in \vs{K}_{\lambda_i}.     &  & \by{7.1.4}
  \end{align*}
  Now we show the claim is true.
  Since \((\T - \lambda_{k + 1} \IT[\V])^m\) maps \(\vs{K}_{\lambda_i}\) into itself and \(\lambda_{k + 1} \neq \lambda_i\), the restriction of \(\T - \lambda_{k + 1} \IT[\V]\) to \(\vs{K}_{\lambda_i}\) is one-to-one (by \cref{7.1}(b)) and hence is onto (\cref{2.4,2.5}).
  One consequence of this is that for \(i \in \set{1, \dots, k}\), \(\vs{K}_{\lambda_i}\) is contained in \(\W\);
  hence \(\lambda_i\) is an eigenvalue of \(\T_{\W}\) for \(i \in \set{1, \dots, k}\).

  Next, observe that \(\lambda_{k + 1}\) is not an eigenvalue of \(\T_{\W}\).
  For suppose that \(\T(v) = \lambda_{k + 1} v\) for some \(v \in \W\).
  Then \(v = (\T - \lambda_{k + 1} \IT[\V])^m(y)\) for some \(y \in \V\), and it follows from \cref{5.4} that
  \[
    \zv = (\T - \lambda_{k + 1} \IT[\V])(v) = (\T - \lambda_{k + 1} \IT[\V])^{m + 1}(y).
  \]
  Therefore \(y \in \vs{K}_{\lambda_{k + 1}}\).
  So by \cref{7.2}(b), \(v = (\T - \lambda_{k + 1} \IT[\V])^m(y) = \zv\).

  Since every eigenvalue of \(\T_{\W}\) is an eigenvalue of \(\T\), the distinct eigenvalues of \(\T_{\W}\) are \(\seq{\lambda}{1,,k}\).
  For each \(i \in \set{1, \dots, k}\), we defined the generalized eigenspace for \(\T_{\W}\) corresponding to \(\lambda_i\) as follow:
  \[
    \vs{K}_{\lambda_i}' = \set{x \in \W | \exists p \in \Z^+ : (\T_{\W} - \lambda_i)^p(x) = \zv}.
  \]
  Clearly we have \(\vs{K}_{\lambda_i}' \subseteq \vs{K}_{\lambda_i}\) for each \(i \in \set{1, \dots, k}\).

  Now let \(x \in \V\).
  Then \((\T - \lambda_{k + 1} \IT[\V])^m(x) \in \W\).
  Since \(\T_{\W}\) has the \(k\) distinct eigenvalues \(\seq{\lambda}{1,,k}\), the induction hypothesis applies.
  Hence there are vectors \(w_i \in \vs{K}_{\lambda_i}'\), \(i \in \set{1, \dots, k}\), such that
  \[
    (\T - \lambda_{k + 1} \IT[\V])^m(x) = \seq[+]{w}{1,,k}.
  \]
  Since \(\vs{K}_{\lambda_i}' \subseteq \vs{K}_{\lambda_i}\) for \(i \in \set{1, \dots, k}\) and \((\T - \lambda_{k + 1} \IT[\V])^m\) maps \(\vs{K}_{\lambda_i}\) onto itself for \(i \in \set{1, \dots, k}\), there exist vectors \(v_i \in \vs{K}_{\lambda_i}\) such that \((\T - \lambda_{k + 1} \IT[\V])^m(v_i) = w_i\) for \(i \in \set{1, \dots, k}\).
  Thus we have
  \[
    (\T - \lambda_{k + 1} \IT[\V])^m(x) = (\T - \lambda_{k + 1} \IT[\V])^m(v_1) + \cdots + (\T - \lambda_{k + 1} \IT[\V])^m(v_{k}),
  \]
  and it follows that \(x - (\seq[+]{v}{1,,k}) \in \vs{K}_{\lambda_{k + 1}}\) since
  \begin{align*}
    \zv & = (\T - \lambda_{k + 1} \IT[\V])^m(x) - (\T - \lambda_{k + 1} \IT[\V])^m(v_1) + \cdots + (\T - \lambda_{k + 1} \IT[\V])^m(v_{k}) \\
        & = (\T - \lambda_{k + 1} \IT[\V])^m(x - (\seq[+]{v}{1,,k})).
  \end{align*}
  Therefore there exists a vector \(v_k \in \vs{K}_{\lambda_{k + 1}}\) such that
  \[
    x = \seq[+]{v}{1,,k+1}.
  \]
\end{proof}

\begin{thm}\label{7.4}
  Let \(\T\) be a linear operator on a finite-dimensional vector space \(\V\) over \(\F\) such that the characteristic polynomial of \(\T\) splits, and let \(\seq{\lambda}{1,,k} \in \F\) be the distinct eigenvalues of \(\T\) with corresponding multiplicities \(\seq{m}{1,,k}\).
  For \(i \in \set{1, \dots, k}\), let \(\beta_i\) be an ordered basis for \(\vs{K}_{\lambda_i}\) over \(\F\).
  Then the following statements are true.
  \begin{enumerate}
    \item \(\beta_i \cap \beta_j = \varnothing\) for \(i \neq j\).
    \item \(\beta = \seq[\cup]{\beta}{1,,k}\) is an ordered basis for \(\V\) over \(\F\).
    \item \(\dim(\vs{K}_{\lambda_i}) = m_i\) for all \(i \in \set{1, \dots, k}\).
  \end{enumerate}
\end{thm}

\begin{proof}[\pf{7.4}(a)]
  Suppose for sake of contradiction that \(x \in \beta_i \cap \beta_j \subseteq \vs{K}_{\lambda_i} \cap \vs{K}_{\lambda_j}\), where \(i \neq j\).
  By \cref{7.1}(b), \(\T - \lambda_i \IT[\V]\) is one-to-one on \(\vs{K}_{\lambda_j}\), and therefore \((\T - \lambda_i \IT[\V])^p(x) \neq \zv\) for any positive integer \(p\).
  But this contradicts the fact that \(x \in \vs{K}_{\lambda_i}\), and the result follows.
\end{proof}

\begin{proof}[\pf{7.4}(b)]
  Let \(x \in \V\).
  By \cref{7.3}, for \(i \in \set{1, \dots, k}\), there exist vectors \(v_i \in \vs{K}_{\lambda_i}\) such that \(x = \seq[+]{v}{1,,k}\).
  Since each \(v_i\) is a linear combination of the vectors of \(\beta_i\), it follows that \(x\) is a linear combination of the vectors of \(\beta\).
  Therefore \(\beta\) spans \(\V\).
  Let \(q\) be the number of vectors in \(\beta\).
  Then \(\dim(\V) \leq q\).
  For each \(i \in \set{1, \dots, k}\), let \(d_i = \dim(\vs{K}_{\lambda_i})\).
  Then, by \cref{5.3} and \cref{7.2}(a),
  \[
    q = \sum_{i = 1}^k d_i \leq \sum_{i = 1}^k m_i = \dim(\V).
  \]
  Hence \(q = \dim(\V)\).
  Consequently \(\beta\) is a basis for \(\V\) over \(\F\) by \cref{1.6.15}(a).
\end{proof}

\begin{proof}[\pf{7.4}(c)]
  Using the notation and result of (b), we see that \(\sum_{i = 1}^k d_i = \sum_{i = 1}^k m_i\).
  But \(d_i \leq m_i\) by \cref{7.2}(a), and therefore \(d_i = m_i\) for all \(i \in \set{1, \dots, k}\).
\end{proof}

\begin{cor}\label{7.1.5}
  Let \(\T\) be a linear operator on a finite-dimensional vector space \(\V\) over \(\F\) such that the characteristic polynomial of \(\T\) splits.
  Then \(\T\) is diagonalizable iff \(\vs{E}_{\lambda} = \vs{K}_{\lambda}\) for every eigenvalue \(\lambda\) of \(\T\).
\end{cor}

\begin{proof}[\pf{7.1.5}]
  Combining \cref{7.4}(c) and \cref{5.9}(a), we see that \(\T\) is diagonalizable iff \(\dim(\vs{E}_{\lambda}) = \dim(\vs{K}_{\lambda})\) for each eigenvalue \(\lambda\) of \(\T\).
  But \(\vs{E}_{\lambda} \subseteq \vs{K}_{\lambda}\) (\cref{7.1}(a)), and hence these subspaces have the same dimension iff they are equal (\cref{1.11}).
\end{proof}

\section{The Jordan Canonical Form II}\label{sec:7.2}

\begin{defn}\label{7.2.1}
  For the purposes of this section, we fix a linear operator \(\T\) on an \(n\)-dimensional vector space \(\V\) over \(\F\) such that the characteristic polynomial of \(\T\) splits.
  Let \(\seq{\lambda}{1,,k}\) be the distinct eigenvalues of \(\T\).

  By \cref{7.7}, each generalized eigenspace \(\vs{K}_{\lambda_i}\) contains an ordered basis \(\beta_i\) consisting of a union of disjoint cycles of generalized eigenvectors corresponding to \(\lambda_i\).
  So by \cref{7.4}(b) and \cref{7.5}, the union \(\beta = \bigcup_{i = 1}^k \beta_i\) is a Jordan canonical basis for \(\T\).
  For each \(i \in \set{1, \dots, k}\), let \(\T_i\) be the restriction of \(\T\) to \(\vs{K}_{\lambda_i}\), and let \(A_i = [\T_i]_{\beta_i}\).
  Then \(A_i\) is a Jordan canonical form of \(\T_i\), and
  \[
    J = [\T]_{\beta} = \begin{pmatrix}
      A_1    & \zm    & \cdots & \zm    \\
      \zm    & A_2    & \cdots & \zm    \\
      \vdots & \vdots &        & \vdots \\
      \zm    & \zm    & \cdots & A_k
    \end{pmatrix}
  \]
  is a Jordan canonical form of \(\T\).
  In this matrix, each \(\zm\) is a zero matrix of appropriate size.

  In this section, we compute the matrices \(A_i\) and the bases \(\beta_i\), thereby computing \(J\) and \(\beta\) as well.
  While developing a method for finding \(J\), it becomes evident that in some sense the matrices \(A_i\) are unique.

  To aid in formulating the uniqueness theorem for \(J\), we adopt the following convention:
  The basis \(\beta_i\) for \(\vs{K}_{\lambda_i}\) will henceforth be ordered in such a way that the cycles appear in order of decreasing length.
  That is, if \(\beta_i\) is a disjoint union of cycles \(\seq{\gamma}{1,,n_i}\) and if the length of the cycle \(\gamma_j\) is \(p_j\), we index the cycles so that \(\seq[\geq]{p}{1,,n_i}\).
  This ordering of the cycles limits the possible orderings of vectors in \(\beta_i\), which in turn determines the matrix \(A_i\).
  It is in this sense that \(A_i\) is unique.
  It then follows that the Jordan canonical form for \(\T\) is unique up to an ordering of the eigenvalues of \(\T\).
  As we will see, there is no uniqueness theorem for the bases \(\beta_i\) or for \(\beta\)
  (See \cref{ex:7.2.8}, it is for this reason that we associate the dot diagram with \(\T_i\) rather than with \(\beta_i\)).
  Specifically, we show that for each \(i \in \set{1, \dots, k}\), the number \(n_i\) of cycles that form \(\beta_i\), and the length \(p_j\) (\(j \in \set{1, \dots, n_i}\)) of each cycle, is completely determined by \(\T\).

  To help us visualize each of the matrices \(A_i\) and ordered bases \(\beta_i\), we use an array of dots called a \textbf{dot diagram} of \(\T_i\), where \(\T_i\) is the restriction of \(\T\) to \(\vs{K}_{\lambda_i}\).
  Suppose that \(\beta_i\) is a disjoint union of cycles of generalized eigenvectors \(\seq{\gamma}{1,,n_i}\) with lengths \(\seq[\geq]{p}{1,,n_i}\), respectively.
  The dot diagram of \(\T_i\) contains one dot for each vector in \(\beta_i\), and the dots are configured according to the following rules.
  \begin{itemize}
    \item The array consists of \(n_i\) columns (one column for each cycle).
    \item Counting from left to right, the \(j\)th column consists of the \(p_j\) dots that correspond to the vectors of \(\gamma_j\) starting with the initial vector at the top and continuing down to the end vector.
  \end{itemize}

  Denote the end vectors of the cycles by \(\seq{v}{1,,n_i}\).
  In the following dot diagram of \(\T_i\), each dot is labeled with the name of the vector in \(\beta_i\) to which it corresponds.
  \begin{align*}
     & \bullet (\T - \lambda_i \IT[\V])^{p_1 - 1}(v_1) &  & \bullet (\T - \lambda_i \IT[\V])^{p_2 - 1}(v_2) &  & \cdots &  & \bullet (\T - \lambda_i \IT[\V])^{p_{n_i} - 1}(v_{n_i}) \\
     & \bullet (\T - \lambda_i \IT[\V])^{p_1 - 2}(v_1) &  & \bullet (\T - \lambda_i \IT[\V])^{p_2 - 2}(v_2) &  & \cdots &  & \bullet (\T - \lambda_i \IT[\V])^{p_{n_i} - 2}(v_{n_i}) \\
     & \vdots                                          &  & \vdots                                          &  &        &  & \vdots                                                  \\
     &                                                 &  &                                                 &  &        &  & \bullet (\T - \lambda_i \IT[\V])(v_{n_i})               \\
     &                                                 &  &                                                 &  &        &  & \bullet v_{n_i}                                         \\
     &                                                 &  & \bullet (\T - \lambda_i \IT[\V])(v_2)           &  &        &  &                                                         \\
     &                                                 &  & \bullet v_2                                     &  &        &  &                                                         \\
     & \bullet (\T - \lambda_i \IT[\V])(v_1)           &  &                                                 &  &        &  &                                                         \\
     & \bullet v_1                                     &  &                                                 &  &        &  &
  \end{align*}
  Notice that the dot diagram of \(\T_i\) has \(n_i\) columns (one for each cycle) and \(p_1\) rows.
  Since \(\seq[\geq]{p}{1,,n_i}\), the columns of the dot diagram become shorter (or at least not longer) as we move from left to right.

  Now let \(r_j\) denote the number of dots in the \(j\)th row of the dot diagram.
  Observe that \(\seq[\geq]{r}{1,,p_1}\).
  Furthermore, the diagram can be reconstructed from the values of the \(r_i\)'s.
  The proofs of these facts, which are combinatorial in nature, are treated in \cref{ex:7.2.9}.
\end{defn}

\begin{thm}\label{7.9}
  Using the notations in \cref{7.2.1}.
  For any positive integer \(r\), the vectors in \(\beta_i\) that are associated with the dots in the first \(r\) rows of the dot diagram of \(\T_i\) constitute a basis for \(\ns{(\T - \lambda_i \IT[\V])^r}\) over \(\F\).
  Hence the number of dots in the first \(r\) rows of the dot diagram equals \(\nt{(\T - \lambda_i \IT[\V])^r}\).
\end{thm}

\begin{proof}[\pf{7.9}]
  By \cref{7.1.4} \(\ns{(\T - \lambda_i \IT[\V])^r} \subseteq \vs{K}_{\lambda_i}\), and \(\vs{K}_{\lambda_i}\) is invariant under \((\T - \lambda_i \IT[\V])^r\).
  Let \(\U\) denote the restriction of \((\T - \lambda_i \IT[\V])^r\) to \(\vs{K}_{\lambda_i}\).
  By \cref{7.1}(b), \(\ns{(\T - \lambda_i \IT[\V])^r} = \ns{\U}\), and hence it suffices to establish the theorem for \(\U\).
  Now define
  \[
    S_1 = \set{x \in \beta_i : \U(x) = \zv} \quad \text{and} \quad S_2 = \set{x \in \beta_i : \U(x) \neq \zv}.
  \]
  Let \(a\) and \(b\) denote the number of vectors in \(S_1\) and \(S_2\), respectively, and let \(m_i = \dim(\vs{K}_{\lambda_i})\).
  Then \(a + b = m_i\).
  For any \(x \in \beta_i\), \(x \in S_1\) iff \(x\) is one of the first \(r\) vectors of a cycle (\cref{7.1.6}), and this is true iff \(x\) corresponds to a dot in the first \(r\) rows of the dot diagram (\cref{7.2.1}).
  Hence \(a\) is the number of dots in the first \(r\) rows of the dot diagram.
  For any \(x \in S_2\), the effect of applying \(\U\) to \(x\) is to move the dot corresponding to \(x\) exactly \(r\) places up its column to another dot.
  It follows that \(\U\) maps \(S_2\) in a one-to-one fashion into \(\beta_i\) (since \(\beta_i\) is linearly independent).
  Thus \(\set{\U(x) : x \in S_2}\) is a basis for \(\rg{\U}\) over \(\F\) consisting of \(b\) vectors (\cref{2.2}).
  Hence \(\rk{\U} = b\), and so \(\nt{\U} = m_i - b = a\) (\cref{2.3}).
  But \(S_1\) is a linearly independent subset of \(\ns{\U}\) consisting of \(a\) vectors;
  therefore \(S_1\) is a basis for \(\ns{\U}\).
\end{proof}

\begin{note}
  In the case that \(r = 1\), \cref{7.9} yields \cref{7.2.2}.
\end{note}

\begin{cor}\label{7.2.2}
  Using the notations in \cref{7.2.1}.
  The dimension of \(\vs{E}_{\lambda_i}\) is \(n_i\).
  Hence in a Jordan canonical form of \(\T\), the number of Jordan blocks corresponding to \(\lambda_i\) equals the dimension of \(\vs{E}_{\lambda_i}\).
\end{cor}

\begin{proof}[\pf{7.2.2}]
  By \cref{5.4} we have \(\vs{E}_{\lambda_i} = \ns{\T - \lambda_i \IT[\V]}\).
  Thus by \cref{7.9} the first row of the dot diagram of \(\T_i\) constitute a basis for \(\vs{E}_{\lambda_i}\) over \(\F\).
\end{proof}

\begin{thm}\label{7.10}
  Using the notations in \cref{7.2.1}.
  Let \(r_j\) denote the number of dots in the \(j\)th row of the dot diagram of \(\T_i\), the restriction of \(\T\) to \(\vs{K}_{\lambda_i}\).
  Then the following statements are true.
  \begin{enumerate}
    \item \(r_1 = \dim(\V) - \rk{\T - \lambda_i \IT[\V]}\).
    \item \(r_j = \rk{(\T - \lambda_i \IT[\V])^{j - 1}} - \rk{(\T - \lambda_i \IT[\V])^j}\) if \(j \in \set{2, \dots, p_1}\).
  \end{enumerate}
\end{thm}

\begin{proof}[\pf{7.10}]
  By \cref{7.9}, for \(j \in \set{1, \dots, p_1}\), we have
  \begin{align*}
    \seq[+]{r}{1,,j} & = \nt{(\T - \lambda_i \IT[\V])^j}             &  & \by{7.9} \\
                     & = \dim(\V) - \rk{(\T - \lambda_i \IT[\V])^j}. &  & \by{2.3}
  \end{align*}
  Hence \(r_1 = \dim(\V) - \rk{\T - \lambda_i \IT[\V]}\), and for \(j \in \set{2, \dots, p_1}\),
  \begin{align*}
    r_j & = (\seq[+]{r}{1,,j}) - (\seq[+]{r}{1,,j-1})                                                               \\
        & = \pa{\dim(\V) - \rk{(\T - \lambda_i \IT[\V])^j}} - \pa{\dim(\V) - \rk{(\T - \lambda_i \IT[\V])^{j - 1}}} \\
        & = \rk{(\T - \lambda_i \IT[\V])^{j - 1}} - \rk{(\T - \lambda_i \IT[\V])^j}.
  \end{align*}
\end{proof}

\begin{cor}\label{7.2.3}
  Using the notations in \cref{7.2.1}.
  For any eigenvalue \(\lambda_i\) of \(\T\), the dot diagram of \(\T_i\) is unique.
  Thus, subject to the convention that the cycles of generalized eigenvectors for the bases of each generalized eigenspace are listed in order of decreasing length, the Jordan canonical form of a linear operator or a matrix is unique up to the ordering of the eigenvalues.
\end{cor}

\begin{proof}[\pf{7.2.3}]
  \cref{7.10} shows that the dot diagram of \(\T_i\) is completely determined by \(\T\) and \(\lambda_i\).
\end{proof}

\begin{thm}\label{7.11}
  Let \(A\) and \(B\) be \(n \times n\) matrices, each having Jordan canonical forms computed according to the conventions of this section.
  Then \(A\) and \(B\) are similar iff they have (up to an ordering of their eigenvalues) the same Jordan canonical form.
\end{thm}

\begin{proof}[\pf{7.11}]
  If \(A\) and \(B\) have the same Jordan canonical form \(J\), then \(A\) and \(B\) are each similar to \(J\) and hence are similar to each other.

  Conversely, suppose that \(A\) and \(B\) are similar.
  Then \(A\) and \(B\) have the same eigenvalues (\cref{ex:5.1.12}).
  Let \(J_A\) and \(J_B\) denote the Jordan canonical forms of \(A\) and \(B\), respectively, with the same ordering of their eigenvalues.
  Then \(A\) is similar to both \(J_A\) and \(J_B\), and therefore, by the \cref{2.5.3}, \(J_A\) and \(J_B\) are matrix representations of \(\L_A\).
  Hence \(J_A\) and \(J_B\) are Jordan canonical forms of \(\L_A\). Thus \(J_A = J_B\) by the \cref{7.2.3}.
\end{proof}

\begin{cor}\label{7.2.4}
  A linear operator \(\T\) on a finite-dimensional vector space \(\V\) over \(\F\) is diagonalizable iff its Jordan canonical form is a diagonal matrix.
  Hence \(\T\) is diagonalizable iff the Jordan canonical basis for \(\T\) consists of eigenvectors of \(\T\).
\end{cor}

\begin{proof}[\pf{7.2.4}]
  By \cref{7.1.5,7.2.3} we see that this is true.
\end{proof}

\exercisesection

\setcounter{ex}{5}
\begin{ex}\label{ex:7.2.6}
  Let \(A \in \ms[n][n][\F]\) whose characteristic polynomial splits.
  Prove that \(A\) and \(\tp{A}\) have the same Jordan canonical form, and conclude that \(A\) and \(\tp{A}\) are similar.
\end{ex}

\begin{proof}[\pf{ex:7.2.6}]
  Let \(\lambda \in \F\) be an eigenvalue of \(A\).
  Since
  \begin{align*}
    \forall r \in \Z^+, \rk{\pa{A - \lambda I_n}^r} & = \rk{\tp{\pa{(A - \lambda I_n)^r}}} &  & \by{3.2.5}[a] \\
                                                    & = \rk{\pa{\tp{(A - \lambda I_n)}}^r} &  & \by{2.3.2}    \\
                                                    & = \rk{\pa{\tp{A} - \lambda I_n}^r},  &  & \by{ex:1.3.3}
  \end{align*}
  by \cref{7.10} we see that the dot diagrams of \(A\) and \(\tp{A}\) correspond to \(\lambda\) are the same.
  Since \(\lambda\) is arbitrary, by \cref{7.2.3} we conclude that \(A\) and \(\tp{A}\) have the same Jordan canonical form.
  By \cref{7.11} this means \(A\) and \(\tp{A}\) are similar.
\end{proof}

\begin{ex}\label{ex:7.2.7}
  Let \(\T\) be a linear operator on a finite-dimensional vector space \(\V\) over \(\F\) such that the characteristic polynomial of \(\T\) splits.
  Let \(\gamma\) be a cycle of generalized eigenvectors corresponding to an eigenvalue \(\lambda\), and \(\W\) be the subspace spanned by \(\gamma\).
  Define \(\gamma'\) to be the ordered set obtained from \(\gamma\) by reversing the order of the vectors in \(\gamma\).
  \begin{enumerate}
    \item Prove that \([\T_{\W}]_{\gamma'} = \tp{([\T_{\W}]_{\gamma})}\).
    \item Let \(J\) be the Jordan canonical form of \(\T\).
          Use (a) to prove that \(J\) and \(\tp{J}\) are similar.
    \item Let \(A \in \ms[n][n][\F]\) whose characteristic polynomial splits.
          Use (b) to prove that \(A\) and \(\tp{A}\) are similar.
  \end{enumerate}
\end{ex}

\begin{proof}[\pf{ex:7.2.7}(a)]
  Let \(\gamma = \set{\seq{v}{1,,m}}\).
  By \cref{7.1.6} we have
  \[
    \forall i \in \set{1, \dots, m}, \T(v_i) = \begin{dcases}
      \lambda v_i             & \text{if } i = 1                   \\
      \lambda v_i + v_{i - 1} & \text{if } i \in \set{2, \dots, m}
    \end{dcases}.
  \]
  Now define \(\gamma' = \set{v_1', \dots, v_m'}\).
  By definition we have \(v_i' = v_{m + 1 - i}\) for all \(i \in \set{1, \dots, m}\) and
  \begin{align*}
    \forall i \in \set{1, \dots, m}, \T(v_i') & = \T(v_{m + 1 - i})                                                              \\
                                              & = \begin{dcases}
                                                    \lambda v_{m + 1 - i}             & \text{if } m + 1 - i = 1                   \\
                                                    \lambda v_{m + 1 - i} + v_{m - i} & \text{if } m + 1 - i \in \set{2, \dots, m}
                                                  \end{dcases} \\
                                              & = \begin{dcases}
                                                    \lambda v_i'              & \text{if } i = m                       \\
                                                    \lambda v_i' + v_{i + 1}' & \text{if } i \in \set{1, \dots, m - 1}
                                                  \end{dcases}.
  \end{align*}
  Thus we have
  \begin{align*}
    \forall i, j \in \set{1, \dots, m}, \pa{\tp{([\T_{\W}]_{\gamma})}}_{i j} & = ([\T_{\W}]_{\gamma})_{j i}      &  & \by{1.3.3} \\
                                                                             & = \begin{dcases}
                                                                                   \lambda & \text{if } i = j     \\
                                                                                   1       & \text{if } i + 1 = j \\
                                                                                   0       & \text{otherwise}
                                                                                 \end{dcases} &  & \by{7.1.1}                  \\
                                                                             & = ([\T_{\W}]_{\gamma'})_{i j}.    &  & \by{2.2.4}
  \end{align*}
  By \cref{1.2.8} this means \(\tp{([\T_{\W}]_{\beta})} = [\T_{\W}]_{\gamma'}\).
\end{proof}

\begin{proof}[\pf{ex:7.2.7}(b)]
  Let \(\beta\) be an Jordan canonical basis following the convention in \cref{7.2.1}.
  Let \(J = [\T]_{\beta}\).
  Let \(\beta'\) be the ordered set obtained from \(\beta\) by reversing the ordered of each disjoint cycles in \(\beta\).
  By \cref{5.25} and \cref{ex:7.2.7}(a) we see that \([\T]_{\beta'} = \tp{([\T]_{\beta})} = \tp{J}\).
  By \cref{2.23} we hve \([\T]_{\beta'} = \pa{[\IT[\V]]_{\beta'}^{\beta}}^{-1} [\T]_{\beta} [\IT[\V]]_{\beta'}^{\beta}\).
  Thus by \cref{2.5.4} \(J\) and \(\tp{J}\) are similar.
\end{proof}

\begin{proof}[\pf{ex:7.2.7}(c)]
  Let \(\beta\) be the standard ordered basis for \(\vs{F}^n\) over \(\F\) and let \(\alpha\) be a Jordan canonical basis for \(A\).
  By \cref{ex:7.2.7}(b) we see that \([\L_A]_{\alpha}\) and \(\tp{([\L_A]_{\alpha})}\) are similar.
  By \cref{2.23} we know that \(A = [\L_A]_{\beta}\) and \([\L_A]_{\alpha}\) are similar.
  Thus by \cref{ex:2.5.9} we know that \(A\) and \(\tp{([\L_A]_{\alpha})}\) are similar.
  If we can show that \(\tp{A}\) and \(\tp{([\L_A]_{\alpha})}\) are similar, then by \cref{ex:2.5.9} again we see that \(A\) and \(\tp{A}\) are similar.
  This is true since
  \begin{align*}
    \tp{A} & = \tp{([\L_A]_{\beta})}                                                                                       &  & \by{2.15}[a]  \\
           & = \tp{\pa{\pa{[\IT[\V]]_{\beta}^{\alpha}}^{-1} [\L_A]_{\alpha} [\IT[\V]]_{\beta}^{\alpha}}}                   &  & \by{2.23}     \\
           & = \tp{([\IT[\V]]_{\beta}^{\alpha})} \tp{([\L_A]_{\alpha})} \tp{\pa{\pa{[\IT[\V]]_{\beta}^{\alpha}}^{-1}}}     &  & \by{2.3.2}    \\
           & = \tp{\pa{\pa{[\IT[\V]]_{\alpha}^{\beta}}^{-1}}} \tp{([\L_A]_{\alpha})} \tp{\pa{[\IT[\V]]_{\alpha}^{\beta}}}  &  & \by{2.23}     \\
           & = \pa{\tp{\pa{[\IT[\V]]_{\alpha}^{\beta}}}}^{-1} \tp{([\L_A]_{\alpha})} \tp{\pa{[\IT[\V]]_{\alpha}^{\beta}}}. &  & \by{ex:2.4.5}
  \end{align*}
\end{proof}

\begin{ex}\label{ex:7.2.8}
  Let \(\T\) be a linear operator on a finite-dimensional vector space \(\V\) over \(\F\), and suppose that the characteristic polynomial of \(\T\) splits.
  Let \(\beta\) be a Jordan canonical basis for \(\T\).
  \begin{enumerate}
    \item Prove that for any nonzero scalar \(c\), \(\set{cx : x \in \beta}\) is a Jordan canonical basis for \(\T\).
    \item Suppose that \(\gamma\) is one of the cycles of generalized eigenvectors that forms \(\beta\), and suppose that \(\gamma\) corresponds to the eigenvalue \(\lambda\) and has length greater than \(1\).
          Let \(x\) be the end vector of \(\gamma\), and let \(y\) be a nonzero vector in \(\vs{E}_{\lambda}\).
          Let \(\gamma'\) be the ordered set obtained from \(\gamma\) by replacing \(x\) by \(x + y\).
          Prove that \(\gamma'\) is a cycle of generalized eigenvectors corresponding to \(\lambda\), and that if \(\gamma'\) replaces \(\gamma\) in the union that defines \(\beta\), then the new union is also a Jordan canonical basis for \(\T\).
  \end{enumerate}
\end{ex}

\begin{proof}[\pf{ex:7.2.8}(a)]
  Let \(\lambda\) be an eigenvalue of \(\T\) and let \(x \in \beta\) be an generalized eigenvector of \(\T\) corresponding to \(\lambda\).
  If \(x \in \vs{E}_{\lambda}\), then by \cref{5.1.2} we have \(\T(x) = \lambda x\).
  If \(x \in \vs{K}_{\lambda} \setminus \vs{E}_{\lambda}\), then by \cref{7.1.1} there exists a \(v \in \beta\) such that \(\T(x) = \lambda x + v\).
  In either cases we have
  \[
    \T(cx) = c \T(x) = \begin{dcases}
      c \lambda x \\
      c \lambda x + cv
    \end{dcases} = \begin{dcases}
      \lambda (cx) \\
      \lambda (cx) + (cv)
    \end{dcases}.
  \]
  Thus by \cref{7.1.1} \(\set{cx : x \in \beta}\) is a Jordan canonical basis for \(\T\).
\end{proof}

\begin{proof}[\pf{ex:7.2.8}(b)]
  Since
  \begin{align*}
    (\T - \lambda \IT[\V])(x + y) & = (\T - \lambda \IT[\V])(x) + (\T - \lambda \IT[\V])(y) &  & \by{2.1.1}[a] \\
                                  & = (\T - \lambda \IT[\V])(x) + \zv                       &  & \by{5.4}      \\
                                  & = (\T - \lambda \IT[\V])(x),                            &  & \by{1.2.1}
  \end{align*}
  we see that \((\T - \lambda \IT[\V])^p(x + y) = (\T - \lambda \IT[\V])^p(x)\) for any \(p \in \Z^+\).
  Thus \(\gamma'\) is a cycle of generalized eigenvectors corresponding to \(\lambda\), and the rest claim follows.
\end{proof}

\begin{ex}\label{ex:7.2.9}
  Suppose that a dot diagram has \(k\) columns and \(m\) rows with \(p_j\) dots in column \(j\) and \(r_i\) dots in row \(i\).
  Prove the following results.
  \begin{enumerate}
    \item \(m = p_1\) and \(k = r_1\).
    \item We have
          \begin{align*}
             & \forall j \in \set{1, \dots, k}, p_j = \max \set{i \in \set{1, \dots, m} : r_i \geq j}; \\
             & \forall i \in \set{1, \dots, m}, r_i = \max \set{j \in \set{1, \dots, k} : p_j \geq i}.
          \end{align*}
    \item \(\seq[\geq]{r}{1,,m}\).
    \item Deduce that the number of dots in each column of a dot diagram is completely determined by the number of dots in the rows.
  \end{enumerate}
\end{ex}

\begin{proof}[\pf{ex:7.2.9}(a)]
  By \cref{7.2.1} we know that \(p_1\) is the column with the most number of dots.
  Thus \(m = p_1\).
  Since the number of columns equals the number of disjoint cycles, we have \(k = r_1\).
\end{proof}


\begin{proof}[\pf{ex:7.2.9}(b)]
  We first fix \(k\) and use induction on \(m\) to prove that
  \[
    \forall i \in \set{1, \dots, m}, r_i = \max \set{j \in \set{1, \dots, k} : p_j \geq i}.
  \]
  For \(m = 1\), our dot diagram have \(1\) row and \(k\) columns.
  This means the first row has \(k\) dots and each column has one dot.
  Thus
  \begin{align*}
    \forall i \in \set{1}, r_i & = k                                                    \\
                               & = \max \set{1, \dots, k}                               \\
                               & = \max \set{j \in \set{1, \dots, k} : 1 = p_j \geq i}.
  \end{align*}
  So the base case holds.
  Suppose inductively that for some \(m \geq 1\) the statement is true.
  We need to show that for \(m + 1\) the statement is also true.
  So suppose that there are \(k\) columns and \(m + 1\) rows in a dot diagram.
  By \cref{7.2.1} we see that columns in dot diagram are ordered by decreasing length.
  Thus the first \(r_{m + 1}\) columns are the longest columns of all.
  Since there are \(m + 1\) rows, we know that the first \(r_{m + 1}\) columns have \(m + 1\) dots and the rest \((k - r_{m + 1})\) columns have less than \(m + 1\) dots.
  Thus we have
  \begin{align*}
             & \begin{dcases}
                 \forall j \in \set{1, \dots, r_{m + 1}}, p_j = m + 1 \\
                 \forall j \in \set{r_{m + 1} + 1, \dots, k}, p_j < m + 1
               \end{dcases} \\
    \implies & r_{m + 1} = \max \set{1, \dots, r_{m + 1}}                                   \\
             & = \max \set{j \in \set{1, \dots, r_{m + 1}} : p_j \geq m + 1}                \\
             & = \max \set{j \in \set{1, \dots, k} : p_j \geq m + 1}.
  \end{align*}
  By induction hypothesis we have
  \[
    \forall i \in \set{1, \dots, m + 1}, r_i = \max \set{j \in \set{1, \dots, k} : p_j \geq i}.
  \]
  This closes the induction.

  Now we fix \(m\) and use induction on \(k\) to prove that
  \[
    \forall j \in \set{1, \dots, k}, p_j = \max \set{i \in \set{1, \dots, m} : r_i \geq j}.
  \]
  For \(k = 1\), our dot diagram have \(m\) rows and \(1\) column.
  This means the first column has \(m\) dots and each row has one dot.
  Thus
  \begin{align*}
    \forall j \in \set{1}, p_j & = m                                                    \\
                               & = \max \set{1, \dots, m}                               \\
                               & = \max \set{i \in \set{1, \dots, m} : 1 = r_i \geq j}.
  \end{align*}
  So the base case holds.
  Suppose inductively that for some \(k \geq 1\) the statement is true.
  We need to show that for \(k + 1\) the statement is also true.
  So suppose that there are \(k + 1\) columns and \(m\) rows in a dot diagram.
  By \cref{7.2.1} we see that columns in dot diagram are ordered by decreasing length.
  Thus the first \(p_{k + 1}\) rows are the longest rows of all.
  Since there are \(k + 1\) columns, we know that the first \(p_{k + 1}\) rows have \(k + 1\) dots and the rest \((m - p_{k + 1})\) rows have less than \(k + 1\) dots.
  Thus we have
  \begin{align*}
             & \begin{dcases}
                 \forall i \in \set{1, \dots, p_{k + 1}}, r_i = k + 1 \\
                 \forall i \in \set{p_{k + 1} + 1, \dots, m}, r_i < k + 1
               \end{dcases} \\
    \implies & p_{k + 1} = \max \set{1, \dots, p_{k + 1}}                                   \\
             & = \max \set{i \in \set{1, \dots, p_{k + 1}} : r_i \geq k + 1}                \\
             & = \max \set{i \in \set{1, \dots, m} : r_i \geq k + 1}.
  \end{align*}
  By induction hypothesis we have
  \[
    \forall j \in \set{1, \dots, k + 1}, p_j = \max \set{i \in \set{1, \dots, m} : r_i \geq j}.
  \]
  This closes the induction.
\end{proof}

\begin{proof}[\pf{ex:7.2.9}(c)]
  Suppose that \(i_1, i_2 \in \set{1, \dots, m}\) and \(i_1 < i_2\).
  Since
  \begin{align*}
             & i_1 < i_2                                                                                                                      \\
    \implies & \set{j \in \set{1, \dots, k} : p_j \geq i_2} \subseteq \set{j \in \set{1, \dots, k} : p_j \geq i_1}                            \\
    \implies & \max \set{j \in \set{1, \dots, k} : p_j \geq i_2} \leq \max \set{j \in \set{1, \dots, k} : p_j \geq i_1} &  & \by{7.2.1}       \\
    \implies & r_{i_2} \leq r_{i_1}                                                                                     &  & \by{ex:7.2.9}[b]
  \end{align*}
  and \(i_1, i_2\) are arbitrary, we have \(\seq[\geq]{r}{1,,m}\).
\end{proof}

\begin{proof}[\pf{ex:7.2.9}(d)]
  By \cref{ex:7.2.9}(b) we have
  \[
    \forall j \in \set{1, \dots, k}, p_j = \max{i \in \set{1, \dots, m} : r_i \geq j}.
  \]
  This means when number of dots in the rows of a dot diagram is defined, the number of dots in the columns of the same dot diagram is also defined.
\end{proof}

\begin{ex}\label{ex:7.2.10}
  Let \(\T\) be a linear operator whose characteristic polynomial splits, and let \(\lambda\) be an eigenvalue of \(\T\).
  \begin{enumerate}
    \item Prove that \(\dim(\vs{K}_{\lambda})\) is the sum of the lengths of all the cycles corresponding to \(\lambda\) in the Jordan canonical form of \(\T\).
    \item Deduce that \(\vs{E}_{\lambda} = \vs{K}_{\lambda}\) iff all the Jordan blocks corresponding to \(\lambda\) are \(1 \times 1\) matrices.
  \end{enumerate}
\end{ex}

\begin{proof}[\pf{ex:7.2.10}(a)]
  By \cref{7.6} we know that the union of disjoint cycles are linearly independent.
  By \cref{7.5} each cycle forms a Jordan block corresponding to \(\lambda\), and the union of cycles is a basis for \(\vs{K}_{\lambda}\).
  Thus \(\dim(\vs{K}_{\lambda})\) is the sum of the lengths of all cycles corresponding to \(\lambda\).
\end{proof}

\begin{proof}[\pf{ex:7.2.10}(b)]
  We have
  \begin{align*}
         & \vs{E}_{\lambda} = \vs{K}_{\lambda}                                                         \\
    \iff & \text{each cycle has length } 1                                             &  & \by{5.4}   \\
    \iff & \text{all Jordan blocks corresponding to } \lambda \text{ are } 1 \times 1. &  & \by{7.1.1}
  \end{align*}
\end{proof}

\begin{defn}\label{7.2.5}
  A linear operator \(\T\) on a vector space \(\V\) is called \textbf{nilpotent} if \(\T^p = \zT\) for some positive integer \(p\).
  An \(n \times n\) matrix \(A\) is called \textbf{nilpotent} if \(A^p = \zm\) for some positive integer \(p\).
\end{defn}

\begin{ex}\label{ex:7.2.11}
  Let \(\T\) be a linear operator on a finite-dimensional vector space \(\V\) over \(\F\), and let \(\beta\) be an ordered basis for \(\V\) over \(\F\).
  Prove that \(\T\) is nilpotent iff \([\T]_{\beta}\) is nilpotent.
\end{ex}

\begin{proof}[\pf{ex:7.2.11}]
  We have
  \begin{align*}
         & \T \text{ is nilpotent}                                                                            \\
    \iff & \exists p \in \Z^+ : \T^p = \zT                                              &  & \by{7.2.5}       \\
    \iff & \exists p \in \Z^+ : ([\T]_{\beta})^p = [\T^p]_{\beta} = [\zT]_{\beta} = \zm &  & \by{2.11,2.1.13} \\
    \iff & [\T]_{\beta} \text{ is nilpotent}.                                           &  & \by{7.2.5}
  \end{align*}
\end{proof}

\begin{ex}\label{ex:7.2.12}
  Prove that any square upper triangular matrix with each diagonal entry equal to zero is nilpotent.
\end{ex}

\begin{proof}[\pf{ex:7.2.12}]
  Let \(A \in \ms[n][n][\F]\) be an upper triangular matrix with each diagonal entry equal to zero.
  Let \(\beta = \set{\seq{e}{1,,n}}\) be the standard ordered basis for \(\vs{F}^n\) over \(\F\).
  By \cref{2.2.4} we have \(\L_A(e_1) = \zv\) and
  \[
    \forall i \in \set{2, \dots, n}, \L_A(e_i) = \sum_{k = 1}^{i - 1} A_{k i} e_k.
  \]
  This means \(\L_A^n(e_i) = \zv\) for all \(i \in \set{1, \dots, n}\).
  Thus by \cref{2.1.13} \(\L_A^p = \zT\).
  By \cref{7.2.5} this means \(\L_A\) is nilpotent.
  By \cref{ex:7.2.11} we conclude that \(A\) is nilpotent.
\end{proof}

\begin{ex}\label{ex:7.2.13}
  Let \(\T\) be a nilpotent operator on an \(n\)-dimensional vector space \(\V\) over \(\F\), and suppose that \(p\) is the smallest positive integer for which \(\T^p = \zT\).
  Prove the following results.
  \begin{enumerate}
    \item \(\ns{\T^i} \subseteq \ns{\T^{i + 1}}\) for every positive integer \(i\).
    \item There is a sequence of ordered bases \(\seq{\beta}{1,,p}\) such that \(\beta_i\) is a basis for \(\ns{\T^i}\) over \(\F\) and \(\beta_{i + 1}\) contains \(\beta_i\) for \(i \in \set{1, \dots, p - 1}\).
    \item Let \(\beta = \beta_p\) be the ordered basis for \(\ns{\T^p} = \V\) over \(\F\) in (b).
          Then \([\T]_{\beta}\) is an upper triangular matrix with each diagonal entry equal to zero.
    \item The characteristic polynomial of \(\T\) is \((-1)^n t^n\).
          Hence the characteristic polynomial of \(\T\) splits, and \(0\) is the only eigenvalue of \(\T\).
  \end{enumerate}
\end{ex}

\begin{proof}[\pf{ex:7.2.13}(a)]
  See \cref{ex:7.1.7}(a).
\end{proof}

\begin{proof}[\pf{ex:7.2.13}(b)]
  Let \(\beta_1\) be a basis for \(\ns{\T}\) over \(\F\).
  Since \(\beta_1 \subseteq \ns{\T} \subseteq \ns{\T^2}\), by \cref{1.6.15}(c) we can extend \(\beta_1\) to an ordered basis \(\beta_2\) for \(\ns{\T^2}\) over \(\F\).
  Similarly we can extend \(\beta_2\) to an ordered basis \(\beta_3\) for \(\ns{\T^3}\) over \(\F\).
  Continue this process we can construct the sequence \(\seq{\beta}{1,,p}\) which satisfy the requirement of \cref{ex:7.2.13}(b).
\end{proof}

\begin{proof}[\pf{ex:7.2.13}(c)]
  Let \(i \in \set{1, \dots, p - 1}\).
  By \cref{ex:7.1.7}(c) we have \(\ns{\T^i} \neq \ns{\T^{i + 1}}\).
  Thus
  \begin{align*}
             & \beta_{i + 1} \setminus \beta_i \neq \varnothing                   &  & \by{ex:7.2.13}[b]                 \\
    \implies & \forall v \in \beta_{i + 1} \setminus \beta_i, \begin{dcases}
                                                                \T^i(v) \neq \zv \\
                                                                \T^{i + 1}(v) = \zv
                                                              \end{dcases}      &  & \by{2.1.10}                         \\
    \implies & \forall v \in \beta_{i + 1} \setminus \beta_i, \T(v) \in \ns{\T^i} &  & \by{2.1.10}                       \\
    \implies & \forall v \in \beta_{i + 1}, \T(v) \in \ns{\T^i}                   &  & (\beta_i \subseteq \beta_{i + 1}) \\
    \implies & \T(\beta_{i + 1}) \subseteq \ns{\T^i} = \spn{\beta_i}.             &  & \by{ex:7.2.13}[b]
  \end{align*}
  By \cref{2.2.4} this means \([\T]_{\beta}\) is an upper triangular matrix with each diagonal entry equal to \(0\).
\end{proof}

\begin{proof}[\pf{ex:7.2.13}(d)]
  By \cref{ex:7.2.13}[c] we know that \([\T]_{\beta}\) is an upper triangular matrix with each diagonal entry equal to \(0\).
  Thus by \cref{ex:4.2.23} we see that the characteristic polynomial of \(\T\) is \((-1)^n t^n\).
  By \cref{5.2} we know that \(0\) is the only eigenvalue of \(\T\).
\end{proof}

\begin{ex}\label{ex:7.2.14}
  Prove the converse of \cref{ex:7.2.13}(d):
  If \(\T\) is a linear operator on an \(n\)-dimensional vector space \(\V\) over \(\F\) and \((-1)^n t^n\) is the characteristic polynomial of \(\T\), then \(\T\) is nilpotent.
\end{ex}

\begin{proof}[\pf{ex:7.2.14}]
  Since the characteristic polynomial of \(\T\) splits, by \cref{7.1.8} we know that \(\T\) has a Jordan form and a Jordan canonical basis \(\beta\).
  By \cref{5.2} we know that \(0\) is the only eigenvalue of \(\T\), thus by \cref{7.1.1} \([\T]_{\beta}\) is an upper triangular matrix with each diagonal entry equal to \(0\).
  By \cref{ex:7.2.12} we know that \([\T]_{\beta}\) is nilpotent.
  Thus by \cref{ex:7.2.11} we conclude that \(\T\) is nilpotent.
\end{proof}

\begin{ex}\label{ex:7.2.15}
  Give an example of a linear operator \(\T\) on a finite-dimensional vector space \(\V\) over \(\F\) such that \(\T\) is not nilpotent, but zero is the only eigenvalue of \(\T\).
  Characterize all such operators.
\end{ex}

\begin{proof}[\pf{ex:7.2.15}]
  First we give an example as required.
  Let \(\V = \R^3\) and let \(\F = \R\).
  Define
  \[
    A = \begin{pmatrix}
      0 & 0  & 0 \\
      0 & 0  & 1 \\
      0 & -1 & 0
    \end{pmatrix}.
  \]
  Then \(\det(A - t I_3) = (-t)(t^2 + 1)\).
  Thus the characteristic polynomial of \(A\) does not split and \(0\) is the only eigenvalue of \(A\).
  Since \(A^3 = -A\), we know that \(A^p \neq \zm\) for all \(p \in \Z^+\).
  Thus by \cref{7.2.5} \(A\) is not nilpotent.

  Now we characterize all such operator.
  By \cref{ex:7.2.13}(d) and \cref{ex:7.2.14} we know that \(\T\) is nilpotent iff the characteristic polynomial of \(\T\) is \((-1)^n t^n\).
  Thus \(\T\) is not nilpotent iff the characteristic polynomial of \(\T\) is not \((-1)^n t^n\).
\end{proof}

\begin{ex}\label{ex:7.2.16}
  Let \(\T\) be a nilpotent linear operator on a finite-dimensional vector space \(\V\) over \(\F\).
  Recall from \cref{ex:7.2.13} that \(\lambda = 0\) is the only eigenvalue of \(\T\), and hence \(\V = \vs{K}_0\).
  Let \(\beta\) be a Jordan canonical basis for \(\T\).
  Prove that for any positive integer \(i\), if we delete from \(\beta\) the vectors corresponding to the last \(i\) dots in each column of a dot diagram of \(\beta\), the resulting set is a basis for \(\rg{\T^i}\) over \(\F\).
  (If a column of the dot diagram contains fewer than \(i\) dots, all the vectors associated with that column are removed from \(\beta\).)
\end{ex}

\begin{proof}[\pf{ex:7.2.16}]
  Let \(\gamma\) be a cycle in \(\beta\) with length \(p\) and end vector \(x\).
  By \cref{7.1.6} we have
  \[
    \gamma = \set{\T^{p - 1}(x), \dots, \T^i(x), \T^{i - 1}(x), \dots, \T(x), x}.
  \]
  Let \(\alpha\) be the set obtained from removing the last \(i\) vector in \(\gamma\), i.e.,
  \[
    \alpha = \set{\T^{p - 1}(x), \dots, \T^i(x)}.
  \]
  Note that \(\alpha = \varnothing\) when \(p \leq i\).
  If \(\alpha \neq \varnothing\), then each vector in \(\alpha\) is mapped by \(\T^i\) from exactly one vector in \(\gamma \setminus \alpha\).
  Thus by \cref{2.2} \(\alpha\) is linearly independent and the union of all \(\alpha\) is a basis for \(\rg{\T^i}\) over \(\F\).
\end{proof}

\begin{ex}\label{ex:7.2.20}
\end{ex}

\begin{ex}\label{ex:7.2.21}
\end{ex}

\section{The Minimal Polynomial}\label{sec:7.3}

\begin{defn}\label{7.3.1}
  Let \(\T\) be a linear operator on a finite-dimensional vector space.
  A polynomial \(p\) is called a \textbf{minimal polynomial} of \(\T\) if \(p\) is a monic polynomial of least positive degree for which \(p(\T) = \zT\).
\end{defn}

\begin{cor}\label{7.3.2}
  Every linear operator on a finite-dimensional vector space has a minimal polynomial.
\end{cor}

\begin{proof}[\pf{7.3.2}]
  The Cayley--Hamilton theorem (\cref{5.23}) tells us that for any linear operator \(\T\) on an \(n\)-dimensional vector space, there is a polynomial \(f\) of degree \(n\) such that \(f(\T) = \zT\), namely, the characteristic polynomial of \(\T\).
  Hence there is a polynomial of least degree with this property, and this degree is at most \(n\).
  If \(g\) is such a polynomial, we can divide \(g\) by its leading coefficient to obtain another polynomial \(p\) of the same degree with leading coefficient \(1\), that is, \(p\) is a \emph{monic} polynomial.
\end{proof}

\begin{thm}\label{7.12}
  Let \(p\) be a minimal polynomial of a linear operator \(\T\) on a finite-dimensional vector space \(\V\) over \(\F\).
  \begin{enumerate}
    \item For any polynomial \(g\), if \(g(\T) = \zT\), then \(p\) divides \(g\).
          In particular, \(p\) divides the characteristic polynomial of \(\T\).
    \item The minimal polynomial of \(\T\) is unique.
  \end{enumerate}
\end{thm}

\begin{proof}[\pf{7.12}(a)]
  Let \(g\) be a polynomial for which \(g(\T) = \zT\).
  By the division algorithm for polynomials (\cref{e.1}), there exist polynomials \(q\) and \(r\) such that
  \begin{equation}\label{eq:7.3.1}
    g(t) = q(t) p(t) + r(t),
  \end{equation}
  where \(r\) has degree less than the degree of \(p\).
  Substituting \(\T\) into \cref{eq:7.3.1} and using that \(g(\T) = p(\T) = \zT\), we have \(r(\T) = \zT\).
  Since \(r\) has degree less than \(p\) and \(p\) is the minimal polynomial of \(\T\), \(r\) must be the zero polynomial.
  Thus \cref{eq:7.3.1} simplifies to \(g = qp\), proving (a).
\end{proof}

\begin{proof}[\pf{7.12}(b)]
  Suppose that \(p_1\) and \(p_2\) are each minimal polynomials of \(\T\).
  Then \(p_1\) divides \(p_2\) by (a).
  Since \(p_1\) and \(p_2\) have the same degree, we have that \(p_2 = c p_1\) for some nonzero scalar \(c\).
  Because \(p_1\) and \(p_2\) are monic, \(c = 1\);
  hence \(p_1 = p_2\).
\end{proof}

\begin{defn}\label{7.3.3}
  Let \(A \in \ms[n][n][\F]\).
  The \textbf{minimal polynomial} \(p\) of \(A\) is the monic polynomial of least positive degree for which \(p(A) = \zm\).
\end{defn}

\begin{thm}\label{7.13}
  Let \(\T\) be a linear operator on a finite-dimensional vector space \(\V\) over \(\F\), and let \(\beta\) be an ordered basis for \(\V\) over \(\F\).
  Then the minimal polynomial of \(\T\) is the same as the minimal polynomial of \([\T]_{\beta}\).
\end{thm}

\begin{proof}[\pf{7.13}]
  Let \(p\) be the minimal polynomial of \(\T\).
  By \cref{7.3.1} we have \(p(\T) = \zT\).
  Thus by \cref{e.3}(b) we have \(\zm = [p(\T)]_{\beta} = p([\T]_{\beta})\).
  Suppose for sake of contradiction that \(q\) is the minimal polynomial of \(A\) and \(q \neq p\).
  Then the degree of \(q\) must be less than the degree of \(p\).
  Thus by \cref{e.3}(b) this means \(\zm = q([\T]_{\beta}) = [q(\T)]_{\beta}\).
  But by \cref{2.2.4} this means \(q(\T) = \zT\), which contradict to the uniqueness of \(p\) (\cref{7.12}(b)).
  Thus \(p = q\).
\end{proof}

\begin{cor}\label{7.3.4}
  For any \(A \in \ms[n][n][\F]\), the minimal polynomial of \(A\) is the same as the minimal polynomial of \(\L_A\).
\end{cor}

\begin{proof}[\pf{7.3.4}]
  Let \(\beta\) be the standard ordered basis for \(\vs{F}^n\) over \(\F\).
  By \cref{2.15}(a) we have \([\L_A]_{\beta} = A\).
  Thus by \cref{7.13} the minimal polynomial of \(A\) is the same as the minimal polynomial of \(\L_A\).
\end{proof}

\begin{note}
  In view of \cref{7.13,7.3.4}, \cref{7.12} and all subsequent theorems in this section that are stated for operators are also valid for matrices.
\end{note}

\begin{thm}\label{7.14}
  Let \(\T\) be a linear operator on a finite-dimensional vector space \(\V\) over \(\F\), and let \(p\) be the minimal polynomial of \(\T\).
  A scalar \(\lambda\) is an eigenvalue of \(\T\) iff \(p(\lambda) = 0\).
  Hence the characteristic polynomial and the minimal polynomial of \(\T\) have the same zeros.
\end{thm}

\begin{proof}[\pf{7.14}]
  Let \(f\) be the characteristic polynomial of \(\T\).
  Since \(p\) divides \(f\) (\cref{7.12}(a)), there exists a polynomial \(q\) such that \(f = qp\).
  If \(\lambda\) is a zero of \(p\), then
  \[
    f(\lambda) = q(\lambda) p(\lambda) = q(\lambda) \cdot 0 = 0.
  \]
  So \(\lambda\) is a zero of \(f\);
  that is, \(\lambda\) is an eigenvalue of \(\T\).

  Conversely, suppose that \(\lambda\) is an eigenvalue of \(\T\), and let \(x \in \V\) be an eigenvector corresponding to \(\lambda\).
  By \cref{ex:5.1.22}, we have
  \[
    \zv = \zT(x) = p(\T)(x) = p(\lambda)(x).
  \]
  Since \(x \neq \zv\), it follows that \(p(\lambda) = 0\), and so \(\lambda\) is a zero of \(p\).
\end{proof}

\begin{cor}\label{7.3.5}
  Let \(\T\) be a linear operator on a finite-dimensional vector space \(\V\) over \(\F\) with minimal polynomial \(p\) and characteristic polynomial \(f\).
  Suppose that \(f\) factors as
  \[
    f(t) = (\lambda_1 - t)^{n_1} \cdots (\lambda_k - t)^{n_k},
  \]
  where \(\seq{\lambda}{1,,k}\) are the distinct eigenvalues of \(\T\).
  Then there exist integers \(\seq{m}{1,,k}\) such that \(1 \leq m_i \leq n_i\) for all \(i \in \set{1, \dots, k}\) and
  \[
    p(t) = (t - \lambda_1)^{m_1} \cdots (t - \lambda_k)^{m_k}.
  \]
\end{cor}

\begin{proof}[\pf{7.3.5}]
  By \cref{7.14} we know that \(f\) and \(p\) have the same zeros, and by \cref{5.2} these zeros are exactly the eigenvalues of \(\T\).
  Thus \(\seq{m}{1,,k}\) exist and \(m_i \geq 1\) for all \(i \in \set{1, \dots, k}\).
  Since \(p\) divides \(f\) (\cref{7.12}(a)), we know that \(m_i \leq n_i\) for all \(i \in \set{1, \dots, k}\).
\end{proof}

\section{The Rational Canonical Form}\label{sec:7.4}

\exercisesection

\begin{ex}\label{ex:7.4.7}

\end{ex}



% Begin of appendix.
\appendix

% All appendices are in separated files.  We include them here.
\chapter{Sets}\label{ch:a}

\begin{defn}\label{a.0.1}
  A \textbf{set} is a collection of objects, called \textbf{elements} of the set.
  If \(x\) is an element of the set \(A\), then we write \(x \in A\);
  otherwise, we write \(x \notin A\).
\end{defn}

\begin{note}
  One set that appears frequently is the set of real numbers, which we denote by \(\R\) throughout this text.
\end{note}

\begin{defn}\label{a.0.2}
  Two sets \(A\) and \(B\) are called equal, written \(A = B\), if they contain exactly the same elements.
  Sets may be described in one of two ways:
  \begin{enumerate}
    \item By listing the elements of the set between set braces \(\set{}\).
    \item By describing the elements of the set in terms of some characteristic property.
  \end{enumerate}
\end{defn}

\begin{note}
  The order in which the elements of a set are listed is immaterial.
\end{note}

\begin{defn}\label{a.0.3}
  A set \(B\) is called a \textbf{subset} of a set \(A\), written \(B \subseteq A\) or \(A \supseteq B\), if every element of \(B\) is an element of \(A\).
  If \(B \subseteq A\), and \(B \neq A\), then \(B\) is called a \textbf{proper subset} of \(A\).
  Observe that \(A = B\) iff \(A \subseteq B\) and \(B \subseteq A\), a fact that is often used to prove that two sets are equal.
\end{defn}

\begin{defn}\label{a.0.4}
  The \textbf{empty set}, denoted by \(\varnothing\), is the set containing no elements.
  The empty set is a subset of every set.
\end{defn}

\begin{defn}\label{a.0.5}
  Sets may be combined to form other sets in two basic ways.
  The \textbf{union} of two sets \(A\) and \(B\), denoted \(A \cup B\), is the set of elements that are in \(A\), or \(B\), or both;
  that is,
  \[
    A \cup B = \set{x : x \in A \text{ or } x \in B}.
  \]
  The \textbf{intersection} of two sets \(A\) and \(B\), denoted \(A \cap B\), is the set of elements that are in both \(A\) and \(B\);
  that is,
  \[
    A \cap B = \set{x : x \in A \text{ and } x \in B}.
  \]
  Two sets are called \textbf{disjoint} if their intersection equals the empty set.
  The union and intersection of more than two sets can be defined analogously.
  Specifically, if \(\seq{A}{1,,n}\) are sets, then the union and intersections of these sets are defined, respectively, by
  \[
    \bigcup_{i = 1}^n A_i = \set{x : x \in A_i \text{ for some } i \in \set{1, \dots, n}}
  \]
  and
  \[
    \bigcap_{i = 1}^n A_i = \set{x : x \in A_i \text{ for all } i \in \set{1, \dots, n}}.
  \]
  Similarly, if \(\Lambda\) is an index set and \(\set{A_{\alpha} : \alpha \in \Lambda}\) is a collection of sets, the union and intersection of these sets are defined, respectively, by
  \[
    \bigcup_{\alpha \in \Lambda} A_{\alpha} = \set{x : x \in A_{\alpha} \text{ for some } \alpha \in \Lambda}
  \]
  and
  \[
    \bigcap_{\alpha \in \Lambda} A_{\alpha} = \set{x : x \in A_{\alpha} \text{ for all } \alpha \in \Lambda}.
  \]
\end{defn}

\begin{defn}\label{a.0.6}
  By a relation on a set \(A\), we mean a rule for determining whether or not, for any elements \(x\) and \(y\) in \(A\), \(x\) stands in a given relationship to \(y\).
  More precisely, a \textbf{relation} on \(A\) is a set \(S\) of ordered pairs of elements of \(A\) such that \((x, y) \in S\) iff \(x\) stands in the given relationship to \(y\).
  If \(S\) is a relation on a set \(A\), we often write \(x \sim y\) in place of \((x, y) \in S\).
\end{defn}

\begin{defn}\label{a.0.7}
  A relation \(S\) on a set \(A\) is called an \textbf{equivalence relation} on \(A\) if the following three conditions hold:
  \begin{description}
    \item[Reflexivity:]
      For each \(x \in A\), \(x \sim x\).
    \item[Symmetry:]
      If \(x \sim y\), then \(y \sim x\).
    \item[Transitivity:]
      If \(x \sim y\) and \(y \sim z\), then \(x \sim z\).
  \end{description}
\end{defn}

\chapter{Functions}\label{ch:b}

\begin{defn}\label{b.0.1}
  If \(A\) and \(B\) are sets, then a \textbf{function} \(f\) from \(A\) to \(B\), written \(f : A \to B\), is a rule that associates to each element \(x\) in \(A\) a unique element denoted \(f(x)\) in \(B\).
  The element \(f(x)\) is called the \textbf{image} of \(x\) (under \(f\)), and \(x\) is called a preimage of \(f(x)\) (under \(f\)).
  If \(f : A \to B\), then \(A\) is called the \textbf{domain} of \(f\), \(B\) is called the \textbf{codomain} of \(f\), and the set \(\set{f(x) : x \in A}\) is called the \textbf{range} of \(f\).
  Note that the range of \(f\) is a subset of \(B\).
  If \(S \subseteq A\), we denote by \(f(S)\) the set \(\set{f(x) : x \in S}\) of all images of elements of \(S\).
  Likewise, if \(T \subseteq B\), we denote by \(f^{-1}(T)\) the set \(\set{x \in A : f(x) \in T}\) of all preimages of elements in \(T\).
  Finally, two functions \(f : A \to B\) and \(g : A \to B\) are \textbf{equal}, written \(f = g\), if \(f(x) = g(x)\) for all \(x \in A\).
\end{defn}

\begin{defn}\label{b.0.2}
  the preimage of an element in the range need not be unique.
  Functions such that each element of the range has a unique preimage are called \textbf{one-to-one};
  that is \(f : A \to B\) is one-to-one if \(f(x) = f(y)\) implies \(x = y\) or, equivalently, if \(x \neq y\) implies \(f(x) \neq f(y)\).
\end{defn}

\begin{defn}\label{b.0.3}
  If \(f : A \to B\) is a function with range \(B\), that is, if \(f(A) = B\), then \(f\) is called \textbf{onto}.
  So \(f\) is onto iff the range of \(f\) equals the codomain of \(f\).
\end{defn}

\begin{defn}\label{b.0.4}
  Let \(f : A \to B\) be a function and \(S \subseteq A\).
  Then a function \(f_S : S \to B\), called the \textbf{restriction} of \(f\) to \(S\), can be formed by defining \(f_S(x) = f(x)\) for each \(x \in S\).
\end{defn}

\begin{defn}\label{b.0.5}
  Let \(A\), \(B\), and \(C\) be sets and \(f : A \to B\) and \(g : B \to C\) be functions.
  By following \(f\) with \(g\), we obtain a function \(g \circ f : A \to C\) called the \textbf{composite} of \(g\) and \(f\).
  Thus \((g \circ f)(x) = g(f(x))\) for all \(x \in A\).
  Functional composition is associative, however;
  that is, if \(h : C \to D\) is another function, then \(h \circ (g \circ f) = (h \circ g) \circ f\).
\end{defn}

\begin{defn}\label{b.0.6}
  A function \(f : A \to B\) is said to be \textbf{invertible} if there exists a function \(g : B \to A\) such that \((f \circ g)(y) = y\) for all \(y \in B\) and \((g \circ f)(x) = x\) for all \(x \in A\).
  If such a function \(g\) exists, then it is unique and is called the \textbf{inverse} of \(f\).
  We denote the inverse of \(f\) (when it exists) by \(f^{-1}\).
  It can be shown that \(f\) is invertible iff \(f\) is both one-to-one and onto.
  The following facts about invertible functions are easily proved.
  \begin{itemize}
    \item If \(f : A \to B\) is invertible, then \(f^{-1}\) is invertible, and \((f^{-1})^{-1} = f\).
    \item If \(f : A \to B\) and \(g : B \to C\) are invertible, then \(g \circ f\) is invertible, and \((g \circ f)^{-1} = f^{-1} \circ g^{-1}\).
  \end{itemize}
\end{defn}

\chapter{Fields}\label{ch:c}

\begin{defn}\label{c.0.1}
  A field \(\F\) is a set on which two operations \(+\) and \(\cdot\) (called \textbf{addition} and \textbf{multiplication}, respectively) are defined so that, for each pair of elements \(x, y\) in \(F\), there are unique elements \(x + y\) and \(x \cdot y\) in \(\F\) for which the following conditions hold for all elements \(a, b, c\) in \(\F\).
  \begin{enumerate}[label=(F \arabic*), ref=F \arabic*]
    \item\label{f1} \(a + b = b + a\) and \(a \cdot b = b \cdot a\)
    (commutativity of addition and multiplication).
    \item\label{f2} \((a + b) + c = a + (b + c)\) and \((a \cdot b) \cdot c = a \cdot (b \cdot c)\)
    (associativity of addition and multiplication).
    \item\label{f3} There exist distinct elements \(0\) and \(1\) in \(\F\) such that
    \[
      0 + a = a \quad \text{and} \quad 1 \cdot a = a
    \]
    (existence of identity elements for addition and multiplication).
    \item\label{f4} For each element \(a\) in \(F\) and each nonzero element \(b\) in \(\F\), there exist elements \(c\) and \(d\) in \(\F\) such that
    \[
      a + c = 0 \quad \text{and} \quad b \cdot d = 1
    \]
    (existence of inverses for addition and multiplication).
    \item\label{f5} \(a \cdot (b + c) = a \cdot b + a \cdot c\)
    (distributivity of multiplication over addition).
  \end{enumerate}
  The elements \(x + y\) and \(x \cdot y\) are called the \textbf{sum} and \textbf{product}, respectively, of \(x\) and \(y\).
  The elements \(0\) (read ``zero'') and \(1\) (read ``one'') mentioned in \ref{f3} are called \textbf{identity elements} for addition and multiplication, respectively, and the elements \(c\) and \(d\) referred to in \ref{f4} are called an \textbf{additive inverse} for \(a\) and a \textbf{multiplicative inverse} for \(b\), respectively.
\end{defn}

\begin{eg}\label{c.0.2}
  The set of real numbers \(\R\) with the usual definitions of addition and multiplication is a field.
\end{eg}

\begin{eg}\label{c.0.3}
  The set of rational numbers \(\Q\) with the usual definitions of addition and multiplication is a field.
\end{eg}

\begin{eg}\label{c.0.4}
  The field \(\Z_2\) consists of two elements \(0\) and \(1\) with the operations of addition and multiplication defined by the equations
  \begin{align*}
    0 + 0     & = 0             \\
    0 + 1     & = 1 + 0 = 1     \\
    1 + 1     & = 0             \\
    0 \cdot 0 & = 0             \\
    0 \cdot 1 & = 1 \cdot 0 = 0 \\
    1 \cdot 1 & = 1.
  \end{align*}
\end{eg}

\begin{thm}[Cancellation Laws]\label{c.1}
  For arbitrary elements \(a\), \(b\), and \(c\) in a field, the following statements are true.
  \begin{enumerate}
    \item If \(a + b = c + b\), then \(a = c\).
    \item If \(a \cdot b = c \cdot b\) and \(b \neq 0\), then \(a = c\).
  \end{enumerate}
\end{thm}

\begin{proof}[\pf{c.1}(a)]
  We have
  \begin{align*}
             & \exists d \in \F : b + d = 0 &  & \text{(by \ref{f4})} \\
    \implies & (a + b) + d = (c + b) + d    &  & \by{c.0.1}           \\
    \implies & a + (b + d) = c + (b + d)    &  & \text{(by \ref{f2})} \\
    \implies & a + 0 = c + 0                &  & \text{(by \ref{f4})} \\
    \implies & a = c.                       &  & \text{(by \ref{f3})}
  \end{align*}
\end{proof}

\begin{proof}[\pf{c.1}(b)]
  If \(b \neq 0\), then \ref{f4} guarantees the existence of an element \(d\) in the field such that \(b \cdot d = 1\).
  Multiply both sides of the equality \(a \cdot b = c \cdot b\) by \(d\) to obtain \((a \cdot b) \cdot d = (c \cdot b) \cdot d\).
  Consider the left side of this equality:
  By \ref{f2} and \ref{f3}, we have
  \[
    (a \cdot b) \cdot d = a \cdot (b \cdot d) = a \cdot 1 = a.
  \]
  Similarly, the right side of the equality reduces to \(c\).
  Thus \(a = c\).
\end{proof}

\begin{cor}\label{c.0.5}
  The elements \(0\) and \(1\) mentioned in \ref{f3}, and the elements \(c\) and \(d\) mentioned in \ref{f4}, are unique.
\end{cor}

\begin{proof}[\pf{c.0.5}]
  Suppose that \(0' \in \F\) satisfies \(0' + a = a\) for each \(a \in \F\).
  Since \(0 + a = a\) for each \(a \in \F\) , we have \(0' + a = 0 + a\) for each \(a \in \F\).
  Thus \(0' = 0\) by \cref{c.1}.
  The proofs of the remaining parts are similar.
\end{proof}

\begin{defn}\label{c.0.6}
  The additive inverse and the multiplicative inverse of \(b\) are denoted by \(-b\) and \(b^{-1}\), respectively.
  Note that \(-(-b) = b\) and \((b^{-1})^{-1} = b\).
\end{defn}

\begin{defn}\label{c.0.7}
  \textbf{Subtraction} and \textbf{division} can be defined in terms of addition and multiplication by using the additive and multiplicative inverses.
  Specifically, subtraction of \(b\) is defined to be addition of \(-b\) and division by \(b \neq 0\) is defined to be multiplication by \(b^{-1}\);
  that is,
  \[
    a - b = a + (-b) \quad \text{and} \quad \dfrac{a}{b} = a \cdot b^{-1}.
  \]
  In particular, the symbol \(\dfrac{1}{b}\) denotes \(b^{-1}\).
  Division by zero is undefined, but, with this exception, the sum, product, difference, and quotient of any two elements of a field are defined.
\end{defn}

\begin{thm}\label{c.2}
  Let \(a\) and \(b\) be arbitrary elements of a field.
  Then each of the following statements are true.
  \begin{enumerate}
    \item \(a \cdot 0 = 0\).
    \item \((-a) \cdot b = a \cdot (-b) = -(a \cdot b)\).
    \item \((-a) \cdot (-b) = a \cdot b\).
  \end{enumerate}
\end{thm}

\begin{proof}[\pf{c.2}(a)]
  Since \(0 + 0 = 0\), \ref{f5} shows that
  \[
    0 + a \cdot 0 = a \cdot 0 = a \cdot (0 + 0) = a \cdot 0 + a \cdot 0.
  \]
  Thus \(0 = a \cdot 0\) by \cref{c.1}.
\end{proof}

\begin{proof}[\pf{c.2}(b)]
  By definition, \(-(a \cdot b)\) is the unique element of \(\F\) with the property \(a \cdot b + [-(a \cdot b)] = 0\).
  So in order to prove that \((-a) \cdot b = -(a \cdot b)\), it suffices to show that \(a \cdot b + (-a) \cdot b = 0\).
  But \(-a\) is the element of \(\F\) such that \(a + (-a) = 0\);
  so
  \[
    a \cdot b + (-a) \cdot b = [a + (-a)] \cdot b = 0 \cdot b = b  \cdot 0 = 0
  \]
  by \ref{f5} and \cref{c.2}(a).
  Thus \((-a) \cdot b = -(a \cdot b)\).
  The proof that \(a \cdot (-b) = -(a \cdot b)\) is similar.
\end{proof}

\begin{proof}[\pf{c.2}(c)]
  By applying \cref{c.2}(b) twice, we find that
  \[
    (-a) \cdot (-b) = -[a \cdot (-b)] = -[-(a \cdot b)] = a \cdot b.
  \]
\end{proof}

\begin{cor}\label{c.0.8}
  The additive identity of a field has no multiplicative inverse.
\end{cor}

\begin{proof}[\pf{c.0.8}]
  Suppose for sake of contradiction that \(0^{-1}\) exists.
  But then we have
  \begin{align*}
    \forall a \in \F \setminus \set{1}, a & = a \cdot 1                &  & \text{(by \ref{f3})} \\
                                          & = a \cdot (0 \cdot 0^{-1}) &  & \text{(by \ref{f4})} \\
                                          & = (a \cdot 0) \cdot 0^{-1} &  & \text{(by \ref{f2})} \\
                                          & = 0 \cdot 0^{-1}           &  & \by{c.2}[a]          \\
                                          & = 1,                       &  & \text{(by \ref{f4})}
  \end{align*}
  a contradiction.
  Thus \(0^{-1}\) does not exist.
\end{proof}

\begin{defn}\label{c.0.9}
  In an arbitrary field \(\F\), it may happen that a sum \(1 + 1 + \cdots + 1\) (\(p\) summands) equals \(0\) for some positive integer \(p\).
  For example, in the field \(\Z_2\) (defined in \cref{c.0.4}), \(1 + 1 = 0\).
  In this case, the smallest positive integer \(p\) for which a sum of \(p\) \(1\)'s equals \(0\) is called the \textbf{characteristic} of \(\F\);
  if no such positive integer exists, then \(\F\) is said to have \textbf{characteristic zero}.
  Thus \(\Z_2\) has characteristic two, and \(\R\) has characteristic zero.
  Observe that if \(\F\) is a field of characteristic \(p \neq 0\), then \(x + x + \cdots + x\) (\(p\) summands) equals \(0\) for all \(x \in \F\).
  In a field having nonzero characteristic (especially characteristic two), many unnatural problems arise.
  For this reason, some of the results about vector spaces stated in this book require that the field over which the vector space is defined be of characteristic zero (or, at least, of some characteristic other than two).
\end{defn}

\chapter{Complex Numbers}\label{ch:d}

\begin{defn}\label{d.0.1}
  A complex number is an expression of the form \(z = a + bi\), where \(a\) and \(b\) are real numbers called the \textbf{real part} and the \textbf{imaginary part} of \(z\), respectively.

  The \textbf{sum} and \textbf{product} of two complex numbers \(z = a + bi\) and \(w = c + di\) (where \(a, b, c\), and \(d\) are real numbers) are defined, respectively, as follows:
  \[
    z + w = (a + bi) + (c + di) = (a + c) + (b + d) i
  \]
  and
  \[
    zw = (a + bi)(c + di) = (ac - bd) + (ad + bc) i.
  \]
\end{defn}

\begin{defn}\label{d.0.2}
  Any real number \(c\) may be regarded as a complex number by identifying \(c\) with the complex number \(c + 0i\).
  Observe that this correspondence preserves sums and products;
  that is,
  \[
    (c + 0i) + (d + 0i) = (c + d) + 0i \quad \text{and} \quad (c + 0i)(d + 0i) = cd + 0i.
  \]
  Any complex number of the form \(bi = 0 + bi\), where \(b\) is a nonzero real number, is called \textbf{imaginary}.
  The product of two imaginary numbers is real since
  \[
    (bi)(di) = (0 + bi)(0 + di) = (0 - bd) + (0 \cdot d + b \cdot 0) i = -bd.
  \]
  In particular, for \(i = 0 + 1i\), we have \(i \cdot i = -1\).
\end{defn}

\begin{note}
  The observation that \(i^2 = i \cdot i = -1\) provides an easy way to remember the definition of multiplication of complex numbers:
  simply multiply two complex numbers as you would any two algebraic expressions, and replace \(i^2\) by \(-1\).
\end{note}

\begin{defn}\label{d.0.3}
  The real number \(0\), regarded as a complex number, is an additive identity element for the complex numbers since
  \[
    (a + bi) + 0 = (a + bi) + (0 + 0i) = (a + 0) + (b + 0) i = a + bi.
  \]
  Likewise the real number \(1\), regarded as a complex number, is a multiplicative identity element for the set of complex numbers since
  \[
    (a + bi) \cdot 1 = (a + bi)(1 + 0i) = (a \cdot 1 - b \cdot 0) + (a \cdot 0 + b \cdot 1) i = a + bi.
  \]
  Every complex number \(a + bi\) has an additive inverse, namely \((-a) + (-b)i\).
  But also each complex number except \(0\) has a multiplicative inverse.
  In fact,
  \[
    (a + bi)^{-1} = \frac{a}{a^2 + b^2} - \frac{b}{a^2 + b^2} i.
  \]
\end{defn}

\begin{thm}\label{d.1}
  The set of complex numbers with the operations of addition and multiplication previously defined is a field.
\end{thm}

\begin{proof}[\pf{d.1}]
  Let \(x, y, z \in \C\).
  By \cref{d.0.1} we know that \(x = a + bi, y = c + di, z = e + fi\) for some \(a, b, c, d, e, f \in \R\).
  By \cref{d.0.1} we know that \(x + y \in \C\) and \(xy \in \C\).
  Thus by \cref{c.0.1} we only need to show that \ref{f1} -- \ref{f5} hold.
  \begin{description}
    \item[For \ref{f1}:]
      We have
      \begin{align*}
        x + y & = (a + c) + (b + d) i &  & \text{(by \cref{d.0.1})} \\
              & = (c + a) + (d + b) i &  & (a, b, c, d \in \R)      \\
              & = y + x               &  & \text{(by \cref{d.0.1})}
      \end{align*}
      and
      \begin{align*}
        xy & = (ac - bd) + (ad + bc) i &  & \text{(by \cref{d.0.1})} \\
           & = (ca - db) + (da + cb) i &  & (a, b, c, d \in \R)      \\
           & = yx.                     &  & \text{(by \cref{d.0.1})}
      \end{align*}
    \item[For \ref{f2}:]
      We have
      \begin{align*}
        (x + y) + z & = ((a + c) + (b + d)i) + z        &  & \text{(by \cref{d.0.1})}  \\
                    & = ((a + c) + e) + ((b + d) + f) i &  & \text{(by \cref{d.0.1})}  \\
                    & = (a + (c + e)) + (b + (d + f)) i &  & (a, b, c, d, e, f \in \R) \\
                    & = x + ((c + e) + (d + f)i)        &  & \text{(by \cref{d.0.1})}  \\
                    & = x + (y + z)                     &  & \text{(by \cref{d.0.1})}
      \end{align*}
      and
      \begin{align*}
        (x \cdot y) \cdot z & = ((ac - bd) + (ad + bc) i) \cdot z   &  & \text{(by \cref{d.0.1})}  \\
                            & = ((ac - bd) e - (ad + bc) f)                                        \\
                            & \quad + ((ac - bd) f + (ad + bc) e) i &  & \text{(by \cref{d.0.1})}  \\
                            & = (a (ce - df) - b (cf + de))                                        \\
                            & \quad + (a (cf + de) + b(ce - df)) i  &  & (a, b, c, d, e, f \in \R) \\
                            & = x \cdot ((ce - df) + (cf + de) i)   &  & \text{(by \cref{d.0.1})}  \\
                            & = x \cdot (y \cdot z).                &  & \text{(by \cref{d.0.1})}
      \end{align*}
    \item[For \ref{f3}:]
      See \cref{d.0.3}.
    \item[For \ref{f4}:]
      See \cref{d.0.3}.
    \item[For \ref{f5}:]
      We have
      \begin{align*}
        x \cdot (y + z) & = x \cdot ((c + e) + (d + f) i)               &  & \text{(by \cref{d.0.1})}          \\
                        & = (a (c + e) - b (d + f))                                                            \\
                        & \quad + (a (d + f) + b (c + e)) i             &  & \text{(by \cref{d.0.1})}          \\
                        & = (ac + ae - bd - bf) + (ad + af + bc + be) i &  & (a, b, c, d, e, f \in \R)         \\
                        & = (ac - bd) + (ad + bc) i                                                            \\
                        & \quad + (ae - bf) + (af + be) i               &  & \text{(by \ref{f1} and \ref{f2})} \\
                        & = xy + xz.                                    &  & \text{(by \cref{d.0.1})}
      \end{align*}
      From all cases above we conclude by \cref{c.0.1} that \(\C\) is a field.
  \end{description}
\end{proof}

\begin{defn}\label{d.0.4}
  The \textbf{(complex) conjugate} of a complex number \(a + bi\) is the complex number \(a - bi\).
  We denote the conjugate of the complex number \(z\) by \(\conj{z}\).
\end{defn}

\begin{thm}\label{d.2}
  Let \(z\) and \(w\) be complex numbers.
  Then the following statements are true.
  \begin{enumerate}
    \item \(\conj{\conj{z}} = z\).
    \item \(\conj{z + w} = \conj{z} + \conj{w}\).
    \item \(\conj{zw} = \conj{z} \cdot \conj{w}\).
    \item \(\conj{\frac{z}{w}} = \frac{\conj{z}}{\conj{w}}\) if \(w \neq 0\).
    \item \(z \in \R\) iff \(\conj{z} = z\).
  \end{enumerate}
\end{thm}

\begin{proof}[\pf{d.2}(a)]
  We have
  \begin{align*}
    \conj{\conj{z}} & = \conj{\Re(z) - \Im(z) i} &  & \text{(by \cref{d.0.4})} \\
                    & = \Re(z) + \Im(z) i        &  & \text{(by \cref{d.0.4})} \\
                    & = z.
  \end{align*}
\end{proof}

\begin{proof}[\pf{d.2}(b)]
  We have
  \begin{align*}
    \conj{z + w} & = \conj{\Re(z) + \Re(w) + (\Im(z) + \Im(w)) i} &  & \text{(by \cref{d.0.1})} \\
                 & = \Re(z) + \Re(w) - (\Im(z) + \Im(w)) i        &  & \text{(by \cref{d.0.4})} \\
                 & = \Re(z) - \Im(z) i + \Re(w) - \Im(w) i        &  & \text{(by \cref{d.1})}   \\
                 & = \conj{z} + \conj{w}.                         &  & \text{(by \cref{d.0.4})}
  \end{align*}
\end{proof}

\begin{proof}[\pf{d.2}(c)]
  We have
  \begin{align*}
    \conj{zw} & = \conj{(\Re(z) \Re(w) - \Im(z) \Im(w)) + (\Re(z) \Im(w) + \Im(z) \Re(w)) i} &  & \text{(by \cref{d.0.1})} \\
              & = (\Re(z) \Re(w) - \Im(z) \Im(w)) - (\Re(z) \Im(w) + \Im(z) \Re(w)) i        &  & \text{(by \cref{d.0.4})} \\
              & = (\Re(z) - \Im(z) i) (\Re(w) - \Im(w) i)                                    &  & \text{(by \cref{d.0.1})} \\
              & = \conj{z} \cdot \conj{w}.                                                   &  & \text{(by \cref{d.0.4})}
  \end{align*}
\end{proof}

\begin{proof}[\pf{d.2}(d)]
  We have
  \begin{align*}
    \conj{\frac{z}{w}} & = \conj{z \cdot w^{-1}}                                                                                        &  & \text{(by \cref{d.0.3})}  \\
                       & = \conj{z} \cdot \conj{w^{-1}}                                                                                 &  & \text{(by \cref{d.2}(c))} \\
                       & = \conj{z} \cdot \pa{\conj{\frac{\Re(w)}{(\Re(w))^2 + (\Im(w))^2} - \frac{\Im(w)}{(\Re(w))^2 + (\Im(w))^2} i}} &  & \text{(by \cref{d.0.3})}  \\
                       & = \conj{z} \cdot \pa{\frac{\Re(w)}{(\Re(w))^2 + (\Im(w))^2} + \frac{\Im(w)}{(\Re(w))^2 + (\Im(w))^2} i}        &  & \text{(by \cref{d.0.4})}  \\
                       & = \conj{z} \cdot (\Re(w) - \Im(w))^{-1}                                                                        &  & \text{(by \cref{d.0.3})}  \\
                       & = \conj{z} \cdot \conj{w}^{-1}                                                                                 &  & \text{(by \cref{d.0.4})}  \\
                       & = \frac{\conj{z}}{\conj{w}}.                                                                                   &  & \text{(by \cref{d.0.3})}
  \end{align*}
\end{proof}

\begin{proof}[\pf{d.2}(e)]
  We have
  \begin{align*}
         & z \in \R                                                            \\
    \iff & \Im(z) = 0                            &  & \text{(by \cref{d.0.2})} \\
    \iff & \Re(z) + \Im(z) i = \Re(z) - \Im(z) i                               \\
    \iff & \conj{z} = z.                         &  & \text{(by \cref{d.0.4})}
  \end{align*}
\end{proof}

\begin{defn}\label{d.0.5}
  Let \(z = a + bi\), where \(a, b \in \R\).
  The \textbf{absolute value} (or \textbf{modulus}) of \(z\) is the real number \(\sqrt{a^2 + b^2}\).
  We denote the absolute value of \(z\) by \(\abs{z}\).

  Observe that \(z \conj{z} = \abs{z}^2\).
  The fact that the product of a complex number and its conjugate is real provides an easy method for determining the quotient of two complex numbers;
  for if \(c + di \neq 0\), then
  \[
    \frac{a + bi}{c + di} = \frac{a + bi}{c + di} \cdot \frac{c - di}{c - di} = \frac{(ac + bd) + (bc - ad) i}{c^2 + d^2} = \frac{ac + bd}{c^2 + d^2} + \frac{bc - ad}{c^2 + d^2} i.
  \]
\end{defn}

\begin{thm}\label{d.3}
  Let \(z\) and \(w\) denote any two complex numbers.
  Then the following statements are true.
  \begin{enumerate}
    \item \(\abs{zw} = \abs{z} \cdot \abs{w}\).
    \item \(\abs{\frac{z}{w}} = \frac{\abs{z}}{\abs{w}}\) if \(w \neq 0\).
    \item \(\abs{z + w} \leq \abs{z} + \abs{w}\).
    \item \(\abs{z} - \abs{w} \leq \abs{z + w}\).
  \end{enumerate}
\end{thm}

\begin{proof}[\pf{d.3}(a)]
  By \cref{d.2} we have
  \[
    \abs{zw}^2 = (zw) \conj{(zw)} = (zw) (\conj{z} \cdot \conj{w}) = (z \conj{z}) (w \conj{w}) = \abs{z}^2 \abs{w}^2
  \]
  and thus \(\abs{zw} = \abs{z} \abs{w}\).
\end{proof}

\begin{proof}[\pf{d.3}(b)]
  We have
  \begin{align*}
    \abs{z} & = \abs{\frac{z}{w} w}       &  & \text{(by \cref{d.0.3})}  \\
            & = \abs{\frac{z}{w}} \abs{w} &  & \text{(by \cref{d.3}(a))}
  \end{align*}
  and thus
  \[
    \frac{\abs{z}}{\abs{w}} = \abs{\frac{z}{w}}.
  \]
\end{proof}

\begin{proof}[\pf{d.3}(c)]
  For any complex number \(x = a + bi\), where \(a, b \in \R\), observe that
  \[
    x + \conj{x} = (a + bi) + (a - bi) = 2a \leq 2 \sqrt{a^2 + b^2} = 2 \abs{x}.
  \]
  Thus \(x + \conj{x}\) is real and satisfies the inequality \(x + \conj{x} \leq 2 \abs{x}\).
  Taking \(x = w \conj{z}\), we have, by \cref{d.2} and \cref{d.3}(a),
  \[
    w \conj{z} + \conj{w} z \leq 2 \abs{w \conj{z}} = 2 \abs{w} \abs{\conj{z}} = 2 \abs{z} \abs{w}.
  \]
  Using \cref{d.2} again gives
  \begin{align*}
    \abs{z + w}^2 & = (z + w) \conj{(z + w)} = (z + w) (\conj{z} + \conj{w}) = z \conj{z} + w \conj{z} + z \conj{w} + w \conj{w} \\
                  & \leq \abs{z}^2 + 2 \abs{z} \abs{w} + \abs{w}^2 = (\abs{z} + \abs{w})^2.
  \end{align*}
  By taking square roots, we obtain \cref{d.3}(c).
\end{proof}

\begin{proof}[\pf{d.3}(d)]
  From \cref{d.3}(a) and (c), it follows that
  \[
    \abs{z} = \abs{(z + w) - w} \leq \abs{z + w} + \abs{-w} = \abs{z + w} + \abs{w}.
  \]
  So
  \[
    \abs{z} - \abs{w} \leq \abs{z + w},
  \]
  proving \cref{d.3}(d).
\end{proof}

\begin{defn}\label{d.0.6}
  It is interesting as well as useful that complex numbers have both a geometric and an algebraic representation.
  Suppose that \(z = a + bi\), where \(a\) and \(b\) are real numbers.
  We may represent \(z\) as a vector in the complex plane.
  Notice that, as in \(\R^2\), there are two axes, the \textbf{real axis} and the \textbf{imaginary axis}.
  The real and imaginary parts of \(z\) are the first and second coordinates, and the absolute value of \(z\) gives the length of the vector \(z\).
  It is clear that addition of complex numbers may be represented as in \(\R^2\) using the parallelogram law.

  In \cref{sec:2.7}, we introduce Euler's formula.
  The special case \(e^{i \theta} = \cos(\theta) + i \sin(\theta)\) is of particular interest.
  \(e^{i \theta}\) is the unit vector that makes an angle \(\theta\) with the positive real axis.
  We see that any nonzero complex number \(z\) may be depicted as a multiple of a unit vector, namely, \(z = \abs{z} e^{i \phi}\), where \(\phi\) is the angle that the vector \(z\) makes with the positive real axis.
  Thus multiplication, as well as addition, has a simple geometric interpretation:
  If \(z = \abs{z} e^{i \theta}\) and \(w = \abs{w} e^{i \omega}\) are two nonzero complex numbers, then from the properties established in \cref{sec:2.7} and \cref{d.3}, we have
  \[
    zw = \abs{z} e^{i \theta} \cdot \abs{w} e^{i \omega} = \abs{zw} e^{i (\theta + \omega)}.
  \]
  So \(zw\) is the vector whose length is the product of the lengths of \(z\) and \(w\), and makes the angle \(\theta + \omega\) with the positive real axis.
\end{defn}

\begin{thm}[The Fundamental Theorem of Algebra]\label{d.4}
  Suppose that \(p(z) = a_n z^n + a_{n - 1} z^{n - 1} + \cdots + a_1 z + a_0\) is a polynomial in \(\ps{\C}\) of degree \(n \geq 1\).
  Then \(p(z)\) has a zero.
\end{thm}

\begin{proof}[\pf{d.4}]
  We want to find \(z_0\) in \(\C\) such that \(p(z_0) = 0\).
  Let \(m\) be the greatest lower bound of \(\set{\abs{p(z)} : z \in \C}\).
  For \(\abs{z} = s > 0\), we have
  \begin{align*}
    \abs{p(z)} & = \abs{a_n z^n + a_{n - 1} z^{n - 1} + \cdots + a_0}                                                              \\
               & \geq \abs{a_n} \abs{z^n} - \abs{a_{n - 1}} \abs{z^{n - 1}} - \cdots - \abs{a_0} &  & \text{(by \cref{d.3}(a)(d))} \\
               & = \abs{a_n} s^n - \abs{a_{n - 1}} s^{n - 1} - \cdots - \abs{a_0}                                                  \\
               & = s^{n} \pa{\abs{a_n} - \abs{a_{n - 1}} s^{-1} - \cdots - \abs{a_0} s^{-n}}.
  \end{align*}
  Because the last expression approaches infinity as \(s\) approaches infinity, the set \(\set{\abs{p(z)} : (z \in \C) \land (\abs{p(z)} \leq m + 1)}\) is non-empty and closed.
  Since \(p\) is continuous, we know that the set \(D = \set{z \in \C : \abs{p(z)} \leq m + 1}\) is also closed.
  It follows that \(m\) is the greatest lower bound of \(\set{\abs{p(z)} : z \in D}\).
  Because \(D\) is closed and bounded and \(p\) is continuous, there exists \(z_0\) in \(D\) such that \(\abs{p(z_0)} = m\).
  We want to show that \(m = 0\).
  We argue by contradiction.

  Assume that \(m \neq 0\).
  Let \(q(z) = \frac{p(z + z_0)}{p(z_0)}\).
  Then \(q(z)\) is a polynomial of degree \(n\), \(q(0) = 1\), and \(\abs{q(z)} \geq 1\) for all \(z \in \C\)
  (this is true since \(\abs{p(z + z_0)} \geq \abs{p(z_0)}\)).
  So we may write
  \[
    q(z) = 1 + b_k z^k + b_{k + 1} z^{k + 1} + \cdots + b_n z^n,
  \]
  where \(k \in \N\), \(1 \leq k \leq n\) and \(b_k \neq 0\).
  Because \(-\frac{\abs{b_k}}{b_k}\) has modulus one (\(\abs{-\frac{\abs{b_k}}{b_k}} = 1\)), we may pick a real number \(\theta\) such that \(e^{i k \theta} = - \frac{\abs{b_k}}{b_k}\), or \(e^{i k \theta} b_k = -\abs{b_k}\).
  For any \(r > 0\), we have
  \begin{align*}
    q(r e^{i \theta}) & = 1 + b_k r^k e^{i k \theta} + b_{k + 1} r^{k + 1} e^{i (k + 1) \theta} + \cdots + b_n r^n e^{i n \theta} \\
                      & = 1 - \abs{b_k} r^k + b_{k + 1} r^{k + 1} e^{i (k + 1) \theta} + \cdots + b_n r^n e^{i n \theta}.
  \end{align*}
  Choose \(r\) small enough so that \(1 - \abs{b_k} r^k > 0\).
  Then
  \begin{align*}
    \abs{q(r e^{i \theta})} & \leq \abs{1 - \abs{b_k} r^k} + \abs{b_{k + 1} r^{k + 1} e^{i (k + 1) \theta}} + \cdots + \abs{b_n r^n e^{i n \theta}} &  & \text{(by \cref{d.3}(c))} \\
                            & = 1 - \abs{b_k} r^k + \abs{b_{k + 1}} r^{k + 1} + \cdots + \abs{b_n} r^n                                              &  & \text{(by \cref{d.3}(a))} \\
                            & = 1 - r^k \pa{\abs{b_k} - \abs{b_{k + 1}} r - \cdots - \abs{b_n} r^{n - k}}.
  \end{align*}
  Now choose \(r\) even smaller, if necessary, so that the expression within the brackets is positive.
  We obtain that \(\abs{q(r e^{i \theta})} < 1\).
  But this is a contradiction.
\end{proof}

\begin{cor}\label{d.0.7}
  If \(p(z) = a_n z^n + a_{n - 1} z^{n - 1} + \cdots + a_1 z + a_0\) is a polynomial of degree \(n \geq 1\) with complex coefficients, then there exist complex numbers \(\seq{c}{1,,n}\) (not necessarily distinct) such that
  \[
    p(z) = a_n (z - c_1)(z - c_2) \cdots (z - c_n).
  \]
\end{cor}

\begin{proof}[\pf{d.0.7}]
  We use induction on \(n\).
  For \(n = 1\), let \(p \in \ps{\C}\) be the function
  \[
    \forall z \in \C, p(z) = a_1 z + a_0
  \]
  where \(\seq{a}{0,1} \in \C\) and \(a_1 \neq 0\).
  Then we have
  \[
    p(z) = a_1 z + a_0 = a_1 (z - (-a_0 / a_1))
  \]
  and thus the base case holds.
  Suppose inductively that \cref{d.0.7} is true for some \(n \geq 1\).
  Then we need to show that \cref{d.0.7} is true for \(n + 1\).
  Let \(p \in \ps{\C}\) be the function
  \[
    \forall z \in \C, p(z) = a_{n + 1} z^{n + 1} + a_n z^n + \cdots + a_1 z + a_0
  \]
  where \(\seq{a}{0,,n+1} \in \C\) and \(a_{n + 1} \neq 0\).
  By \cref{d.4} we know that there exists a \(c_{n + 1} \in \C\) such that \(p(c_{n + 1}) = 0\).
  By \cref{e.0.3} there exists a polynomial \(q_1 \in \ps{\C}\) such that
  \[
    \forall z \in \C, p(z) = q_1(z) (z - c_{n + 1}).
  \]
  Since \(a_{n + 1} \neq 0\), we can rewrite the above equation as
  \[
    \forall z \in \C, p(z) = a_{n + 1} (z - c_{n + 1}) q_2(z)
  \]
  where \(q_2 = \frac{1}{a_{n + 1}} q_1\).
  Since \(p\) has degree \(n + 1\), we know that \(q_2\) has degree \(n\).
  Then there exist \(\seq{b}{0,,n-1} \in \C\) such that
  \[
    \forall z \in \C, q_2(z) = b_0 + b_1 z + \cdots + b_n z^n.
  \]
  Note that we must have \(b_n = 1\), otherwise the \((n + 1)\)th coefficient of \(p\) is \(a_n b_n\) instead of \(a_n\).
  By induction hypothesis \(q_2\) can be written as
  \[
    \forall z \in \C, q_2(z) = (z - c_1) (z - c_2) \cdots (z - c_n)
  \]
  where \(\seq{c}{1,,n} \in \C\).
  Then we have
  \begin{align*}
    \forall z \in \C, p(z) & = a_{n + 1} (z - c_{n + 1}) q_2(z)                               \\
                           & = a_{n + 1} (z - c_1) (z - c_2) \cdots (z - c_n) (z - c_{n + 1})
  \end{align*}
  and this closes the induction.
\end{proof}

\begin{defn}\label{d.0.8}
  A field is called \textbf{algebraically closed} if it has the property that every polynomial of positive degree with coefficients from that field factors as a product of polynomials of degree \(1\).
  Thus \cref{d.0.7} asserts that the field of complex numbers is algebraically closed.
\end{defn}

\chapter{Polynomial}\label{ch:e}


%------------------------------------------------------------------------------
% Back matters.
%------------------------------------------------------------------------------

\backmatter

\end{document}

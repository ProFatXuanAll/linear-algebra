% We use chapter structure.
\documentclass[12pt,oneside]{book}

%==============================================================================
% Preamble.
%==============================================================================

% Correctly showing characters outside ASCII.
\usepackage[T1]{fontenc}
% File is written and read with utf8 encoding.
\usepackage[utf8]{inputenc}
% Set paging layout.
\usepackage[margin=1.2in]{geometry}
% Including `amsfonts'.  Must be loaded before `mathtools'.
\usepackage{amssymb}
% Including `amsmath' and fixing bugs for `amsmath'.
\usepackage{mathtools}
% Must be loaded after `amsmath' and `mathtools'.
\usepackage{amsthm}
% Automatically adjust character spacing at margins.
\usepackage{microtype}
% Provide further utilities and fix bugs for `enumerate', `itemize' and
% `description'.
\usepackage{enumitem}
% Provide better quoting environment.
\usepackage{dirtytalk}
% Automatically add hyperlinks to labels/refs.  Must be loaded after all
% packages above and before `cleveref'.  Recommend to use with `natbib' when
% you need bibtex.
\usepackage{hyperref}

\hypersetup{         % This macro come with `hyperref'.
    colorlinks=true, % Color hyperlinks.
    linkcolor=blue,  % Color local hyperlinks with blue.
    urlcolor=cyan,   % Color url links with cyan.
}

% Must be loaded after `hyperref'.  We always capitalize each cross-references'
% type name.  See `cleveref' for details.
\usepackage[capitalize]{cleveref}

%------------------------------------------------------------------------------
% Define environments.
%------------------------------------------------------------------------------

% Text inside the body of theorem-like environments are set to Roman font.
% theorem-like environments share their counters, counters follow section and
% reset in every sections.  Exercises has their owned counter.  Notes do not
% use counter.  See `amsthm' for details.
\theoremstyle{definition}
\newtheorem{ax}{Axiom}[section]
\newtheorem{cor}[ax]{Corollary}
\newtheorem{defn}[ax]{Definition}
\newtheorem{eg}[ax]{Example}
\newtheorem{ex}{Exercise}[section]
\newtheorem{lem}[ax]{Lemma}
\newtheorem*{note}{Note}
\newtheorem{prop}[ax]{Proposition}
\newtheorem{rem}[ax]{Remark}
\newtheorem{thm}[ax]{Theorem}

% In `enumerate' enviroments, items' label are alphabets and surrounded by
% parentheses.  See `enumitem' for details.
\renewcommand{\labelenumi}{\textnormal{(}\alph{enumi}\textnormal{)}}

% Formatting equations tag appearence.  See `mathtools' for details.
\renewcommand{\theequation}{\thechapter.\thesection.\arabic{equation}}
\numberwithin{equation}{section}

%------------------------------------------------------------------------------
% Define operators and symbols.
%------------------------------------------------------------------------------

% Define common operators with paired of delimiters.  Always use star versions
% of these operator to automatically adjust height.  See `mathtools' for
% details.

% Absolute value.
\DeclarePairedDelimiter{\absTmp}{\lvert}{\rvert}
\newcommand{\abs}[1]{\absTmp*{#1}}
% Ceiling.
\DeclarePairedDelimiter{\ceilTmp}{\lceil}{\rceil}
\newcommand{\ceil}[1]{\ceilTmp*{#1}}
% Floor.
\DeclarePairedDelimiter{\floorTmp}{\lfloor}{\rfloor}
\newcommand{\floor}[1]{\floorTmp*{#1}}
% Parenthese.
\DeclarePairedDelimiter{\pTmp}{\lparen}{\rparen}
\newcommand{\p}[1]{\pTmp*{#1}}
% Bracket.
\DeclarePairedDelimiter{\bkTmp}{\lbrack}{\rbrack}
\newcommand{\bk}[1]{\bkTmp*{#1}}
% Brace.
\DeclarePairedDelimiter{\BTmp}{\lbrace}{\rbrace}
\newcommand{\B}[1]{\BTmp*{#1}}
% Set.
\newcommand{\set}[1]{\B{#1}}

% Define common symbols.  See `amsmath' section 9.2 for details.

% Fields.
\newcommand{\field}[1]{\mathbf{#1}}
% General field.
\newcommand{\F}{\field{F}}
% Complex number field.
\newcommand{\C}{\field{C}}
% Natural number field.
\newcommand{\N}{\field{N}}
% Rational number field.
\newcommand{\Q}{\field{Q}}
% Real number field.
\newcommand{\R}{\field{R}}
% Integer number field.
\newcommand{\Z}{\field{Z}}

% Vector spaces.
\newcommand{\vs}[1]{\mathsf{#1}}
% General vector space.
\newcommand{\V}{\vs{V}}
% Metric spaces.
\newcommand{\ms}[3]{{\vs{M}_{{#1} \times #2}({#3})}}
% Polynomial spaces.
\newcommand{\ps}[1]{{\vs{P}\p{#1}}}

% n-tuple.
\newcommand{\tp}[2]{{\p{{#1}_{1}, {#1}_{2}, \dots, {#1}_{#2}}}}

% 0 vector.
\newcommand{\zv}{\mathit{0}}
% 0 metric.
\newcommand{\zm}{\mathit{O}}

% Formatting exercises section.
\newcommand{\exercisesection}{
    \begin{center}
        \textbf{EXERCISES}
    \end{center}
}

%==============================================================================
% Document.
%==============================================================================

\begin{document}

%------------------------------------------------------------------------------
% Front matters.
%------------------------------------------------------------------------------

\frontmatter

% Author informations.
\title{Linear Algebra}
\author{ProFatXuanAll}
\maketitle

% Table of contents.
\tableofcontents

%------------------------------------------------------------------------------
% Main matters.
%------------------------------------------------------------------------------

\mainmatter

% All chapters are in separated files.  We include them here.
\chapter{Vector Spaces}\label{ch:1}

% All sections are in separated files.  We include them here.
\section{Introduction}\label{sec 1.1}

\begin{note}
    Experiments show that if two like quantities act together, their effect is predictable.
    In this case, the vectors used to represent these quantities can be combined to form a resultant vector that represents the combined effects of the original quantities.
    This resultant vector is called the \emph{sum} of the original vectors, and the rule for their combination is called the \emph{parallelogram law}.
\end{note}

\begin{axiom}[Parallelogram Law for Vector Addition]\label{ac 1.1.1}
    The sum of two vectors $x$ and $y$ that act at the same point $P$ is the vector beginning at $P$ that is represented by the diagonal of parallelogram having $x$ and $y$ as adjacent sides.
\end{axiom}

\begin{note}
    Since a vector beginning at the origin is completely determined by its endpoint, we sometimes refer to \emph{the point $x$} rather than \emph{the endpoint of the vector $x$} if $x$ is a vector emanating from the origin.
\end{note}

\begin{note}
    Besides the operation of vector addition, there is another natural operation that can be performed on vectors
    --- the length of a vector may be magnified or contracted.
    This operation, called \emph{scalar multiplication}, consists of multiplying the vector by a real number.
    If the vector $x$ is represented by an arrow, then for any real number $t$, the vector $tx$ is represented by an arrow in the same direction if $t \geq 0$ and in the opposite direction if $t < 0$.
    The length of the arrow $tx$ is $\abs*{t}$ times the length of the arrow $x$.
    Two nonzero vectors $x$ and $y$ are called \textbf{parallel} if $y = tx$ for some nonzero real number $t$.
    (Thus nonzero vectors having the same or opposite directions are parallel.)
\end{note}
\section{Vector Spaces}\label{sec:1.2}

\begin{defn}\label{1.2.1}
    A \textbf{vector space} (or \textbf{linear space}) \(\V\) over a field \(\F\) consists of a set on which two operations (called \textbf{addition} and \textbf{scalar multiplication}, respectively) are defined so that for each pair of elements \(x\), \(y\) in \(\V\) there is a unique element \(x + y\) in \(\V\), and for each element \(a\) in \(\F\) and each element \(x\) in \(\V\) there is a unique element \(ax\) in \(\V\), such that the following conditions hold.
    \begin{enumerate}[label=(VS \arabic*), ref=VS \arabic*]
        \item\label{VS1}
        For all \(x, y\) in \(\V\), \(x + y = y + x\)
        (commutativity of addition).
        \item\label{VS2}
        For all \(x, y, z\) in \(\V\), \(\p{x + y} + z = x + \p{y + z}\)
        (associativity of addition).
        \item\label{VS3}
        There exists an element in \(\V\) denoted by \(0\) such that \(x + 0 = x\) for each \(x\) in \(\V\).
        \item\label{VS4}
        For each element \(x\) in \(\V\) there exists an element \(y\) in \(\V\) such that \(x + y = 0\).
        \item\label{VS5}
        For each element \(x\) in \(\V\), \(1x = x\).
        \item\label{VS6}
        For each pair of elements \(a, b\) in \(\F\) and each element \(x\) in \(\V\), \(\p{ab} x = a \p{bx}\).
        \item\label{VS7}
        For each element \(a\) in \(\F\) and each pair of elements \(x, y\) in \(\V\), \(a \p{x + y} = ax + ay\).
        \item\label{VS8}
        For each pair of elements \(a, b\) in \(\F\) and each element \(x\) in \(\V\), \(\p{a + b} x = ax + bx\).
    \end{enumerate}
    The elements \(x + y\) and \(ax\) are called the \textbf{sum} of \(x\) and \(y\) and the \textbf{product} of \(a\) and \(x\), respectively.
\end{defn}

\begin{defn}\label{1.2.2}
    The elements of the field \(\F\) are called \textbf{scalars} and the elements of the vector space \(\V\) are called \textbf{vectors}.
\end{defn}

\begin{note}
    A vector space is frequently discussed in the text without explicitly mentioning its field of scalars.
    The reader is cautioned to remember, however, that \emph{every vector space is regarded as a vector space over a given field, which is denoted by \(\F\)}.
    Occasionally we restrict our attention to the fields of real and complex numbers, which are denoted \(\R\) and \(\C\), respectively.
\end{note}

\begin{note}
    \ref{VS2} permits us to unambiguously define the addition of any finite number of vectors
    (without the use of parentheses).
\end{note}

\begin{defn}\label{1.2.3}
    An object of the form \(\tp{a}{n}\), where the entries \(a_{1}, a_{2}, \dots, a_{n}\) are elements of a field \(\F\), is called an \textbf{\(n\)-tuple} with entries from \(\F\).
    The elements \(a_{1}, a_{2}, \dots, a_{n}\) are called the \textbf{entries} or \textbf{components} of the \(n\)-tuple.
    Two \(n\)-tuples \(\tp{a}{n}\) and \(\tp{b}{n}\) with entries from a field \(\F\) are called \textbf{equal} if \(a_i = b_i\) for \(i = 1, 2, \dots, n\).
\end{defn}

\begin{eg}\label{1.2.4}
    The set of all \(n\)-tuples with entries from a field \(\F\) is denoted by \(\vs{F}^n\).
    This set is a vector space over \(\F\) with the operations of coordinatewise addition and scalar multiplication;
    that is, if \(u = \tp{a}{n} \in \vs{F}^n\), \(v = \tp{b}{n} \in \vs{F}^n\), and \(c \in \F\), then
    \[
        u + v = (a_{1} + b_{1}, a_{2} + b_{2}, \dots, a_{n} + b_{n}) \quad \text{ and } \quad cu = \tp{ca}{n}.
    \]
\end{eg}


\section{Subspaces}\label{sec:1.3}

\begin{defn}\label{1.3.1}
  A subset \(\W\) of a vector space \(\V\) over a field \(\F\) is called a \textbf{subspace} of \(\V\) over \(\F\) if \(\W\) is a vector space over \(\F\) with the operations of addition and scalar multiplication defined on \(\V\).
\end{defn}

\begin{eg}\label{1.3.2}
  In any vector space \(\V\) over \(\F\), note that \(\V\) and \(\set{\zv}\) are subspaces.
  The latter is called the \textbf{zero subspace} of \(\V\) over \(\F\).
\end{eg}

\begin{proof}
  Since \(\V \subseteq \V\) and \(\V\) is a vector space over \(\F\) with the operations of addition and scalar multiplication defined on \(\V\), by \cref{1.3.1} we know that \(\V\) is a subspace of \(\V\) over \(\F\).
  Since \(\zv \in \V\) (by \ref{vs3}), we know that \(\set{\zv} \subseteq \V\).
  Thus by \cref{ex:1.2.11} and \cref{1.3.1} \(\set{\zv}\) is a subspace of \(\V\) over \(\F\).
\end{proof}

\begin{thm}\label{1.3}
  Let \(\V\) be a vector space over \(\F\) and \(\W\) a subset of \(\V\).
  Then \(\W\) is a subspace of \(\V\) over \(\F\) if and only if the following three conditions hold for the operations defined in \(\V\).
  \begin{enumerate}
    \item \(\zv \in \W\).
    \item (\(\W\) is \textbf{closed under addition}.)
          \(x + y \in \W\) whenever \(x \in \W\) and \(y \in \W\).
    \item (\(\W\) is \textbf{closed under scalar multiplication}.)
          \(cx \in \W\) whenever \(c \in \F\) and \(x \in \W\).
  \end{enumerate}
\end{thm}

\begin{proof}
  If \(\W\) is a subspace of \(\V\) over \(\F\), then \(\W\) is a vector space over \(\F\) with the operations of addition and scalar multiplication defined on \(\V\).
  Hence conditions (b) and (c) hold, and there exists a vector \(\zv' \in \W\) such that \(x + \zv' = x\) for each \(x \in \W\).
  But also \(x + \zv = x\), and thus \(\zv' = \zv\) by \cref{1.1}.
  So condition (a) holds.

  Since properties \ref{vs1}, \ref{vs2}, \ref{vs5}, \ref{vs6}, \ref{vs7}, and \ref{vs8} hold for all vectors in the vector space, these properties automatically hold for the vectors in any subset.
  Thus if conditions (a), (b), and (c) hold, then \(\W\) is a subspace of \(\V\) over \(\F\) if the additive inverse of each vector in \(\W\) lies in \(\W\).
  But if \(x \in W\), then \(\p{-1} x \in \W\) by condition (c), and \(-x = \p{-1} x\) by \cref{1.2}.
  Hence \(\W\) is a subspace of \(\V\) over \(\F\).
\end{proof}

\begin{defn}\label{1.3.3}
  The \textbf{transpose} \(\tp{A}\) of an \(m \times n\) matrix \(A\) is the \(n \times m\) matrix obtained from \(A\) by interchanging the rows with the columns;
  that is, \(\p{\tp{A}}_{i j} = A_{j i}\).
\end{defn}

\begin{defn}\label{1.3.4}
  A \textbf{symmetric matrix} is a matrix \(A\) such that \(\tp{A} = A\).
  Clearly, a symmetric matrix must be square.
\end{defn}

\begin{eg}\label{1.3.5}
  The set \(\W\) of all symmetric matrices in \(\ms{n}{n}{\F}\) is a subspace of \(\ms{n}{n}{\F}\) over \(\F\).
\end{eg}

\begin{proof}
  Observe that
  \begin{itemize}
    \item The zero matrix is equal to its transpose and hence belongs to \(\W\).
    \item If \(A \in \W\) and \(B \in \W\), then \(\tp{A} = A\) and \(\tp{B} = B\).
          Thus by \cref{ex:1.3.3} \(\tp{\p{A + B}} = \tp{A} + \tp{B} = A + B\), so that \(A + B \in \W\).
    \item If \(A \in \W\), then \(\tp{A} = A\).
          So for any \(a \in \F\), we have \(\tp{\p{aA}} = a\tp{A} = aA\) by \cref{ex:1.3.3}
          Thus \(aA \in W\).
  \end{itemize}
  Thus by \cref{1.3} \(\W\) is a subspace of \(\ms{n}{n}{\F}\) over \(\F\).
\end{proof}

\begin{eg}\label{1.3.6}
  Let \(n\) be a nonnegative integer, and let \(\ps[n]{\F}\) consist of all polynomials in \(\ps{\F}\) having degree less than or equal to \(n\).
  Then \(\ps[n]{\F}\) is a subspace of \(\ps{\F}\) over \(\F\).
\end{eg}

\begin{proof}
  Since the zero polynomial has degree \(-1\), it is in \(\ps[n]{\F}\).
  Moreover, the sum of two polynomials with degrees less than or equal to \(n\) is another polynomial of degree less than or equal to \(n\), and the product of a scalar and a polynomial of degree less than or equal to \(n\) is a polynomial of degree less than or equal to \(n\).
  So \(\ps[n]{\F}\) is closed under addition and scalar multiplication.
  It therefore follows from \cref{1.3} that \(\ps[n]{\F}\) is a subspace of \(\ps{\F}\) over \(\F\).
\end{proof}

\begin{eg}\label{1.3.7}
  Let \(\cfs{\R}\) denote the set of all continuous real-valued functions defined on \(\R\).
  Clearly \(\cfs{\R}\) is a subset of the vector space \(\fs{\R}{\R}\) defined in \cref{1.2.10} of \cref{sec:1.2}.
  We claim that \(\cfs{\R}\) is a subspace of \(\fs{\R}{\R}\) over \(\R\).
\end{eg}

\begin{proof}
  First note that the zero of \(\fs{\R}{\R}\) is the constant function defined by \(f\p{t} = 0\) for all \(t \in \R\).
  Since constant functions are continuous, we have \(f \in \cfs{\R}\).
  Moreover, the sum of two continuous functions is continuous, and the product of a real number and a continuous function is continuous.
  So \(\cfs{\R}\) is closed under addition and scalar multiplication and hence is a subspace of \(\fs{\R}{\R}\) over \(\R\) by \cref{1.3}.
\end{proof}

\begin{eg}\label{1.3.8}
  An \(n \times n\) matrix \(M\) is called a \textbf{diagonal matrix} if \(M_{i j} = 0\) whenever \(i \neq j\), that is, if all its nondiagonal entries are zero.
  Then the set of diagonal matrices is a subspace of \(\ms{n}{n}{\F}\) over \(\F\).
\end{eg}

\begin{proof}
  Clearly the zero matrix is a diagonal matrix because all of its entries are \(0\).
  Moreover, if \(A\) and \(B\) are diagonal \(n \times n\) matrices, then whenever \(i \neq j\),
  \[
    \p{A + B}_{i j} = A_{i j} + B_{i j} = 0 + 0 = 0 \quad \text{and} \quad \p{cA}_{i j} = cA_{i j} = c0 = 0
  \]
  for any scalar \(c \in \F\).
  Hence \(A + B\) and \(cA\) are diagonal matrices for any scalar \(c \in \F\).
  Therefore the set of diagonal matrices is a subspace of \(\ms{n}{n}{\F}\) over \(\F\) by \cref{1.3}.
\end{proof}

\begin{eg}\label{1.3.9}
  The \textbf{trace} of an \(n \times n\) matrix \(M\), denoted \(\tr{M}\), is the sum of the diagonal entries of \(M\);
  that is,
  \[
    \tr{M} = M_{1 1} + M_{2 2} + \dots + M_{n n}.
  \]
  The set \(\W\) of \(n \times n\) matrices having trace equal to \(0\) is a subspace of \(\ms{n}{n}{\F}\) over \(\F\).
\end{eg}

\begin{proof}
  Clearly we have \(\W \subseteq \ms{n}{n}{\F}\), \(\tr{\zm} = 0\) and \(\zm \in \W\).
  Moreover, if \(A, B \in \W\), then
  \begin{align*}
    \tr{A + B} & = \p{A + B}_{1 1} + \p{A + B}_{2 2} + \dots + \p{A + B}_{n n}               &  & \text{(by \cref{1.3.9})}  \\
               & = A_{1 1} + B_{1 1} + A_{2 2} + B_{2 2} + \dots + A_{n n} + B_{n n}         &  & \text{(by \cref{1.2.9})}  \\
               & = A_{1 1} + A_{2 2} + \dots + A_{n n} + B_{1 1} + B_{2 2} + \dots + B_{n n} &  & (A_{i i}, B_{i i} \in \F) \\
               & = 0                                                                         &  & (A, B \in \W)
  \end{align*}
  and
  \begin{align*}
    \tr{cA} & = \p{cA}_{1 1} + \p{cA}_{2 2} + \dots + \p{cA}_{n n} &  & \text{(by \cref{1.3.9})} \\
            & = cA_{1 1} + cA_{2 2} + \dots + cA_{n n}             &  & \text{(by \cref{1.2.9})} \\
            & = c\p{A_{1 1} + A_{2 2} + A_{n n}}                   &  & (c, A_{i i} \in \F)      \\
            & = c0                                                 &  & (A \in \W)               \\
            & = 0                                                  &  & (c \in \F)
  \end{align*}
  for any scalar \(c \in \F\).
  Therefore \(\W\) is a subspace of \(\ms{n}{n}{\F}\) over \(\F\) by \cref{1.3}.
\end{proof}

\begin{thm}\label{1.4}
  Any intersection of subspaces of a vector space \(\V\) is a subspace of \(\V\).
\end{thm}

\begin{proof}
  Let \(\cvs\) be a collection of subspaces of \(\V\) over \(\F\), and let \(\W\) denote the intersection of the subspaces in \(\cvs\).
  Since every subspace contains the zero vector, \(\zv \in \W\).
  Let \(a \in \F\) and \(x, y \in \W\).
  Then \(x\) and \(y\) are contained in each subspace in \(\cvs\).
  Because each subspace in \(\cvs\) is closed under addition and scalar multiplication, it follows that \(x + y\) and \(ax\) are contained in each subspace in \(\cvs\).
  Hence \(x + y\) and \(ax\) are also contained in \(\W\), so that \(\W\) is a subspace of \(\V\) over \(\F\) by \cref{1.3}.
\end{proof}

\exercisesection

\setcounter{ex}{2}
\begin{ex}\label{ex:1.3.3}
  Prove that \(\tp{\p{aA + bB}} = a\tp{A} + b\tp{B}\) for any \(A, B \in \ms{m}{n}{\F}\) and any \(a, b \in \F\).
\end{ex}

\begin{proof}
  Let \(i = 1, \dots, n\) and \(j = 1, \dots, m\).
  Since
  \begin{align*}
    \p{a\tp{A} + b\tp{B}}_{i j} & = a\p{\tp{A}}_{i j} + b\p{\tp{B}}_{i j} &  & \text{(by \cref{1.2.9})} \\
                                & = aA_{j i} + bB_{j i}                   &  & \text{(by \cref{1.3.3})} \\
                                & = \p{aA}_{j i} + \p{bB}_{j i}           &  & \text{(by \cref{1.2.9})} \\
                                & = \p{aA + bB}_{j i}                     &  & \text{(by \cref{1.2.9})} \\
                                & = \tp{\p{aA + bB}}_{i j},               &  & \text{(by \cref{1.3.3})}
  \end{align*}
  by \cref{1.2.7} we know that \(\tp{\p{aA + bB}} = a\tp{A} + b\tp{B}\).
\end{proof}

\begin{ex}\label{ex:1.3.4}
  Prove that \(\tp{\p{\tp{A}}} = A\) for each \(A \in \MS\).
\end{ex}

\begin{proof}
  Let \(i = 1, \dots, m\) and \(j = 1, \dots, n\).
  Since
  \begin{align*}
    A_{i j} & = \p{\tp{A}}_{j i}           &  & \text{(by \cref{1.3.3})} \\
            & = \p{\tp{\p{\tp{A}}}}_{i j}, &  & \text{(by \cref{1.3.3})}
  \end{align*}
  by \cref{1.2.8} we know that \(\tp{\p{\tp{A}}} = A\).
\end{proof}

\begin{ex}\label{ex:1.3.5}
  Prove that \(A + \tp{A}\) is symmetric for any square matrix \(A \in \ms{n}{n}{\F}\).
\end{ex}

\begin{proof}
  Let \(i, j = 1, \dots, n\).
  Since
  \begin{align*}
    \p{\tp{\p{A + \tp{A}}}}_{i j} & = \p{A + \tp{A}}_{j i}       &  & \text{(by \cref{1.3.3})}  \\
                                  & = A_{j i} + \p{\tp{A}}_{j i} &  & \text{(by \cref{1.2.9})}  \\
                                  & = A_{j i} + A_{i j}          &  & \text{(by \cref{1.3.3})}  \\
                                  & = A_{i j} + A_{j i}          &  & (A_{i j}, A_{j i} \in \F) \\
                                  & = A_{i j} + \p{\tp{A}}_{i j} &  & \text{(by \cref{1.3.3})}  \\
                                  & = \p{A + \tp{A}}_{i j},      &  & \text{(by \cref{1.2.9})}
  \end{align*}
  by \cref{1.2.8} we know that \(\tp{\p{A + \tp{A}}} = A + \tp{A}\).
  Thus by \cref{1.3.4} \(A + \tp{A}\) is symmetric.
\end{proof}

\begin{ex}\label{ex:1.3.6}
  Prove that \(\tr{aA + bB} = a\tr{A} + b\tr{B}\) for any \(A, B \in \ms{n}{n}{\F}\).
\end{ex}

\begin{proof}
  We have
  \begin{align*}
     & \tr{aA + bB}                                                                                                         \\
     & = (aA + bB)_{1 1} + (aA + bB)_{2 2} + \dots + (aA + bB)_{n n}                       &  & \text{(by \cref{1.3.9})}    \\
     & = aA_{1 1} + bB_{1 1} + aA_{2 2} + bB_{2 2} + \dots + aA_{n n} + bB_{n n}           &  & \text{(by \cref{1.2.9})}    \\
     & = a\p{A_{1 1} + A_{2 2} + \dots + A_{n n}} + b\p{B_{1 1} + B_{2 2} + \dots B_{n n}} &  & (aA_{i i}, bB_{i i} \in \F) \\
     & = a\tr{A} + b\tr{B}.                                                                &  & \text{(by \cref{1.3.9})}
  \end{align*}
\end{proof}

\begin{ex}\label{ex:1.3.7}
  Prove that diagonal matrices are symmetric matrices.
\end{ex}

\begin{proof}
  Let \(A \in \ms{n}{n}{\F}\) be a diagonal matrices and let \(i, j = 1, \dots, n\).
  Then we have
  \begin{align*}
             & \begin{dcases}
      A_{i j} = A_{j i} = 0 & \text{if } i \neq j \\
      A_{i i} = A_{i i}     & \text{otherwise}
    \end{dcases}           &  & \text{(by \cref{1.3.8})} \\
    \implies & A_{i j} = A_{j i} = \p{\tp{A}}_{i j} &  & \text{(by \cref{1.3.3})} \\
    \implies & A = \tp{A}                           &  & \text{(by \cref{1.2.8})}
  \end{align*}
  and thus by \cref{1.3.4} diagonal matrices are symmetric.
\end{proof}

%------------------------------------------------------------------------------
% Back matters.
%------------------------------------------------------------------------------

\backmatter

\end{document}

% We use chapter structure.
\documentclass[12pt,oneside]{book}

%==============================================================================
% Preamble.
%==============================================================================

% Correctly showing characters outside ASCII.
\usepackage[T1]{fontenc}
% File is written and read with utf8 encoding.
\usepackage[utf8]{inputenc}
% Set paging layout.
\usepackage[margin=1.2in]{geometry}
% Including `amsfonts'.  Must be loaded before `mathtools'.
\usepackage{amssymb}
% Including `amsmath' and fixing bugs for `amsmath'.
\usepackage{mathtools}
% Must be loaded after `amsmath' and `mathtools'.
\usepackage{amsthm}
% Automatically adjust character spacing at margins.
\usepackage{microtype}
% Provide further utilities and fix bugs for `enumerate', `itemize' and
% `description'.
\usepackage{enumitem}
% Provide better quoting environment.
\usepackage{dirtytalk}
% Parsing list inside `\newcommand'.
\usepackage{listofitems}
% Nice looking if-then-else structure with comparison functionality.
\usepackage{ifthen}
% Automatically add hyperlinks to labels/refs.  Must be loaded after all
% packages above and before `cleveref'.  Recommend to use with `natbib' when
% you need bibtex.
\usepackage{hyperref}

\hypersetup{       % This macro come with `hyperref'.
	colorlinks=true, % Color hyperlinks.
	linkcolor=blue,  % Color local hyperlinks with blue.
	urlcolor=cyan,   % Color url links with cyan.
}

% Must be loaded after `hyperref'.  We always capitalize each cross-references'
% type name.  See `cleveref' for details.
\usepackage[capitalize]{cleveref}

%------------------------------------------------------------------------------
% Define environments.
%------------------------------------------------------------------------------

% Text inside the body of theorem-like environments are set to Roman font.
% theorem-like environments share their counters, counters follow section and
% reset in every sections (except for theorems, theorems counters are reset for
% each chapter).  Theorems and exercises has their owned counter.  Notes do not
% use counter.  See `amsthm' for details.
\theoremstyle{definition}
\newtheorem{ax}{Axiom}[section]
\newtheorem{cor}[ax]{Corollary}
\newtheorem{defn}[ax]{Definition}
\newtheorem{eg}[ax]{Example}
\newtheorem{ex}{Exercise}[section]
\newtheorem{lem}[ax]{Lemma}
\newtheorem*{note}{Note}
\newtheorem{prop}[ax]{Proposition}
\newtheorem{rem}[ax]{Remark}
\newtheorem{thm}{Theorem}[chapter]

% Proof reference text.
\newcommand{\pf}[1]{Proof of \cref{#1}}

% In `enumerate' enviroments, items' label are alphabets and surrounded by
% parentheses.  See `enumitem' for details.
\renewcommand{\labelenumi}{\textnormal{(}\alph{enumi}\textnormal{)}}

% Formatting equations tag appearence.  See `mathtools' for details.
\renewcommand{\theequation}{\thechapter.\thesection.\arabic{equation}}
\numberwithin{equation}{section}

%------------------------------------------------------------------------------
% Define operators and symbols.
%------------------------------------------------------------------------------

% Define common operators with paired of delimiters.  Always use star versions
% of these operator to automatically adjust height.  See `mathtools' for
% details.

% Absolute value.
\DeclarePairedDelimiter{\absTmp}{\lvert}{\rvert}
\newcommand{\abs}[1]{\absTmp*{#1}}
% Ceiling.
\DeclarePairedDelimiter{\ceilTmp}{\lceil}{\rceil}
\newcommand{\ceil}[1]{\ceilTmp*{#1}}
% Floor.
\DeclarePairedDelimiter{\floorTmp}{\lfloor}{\rfloor}
\newcommand{\floor}[1]{\floorTmp*{#1}}
% Evaluate.
\DeclarePairedDelimiter{\evalTmp}{.}{\rvert}
\newcommand{\eval}[1]{\evalTmp*{#1}}
% Parenthese.
\DeclarePairedDelimiter{\paTmp}{\lparen}{\rparen}
\newcommand{\pa}[1]{\paTmp*{#1}}
% Bracket.
\DeclarePairedDelimiter{\brTmp}{\lbrack}{\rbrack}
\newcommand{\br}[1]{\brTmp*{#1}}
% Brace.
\DeclarePairedDelimiter{\BTmp}{\lbrace}{\rbrace}
\newcommand{\B}[1]{\BTmp*{#1}}
% Set.
\newcommand{\set}[1]{\B{#1}}

% Define common symbols.  See `amsmath' section 9.2 for details.

% Fields.
\newcommand{\field}[1]{\mathit{#1}}
% General field.
\newcommand{\F}{\field{F}}
% Complex number field.
\newcommand{\C}{\mathbb{C}}
% Natural number field.
\newcommand{\N}{\mathbb{N}}
% Rational number field.
\newcommand{\Q}{\mathbb{Q}}
% Real number field.
\newcommand{\R}{\mathbb{R}}
% Integer number field.
\newcommand{\Z}{\mathbb{Z}}

% Vector space.
\newcommand{\vs}[1]{\mathsf{#1}}
% General vector space.
\newcommand{\V}{\vs{V}}
\newcommand{\W}{\vs{W}}
% Collection of vector space.
\newcommand{\cvs}{\mathcal{C}}

% Metric space with shape #1 x #2 over field #3.
\newcommand{\ms}[3]{\vs{M}_{{#1} \times #2}({#3})}
% General metric space.
\newcommand{\MS}{\ms{m}{n}{\F}}

% Function space from domain #1 to range #2.
\newcommand{\fs}[2]{\mathcal{F}\pa{#1, #2}}
% General function space.
\newcommand{\FS}{\fs{S}{\F}}

% Continuous function space over matric space #2.
% Function with continuous #1-th derivative
\newcommand{\cfs}[2][]{%
	\ifthenelse{%
		\equal{#1}{}%
	}{%
	\vs{C}\pa{#2}%
	}{%
	\vs{C}^{#1}\pa{#2}%
	}%
}

% Polynomial spaces of degree #1 over field #2.
% #1 (optional) is the degree of polynomial.
% #2 is field.
%
% For example:
% If we use \(\ps{\F}\), we will get
%
%    \vs{P}(\F)
%
% If we use \(\ps[n]{\F}\), we will get
%
%    \vs{P}_{n}(\F)
%
\newcommand{\ps}[2][]{%
\ifthenelse{%
	\equal{#1}{}%
}%
{%
	\vs{P}\pa{#2}%
}%
{%
	\vs{P}_{#1}\pa{#2}%
}%
}

% Sequence.
% #1 is a join operator (defaults to comma).
% #2 is a comma-separated list of sequence symbols.
% #3 is a comma-separated list of sequence index.
%
% For example:
% If we use \(\seq{a}{1,2,3}\), we will get
%
%    a_{1}, a_{2}, a_{3}
%
% If we use \(\seq[+]{a,b}{1,2,3}\), we will get
%
%    a_{1} b_{1} + a_{2} b_{2} + a_{3} b_{3}
%
% If we use \(\seq{a}{1}\), we will get
%
%    a_{1}
%
% If we use \(\seq{a}{1,2,,n}\), we will get
%
%    a_{1}, a_{2}, \dots, a_{n}
%
\newcommand{\seq}[3][,]{%
	\setsepchar{,}%                              List items are separated by comma.
	\readlist\SeqSymbols{#2}%                    Define macro `\SeqSymbols' using #1.
	\readlist\SeqIndices{#3}%                    Define macro `\SeqIndices' using #2.
	\foreachitem\SeqIndex\in\SeqIndices{%        Loop over indices.
		\ifthenelse{%                              Output join operator if `\SeqIndex' is not the first index.
			\NOT\equal{\SeqIndex}{\SeqIndices[1]}%
		}%
		{#1}%
		{}%
		\ifthenelse{%                              Output dots if `\SeqIndex' is empty.
			\equal{\SeqIndex}{}%
		}%
		{%
			\ifthenelse{%                            If #1 is `,' use `\dots', otherwise use `\cdots'.
				\equal{#1}{,}%
			}%
			{\dots}%
			{\cdots}%
		}%
		{%
			\foreachitem\SeqSymbol\in\SeqSymbols{%   Loop over symbols.
				{\SeqSymbol}_{\SeqIndex}%              Output symbols with index.
			}%
		}%
	}%
}

% Tuple.
\newcommand{\tuple}[2]{\pa{\seq{#1}{#2}}}

% 0 vector.
\newcommand{\zv}{\mathit{0}}
% 0 metric.
\newcommand{\zm}{\mathit{O}}

% Transpose of matrix #1.
\newcommand{\tp}[1]{{#1}^{t}}

% Trace of matrix #1.
\newcommand{\tr}[1]{\operatorname{tr}\pa{#1}}

% Span of a set.
% We cannot use `\span' since it is a tex primitive.
\newcommand{\spn}[1]{\operatorname{span}\pa{#1}}

% Linear transformation T.
\newcommand{\T}{\mathsf{T}}
% Identity transformation I.
\newcommand{\IT}[1][]{%
\ifthenelse{%
	\equal{#1}{}%
}{%
\mathsf{I}%
}{%
\mathsf{I}_{#1}%
}%
}
% Zero transformation T_0.
\newcommand{\zT}{\T_0}

% Null space N(T).
\newcommand{\ns}[1]{\mathsf{N}\pa{#1}}

% Range R(T).
\newcommand{\rg}[1]{\mathsf{R}\pa{#1}}

% Nullity nullity(T).
\newcommand{\nt}[1]{\operatorname{nullity}\pa{#1}}

% Rank rank(T).
\newcommand{\rk}[1]{\operatorname{rank}\pa{#1}}

% Formatting exercises section.
\newcommand{\exercisesection}{
	\begin{center}
		\textbf{EXERCISES}
	\end{center}
}

%==============================================================================
% Document.
%==============================================================================

\begin{document}

%------------------------------------------------------------------------------
% Front matters.
%------------------------------------------------------------------------------

\frontmatter

% Author informations.
\title{Linear Algebra}
\author{ProFatXuanAll}
\maketitle

% Table of contents.
\tableofcontents

%------------------------------------------------------------------------------
% Main matters.
%------------------------------------------------------------------------------

\mainmatter

% All chapters are in separated files.  We include them here.
\chapter{Vector Spaces}\label{ch:1}

% All sections are in separated files.  We include them here.
\section{Introduction}\label{sec:1.1}

\begin{note}
    Experiments show that if two like quantities act together, their effect is predictable.
    In this case, the vectors used to represent these quantities can be combined to form a resultant vector that represents the combined effects of the original quantities.
    This resultant vector is called the \emph{sum} of the original vectors, and the rule for their combination is called the \emph{parallelogram law}.
\end{note}

\begin{axiom}[Parallelogram Law for Vector Addition]\label{ax:1.1.1}
    The sum of two vectors \(x\) and \(y\) that act at the same point \(P\) is the vector beginning at \(P\) that is represented by the diagonal of parallelogram having \(x\) and \(y\) as adjacent sides.
\end{axiom}

\begin{note}
    Since a vector beginning at the origin is completely determined by its endpoint, we sometimes refer to \emph{the point \(x\)} rather than \emph{the endpoint of the vector \(x\)} if \(x\) is a vector emanating from the origin.
\end{note}

\begin{note}
    Besides the operation of vector addition, there is another natural operation that can be performed on vectors
    --- the length of a vector may be magnified or contracted.
    This operation, called \emph{scalar multiplication}, consists of multiplying the vector by a real number.
    If the vector \(x\) is represented by an arrow, then for any real number \(t\), the vector \(tx\) is represented by an arrow in the same direction if \(t \geq 0\) and in the opposite direction if \(t < 0\).
    The length of the arrow \(tx\) is \(\abs*{t}\) times the length of the arrow \(x\).
    Two nonzero vectors \(x\) and \(y\) are called \textbf{parallel} if \(y = tx\) for some nonzero real number \(t\).
    (Thus nonzero vectors having the same or opposite directions are parallel.)
\end{note}
\section{Vector Spaces}\label{sec:1.2}

\begin{defn}\label{1.2.1}
    A \textbf{vector space} (or \textbf{linear space}) \(\V\) over a field \(\F\) consists of a set on which two operations (called \textbf{addition} and \textbf{scalar multiplication}, respectively) are defined so that for each pair of elements \(x\), \(y\) in \(\V\) there is a unique element \(x + y\) in \(\V\), and for each element \(a\) in \(\F\) and each element \(x\) in \(\V\) there is a unique element \(ax\) in \(\V\), such that the following conditions hold.
    \begin{enumerate}[label=(VS \arabic*), ref=VS \arabic*]
        \item\label{vs1}
        For all \(x, y\) in \(\V\), \(x + y = y + x\)
        (commutativity of addition).
        \item\label{vs2}
        For all \(x, y, z\) in \(\V\), \(\p{x + y} + z = x + \p{y + z}\)
        (associativity of addition).
        \item\label{vs3}
        There exists an element in \(\V\) denoted by \(0\) such that \(x + 0 = x\) for each \(x\) in \(\V\).
        \item\label{vs4}
        For each element \(x\) in \(\V\) there exists an element \(y\) in \(\V\) such that \(x + y = 0\).
        \item\label{vs5}
        For each element \(x\) in \(\V\), \(1x = x\).
        \item\label{vs6}
        For each pair of elements \(a, b\) in \(\F\) and each element \(x\) in \(\V\), \(\p{ab} x = a \p{bx}\).
        \item\label{vs7}
        For each element \(a\) in \(\F\) and each pair of elements \(x, y\) in \(\V\), \(a \p{x + y} = ax + ay\).
        \item\label{vs8}
        For each pair of elements \(a, b\) in \(\F\) and each element \(x\) in \(\V\), \(\p{a + b} x = ax + bx\).
    \end{enumerate}
    The elements \(x + y\) and \(ax\) are called the \textbf{sum} of \(x\) and \(y\) and the \textbf{product} of \(a\) and \(x\), respectively.
\end{defn}

\begin{defn}\label{1.2.2}
    The elements of the field \(\F\) are called \textbf{scalars} and the elements of the vector space \(\V\) are called \textbf{vectors}.
\end{defn}

\begin{note}
    A vector space is frequently discussed in the text without explicitly mentioning its field of scalars.
    The reader is cautioned to remember, however, that \emph{every vector space is regarded as a vector space over a given field, which is denoted by \(\F\)}.
    Occasionally we restrict our attention to the fields of real and complex numbers, which are denoted \(\R\) and \(\C\), respectively.
\end{note}

\begin{note}
    \ref{vs2} permits us to unambiguously define the addition of any finite number of vectors
    (without the use of parentheses).
\end{note}

\begin{defn}\label{1.2.3}
    An object of the form \(\tp{a}{n}\), where the entries \(a_{1}, a_{2}, \dots, a_{n}\) are elements of a field \(\F\), is called an \textbf{\(n\)-tuple} with entries from \(\F\).
    The elements \(a_{1}, a_{2}, \dots, a_{n}\) are called the \textbf{entries} or \textbf{components} of the \(n\)-tuple.
    Two \(n\)-tuples \(\tp{a}{n}\) and \(\tp{b}{n}\) with entries from a field \(\F\) are called \textbf{equal} if \(a_i = b_i\) for \(i = 1, 2, \dots, n\).
\end{defn}

\begin{eg}\label{1.2.4}
    The set of all \(n\)-tuples with entries from a field \(\F\) is denoted by \(\vs{F}^{n}\).
    This set is a vector space over \(\F\) with the operations of coordinatewise addition and scalar multiplication;
    that is, if \(u = \tp{a}{n} \in \vs{F}^{n}\), \(v = \tp{b}{n} \in \vs{F}^{n}\), and \(c \in \F\), then
    \[
        u + v = (a_{1} + b_{1}, a_{2} + b_{2}, \dots, a_{n} + b_{n}) \quad \text{ and } \quad cu = \tp{ca}{n}.
    \]
\end{eg}

\begin{proof}
    Clearly we have
    \[
        \forall u, v \in \vs{F}^{n}, u + v \in \vs{F}^{n}
    \]
    and
    \[
        \begin{dcases}
            \forall u \in \vs{F}^{n} \\
            \forall c \in \F
        \end{dcases}, cu \in \vs{F}^{n}.
    \]

    Let \(I_{n} = \set{i \in \N : 1 \leq i \leq n}\).
    First we show that addition and scalar multiplication in \(\vs{F}^{n}\) over \(\F\) are unique.
    Suppose that \(u, u', v, v' \in \vs{F}^{n}\) such that \(u = u'\) and \(v = v'\).
    Then we have
    \begin{align*}
                 & \p{u = u'} \land \p{v = v'}                                                                                 \\
        \implies & \forall i \in I_{n}, \begin{dcases}
            u_{i} = u_{i}' \\
            v_{i} = v_{i}'
        \end{dcases}       &  & \text{(this is the definition of \(\vs{F}^{n}\))} \\
        \implies & \forall i \in I_{n}, u_{i} + v_{i} = u_{i}' + v_{i}' &  & \text{(\(\F\) is a field)}                        \\
        \implies & u + v = u' + v'                                      &  & \text{(this is the definition of \(\vs{F}^{n}\))}
    \end{align*}
    and thus the addition in \(\vs{F}^{n}\) over \(\F\) is unique.
    Now suppose that \(u, u' \in \vs{F}^{n}\) and \(c, c' \in \F\) such that \(u = u'\) and \(c = c'\).
    Then we have
    \begin{align*}
                 & \p{u = u'} \land \p{c = c'}                                                                     \\
        \implies & \begin{dcases}
            \forall i \in I_{n}, u_{i} = u_{i}' \\
            c = c'
        \end{dcases}                &  & \text{(this is the definition of \(\vs{F}^{n}\))} \\
        \implies & \forall i \in I_{n}, c u_{i} = c' u_{i}' &  & \text{(\(\F\) is a field)}                        \\
        \implies & cu = c' u'                               &  & \text{(this is the definition of \(\vs{F}^{n}\))}
    \end{align*}
    and thus the scalar multiplication in \(\vs{F}^{n}\) over \(\F\) is unique.

    Now we show that \ref{vs1}--\ref{vs8} holds for \cref{1.2.4}.
    \begin{description}
        \item[For \ref{vs1}:]
            For all \(x, y \in \vs{F}^{n}\), we have
            \begin{align*}
                x + y & = \p{x_{1} + y_{1}, x_{2} + y_{2}, \dots, x_{n} + y_{n}} &  & \text{(by \cref{1.2.4})}   \\
                      & = \p{y_{1} + x_{1}, y_{2} + x_{2}, \dots, y_{n} + x_{n}} &  & \text{(\(\F\) is a field)} \\
                      & = y + x.                                                 &  & \text{(by \cref{1.2.4})}
            \end{align*}
        \item[For \ref{vs2}:]
            For all \(x, y, z \in \vs{F}^{n}\), we have
            \begin{align*}
                 & \p{x + y} + z                                                                                                                \\
                 & = \p{x_{1} + y_{1}, x_{2} + y_{2}, \dots, x_{n} + y_{n}} + z                                 &  & \text{(by \cref{1.2.4})}   \\
                 & = \p{\p{x_{1} + y_{1}} + z_{1}, \p{x_{2} + y_{2}} + z_{2}, \dots, \p{x_{n} + y_{n}} + z_{n}} &  & \text{(by \cref{1.2.4})}   \\
                 & = \p{x_{1} + \p{y_{1} + z_{1}}, x_{2} + \p{y_{2} + z_{2}}, \dots, x_{n} + \p{y_{n} + z_{n}}} &  & \text{(\(\F\) is a field)} \\
                 & = x + \p{y_{1} + z_{1}, y_{2} + z_{2}, \dots, y_{n} + z_{n}}                                 &  & \text{(by \cref{1.2.4})}   \\
                 & = x + \p{y + z}.                                                                             &  & \text{(by \cref{1.2.4})}
            \end{align*}
        \item[For \ref{vs3}:]
            Since \(\F\) is a field, we know that \(0 \in \F\) and thus \(\p{0, \dots, 0} \in \vs{F}^{n}\).
            Then for all \(x \in \vs{F}^{n}\), we have
            \begin{align*}
                x + \p{0, \dots, 0} & = \p{x_{1} + 0, x_{2} + 0, \dots, x_{n} + 0} &  & \text{(by \cref{1.2.4})}   \\
                                    & = \tp{x}{n}                                  &  & \text{(\(\F\) is a field)} \\
                                    & = x.
            \end{align*}
            We denote \(\zv = \p{0, \dots, 0}\).
        \item[For \ref{vs4}:]
            For all \(x \in \vs{F}^{n}\), we have
            \begin{align*}
                         & \forall i \in I_{n}, x_{i} \in \F                                                                                 \\
                \implies & \forall i \in I_{n}, \exists y_{i} \in \F : x_{i} + y_{i} = 0                  &  & \text{(\(\F\) is a field)}    \\
                \implies & \exists y_{1}, y_{2}, \dots, y_{n} \in \F :                                                                       \\
                         & \p{x_{1} + y_{1}, x_{2} + y_{2}, \dots, x_{n} + y_{n}} = \p{0, \dots, 0} = \zv                                    \\
                \implies & \exists y \in \vs{F}^{n} : x + y = \zv.                                        &  & \text{(from the proof above)}
            \end{align*}
        \item[For \ref{vs5}:]
            Since \(\F\) is a field, we know that \(1 \in \F\).
            Then for all \(x \in \vs{F}^{n}\), we have
            \begin{align*}
                1x & = \p{1 x_{1}, 1 x_{2}, \dots, 1 x_{n}} &  & \text{(by \cref{1.2.4})}   \\
                   & = \tp{x}{n}                            &  & \text{(\(\F\) is a field)} \\
                   & = x.
            \end{align*}
        \item[For \ref{vs6}:]
            For all \(a, b \in \F\) and \(x \in \vs{F}^{n}\), we have
            \begin{align*}
                \p{ab} x & = \p{\p{ab} x_{1}, \p{ab} x_{2}, \dots, \p{ab} x_{n}}    &  & \text{(by \cref{1.2.4})}   \\
                         & = \p{a \p{b x_{1}}, a \p{b x_{2}}, \dots, a \p{b x_{n}}} &  & \text{(\(\F\) is a field)} \\
                         & = a \p{b x_{1}, b x_{2}, \dots, b x_{n}}                 &  & \text{(by \cref{1.2.4})}   \\
                         & = a \p{bx}.                                              &  & \text{(by \cref{1.2.4})}
            \end{align*}
        \item[For \ref{vs7}:]
            For all \(a \in \F\) and \(x, y \in \vs{F}^{n}\), we have
            \begin{align*}
                a \p{x + y} & = a \p{x_{1} + y_{1}, x_{2} + y_{2}, \dots, x_{n} + y_{n}}                    &  & \text{(by \cref{1.2.4})}   \\
                            & = \p{a \p{x_{1} + y_{1}}, a \p{x_{2} + y_{2}}, \dots, a \p{x_{n} + y_{n}}}    &  & \text{(by \cref{1.2.4})}   \\
                            & = \p{a x_{1} + a y_{1}, a x_{2} + a y_{2}, \dots, a x_{n} + a y_{n}}          &  & \text{(\(\F\) is a field)} \\
                            & = \p{a x_{1}, a x_{2}, \dots, a x_{n}} + \p{a y_{1}, a y_{2}, \dots, a y_{n}} &  & \text{(by \cref{1.2.4})}   \\
                            & = a x + a y.                                                                  &  & \text{(by \cref{1.2.4})}
            \end{align*}
        \item[For \ref{vs8}:]
            For all \(a, b \in \F\) and \(x \in \vs{F}^{n}\), we have
            \begin{align*}
                \p{a + b} x & = \p{\p{a + b} x_{1}, \p{a + b} x_{2}, \dots, \p{a + b} x_{n}}                &  & \text{(by \cref{1.2.4})}   \\
                            & = \p{a x_{1} + b x_{1}, a x_{2} + b x_{2}, \dots, a x_{n} + b x_{n}}          &  & \text{(\(\F\) is a field)} \\
                            & = \p{a x_{1}, a x_{2}, \dots, a x_{n}} + \p{b x_{1}, b x_{2}, \dots, b x_{n}} &  & \text{(by \cref{1.2.4})}   \\
                            & = a x + b x.                                                                  &  & \text{(by \cref{1.2.4})}
            \end{align*}
    \end{description}
    From all proofs above we conclude by \cref{1.2.1} that \cref{1.2.4} is indeed a vector space.
\end{proof}

\section{Subspaces}\label{sec:1.3}

\begin{defn}\label{1.3.1}
  A subset \(\W\) of a vector space \(\V\) over a field \(\F\) is called a \textbf{subspace} of \(\V\) if \(\W\) is a vector space over \(\F\) with the operations of addition and scalar multiplication defined on \(\V\).
\end{defn}

\begin{eg}\label{1.3.2}
  In any vector space \(\V\), note that \(\V\) and \(\set{\zv}\) are subspaces.
  The latter is called the \textbf{zero subspace} of \(\V\).
\end{eg}

\begin{proof}
  Since \(\V \subseteq \V\) and \(\V\) is a vector space over \(\F\) with the operations of addition and scalar multiplication defined on \(\V\), by \cref{1.3.1} we know that \(\V\) is a subspace of \(\V\).
  Since \(\zv \in \V\) (by \ref{vs3}), we know that \(\set{\zv} \subseteq \V\).
  Thus by \cref{ex:1.2.11} and \cref{1.3.1} \(\set{\zv}\) is a subspace of \(\V\).
\end{proof}
\section{Linear Combinations and Systems of Linear Equations}\label{sec:1.4}

\begin{defn}\label{1.4.1}
  Let \(\V\) be a vector space over \(\F\) and \(S\) a nonempty subset of \(\V\).
  A vector \(v \in \V\) is called a \textbf{linear combination} of vectors of \(S\) if there exist a finite number of vectors \(\seq{u}{1,2,,n}\) in \(S\) and scalars \(\seq{a}{1,2,,n}\) in \(\F\) such that \(v = \seq[+]{a,u}{1,2,,n}\).
  In this case we also say that \(v\) is a linear combination of \(\seq{u}{1,2,,n}\) and call \(\seq{a}{1,2,,n}\) the \textbf{coefficients} of the linear combination.
\end{defn}

\begin{eg}\label{1.4.2}
  Observe that in any vector space \(\V\), \(0v = \zv\) for each \(v \in \V\).
  Thus the zero vector is a linear combination of any nonempty subset of \(\V\).
\end{eg}

\begin{defn}\label{1.4.3}
  Let \(S\) be a nonempty subset of a vector space \(\V\) over \(\F\).
  The \textbf{span} of \(S\), denoted \(\spn{S}\), is the set consisting of all linear combinations of the vectors in \(S\).
  For convenience, we define \(\spn{\varnothing} = \set{\zv}\).
\end{defn}

\begin{thm}\label{1.5}
  The span of any subset \(S\) of a vector space \(\V\) over \(\F\) is a subspace of \(\V\) over \(\F\).
  Moreover, any subspace of \(\V\) over \(\F\) that contains \(S\) must also contain the span of \(S\).
\end{thm}

\begin{proof}
  This result is immediate if \(S = \varnothing\) because \(\spn{\varnothing} = \set{\zv}\), which is a subspace that is contained in any subspace of \(\V\) over \(\F\)
  (see \cref{1.3.2}).

  If \(S \neq \varnothing\), then \(S\) contains a vector \(z\).
  So \(0z = \zv\) is in \(\spn{S}\).
  Let \(x, y \in \spn{S}\).
  Then there exist vectors \(\seq{u}{1,2,,m}\), \(\seq{v}{1,2,,n}\) in \(S\) and scalars \(\seq{a}{1,2,,m}\), \(\seq{b}{1,2,,n}\) in \(\F\) such that
  \[
    x = \seq[+]{a,u}{1,2,,m} \quad \text{and} \quad y = \seq[+]{b,v}{1,2,,n}
  \]
  Then
  \[
    x + y = \seq[+]{a,u}{1,2,,m} + \seq[+]{b,v}{1,2,,n}
  \]
  and, for any scalar \(c \in \F\),
  \[
    cx = (ca_1) u_1 + (ca_2) u_2 + \cdots + (ca_m) u_m
  \]
  are clearly linear combinations of the vectors in \(S\);
  so \(x + y\) and \(cx\) are in \(\spn{S}\).
  Thus \(\spn{S}\) is a subspace of \(\V\) over \(\F\) (by \cref{1.3}).

  Now let \(\W\) denote any subspace of \(\V\) over \(\F\) that contains \(S\).
  If \(w \in \spn{S}\), then \(w\) has the form \(w = \seq[+]{c,w}{1,2,,k}\) for some vectors \(\seq{w}{1,2,,k}\) in \(S\) and some scalars \(\seq{c}{1,2,,k}\).
  Since \(S \subseteq \W\), we have \(\seq{w}{1,2,,k} \in \W\).
  Therefore \(w = \seq[+]{c,w}{1,2,,k}\) is in \(\W\) by \cref{ex:1.3.20} of \cref{sec:1.3}.
  Because \(w\), an arbitrary vector in \(\spn{S}\), belongs to \(\W\), it follows that \(\spn{S} \subseteq \W\).
\end{proof}

\begin{defn}\label{1.4.4}
  A subset \(S\) of a vector space \(\V\) over \(\F\) \textbf{generates} (or \textbf{spans}) \(\V\) if \(\spn{S} = \V\).
  In this case, we also say that the vectors of \(S\) generate (or span) \(\V\).
\end{defn}

\begin{note}
  Usually there are many different subsets that generate a subspace \(\W\).
  It is natural to seek a subset of \(\W\) that generates \(\W\) and is as small as possible.
\end{note}

\section{Linear Dependence and Linear Independence}\label{sec:1.5}

\begin{note}
  Suppose that \(\V\) is a vector space over an infinite field and that \(\W\) is a subspace of \(\V\).
  Unless \(\W\) is the zero subspace, \(\W\) is an infinite set.
  It is desirable to find a ``small'' finite subset \(S\) that generates \(\W\) because we can then describe each vector in \(\W\) as a linear combination of the finite number of vectors in \(S\).
  Indeed, the smaller that \(S\) is, the fewer computations that are required to represent vectors in \(\W\).

  Checking that some vector in \(S\) is a linear combination of the other vectors in \(S\) could require that we solve several different systems of linear equations before we determine which, if any, vectors in \(S\) is a linear combination of the others.

  Because some vector in \(S\) is a linear combination of the others, the zero vector can be expressed as a linear combination of the vectors in \(S\) using coefficients that are not all zero.
  The converse of this statement is also true:
  If the zero vector can be written as a linear combination of the vectors in \(S\) in which not all the coefficients are zero, then some vector in \(S\) is a linear combination of the others.
  Thus, rather than asking whether some vector in \(S\) is a linear combination of the other vectors in \(S\), it is more efficient to ask whether the zero vector can be expressed as a linear combination of the vectors in \(S\) with coefficients that are not all zero.
\end{note}

\begin{defn}\label{1.5.1}
  A subset \(S\) of a vector space \(\V\) over \(\F\) is called \textbf{linearly dependent} if there exist a finite number of distinct vectors \(\seq{u}{1,,n}\) in \(S\) and scalars \(\seq{a}{1,,n}\) in \(\F\), not all zero, such that
  \[
    \seq[+]{a,u}{1,,n} = \zv.
  \]
  In this case we also say that the vectors of \(S\) are linearly dependent.
\end{defn}

\begin{defn}\label{1.5.2}
  For any vectors \(\seq{u}{1,,n}\), we have \(\seq[+]{a,u}{1,,n} = \zv\) if \(\seq[=]{a}{1,,n} = 0\).
  We call this the \textbf{trivial representation} of \(\zv\) as a linear combination of \(\seq{u}{1,,n}\).
  Thus, for a set to be linearly dependent, there must exist a nontrivial representation of \(\zv\) as a linear combination of vectors in the set.
  Consequently, any subset of a vector space that contains the zero vector is linearly dependent, because \(\zv = 1 \cdot \zv\) is a nontrivial representation of \(\zv\) as a linear combination of vectors in the set.
\end{defn}

\begin{defn}\label{1.5.3}
  A subset \(S\) of a vector space over \(\F\) that is not linearly dependent is called \textbf{linearly independent}.
  As before, we also say that the vectors of \(S\) are linearly independent.
\end{defn}

\begin{eg}\label{1.5.4}
  The following facts about linearly independent sets are true in any vector space.
  \begin{enumerate}
    \item The empty set is linearly independent, for linearly dependent sets must be nonempty.
    \item A set consisting of a single nonzero vector is linearly independent.
          For if \(\set{u}\) is linearly dependent, then \(au = \zv\) for some nonzero scalar \(a\).
          Thus
          \[
            u = a^{-1} (au) = a^{-1} \zv = \zv.
          \]
    \item A set is linearly independent iff the only representations of \(\zv\) as linear combinations of its vectors are trivial representations.
  \end{enumerate}
\end{eg}

\begin{eg}\label{1.5.5}
  For \(k = 0, 1, \dots, n\) let \(p_k(x) = x^k + x^{k + 1} + \cdots + x^n\).
  The set
  \[
    \set{p_0(x), p_1(x), \dots, p_n(x)}
  \]
  is linearly independent in \(\ps[n]{\F}\).
  For if
  \[
    a_0 p_0(x) + a_1 p_1(x) + \cdots + a_n p_n(x) = \zv
  \]
  for some scalars \(\seq{a}{0,1,,n}\), then
  \[
    a_0 + (a_0 + a_1) x + (a_0 + a_1 + a_2) x^2 + \cdots + (a_0 + a_1 + \cdots + a_n) x^n = \zv.
  \]
  By equating the coefficients of \(x^k\) on both sides of this equation for \(k = 0, 1, \dots, n\), we obtain
  \[
    \begin{matrix*}[l]
      & \seq[+]{a}{0}      & = 0 \\
      & \seq[+]{a}{0,1}    & = 0 \\
      & \seq[+]{a}{0,1,2}  & = 0 \\
      & \vdots & \\
      & \seq[+]{a}{0,1,2,,n} & = 0
    \end{matrix*}
  \]
  Clearly the only solution to this system of linear equations is \(\seq[=]{a}{0,1,,n} = 0\).
\end{eg}

\begin{thm}\label{1.6}
  Let \(\V\) be a vector space over \(\F\), and let \(S_1 \subseteq S_2 \subseteq \V\).
  If \(S_1\) is linearly dependent, then \(S_2\) is linearly dependent.
\end{thm}

\begin{proof}[\pf{1.6}]
  We have
  \begin{align*}
             & \begin{dcases}
                 S_1 \subseteq S_2 \\
                 S_1 \text{ is linearly dependent}
               \end{dcases}                               \\
    \implies & \begin{dcases}
                 \exists \seq{u}{1,,n} \in S_1 \subseteq S_2 \\
                 \exists \seq{a}{1,,n} \in \F
               \end{dcases} :                    \\
             & \begin{dcases}
                 \seq[+]{a,u}{1,,n} = \zv \\
                 \lnot (\seq[=]{a}{1,,n} = 0)
               \end{dcases}                   &  & \by{1.5.1}                 \\
    \implies & S_2 \text{ is linearly dependent}.             &  & \by{1.5.1}
  \end{align*}
\end{proof}

\begin{cor}\label{1.5.6}
  Let \(\V\) be a vector space, and let \(S_1 \subseteq S_2 \subseteq \V\).
  If \(S_2\) is linearly independent, then \(S_1\) is linearly independent.
\end{cor}

\begin{proof}[\pf{1.5.6}]
  Suppose for sake of contradiction that \(S_1\) is linearly dependent.
  But by \cref{1.6} \(S_1 \subseteq S_2\) implies \(S_2\) is linearly dependent, which contradicts to the fact that \(S_2\) is linearly independent.
  Thus \(S_1\) is linearly independent.
\end{proof}

\begin{note}
  Earlier in this section, we noted that the issue of whether \(S\) is the smallest generating set for its span is related to the question of whether some vector in \(S\) is a linear combination of the other vectors in \(S\).
  Thus the issue of whether \(S\) is the smallest generating set for its span is related to the question of whether \(S\) is linearly dependent.

  More generally, suppose that \(S\) is any linearly dependent set containing two or more vectors.
  Then some vector \(v \in S\) can be written as a linear combination of the other vectors in \(S\), and the subset obtained by removing \(v\) from \(S\) has the same span as \(S\).
  It follows that \emph{if no proper subset of \(S\) generates the span of \(S\), then \(S\) must be linearly independent.}
\end{note}

\begin{thm}\label{1.7}
  Let \(S\) be a linearly independent subset of a vector space \(\V\) over \(\F\), and let \(v\) be a vector in \(\V\) that is not in \(S\).
  Then \(S \cup \set{v}\) is linearly dependent iff \(v \in \spn{S}\).
\end{thm}

\begin{proof}[\pf{1.7}]
  By \cref{1.5.2} we see that when \(v = \zv\) the statement holds.
  So suppose that \(v \neq \zv\).

  If \(S \cup \set{v}\) is linearly dependent, then there are vectors \\
  \(\seq{u}{1,,n}\) in \(S\) such that \(a_0 v + \seq[+]{a,u}{1,,n} = \zv\) for some nonzero scalars \(\seq{a}{0,1,2,,n}\) in \(\F\).
  We claim that \(a_0 \neq 0\).
  So suppose for sake of contradiction that \(a_0 = 0\).
  Because \(S\) is linearly independent, we know that
  \begin{align*}
             & a_0 v + \seq[+]{a,u}{1,,n}                            \\
             & = 0v + \seq[+]{a,u}{1,,n}                             \\
             & = \zv + \seq[+]{a,u}{1,,n} &  & \by{1.2}[a]           \\
             & = \seq[+]{a,u}{1,,n}       &  & \text{(by \ref{vs3})} \\
             & = \zv                                                 \\
    \implies & \seq[=]{a}{1,,n} = 0.      &  & \by{1.5.3}
  \end{align*}
  But this means \(\seq[=]{a}{0,1,2,,n} = 0\), a contradiction.
  Thus we must have \(a_0 \neq 0\), and so
  \[
    v = a_0^{-1} (-\seq[-]{a,u}{1,,n}) = -(a_0^{-1} a_1) u_1 - (a_0^{-1} a_2) u_2 - \cdots - (a_0^{-1} a_n) u_n.
  \]
  Since \(v\) is a linear combination of \(\seq{u}{1,,n}\), which are in \(S\), we have \(v \in \spn{S}\).

  Conversely, let \(v \in \spn{S}\).
  Then there exist vectors \(\seq{v}{1,,m}\) in \(S\) and scalars \(\seq{b}{1,,m}\) such that \(v = \seq[+]{b,v}{1,,m}\).
  Hence
  \[
    \zv = \seq[+]{b,v}{1,,m} + (-1)v.
  \]
  Since \(v \neq v_i\) for \(i = 1, 2, \dots, m\), the coefficient of \(v\) in this linear combination is nonzero, and so the set \(\set{\seq{v}{1,,m}, v}\) is linearly dependent.
  Therefore \(S \cup \set{v}\) is linearly dependent by \cref{1.6}.
\end{proof}

\exercisesection

\setcounter{ex}{3}
\begin{ex}\label{ex:1.5.4}
  In \(\vs{F}^n\), let \(e_j\) denote the vector whose \(j\)th coordinate is \(1\) and whose other coordinates are \(0\).
  Prove that \(\set{\seq{e}{1,,n}}\) is linearly independent.
\end{ex}

\begin{proof}[\pf{ex:1.5.4}]
  Let \(\seq{a}{1,,n} \in \F\).
  Since
  \begin{align*}
             & \seq[+]{a,e}{1,,n} = \zv                          \\
    \implies & \begin{dcases}
                 a_1 1 + a_2 0 + \cdots + a_n 0 = 0 \\
                 a_1 0 + a_2 1 + \cdots + a_n 0 = 0 \\
                 \vdots                             \\
                 a_1 0 + a_2 0 + \cdots + a_n 1 = 0
               \end{dcases}             &  & \by{1.2.4}          \\
    \implies & \forall i \in \set{1, \dots, n}, a_i 1 = a_i = 0,
  \end{align*}
  by \cref{1.5.3} we know that \(\set{\seq{e}{1,,n}}\) is linearly independent.
\end{proof}

\begin{ex}\label{ex:1.5.5}
  Show that the set \(\set{1, x, x^2, \dots, x^n}\) is linearly independent in \(\ps[n]{\F}\).
\end{ex}

\begin{proof}[\pf{ex:1.5.5}]
  Let \(\seq{a}{0,1,2,,n} \in \F\).
  Since
  \begin{align*}
             & a_0 + a_1 x^1 + a_2 x^2 + \cdots + a_n x^n = \zv = 0 + 0x + 0x^2 + \cdots + 0x^n \\
    \implies & \seq[=]{a}{0,1,2,,n} = 0,
  \end{align*}
  by \cref{1.5.3} we know that \(\set{1, x, x^2, \dots, x^n}\) is linearly independent.
\end{proof}

\begin{ex}\label{ex:1.5.6}
  In \(\ms\), let \(E^{i j}\) denote the matrix whose only nonzero entry is \(1\) in the \(i\)th row and \(j\)th column.
  Prove that \(\set{E^{i j} : 1 \leq i \leq m, 1 \leq j \leq n}\) is linearly independent.
\end{ex}

\begin{proof}[\pf{ex:1.5.6}]
  Let \(a_{i j} \in \F\) where \(1 \leq i \leq m\) and \(1 \leq j \leq n\).
  Since
  \begin{align*}
             & \sum_{i = 1}^m \sum_{j = 1}^{n} a_{i j} E^{i j} = \zm                 \\
    \implies & \begin{pmatrix}
                 a_{1 1} 1 & a_{1 2} 1 & \cdots & a_{1 n} 1 \\
                 a_{2 1} 1 & a_{2 2} 1 & \cdots & a_{2 n} 1 \\
                 \vdots    & \vdots    & \ddots & \vdots    \\
                 a_{m 1} 1 & a_{m 2} 1 & \cdots & a_{m n} 1
               \end{pmatrix}                            \\
             & = \begin{pmatrix}
                   a_{1 1} & a_{1 2} & \cdots & a_{1 n} \\
                   a_{2 1} & a_{2 2} & \cdots & a_{2 n} \\
                   \vdots  & \vdots  & \ddots & \vdots  \\
                   a_{m 1} & a_{m 2} & \cdots & a_{m n}
                 \end{pmatrix} = \zm                                \\
    \implies & a_{i j} = 0,                                          &  & \by{1.2.8}
  \end{align*}
  by \cref{1.5.3} we know that \(\set{E^{i j} : 1 \leq i \leq m, 1 \leq j \leq n}\) is linearly independent.
\end{proof}

\setcounter{ex}{7}
\begin{ex}\label{ex:1.5.8}
  Let \(S = \set{(1, 1, 0), (1, 0, 1), (0, 1, 1)}\) be a subset of the vector space \(\vs{F}^3\) over \(\F\).
  \begin{enumerate}
    \item Prove that if \(\F = \R\), then \(S\) is linearly independent.
    \item Prove that if \(\F\) has characteristic \(2\), then \(S\) is linearly dependent.
  \end{enumerate}
\end{ex}

\begin{proof}[\pf{ex:1.5.8}(a)]
  Let \(\seq{a}{1,2,3} \in \R\).
  Since
  \begin{align*}
             & a_1 (1, 1, 0) + a_2 (1, 0, 1) + a_3 (0, 1, 1) = (0, 0, 0) \\
    \implies & \begin{dcases}
                 a_1 1 + a_2 1 + a_3 0 = 0 \\
                 a_1 1 + a_2 0 + a_3 1 = 0 \\
                 a_1 0 + a_2 1 + a_3 1 = 0
               \end{dcases}                              &  & \by{1.2.4} \\
    \implies & \seq[=]{a}{1,2,3} = 0,
  \end{align*}
  by \cref{1.5.3} we know that \(S\) is linearly independent.
\end{proof}

\begin{proof}[\pf{ex:1.5.8}(b)]
  Observe that
  \begin{align*}
    (1, 1, 0) + (1, 0, 1) + (0, 1, 1) & = (1 + 1, 1 + 0, 0 + 1) + (0, 1, 1) \\
                                      & = (0, 1, 1) + (0, 1, 1)             \\
                                      & = (0 + 0, 1 + 1, 1 + 1)             \\
                                      & = (0, 0, 0).
  \end{align*}
  Thus by \cref{1.5.1} \(S\) is linearly dependent.
\end{proof}

\begin{ex}\label{ex:1.5.9}
  Let \(u\) and \(v\) be distinct vectors in a vector space \(\V\) over \(\F\).
  Show that \(\set{u, v}\) is linearly dependent iff \(u\) or \(v\) is a multiple of the other.
\end{ex}

\begin{proof}[\pf{ex:1.5.9}]
  We have
  \begin{align*}
         & \set{u, v} \text{ is linearly dependent}                        \\
    \iff & \exists a, b \in \F : \begin{dcases}
                                   au + bv = \zv \\
                                   \lnot (a = b = 0)
                                 \end{dcases}         &  & \by{1.5.1}      \\
    \iff & \exists a, b \in \F : \begin{dcases}
                                   au = -bv \\
                                   \lnot (a = b = 0)
                                 \end{dcases}                          \\
    \iff & \exists a, b \in \F : \begin{dcases}
                                   u = -\dfrac{b}{a} v & \text{if } a \neq 0 \\
                                   v = -\dfrac{a}{b} u & \text{if } b \neq 0
                                 \end{dcases} \\
    \iff & \exists c \in \F : u = cv.
  \end{align*}
\end{proof}

\setcounter{ex}{10}
\begin{ex}\label{ex:1.5.11}
  Let \(S = \set{\seq{u}{1,,n}}\) be a linearly independent subset of a vector space \(\V\) over the field \(\Z_2\).
  How many vectors are there in \(\spn{S}\)?
  Justify your answer.
\end{ex}

\begin{proof}[\pf{ex:1.5.11}]
  We have
  \[
    \forall v \in \spn{S}, \exists \seq{a}{1,,n} \in \Z_2 : \seq[+]{a,u}{1,,n} = v.
  \]
  Since \(a_i \in \Z_2\) for all \(i \in \set{1, \dots, n}\), \(a_i\) has only two choices, which are \(0\) or \(1\).
  Since there are \(n\) variables with \(2\) choices for each, there are \(2^n\) vectors in \(\spn{S}\).
\end{proof}

\setcounter{ex}{12}
\begin{ex}\label{ex:1.5.13}
  Let \(\V\) be a vector space over a field of characteristic not equal to two.
  \begin{enumerate}
    \item Let \(u\) and \(v\) be distinct vectors in \(\V\).
          Prove that \(\set{u,v}\) is linearly independent iff \(\set{u + v, u - v}\) is linearly independent.
    \item Let \(u, v\), and \(w\) be distinct vectors in \(\V\).
          Prove that \(\set{u, v, w}\) is linearly independent iff \(\set{u + v, u + w, v + w}\) is linearly independent.
  \end{enumerate}
\end{ex}

\begin{proof}[\pf{ex:1.5.13}(a)]
  We have
  \begin{align*}
         & \set{u, v} \text{ is linearly independent}                          \\
    \iff & \forall a, b \in \F,                                                \\
         & [au + bv = \zv \implies a = b = 0]                  &  & \by{1.5.3} \\
    \iff & \forall a, b \in \F,                                                \\
         & [a(u + v) + b(u - v) = (a + b) u + (a - b) v                        \\
         & = \zv \implies a + b = a - b = 0]                                   \\
    \iff & \set{u + v, u - v} \text{ is linearly independent}. &  & \by{1.5.3}
  \end{align*}
\end{proof}

\begin{proof}[\pf{ex:1.5.13}(b)]
  We have
  \begin{align*}
         & \set{u, v, w} \text{ is linearly independent}                              \\
    \iff & \forall a, b, c \in \F,                                                    \\
         & [au + bv + cw = \zv \implies a = b = c = 0]                &  & \by{1.5.3} \\
    \iff & \forall a, b \in \F,                                                       \\
         & [a(u + v) + b(u + w) + c(v + w)                                            \\
         & = (a + b) u + (a + c) v + (b + c) w                                        \\
         & = \zv \implies a + b = a + c = b + c = 0]                                  \\
    \iff & \set{u + v, u + w, v + w} \text{ is linearly independent}. &  & \by{1.5.3}
  \end{align*}
\end{proof}

\begin{ex}\label{ex:1.5.14}
  Prove that a set \(S\) is linearly dependent iff \(S = \set{\zv}\) or there exist distinct vectors \(v, \seq{u}{1,,n}\) in \(S\) such that \(v\) is a linear combination of \(\seq{u}{1,,n}\).
\end{ex}

\begin{proof}[\pf{ex:1.5.14}]
  By \cref{1.5.4}(a) we know that \(\varnothing\) is linearly independent, so \(S\) has at least one element.
  First suppose that \(S\) has only one element.
  Then by \cref{1.5.4}(b) we know that \(S\) is linearly dependent iff \(S = \set{\zv}\).

  Now suppose that \(S\) has more than one element.
  Then we have
  \begin{align*}
         & S \text{ is linearly dependent}                                                      \\
    \iff & \begin{dcases}
             \exists v \in S                               \\
             \exists \seq{u}{1,,n} \in S \setminus \set{v} \\
             \exists a_0 \in \F \setminus \set{0}          \\
             \exists \seq{a}{1,,n} \in \F
           \end{dcases} :                                        \\
         & a_0 v + \seq[+]{a,u}{1,,n} = \zv                                     &  & \by{1.5.1} \\
    \iff & \begin{dcases}
             \exists v \in S                               \\
             \exists \seq{u}{1,,n} \in S \setminus \set{v} \\
             \exists a_0 \in \F \setminus \set{0}          \\
             \exists \seq{a}{1,,n} \in \F
           \end{dcases} :                                        \\
         & v = -a_0^{-1} a_1 u_1 - a_0^{-1} a_2 u_2 - \cdots - a_0^{-1} a_n u_n                 \\
    \iff & \exists v, \seq{u}{1,,n} \in S :                                                     \\
         & v \text{ is a linear combination of } \seq{u}{1,,n}.                 &  & \by{1.4.1}
  \end{align*}
  Combine all proofs above we conclude that \(S\) is linearly dependent iff \(S = \set{\zv}\) or there exist distinct vectors \(v, \seq{u}{1,,n}\) in \(S\) such that \(v\) is a linear combination of \(\seq{u}{1,,n}\).
\end{proof}

\begin{ex}\label{ex:1.5.15}
  Let \(S = \set{\seq{u}{1,,n}}\) be a finite set of vectors.
  Prove that \(S\) is linearly dependent iff \(u_1 = \zv\) or \(u_{k + 1} \in \spn{\set{\seq{u}{1,,k}}}\) for some \(k\) (\(1 \leq k < n\)).
\end{ex}

\begin{proof}[\pf{ex:1.5.15}]
  By \cref{1.5.1} and \cref{1.5.2} we know that if \(u_1 = \zv\) or \(u_{k + 1} \in \spn{\set{\seq{u}{1,,k}}}\) for some \(k\) (\(1 \leq k < n\)), then \(S\) is linearly dependent.
  So we only need to show the converse is also true.

  Let \(S = \set{\seq{u}{1,,n}}\) be linearly dependent.
  Suppose for sake of contradiction that \(u_1 \neq \zv\) and
  \[
    \forall k \in \set{1, \dots, n - 1}, u_{k + 1} \notin \spn{\set{\seq{u}{1,,k}}}.
  \]
  By \cref{1.4.2} we know that \(u_k \neq \zv\) for all \(k \in \set{1, \dots, n}\).
  But then we have
  \begin{align*}
             & u_n \notin \spn{\set{\seq{u}{1,,n - 1}}}                                \\
    \implies & \forall \seq{a}{1,,n - 1} \in \F,                                       \\
             & \seq[+]{a,u}{1,,n - 1} \neq u_n                         &  & \by{1.4.3} \\
    \implies & \begin{dcases}
                 \forall \seq{a}{1,,n - 1} \in \F \\
                 \forall a_n \in \F \setminus \set{0}
               \end{dcases},                                     \\
             & \seq[+]{a,u}{1,,n - 1} \neq a_n u_n                                     \\
    \implies & \begin{dcases}
                 \forall \seq{a}{1,,n - 1} \in \F \\
                 \forall a_n \in \F \setminus \set{0}
               \end{dcases},                                     \\
             & \seq[+]{a,u}{1,,n} \neq \zv                                             \\
    \implies & \forall \seq{a}{1,,n} \in \F, [\seq[+]{a,u}{1,,n} = \zv                 \\
             & \iff \seq[=]{a}{1,,n} = 0]                                              \\
    \implies & S \text{ is linearly independent},                      &  & \by{1.5.3}
  \end{align*}
  a contradiction.
  Thus the converse must be true.
\end{proof}

\begin{ex}\label{ex:1.5.16}
  Prove that a set \(S\) of vectors is linearly independent iff each finite subset of \(S\) is linearly independent.
\end{ex}

\begin{proof}[\pf{ex:1.5.16}]
  First suppose that \(S\) is linearly independent.
  Then by \cref{1.5.6} we know that every finite subset of \(S\) is linearly independent.

  Now suppose that every finite subset of \(S\) is linearly independent.
  Suppose for sake of contradiction that \(S\) is linearly dependent.
  Then by \cref{ex:1.5.14} we know that \(S = \set{\zv}\) or there exist \(v, \seq{u}{1,,n} \in S\) such that \(v\) is a linear combination of \(\seq{u}{1,,n}\).
  Clearly \(S \neq \set{\zv}\) since \(\set{\zv}\) is a finite subset of \(S\) but \(\set{\zv}\) is linearly dependent.
  So we must have \(v\) as a linear combination of \(\seq{u}{1,,n}\).
  Since the set \(\set{v,\seq{u}{1,,n}}\) is finite, by hypothese it is linearly independent.
  But this contradicts to the fact that \(v\) is a linear combination of \(\seq{u}{1,,n}\).
  Thus \(S\) is linearly independent.
\end{proof}

\begin{ex}\label{ex:1.5.17}
  Let \(M \in \ms[n][n][\F]\) be a square upper triangular matrix (as defined in \cref{ex:1.3.12}) with nonzero diagonal entries.
  Prove that the columns of \(M\) are linearly independent.
\end{ex}

\begin{proof}[\pf{ex:1.5.17}]
  The columns of \(M\) are vectors in \(\vs{F}^n\), thus we can write the \(i\)th column (\(1 \leq i \leq n\)) of \(M\) as
  \[
    \begin{pmatrix}
      M_{1 i} \\
      M_{2 i} \\
      \vdots  \\
      M_{n i}
    \end{pmatrix} = M_{1 i} \cdot e_1 + M_{2 i} \cdot e_2 + \cdots + M_{n i} \cdot e_n = \sum_{j = 1}^n M_{j i} \cdot e_j,
  \]
  where \(\seq{e}{1,,n}\) are defined as in \cref{ex:1.5.4}.
  We now show that the set of column vectors of \(M\) is linearly independent.
  Let \(\seq{a}{1,,n} \in \F\).
  Since
  \begin{align*}
             & \sum_{i = 1}^n \br{a_i \cdot \pa{\sum_{j = 1}^n M_{j i} \cdot e_j}} = \zv                                                        \\
    \implies & \sum_{i = 1}^n \pa{\sum_{j = 1}^n a_i \cdot M_{j i} \cdot e_j} = \zv                                                             \\
    \implies & \sum_{j = 1}^n \pa{\sum_{i = 1}^n a_i \cdot M_{j i} \cdot e_j} = \zv      &  & \text{(by \ref{vs1} and \ref{vs2})}               \\
    \implies & \sum_{j = 1}^n \pa{\sum_{i = 1}^n a_i \cdot M_{j i}} \cdot e_j = \zv                                                             \\
    \implies & \forall j \in \set{1, \dots, n}, \sum_{i = 1}^n a_i \cdot M_{j i} = 0     &  & \by{ex:1.5.4}                                     \\
    \implies & \forall j \in \set{1, \dots, n}, \sum_{i = j}^n a_i \cdot M_{j i} = 0     &  & \by{ex:1.3.12}                                    \\
    \implies & \seq[=]{a}{1,,n} = 0,                                                     &  & (\forall j \in \set{1, \dots, n}, M_{j j} \neq 0)
  \end{align*}
  by \cref{1.5.3} we know that the column vectors of \(M\) is linearly independent.
\end{proof}

\begin{ex}\label{ex:1.5.18}
  Let \(S\) be a set of nonzero polynomials in \(\ps{\F}\) such that no two have the same degree.
  Prove that \(S\) is linearly independent.
\end{ex}

\begin{proof}[\pf{ex:1.5.18}]
  Suppose for sake of contradiction that \(S\) is linearly dependent.
  Then by \cref{ex:1.5.14} we know that
  \[
    \begin{dcases}
      \exists g, \seq{f}{1,,n} \in S \\
      \exists \seq{a}{1,,n} \in \F
    \end{dcases} : \begin{dcases}
      g = \seq[+]{a,f}{1,,n} \\
      \lnot (\seq[=]{a}{1,,n} = 0)
    \end{dcases}.
  \]
  Let \(\deg(h)\) be the degree of any function \(h \in \ps{\F}\) and let \(m = \deg(g)\).
  Then by \cref{1.2.11} we have
  \[
    \exists \seq{c}{0,1,,m} \in \F : \begin{dcases}
      g(x) = \sum_{j = 0}^m c_j x^j \\
      c_m \neq 0
    \end{dcases}.
  \]
  By hypothese we know that
  \[
    \forall i \in \set{1, \dots, n}, \deg(f_i) \neq m.
  \]
  But by \cref{ex:1.5.5} we have
  \[
    c_m x^m = \sum_{i = 1}^n a_i \cdot (0x^m) = 0 \implies c_m = 0,
  \]
  a contradiction.
  Thus \(S\) is linearly independent.
\end{proof}

\begin{ex}\label{ex:1.5.19}
  Prove that if \(\set{\seq{A}{1,,k}}\) is a linearly independent subset of \(\ms[n][n][\F]\), then \(\set{\tp{A}_1, \tp{A}_2, \dots, \tp{A}_k}\) is also linearly independent.
\end{ex}

\begin{proof}[\pf{ex:1.5.19}]
  Let \(i, j \in \set{1, \dots, n}\).
  Then we have
  \begin{align*}
             & \set{\seq{A}{1,,k}} \text{ is linearly independent}                                                         \\
    \implies & \br{\forall \seq{c}{1,,k} \in \F, \sum_{p = 1}^k c_p A_p = \zm \implies c_p = 0}            &  & \by{1.5.3} \\
    \implies & \br{\forall \seq{c}{1,,k} \in \F, \sum_{p = 1}^k c_p (A_p)_{i j} = 0 \implies c_p = 0}      &  & \by{1.2.9} \\
    \implies & \br{\forall \seq{c}{1,,k} \in \F, \sum_{p = 1}^k c_p \tp{(A_p)}_{j i} = 0 \implies c_p = 0} &  & \by{1.3.3} \\
    \implies & \br{\forall \seq{c}{1,,k} \in \F, \sum_{p = 1}^k c_p \tp{(A_p)} = \zm \implies c_p = 0}     &  & \by{1.2.9} \\
    \implies & \set{\tp{A}_1, \tp{A}_2, \dots, \tp{A}_k} \text{ is linearly independent}.                  &  & \by{1.5.3}
  \end{align*}
\end{proof}

\begin{ex}\label{ex:1.5.20}
  Let \(f, g \in \fs(\R, \R)\) be the functions defined by \(f(t) = e^{rt}\) and \(g(t) = e^{st}\), where \(r \neq s\).
  Prove that \(f\) and \(g\) are linearly independent in \(\fs(\R, \R)\).
\end{ex}

\begin{proof}[\pf{ex:1.5.20}]
  Suppose for sake of contradiction that \(f, g\) are linearly dependent.
  Then by \cref{1.5.1} we have
  \[
    \exists a, b \in \R : \begin{dcases}
      af + bg = \zv \\
      \lnot (a = b = 0)
    \end{dcases}.
  \]
  But this means
  \begin{align*}
             & \forall t \in \R, ae^{rt} + be^{st} = 0                                                         \\
    \implies & \forall t \in \R, a = -b e^{(s - r) t}                                                          \\
    \implies & a = b = 0,                              &  & \text{(\(t \mapsto e^{(s - r) t}\) is one-to-one)}
  \end{align*}
  a contradiction.
  Thus \(f, g\) are linearly independent.
\end{proof}

\section{Bases and Dimension}\label{sec:1.6}

\begin{defn}\label{1.6.1}
  A \textbf{basis} \(\beta\) for a vector space \(\V\) over \(\F\) is a linearly independent subset of \(\V\) that generates \(\V\).
  If \(\beta\) is a basis for \(\V\), we also say that the vectors of \(\beta\) form a basis for \(\V\).
\end{defn}

\begin{eg}\label{1.6.2}
  Recalling that \(\spn{\varnothing} = \set{\zv}\) and \(\varnothing\) is linearly independent, we see that \(\varnothing\) is a basis for the zero vector space.
\end{eg}

\begin{eg}\label{1.6.3}
  In \(\vs{F}^n\), let \(e_1 = (1, 0, 0, \dots, 0)\), \(e_2 = (0, 1, 0, \dots, 0)\), \dots, \(e_n = (0, 0, \dots, 0, 1)\);
  \(\set{\seq{e}{1,2,,n}}\) is readily seen to be a basis for \(\vs{F}^n\) and is called the \textbf{standard basis} for \(\vs{F}^n\).
\end{eg}

\begin{proof}[\pf{1.6.3}]
  By \cref{ex:1.5.4} we know that \(\set{\seq{e}{1,2,,n}}\) is linearly independent.
  By \cref{ex:1.4.7} we know that \(\vs{F}^n = \spn{\set{\seq{e}{1,2,,n}}}\).
  Thus by \cref{1.6.1} \(\set{\seq{e}{1,2,,n}}\) is a basis for \(\vs{F}^n\) over \(\F\).
\end{proof}

\begin{eg}\label{1.6.4}
  In \(\MS\), let \(E^{i j}\) denote the matrix whose only nonzero entry is a \(1\) in the \(i\)th row and \(j\)th column.
  Then \(\set{E^{i j} : 1 \leq i \leq m, 1 \leq j \leq n}\) is a basis for \(\MS\).
\end{eg}

\begin{proof}[\pf{1.6.4}]
  By \cref{ex:1.5.6} we know that \(\set{E^{i j} : 1 \leq i \leq m, 1 \leq j \leq n}\) is linearly independent.
  Since
  \begin{align*}
             & \forall A \in \MS, A = \begin{pmatrix}
      A_{1 1} & A_{1 2} & \cdots & A_{1 n} \\
      A_{2 1} & A_{2 2} & \cdots & A_{2 n} \\
      \vdots  & \vdots  & \ddots & \vdots  \\
      A_{m 1} & A_{m 2} & \cdots & A_{m n}
    \end{pmatrix}                                                              \\
             & = \sum_{i = 1}^m \sum_{j = 1}^n A_{i j} E^{i j}                                 &  & \text{(by \cref{1.2.9})} \\
    \implies & \forall A \in \MS, A \in \spn{\set{E^{i j} : 1 \leq i \leq m, 1 \leq j \leq n}} &  & \text{(by \cref{1.4.3})} \\
    \implies & \MS = \spn{\set{E^{i j} : 1 \leq i \leq m, 1 \leq j \leq n}},                   &  & \text{(by \cref{1.5})}
  \end{align*}
  by \cref{1.6.1} we know that \(\set{E^{i j} : 1 \leq i \leq m, 1 \leq j \leq n}\) is a basis for \(\MS\) over \(\F\).
\end{proof}

\begin{eg}\label{1.6.5}
  In \(\ps[n]{\F}\) the set \(\set{1, x, x^2, \dots, x^n}\) is a basis.
  We call this basis the \textbf{standard basis} for \(\ps[n]{\F}\).
\end{eg}

\begin{proof}[\pf{1.6.5}]
  By \cref{ex:1.5.5} we know that \(\set{1, x, x^2, \dots, x^n}\) is linearly independent.
  By \cref{ex:1.4.8} we know that \(\ps[n]{\F} = \spn{\set{1, x, x^2, \dots, x^n}}\).
  Thus by \cref{1.6.1} \(\set{1, x, x^2, \dots, x^n}\) is a basis for \(\ps[n]{\F}\) over \(\F\).
\end{proof}

\begin{eg}\label{1.6.6}
  In \(\ps{\F}\) the set \(\set{1, x, x^2, \dots}\) is a basis.
\end{eg}

\begin{proof}[\pf{1.6.6}]
  Suppose for sake of contradiction that \(\set{1, x, x^2, \dots}\) is not a basis for \(\ps{\F}\).
  Then by \cref{1.6.1} we can split into two cases:
  \begin{itemize}
    \item If \(\set{1, x, x^2, \dots}\) is linearly dependent, then by \cref{ex:1.5.14} we have
          \[
            \begin{dcases}
              \exists x^n \in \set{1, x, x^2, \dots} \\
              \exists \set{\seq{a}{0,1,2,}} \subseteq \F
            \end{dcases} : x^n = \sum_{i \in \N : i \neq n} a_i x^i.
          \]
          By setting \(a_n = -1\) we have
          \begin{align*}
                     & \sum_{i \in \N} a_i x^i = \zv = \sum_{i \in \N} 0x^i                                \\
            \implies & \seq[=]{a}{0,1,2,} = 0.                              &  & \text{(by \cref{1.2.11})}
          \end{align*}
          But this means \(a_n = 0\), a contradiction.
    \item If \(\ps{\F} \neq \spn{\set{1, x, x^2, \dots}}\), then by \cref{1.4.3} we have
          \[
            \exists f \in \ps{\F} : \forall \set{\seq{a}{0,1,2,}} \subseteq \F, f(x) \neq \sum_{i \in \N} a_i x^i.
          \]
          Let \(m\) be the degree of \(f\).
          Then by \cref{1.2.11} we have
          \[
            \exists \seq{c}{0,1,,m} \in \F : f(x) = c_0 + c_1 x + \cdots + c_m x^m.
          \]
          But by setting
          \[
            \begin{dcases}
              a_i = c_i & \text{if } i \leq m \\
              a_i = 0   & \text{if } i > m
            \end{dcases}
          \]
          we have \(f(x) = \sum_{i \in \N} a_i x^i\), a contradiction.
  \end{itemize}
  From all cases above we derived contradictions.
  Thus \(\set{1, x, x^2, \dots}\) is a basis for \(\ps{\F}\).
\end{proof}

\begin{note}
  Observe that \cref{1.6.6} shows that a basis need not be finite.
  In fact, later in \cref{sec:1.6} it is shown that no basis for \(\ps{\F}\) can be finite.
  Hence not every vector space has a finite basis.
\end{note}

\begin{thm}\label{1.8}
  Let \(\V\) be a vector space over \(\F\) and \(\beta = \set{\seq{u}{1,2,,n}}\) be a subset of \(\V\).
  Then \(\beta\) is a basis for \(\V\) if and only if each \(v \in \V\) can be uniquely expressed as a linear combination of vectors of \(\beta\), that is, can be expressed in the form
  \[
    v = \seq[+]{a,u}{1,2,,n}
  \]
  for unique scalars \(\seq{a}{1,2,,n} \in \F\).
\end{thm}

\begin{proof}[\pf{1.8}]
  First suppose that \(\beta\) be a basis for \(\V\).
  If \(v \in V\), then \(v \in \spn{\beta}\) because \(\spn{\beta} = \V\).
  Thus \(v\) is a linear combination of the vectors of \(\beta\).
  Suppose that
  \[
    v = \seq[+]{a,u}{1,2,,n} \quad \text{and} \quad v = \seq[+]{b,u}{1,2,,n}
  \]
  are two such representations of \(v\).
  Subtracting the second equation from the first gives
  \[
    \zv = (a_1 - b_1) u_1 + (a_2 - b_2) u_2 + \cdots + (a_n - b_n) u_n.
  \]
  Since \(\beta\) is linearly independent, it follows that \(a_1 - b_1 = a_2 - b_2 = \cdots = a_n - b_n = 0\).
  Hence \(a_1 = b_1, a_2 = b_2, \dots, a_n = b_n\), and so \(v\) is uniquely expressible as a linear combination of the vectors of \(\beta\).

  Now suppose that each \(v \in \V\) can be uniquely expressed as a linear combination of vectors of \(\beta\).
  By \cref{1.4.3} and \cref{1.5} this means \(\V = \spn{\beta}\).
  Thus to show that \(\beta\) is a basis for \(\V\), by \cref{1.6.1} we need to show that \(\beta\) is linearly independent.
  This is true since
  \begin{align*}
             & \zv \in \V                                             &  & \text{(by \ref{vs3})}    \\
    \implies & \exists! \seq{a}{1,2,,n} \in \F :                                                    \\
             & \zv = \seq[+]{a,u}{1,2,,n} = \seq[+]{0u}{1,2,,n}       &  & \text{(by hypothesis)}   \\
    \implies & \seq[=]{a}{1,2,,n} = 0                                                               \\
    \implies & \set{\seq{u}{1,2,,n}} \text{ is linearly independent}. &  & \text{(by \cref{1.5.3})}
  \end{align*}
\end{proof}

\begin{note}
  \cref{1.8} shows that if the vectors \(\seq{u}{1,2,,n}\) form a basis for a vector space \(\V\), then every vector in \(\V\) can be uniquely expressed in the form
  \[
    v = \seq[+]{a,u}{1,2,,n}
  \]
  for appropriately chosen scalars \(\seq{a}{1,2,,n}\).
  Thus \(v\) determines a unique \(n\)-tuple of scalars \(\tuple{a}{1,2,,n}\) and, conversely, each \(n\)-tuple of scalars determines a unique vector \(v \in \V\) by using the entries of the \(n\)-tuple as the coefficients of a linear combination of \(\seq{u}{1,2,,n}\).
  This fact suggests that \(\V\) is like the vector space \(\vs{F}^n\), where \(n\) is the number of vectors in the basis for \(\V\).
  We see in \cref{sec:2.4} that this is indeed the case.
\end{note}

\begin{thm}\label{1.9}
  If a vector space \(\V\) over \(\F\) is generated by a finite set \(S\), then some subset of \(S\) is a basis for \(\V\).
  Hence \(\V\) has a finite basis.
\end{thm}

\begin{proof}[\pf{1.9}]
  If \(S = \varnothing\) or \(S = \set{\zv}\), then \(\V = \set{\zv}\) and \(\varnothing\) is a subset of \(S\) that is a basis for \(\V\).
  Otherwise \(S\) contains a nonzero vector \(u_1\).
  By \cref{1.5.4}(b), \(\set{u_1}\) is a linearly independent set.
  Continue, if possible, choosing vectors \(\seq{u}{2,,k}\) in \(S\) such that \(\set{\seq{u}{1,2,,k}}\) is linearly independent.
  Since \(S\) is a finite set, we must eventually reach a stage at which \(\beta = \set{\seq{u}{1,2,,k}}\) is a linearly independent subset of \(S\), but adjoining to \(\beta\) any vector in \(S \setminus \beta\) produces a linearly dependent set.
  We claim that \(\beta\) is a basis for \(\V\).
  Because \(\beta\) is linearly independent by construction, it suffices to show that \(\beta\) spans \(\V\).
  By \cref{1.5} we need to show that \(S \subseteq \spn{\beta}\).
  Let \(v \in S\).
  If \(v \in \beta\), then clearly \(v \in \spn{\beta}\).
  Otherwise, if \(v \notin \beta\), then the preceding construction shows that \(\beta \cup \set{v}\) is linearly dependent.
  So \(v \in \spn{\beta}\) by \cref{1.7}.
  Thus \(S \subseteq \spn{\beta}\).
\end{proof}

\begin{note}
  Because of the method by which the basis \(\beta\) was obtained in the proof of \cref{1.9}, this theorem is often remembered as saying that \emph{a finite spanning set for \(\V\) can be reduced to a basis for \(\V\).}
\end{note}

\begin{thm}[Replacement Theorem]\label{1.10}
  Let \(\V\) be a vector space over \(\F\) that is generated by a set \(G\) containing exactly \(n\) vectors, and let \(L\) be a linearly independent subset of \(\V\) containing exactly \(m\) vectors.
  Then \(m \leq n\) and there exists a subset \(H\) of \(G\) containing exactly \(n - m\) vectors such that \(L \cup H\) generates \(\V\).
\end{thm}

\begin{proof}[\pf{1.10}]
  The proof is by mathematical induction on \(m\).
  The induction begins with \(m = 0\);
  for in this case \(L = \varnothing\), and so taking \(H = G\) gives the desired result.

  Now suppose that the theorem is true for some integer \(m \geq 0\).
  We prove that the theorem is true for \(m + 1\).
  Let \(L = \set{\seq{v}{1,2,,m + 1}}\) be a linearly independent subset of \(\V\) consisting of \(m + 1\) vectors.
  By \cref{1.5.6} \(\set{\seq{v}{1,2,,m}} \subseteq L\) is linearly independent, and so we may apply the induction hypothesis to conclude that \(m \leq n\) and that there is a subset \(\set{\seq{u}{1,2,,n - m}}\) of \(G\) such that \(\set{\seq{v}{1,2,,m}} \cup \set{\seq{u}{1,2,,n - m}}\) generates \(\V\).
  Thus there exist scalars \(\seq{a}{1,2,,m}, \seq{b}{1,2,,n - m} \in \F\) such that
  \begin{equation}\label{eq:1.6.1}
    \seq[+]{a,v}{1,2,,m} + \seq[+]{b,u}{1,2,,n - m} = v_{m + 1}
  \end{equation}
  Note that \(n - m > 0\), otherwise \(n - m = 0\) implies \(v_{m + 1}\) is a linear combination of \(\seq{v}{1,2,,m}\), which by \cref{1.7} contradicts the assumption that \(L\) is linearly independent.
  Hence \(n > m\);
  that is, \(n \geq m + 1\).
  Moreover, some \(b_i\), say \(b_1\), is nonzero, for otherwise we obtain the same contradiction.
  Solving \cref{eq:1.6.1} for \(u_1\) gives
  \begin{multline*}
    u_1 = (-b_1^{-1} a_1) v_1 + (-b_1^{-1} a_2) v_2 + \cdots + (-b_1^{-1} a_m) v_m + (b_1^{-1}) v_{m + 1} \\
    + (-b_1^{-1} b_2) u_2 + \cdots + (-b_1^{-1} b_{n - m}) u_{n - m}.
  \end{multline*}
  Let \(H = \set{\seq{u}{2,,n - m}}\).
  Then \(u_1 \in \spn{L \cup H}\), and because \(\seq{v}{1,2,,m}\), \\
  \(\seq{u}{2,,n - m}\) are clearly in \(\spn{L \cup H}\), it follows that
  \[
    \set{\seq{v}{1,2,,m}, \seq{u}{1,2,,n - m}} \subseteq \spn{L \cup H}.
  \]
  Because \(\set{\seq{v}{1,2,,m}, \seq{u}{1,2,n - m}}\) generates \(\V\), \cref{1.5} implies that \\
  \(\spn{L \cup H} = \V\).
  Since \(H\) is a subset of \(G\) that contains \((n - m) - 1 = n - (m + 1)\) vectors, the theorem is true for \(m + 1\).
  This completes the induction.
\end{proof}

\begin{cor}\label{1.6.7}
  Let \(\V\) be a vector space over \(\F\) having a finite basis.
  Then every basis for \(\V\) contains the same number of vectors.
\end{cor}

\begin{proof}[\pf{1.6.7}]
  Suppose that \(\beta\) is a finite basis for \(\V\) that contains exactly \(n\) vectors, and let \(\gamma\) be any other basis for \(\V\).
  If \(\gamma\) contains more than \(n\) vectors, then we can select a subset \(S\) of \(\gamma\) containing exactly \(n + 1\) vectors.
  Since \(S\) is linearly independent and \(\beta\) generates \(\V\), the replacement theorem (\cref{1.10}) implies that \(n + 1 \leq n\), a contradiction.
  Therefore \(\gamma\) is finite, and the number \(m\) of vectors in \(\gamma\) satisfies \(m \leq n\).
  Reversing the roles of \(\beta\) and \(\gamma\) and arguing as above, we obtain \(n \leq m\).
  Hence \(m = n\).
\end{proof}

\begin{note}
  If a vector space has a finite basis, \cref{1.6.7} asserts that the number of vectors in \emph{any} basis for \(\V\) is an intrinsic property of \(\V\).
\end{note}

\begin{defn}\label{1.6.8}
  A vector space is called \textbf{finite-dimensional} if it has a basis consisting of a finite number of vectors.
  The unique number of vectors in each basis for \(\V\) is called the \textbf{dimension} of \(\V\) and is denoted by \(\dim(\V)\).
  A vector space that is not finite-dimensional is called \textbf{infinite-dimensional}.
\end{defn}

\begin{eg}\label{1.6.9}
  The vector space \(\set{\zv}\) has dimension zero.
\end{eg}

\begin{proof}[\pf{1.6.9}]
  By \cref{1.6.2} and \cref{1.6.7} we are done.
\end{proof}

\begin{eg}\label{1.6.10}
  The vector space \(\vs{F}^n\) has dimension \(n\).
\end{eg}

\begin{proof}[\pf{1.6.10}]
  By \cref{1.6.3} and \cref{1.6.7} we are done.
\end{proof}

\begin{eg}\label{1.6.11}
  The vector space \(\MS\) has dimension \(mn\).
\end{eg}

\begin{proof}[\pf{1.6.11}]
  By \cref{1.6.4} and \cref{1.6.7} we are done.
\end{proof}

\begin{eg}\label{1.6.12}
  The vector space \(\ps[n]{\F}\) has dimension \(n + 1\).
\end{eg}

\begin{proof}[\pf{1.6.12}]
  By \cref{1.6.5} and \cref{1.6.7} we are done.
\end{proof}

\begin{eg}\label{1.6.13}
  Over the field of complex numbers, the vector space of complex numbers has dimension \(1\).
  (A basis is \(\set{1}\).)
\end{eg}

\begin{proof}[\pf{1.6.13}]
  We have
  \begin{align*}
             & \forall c \in \C, c = c \cdot 1                               \\
    \implies & \spn{\set{1}} = \C              &  & \text{(by \cref{1.5})}   \\
    \implies & \#\pa{\set{1}} = 1 = \dim(\C).  &  & \text{(by \cref{1.6.8})}
  \end{align*}
\end{proof}

\begin{eg}\label{1.6.14}
  Over the field of real numbers, the vector space of complex numbers has dimension \(2\).
  (A basis is \(\set{1, i}\).)
\end{eg}

\begin{proof}[\pf{1.6.14}]
  We have
  \begin{align*}
             & \forall c \in \C, c = \Re(c) + i \Im(c) = \Re(c) \cdot 1 + \Im(c) \cdot i &  & (\Re(c), \Im(c) \in \R)  \\
    \implies & \spn{\set{1, i}} = \C                                                     &  & \text{(by \cref{1.5})}   \\
    \implies & \#\pa{\set{1, i}} = 2 = \dim(\C).                                         &  & \text{(by \cref{1.6.8})}
  \end{align*}
\end{proof}

\begin{note}
  From \cref{1.6.13} and \cref{1.6.14} we see that the dimension of a vector space depends on its field of scalars.
\end{note}

\begin{note}
  In the terminology of dimension, the first conclusion in the replacement theorem states that if \(\V\) is a finite-dimensional vector space over \(\F\), then no linearly independent subset of \(\V\) can contain more than \(\dim(\V)\) vectors.
  From this fact it follows that the vector space \(\ps{\F}\) over \(\F\) is infinite-dimensional because it has an infinite linearly independent set, namely \(\set{1, x, x^2, \dots}\).
  This set is, in fact, a basis for \(\ps{\F}\).
  Yet nothing that we have proved in this section guarantees an infinite-dimensional vector space must have a basis.
  In \cref{sec:1.7} it is shown, however, that \emph{every vector space has a basis}.
\end{note}

\begin{cor}\label{1.6.15}
  Let \(\V\) be a vector space over \(\F\) with dimension \(n\).
  \begin{enumerate}
    \item Any finite generating set for \(\V\) contains at least \(n\) vectors, and a generating set for \(\V\) that contains exactly \(n\) vectors is a basis for \(\V\).
    \item Any linearly independent subset of \(\V\) that contains exactly \(n\) vectors is a basis for \(\V\).
    \item Every linearly independent subset of \(\V\) can be extended to a basis for \(\V\).
  \end{enumerate}
\end{cor}

\begin{proof}[\pf{1.6.15}]
  Let \(\beta\) be a basis for \(\V\).
  \begin{enumerate}
    \item Let \(G\) be a finite generating set for \(\V\).
          By \cref{1.9} some subset \(H\) of \(G\) is a basis for \(\V\).
          \cref{1.6.7} implies that \(H\) contains exactly \(n\) vectors.
          Since a subset of \(G\) contains \(n\) vectors, \(G\) must contain at least \(n\) vectors.
          Moreover, if \(G\) contains exactly \(n\) vectors, then we must have \(H = G\), so that \(G\) is a basis for \(\V\).
    \item Let \(L\) be a linearly independent subset of \(\V\) containing exactly \(n\) vectors.
          It follows from the replacement theorem that there is a subset \(H\) of \(\beta\) containing \(n - n = 0\) vectors such that \(L \cup H\) generates \(\V\).
          Thus \(H = \varnothing\), and \(L\) generates \(\V\).
          Since \(L\) is also linearly independent, \(L\) is a basis for \(\V\).
    \item If \(L\) is a linearly independent subset of \(\V\) containing \(m\) vectors, then the replacement theorem asserts that there is a subset \(H\) of \(\beta\) containing exactly \(n - m\) vectors such that \(L \cup H\) generates \(\V\).
          Now \(L \cup H\) contains at most \(n\) vectors;
          therefore (a) implies that \(L \cup H\) contains exactly \(n\) vectors and that \(L \cup H\) is a basis for \(\V\).
  \end{enumerate}
\end{proof}

\begin{eg}\label{1.6.16}
  For \(k = 0, 1, \dots, n\), let \(p_k(x) = x^k + x^{k + 1} + \cdots + x^n\).
  It follows from \cref{1.5.5} and \cref{1.6.15}(b) that
  \[
    \set{p_0(x), p_1(x), \dots, p_n(x)}
  \]
  is a basis for \(\ps[n]{\F}\).
\end{eg}

\begin{thm}\label{1.11}
  Let \(\W\) be a subspace of a finite-dimensional vector space \(\V\) over \(\F\).
  Then \(\W\) is finite-dimensional and \(\dim(\W) \leq \dim(\V)\).
  Moreover, if \(\dim(\W) = \dim(\V)\), then \(\V = \W\).
\end{thm}

\begin{proof}[\pf{1.11}]
  Let \(\dim(\V) = n\).
  If \(\W = \set{\zv}\), then \(\W\) is finite-dimensional and \(\dim(\W) = 0 \leq n\).
  Otherwise, \(\W\) contains a nonzero vector \(x_1\);
  so \(\set{x_1}\) is a linearly independent set.
  Continue choosing vectors, \(\seq{x}{1,2,,k}\) in \(\W\) such that \(\set{\seq{x}{1,2,,k}}\) is linearly independent.
  Since no linearly independent subset of \(\V\) can contain more than \(n\) vectors, this process must stop at a stage where \(k \leq n\) and \(\set{\seq{x}{1,2,,k}}\) is linearly independent but adjoining any other vector from \(\W\) produces a linearly dependent set.
  \cref{1.7} implies that \(\set{\seq{x}{1,2,,k}}\) generates \(\W\), and hence it is a basis for \(\W\).
  Therefore \(\dim(\W) = k \leq n\).

  If \(\dim(\W) = n\), then a basis for \(\W\) is a linearly independent subset of \(\V\) containing \(n\) vectors.
  But \cref{1.6.15}(b) implies that this basis for \(\W\) is also a basis for \(\V\);
  so \(\W = \V\).
\end{proof}

\begin{eg}\label{1.6.17}
  The set of diagonal \(n \times n\) matrices is a subspace \(\W\) of \(\ms{n}{n}{\F}\)
  (see \cref{1.3.8}).
  A basis for \(\W\) is
  \[
    \set{E^{1 1}, E^{2 2}, \dots, E^{n n}},
  \]
  where \(E^{i j}\) is the matrix in which the only nonzero entry is a \(1\) in the \(i\)th row and \(j\)th column.
  Thus \(\dim(\W) = n\).
\end{eg}

\begin{proof}[\pf{1.6.17}]
  By \cref{ex:1.5.6} we know that \(\set{E^{1 1}, E^{2 2}, \dots, E^{n n}}\) is linearly independent.
  Since \(\W = \spn{\set{E^{1 1}, E^{2 2}, \dots, E^{n n}}}\), by \cref{1.6.15}(a) we know that \(\dim(\W) \leq n\).
  By \cref{1.6.15}(c) we also know that \(\dim(\W) \geq n\).
  Thus we have \(\dim(\W) = n\).
\end{proof}

\begin{eg}\label{1.6.18}
  The set of symmetric \(n \times n\) matrices is a subspace \(\W\) of \(\ms{n}{n}{\F}\) over \(\F\).
  A basis for \(\W\) is
  \[
    \set{A^{i j} : 1 \leq i \leq j \leq n}
  \]
  where \(A^{i j}\) is the \(n \times n\) matrix having \(1\) in the \(i\)th row and \(j\)th column, \(1\) in the \(j\)th row and \(i\)th column, and \(0\) elsewhere.
  It follows that
  \[
    \dim(\W) = n + (n - 1) + \cdots + 1 = \frac{1}{2} n(n + 1).
  \]
\end{eg}

\begin{proof}[\pf{1.6.18}]
  By \cref{ex:1.5.6} we see that each \(A^{i j}\) can only express as \(E^{i j} + E^{j i}\).
  Thus \(\set{A^{i j} : 1 \leq i \leq j \leq n}\) is linearly independent and by \cref{1.6.15}(c)we have  \(\dim(\W) \geq \#\pa{\set{A^{i j} : 1 \leq i \leq j \leq n}}\).
  Since
  \begin{align*}
             & \forall M \in \W, M_{i j} = M_{j i} = M_{i j} \cdot 1                                                 \\
    \implies & \forall M \in \W, M = \sum_{i = 1}^n \sum_{j = i}^n M_{i j} A^{i j} &  & \text{(by \cref{1.2.9})}     \\
    \implies & \W = \spn{\set{A^{i j} : 1 \leq i \leq j \leq n}}                   &  & \text{(by \cref{1.5})}       \\
    \implies & \dim(\W) \leq \#\pa{\set{A^{i j} : 1 \leq i \leq j \leq n}},        &  & \text{(by \cref{1.6.15}(a))}
  \end{align*}
  we have \(\dim(\W) = \#\pa{\set{A^{i j} : 1 \leq i \leq j \leq n}} = \frac{1}{2} n(n + 1)\).
\end{proof}

\begin{cor}\label{1.6.19}
  If \(\W\) is a subspace of a finite-dimensional vector space \(\V\) over \(\F\), then any basis for \(\W\) can be extended to a basis for \(\V\).
\end{cor}

\begin{proof}[\pf{1.6.19}]
  Let \(S\) be a basis for \(\W\).
  Because \(S\) is a linearly independent subset of \(\V\), \cref{1.6.15}(c) guarantees that \(S\) can be extended to a basis for \(\V\).
\end{proof}

\begin{defn}[The Lagrange Interpolation Formula]\label{1.6.20}
  Let \(\seq{c}{0,1,,n}\) be distinct scalars in an infinite field \(\F\).
  The polynomials \(f_0(x), f_1(x), \dots, f_n(x)\) defined by
  \[
    f_i(x) = \frac{(x - c_0) \cdots (x - c_{i - 1}) (x - c_{i + 1}) \cdots (x - c_n)}{(c_i - c_0) \cdots (c_i - c_{i - 1}) (c_i - c_{i + 1}) \cdots (c_i - c_n)} = \prod_{\substack{k = 0 \\ k \neq i}}^n \frac{x - c_k}{c_i - c_k}
  \]
  are called the \textbf{Lagrange polynomials} (associated with \(\seq{c}{0,1,,n}\)).
  Note that each \(f_i(x)\) is a polynomial of degree \(n\) and hence is in \(\ps[n]{\F}\).
  By regarding \(f_i(x)\) as a polynomial function \(f_i : \F \to \F\), we see that
  \begin{equation}\label{eq:1.6.2}
    f_i(c_j) = \begin{dcases}
      0 & \text{if } i \neq j \\
      1 & \text{if } i = j
    \end{dcases}.
  \end{equation}

  This property of Lagrange polynomials can be used to show that \(\beta = \set{\seq{f}{0,1,,n}}\) is a linearly independent subset of \(\ps[n]{\F}\).
  Suppose that
  \[
    \sum_{i = 0}^n a_i f_i = \zv \quad \text{for some scalars } \seq{a}{0,1,,n},
  \]
  where \(\zv\) denotes the zero function.
  Then
  \[
    \sum_{i = 0}^n a_i f_i(c_j) = 0 \quad \text{for } j = 0, 1, \dots, n.
  \]
  But also
  \[
    \sum_{i = 0}^n a_i f_i(c_j) = a_j
  \]
  by \cref{eq:1.6.2}.
  Hence \(a_j = 0\) for \(j = 0, 1, \dots, n\);
  so \(\beta\) is linearly independent.
  Since the dimension of \(\ps[n]{\F}\) is \(n + 1\), it follows from \cref{1.6.15} that \(\beta\) is a basis for \(\ps[n]{\F}\).

  Because \(\beta\) is a basis for \(\ps[n]{\F}\), every polynomial function \(g\) in \(\ps[n]{\F}\) is a linear combination of polynomial functions of \(\beta\), say,
  \[
    g = \sum_{i = 0}^n b_i f_i.
  \]
  It follows that
  \[
    g(c_j) = \sum_{i = 0}^n b_i f_i(c_j) = b_j;
  \]
  so
  \[
    g = \sum_{i = 0}^n g(c_i) f_i
  \]
  is the unique representation of \(g\) as a linear combination of elements of \(\beta\).
  This representation is called the \textbf{Lagrange interpolation formula}.
  Notice that the preceding argument shows that if \(\seq{b}{0,1,,n}\) are any \(n + 1\) scalars in \(\F\) (not necessarily distinct), then the polynomial function
  \[
    g = \sum_{i = 0}^n b_i f_i
  \]
  is the unique polynomial in \(\ps[n]{\F}\) such that \(g(c_j) = b_j\).
  Thus we have found the unique polynomial of degree not exceeding \(n\) that has specified values \(b_j\) at given points \(c_j\) in its domain (\(j = 0, 1, \dots, n\)).

  An important consequence of the Lagrange interpolation formula is the following result:
  If \(f \in \ps[n]{\F}\) and \(f(c_i) = 0\) for \(n + 1\) distinct scalars \(\seq{c}{0,1,,n}\) in \(\F\), then \(f\) is the zero function.
\end{defn}

\exercisesection

\setcounter{ex}{10}
\begin{ex}\label{ex:1.6.11}
  Let \(u\) and \(v\) be distinct vectors of a vector space \(\V\) over \(\F\).
  Show that if \(\set{u, v}\) is a basis for \(\V\) and \(a\) and \(b\) are nonzero scalars, then both \(\set{u + v, au}\) and \(\set{au, bv}\) are also bases for \(\V\).
\end{ex}

\begin{proof}[\pf{ex:1.6.11}]
  Let \(c_1, c_2 \in \F\).
  Since
  \begin{align*}
             & c_1 (u + v) + c_2 (au) = \zv                                \\
    \implies & (c_1 + c_2 a) u + c_1 v = \zv &  & \text{(by \cref{1.2.1})} \\
    \implies & \begin{dcases}
      c_1 + c_2 a = 0 \\
      c_1 = 0
    \end{dcases}    &  & \text{(by \cref{1.5.3})} \\
    \implies & \begin{dcases}
      c_2 a = 0 \\
      c_1 = 0
    \end{dcases}                                  \\
    \implies & c_1 = c_2 = 0                 &  & (a \neq 0)
  \end{align*}
  and
  \begin{align*}
             & c_1 (au) + c_2 (bv) = \zv                                 \\
    \implies & (c_1 a) u + (c_2 b) v = \zv &  & \text{(by \cref{1.2.1})} \\
    \implies & \begin{dcases}
      c_1 a = 0 \\
      c_2 b = 0
    \end{dcases}  &  & \text{(by \cref{1.5.3})} \\
    \implies & c_1 = c_2 = 0,              &  & (a \neq 0 \neq b)
  \end{align*}
  by \cref{1.5.3} we know that \(\set{u + v, au}\) and \(\set{au, bv}\) are linearly independent.
  Since
  \[
    \#(\set{u + v, au}) = \#(\set{au, bv}) = 2 = \#(\set{u, v}),
  \]
  by \cref{1.6.15}(a) we know that \(\set{u + v, au}\) and \(\set{au, bv}\) are basis for \(\V\).
\end{proof}

\begin{ex}\label{ex:1.6.12}
  Let \(u, v\), and \(w\) be distinct vectors of a vector space \(\V\) over \(\F\).
  Show that if \(\set{u, v, w}\) is a basis for \(\V\), then \(\set{u + v + w, v + w, w}\) is also a basis for \(\V\).
\end{ex}

\begin{proof}[\pf{ex:1.6.12}]
  Let \(a, b, c \in \F\).
  Since
  \begin{align*}
             & a(u + v + w) + b(v + w) + cw = \zv                               \\
    \implies & au + (a + b)v + (a + b + c)w = \zv &  & \text{(by \cref{1.2.1})} \\
    \implies & \begin{dcases}
      a = 0     \\
      a + b = 0 \\
      a + b + c = 0
    \end{dcases}         &  & \text{(by \cref{1.5.3})} \\
    \implies & a = b = c = 0,
  \end{align*}
  by \cref{1.5.3} we know that \(\set{u + v + w, v + w, w}\) is linearly independent.
  Since
  \[
    \#(\set{u + v + w, v + w, w}) = 3 = \#(\set{u, v, w}),
  \]
  by \cref{1.6.15}(a) we know that \(\set{u + v + w, v + w, w}\) is a basis for \(\V\).
\end{proof}

\setcounter{ex}{14}
\begin{ex}\label{ex:1.6.15}
  The set of all \(n \times n\) matrices having trace equal to zero is a subspace \(\W\) of \(\ms{n}{n}{\F}\) (see \cref{1.3.9}).
  Find a basis for \(\W\).
  What is the dimension of \(\W\)?
\end{ex}

\begin{proof}[\pf{ex:1.6.15}]
  Let \(E^{i j} \in \ms{n}{n}{\F}\) be matrix defined as in \cref{ex:1.5.6} and let \(\beta\) be the set
  \[
    \beta = \set{E^{i j} : i, j \in \set{1, \dots, n}, i \neq j} \cup \set{E^{i i} - E^{1 1} : 2 \leq i \leq n}.
  \]
  Observe that
  \begin{align*}
             & \forall A \in \W, \begin{dcases}
      \tr{A} = 0 \\
      A = \sum_{i = 1}^n \sum_{j = 1}^n A_{i j} E^{i j}
    \end{dcases}                   &  & \text{(by \cref{ex:1.5.6})} \\
    \implies & \forall A \in \W, \begin{dcases}
      A_{1 1} + A_{2 2} + \cdots + A_{n n} = 0 \\
      A = \sum_{i = 1}^n \sum_{j = 1}^n A_{i j} E^{i j}
    \end{dcases}                   &  & \text{(by \cref{1.3.9})}    \\
    \implies & \forall A \in \W, \begin{dcases}
      A_{2 2} + \cdots + A_{n n} = -A_{1 1}        \\
      A = \pa{\sum_{i = 1}^n \sum_{\substack{j = 1 \\ j \neq i}}^n A_{i j} E^{i j}} + \pa{\sum_{i = 1}^n A_{i i} E^{i i}}
    \end{dcases}                                                    \\
    \implies & \forall A \in \W, A = \pa{\sum_{i = 1}^n \sum_{\substack{j = 1                                  \\ j \neq i}}^n A_{i j} E^{i j}} + \pa{\sum_{i = 2}^n A_{i i} (E^{i i} - E^{1 1})} \\
    \implies & \forall A \in \W, A \in \spn{\beta}                            &  & \text{(by \cref{1.4.3})}    \\
    \implies & \W \subseteq \spn{\beta}.
  \end{align*}
  Since
  \[
    \tr{E^{i j}} = 0 \quad \forall i, j \in \set{1, \dots, n} \text{ and } i \neq j
  \]
  and
  \[
    \tr{E^{i i} - E^{1 1}} = 1 + (-1) = 0 \quad \forall i \in \set{2, \dots, n},
  \]
  we know that \(\beta \subseteq \W\).
  Thus by \cref{1.5} we have \(\W = \spn{\beta}\).
  By \cref{ex:1.5.6} we know that \(\beta\) is linearly independent, thus by \cref{1.6.1} \(\beta\) is a basis for \(\W\), and by \cref{1.6.8} we know that \(\dim(\W) = n^2 - 1\).
\end{proof}

\begin{ex}\label{ex:1.6.16}
  The set of all upper triangular \(n \times n\) matrices is a subspace \(\W\) of \(\ms{n}{n}{\F}\) (see \cref{ex:1.3.12}).
  Find a basis for \(\W\).
  What is the dimension of \(\W\)?
\end{ex}

\begin{proof}[\pf{ex:1.6.16}]
  Let \(E^{i j} \in \ms{n}{n}{\F}\) be matrix defined as in \cref{ex:1.5.6} and let \(\beta\) be the set
  \[
    \beta = \set{E^{i j} : i, j \in \set{1, \dots, n}, i \leq j}.
  \]
  Clearly \(\beta \subseteq \W\).
  Since
  \begin{align*}
             & \forall A \in \W, A = \sum_{i = 1}^n \sum_{j = 1}^n A_{i j} E^{i j} = \sum_{i = 1}^n \sum_{j = i}^n A_{i j} E^{i j} &  & \text{(by \cref{ex:1.3.12})} \\
    \implies & \forall A \in \W, A \in \spn{\beta}                                                                                 &  & \text{(by \cref{1.4.3})}     \\
    \implies & \W = \spn{\beta}                                                                                                    &  & \text{(by \cref{1.5})}
  \end{align*}
  and \(\beta\) is linearly independent (by \cref{ex:1.5.6}), by \cref{1.6.1} we know that \(\beta\) is a basis for \(\W\).
  Thus by \cref{1.6.8} we have \(\dim(\W) = \frac{1}{2} n(n + 1)\).
\end{proof}

\begin{ex}\label{ex:1.6.17}
  The set of all skew-symmetric \(n \times n\) matrices is a subspace \(\W\) of \(\ms{n}{n}{\F}\) (see \cref{ex:1.3.28}).
  Find a basis for \(\W\).
  What is the dimension of \(\W\)?
\end{ex}

\begin{proof}[\pf{ex:1.6.17}]
  Let \(E^{i j} \in \ms{n}{n}{\F}\) be matrix defined as in \cref{ex:1.5.6} and let \(\beta\) be the set
  \[
    \beta = \set{E^{i j} - E^{j i} : i, j \in \set{1, \dots, n}, i < j}.
  \]
  By \cref{ex:1.3.28} we know that \(\beta \subseteq \W\).
  Since
  \begin{align*}
             & \forall A \in \W, \tp{A} = -A                                                  &  & \text{(by \cref{ex:1.3.28})} \\
    \implies & \forall A \in \W, A_{j i} = -A_{i j} \text{ where } i, j \in \set{1, \dots, n} &  & \text{(by \cref{1.3.3})}     \\
    \implies & \forall A \in \W, \begin{dcases}
      A_{i j} = -A_{j i} & \text{if } i \neq j \\
      A_{i j} = 0        & \text{if } i = j
    \end{dcases}                                                                     \\
    \implies & \forall A \in \W, A = \sum_{i = 1}^n \sum_{j = 1}^n A_{i j} E^{i j}                                              \\
             & = \sum_{i = 1}^n \sum_{j = 1}^{i - 1} (A_{i j} E^{i j} + A_{j i} E^{j i})                                        \\
             & = \sum_{i = 1}^n \sum_{j = 1}^{i - 1} (A_{i j} E^{i j} - A_{i j} E^{j i})                                        \\
             & = \sum_{i = 1}^n \sum_{j = 1}^{i - 1} A_{i j} (E^{i j} - E^{j i})                                                \\
    \implies & \forall A \in \W, A \in \spn{\beta}                                            &  & \text{(by \cref{1.4.3})}     \\
    \implies & \W = \spn{\beta}                                                               &  & \text{(by \cref{1.5})}
  \end{align*}
  and \(\beta\) is linearly independent (by \cref{ex:1.5.6}), by \cref{1.6.1} we know that \(\beta\) is a basis for \(\W\).
  Thus by \cref{1.6.8} we have \(\dim(\W) = \frac{1}{2} n(n - 1)\).
\end{proof}

\begin{ex}\label{ex:1.6.18}
  Find a basis for the vector space in \cref{1.2.13}.
  Justify your answer.
\end{ex}

\begin{proof}[\pf{ex:1.6.18}]
  Let \(\V\) be the vector space in \cref{1.2.13} over field \(\F\).
  We claim that the set
  \[
    \beta = \set{\set{e_n^i} : e_n^i = 0 \text{ when } n \neq i ; e_n^i = 1 \text{ when } n = i}
  \]
  is a basis for \(\V\).
  Clearly \(\beta \subseteq \V\).
  Since
  \begin{align*}
             & \forall \set{a_n} \in \V, \set{a_n} = \sum_{i \in \N} a_n \set{e_n^i} &  & \text{(by \cref{1.2.13})} \\
    \implies & \V = \spn{\beta}                                                      &  & \text{(by \cref{1.5})}
  \end{align*}
  and \(\beta\) is linearly independent (for obvious reason), by \cref{1.6.1} we know that \(\beta\) is a basis for \(\V\).
\end{proof}

\setcounter{ex}{19}
\begin{ex}\label{ex:1.6.20}
  Let \(\V\) be a vector space over \(\F\) having dimension \(n\), and let \(S\) be a subset of \(\V\) that generates \(\V\).
  \begin{enumerate}
    \item Prove that there is a subset of \(S\) that is a basis for \(\V\).
          (Be careful not to assume that \(S\) is finite.)
    \item Prove that \(S\) contains at least \(n\) vectors.
  \end{enumerate}
\end{ex}

\begin{proof}[\pf{ex:1.6.20}(a)]
  Note that the hypothesis of \cref{ex:1.6.20}(a) is different from \cref{1.9} that in \cref{1.9} \(S\) is assumed to be finite.
  Suppose for sake of contradiction that such subset does not exist.
  By \cref{1.6.1} this means every subset of \(S\) must either be linearly dependent or cannot generate \(\V\).
  Since \(S\) is a subset of \(S\) and \(\spn{S} = \V\), we know that \(S\) must be linearly dependent.

  Now we let \(\beta_0\) be a finite, linearly independent subset of \(S\).
  From previous paragraph we know that \(\beta_0 \neq S\) and \(\V \neq \spn{\beta_0}\).
  By \cref{1.10} we know that \(\#(\beta_0) \leq n\).
  Then there exists a \(v_1 \in S\) such that \(v_1 \cup \beta_0\) is linearly independent.
  If such \(v_1\) does not exist, then we would have \(S \subseteq \spn{\beta_0}\), which implies \(\spn{\beta_0} = \V\), a contradiction.
  So such \(v_1\) exists and we let \(\beta_1 = \beta_0 \cup v_1\).
  Again we must have \(\beta_1 \neq S\), \(\V \neq \spn{\beta_1}\) and \(\#(\beta_1) \leq n\).
  Using the same argument as above there must exist a \(v_2 \in S\) such that \(v_2 \cup \beta_1\) is linearly independent.
  Continue this definition we can define \(\beta_n = \set{\seq{v}{1,2,,n}}\).
  But by \cref{1.6.15}(b) we know that \(\beta_n\) must be a basis for \(\V\), a contradiction.
  Thus there must exists a subset of \(S\) which is a basis for \(\V\).
\end{proof}

\begin{proof}[\pf{ex:1.6.20}(b)]
  From \cref{ex:1.6.20}(a) we know that there exists a subset of \(S\) which is a basis for \(\V\).
  Thus by \cref{1.6.15}(a) \(S\) must has at least \(n\) vectors.
\end{proof}

\begin{ex}\label{ex:1.6.21}
  Prove that a vector space is infinite-dimensional if and only if it contains an infinite linearly independent subset.
\end{ex}

\begin{proof}[\pf{ex:1.6.21}]
  Let \(\V\) be a vector spaces over \(\F\).
  Then
  \begin{align*}
         & \V \text{ is infinite-dimensional}                                                    \\
    \iff & \V \text{ has an infinite basis}                        &  & \text{(by \cref{1.6.8})} \\
    \iff & \V \text{ has an infinite linearly independent subset}. &  & \text{(by \cref{1.6.1})}
  \end{align*}
\end{proof}

\begin{ex}\label{ex:1.6.22}
  Let \(\W_1\) and \(\W_2\) be subspaces of a finite-dimensional vector space \(\V\) over \(\F\).
  Determine necessary and sufficient conditions on \(\W_1\) and \(\W_2\) so that \(\dim(\W_1 \cap \W_2) = \dim(\W_1)\).
\end{ex}

\begin{proof}[\pf{ex:1.6.22}]
  We have
  \begin{align*}
         & \dim(\W_1 \cap \W_2) = \dim(\W_1)                                                     \\
    \iff & \begin{dcases}
      \exists \beta_1 \subseteq \W_1 \cap \W_2 \\
      \exists \beta_2 \subseteq \W_1
    \end{dcases} : \begin{dcases}
      \beta_1 \text{ is a basis for } \W_1 \cap \W_2 \\
      \beta_2 \text{ is a basis for } \W_1           \\
      \#(\beta_1) = \#(\beta_2)
    \end{dcases} &  & \text{(by \cref{1.6.8})} \\
    \iff & \W_1 \cap \W_2 = \W_1                                   &  & \text{(by \cref{1.11})}  \\
    \iff & \W_1 \subseteq \W_2.
  \end{align*}
\end{proof}

\begin{ex}\label{ex:1.6.23}
  Let \(\seq{v}{1,2,,k}, v\) be vectors in a vector space \(\V\), and define \(W_1 = \spn{\set{\seq{v}{1,2,,k}}}\), and \(\W_2 = \spn{\set{\seq{v}{1,2,,k}, v}}\).
  \begin{enumerate}
    \item Find necessary and sufficient conditions on \(v\) such that \(\dim(\W_1) = \dim(\W_2)\).
    \item State and prove a relationship involving \(\dim(\W_1)\) and \(\dim(\W_2)\) in the case that \(\dim(\W_1) \neq \dim(\W_2)\).
  \end{enumerate}
\end{ex}

\begin{proof}[\pf{ex:1.6.23}(a)]
  We have
  \begin{align*}
         & \begin{dcases}
      \W_1 \subseteq \W_2 \\
      \dim(\W_1) = \dim(\W_2)
    \end{dcases}                                  &  & \text{(by \cref{ex:1.4.13})} \\
    \iff & \W_1 = \W_2                                                 &  & \text{(by \cref{1.11})}      \\
    \iff & v \in \spn{\set{\seq{v}{1,2,,k}}}                           &  & \text{(by \cref{1.4.3})}     \\
    \iff & v \cup \set{\seq{v}{1,2,,k}} \text{ is linearly dependent}. &  & \text{(by \cref{1.7})}
  \end{align*}
\end{proof}

\begin{proof}[\pf{ex:1.6.23}(b)]
  By \cref{1.11} we have
  \[
    \begin{dcases}
      \W_1 \subseteq \W_2 \\
      \W_1 \neq \W_2
    \end{dcases} \implies \dim(\W_1) \neq \dim(\W_2).
  \]
\end{proof}

\begin{ex}\label{ex:1.6.24}
  Let \(f(x)\) be a polynomial of degree n in \(\ps[n]{\R}\).
  Prove that for any \(g(x) \in \ps[n]{\R}\) there exist scalars \(\seq{c}{0,1,,n}\) such that
  \[
    g(x) = c_0 f(x) + c_1 f'(x) + c_2 f''(x) + \cdots + c_n f^{(n)}(x),
  \]
  where \(f^{(n)}(x)\) denotes the \(n\)th derivative of \(f(x)\).
\end{ex}

\begin{proof}[\pf{ex:1.6.24}]
  We denote \(f^{(0)} = f\).
  Since \(f^{(i)}(x)\) has degree \(n - i\) for all \(i = 0, 1, \dots, n\), we know that the set
  \[
    \beta = \set{f^{(i)} : i = 0, 1, \dots, n}
  \]
  is linearly independent.
  Since \(\#(\beta) = n + 1\), by \cref{1.6.15}(b) we know that \(\beta\) is a basis for \(\ps[n]{\R}\), thus there exist \(\seq{c}{0,1,,n}\) such that \(g = \sum_{i = 0}^n c_i f^{(i)}\).
\end{proof}

\begin{ex}\label{ex:1.6.25}
  If \(\V\) and \(\W\) are vector spaces over \(\F\) of dimensions \(m\) and \(n\), determine the dimension of \(\V \times \W\) (see \cref{ex:1.2.21}).
\end{ex}

\begin{proof}[\pf{ex:1.6.25}]
  Let \(\beta_v = \set{\seq{v}{1,2,,m}}, \beta_w = \set{\seq{w}{1,2,n}}\) be a basis for \(\V, \W\), respectively.
  Let \(\zv_v, \zv_w\) be the zero vectors of \(\V, \W\), respectively.
  Then we claim the set
  \[
    \beta = \pa{\beta_v \times \set{\zv_w}} \cup \pa{\set{\zv_v} \times \beta_w}
  \]
  is a basis for \(\V \times \W\).
  Clearly \(\beta \subseteq \V \times \W\).
  Since
  \begin{align*}
             & \forall \seq{a}{1,2,,m + n} \in \F, \sum_{i = 1}^m a_i (v_i, \zv_w) + \sum_{i = 1}^n a_{m + i} (\zv_v, w_i)                                   \\
             & = (\sum_{i = 1}^m a_i v_i, \sum_{i = 1}^n a_{m + i} w_i) = (\zv_v, \zv_w)                                   &  & \text{(by \cref{ex:1.2.21})} \\
    \implies & \begin{dcases}
      \sum_{i = 1}^m a_i v_i = \zv_v \\
      \sum_{i = 1}^n a_{m + i} w_i = \zv_w
    \end{dcases}                                                                                                                    \\
    \implies & \seq[=]{a}{1,2,,m + n} = 0,
  \end{align*}
  by \cref{1.5.3} we know that \(\beta\) is linearly independent.
  Since
  \begin{align*}
             & \forall (v, w) \in \V \times \W, \exists \seq{a}{1,2,,m + n} \in \F :                                       \\
             & (v, w) = (\sum_{i = 1}^m a_i v_i, \sum_{i = 1}^n a_{m + i} w_i)                                             \\
             & = \sum_{i = 1}^m a_i (v_i, \zv_w) + \sum_{i = 1}^n a_{m + i} (\zv_v, w_i) &  & \text{(by \cref{ex:1.2.21})} \\
    \implies & \forall (v, w) \in \V \times \W, (v, w) \in \spn{\beta}                                                     \\
    \implies & \V \times \W = \spn{\beta},                                               &  & \text{(by \cref{1.5})}
  \end{align*}
  by \cref{1.6.1} we know that \(\beta\) is a basis for \(\V \times \W\) and \(\dim(\V \times \W) = m + n\).
\end{proof}

\begin{ex}\label{ex:1.6.26}
  For a fixed \(a \in \R\), determine the dimension of the subspace of \(\ps[n]{\R}\) defined by \(\set{f \in \ps[n]{\R} : f(a) = 0}\).
\end{ex}

\begin{proof}[\pf{ex:1.6.26}]
  Since the set
  \[
    \beta = \set{x - a, x^2 - a^2, \dots, x^n - a^n} \subseteq \set{f \in \ps[n]{\R} : f(a) = 0} \subseteq \ps[n]{\R}
  \]
  is linearly independent (by \cref{ex:1.5.5}) and
  \begin{align*}
             & \forall g \in \set{f \in \ps[n]{\R} : f(a) = 0}, \exists \seq{c}{1,2,,n} \in \R :                               \\
             & \forall x \in \R, g(x) = c_1 (x - a) + c_2 (x^2 - a^2) + \cdots + c_n (x^n - a^n)                               \\
    \implies & \forall g \in \set{f \in \ps[n]{\R} : f(a) = 0}, g \in \spn{\beta}                &  & \text{(by \cref{1.4.3})} \\
    \implies & \set{f \in \ps[n]{\R} : f(a) = 0} = \spn{\beta},                                  &  & \text{(by \cref{1.5})}
  \end{align*}
  by \cref{1.6.1} we know that \(\beta\) is a basis for \(\set{f \in \ps[n]{\R} : f(a) = 0}\).
  Thus by \cref{1.6.8} we have \(\dim(\set{f \in \ps[n]{\R} : f(a) = 0}) = \#(\beta) = n\).
\end{proof}

\begin{ex}\label{ex:1.6.27}
  Let \(\W_1\) and \(\W_2\) be the subspaces of \(\ps{\F}\) defined in \cref{ex:1.3.25}.
  Determine the dimensions of the subspaces \(\W_1 \cap \ps[n]{\F}\) and \(\W_2 \cap \ps[n]{\F}\).
\end{ex}

\begin{proof}[\pf{ex:1.6.27}]
  First suppose that \(n\) is odd.
  We define
  \begin{align*}
    \beta_1 & = \set{1, x^2, x^4 \dots, x^{n - 1}} \\
    \beta_2 & = \set{x, x^3, x^5 \dots, x^n}.
  \end{align*}
  Then we have
  \begin{align*}
             & \begin{dcases}
      \beta_1 \text{ is linearly independent} \\
      \beta_2 \text{ is linearly independent} \\
      \W_1 = \spn{\beta_1}                    \\
      \W_2 = \spn{\beta_2}
    \end{dcases}  &  & \text{(by \cref{ex:1.3.25})} \\
    \implies & \begin{dcases}
      \beta_1 \text{ is a basis for } \W_1 \\
      \beta_2 \text{ is a basis for } \W_2
    \end{dcases}  &  & \text{(by \cref{1.6.1})}     \\
    \implies & \begin{dcases}
      \dim(\W_1) = \#(\beta_1) = \frac{n - 1}{2} \\
      \dim(\W_2) = \#(\beta_2) = \frac{n - 1}{2}
    \end{dcases}. &  & \text{(by \cref{1.6.8})}
  \end{align*}
  Using similar arguments as above we can show that when \(n\) is even we have
  \[
    \begin{dcases}
      \dim(\W_1) = \#(\beta_1) = \frac{n}{2} + 1 \\
      \dim(\W_2) = \#(\beta_2) = \frac{n}{2}
    \end{dcases}.
  \]
\end{proof}

\begin{ex}\label{ex:1.6.28}
  Let \(\V\) be a finite-dimensional vector space over \(\C\) with dimension \(n\).
  Prove that if \(\V\) is now regarded as a vector space over \(\R\), then \(\dim(\V) = 2n\).
\end{ex}

\begin{proof}[\pf{ex:1.6.28}]
  Let \(\beta = \set{\seq{v}{1,2,,n}}\) be a basis for \(\V\) over \(\C\).
  Since
  \begin{align*}
             & \spn{\beta} = \V                                                           &  & \text{(by \cref{1.6.1})}    \\
    \implies & \forall v \in \V, \exists \seq{a}{1,2,,n} \in \C :                                                          \\
             & v = \sum_{j = 1}^n a_j v_j = \sum_{j = 1}^n \pa{\Re(a_j) + i \Im(a_j)} v_j                                  \\
             & = \sum_{j = 1}^n \Re(a_j) (1 v_j) + \sum_{j = 1}^n \Im(a_j) (i v_j)        &  & \text{(by \cref{1.4.3})}    \\
    \implies & \forall v \in \V, \exists \seq{b}{1,2,,2n} \in \R :                                                         \\
             & v = \sum_{j = 1}^n b_j v_j + \sum_{j = 1}^n b_{n + j} (i v_j)              &  & (\Re(a_j), \Im(a_j) \in \R) \\
    \implies & \spn{\set{\seq{v}{1,2,,n}, \seq{iv}{1,2,,n}}} = \V                         &  & \text{(by \cref{1.4.3})}
  \end{align*}
  and
  \begin{align*}
             & \forall \seq{b}{1,2,,2n} \in \R, \sum_{j = 1}^n b_j v_j + \sum_{j = 1}^n b_{n + j} (i v_j) = \zv                                            \\
    \implies & \sum_{j = 1}^n (b_j + i b_{n + j}) v_j = \zv                                                                                                \\
    \implies & b_1 + i b_{n + 1} = b_2 + i b_{n + 2} = \cdots = b_n + i b_{2n} = 0                              &  & \text{(by \cref{1.5.3})}              \\
    \implies & \seq[=]{b}{1,2,,2n} = 0,                                                                         &  & (\set{\seq{b}{1,2,,2n}} \subseteq \R)
  \end{align*}
  by \cref{1.6.1} we know that \(\set{\seq{v}{1,2,,n}, \seq{iv}{1,2,,n}}\) is a basis for \(\V\) over \(\R\) with dimension \(2n\).
\end{proof}

\begin{ex}\label{ex:1.6.29}
  \quad
  \begin{enumerate}
    \item Prove that if \(\W_1\) and \(\W_2\) are finite-dimensional subspaces of a vector space \(\V\) over \(\F\), then the subspace \(\W_1 + \W_2\) is finite-dimensional, and \(\dim(\W_1 + \W_2) = \dim(\W_1) + \dim(\W_2) - \dim(\W_1 \cap \W_2)\).
    \item Let \(\W_1\) and \(\W_2\) be finite-dimensional subspaces of a vector space \(\V\) over \(\F\), and let \(\V = \W_1 + \W_2\).
          Deduce that \(\V\) is the direct sum of \(\W_1\) and \(\W_2\) if and only if \(\dim(\V) = \dim(\W_1) + \dim(\W_2)\).
  \end{enumerate}
\end{ex}

\begin{proof}[\pf{ex:1.6.29}(a)]
  If \(\dim(\W_1) = \dim(\W_2) = 0\), then we have
  \[
    \dim(\W_1 + \W_2) = \dim(\set{\zv}) = 0 = 0 + 0 - 0 = \dim(\W_1) + \dim(\W_2) - \dim(\W_1 \cap \W_2).
  \]
  So suppose that \(\dim(\W_1) > 0\) or \(\dim(\W_2) > 0\).

  By \cref{1.4} we know that \(\W_1 \cap \W_2\) is a subspace of \(\W_1\) and \(\W_2\) over \(\F\).
  Since \(\W_1\) and \(\W_2\) are finite-dimensional, by \cref{1.11} we know that \(\W_1 \cap \W_2\) is finite-dimensional.

  Let \(\beta = \set{\seq{u}{1,2,,k}}\) be a basis for \(\W_1 \cap \W_2\) over \(\F\).
  Note that \(k \geq 0\) and when \(k = 0\) we have \(\beta = \varnothing\).
  By \cref{1.6.15}(c) we can extend \(\beta\) to another set \(\beta_1 = \set{\seq{u}{1,2,,k}, \seq{v}{1,2,,m}}\) which is basis for \(\W_1\) over \(\F\).
  Using similar argument we can extend \(\beta\) to \(\beta_2 = \set{\seq{u}{1,2,,k}, \seq{w}{1,2,,p}}\) which is a basis for \(\W_2\) over \(\F\).
  Note that \(m, p \geq 0\) and \(\max(m, p) \geq 1\).
  Since
  \begin{align*}
             & \begin{dcases}
      \forall v \in \beta_1, v + \zv \in \W_1 + \W_2 \\
      \forall w \in \beta_2, \zv + w \in \W_1 + \W_2
    \end{dcases}                 &  & \text{(by \cref{1.3.10})} \\
    \implies & \beta_1 \cup \beta_2 \subseteq \W_1 + \W_2
  \end{align*}
  and
  \begin{align*}
             & \forall x \in \W_1 + \W_2, \exists (y, z) \in \W_1 \times \W_2 : x = y + z                                                          &  & \text{(by \cref{1.3.10})} \\
    \implies & \forall x \in \W_1 + \W_2, \begin{dcases}
      \exists \seq{a}{1,2,,k + m} \in \F \\
      \exists \seq{b}{1,2,,k + p} \in \F
    \end{dcases} :                                                                                                            \\
             & x = \pa{\sum_{i = 1}^k a_i u_i + \sum_{i = k + 1}^{k + m} a_i v_i} + \pa{\sum_{i = 1}^k b_i u_i + \sum_{i = k + 1}^{k + p} b_i w_i} &  & \text{(by \cref{1.6.1})}  \\
             & = \sum_{i = 1}^k (a_i + b_i) u_i + \sum_{i = k + 1}^{k + m} a_i v_i + \sum_{i = k + 1}^{k + p} b_i w_i                              &  & \text{(by \cref{1.2.1})}  \\
    \implies & \forall x \in \W_1 + \W_2, \exists \seq{a}{1,2,,k + m + p} \in \F :                                                                                                \\
             & x = \sum_{i = 1}^k a_i u_i + \sum_{i = k + 1}^{k + m} a_i v_i + \sum_{i = k + m + 1}^{k + m + p} a_i w_i                                                           \\
    \implies & \forall x \in \W_1 + \W_2, x \in \spn{\beta_1 \cup \beta_2}                                                                         &  & \text{(by \cref{1.4.3})}  \\
    \implies & \W_1 + \W_2 = \spn{\beta_1 \cup \beta_2},                                                                                           &  & \text{(by \cref{1.5})}
  \end{align*}
  by \cref{1.9} we know that \(\dim(\W_1 + \W_2) \leq \#(\beta_1 \cup \beta_2)\) and there exist some \(S \subseteq \beta_1 \cup \beta_2\) such that \(S\) is a basis for \(\W_1 + \W_2\) over \(\F\).
  We now claim that \(\beta_1 \cup \beta_2\) is a basis for \(\W_1 + \W_2\) over \(\F\).
  By \cref{1.6.1} we need to show that \(\beta_1 \cup \beta_2\) is linearly independent.
  Since
  \begin{align*}
             & \beta_1 \cap \beta_2 = \beta                                     \\
    \implies & \begin{dcases}
      \spn{\beta_1 \setminus \beta} \cap \beta_2 = \varnothing \\
      \spn{\beta_2 \setminus \beta} \cap \beta_1 = \varnothing
    \end{dcases}   &  & (\spn{\beta} = \W_1 \cap \W_2) \\
    \implies & \begin{dcases}
      \spn{\beta_1} \cap \beta_2 = \varnothing \\
      \spn{\beta_2} \cap \beta_1 = \varnothing
    \end{dcases},
  \end{align*}
  we know that
  \begin{align*}
             & \forall \seq{a}{1,2,,k + m + p} \in \F,                                                                                                  \\
             & \sum_{i = 1}^k a_i u_i + \sum_{i = k + 1}^{k + m} a_i v_i + \sum_{i = k + m + 1}^{k + m + p} a_i w_i = \zv                               \\
    \implies & \sum_{i = 1}^k a_i u_i + \sum_{i = k + 1}^{k + m} a_i v_i = \sum_{i = k + m + 1}^{k + m + p} -a_i w_i      &  & \text{(by \cref{1.2.1})} \\
    \implies & \seq[=]{a}{1,2,,k + m + p} = 0.                                                                            &  & \text{(by \cref{1.5.3})}
  \end{align*}
  Thus by \cref{1.5.3} \(\beta_1 \cup \beta_2\) is linearly independent and by \cref{1.6.1} \(\beta_1 \cup \beta_2\) is a basis for \(\W_1 + \W_2\).
  Then we have
  \[
    \dim(\W_1 + \W_2) = k + m + p = (k + m) + (k + p) - k = \dim(\W_1) + \dim(\W_2) - \dim(\W_1 \cap \W_2).
  \]
\end{proof}

\begin{proof}[\pf{ex:1.6.29}(b)]
  We have
  \begin{align*}
         & \V = \W_1 \oplus \W_2                                                                        \\
    \iff & \begin{dcases}
      \V = \W_1 + \W_2 \\
      \W_1 \cap \W_2 = \varnothing
    \end{dcases}                             &  & \text{(by \cref{1.3.11})}       \\
    \iff & \begin{dcases}
      \V = \W_1 + \W_2 \\
      \dim(\W_1 \cap \W_2) = 0
    \end{dcases}                             &  & \text{(by \cref{1.6.2})}        \\
    \iff & \dim(\V) = \dim(\W_1 + \W_2) = \dim(\W_1) + \dim(\W_2). &  & \text{(by \cref{ex:1.6.29}(a))}
  \end{align*}
\end{proof}

\setcounter{ex}{30}
\begin{ex}\label{ex:1.6.31}
  Let \(\W_1\) and \(\W_2\) be subspaces of a vector space \(\V\) over \(\F\) having dimensions \(m\) and \(n\), respectively, where \(m \geq n\).
  \begin{enumerate}
    \item Prove that \(\dim(\W_1 \cap \W_2) \leq n\).
    \item Prove that \(\dim(\W_1 + \W_2) \leq m + n\).
  \end{enumerate}
\end{ex}

\begin{proof}[\pf{ex:1.6.31}(a)]
  We have
  \begin{align*}
             & \W_1 \cap \W_2 \subseteq \W_2                                          \\
    \implies & \dim(\W_1 \cap \W_2) \leq \dim(\W_2) = n. &  & \text{(by \cref{1.11})}
  \end{align*}
\end{proof}

\begin{proof}[\pf{ex:1.6.31}(b)]
  We have
  \begin{align*}
    \dim(\W_1 + \W_2) & = \dim(\W_1) + \dim(\W_2) - \dim(\W_1 \cap \W_2) &  & \text{(by \cref{ex:1.6.29}(a))} \\
                      & = m + n - \dim(\W_1 \cap \W_2)                                                        \\
                      & \leq m + n.
  \end{align*}
\end{proof}

\setcounter{ex}{32}
\begin{ex}\label{ex:1.6.33}
  \quad
  \begin{enumerate}
    \item Let \(\W_1\) and \(\W_2\) be subspaces of a vector space \(\V\) over \(\F\) such that \(\V = \W_1 \oplus \W_2\).
          If \(\beta_1\) and \(\beta_2\) are bases for \(\W_1\) and \(\W_2\) over \(\F\), respectively, show that \(\beta_1 \cap \beta_2 = \varnothing\) and \(\beta_1 \cup \beta_2\) is a basis for \(\V\) over \(\F\).
    \item Conversely, let \(\beta_1\) and \(\beta_2\) be disjoint bases for subspaces \(\W_1\) and \(\W_2\), respectively, of a vector space \(\V\) over \(\F\).
          Prove that if \(\beta_1 \cup \beta_2\) is a basis for \(\V\) over \(\F\), then \(\V = \W_1 \oplus \W_2\).
  \end{enumerate}
\end{ex}

\begin{proof}[\pf{ex:1.6.33}(a)]
  First observe that
  \begin{align*}
             & \V = \W_1 \oplus \W_2                                                                                              \\
    \implies & \V = \W_1 + \W_2                                                                    &  & \text{(by \cref{1.3.11})} \\
    \implies & \forall v \in \V, \exists (x, y) \in \W_1 \times \W_2 : v = x + y                                                  \\
    \implies & \forall v \in \V, \exists (x, y) \in \spn{\beta_1} \times \spn{\beta_2} : v = x + y &  & \text{(by \cref{1.6.1})}  \\
    \implies & \forall v \in \V, v \in \spn{\beta_1 \cup \beta_2}                                  &  & \text{(by \cref{1.2.1})}  \\
    \implies & \V = \spn{\beta_1 \cup \beta_2}.                                                    &  & \text{(by \cref{1.5})}
  \end{align*}
  Since
  \begin{align*}
             & V = \W_1 \oplus \W_2                                                                             \\
    \implies & \W_1 \cap \W_2 = \set{\zv}                                  &  & \text{(by \cref{1.3.11})}       \\
    \implies & \beta_1 \cap \beta_2 = \varnothing                                                               \\
    \implies & \dim(\V) = \dim(\W_1 \oplus \W_2) = \dim(\W_1) + \dim(\W_2) &  & \text{(by \cref{ex:1.6.29}(b))} \\
             & = \#(\beta_1) + \#(\beta_2) = \#(\beta_1 \cup \beta_2),
  \end{align*}
  by \cref{1.6.15}(a) we know that \(\beta_1 \cup \beta_2\) is a basis for \(\V\) over \(\F\).
\end{proof}

\begin{proof}[\pf{ex:1.6.33}(b)]
  Since
  \begin{align*}
             & \begin{dcases}
      \beta_1 \cap \beta_2 = \varnothing \\
      \spn{\beta_1 \cup \beta_2} = \V
    \end{dcases}                                                                                           \\
    \implies & \forall v \in \V, v \in \spn{\beta_1 \cup \beta_2}                                                                    \\
    \implies & \forall v \in \V, \exists (x, y) \in \spn{\beta_1} \times \spn{\beta_2} : v = x + y &  & \text{(by \cref{1.2.1})}     \\
    \implies & \forall v \in \V, v \in \W_1 + \W_2                                                 &  & \text{(by \cref{1.3.10})}    \\
    \implies & \V \subseteq \W_1 + \W_2                                                            &  & \text{(by \cref{1.5})}       \\
    \implies & \V = \W_1 + \W_2                                                                    &  & \text{(by \cref{ex:1.3.23})}
  \end{align*}
  and
  \begin{align*}
             & \begin{dcases}
      \beta_1 \cap \beta_2 = \varnothing \\
      \beta_1 \cup \beta_2 \text{ is a basis for } \V \text{ over } \F
    \end{dcases}                                                \\
    \implies & \begin{dcases}
      \spn{\beta_1} \cap \beta_2 = \varnothing \\
      \spn{\beta_2} \cap \beta_1 = \varnothing
    \end{dcases}                  &  & \text{(by \cref{1.7})}   \\
    \implies & \spn{\beta_1} \cap \spn{\beta_2} = \set{\zv} &  & \text{(by \cref{1.5.3})} \\
    \implies & \W_1 \cap \W_2 = \set{\zv},
  \end{align*}
  by \cref{1.3.11} we know that \(\V = \W_1 \oplus \W_2\).
\end{proof}


\chapter{Linear Transformations and Matrices}\label{ch:2}

% All sections are in separated files.  We include them here.
\section{Linear Transformations, Null Spaces and Ranges}\label{sec:2.1}

\begin{defn}\label{2.1.1}
  Let \(\V\) and \(\W\) be vector spaces over \(\F\).
  We call a function \(\T : \V \to \W\) a \textbf{linear transformation from \(\V\) to \(\W\)} if, for all \(x, y \in \V\) and \(c \in \F\), we have
  \begin{enumerate}
    \item \(\T(x + y) = \T(x) + \T(y)\) and
    \item \(\T(cx) = c\T(x)\).
  \end{enumerate}
  We often simply call \(\T\) \textbf{linear}.
\end{defn}

\begin{note}
  If the underlying field \(\F\) is the field of rational numbers, then \cref{2.1.1}(a) implies \cref{2.1.1}(b) (see \cref{ex:2.1.37}), but, in general \cref{2.1.1}(a)(b) are logically independent.
\end{note}

\begin{prop}\label{2.1.2}
  Let \(\V\) and \(\W\) be vector spaces over \(\F\) and let \(\T : \V \to \W\) be a function.
  \begin{enumerate}
    \item If \(\T\) is linear, then \(\T(\zv_{\V}) = \zv_{\W}\).
    \item \(\T\) is linear if and only if \(\T(cx + y) = c\T(x) + \T(y)\) for all \(x, y \in \V\) and \(c \in \F\).
    \item If \(\T\) is linear, then \(\T(x - y) = \T(x) - \T(y)\) for all \(x, y \in \V\).
    \item \(\T\) is linear if and only if, for \(\seq{x}{1,2,,n} \in \V\) and \(\seq{a}{1,2,,n} \in F\), we have
          \[
            \T\pa{\sum_{i = 1}^n a_i x_i} = \sum_{i = 1}^n a_i \T(x_i).
          \]
  \end{enumerate}
\end{prop}

\begin{proof}[\pf{2.1.2}(a)]
  We have
  \begin{align*}
             & \T \text{ is linear}                                                                   \\
    \implies & \T(\zv_{\V}) + \T(\zv_{\V}) = \T(\zv_{\V} + \zv_{\V}) &  & \text{(by \cref{2.1.1}(a))} \\
             & = \T(\zv_{\V}) = \T(\zv_{\V}) + \zv_{\W}              &  & \text{(by \ref{vs3})}       \\
    \implies & \T(\zv_{\V}) = \zv_{\W}.                              &  & \text{(by \cref{1.1})}
  \end{align*}
\end{proof}

\begin{proof}[\pf{2.1.2}(b)]
  We have
  \begin{align*}
             & \T \text{ is linear}                                                                                  \\
    \implies & \begin{dcases}
      \forall x, y \in \V \\
      \forall c \in \F
    \end{dcases}, \begin{dcases}
      \T(x + y) = \T(x) + \T(y) \\
      \T(cx) = c\T(x)
    \end{dcases}                    &  & \text{(by \cref{2.1.1})} \\
    \implies & \begin{dcases}
      \forall x, y \in \V \\
      \forall c \in \F
    \end{dcases}, \T(cx + y) = \T(cx) + \T(y) = c\T(x) + \T(y) &  & \text{(by \cref{2.1.1})}
  \end{align*}
  and
  \begin{align*}
             & \begin{dcases}
      \forall x, y \in \V \\
      \forall c \in \F
    \end{dcases}, \T(cx + y) = c\T(x) + \T(y)                                  \\
    \implies & \begin{dcases}
      \forall x, y \in \V \\
      \forall c \in \F
    \end{dcases}, \begin{dcases}
      \T(x + y) = \T(x) + \T(y)             & \text{if } c = 0        \\
      \T(cx + \zv_{\V}) = c\T(x) + \zv_{\W} & \text{if } y = \zv_{\V}
    \end{dcases}  &  & \text{(by \cref{2.1.2}(a))} \\
    \implies & \begin{dcases}
      \forall x, y \in \V \\
      \forall c \in \F
    \end{dcases}, \begin{dcases}
      \T(x + y) = \T(x) + \T(y) & \text{if } c = 0        \\
      \T(cx) = c\T(x)           & \text{if } y = \zv_{\V}
    \end{dcases} &  & \text{(by \ref{vs3})}       \\
    \implies & \T \text{ is linear}.                                  &  & \text{(by \cref{2.1.1})}
  \end{align*}
  Thus
  \[
    \T \text{ is linear} \iff \begin{dcases}
      \forall x, y \in \V \\
      \forall c \in \F
    \end{dcases}, \T(cx + y) = c\T(x) + \T(y).
  \]
\end{proof}

\begin{proof}[\pf{2.1.2}(c)]
  For all \(x, y \in \V\), we have
  \begin{align*}
    \T(x - y) & = \T(x + (-1)y)      &  & \text{(by \cref{1.2}(b))}   \\
              & = \T(x) + \T((-1)y)  &  & \text{(by \cref{2.1.1}(a))} \\
              & = \T(x) + (-1) \T(y) &  & \text{(by \cref{2.1.1}(b))} \\
              & = \T(x) - \T(y).     &  & \text{(by \cref{1.2}(b))}
  \end{align*}
\end{proof}

\begin{proof}[\pf{2.1.2}(d)]
  We have
  \begin{align*}
             & \T \text{ is linear}                                                                                                  \\
    \implies & \begin{dcases}
      \forall \seq{x}{1,2,,n} \in \V \\
      \forall \seq{a}{1,2,,n} \in \F
    \end{dcases},                                                                                           \\
             & \T\pa{\sum_{i = 1}^n a_i x_i} = \sum_{i = 1}^n \T(a_i x_i) = \sum_{i = 1}^n a_i \T(x_i) &  & \text{(by \cref{2.1.1})}
  \end{align*}
  and
  \begin{align*}
             & \begin{dcases}
      \forall \seq{x}{1,2,,n} \in \V \\
      \forall \seq{a}{1,2,,n} \in \F
    \end{dcases},                                                                 \\
             & \T\pa{\sum_{i = 1}^n a_i x_i} = \sum_{i = 1}^n a_i \T(x_i) &  & \text{(by \cref{2.1.1})}    \\
    \implies & \begin{dcases}
      \forall x, y \in \V \\
      \forall c \in \F
    \end{dcases},                                                                 \\
             & \T(cx + 1y) = c\T(x) + 1\T(y) = c\T(x) + \T(y)             &  & \text{(by \ref{vs5})}       \\
    \implies & \T \text{ is linear}.                                      &  & \text{(by \cref{2.1.2}(b))}
  \end{align*}
  Thus
  \[
    \T \text{ is linear} \iff \begin{dcases}
      \forall \seq{x}{1,2,,n} \in \V \\
      \forall \seq{a}{1,2,,n} \in \F
    \end{dcases}, \T\pa{\sum_{i = 1}^n a_i x_i} = \sum_{i = 1}^n a_i \T(x_i).
  \]
\end{proof}

\begin{note}
  We generally use \cref{2.1.2}(b) to prove that a given transformation is linear.
\end{note}

\begin{eg}\label{2.1.3}
  For any angle \(\theta\), define \(\T_{\theta} : \R^2 \to \R^2\) by the rule: \(\T_{\theta}(a_1, a_2)\) is the vector obtained by rotating \((a_1, a_2)\) counterclockwise by \(\theta\) if \((a_1, a_2) \neq (0, 0)\), and \(\T_{\theta}(0, 0) = (0, 0)\).
  Then \(\T_{\theta} : \R^2 \to \R^2\) is a linear transformation that is called the \textbf{rotation by \(\theta\)}.

  We determine an explicit formula for \(\T_{\theta}\).
  Fix a nonzero vector \((a_1, a_2) \in \R^2\).
  Let \(\alpha\) be the angle that \((a_1, a_2)\) makes with the positive \(x\)-axis, and let \(r = \sqrt{a_1^2 +a_2^2}\).
  Then \(a_1 = r \cos(\alpha)\) and \(a_2 = r \sin(\alpha)\).
  Also, \(\T_{\theta}(a_1, a_2)\) has length \(r\) and makes an angle \(\alpha + \theta\) with the positive \(x\)-axis.
  It follows that
  \begin{align*}
    \T_{\theta}(a_1, a_2) & = (r \cos(\alpha + \theta), r \sin(\alpha + \theta))                                                                     \\
                          & = (r \cos(\alpha) \cos(\theta) - r \sin(\alpha) \sin(\theta), r \cos(\alpha) \sin(\theta) + r \sin(\alpha) \cos(\theta)) \\
                          & = (a_1 \cos(\theta) - a_2 \sin(\theta), a_1 \sin(\theta) + a_2 \cos(\theta)).
  \end{align*}
  Finally, observe that this same formula is valid for \((a_1 ,a_2) = (0, 0)\).
  It is now easy to show that \(\T_{\theta}\) is linear.
\end{eg}

\begin{proof}[\pf{2.1.3}]
  For all \(x, y \in \R^2\) and \(c \in \R\), we have
  \begin{align*}
    \T_{\theta}(cx + y) & = \T_{\theta}(cx_1 + y_1, cx_2 + y_2)                                              &  & \text{(by \cref{1.2.4})} \\
                        & = ((cx_1 + y_1) \cos(\theta) - (cx_2 + y_2)\sin(\theta),                                                         \\
                        & \quad (cx_1 + y_1) \sin(\theta) + (cx_2 + y_2) \cos(\theta))                       &  & \text{(by \cref{2.1.3})} \\
                        & = c (x_1 \cos(\theta) - x_2 \sin(\theta), x_1 \sin(\theta) + x_2 \cos(\theta))     &  & \text{(by \cref{1.2.1})} \\
                        & \quad + (y_1 \cos(\theta) - y_2 \sin(\theta), y_1 \sin(\theta) + y_2 \cos(\theta))                               \\
                        & = c\T(x_1, x_2) + \T(y_1, y_2)                                                     &  & \text{(by \cref{2.1.3})} \\
                        & = c\T_{\theta}(x) + \T_{\theta}(y).                                                &  & \text{(by \cref{1.2.4})}
  \end{align*}
  Thus by \cref{2.1.2}(b) \(\T_{\theta}\) is linear.
\end{proof}

\begin{eg}\label{2.1.4}
  Define \(\T : \R^2 \to \R^2\) by \(\T(a_1, a_2) = (a_1, -a_2)\).
  \(\T\) is called the \textbf{reflection about the \(x\)-axis} and \(\T\) is linear.
\end{eg}

\begin{proof}[\pf{2.1.4}]
  For all \(x, y \in \R^2\) and \(c \in \R\), we have
  \begin{align*}
    \T(cx + y) & = \T(cx_1 + y_1, cx_2 + y_2)   &  & \text{(by \cref{1.2.4})} \\
               & = (cx_1 + y_1, -cx_2 - y_2)    &  & \text{(by \cref{2.1.4})} \\
               & = c(x_1, -x_2) + (y_1, -y_2)   &  & \text{(by \cref{1.2.1})} \\
               & = c\T(x_1, x_2) + \T(y_1, y_2) &  & \text{(by \cref{2.1.4})} \\
               & = c\T(x) + \T(y).              &  & \text{(by \cref{1.2.4})}
  \end{align*}
  Thus by \cref{2.1.2}(b) \(\T\) is linear.
\end{proof}

\begin{eg}\label{2.1.5}
  Define \(\T : \R^2 \to \R^2\) by \(\T(a_1, a_2) = (a_1, 0)\).
  \(\T\) is called the \textbf{projection on the \(x\)-axis} and \(\T\) is linear.
\end{eg}

\begin{proof}[\pf{2.1.5}]
  For all \(x, y \in \R^2\) and \(c \in \R\), we have
  \begin{align*}
    \T(cx + y) & = \T(cx_1 + y_1, cx_2 + y_2)   &  & \text{(by \cref{1.2.4})} \\
               & = (cx_1 + y_1, 0)              &  & \text{(by \cref{2.1.5})} \\
               & = c(x_1, 0) + (y_1, 0)         &  & \text{(by \cref{1.2.1})} \\
               & = c\T(x_1, x_2) + \T(y_1, y_2) &  & \text{(by \cref{2.1.5})} \\
               & = c\T(x) + \T(y).              &  & \text{(by \cref{1.2.4})}
  \end{align*}
  Thus by \cref{2.1.2}(b) \(\T\) is linear.
\end{proof}

\begin{eg}\label{2.1.6}
  Define \(\T : \ms{m}{n}{\F} \to \ms{n}{m}{\F}\) by \(\T(A) = \tp{A}\), where \(\tp{A}\) is the transpose of \(A\), defined in \cref{1.3.3}.
  Then \(\T\) is linear.
\end{eg}

\begin{proof}[\pf{2.1.6}]
  By \cref{ex:1.3.3} and \cref{2.1.2}(b) we see that \(\T\) is linear.
\end{proof}

\begin{eg}\label{2.1.7}
  Define \(\T : \ps{\R} \to \ps{\R}\) by \(\T(f) = f'\), where \(f'\) denotes the derivative of \(f\).
  Then \(\T\) is linear.
\end{eg}

\begin{proof}[\pf{2.1.7}]
  Let \(g, h \in \ps{\R}\) and \(a \in \R\).
  Now
  \[
    \T(ag + h) = (ag + h)' = ag' + h' = a\T(g) + \T(h).
  \]
  So by \cref{2.1.2}(b) \(\T\) is linear.
\end{proof}

\begin{eg}\label{2.1.8}
  Let \(\V = \cfs{\R}\), the vector space of continuous real-valued functions on \(\R\).
  Let \(a, b \in \R\), \(a < b\).
  Define \(\T : \V \to \R\) by
  \[
    \T(f) = \int_a^b f(t) \; dt
  \]
  for all \(f \in \V\).
  Then \(\T\) is linear.
\end{eg}

\begin{proof}[\pf{2.1.8}]
  Let \(g, h \in \cfs{\R}\) and \(a \in \R\).
  Now
  \begin{align*}
    \T(cg + h) & = \int_a^b (cg + h)(t) \; dt                  &  & \text{(by \cref{2.1.8})}  \\
               & = \int_a^b cg(t) + h(t) \; dt                 &  & \text{(by \cref{1.2.10})} \\
               & = c \int_a^b g(t) \; dt + \int_a^b h(t) \; dt                                \\
               & = c \T(g) + \T(h).                            &  & \text{(by \cref{2.1.8})}
  \end{align*}
  So by \cref{2.1.2}(b) \(\T\) is linear.
\end{proof}

\begin{eg}\label{2.1.9}
  For vector spaces \(\V\) and \(\W\) over \(\F\), we define the \textbf{identity transformation} \(\IT[\V] : \V \to \V\) by \(\IT[\V](x) = x\) for all \(x \in \V\) and the \textbf{zero transformation} \(\zT : \V \to \W\) by \(\zT(x) = \zv_{\W}\) for all \(x \in \V\).
  It is clear that both of these transformations are linear.
  We often write \(\IT\) instead of \(\IT[\V]\).
\end{eg}

\begin{proof}[\pf{2.1.9}]
  For all \(x, y \in \V\) and \(c \in \F\), we have
  \begin{align*}
    \IT[\V](cx + y) & = cx + y                    &  & \text{(by \cref{2.1.9})} \\
                    & = c \IT[\V](x) + \IT[\V](y) &  & \text{(by \cref{2.1.9})}
  \end{align*}
  and
  \begin{align*}
    \zT(cx + y) & = \zv_{\W}              &  & \text{(by \cref{2.1.9})}  \\
                & = c \zv_{\W}            &  & \text{(by \cref{1.2}(c))} \\
                & = c \zv_{\W} + \zv_{\W} &  & \text{(by \ref{vs3})}     \\
                & = c\T(x) + \T(y).       &  & \text{(by \cref{2.1.9})}
  \end{align*}
  Thus by \cref{2.1.2}(b) \(\IT[\V], \zT\) are linear.
\end{proof}

\begin{defn}\label{2.1.10}
  Let \(\V\) and \(\W\) be vector spaces over \(\F\), and let \(\T : \V \to \W\) be linear.
  We define the \textbf{null space} (or \textbf{kernel}) \(\ns{\T}\) of \(\T\) to be the set of all vectors \(x\) in \(\V\) such that \(\T(x) = \zv_{\W}\);
  that is, \(\ns{\T} = \set{x \in \V : \T(x) = \zv_{\W}}\).

  We define the \textbf{range} (or \textbf{image}) \(\rg{\T}\) of \(\T\) to be the subset of \(\W\) consisting of all images (under \(\T\)) of vectors in \(\V\);
  that is, \(\rg{\T} = \set{\T(x) : x \in V}\).
\end{defn}

\begin{eg}\label{2.1.11}
  Let \(\V\) and \(\W\) be vector spaces over \(\F\), and let \(\IT : \V \to \V\) and \(\zT : \V \to \W\) be the identity and zero transformations, respectively.
  Then \(\ns{\IT} = \set{\zv_{\V}}\), \(\rg{\IT} = \V\), \(\ns{\zT} = \V\), and \(\rg{\zT} = \set{\zv_{\W}}\).
\end{eg}

\begin{thm}\label{2.1}
  Let \(\V\) and \(\W\) be vector spaces over \(\F\) and \(\T : \V \to \W\) be linear.
  Then \(\ns{\T}\) and \(\rg{\T}\) are subspaces of \(\V\) and \(\W\) over \(\F\), respectively.
\end{thm}

\begin{proof}[\pf{2.1}]
  To clarify the notation, we use the symbols \(\zv_{\V}\) and \(\zv_{\W}\) to denote the zero vectors of \(\V\) and \(\W\), respectively.

  Since \(\T(\zv_{\V}) = \zv_{\W}\), we have that \(\zv_{\V} \in \ns{\T}\).
  Let \(x, y \in \ns{\T}\) and \(c \in \F\).
  Then \(\T(x + y) = \T(x) + \T(y) = \zv_{\W} +\zv_{\W} = \zv_{\W}\), and \(\T(cx) = c \T(x) = c \zv_{\W} = \zv_{\W}\).
  Hence \(x + y \in \ns{\T}\) and \(cx \in \ns{\T}\), so that \(\ns{\T}\) is a subspace of \(\V\) over \(\F\) (see \cref{1.3}).

  Because \(\T(\zv_{\V}) = \zv_{\W}\), we have that \(\zv_{\W} \in \rg{\T}\).
  Now let \(x, y \in \rg{\T}\) and \(c \in \F\).
  Then there exist \(v\) and \(w\) in \(\V\) such that \(\T(v) = x\) and \(\T(w) = y\).
  So \(\T(v + w) = \T(v) + \T(w) = x + y\), and \(\T(cv) = c \T(v) = cx\).
  Thus \(x + y \in \rg{\T}\) and \(cx \in \rg{\T}\), so \(\rg{\T}\) is a subspace of \(\W\) over \(\F\) (see \cref{1.3}).
\end{proof}

\begin{thm}\label{2.2}
  Let \(\V\) and \(\W\) be vector spaces over \(\F\), and let \(\T : \V \to \W\) be linear.
  If \(\beta = \set{\seq{v}{1,2,,n}}\) is a basis for \(\V\) over \(\F\), then
  \[
    \rg{\T} = \spn{\T(\beta)} = \spn{\set{\T(v_1), \T(v_2), \dots, \T(v_n)}}.
  \]
\end{thm}

\begin{proof}[\pf{2.2}]
  Clearly \(\T(v_i) \in \rg{\T}\) for each \(i\).
  Because \(\rg{\T}\) is a subspace, \(\rg{\T}\) contains \(\spn{\set{\T(v_1), \T(v_2), \dots, \T(v_n)}} = \spn{\T(\beta)}\) by \cref{1.5}.

  Now suppose that \(w \in \rg{\T}\).
  Then \(w = \T(v)\) for some \(v \in \V\).
  Because \(\beta\) is a basis for \(\V\) over \(\F\), we have
  \[
    v = \sum_{i = 1}^n a_i v_i \quad \text{for some } \seq{a}{1,2,,n} \in \F.
  \]
  Since \(\T\) is linear, it follows that
  \[
    w = \T(v) = \sum_{i = 1}^n a_i \T(v_i) \in \spn{T(\beta)}.
  \]
  So \(\rg{\T}\) is contained in \(\spn{\T(\beta)}\).
\end{proof}

\begin{note}
  It should be noted that \cref{2.2} is true if \(\beta\) is infinite.
  (See \cref{ex:2.1.33}.)
\end{note}

\begin{defn}\label{2.1.12}
  Let \(\V\) and \(\W\) be vector spaces over \(\F\), and let \(\T : \V \to \W\) be linear.
  If \(\ns{\T}\) and \(\rg{\T}\) are finite-dimensional, then we define the \textbf{nullity} of \(\T\), denoted \(\nt{\T}\), and the \textbf{rank} of \(\T\), denoted \(\rk{\T}\), to be the dimensions of \(\ns{\T}\) and \(\rg{\T}\), respectively.
\end{defn}

\begin{note}
  Reflecting on the action of a linear transformation, we see intuitively that the larger the nullity, the smaller the rank.
  In other words, the more vectors that are carried into \(\zv\), the smaller the range.
  The same heuristic reasoning tells us that the larger the rank, the smaller the nullity.
\end{note}

\begin{thm}[Dimension Theorem]\label{2.3}
  Let \(\V\) and \(\W\) be vector spaces over \(\F\), and let \(\T : \V \to \W\) be linear.
  If \(\V\) is finite-dimensional, then
  \[
    \nt{\T} + \rk{\T} = \dim(\V).
  \]
\end{thm}

\begin{proof}[\pf{2.3}]
  Suppose that \(\dim(\V) = n\), \(\dim(\ns{\T}) = k\), and \(\set{\seq{v}{1,2,,k}}\) is a basis for \(\ns{\T}\) over \(\F\).
  By the \cref{1.6.19} we may extend \(\set{\seq{v}{1,2,,k}}\) to a basis \(\beta = \set{\seq{v}{1,2,,n}}\) for \(\V\) over \(\F\).
  We claim that \(S = \set{\T(v_{k + 1}), \T(v_{k + 2}), \dots, \T(v_n)}\) is a basis for \(\rg{\T}\) over \(\F\).

  First we prove that \(S\) generates \(\rg{\T}\).
  Using \cref{2.2} and the fact that \(\T(v_i) = \zv\) for \(1 \leq i \leq k\), we have
  \begin{align*}
    \rg{\T} & = \spn{\set{\T(v_1), \T(v_2), \dots, \T(v_n)}}             \\
            & = \spn{\set{\T(v_{k + 1}), \T(v_{k + 2}), \dots, \T(v_n)}} \\
            & = \spn{S}.
  \end{align*}
  Now we prove that \(S\) is linearly independent.
  Suppose that
  \[
    \sum_{i = k + 1}^n b_i \T(v_i) = \zv \quad \text{for } \seq{b}{k + 1,k + 2,,n} \in \F.
  \]
  Using the fact that \(\T\) is linear, we have
  \[
    \T\pa{\sum_{i = k + 1}^n b_i v_i} = \zv.
  \]
  So
  \[
    \sum_{i = k + 1}^n b_i v_i \in \ns{\T}.
  \]
  Hence there exist \(\seq{c}{1,2,,k} \in \F\) such that
  \[
    \sum_{i = k + 1}^n b_i v_i = \sum_{i = 1}^k c_i v_i \quad \text{or} \quad \sum_{i = 1}^k (-c_i) v_i + \sum_{i = k + 1}^n b_i v_i = \zv.
  \]
  Since \(\beta\) is a basis for \(\V\) over \(\F\), we have \(b_i = 0\) for all \(i\).
  Hence \(S\) is linearly independent.
  Notice that this argument also shows that \(\T(v_{k + 1}), \T(v_{k + 2}), \dots, \T(v_n)\) are distinct;
  therefore \(\rk{\T} = n - k\).
\end{proof}

\begin{thm}\label{2.4}
  Let \(\V\) and \(\W\) be vector spaces over \(\F\), and let \(\T : \V \to \W\) be linear.
  Then \(\T\) is one-to-one if and only if \(\ns{\T} = \set{\zv_{\V}}\).
\end{thm}

\begin{proof}[\pf{2.4}]
  Suppose that \(\T\) is one-to-one and \(x \in \ns{\T}\).
  Then \(\T(x) = \zv_{\W} = \T(\zv_{\V})\).
  Since \(\T\) is one-to-one, we have \(x = \zv_{\V}\).
  Hence \(\ns{\T} = \set{\zv_{\V}}\).

  Now assume that \(\ns{\T} = \set{\zv_{\V}}\), and suppose that \(\T(x) = \T(y)\).
  Then \(\zv_{\W} = \T(x) - \T(y) = \T(x - y)\) by \cref{2.1.2}(c).
  Therefore \(x - y \in \ns{\T} = \set{\zv_{\V}}\).
  So \(x - y = \zv_{\V}\), or \(x = y\).
  This means that \(\T\) is one-to-one.
\end{proof}

\begin{thm}\label{2.5}
  Let \(\V\) and \(\W\) be vector spaces over \(\F\) of equal (finite) dimension, and let \(\T : \V \to \W\) be linear.
  Then the following are equivalent.
  \begin{enumerate}
    \item \(\T\) is one-to-one.
    \item \(\T\) is onto.
    \item \(\rk{\T} = \dim(\V)\)
  \end{enumerate}
\end{thm}

\begin{proof}[\pf{2.5}]
  From the dimension theorem (\cref{2.3}), we have
  \[
    \nt{\T} + \rk{\T} = \dim(\V).
  \]
  Now, with the use of \cref{2.4}, we have that \(\T\) is one-to-one if and only if \(\ns{\T} = \set{\zv_{\V}}\), if and only if \(\nt{\T} = 0\), if and only if \(\rk{\T} = \dim(\V)\), if and only if \(\rk{\T} = \dim(\W)\), and if and only if \(\dim(\rg{\T}) = \dim(\W)\).
  By \cref{1.11}, this equality is equivalent to \(\rg{\T} = \W\), the definition of \(\T\) being onto.
\end{proof}

\begin{note}
  We note that if \(\V\) is not finite-dimensional and \(\T : \V \to \V\) is linear, then it does \emph{not} follow that one-to-one and onto are equivalent.

  The linearity of \(\T\) in \cref{2.4} and \cref{2.5} is essential, for it is easy to construct examples of functions from \(\R\) into \(\R\) that are not one-to-one, but are onto, and vice versa.
\end{note}

\begin{thm}\label{2.6}
  Let \(\V\) and \(\W\) be vector spaces over \(\F\), and suppose that \(\set{\seq{v}{1,2,,n}}\) is a basis for \(\V\) over \(\F\).
  For \(\seq{w}{1,2,,n}\) in \(\W\), there exists exactly one linear transformation \(\T : \V \to \W\) such that \(\T(v_i) = w_i\) for \(i = 1, 2, \dots, n\).
\end{thm}

\begin{proof}[\pf{2.6}]
  Let \(x \in \V\).
  Then
  \[
    x = \sum_{i = 1}^n a_i v_i
  \]
  where \(\seq{a}{1,2,,n}\) are unique scalars.
  Define
  \[
    \T : \V \to \W \quad \text{by} \quad \T(x) = \sum_{i = 1}^n a_i w_i.
  \]
  \begin{enumerate}
    \item \(\T\) is linear:
          Suppose that \(u, v \in \V\) and \(d \in \F\).
          Then we may write
          \[
            u = \sum_{i = 1}^n b_i v_i \quad \text{and} \quad v = \sum_{i = 1}^n c_i v_i
          \]
          for some scalars \(\seq{b}{1,2,,n}, \seq{c}{1,2,,n}\).
          Thus
          \[
            du + v = \sum_{i = 1}^n (db_i + c_i) v_i.
          \]
          So
          \[
            \T(du + v) = \sum_{i = 1}^n (db_i + c_i) w_i = d \sum_{i = 1}^n b_i w_i + \sum_{i = 1}^n c_i w_i = d \T(u) + \T(v).
          \]
    \item Clearly
          \[
            \T(v_i) = w_i \quad \text{for } i = 1, 2, \dots, n.
          \]
    \item \(\T\) is unique:
          Suppose that \(\U : \V \to \W\) is linear and \(\U(v_i) = w_i\) for \(i = 1, 2, \dots, n\).
          Then for \(x \in \V\) with
          \[
            x = \sum_{i = 1}^n a_i v_i,
          \]
          we have
          \[
            \U(x) = \sum_{i = 1}^n a_i \U(v_i) = \sum_{i = 1}^n a_i w_i = \T(x).
          \]
          Hence \(\U = \T\).
  \end{enumerate}
\end{proof}

\begin{cor}\label{2.1.13}
  Let \(\V\) and \(\W\) be vector spaces over \(\F\), and suppose that \(\V\) has a finite basis \(\set{\seq{v}{1,2,,n}}\) over \(\F\).
  If \(\U, \T : \V \to \W\) are linear and \(\U(v_i) = \T(v_i)\) for \(i = 1, 2, \dots, n\), then \(\U = \T\).
\end{cor}

\begin{proof}[\pf{2.1.13}]
  Since \(\U(v_i) = \T(v_i)\) for all \(i = 1, 2, \dots, n\), by \cref{2.6} we know that \(\U = \T\).
\end{proof}

\exercisesection

\setcounter{ex}{5}
\begin{ex}\label{ex:2.1.6}
  Define \(\T : \ms{n}{n}{\F} \to \F\) by \(\T(A) = \tr[A]\).
  Prove that \(\T\) is a linear transformation, and find bases for both \(\ns{\T}\) and \(\rg{\T}\) over \(\F\).
  Then compute the nullity and rank of \(\T\), and verify the dimension theorem.
  Finally, use the appropriate theorems in \cref{sec:2.1} to determine whether \(\T\) is one-to-one or onto.
\end{ex}

\begin{proof}[\pf{ex:2.1.6}]
  Let \(A, B \in \ms{n}{n}{\F}\) and let \(c \in \F\).
  First we show that \(\T\) is linear.
  Since
  \begin{align*}
    \T(cA + B) & = \tr[cA + B]                                       &  & \text{(by \cref{ex:2.1.6})}  \\
               & = \sum_{i = 1}^n \pa{cA + B}_{i i}                  &  & \text{(by \cref{1.3.9})}     \\
               & = \sum_{i = 1}^n \pa{cA_{i i} + B_{i i}}            &  & \text{(by \cref{1.2.9})}     \\
               & = c \sum_{i = 1}^n A_{i i} + \sum_{i = 1}^n B_{i i} &  & (A_{i i}, B_{i i}, c \in \F) \\
               & = c \tr[A] + \tr[B]                                 &  & \text{(by \cref{1.3.9})}     \\
               & = c \T(A) + \T(B),                                  &  & \text{(by \cref{ex:2.1.6})}
  \end{align*}
  by \cref{2.1.2}(b) we know that \(\T\) is linear.

  Next we find a basis for \(\ns{\T}\) over \(\F\).
  Let \(\W\) and \(\beta\) be the sets defined in \cref{ex:1.6.15}.
  From \cref{ex:1.6.15} we see that \(\W = \ns{\T}\) and \(\beta\) is a basis for \(\ns{\T}\) over \(\F\).
  Thus we have \(\dim(\ns{\T}) = \nt{\T} = n^2 - 1\).

  Next we find a basis for \(\rg{\T}\) over \(\F\).
  Since \(\tr[A] \in \F\) for all \(A \in \ms{n}{n}{\F}\), we know that \(\tr[\ms{n}{n}{\F}] \subseteq \F\).
  Since
  \[
    \forall c \in \F, \tr\begin{pmatrix}
      c      & 0      & \cdots & 0      \\
      0      & 0      & \cdots & 0      \\
      \vdots & \vdots & \ddots & \vdots \\
      0      & 0      & \cdots & 0
    \end{pmatrix} = c,
  \]
  we know that \(\F \subseteq \tr[\ms{n}{n}{\F}]\).
  Thus we have \(\rg{\T} = \T(\ms{n}{n}{\F}) = \tr[\ms{n}{n}{\F}] = \F\), and \(\dim(\rg{\T}) = \rk{\T} = 1\).

  From the proofs above we see that
  \[
    \dim(\ms{n}{n}{\F}) = n^2 = (n^2 - 1) + 1 = \nt{\T} + \rk{\T},
  \]
  thus the dimension theorem (\cref{2.3}) holds.
  Since \(\ns{\T} \neq \set{\zm}\), by \cref{2.4} we know that \(\T\) is not one-to-one.
  Since \(\rg{\T} = \F\), we know that \(\T\) is onto.
\end{proof}

\setcounter{ex}{12}
\begin{ex}\label{ex:2.1.13}
  Let \(\V\) and \(\W\) be vector spaces over \(\F\), let \(\T : \V \to \W\) be linear, and let \(\set{\seq{w}{1,2,,k}}\) be a linearly independent subset of \(\rg{\T}\).
  Prove that if \(S = \set{\seq{v}{1,2,,k}}\) is chosen so that \(\T(v_i) = w_i\) for \(i = 1, 2, \dots, k\), then \(S\) is linearly independent.
\end{ex}

\begin{proof}[\pf{ex:2.1.13}]
  Let \(\seq{a}{1,2,,k} \in \F\).
  Since
  \begin{align*}
             & \sum_{i = 1}^k a_i v_i = \zv_{\V}                                         \\
    \implies & \T\pa{\sum_{i = 1}^k a_i v_i} = \zv_{\W} &  & \text{(by \cref{2.1.2}(a))} \\
    \implies & \sum_{i = 1}^k a_i \T(v_i) = \zv_{\W}    &  & \text{(by \cref{2.1.2}(d))} \\
    \implies & \sum_{i = 1}^k a_i w_i = \zv_{\W}                                         \\
    \implies & \seq[=]{a}{1,2,,k} = 0,                  &  & \text{(by \cref{1.5.3})}
  \end{align*}
  by \cref{1.5.3} we know that \(\set{\seq{v}{1,2,,k}}\) is linearly independent.
\end{proof}

\begin{ex}\label{ex:2.1.14}
  Let \(\V\) and \(\W\) be vector spaces over \(\F\) and \(\T : \V \to \W\) be linear.
  \begin{enumerate}
    \item Prove that \(\T\) is one-to-one if and only if \(\T\) carries linearly independent subsets of \(\V\) onto linearly independent subsets of \(\W\).
    \item Suppose that \(\T\) is one-to-one and that \(S\) is a subset of \(\V\).
          Prove that \(S\) is linearly independent if and only if \(\T(S)\) is linearly independent.
    \item Suppose \(\beta = \set{\seq{v}{1,2,,n}}\) is a basis for \(\V\) over \(\F\) and \(\T\) is one-to-one and onto.
          Prove that \(\T(\beta) = \set{\T(v_1), \T(v_2), \dots, \T(v_n)}\) is a basis for \(\W\) over \(\F\).
  \end{enumerate}
\end{ex}

\begin{proof}[\pf{ex:2.1.14}(a)]
  First suppose that \(\T\) is one-to-one.
  Let \(S\) be a linearly independent subset of \(\V\).
  Note that \(\V\) can be infinite-dimensional and thus \(S\) can be infinite.
  Since
  \begin{align*}
             & \begin{dcases}
      \forall \seq{w}{1,2,,k} \in \T(S) \\
      \forall \seq{a}{1,2,,k} \in \F
    \end{dcases}, \sum_{i = 1}^k a_i w_i = \zv_{\W}                                    \\
    \implies & \exists \seq{v}{1,2,,k} \in S :                                                                  \\
             & \begin{dcases}
      \forall i \in \set{1, 2, \dots, k}, \T(v_i) = w_i \\
      \sum_{i = 1}^k a_i w_i = \sum_{i = 1}^k a_i \T(v_i) = \T\pa{\sum_{i = 1}^k a_i v_i} = \zv_{\W} = \T(\zv_{\V})
    \end{dcases}                                    &  & \text{(by \cref{2.1.2})}      \\
    \implies & \sum_{i = 1}^k a_i v_i = \zv_{\V}                             &  & \text{(\(\T\) is one-to-one)} \\
    \implies & \seq[=]{a}{1,2,,k} = 0,                                       &  & \text{(by \cref{1.5.3})}
  \end{align*}
  by \cref{1.5.3} we know that \(\T(S)\) is linearly independent.
  Since \(S\) is arbitrary, we conclude that \(\T\) carries linearly independent subsets of \(\V\) onto linearly independent subsets of \(\W\).

  Now suppose that \(\T\) carries linearly independent subsets of \(\V\) onto linearly independent subsets of \(\W\).
  Since
  \begin{align*}
             & \forall x, y \in \V, x \neq y                                                    \\
    \implies & x - y \neq \zv_{\V}                                                              \\
    \implies & \set{x - y} \text{ is linearly independent}     &  & \text{(by \cref{1.5.4}(b))} \\
    \implies & \set{\T(x - y)} \text{ is linearly independent} &  & \text{(by hypothesis)}      \\
    \implies & \T(x - y) \neq \zv_{\W}                         &  & \text{(by \cref{1.5.2})}    \\
    \implies & \T(x) - \T(y) \neq \zv_{\W}                     &  & \text{(by \cref{2.1.2}(c))} \\
    \implies & \T(x) \neq \T(y),
  \end{align*}
  we know that \(\T\) is one-to-one.
  From all proofs above we conclude that \(\T\) is one-to-one if and only if \(\T\) carries linearly independent subsets of \(\V\) onto linearly independent subsets of \(\W\).
\end{proof}

\begin{proof}[\pf{ex:2.1.14}(b)]
  By \cref{ex:2.1.14}(a) and \cref{ex:2.1.13} we are done.
\end{proof}

\begin{proof}[\pf{ex:2.1.14}(c)]
  Since \(\T\) is one-to-one, by \cref{ex:2.1.14}(b) we know that \(\T(\beta)\) is linearly independent.
  Since \(\T\) is onto, by \cref{2.2} we know that \(\spn{\T(\beta)} = \rg{\T} = \W\).
  Thus by \cref{1.6.1} \(\T(\beta)\) is a basis for \(\W\) over \(\F\).
\end{proof}

\begin{ex}\label{ex:2.1.15}
  Define
  \[
    \T : \ps{\R} \to \ps{\R} \quad \text{by} \quad \T(f) = \int_0^x f(t) \; dt.
  \]
  Prove that \(\T\) is linear and one-to-one, but not onto.
\end{ex}

\begin{proof}[\pf{ex:2.1.15}]
  First we show that \(\T\) is linear.
  Let \(f, g \in \ps{\R}\) and let \(c \in \R\).
  Since
  \begin{align*}
    \T(cf + g) & = \int_0^x (cf + g)(t) \; dt                  &  & \text{(by \cref{ex:2.1.15})} \\
               & = \int_0^x cf(t) + g(t) \; dt                 &  & \text{(by \cref{1.2.10})}    \\
               & = c \int_0^x f(t) \; dt + \int_0^x g(t) \; dt                                   \\
               & = c \T(f) + \T(g),
  \end{align*}
  by \cref{2.1.2}(b) we know that \(\T\) is linear.

  Next we show that \(\T\) is one-to-one.
  Let \(\zv : \R \to \R\) denote the zero function.
  Since
  \[
    \forall f \in \ps{\R}, \int_0^x f(t) \; dt = \zv \implies f = \zv \implies \ns{\T} = \set{\zv},
  \]
  by \cref{2.4} we know that \(\T\) is one-to-one.

  Now we show that \(\T\) is not onto.
  Let \(c \in \F \setminus \set{0}\).
  Since \(c \in \ps{\R}\) and no polynomial function has indefinite integral equals to \(c\), we know that \(\T\) is not onto.
\end{proof}

\begin{ex}\label{ex:2.1.16}
  Let \(\T : \ps{\R} \to \ps{\R}\) be defined by \(\T(f) = f'\).
  Recall that \(\T\) is linear (\cref{2.1.7}).
  Prove that \(\T\) is onto, but not one-to-one.
\end{ex}

\begin{proof}[\pf{ex:2.1.16}]
  First we show that \(\T\) is onto.
  Let \(f \in \ps{\R}\).
  Since
  \begin{align*}
             & f \in \ps{\R}                                                          \\
    \implies & \exists \seq{a}{0,1,,n} \in \R : f(x) = a_0 + a_1 x + \cdots + a_n x^n \\
    \implies & \begin{dcases}
      x \mapsto a_0 x + \frac{a_1}{2} x^2 + \cdots + \frac{a_n}{n + 1} x^{n + 1} \in \ps{\R} \\
      (x \mapsto a_0 x + \frac{a_1}{2} x^2 + \cdots + \frac{a_n}{n + 1} x^{n + 1})' = f
    \end{dcases}
  \end{align*}
  and \(f\) is arbitrary, we know that \(\T\) is onto.

  Now we show that \(\T\) is not one-to-one.
  Observe that
  \begin{align*}
     & x \mapsto x \in \ps{\R};          \\
     & x \mapsto x + 1 \in \ps{\R};      \\
     & (x \mapsto x)' = 1;               \\
     & (x \mapsto x + 1)' = 1;           \\
     & x \mapsto x \neq x \mapsto x + 1.
  \end{align*}
  Thus \(\T\) is not one-to-one.
\end{proof}

\begin{ex}\label{ex:2.1.17}
  Let \(\V\) and \(\W\) be finite-dimensional vector spaces over \(\F\) and \(\T : \V \to \W\) be linear.
  \begin{enumerate}
    \item Prove that if \(\dim(\V) < \dim(\W)\), then \(\T\) cannot be onto.
    \item Prove that if \(\dim(\V) > \dim(\W)\), then \(\T\) cannot be one-to-one.
  \end{enumerate}
\end{ex}

\begin{proof}[\pf{ex:2.1.17}(a)]
  Suppose that \(\dim(\V) < \dim(\W)\).
  Suppose for sake of contradiction that \(\T\) is onto.
  Let \(\beta_{\W} = \set{\seq{w}{1,2,,n}}\) be a basis for \(\W\) over \(\F\).
  Since \(\T\) is onto, there exists a set \(\beta_{\V} = \set{\seq{v}{1,2,,n}}\) such that \(\T(v_i) = w_i\) for all \(i \in \set{1, 2, \dots, n}\).
  But by \cref{ex:2.1.13} we know that \(\beta_{\V}\) is linearly independent, by \cref{1.6.8} this means \(\dim(\V) \geq \dim(\W)\), a contradiction.
  Thus \(\T\) cannot be onto.
\end{proof}

\begin{proof}[\pf{ex:2.1.17}(b)]
  Suppose that \(\dim(\V) > \dim(\W)\).
  Suppose for sake of contradiction that \(\T\) is one-to-one.
  Since \(\T\) is one-to-one, by \cref{2.4} we know that \(\ns{\T} = \set{\zv_{\V}}\).
  Thus we have
  \begin{align*}
    \dim(\V) & = \rk{\T} + \nt{\T} &  & \text{(by \cref{2.3})}    \\
             & = \rk{\T} + 0       &  & \text{(by \cref{1.6.9})}  \\
             & = \rk{\T}                                          \\
             & = \dim(\rg{\T})     &  & \text{(by \cref{2.1.12})} \\
             & \leq \dim(\W).      &  & \text{(by \cref{1.11})}
  \end{align*}
  But this contradict to the fact that \(\dim(\V) > \dim(\W)\).
  Thus \(\T\) cannot be one-to-one.
\end{proof}

\setcounter{ex}{19}
\begin{ex}\label{ex:2.1.20}
  Let \(\V\) and \(\W\) be vector spaces over \(\F\) with subspaces \(\V_1\) and \(\W_1\) over \(\F\), respectively.
  If \(\T : \V \to \W\) is linear, prove that \(\T(\V_1)\) is a subspace of \(\W\) over \(\F\) and that \(\set{x \in \V : \T(x) \in \W_1}\) is a subspace of \(\V\) over \(\F\).
\end{ex}

\begin{proof}[\pf{ex:2.1.20}]
  First we show that \(\T(\V_1)\) is a subspace of \(\W\) over \(\F\).
  Let \(w_1, w_2 \in \T(\V_1)\) and let \(c \in \F\).
  Since
  \begin{align*}
             & \zv_{\V} \in \V_1                    &  & \text{(by \cref{1.3}(a))}   \\
    \implies & \T(\zv_{\V}) = \zv_{\W} \in \T(\V_1) &  & \text{(by \cref{2.1.2}(a))}
  \end{align*}
  and
  \begin{align*}
             & w_1, w_2 \in \T(\V_1)                                                                    \\
    \implies & \exists v_1, v_2 \in \V_1 : \begin{dcases}
      \T(v_1) = w_1 \\
      \T(v_2) = w_2
    \end{dcases}                                   \\
    \implies & \begin{dcases}
      v_1 + v_2 \in \V_1 \\
      c v_1 \in \V_1
    \end{dcases}                             &  & \text{(by \cref{1.3}(b)(c))} \\
    \implies & \begin{dcases}
      w_1 + w_2 = \T(v_1) + \T(v_2) = \T(v_1 + v_2) \in \T(\V_1) \\
      c w_1 = c \T(v_1) = \T(c v_1) \in \T(\V_1)
    \end{dcases},                            &  & \text{(by \cref{2.1.1})}
  \end{align*}
  by \cref{1.3} we know that \(\T(\V_1)\) is a subspace of \(\W\) over \(\F\).

  Now we show that \(\V' = \set{x \in \V : \T(x) \in \W_1}\) is a subspace of \(\V\) over \(\F\).
  Let \(v_1, v_2 \in \V'\) and let \(c \in \F\).
  Since
  \begin{align*}
             & \zv_{\V} \in \V                  &  & \text{(by \cref{1.3}(a))}   \\
    \implies & \T(\zv_{\V}) = \zv_{\W} \in \W   &  & \text{(by \cref{2.1.2}(a))} \\
    \implies & \T(\zv_{\V}) = \zv_{\W} \in \W_1 &  & \text{(by \cref{1.3}(a))}   \\
    \implies & \zv_{\V} \in \V'
  \end{align*}
  and
  \begin{align*}
             & v_1, v_2 \in \V'                                              \\
    \implies & \T(v_1), \T(v_2) \in \W_1                                     \\
    \implies & \begin{dcases}
      \T(v_1) + \T(v_2) \in \W_1 \\
      c \T(v_1) \in \W_1
    \end{dcases}  &  & \text{(by \cref{1.3}(b)(c))} \\
    \implies & \begin{dcases}
      \T(v_1) + \T(v_2) = \T(v_1 + v_2) \in \W_1 \\
      c \T(v_1) = \T(c v_1) \in \W_1
    \end{dcases}  &  & \text{(by \cref{2.1.1})}     \\
    \implies & \begin{dcases}
      v_1 + v_2 \in \V' \\
      c v_1 \in \V'
    \end{dcases},
  \end{align*}
  by \cref{1.3} we know that \(\V'\) is a subspace of \(\V\) over \(\F\).
\end{proof}

\begin{ex}\label{ex:2.1.21}
  Let \(\V\) be the vector space of sequences over \(\F\) described in \cref{1.2.13}.
  Define the functions \(\T, \U : \V \to \V\) by
  \[
    \T(\seq{a}{1,2,}) = \tuple{a}{2,3,} \quad \text{and} \quad \U(\seq{a}{1,2,}) = (0, \seq{a}{1,2,}).
  \]
  \(\T\) and \(\U\) are called the \textbf{left shift} and \textbf{right shift} operators on \(\V\), respectively.
  \begin{enumerate}
    \item Prove that \(\T\) and \(\U\) are linear.
    \item Prove that \(\T\) is onto, but not one-to-one.
    \item Prove that \(\U\) is one-to-one, but not onto.
  \end{enumerate}
\end{ex}

\begin{proof}[\pf{ex:2.1.21}(a)]
  Let \(\set{a_n}, \set{b_n} \in \V\) and let \(t \in \F\).
  Since
  \begin{align*}
    \T(c \set{a_n} + \set{b_n}) & = \T(\set{c a_n + b_n})                   &  & \text{(by \cref{1.2.13})}    \\
                                & = \T(c a_1 + b_1, c a_2 + b_2, \dots)                                       \\
                                & = (c a_2 + b_2, c a_3 + b_3, \dots)       &  & \text{(by \cref{ex:2.1.21})} \\
                                & = c \tuple{a}{2,3,} + \tuple{b}{2,3,}     &  & \text{(by \cref{1.2.13})}    \\
                                & = c \T(\seq{a}{1,2,}) + \T(\seq{b}{1,2,}) &  & \text{(by \cref{ex:2.1.21})} \\
                                & = c \T(\set{a_n}) + \T(\set{b_n}),
  \end{align*}
  by \cref{2.1.2}(b) we know that \(\T\) is linear.
  Since
  \begin{align*}
    \U(c \set{a_n} + \set{b_n}) & = \U(\set{c a_n + b_n})                     &  & \text{(by \cref{1.2.13})}    \\
                                & = \U(c a_1 + b_1, c a_2 + b_2, \dots)                                         \\
                                & = (0, c a_1 + b_1, c a_2 + b_2, \dots)      &  & \text{(by \cref{ex:2.1.21})} \\
                                & = c (0, \seq{a}{1,2,}) + (0, \seq{b}{1,2,}) &  & \text{(by \cref{1.2.13})}    \\
                                & = c \U(\seq{a}{1,2,}) + \U(\seq{b}{1,2,})   &  & \text{(by \cref{ex:2.1.21})} \\
                                & = c \U(\set{a_n}) + \U(\set{b_n}),
  \end{align*}
  by \cref{2.1.2}(b) we know that \(\U\) is linear.
\end{proof}

\begin{proof}[\pf{ex:2.1.21}(b)]
  Since
  \[
    \forall \set{a_n} \in \V, \T(0, \seq{a}{1,2,}) = \tuple{a}{1,2,},
  \]
  we know that \(\T\) is onto.
  Since
  \[
    \forall \set{a_n} \in \V, \T(0, \seq{a}{1,2,}) = \tuple{a}{1,2,} = \T(1, \seq{a}{1,2,}),
  \]
  we know that \(\T\) is not one-to-one.
\end{proof}

\begin{proof}[\pf{ex:2.1.21}(c)]
  Since
  \begin{align*}
             & \forall \set{a_n}, \set{b_n} \in \V, \set{a_n} \neq \set{b_n} \\
    \implies & \tuple{a}{1,2,} \neq \tuple{b}{1,2,}                          \\
    \implies & (0, \seq{a}{1,2,}) \neq (0, \seq{b}{1,2,})                    \\
    \implies & \U(\seq{a}{1,2,}) \neq \U(\seq{b}{1,2,}),
  \end{align*}
  we know that \(\U\) is one-to-one.
  Since
  \[
    \forall \set{a_n} \in \V, (1, \seq{a}{1,2,}) \notin \T(\V),
  \]
  we know that \(\U\) is not onto.
\end{proof}

\begin{ex}\label{ex:2.1.22}
  Let \(\T: \vs{F}^n \to \F\) be linear.
  Show that
  \[
    \forall x = \tuple{x}{1,2,,n} \in \vs{F}^n, \exists \seq{a}{1,2,,n} \in \F : \T(\seq{x}{1,2,,n}) = \sum_{i = 1}^n a_i x_i.
  \]
  State and prove an analogous result for \(\T : \vs{F}^n \to \vs{F}^m\).
\end{ex}

\begin{proof}[\pf{ex:2.1.22}]
  For all \(i \in \set{1, \dots, n}\), defined \(e_i \in \vs{F}^n\) as in \cref{1.6.3}.
  We claim that if \(\T : \vs{F}^n \to \vs{F}^m\) is linear, then
  \[
    \forall x \in \vs{F}^n, \exists \seq{a}{1,2,,n} \in \vs{F}^m : \T(x) = \sum_{i = 1}^n x_i a_i.
  \]
  Since
  \begin{align*}
             & \vs{F}^n = \spn{\seq{e}{1,2,,n}}                                       &  & \text{(by \cref{1.6.3})}    \\
    \implies & \forall x \in \vs{F}^n, x = \tuple{x}{1,2,,n} = \sum_{i = 1}^n x_i e_i &  & \text{(by \cref{1.4.3})}    \\
    \implies & \forall x \in \vs{F}^n, \T(x) = \sum_{i = 1}^n x_i \T(e_i),            &  & \text{(by \cref{2.1.2}(d))}
  \end{align*}
  by setting \(\T(e_i) = a_i\) for all \(i \in \set{1, 2, \dots, n}\) we see that our claim is true.
  In particular, when \(m = 1\) we see that
  \[
    \forall x \in \vs{F}^n, \exists \seq{a}{1,2,,n} \in \F : \T(x) = \sum_{i = 1}^n x_i a_i = \sum_{i = 1}^n a_i x_i.
  \]
\end{proof}

\begin{defn}\label{2.1.14}
  Let \(\V\) be a vector space over \(\F\) and \(\W_1\) and \(\W_2\) be subspaces of \(\V\) over \(\F\) such that \(V = \W_1 \oplus \W_2\).
  (See \cref{1.3.11}.)
  A function \(\T : \V \to \V\) is called a \textbf{projection on \(\W_1\) along \(\W_2\)} if, for \(x = x_1 + x_2\) with \(x_1 \in \W_1\) and \(x_2 \in \W_2\), we have \(\T(x) = x_1\).
\end{defn}

\setcounter{ex}{25}
\begin{ex}\label{ex:2.1.26}
  Using the notation in \cref{2.1.14}, assume that \(\T : \V \to \V\) is the projection on \(\W_1\) along \(\W_2\).
  \begin{enumerate}
    \item Prove that \(\T\) is linear and \(\W_1 = \set{x \in \V : \T(x) = x}\).
    \item Prove that \(\W_1 = \rg{\T}\) and \(\W_2 = \ns{\T}\).
    \item Describe \(\T\) if \(\W_1 = \V\).
    \item Describe \(\T\) if \(\W_1\) is the zero subspace.
  \end{enumerate}
\end{ex}

\begin{proof}[\pf{ex:2.1.26}(a)]
  First we show that \(\T\) is linear.
  Let \(x, y \in \V\) and let \(c \in \F\).
  Since
  \begin{align*}
             & \V = \W_1 \oplus \W_2                                                                                           \\
    \implies & \exists (x_1, x_2), (y_1, y_2) \in \W_1 \times \W_2 : \begin{dcases}
      x = x_1 + x_2 \\
      y = y_1 + y_2
    \end{dcases} &  & \text{(by \cref{1.3.11})} \\
    \implies & \T(cx + y) = \T(c (x_1 + x_2) + y_1 + y_2)                                                                      \\
             & = \T(c x_1 + y_1 + c x_2 + y_2) = c x_1 + y_1 = c \T(x) + \T(y),                 &  & \text{(by \cref{2.1.14})}
  \end{align*}
  by \cref{2.1.2}(b) we know that \(\T\) is linear.

  Now we show that \(\W_1 = \set{x \in \V : \T(x) = x}\).
  Since
  \begin{align*}
             & \zv \in \W_2                                &  & \text{(by \cref{1.3}(a))} \\
    \implies & \forall x \in \W_1, x = x + \zv             &  & \text{(by \ref{vs3})}     \\
    \implies & \forall x \in \W_1, \T(x) = \T(x + \zv) = x &  & \text{(by \cref{2.1.14})} \\
    \implies & \W_1 \subseteq \set{x \in \V : \T(x) = x}
  \end{align*}
  and
  \begin{align*}
             & \forall x \in \V, \T(x) = x                                               \\
    \implies & x = \T(x) \in \W_1                         &  & \text{(by \cref{2.1.14})} \\
    \implies & \set{x \in \V : \T(x) = x} \subseteq \W_1,
  \end{align*}
  we know that \(\W_1 = \set{x \in \V : \T(x) = x}\).
\end{proof}

\begin{proof}[\pf{ex:2.1.26}(b)]
  First we show that \(\W_1 = \rg{\T}\).
  Since
  \begin{align*}
             & \W_1 = \set{x \in \V : \T(x) = x} &  & \text{(by \cref{ex:2.1.26}(a))} \\
    \implies & \W_1 \subseteq \rg{\T}            &  & \text{(by \cref{2.1.10})}
  \end{align*}
  and
  \begin{align*}
             & \forall x \in \V, \T(x) \in \W_1 &  & \text{(by \cref{2.1.14})} \\
    \implies & \rg{\T} \subseteq \W_1,
  \end{align*}
  we know that \(\W_1 = \rg{\T}\).

  Now we show that \(\W_2 = \ns{\T}\).
  Since
  \begin{align*}
             & \zv \in \W_1                    &  & \text{(by \cref{1.3}(a))} \\
    \implies & \forall x \in \W_2, x = \zv + x &  & \text{(by \ref{vs3})}     \\
    \implies & \forall x \in \W_2, \T(x) = \zv &  & \text{(by \cref{2.1.14})} \\
    \implies & \W_2 \subseteq \ns{\T}          &  & \text{(by \cref{2.1.10})}
  \end{align*}
  and
  \begin{align*}
             & \ns{\T} \subseteq \V                                                           &  & \text{(by \cref{2.1.10})} \\
    \implies & \forall x \in \ns{\T}, \exists (x_1, x_2) \in \W_1 \times \W_2 : x = x_1 + x_2 &  & \text{(by \cref{1.3.11})} \\
    \implies & \forall x \in \ns{\T}, \exists (x_1, x_2) \in \W_1 \times \W_2 :                                              \\
             & \T(x) = \T(x_1 + x_2) = \zv = x_1                                              &  & \text{(by \cref{2.1.14})} \\
    \implies & \forall x \in \ns{\T}, \exists (x_1, x_2) \in \W_1 \times \W_2 :                                              \\
             & x = x_1 + x_2 = \zv + x_2 = x_2 \in \W_2                                       &  & \text{(by \ref{vs3})}     \\
    \implies & \ns{\T} \subseteq \W_2,
  \end{align*}
  we know that \(\W_2 = \ns{\T}\).
\end{proof}

\begin{proof}[\pf{ex:2.1.26}(c)]
  We have
  \begin{align*}
             & \begin{dcases}
      \V = \W_1 \oplus \W_2 \\
      \V = \W_1
    \end{dcases}                                                            \\
    \implies & \W_1 \cap \W_2 = \V \cap \W_2 = \W_2 = \set{\zv} &  & \text{(by \cref{1.3.11})}       \\
    \implies & \ns{\T} = \W_2 = \set{\zv}                       &  & \text{(by \cref{ex:2.1.26}(b))} \\
    \implies & \T \text{ is one-to-one}                         &  & \text{(by \cref{2.4})}          \\
    \implies & \T \text{ is onto}.                              &  & \text{(by \cref{2.5})}
  \end{align*}
\end{proof}

\begin{proof}[\pf{ex:2.1.26}(d)]
  We have
  \begin{align*}
             & \W_1 = \set{\zv}                                                            \\
    \implies & \V = \W_1 \oplus \W_2 = \W_2           &  & \text{(by \cref{1.3.11})}       \\
             & = \ns{\T}                              &  & \text{(by \cref{ex:2.1.26}(b))} \\
    \implies & \forall x \in \V, \T(x) = \zv          &  & \text{(by \cref{2.1.10})}       \\
    \implies & \T \text{ is the zero transformation}. &  & \text{(by \cref{2.1.9})}
  \end{align*}
\end{proof}

\begin{ex}\label{ex:2.1.27}
  Suppose that \(\W\) is a subspace of a finite-dimensional vector space \(\V\) over \(\F\).
  \begin{enumerate}
    \item Prove that there exists a subspace \(\W'\) and a function \(\T : \V \to \V\) such that \(\T\) is a projection on \(\W\) along \(\W'\).
    \item Give an example of a subspace \(\W\) of a vector space \(\V\) over \(\F\) such that there are two projections on \(\W\) along two (distinct) subspaces.
  \end{enumerate}
\end{ex}

\begin{proof}[\pf{ex:2.1.27}(a)]
  Let \(\beta_{\W} = \set{\seq{v}{1,2,,k}}\) be a basis for \(\V\) over \(\F\).
  By \cref{1.6.19} we can extend \(\beta_{\W}\) to \(\beta = \set{\seq{v}{1,2,,n}}\) such that \(\beta\) is a basis for \(\V\) over \(\F\).
  Let \(\beta_{\W'} = \beta \setminus \beta_{\W}\) and let \(\W' = \spn{\beta_{\W'}}\).
  By \cref{1.5} we know that \(\W'\) is a subspace of \(\V\) over \(\F\).

  First we claim that \(\W \cap \W' = \set{\zv}\).
  Let \(v \in \W \cap \W'\).
  Then we have
  \begin{align*}
             & v \in \W \cap \W'                                                                                                                \\
    \implies & \exists \seq{a}{1,2,,k,k + 1,,n} \in \F : v = \sum_{i = 1}^k a_i v_i = \sum_{i = k + 1}^n a_i v_i &  & \text{(by \cref{1.4.3})}  \\
    \implies & \exists \seq{a}{1,2,,k,k + 1,,n} \in \F :                                                                                        \\
             & \sum_{i = 1}^k a_i v_i + \sum_{i = k + 1}^n (-a_i) v_i = \zv                                      &  & \text{(by \cref{1.2.1})}  \\
    \implies & \seq[=]{a}{1,2,,n} = 0                                                                            &  & \text{(by \cref{1.5.3})}  \\
    \implies & v = \zv                                                                                           &  & \text{(by \cref{1.2}(a))} \\
    \implies & \W \cap \W' \subseteq \set{\zv}                                                                                                  \\
    \implies & \W \cap \W' = \set{\zv}.                                                                          &  & \text{(by \cref{1.3}(a))}
  \end{align*}

  Next we claim that \(\V = \W \oplus \W'\).
  Since
  \begin{align*}
             & \forall v \in \V, \exists \seq{a}{1,2,,n} \in \F : v = \sum_{i = 1}^n a_i v_i &  & \text{(by \cref{1.6.1})}        \\
             & = \sum_{i = 1}^k a_i v_i + \sum_{i = k + 1}^n a_i v_i                         &  & \text{(by \cref{1.2.1})}        \\
    \implies & \forall v \in \V, \exists (u, u') \in \W \times \W' : v = u + u'              &  & \text{(by \cref{1.4.3})}        \\
    \implies & \V \subseteq \W + \W'                                                         &  & \text{(by \cref{1.3.10})}       \\
    \implies & \V = \W + \W'                                                                 &  & \text{(by \cref{ex:1.3.23}(a))}
  \end{align*}
  and \(\W \cap \W' = \set{\zv}\), by \cref{1.3.11} we know that \(\V = \W \oplus \W'\).

  Since \(\beta\) is a basis for \(\V\) over \(\F\), by \cref{1.8} we know that
  \[
    \forall v \in \V, \exists \seq{a}{1,2,,n} \in \F : v = \sum_{i = 1}^n a_i v_i.
  \]
  Now we define a function \(\T\) as follow:
  \[
    \forall v \in \V, \T(v) = \sum_{i = 1}^k a_i v_i.
  \]
  We claim that \(\T\) is a projection on \(\W\) along \(\W'\).
  Since
  \begin{align*}
             & \forall v \in \V, \T(v) = \T\pa{\sum_{i = 1}^n a_i v_i}                                                             \\
             & = \T\pa{\sum_{i = 1}^k a_i v_i + \sum_{i = k + 1}^n a_i v_i} = \sum_{i = 1}^k a_i v_i                               \\
    \implies & \forall v \in \V, \exists (u, u') \in \W \times \W' : \T(v) = \T(u + u') = u,         &  & \text{(by \cref{1.4.3})}
  \end{align*}
  by \cref{2.1.14} we know that \(\T\) is a projection on \(\W\) along \(\W'\).
\end{proof}

\begin{proof}[\pf{ex:2.1.27}(b)]
  See Exercise 2.1.24.
\end{proof}

\begin{defn}\label{2.1.15}
  Let \(\V\) be a vector space over \(\F\), and let \(\T : \V \to \V\) be linear.
  A subspace \(\W\) of \(\V\) over \(\F\) is said to be \textbf{\(\T\)-invariant} if \(\T(x) \in \W\) for every \(x \in \W\), that is, \(\T(\W) \subseteq \W\).
  If \(\W\) is \(\T\)-invariant, we define the \textbf{restriction of \(\T\) on \(\W\)} to be the function \(\T_{\W} : \W \to \W\) defined by \(\T_{\W}(x) = \T(x)\) for all \(x \in \W\).
\end{defn}

\cref{ex:2.1.28,ex:2.1.29,ex:2.1.30,ex:2.1.31,ex:2.1.32} assume that \(\W\) is a subspace of a vector space \(\V\) over \(\F\) and that \(\T : \V \to \V\) is linear.

\begin{ex}\label{ex:2.1.28}
  Prove that the subspaces \(\set{\zv}\), \(\V\), \(\rg{\T}\), and \(\ns{\T}\) are all \(\T\)-invariant.
\end{ex}

\begin{proof}[\pf{ex:2.1.28}]
  First we show that \(\set{\zv}\) is \(\T\)-invariant.
  By \cref{2.1.2}(a) we know that \(\T(\set{\zv}) = \set{\zv} \subseteq \set{\zv}\), thus by \cref{2.1.15} \(\set{\zv}\) is \(\T\)-invariant.

  Next we show that \(\V\) is \(\T\)-invariant.
  Obviously \(\T(\V) \subseteq \T(\V)\), thus by \cref{2.1.15} \(\V\) is \(\T\)-invariant.

  Next we show that \(\rg{\T}\) is \(\T\)-invariant.
  Since
  \begin{align*}
             & \forall y \in \rg{\T}, y \in \V          &  & (\T : \V \to \V)          \\
    \implies & \forall y \in \rg{\T}, \T(y) \in \rg{\T} &  & \text{(by \cref{2.1.10})} \\
    \implies & \T(\rg{\T}) \subseteq \rg{\T},
  \end{align*}
  by \cref{2.1.15} we know that \(\rg{\T}\) is \(\T\)-invariant.

  Finally we show that \(\ns{\T}\) is \(\T\)-invariant.
  Since
  \begin{align*}
             & \forall x \in \ns{\T}, \T(x) = \zv                 &  & \text{(by \cref{2.1.10})} \\
    \implies & \T(\ns{\T}) \subseteq \set{\zv} \subseteq \ns{\T}, &  & \text{(by \cref{1.3}(a))}
  \end{align*}
  by \cref{2.1.15} we know that \(\ns{\T}\) is \(\T\)-invariant.
\end{proof}

\begin{ex}\label{ex:2.1.29}
  If \(\W\) is \(\T\)-invariant, prove that \(\T_{\W}\) is linear.
\end{ex}

\begin{proof}[\pf{ex:2.1.29}]
  Let \(x, y \in \W\) and let \(c \in \F\).
  Since
  \begin{align*}
             & cx + y \in \W                &  & \text{(by \cref{1.3}(b)(c))} \\
    \implies & \T_{\W}(cx + y) = \T(cx + y) &  & \text{(by \cref{2.1.15})}    \\
             & = c \T(x) + \T(y)            &  & \text{(by \cref{2.1.2}(b))}  \\
             & = c \T_{\W}(x) + \T_{\W}(y), &  & \text{(by \cref{2.1.15})}
  \end{align*}
  by \cref{2.1.2}(b) we know that \(\T_{\W}\) is linear.
\end{proof}

\begin{ex}\label{ex:2.1.30}
  Suppose that \(\T\) is a projection on \(\W\) along some subspace \(\W'\).
  Prove that \(\W\) is \(\T\)-invariant and that \(\T_{\W} = \IT[\W]\).
\end{ex}

\begin{proof}[\pf{ex:2.1.30}]
  We have
  \begin{align*}
             & \T \text{ is a projection on } \W \text{ along } \W'                                                                 \\
    \implies & \begin{dcases}
      \V = \W \oplus \W' \\
      \forall x \in \V, \exists (x_1, x_2) \in \W \times \W' : \T(x) = \T(x_1 + x_2) = x_1
    \end{dcases}                                                           &  & \text{(by \cref{2.1.14})} \\
    \implies & \forall x \in \W, \exists (x, \zv) \in \W \times \W' : \T(x) = \T(x + \zv) = x \in \W &  & \text{(by \cref{1.3}(a))} \\
    \implies & \T(\W) \subseteq \W                                                                                                  \\
    \implies & \W \text{ is } \T\text{-invariant}                                                    &  & \text{(by \cref{2.1.15})}
  \end{align*}
  and
  \begin{align*}
             & \forall x \in \W, \T(x) = x                                     \\
    \implies & \forall x \in \W, \T_{\W}(x) = x &  & \text{(by \cref{2.1.15})} \\
    \implies & \T_{\W} = \IT[\W].               &  & \text{(by \cref{2.1.9})}
  \end{align*}
\end{proof}

\begin{ex}\label{ex:2.1.31}
  Suppose that \(\V = \rg{\T} \oplus \W\) and \(\W\) is \(\T\)-invariant.
  (See \cref{1.3.11}.)
  \begin{enumerate}
    \item Prove that \(\W \subseteq \ns{\T}\).
    \item Show that if \(\V\) is finite-dimensional, then \(\W = \ns{\T}\).
    \item Show by example that the conclusion of (b) is not necessarily true if \(\V\) is not finite-dimensional.
  \end{enumerate}
\end{ex}

\begin{proof}[\pf{ex:2.1.31}(a)]
  We have
  \begin{align*}
             & \begin{dcases}
      \V = \rg{\T} \oplus \W \\
      \T(\W) \subseteq \W
    \end{dcases}                               &  & \text{(by \cref{2.1.15})} \\
    \implies & \rg{\T} \cap \T(\W) \subseteq \rg{\T} \cap \W = \set{\zv} &  & \text{(by \cref{1.3.11})} \\
    \implies & \T(\W) = \rg{\T} \cap \T(\W) \subseteq \set{\zv}          &  & \text{(by \cref{2.1.10})} \\
    \implies & \forall x \in \W, \T(x) = \zv                                                            \\
    \implies & \W \subseteq \ns{\T}.                                     &  & \text{(by \cref{2.1.10})}
  \end{align*}
\end{proof}

\begin{proof}[\pf{ex:2.1.31}(b)]
  Since \(\V = \rg{\T} \oplus \W\), by \cref{ex:1.6.29}(b) we know that \(\dim(\V) = \rk{\T} + \dim(\W)\).
  By dimension theorem (\cref{2.3}) we know that \(\ns{\T} = \dim(\V) - \rk{\T} = \dim(\W)\).
  By \cref{ex:2.1.31}(a) we know that \(\W \subseteq \ns{\T}\), thus by \cref{1.11} we have \(\W = \ns{\T}\).
\end{proof}

\begin{proof}[\pf{ex:2.1.31}(c)]
  Let \(\V\) and \(\T\) defined as in \cref{ex:2.1.21}.
  Let \(\W = \set{(0, 0, \dots)}\).
  By \cref{ex:2.1.21}(b) we know that \(\V = \rg{\T} = \rg{\T} \oplus \W\).
  But then we have
  \begin{align*}
             & (1, 0, 0, \dots) \in \V                                                     \\
    \implies & \T(1, 0, 0, \dots) = (0, 0, \dots)        &  & \text{(by \cref{ex:2.1.21})} \\
    \implies & (1, 0, 0, \dots) \in \ns{\T} \setminus \W                                   \\
    \implies & \ns{\T} \neq \W.
  \end{align*}
\end{proof}

\begin{ex}\label{ex:2.1.32}
  Suppose that \(\W\) is \(\T\)-invariant.
  Prove that \(\ns{\T_{\W}} = \ns{\T} \cap \W\) and \(\rg{\T_{\W}} = \T(\W)\).
\end{ex}

\begin{proof}[\pf{ex:2.1.32}]
  First we show that \(\ns{\T_{\W}} = \ns{\T} \cap \W\).
  Since
  \begin{align*}
         & x \in \ns{\T_{\W}}                                         \\
    \iff & \begin{dcases}
      x \in \W \\
      \T_{\W}(x) = \zv
    \end{dcases} &  & \text{(by \cref{2.1.10})} \\
    \iff & \begin{dcases}
      x \in \W \\
      \T(x) = \zv
    \end{dcases} &  & \text{(by \cref{2.1.15})} \\
    \iff & \begin{dcases}
      x \in \W \\
      x \in \ns{\T}
    \end{dcases} &  & \text{(by \cref{2.1.15})} \\
    \iff & x \in \ns{\T} \cap \W,
  \end{align*}
  we know that \(\ns{\T_{\W}} = \ns{\T} \cap \W\).

  Now we show that \(\rg{\T_{\W}} = \T(\W)\).
  This is true since
  \begin{align*}
    \T(\W) & = \T_{\W}(\W)   &  & \text{(by \cref{2.1.15})} \\
           & = \rg{\T_{\W}}. &  & \text{(by \cref{2.1.10})}
  \end{align*}
\end{proof}

\begin{ex}\label{ex:2.1.33}
  Prove \cref{2.2} for the case that \(\beta\) is infinite, that is, \(\rg{\T} = \spn{\set{\T(v) : v \in \beta}}\).
\end{ex}

\begin{proof}[\pf{ex:2.1.33}]
  Clearly we have \(\set{\T(v) : v \in \beta} \subseteq \rg{\T}\).
  By \cref{2.1} we know that \(\rg{\T}\) is a vector space over \(\F\), thus by \cref{1.5} we have
  \[
    \spn{\set{\T(v) : v \in \beta}} \subseteq \rg{\T}.
  \]
  Let \(x \in \V\).
  Since \(\beta\) is a basis for \(\V\) over \(\F\), by \cref{1.6.1} there exists some \(\seq{v}{1,2,,n} \in \beta\) such that \(x \in \spn{\set{\seq{v}{1,2,,n}}}\).
  In particular, there exist some \(\seq{a}{1,2,,n} \in \F\) such that \(x = \sum_{i = 1}^n a_i v_i\).
  Since \(\T\) is linear, by \cref{2.1.2}(d) we know that
  \[
    \T(x) = \T\pa{\sum_{i = 1}^n a_i v_i} = \sum_{i = 1}^n a_i \T(v_i) \in \spn{\set{\T(v) : v \in \beta}}.
  \]
  This means \(\rg{\T} \subseteq \spn{\set{\T(v) : v \in \beta}}\), thus we have \(\rg{\T} = \spn{\set{\T(v) : v \in \beta}}\).
\end{proof}

\begin{ex}\label{ex:2.1.34}
  Prove the following generalization of \cref{2.6}:
  Let \(\V\) and \(\W\) be vector spaces over a common field \(\F\), and let \(\beta\) be a basis for \(\V\) over \(\F\).
  Then for any function \(f : \beta \to \W\) there exists exactly one linear transformation \(\T : \V \to \W\) such that \(\T(x) = f(x)\) for all \(x \in \beta\).
\end{ex}

\begin{proof}[\pf{ex:2.1.34}]
  Note that \(\V\) can be infinite-dimensional.
  Let \(x \in \V\).
  Since \(\beta\) is a basis for \(\V\) over \(\F\), there exists some \(v_1^x, v_2^x, \dots, v_{n^x}^x \in \beta\) and \(a_1^x, a_2^x, \dots, a_{n^x}^x \in \F\) such that \(x = \sum_{i = 1}^{n^x} a_i x_i\).
  Here we use \(v_i^x, a_i^x\) and \(n^x\) to emphasis the choice of \(v_i^x, a_i^x, n^x\) depends on \(x\).
  Now we define \(\T : \V \to \W\) as follow:
  \[
    \forall x \in \V, \T(x) = \sum_{i = 1}^n a_i^x f(v_i^x).
  \]
  \begin{itemize}
    \item \(\T\) is linear:
          Suppose that \(x, y \in \V\) and \(c \in \F\).
          Then we may write
          \[
            x = \sum_{i = 1}^{n^x} a_i^x v_i^x \quad \text{and} \quad y = \sum_{i = 1}^{n^y} a_i^y v_i^y.
          \]
          Since \(\set{v_1^x, \dots, v_{n^x}^x}\) and \(\set{v_1^y, \dots, v_{n^y}^y}\) are finite, we know that \(\gamma = \set{v_1^x, \dots, v_{n^x}^x} \cup \set{v_1^y, \dots, v_{n^y}^y}\) is also finite.
          Let \(n = \#(\gamma)\).
          By relabeling vectors in \(\gamma\) as \(v_1, \dots v_n\) we have
          \[
            x = \sum_{i = 1}^n a_i^x v_i \quad \text{and} \quad y = \sum_{i = 1}^n a_i^y v_i
          \]
          where \(a_i^x = 0\) if \(v_i \notin \set{v_1^x, \dots, v_{n^x}^x}\) and \(a_i^y = 0\) if \(v_i \notin \set{v_1^y, \dots, v_{n^y}^y}\).
          Thus
          \[
            cx + y = \sum_{i = 1}^n c a_i^x v_i + \sum_{i = 1}^n a_i^y v_i = \sum_{i = 1}^n (c a_i^x + a_i^y) v_i.
          \]
          So
          \[
            \T(cx + y) = \sum_{i = 1}^n (c a_i^x + a_i^y) f(v_i) = c \sum_{i = 1}^n a_i^x f(v_i) + \sum_{i = 1}^n a_i^y f(v_i) = c \T(x) + \T(y).
          \]
    \item Clearly
          \[
            \forall v \in \beta, \T(v) = f(v).
          \]
    \item \(\T\) is unique:
          Suppose that \(\U : \V \to \W\) is linear and \(\U(v) = f(v)\) for all \(v \in \beta\).
          Then for \(x \in \V\) with
          \[
            x = \sum_{i = 1}^n a_i^x v_i^x,
          \]
          we have
          \[
            \U(x) = \sum_{i = 1}^n a_i^x \U(v_i^x) = \sum_{i = 1}^n a_i^x f(v_i^x) = \T(x).
          \]
          Hence \(\U = \T\).
  \end{itemize}
\end{proof}

\begin{ex}\label{ex:2.1.35}
  Let \(\V\) be a finite-dimensional vector space over \(\F\) and \(\T : \V \to \V\) be linear.
  \begin{enumerate}
    \item Suppose that \(\V = \rg{\T} + \ns{\T}\).
          Prove that \(\V = \rg{\T} \oplus \ns{\T}\).
    \item Suppose that \(\rg{\T} \cap \ns{\T} = \set{\zv}\).
          Prove that \(\V = \rg{\T} \oplus \ns{\T}\).
  \end{enumerate}
\end{ex}

\begin{proof}[\pf{ex:2.1.35}(a)]
  We have
  \begin{align*}
             & \begin{dcases}
      \V = \rg{\T} + \ns{\T} \\
      \dim(\V) = \rk{\T} + \nt{\T}
    \end{dcases}  &  & \text{(by \cref{2.3})}          \\
    \implies & \V = \rg{\T} \oplus \ns{\T}. &  & \text{(by \cref{ex:1.6.29}(b))}
  \end{align*}
\end{proof}

\begin{proof}[\pf{ex:2.1.35}(b)]
  By \cref{ex:1.3.23}(a) we know that \(\rg{\T} + \ns{\T} \subseteq \V\).
  Since
  \begin{align*}
    \dim(\rg{\T} + \ns{\T}) & = \rk{\T} + \ns{\T} - \dim(\rg{\T} \cap \ns{\T}) &  & \text{(by \cref{ex:1.6.29}(a))} \\
                            & = \rk{\T} + \ns{\T} - 0                          &  & \text{(by hypothesis)}          \\
                            & = \dim(\V),                                      &  & \text{(by \cref{2.3})}
  \end{align*}
  by \cref{1.11} we know that \(\V = \rg{\T} + \ns{\T}\).
  Thus by \cref{ex:2.1.35}(a) we know that \(\V = \rg{\T} \oplus \ns{\T}\).
\end{proof}

\begin{ex}\label{ex:2.1.36}
  Let \(\V\) and \(\T\) be as defined in \cref{ex:2.1.21}.
  \begin{enumerate}
    \item Prove that \(\V = \rg{\T} + \ns{\T}\), but \(\V\) is not a direct sum of these two spaces.
          Thus the result of \cref{ex:2.1.35}(a) cannot be proved without assuming that \(\V\) is finite-dimensional.
    \item Find a linear operator \(\T_1\) on \(\V\) such that \(\rg{\T_1} \cap \ns{\T_1} = \set{\zv}\) but \(\V\) is not a direct sum of \(\rg{\T_1}\) and \(\ns{\T_1}\).
          Conclude that \(\V\) being finite-dimensional is also essential in \cref{ex:2.1.35}(b).
  \end{enumerate}
\end{ex}

\begin{proof}[\pf{ex:2.1.36}(a)]
  We have
  \begin{align*}
             & \T \text{ is onto}                                                   &  & \text{(by \cref{ex:2.1.21}(b))} \\
    \implies & \V = \rg{\T}                                                         &  & \text{(by \cref{2.1.10})}       \\
    \implies & \rg{\T} + \ns{\T} \subseteq \V = \rg{\T} \subseteq \rg{\T} + \ns{\T} &  & \text{(by \cref{ex:1.3.23}(a))} \\
    \implies & \V = \rg{\T} + \ns{\T}.
  \end{align*}
  Since \(\T(1, 0, 0, \dots) = (0, 0, \dots)\), we know that \((1, 0, 0, \dots) \in \ns{\T}\).
  Thus \((1, 0, 0, \dots) \in \rg{\T} \cap \ns{\T}\) and \(\rg{\T} \cap \ns{\T} \neq \set{\zv}\).
  By \cref{1.3.11} this means \(\V\) is not a direct sum of \(\rg{\T}\) and \(\ns{\T}\).
\end{proof}

\begin{proof}[\pf{ex:2.1.36}(b)]
  Define \(\U\) as in \cref{ex:2.1.21}.
  By \cref{ex:2.1.21}(c) we know that \(\U\) is one-to-one, thus by \cref{2.4} we know that \(\ns{\U} = \set{\zv}\) and therefore \(\rg{\U} \cap \ns{\U} = \set{\zv}\).
  By \cref{ex:2.1.21}(c) we know that \(\U\) is not onto, thus \(\V \neq \rg{\U} \cup \ns{\U}\).
  By \cref{1.3.11} this means \(\V\) is not a direct sum of \(\rg{\U}\) and \(\ns{\U}\).
\end{proof}

\begin{ex}\label{ex:2.1.37}
  A function \(\T : \V \to \W\) between vector spaces \(\V\) and \(\W\) over \(\F\) is called \textbf{additive} if \(\T(x + y) = \T(x) + \T(y)\) for all \(x, y \in \V\).
  Prove that if \(\V\) and \(\W\) are vector spaces over the field of rational numbers \(\Q\), then any additive function from \(\V\) into \(\W\) is a linear transformation.
\end{ex}

\begin{proof}[\pf{ex:2.1.37}]
  First observe that for all \(b \in \Z^+\) we have
  \begin{align*}
             & \forall x \in \V, \T(x) = \T\pa{b \frac{1}{b} x} = \T\pa{\sum_{i = 1}^b \frac{1}{b} x} &  & \text{(by \cref{1.2.1})}     \\
             & = \sum_{i = 1}^b \T\pa{\frac{1}{b} x} = b \T\pa{\frac{1}{b} x}                         &  & \text{(by \cref{ex:2.1.37})} \\
    \implies & \forall x \in \V, \frac{1}{b} \T(x) = \T\pa{\frac{1}{b} x}.                            &  & \text{(by \cref{1.2.1})}
  \end{align*}
  Let \(q \in \Q\).
  Then there exists some \((a, b) \in \Z \times \Z^+\) such that \(q = a / b\).
  Now we split into three cases:
  \begin{itemize}
    \item If \(a = 0\), then we have
          \begin{align*}
                     & \forall x \in \V, \T\pa{\frac{0}{b} x} = \T(0x) = \T(0x + \zv_{\V}) &  & \text{(by \ref{vs3})}        \\
                     & = \T(0x + 0x)                                                       &  & \text{(by \cref{1.2}(a))}    \\
                     & = \T(0x) + \T(0x) = \T(0x) + \T\pa{\frac{0}{b} x}                   &  & \text{(by \cref{ex:2.1.37})} \\
                     & = \T(0x) + \zv_{\W}                                                 &  & \text{(by \ref{vs3})}        \\
                     & = \T(0x) + 0 \T(x) = \T(0x) + \frac{0}{b} \T(x)                     &  & \text{(by \cref{1.2}(a))}    \\
            \implies & \forall x \in \V, \T\pa{\frac{0}{b}x} = \frac{0}{b} \T(x).          &  & \text{(by \cref{1.1})}
          \end{align*}
    \item If \(a > 0\), then we have
          \begin{align*}
            \forall x \in \V, \frac{a}{b} \T(x) & = \sum_{i = 1}^a \frac{1}{b} \T(x)    &  & \text{(by \cref{1.2.1})}      \\
                                                & = \sum_{i = 1}^a \T\pa{\frac{1}{b} x} &  & \text{(from the proof above)} \\
                                                & = \T\pa{\sum_{i = 1}^a \frac{1}{b} x} &  & \text{(by \cref{ex:2.1.37})}  \\
                                                & = \T\pa{\frac{a}{b} x}.               &  & \text{(by \cref{1.2.1})}
          \end{align*}
    \item If \(a < 0\), then we have
          \begin{align*}
                     & \forall x \in \V, \T\pa{\frac{a}{b} x} + \T\pa{\frac{-a}{b} x} = \T\pa{\frac{a}{b} x + \frac{-a}{b} x} &  & \text{(by \cref{ex:2.1.37})}  \\
                     & = \T(\zv_{\V})                                                                                         &  & \text{(by \cref{1.2.1})}      \\
                     & = \zv_{\W}                                                                                             &  & \text{(from the proof above)} \\
            \implies & \forall x \in \V, \T\pa{\frac{a}{b} x} + \frac{-a}{b} \T(x) = \zv_{\W}                                 &  & \text{(from the proof above)} \\
            \implies & \forall x \in \V, \T\pa{\frac{a}{b} x} = \frac{a}{b} \T(x).
          \end{align*}
  \end{itemize}
  From all cases above we conclude that \(\T(qx) = q \T(x)\).
  Since \(q\) is arbitrary and \(\T\) is additive, by \cref{2.1.1} we know that \(\T\) is linear.
\end{proof}

\section{The Matrix Representation of a Linear Transformation}\label{sec:2.2}

\begin{note}
  In \cref{sec:2.2}, we embark on one of the most useful approaches to the analysis of a linear transformation on a finite-dimensional vector space:
  the representation of a linear transformation by a matrix.
  In fact, we develop a one-to-one correspondence between matrices and linear transformations that allows us to utilize properties of one to study properties of the other.
\end{note}

\begin{defn}\label{2.2.1}
  Let \(\V\) be a finite-dimensional vector space over \(\F\).
  An \textbf{ordered basis} for \(\V\) over \(\F\) is a basis for \(\V\) over \(\F\) endowed with a specific order;
  that is, an ordered basis for \(\V\) over \(\F\) is a finite sequence of linearly independent vectors in \(\V\) that generates \(\V\).
\end{defn}

\begin{defn}\label{2.2.2}
  For the vector space \(\vs{F}^n\), we call \(\set{\seq{e}{1,2,,n}}\) the \textbf{standard ordered basis} for \(\vs{F}^n\) over \(\F\).
  Similarly, for the vector space \(\ps[n]{\F}\), we call \(\set{1, x, \dots, x^n}\) the \textbf{standard ordered basis} for \(\ps[n]{\F}\) over \(\F\).
\end{defn}

\begin{defn}\label{2.2.3}
  Let \(\beta = \set{\seq{u}{1,2,,n}}\) be an ordered basis for a finite-dimensional vector space \(\V\) over \(\F\).
  For \(x \in \V\), let \(\seq{a}{1,2,,n} \in \F\) be the unique scalars such that
  \[
    x = \sum_{i = 1}^n a_i u_i.
  \]
  We define the \textbf{coordinate vector of \(x\) relative to \(\beta\)}, denoted \([x]_{\beta}\), by
  \[
    [x]_{\beta} = \begin{pmatrix}
      a_1    \\
      a_2    \\
      \vdots \\
      a_n
    \end{pmatrix}.
  \]
\end{defn}

\begin{defn}\label{2.2.4}
  Suppose that \(\V\) and \(\W\) are finite-dimensional vector spaces over \(\F\) with ordered bases \(\beta = \set{\seq{v}{1,2,,n}}\) and \(\gamma = \set{\seq{w}{1,2,,m}}\) over \(\F\), respectively.
  Let \(\T : \V \to \W\) be linear.
  Then for each \(j\), \(1 \leq j \leq n\), there exist unique scalars \(a_{i j} \in \F\), \(1 \leq i \leq m\), such that
  \[
    \T(v_j) = \sum_{i = 1}^m a_{i j} w_i \quad \text{for } 1 \leq j \leq n.
  \]
  We call the \(m \times n\) matrix \(A\) defined by \(A_{i j} = a_{i j}\) the \textbf{matrix representation of \(\T\) in the ordered bases \(\beta\) and \(\gamma\)} and write \(A = [\T]_{\beta}^{\gamma}\).
  If \(\V = \W\) and \(\beta = \gamma\), then we write \(A = [\T]_{\beta}\).
\end{defn}

\begin{note}
  The \(j\)th column of \(A\) is simply \([\T(v_j)]_{\gamma}\).
  Also observe that if \(\U : \V \to \W\) is a linear transformation such that \([\U]_{\beta}^{\gamma} = [\T]_{\beta}^{\gamma}\), then \(\U = \T\) by \cref{2.1.13}.
\end{note}

\begin{defn}\label{2.2.5}
  Let \(\T, \U : \V \to \W\) be arbitrary functions, where \(\V\) and \(\W\) are vector spaces over \(\F\), and let \(a \in \F\).
  We define \(\T + \U : \V \to \W\) by \((\T + \U)(x) = \T(x) + \U(x)\) for all \(x \in \V\), and \(a \T: \V \to \W\) by \((a \T)(x) = a \T(x)\) for all \(x \in \V\).
\end{defn}

\begin{thm}\label{2.7}
  Let \(\V\) and \(\W\) be vector spaces over a field \(\F\), and let \(\T, \U : \V \to \W\) be linear.
  \begin{enumerate}
    \item For all \(a \in \F\), \(a \T + \U\) is linear.
    \item Using the operations of addition and scalar multiplication in \cref{2.2.5}, the collection of all linear transformations from \(\V\) to \(\W\) is a vector space over \(\F\).
  \end{enumerate}
\end{thm}

\begin{proof}[\pf{2.7}(a)]
  Let \(x, y \in \V\) and \(c \in \F\).
  Then
  \begin{align*}
    (a \T + \U)(cx + y) & = a \T(cx + y) + \U(cx + y)            &  & \text{(by \cref{2.2.5})}    \\
                        & = a(\T(cx + y)) + c \U(x) + \U(y)      &  & \text{(by \cref{2.1.2}(b))} \\
                        & = a(c \T(x) + \T(y)) + c \U(x) + \U(y) &  & \text{(by \cref{2.1.2}(b))} \\
                        & = ac \T(x) + c \U(x) + a \T(y) + \U(y) &  & \text{(by \cref{1.2.1})}    \\
                        & = c (a \T + \U)(x) + (a \T + \U)(y).   &  & \text{(by \cref{2.2.5})}
  \end{align*}
  So \(a \T + \U\) is linear.
\end{proof}

\begin{proof}[\pf{2.7}(b)]
  Let \(\ls(\V, \W)\) be the set of all linear transformation from \(\V\) to \(\W\).
  First we show that \ref{vs1}--\ref{vs8} is true.
  Let \(f, g, h \in \ls(\V, \W)\) and let \(a, b \in \F\).
  Now we split into eight cases:
  \begin{description}
    \item[For \ref{vs1}:] We have
      \begin{align*}
        \forall x \in \V, (f + g)(x) & = f(x) + g(x) &  & \text{(by \cref{2.2.5})} \\
                                     & = g(x) + f(x) &  & \text{(by \ref{vs1})}    \\
                                     & = (g + f)(x)  &  & \text{(by \cref{2.2.5})}
      \end{align*}
      and thus \(f + g = g + f\).
    \item[For \ref{vs2}:] We have
      \begin{align*}
        \forall x \in \V, ((f + g) + h)(x) & = (f + g)(x) + h(x)    &  & \text{(by \cref{2.2.5})} \\
                                           & = (f(x) + g(x)) + h(x) &  & \text{(by \cref{2.2.5})} \\
                                           & = f(x) + (g(x) + h(x)) &  & \text{(by \ref{vs2})}    \\
                                           & = f(x) + (g + h)(x)    &  & \text{(by \cref{2.2.5})} \\
                                           & = (f + (g + h))(x)     &  & \text{(by \cref{2.2.5})}
      \end{align*}
      and thus \((f + g) + h = f + (g + h)\).
    \item[For \ref{vs3}:] We have
      \begin{align*}
        \forall x \in \V, (f + \zT)(x) & = f(x) + \zT(x)   &  & \text{(by \cref{2.2.5})} \\
                                       & = f(x) + \zv_{\W} &  & \text{(by \cref{2.1.9})} \\
                                       & = f(x)            &  & \text{(by \ref{vs3})}
      \end{align*}
      and thus \(f + \zT = f\).
    \item[For \ref{vs4}:] We have
      \begin{align*}
        \forall x \in \V, (f + ((-1)f))(x) & = f(x) + ((-1)f)(x) &  & \text{(by \cref{2.2.5})} \\
                                           & = f(x) + (-1)f(x)   &  & \text{(by \cref{2.2.5})} \\
                                           & = \zv_{\W}          &  & \text{(by \ref{vs4})}    \\
                                           & = \zT(x)            &  & \text{(by \cref{2.1.9})}
      \end{align*}
      and thus \(f + (-1)f = \zT\).
    \item[For \ref{vs5}:] We have
      \begin{align*}
        \forall x \in \V, (1f)(x) & = 1f(x) &  & \text{(by \cref{2.2.5})} \\
                                  & = f(x)  &  & \text{(by \ref{vs5})}
      \end{align*}
      and thus \(1f = f\).
    \item[For \ref{vs6}:] We have
      \begin{align*}
        \forall x \in \V, ((ab)f)(x) & = (ab) f(x)   &  & \text{(by \cref{2.2.5})} \\
                                     & = a (bf(x))   &  & \text{(by \ref{vs6})}    \\
                                     & = a ((bf)(x)) &  & \text{(by \cref{2.2.5})} \\
                                     & = (a (bf))(x) &  & \text{(by \cref{2.2.5})}
      \end{align*}
      and thus \((ab)f = a (bf)\).
    \item[For \ref{vs7}:] We have
      \begin{align*}
        \forall x \in \V, (a(f + g))(x) & = a((f + g)(x))     &  & \text{(by \cref{2.2.5})} \\
                                        & = a(f(x) + g(x))    &  & \text{(by \cref{2.2.5})} \\
                                        & = af(x) + ag(x)     &  & \text{(by \ref{vs7})}    \\
                                        & = (af)(x) + (ag)(x) &  & \text{(by \cref{2.2.5})} \\
                                        & = (af + ag)(x)      &  & \text{(by \cref{2.2.5})}
      \end{align*}
      and thus \(a(f + g) = af + ag\).
    \item[For \ref{vs8}:] We have
      \begin{align*}
        \forall x \in \V, ((a + b)f)(x) & = (a + b) f(x)      &  & \text{(by \cref{2.2.5})} \\
                                        & = af(x) + bf(x)     &  & \text{(by \ref{vs8})}    \\
                                        & = (af)(x) + (bf)(x) &  & \text{(by \cref{2.2.5})} \\
                                        & = (af + bf)(x)      &  & \text{(by \cref{2.2.5})}
      \end{align*}
      and thus \((a + b)f = af + bf\).
  \end{description}
  From the proofs above we see that \ref{vs1}--\ref{vs8} is true.

  By \cref{2.7}(a) and the proofs above we see that
  \[
    \forall \T, \U \in \ls(\V, \W), \T + \U = 1 \T + \U \in \ls(\V, \W)
  \]
  and
  \[
    \forall (c, \T) \in \F \times \ls(\V, \W), c \T = c \T + \zT \in \ls(\V, \W).
  \]
  Thus by \cref{1.2.1} \(\ls(\V, \W)\) is a vector space over \(\F\).
\end{proof}

\begin{defn}\label{2.2.6}
  Let \(\V\) and \(\W\) be vector spaces over \(\F\).
  We denote the vector space of all linear transformations from \(\V\) into \(\W\) by \(\ls(\V, \W)\).
  In the case that \(\V = \W\), we write \(\ls(\V)\) instead of \(\ls(\V, \W)\).
\end{defn}

\begin{thm}\label{2.8}
  Let \(\V\) and \(\W\) be finite-dimensional vector spaces over \(\F\) with ordered bases \(\beta\) and \(\gamma\) over \(\F\), respectively, and let \(\T, \U : \V \to \W\) be linear transformations.
  Then
  \begin{enumerate}
    \item \([\T + \U]_{\beta}^{\gamma} = [\T]_{\beta}^{\gamma} + [\U]_{\beta}^{\gamma}\) and
    \item \([c \T]_{\beta}^{\gamma} = c [\T]_{\beta}^{\gamma}\) for all scalars \(c \in \F\).
  \end{enumerate}
\end{thm}

\begin{proof}[\pf{2.8}]
  Let \(\beta = \set{\seq{v}{1,2,,n}}\) and \(\gamma = \set{\seq{w}{1,2,,m}}\).
  There exist unique scalars \(a_{i j}\) and \(b_{i j}\) (\(1 \leq i \leq m\), \(1 \leq j \leq n\)) such that
  \[
    \T(v_j) = \sum_{i = 1}^m a_{i j} w_i \quad \text{and} \quad \U(v_j) = \sum_{i = 1}^m b_{i j} w_i \quad \text{for } 1 \leq j \leq n.
  \]
  Hence
  \begin{align*}
    (\T + \U)(v_j) & = \sum_{i = 1}^m (a_{i j} + b_{i j}) w_i \\
    (c \T)(v_j)    & = \sum_{i = 1}^m (c a_{i j}) w_i.
  \end{align*}
  Thus
  \begin{align*}
    ([\T + \U]_{\beta}^{\gamma})_{i j} & = a_{i j} + b_{i j} = ([\T]_{\beta}^{\gamma} + [\U]_{\beta}^{\gamma})_{i j} \\
    ([c \T]_{\beta}^{\gamma})_{i j}    & = c a_{i j} = c ([\T]_{\beta}^{\gamma})_{i j}.
  \end{align*}
\end{proof}

\exercisesection

\setcounter{ex}{7}
\begin{ex}\label{ex:2.2.8}
  Let \(\V\) be an \(n\)-dimensional vector space over \(\F\) with an ordered basis \(\beta\) over \(\F\).
  Define \(\T : \V \to \vs{F}^n\) by \(\T(x) = [x]_{\beta}\).
  Prove that \(\T\) is linear.
\end{ex}

\begin{proof}[\pf{ex:2.2.8}]
  Let \(\beta = \set{\seq{v}{1,2,,n}}\), let \(x, y \in \V\) and let \(c \in \F\).
  Since
  \begin{align*}
             & \exists \seq{a}{1,2,,n}, \seq{b}{1,2,,n} \in \F : \begin{dcases}
                                                                   x = \sum_{i = 1}^n a_i v_i \\
                                                                   y = \sum_{i = 1}^n b_i v_i
                                                                 \end{dcases} &  & \text{(by \cref{1.8})}       \\
    \implies & cx + y = \sum_{i = 1}^n (ca_i + b_i) v_i                         &  & \text{(by \cref{1.2.1})}   \\
    \implies & \T(cx + y) = [cx + y]_{\beta} = \begin{pmatrix}
                                                 ca_1 + b_1 \\
                                                 \vdots     \\
                                                 ca_n + b_n
                                               \end{pmatrix}                  &  & \text{(by \cref{2.2.3})}     \\
             & = c \begin{pmatrix}
                     a_1    \\
                     \vdots \\
                     a_n
                   \end{pmatrix} + \begin{pmatrix}
                                     b_1    \\
                                     \vdots \\
                                     b_n
                                   \end{pmatrix}                                  &  & \text{(by \cref{1.2.9})} \\
             & = c [x]_{\beta} + [y]_{\beta} = c \T(x) + \T(y),                 &  & \text{(by \cref{2.2.3})}
  \end{align*}
  by \cref{2.1.2}(b) we know that \(\T\) is linear.
\end{proof}

\begin{ex}\label{ex:2.2.9}
  Let \(\V\) be the vector space of complex numbers \(\C\) over the field \(\R\).
  Define \(\T : \V \to \V\) by \(\T(z) = \overline{z}\), where \(\overline{z}\) is the complex conjugate of \(z\).
  Prove that \(\T\) is linear, and compute \([\T]_{\beta}\), where \(\beta = \set{1, i}\).
  (Recall by \cref{ex:2.1.38} that \(\T\) is not linear if \(\V\) is regarded as a vector space over the field \(\C\).)
\end{ex}

\begin{proof}[\pf{ex:2.2.9}]
  Let \(x, y \in \C\) and let \(c \in \R\).
  Since
  \begin{align*}
    \T(cx + y) & = \overline{cx + y}                              \\
               & = \overline{cx} + \overline{y}                   \\
               & = \overline{c} \cdot \overline{x} + \overline{y} \\
               & = c \overline{x} + \overline{y}                  \\
               & = c \T(x) + \T(y),
  \end{align*}
  by \cref{2.1.2}(b) we know that \(\T\) is linear.
  Since
  \begin{align*}
    \T(1) & = 1 = 1 \cdot 1 + 0 \cdot i,     \\
    \T(i) & = -i = 0 \cdot 1 + (-1) \cdot i,
  \end{align*}
  by \cref{2.2.4} we know that
  \[
    [\T]_{\beta} = \begin{pmatrix}
      1 & 0  \\
      0 & -1
    \end{pmatrix}.
  \]
\end{proof}

\begin{ex}\label{ex:2.2.10}
  Let \(\V\) be a vector space over \(\F\) with the ordered basis \(\beta = \set{\seq{v}{1,2,,n}}\) over \(\F\).
  Define \(v_0 = 0\).
  By \cref{2.6} there exists a \(\T \in \ls(\V)\) such that \(\T(v_j) = v_j + v_{j - 1}\) for \(j = 1, 2, \dots, n\).
  Compute \([\T]_{\beta}\).
\end{ex}

\begin{proof}[\pf{ex:2.2.10}]
  By \cref{2.2.4,ex:1.5.6} we have
  \[
    [\T]_{\beta} = \begin{pmatrix}
      1      & 1      & 0      & \cdots & 0      & 0      & 0      \\
      0      & 1      & 1      & \cdots & 0      & 0      & 0      \\
      0      & 0      & 1      & \cdots & 0      & 0      & 0      \\
      \vdots & \vdots & \vdots & \ddots & \vdots & \vdots & \vdots \\
      0      & 0      & 0      & \cdots & 1      & 1      & 0      \\
      0      & 0      & 0      & \cdots & 0      & 1      & 1      \\
      0      & 0      & 0      & \cdots & 0      & 0      & 1
    \end{pmatrix} = \sum_{i = 1}^n E^{i i} + \sum_{i = 2}^n E^{(i - 1) i}.
  \]
\end{proof}

\begin{ex}\label{ex:2.2.11}
  Let \(\V\) be an \(n\)-dimensional vector space over \(\F\), and let \(\T \in \ls(\V)\).
  Suppose that \(\W\) is a \(\T\)-invariant subspace of \(\V\) over \(\F\) (see \cref{2.1.15}) having dimension \(k\).
  Show that there is a basis \(\beta\) for \(\V\) over \(\F\) such that \([\T]_{\beta}\) has the form
  \[
    \begin{pmatrix}
      A   & B \\
      \zm & C
    \end{pmatrix},
  \]
  where \(A\) is a \(k \times k\) matrix and \(\zm\) is the \((n - k) \times k\) zero matrix.
\end{ex}

\begin{proof}[\pf{ex:2.2.11}]
  Let \(\beta_{\W} = \set{\seq{v}{1,2,,k}}\) be a basis for \(\W\) over \(\F\).
  By \cref{1.6.19} we can extend \(\beta_{\W}\) to \(\beta = \set{\seq{v}{1,2,,n}}\) such that \(\beta\) is a basis for \(\V\) over \(\F\).
  Since \(\W\) is \(\T\)-invariant, we know that
  \begin{align*}
             & \T(\W) \subseteq \W                                                                                               &  & \text{(by \cref{2.1.15})} \\
    \implies & \forall v_j \in \beta_{\W}, \T(v_j) \in \W = \spn{\beta_{\W}}                                                     &  & \text{(by \cref{1.6.1})}  \\
    \implies & \forall v_j \in \beta_{\W}, \exists a_{1 j}, a_{2 j}, \dots, a_{k j} \in \F :                                                                    \\
             & \T(v_j) = \sum_{i = 1}^k a_{i j} v_i = \sum_{i = 1}^k a_{i j} v_i + \sum_{i = k + 1}^n 0 v_i                      &  & \text{(by \cref{1.2}(a))} \\
    \implies & \forall v_j \in \beta_{\W}, \exists a_{1 j}, a_{2 j}, \dots, a_{k j} \in \F : [\T(v_j)]_{\beta} = \begin{pmatrix}
                                                                                                                   a_{1 j} \\
                                                                                                                   \vdots  \\
                                                                                                                   a_{k j} \\
                                                                                                                   0       \\
                                                                                                                   \vdots  \\
                                                                                                                   0
                                                                                                                 \end{pmatrix}. &  & \text{(by \cref{2.2.3})}
  \end{align*}
  By setting
  \begin{align*}
    A & = \begin{pmatrix}
            a_{1 1} & \cdots & a_{1 k} \\
            \vdots  & \ddots & \vdots  \\
            a_{k 1} & \cdots & a_{k k}
          \end{pmatrix} \in \ms{k}{k}{\F}                                                                    \\
    B & = \begin{pmatrix}
            ([\T(v_{k + 1})]_{\beta})_1 & \cdots & ([\T(v_n)]_{\beta})_1 \\
            \vdots                      & \ddots & \vdots                \\
            ([\T(v_{k + 1})]_{\beta})_k & \cdots & ([\T(v_n)]_{\beta})_k
          \end{pmatrix} \in \ms{k}{(n - k)}{\F}             \\
    C & = \begin{pmatrix}
            ([\T(v_{k + 1})]_{\beta})_{k + 1} & \cdots & ([\T(v_n)]_{\beta})_{k + 1} \\
            \vdots                            & \ddots & \vdots                      \\
            ([\T(v_{k + 1})]_{\beta})_n       & \cdots & ([\T(v_n)]_{\beta})_n
          \end{pmatrix} \in \ms{(n - k)}{(n - k)}{\F}
  \end{align*}
  we have
  \begin{align*}
    [\T]_{\beta} & = \begin{pmatrix}
                       ([\T(v_1)]_{\beta})_1 & \cdots & ([\T(v_n)]_{\beta})_1   \\
                       \vdots                & \ddots & \vdots                  \\
                       ([\T(v_1)]_{\beta})_n & \cdots & ([\T(v_n)]_{\beta})_{n}
                     \end{pmatrix}                                        &  & \text{(by \cref{2.2.4})}                                        \\
                 & = \begin{pmatrix}
                       a_{1 1} & \cdots & a_{1 k} & ([\T(v_{k + 1})]_{\beta})_1       & \cdots & ([\T(v_n)]_{\beta})_1       \\
                       \vdots  & \ddots & \vdots  & \vdots                            & \ddots & \vdots                      \\
                       a_{k 1} & \cdots & a_{k k} & ([\T(v_{k + 1})]_{\beta})_k       & \cdots & ([\T(v_n)]_{\beta})_k       \\
                       0       & \cdots & 0       & ([\T(v_{k + 1})]_{\beta})_{k + 1} & \cdots & ([\T(v_n)]_{\beta})_{k + 1} \\
                       \vdots  & \ddots & \vdots  & \vdots                            & \ddots & \vdots                      \\
                       0       & \cdots & 0       & ([\T(v_{k + 1})]_{\beta})_{n}     & \cdots & ([\T(v_n)]_{\beta})_{n}
                     \end{pmatrix} \\
                 & = \begin{pmatrix}
                       A   & B \\
                       \zm & C
                     \end{pmatrix}.
  \end{align*}
\end{proof}

\begin{ex}\label{ex:2.2.12}
  Let \(\V\) be a finite-dimensional vector space over \(\F\) and \(\T\) be the projection on \(\W\) along \(\W'\), where \(\W\) and \(\W'\) are subspaces of \(\V\) over \(\F\).
  (See \cref{2.1.14}.)
  Find an ordered basis \(\beta\) for \(\V\) over \(\F\) such that \([\T]_{\beta}\) is a diagonal matrix.
\end{ex}

\begin{proof}[\pf{ex:2.2.12}]
  Let \(\beta_{\W} = \set{\seq{v}{1,2,,k}}\) be a basis for \(\W\) over \(\F\).
  By \cref{1.6.19} we can extend \(\beta_{\W}\) to \(\beta = \set{\seq{v}{1,2,,n}}\) such that \(\beta\) is a basis for \(\V\) over \(\F\).
  Since
  \begin{align*}
             & \begin{dcases}
                 \forall v_j \in \beta_{\W}, v_j = v_j + \zv \in \W + \W' \\
                 \forall v_j \in \beta \setminus \beta_{\W}, v_j = \zv + v_j \in \W + \W'
               \end{dcases}                         &  & \text{(by \cref{1.3.10})}                         \\
    \implies & \begin{dcases}
                 \forall v_j \in \beta_{\W}, \T(v_j) = v_j \\
                 \forall v_j \in \beta \setminus \beta_{\W}, \T(v_j) = \zv
               \end{dcases}                                        &  & \text{(by \cref{2.1.14})}              \\
    \implies & \begin{dcases}
                 \forall v_j \in \beta_{\W}, [\T(v_j)]_{\beta} = e_j \in \vs{F}^n \\
                 \forall v_j \in \beta \setminus \beta_{\W}, [\T(v_j)]_{\beta} = \zv \in \vs{F}^n
               \end{dcases} &  & \text{(by \cref{2.2.2})} \\
    \implies & [\T]_{\beta} = \begin{pmatrix}
                                e_1 & \cdots & e_k & \zv & \cdots & \zv
                              \end{pmatrix}                                          \\
             & = \begin{pmatrix}
                   1      & 0      & \cdots & 0      & 0      & \cdots & 0      \\
                   0      & 1      & \cdots & 0      & 0      & \cdots & 0      \\
                   \vdots & \vdots & \ddots & \vdots & 0      & \cdots & 0      \\
                   0      & 0      & \cdots & 1      & 0      & \cdots & 0      \\
                   0      & 0      & \cdots & 0      & 0      & \cdots & 0      \\
                   \vdots & \vdots & \ddots & \vdots & \vdots & \ddots & \vdots \\
                   0      & 0      & \cdots & 0      & 0      & \cdots & 0
                 \end{pmatrix},
  \end{align*}
  by \cref{1.3.8} we know that \([\T]_{\beta}\) is a diagonal matrix.
\end{proof}

\begin{ex}\label{ex:2.2.13}
  Let \(\V\) and \(\W\) be vector spaces over \(\F\), and let \(\T\) and \(\U\) be nonzero linear transformations from \(\V\) into \(\W\).
  If \(\rg{\T} \cap \rg{\U} = \set{\zv}\), prove that \(\set{\T, \U}\) is a linearly independent subset of \(\ls(\V, \W)\).
\end{ex}

\begin{proof}[\pf{ex:2.2.13}]
  Let \(\zT : \V \to \W\) be the zero transformation in \(\ls(\V, \W)\).
  Suppose for sake of contradiction that \(\set{\T, \U}\) is linearly dependent.
  Since
  \begin{align*}
             & \set{\T} \text{ is linearly independent}        &  & \text{(by \cref{1.5.4}(b))} \\
    \implies & \U \in \spn{\set{\T}}                           &  & \text{(by \cref{1.7})}      \\
    \implies & \exists c \in \F \setminus \set{0} : \U = c \T, &  & \text{(by \cref{1.4.3})}
  \end{align*}
  by fixing such \(c\) we know that
  \begin{align*}
             & \T \neq \zT                                                                             \\
    \implies & \exists x \in \V : \T(x) \neq \zv_{\W}                                                  \\
    \implies & \exists x \in \V : \U(\frac{1}{c} x) = \frac{1}{c} \U(x) &  & \text{(by \cref{2.1.1})}  \\
             & = \frac{1}{c} (c \T(x)) = \T(x) \neq \zv_{\W}                                           \\
    \implies & \rg{\U} \cap \rg{\T} \neq \set{\zv}.                     &  & \text{(by \cref{2.1.10})}
  \end{align*}
  But this contradicts to the fact that \(\rg{\U} \cap \rg{\T} = \set{\zv}\).
  Thus \(\set{\T, \U}\) is linearly independent.
\end{proof}

\begin{ex}\label{ex:2.2.14}
  Let \(f \in \ps{\R}\), and for \(j \geq 1\) define \(\T_j(f) = f^{(j)}\), where \(f^{(j)}\) is the \(j\)th derivative of \(f\).
  Prove that the set \(\set{\seq{\T}{1,2,,n}}\) is a linearly independent subset of \(\ls(\ps{\R})\) for any positive integer \(n\).
\end{ex}

\begin{proof}[\pf{ex:2.2.14}]
  First we show that \(\T_i \in \ls(\ps{\R})\) for all \(i \in \Z^+\).
  Let \(f, g \in \ps{\R}\) and let \(c \in \R\).
  Since
  \[
    \forall i \in \Z^+, \begin{dcases}
      \rg{\T_i} \subseteq \ps{\R} \\
      \T_i(cf + g) = (cf + g)^{(i)} = c f^{(i)} + g^{(i)} = c \T_i(f) + \T_i(g)
    \end{dcases},
  \]
  by \cref{2.1.2}(b) we know that \(\T_i \in \ls(\ps{\R})\) for all \(i \in \Z^+\).

  Now we show that \(\set{\seq{\T}{1,2,,n}}\) is linearly independent.
  Let \(\seq{b}{1,2,,n} \in \R\) such that
  \[
    \sum_{i = 1}^n b_i \T_i = \zT
  \]
  where \(\zT\) is the zero transformation of \(\ls(\ps{\R})\).
  Since
  \begin{align*}
             & \sum_{i = 1}^n b_i \T_i(x^n) = \sum_{i = 1}^n \pa{b_i \cdot \pa{\prod_{k = 0}^{i - 1} (n - k)} x^{n - i}} = 0                               \\
    \implies & \forall i \in \set{1, \dots, n}, b_i \cdot \prod_{k = 0}^{i - 1} (n - k) = 0                                  &  & \text{(by \cref{1.6.5})} \\
    \implies & b_i = 0,                                                                                                      &  & (n - k > 0)
  \end{align*}
  by \cref{1.5.3} we know that \(\set{\seq{\T}{1,2,,n}}\) is linearly independent.
\end{proof}

\begin{ex}\label{ex:2.2.15}
  Let \(\V\) and \(\W\) be vector spaces over \(\F\), and let \(S\) be a subset of \(\V\).
  Define \(S^0 = \set{\T \in \ls(\V, \W) : \T(x) = 0 \text{ for all } x \in S}\).
  Prove the following statements.
  \begin{enumerate}
    \item \(S^0\) is a subspace of \(\ls(\V, \W)\) over \(\F\).
    \item If \(S_1\) and \(S_2\) are subsets of \(V\) and \(S_1 \subseteq S_2\), then \(S_2^0 \subseteq S_1^0\).
    \item If \(\V_1\) and \(\V_2\) are subspaces of \(\V\) over \(\F\), then \((\V_1 + \V_2)^0 = \V_1^0 \cap \V_2^0\).
  \end{enumerate}
\end{ex}

\begin{proof}[\pf{ex:2.2.15}(a)]
  Let \(\zT : \V \to \W\) be the zero transformation in \(\ls(\V, \W)\).
  Since
  \begin{align*}
             & \forall x \in \V, \zT(x) = \zv_{\W} &  & \text{(by \cref{2.1.9})}     \\
    \implies & \forall x \in S, \zT(x) = \zv_{\W}                                    \\
    \implies & \zT \in S^0                         &  & \text{(by \cref{ex:2.2.15})}
  \end{align*}
  and
  \begin{align*}
             & \begin{dcases}
                 \forall \T, \U \in S^0 \\
                 \forall c \in \F       \\
                 \forall x \in S
               \end{dcases}, (c\T + \U)(x) = c \T(x) + \U(x) = \zv_{\W} &  & \text{(by \cref{2.2.5})}     \\
    \implies & \begin{dcases}
                 \forall \T, \U \in S^0 \\
                 \forall c \in \F
               \end{dcases}, c\T + \U \in S^0,                          &  & \text{(by \cref{ex:2.2.15})}
  \end{align*}
  by \cref{ex:1.3.18} we know that \(S^0\) is a subspace of \(\ls(\V, \W)\) over \(\F\).
\end{proof}

\begin{proof}[\pf{ex:2.2.15}(b)]
  Let \(\T \in S_2^0\).
  Since
  \begin{align*}
             & \T \in S_2^0                                                          \\
    \implies & \forall x \in S_2, \T(x) = \zv_{\W} &  & \text{(by \cref{ex:2.2.15})} \\
    \implies & \forall x \in S_1, \T(x) = \zv_{\W} &  & (S_1 \subseteq S_2)          \\
    \implies & \T \in S_1^0                        &  & \text{(by \cref{ex:2.2.15})}
  \end{align*}
  and \(\T\) is arbitrary, we know that \(S_2^0 \subseteq S_1^0\).
\end{proof}

\begin{proof}[\pf{ex:2.2.15}(c)]
  Since
  \begin{align*}
         & \T \in (\V_1 + \V_2)^0                                                                    \\
    \iff & \forall x \in \V_1 + \V_2, \T(x) = \zv_{\W}             &  & \text{(by \cref{ex:2.2.15})} \\
    \iff & \forall (x_1, x_2) \in \V_1 \times \V_2, \begin{dcases}
                                                      x_1 + \zv_{\V} \in \V_1 + \V_2 \\
                                                      \zv_{\V} + x_2 \in \V_1 + \V_2 \\
                                                      x_1 + x_2 \in \V_1 + \V_2      \\
                                                      \T(x_1) = \T(x_2) = \T(x_1 + x_2) = \zv_{\W}
                                                    \end{dcases} &  & \text{(by \cref{1.3.10})}      \\
    \iff & \begin{dcases}
             \T \in \V_1^0 \\
             \T \in \V_2^0
           \end{dcases}                                        &  & \text{(by \cref{ex:2.2.15})}     \\
    \iff & \T \in \V_1^0 \cap \V_2^0,
  \end{align*}
  we know that \((\V_1 + \V_2)^0 = \V_1^0 \cap \V_2^0\).
\end{proof}

\begin{ex}\label{ex:2.2.16}
  Let \(\V\) and \(\W\) be vector spaces over \(\F\) such that \(\dim(\V) = \dim(\W)\), and let \(\T \in \ls(\V, \W)\).
  Show that there exist ordered bases \(\beta\) and \(\gamma\) for \(\V\) and \(\W\) over \(\F\), respectively, such that \([\T]_{\beta}^{\gamma}\) is a diagonal matrix.
\end{ex}

\begin{proof}[\pf{ex:2.2.16}]
  If \(\T\) is the zero transformation, then for arbitrary bases \(\beta\) and \(\gamma\) for \(\V\) and \(\W\) over \(\F\), respectively, we have
  \[
    [\T]_{\beta}^{\gamma} = \zm
  \]
  which by \cref{1.3.8} is a diagonal matrix.
  So suppose that \(\T\) is not the zero transformation.
  Let \(\beta\) be a basis for \(\V\) over \(\F\).
  By \cref{2.2} we know that \(\spn{\T(\beta)} = \rg{\T}\).
  Since \(\T\) is not the zero transformation, we know that
  \begin{align*}
             & \exists x \in \V : \T(x) \neq \zv_{\W}                &  & \text{(by \cref{2.1.9})}  \\
    \implies & 1 \leq \rk{\T} \leq \dim(\W)                          &  & \text{(by \cref{2.1.12})} \\
    \implies & \exists \gamma_1 \subseteq \T(\beta) : \begin{dcases}
                                                        \gamma_1 \text{ is linearly independent} \\
                                                        \spn{\gamma_1} = \rg{\T}
                                                      \end{dcases}. &  & \text{(by \cref{1.6.8})}
  \end{align*}
  Fix such \(\gamma_1\).
  By \cref{2.3} we know that \(\#(\gamma_1) \leq \#(\beta_1)\) for all \(\beta_1 \subseteq \beta\) and \(\T(\beta_1) = \gamma_1\).
  Thus we can define \(\beta_1 \subseteq \beta\) such that \(\T(\beta_1) = \gamma_1\) and \(\#(\beta_1) = \#(\gamma_1)\).
  Let \(k = \#(\beta_1)\) and we can write \(\beta_1 = \set{\seq{v}{1,2,,k}}\) and \(\gamma_1 = \set{\seq{w}{1,2,,k}}\) such that
  \[
    \forall i \in \set{1, \dots, k}, \T(v_i) = w_i.
  \]
  If we let \(\beta\) follow this ordered and we extend \(\gamma_1\) to a basis \(\gamma\) for \(\W\) over \(\F\) (this can be done by \cref{1.6.19}), then we see that (by setting \(n = \dim(\V)\) and \(\beta = \set{\seq{v}{1,2,,n}}\))
  \begin{align*}
             & \forall v_i \in \beta, [\T(v_i)]_{\gamma} = \begin{dcases}
                                                             \zv_{\W}           & \text{if } i \notin \set{1, \dots, k} \\
                                                             e_i \in \vs{F}^{n} & \text{if } i \in \set{1, \dots, k}
                                                           \end{dcases}   &  & \text{(by \cref{2.2.3})} \\
    \implies & [\T]_{\beta}^{\gamma} = \begin{pmatrix}
                                         1      & 0      & \cdots & 0      & 0      & \cdots & 0      \\
                                         0      & 1      & \cdots & 0      & 0      & \cdots & 0      \\
                                         \vdots & \vdots & \ddots & \vdots & \vdots & \ddots & \vdots \\
                                         0      & 0      & \cdots & 1      & 0      & \cdots & 0      \\
                                         0      & 0      & \cdots & 0      & 0      & \cdots & 0      \\
                                         \vdots & \vdots & \ddots & \vdots & \vdots & \ddots & 0      \\
                                         0      & 0      & \cdots & 0      & 0      & \cdots & 0
                                       \end{pmatrix} &  & \text{(by \cref{2.2.4})}
  \end{align*}
  and by \cref{1.3.8} \([\T]_{\beta}^{\gamma}\) is a diagonal matrix.
\end{proof}

\section{Composition of Linear Transformations and Matrix Multiplication}\label{sec:2.3}

\begin{note}
  We use the more convenient notation of \(\U \T\) rather than \(\U \circ \T\) for the composite of linear transformations \(\U\) and \(\T\).
\end{note}

\begin{thm}\label{2.9}
  Let \(\V\), \(\W\), and \(\vs{Z}\) be vector spaces over the same field \(\F\), and let \(\T \in \ls(\V, \W)\) and \(\U \in \ls(\W, \vs{Z})\).
  Then \(\U \T \in \ls(\V, \vs{Z})\).
\end{thm}

\begin{proof}[\pf{2.9}]
  Let \(x, y \in \V\) and \(a \in \F\).
  Then
  \begin{align*}
    \U \T(ax + y) & = \U(\T(ax + y))                                            \\
                  & = \U(a \T(x) + \T(y))      &  & \text{(by \cref{2.1.2}(b))} \\
                  & = a \U(\T(x)) + \U(\T(y))  &  & \text{(by \cref{2.1.2}(b))} \\
                  & = a (\U \T)(x) + \U \T(y).
  \end{align*}
\end{proof}

\begin{thm}\label{2.10}
  Let \(\V, \W, \vs{X}, \vs{Y}\) be vector spaces over \(\F\).
  Let \(\lt{S}, \lt{S}_1, \lt{S}_2 \in \ls(\vs{X}, \vs{Y})\), let \(\T \in \ls(\W, \vs{X})\) and let \(\U, \U_1, \U_2 \in \ls(\V, \W)\).
  Then
  \begin{enumerate}
    \item \(\T(\U_1 + \U_2) = \T \U_1 + \T \U_2\) and \((\lt{S}_1 + \lt{S}_2) \T = \lt{S}_1 \T + \lt{S}_2 \T\).
    \item \(\lt{S} (\T \U) = (\lt{S} \T) \U\).
    \item \(\T \IT[\W] = \IT[\vs{X}] \T = \T\).
    \item \(a(\T \U) = (a \T) \U = \T (a \U)\) for all scalars \(a \in \F\).
  \end{enumerate}
\end{thm}

\begin{proof}[\pf{2.10}(a)]
  For all \(x \in \V\), we have
  \begin{align*}
    (\T(\U_1 + \U_2))(x) & = \T((\U_1 + \U_2)(x))                                      \\
                         & = \T(\U_1(x) + \U_2(x))       &  & \text{(by \cref{2.2.5})} \\
                         & = \T(\U_1(x)) + \T(\U_2(x))   &  & \text{(by \cref{2.1.1})} \\
                         & = (\T \U_1)(x) + (\T \U_2)(x)                               \\
                         & = (\T \U_1 + \T \U_2)(x).     &  & \text{(by \cref{2.2.5})}
  \end{align*}
  Thus \(\T(\U_1 + \U_2) = \T \U_1 + \T \U_2\).
  For all \(x \in \W\), we have
  \begin{align*}
    ((\vs{S}_1 + \vs{S}_2) \T)(x) & = (\vs{S}_1 + \vs{S}_2)(\T(x))                                      \\
                                  & = \vs{S}_1(\T(x)) + \vs{S}_2(\T(x))   &  & \text{(by \cref{2.2.5})} \\
                                  & = (\vs{S}_1 \T)(x) + (\vs{S}_2 \T)(x)                               \\
                                  & = (\vs{S}_1 \T + \vs{S}_2 \T)(x).     &  & \text{(by \cref{2.2.5})}
  \end{align*}
  Thus \((\vs{S}_1 + \vs{S}_2) \T = \vs{S}_1 \T + \vs{S}_2 \T\).
\end{proof}

\begin{proof}[\pf{2.10}(b)]
  For all \(x \in \V\), we have
  \begin{align*}
    (\lt{S} (\T \U))(x) & = \lt{S}((\T \U)(x))  \\
                        & = \lt{S}(\T(\U(x)))   \\
                        & = (\lt{S} \T)(\U(x))  \\
                        & = ((\lt{S} \T) \U)(x)
  \end{align*}
  and thus \(\lt{S} (\T \U) = (\lt{S} \T) \U\).
\end{proof}

\begin{proof}[\pf{2.10}(c)]
  For all \(x \in \V\), we have
  \begin{align*}
    (\T \IT[\W])(x) & = \T(\IT[\W](x))                                    \\
                    & = \T(x)               &  & \text{(by \cref{2.1.9})} \\
                    & = \IT[\vs{X}](\T(x))  &  & \text{(by \cref{2.1.9})} \\
                    & = (\IT[\vs{X}] \T)(x)
  \end{align*}
  and thus \(\T \IT[\W] = \T = \IT[\vs{X}] \T\).
\end{proof}

\begin{proof}[\pf{2.10}(d)]
  For all \(x \in \V\), we have
  \begin{align*}
    (a (\T \U))(x) & = a (\T \U)(x)                                 \\
                   & = a \T(\U(x))                                  \\
                   & = (a \T)(\U(x))  &  & \text{(by \cref{2.2.5})} \\
                   & = ((a \T) \U)(x)                               \\
                   & = \T(a \U(x))    &  & \text{(by \cref{2.1.1})} \\
                   & = \T((a \U)(x))                                \\
                   & = (\T (a \U))(x)
  \end{align*}
  and thus \(a (\T \U) = (a \T) \U = \T (a \U)\).
\end{proof}

\begin{defn}\label{2.3.1}
  Let \(A \in \MS\) matrix and \(B \in \ms{n}{p}{\F}\).
  We define the \textbf{product} of \(A\) and \(B\), denoted \(AB\), to be the \(m \times p\) matrix such that
  \[
    (AB)_{i j} = \sum_{k = 1}^n A_{i k} B_{k j} \quad \text{for } 1 \leq i \leq m, 1 \leq j \leq p.
  \]
  Note that \((AB)_{i j}\) is the sum of products of corresponding entries from the \(i\)th row of \(A\) and the \(j\)th column of \(B\).
\end{defn}

\begin{note}
  The reader should observe that in order for the product \(AB\) to be defined, there are restrictions regarding the relative sizes of \(A\) and \(B\).
  The following mnemonic device is helpful:
  ``\((m \times n) \cdot (n \times p) = (m \times p)\)'';
  that is, in order for the product \(AB\) to be defined, the two ``inner'' dimensions must be equal, and the two ``outer'' dimensions yield the size of the product.
\end{note}

\begin{note}
  As in the case with composition of functions, we have that matrix multiplication is not commutative. Consider the following two products:
  \[
    \begin{pmatrix}
      1 & 1 \\
      0 & 0
    \end{pmatrix} \begin{pmatrix}
      0 & 1 \\
      1 & 0
    \end{pmatrix} = \begin{pmatrix}
      1 & 1 \\
      0 & 0
    \end{pmatrix} \quad \text{and} \quad \begin{pmatrix}
      0 & 1 \\
      1 & 0
    \end{pmatrix} \begin{pmatrix}
      1 & 1 \\
      0 & 0
    \end{pmatrix} = \begin{pmatrix}
      0 & 0 \\
      1 & 1
    \end{pmatrix}.
  \]
  Hence we see that even if both of the matrix products \(AB\) and \(BA\) are defined, it need not be true that \(AB = BA\).
\end{note}

\begin{eg}\label{2.3.2}
  If \(A \in \MS\) and \(B \in \ms{n}{p}{\F}\), then \(\tp{(AB)} = \tp{B} \tp{A}\).
\end{eg}

\begin{proof}[\pf{2.3.2}]
  Since
  \[
    \tp{(AB)}_{i j} = (AB)_{j i} = \sum_{k = 1}^n A_{j k} B_{k i}
  \]
  and
  \[
    (\tp{B} \tp{A})_{i j} = \sum_{k = 1}^n \tp{B}_{i k} \tp{A}_{k j} = \sum_{k = 1}^n B_{k i} A_{j k},
  \]
  we are finished.
  Therefore the transpose of a product is the product of the transposes in the \emph{opposite order}.
\end{proof}

\begin{thm}\label{2.11}
  Let \(\V\), \(\W\), and \(\vs{Z}\) be finite-dimensional vector spaces over \(\F\) with ordered bases \(\alpha\), \(\beta\), and \(\gamma\) over \(\F\), respectively.
  Let \(\T \in \ls(\V, \W)\) and \(\U \in \ls(\W, \vs{Z})\).
  Then
  \[
    [\U \T]_{\alpha}^{\gamma} = [\U]_{\beta}^{\gamma} [\T]_{\alpha}^{\beta}.
  \]
\end{thm}

\begin{proof}[\pf{2.11}]
  Define
  \begin{align*}
    \alpha & = \set{\seq{v}{1,2,,n}}; \\
    \beta  & = \set{\seq{w}{1,2,,m}}; \\
    \gamma & = \set{\seq{z}{1,2,,p}}.
  \end{align*}
  For \(1 \leq j \leq n\), we have
  \begin{align*}
    (\U \T)(v_j) & = \sum_{i = 1}^p ([\U \T]_{\alpha}^{\gamma})_{i j} \cdot z_i                                                      &  & \text{(by \cref{2.2.4})}    \\
                 & = \U(\T(v_j))                                                                                                                                      \\
                 & = \U\pa{\sum_{k = 1}^m ([\T]_{\alpha}^{\beta})_{k j} \cdot w_k}                                                   &  & \text{(by \cref{2.2.4})}    \\
                 & = \sum_{k = 1}^m ([\T]_{\alpha}^{\beta})_{k j} \cdot \U(w_k)                                                      &  & \text{(by \cref{2.1.2}(d))} \\
                 & = \sum_{k = 1}^m ([\T]_{\alpha}^{\beta})_{k j} \cdot \pa{\sum_{i = 1}^p ([\U]_{\beta}^{\gamma})_{i k} \cdot z_i}  &  & \text{(by \cref{2.2.4})}    \\
                 & = \sum_{k = 1}^m \sum_{i = 1}^p \pa{([\T]_{\alpha}^{\beta})_{k j}  \cdot ([\U]_{\beta}^{\gamma})_{i k} \cdot z_i} &  & \text{(by \cref{1.2.1})}    \\
                 & = \sum_{k = 1}^m \sum_{i = 1}^p \pa{([\U]_{\beta}^{\gamma})_{i k} \cdot ([\T]_{\alpha}^{\beta})_{k j} \cdot z_i}  &  & \text{(by \cref{1.2.1})}    \\
                 & = \sum_{i = 1}^p \sum_{k = 1}^m \pa{([\U]_{\beta}^{\gamma})_{i k} \cdot ([\T]_{\alpha}^{\beta})_{k j} \cdot z_i}  &  & \text{(by \cref{1.2.1})}    \\
                 & = \sum_{i = 1}^p \pa{\sum_{k = 1}^m ([\U]_{\beta}^{\gamma})_{i k} \cdot ([\T]_{\alpha}^{\beta})_{k j}} \cdot z_i  &  & \text{(by \cref{1.2.1})}    \\
                 & = \sum_{i = 1}^p \pa{\pa{[\U]_{\beta}^{\gamma} [\T]_{\alpha}^{\beta}}_{i j} \cdot z_i}.                           &  & \text{(by \cref{2.3.1})}
  \end{align*}
  Thus by \cref{1.8} we see that \([\U \T]_{\alpha}^{\gamma} = [\U]_{\beta}^{\gamma} [\T]_{\alpha}^{\beta}\).
\end{proof}

\begin{cor}\label{2.3.3}
  Let \(\V\) be a finite-dimensional vector space over \(\F\) with an ordered basis \(\beta\).
  Let \(\T, \U \in \ls(\V)\).
  Then \([\U \T]_{\beta} = [\U]_{\beta} [\T]_{\beta}\).
\end{cor}

\begin{proof}[\pf{2.3.3}]
  We have
  \begin{align*}
    [\U \T]_{\beta} & = [\U \T]_{\beta}^{\beta}                   &  & \text{(by \cref{2.2.4})} \\
                    & = [\U]_{\beta}^{\beta} [\T]_{\beta}^{\beta} &  & \text{(by \cref{2.11})}  \\
                    & = [\U]_{\beta} [\T]_{\beta}.                &  & \text{(by \cref{2.2.4})}
  \end{align*}
\end{proof}

\begin{defn}\label{2.3.4}
  We define the \textbf{Kronecker delta} \(\delta_{i j}\) by \(\delta_{i j} = 1\) if \(i = j\) and \(\delta_{i j} = 0\) if \(i \neq j\).
  The \(n \times n\) \textbf{identity matrix} \(I_n\) is defined by \((I_n)_{i j} = \delta_{i j}\).
\end{defn}

\begin{thm}\label{2.12}
  Let \(A \in \MS\), let \(B, C \in \ms{n}{p}{\F}\) and let \(D, E \in \ms{q}{m}{\F}\).
  Then
  \begin{enumerate}
    \item \(A (B + C) = AB + AC\) and \((D + E) A = DA + EA\).
    \item \(a (AB) = (aA) B = A (aB)\) for any \(a \in \F\).
    \item \(I_m A = A = A I_n\).
    \item If \(\V\) is an \(n\)-dimensional vector space over \(\F\) with an ordered basis \(\beta\), then \([\IT[\V]]_{\beta} = I_n\).
  \end{enumerate}
\end{thm}

\begin{proof}[\pf{2.12}(a)]
  We have
  \begin{align*}
    (A (B + C))_{i j} & = \sum_{k = 1}^n A_{i k} (B + C)_{k j}                            &  & \text{(by \cref{2.3.1})}           \\
                      & = \sum_{k = 1}^n A_{i k} (B_{k j} + C_{k j})                      &  & \text{(by \cref{1.2.9})}           \\
                      & = \sum_{k = 1}^n (A_{i k} B_{k j} + A_{i k} C_{k j})              &  & (A_{i k}, B_{k j}, C_{k j} \in \F) \\
                      & = \sum_{k = 1}^n A_{i k} B_{k j} + \sum_{k = 1}^n A_{i k} C_{k j} &  & (A_{i k}, B_{k j}, C_{k j} \in \F) \\
                      & = (AB)_{i j} + (AC)_{i j}                                         &  & \text{(by \cref{2.3.1})}           \\
                      & = (AB + AC)_{i j}                                                 &  & \text{(by \cref{1.2.9})}
  \end{align*}
  and
  \begin{align*}
    ((D + E) A)_{i j} & = \sum_{k = 1}^m (D + E)_{i k} A_{k j}                            &  & \text{(by \cref{2.3.1})}           \\
                      & = \sum_{k = 1}^m (D_{i k} + E_{i k}) A_{k j}                      &  & \text{(by \cref{1.2.9})}           \\
                      & = \sum_{k = 1}^m (D_{i k} A_{k j} + E_{i k} A_{k j})              &  & (A_{k j}, D_{i k}, E_{i k} \in \F) \\
                      & = \sum_{k = 1}^m D_{i k} A_{k j} + \sum_{k = 1}^m E_{i k} A_{k j} &  & (A_{k j}, D_{i k}, E_{i k} \in \F) \\
                      & = (DA)_{i j} + (EA)_{i j}                                         &  & \text{(by \cref{2.3.1})}           \\
                      & = (DA + EA)_{i j}.                                                &  & \text{(by \cref{1.2.9})}
  \end{align*}
  Thus by \cref{1.2.8} \(A (B + C) = AB + AC\) and \((D + E) A = DA + EA\).
\end{proof}

\begin{proof}[\pf{2.12}(b)]
  We have
  \begin{align*}
    (a (AB))_{i j} & = a (AB)_{i j}                          &  & \text{(by \cref{1.2.9})} \\
                   & = a \pa{\sum_{k = 1}^n A_{i k} B_{k j}} &  & \text{(by \cref{2.3.1})} \\
                   & = \sum_{k = 1}^n (a A_{i k}) B_{k j}    &  & \text{(by \cref{1.2.9})} \\
                   & = \sum_{k = 1}^n (a A)_{i k} B_{k j}    &  & \text{(by \cref{1.2.9})} \\
                   & = ((aA) B)_{i j}                        &  & \text{(by \cref{2.3.1})} \\
                   & = \sum_{k = 1}^n A_{i k} (a B_{k j})    &  & \text{(by \cref{1.2.9})} \\
                   & = \sum_{k = 1}^n A_{i k} (a B)_{k j}    &  & \text{(by \cref{1.2.9})} \\
                   & = (A (aB))_{i j}                        &  & \text{(by \cref{2.3.1})}
  \end{align*}
  thus by \cref{1.2.8} \(a (AB) = (aA) B = A (aB)\).
\end{proof}

\begin{proof}[\pf{2.12}(c)]
  We have
  \begin{align*}
    (I_m A)_{i j} & = \sum_{k = 1}^m (I_m)_{i k} A_{k j}  &  & \text{(by \cref{2.3.1})} \\
                  & = \sum_{k = 1}^m \delta_{i k} A_{k j} &  & \text{(by \cref{2.3.4})} \\
                  & = A_{i j}                             &  & \text{(by \cref{2.3.4})} \\
                  & = \sum_{k = 1}^n A_{i k} \delta_{k j} &  & \text{(by \cref{2.3.4})} \\
                  & = \sum_{k = 1}^n A_{i k} (I_n)_{k j}  &  & \text{(by \cref{2.3.4})} \\
                  & = (A I_n)_{i j}                       &  & \text{(by \cref{2.3.1})}
  \end{align*}
  thus by \cref{1.2.8} \(I_m A = A = A I_n\).
\end{proof}

\begin{proof}[\pf{2.12}(d)]
  Let \(\beta = \set{\seq{v}{1,2,,n}}\).
  For all \(j \in \set{1, 2, \dots, n}\), we have
  \begin{align*}
             & \IT[\V](v_j) = v_j                                                                          &  & \text{(by \cref{2.1.9})} \\
             & = \sum_{i = 1}^n ([\IT[\V]]_{\beta})_{i j} v_i                                              &  & \text{(by \cref{2.2.4})} \\
    \implies & \forall i \in \set{1, 2, \dots, n}, ([\IT[\V]]_{\beta})_{i j} = \delta_{i j} = (I_n)_{i j}. &  & \text{(by \cref{2.3.4})}
  \end{align*}
  Thus by \cref{1.2.8} \([\IT[\V]]_{\beta} = I_n\).
\end{proof}

\begin{note}
  \cref{2.12} provides analogs of (a), (c), and (d) of \cref{2.10}.
  \cref{2.10}(b) has its analog in \cref{2.16}.
  Observe also that part (c) of \cref{2.12} illustrates that the identity matrix acts as a multiplicative identity in \(\ms{n}{n}{\F}\).
  When the context is clear, we sometimes omit the subscript \(n\) from \(I_n\).
\end{note}

\begin{cor}\label{2.3.5}
  Let
  \begin{align*}
    A               & \in \MS;           \\
    \seq{B}{1,2,,k} & \in \ms{n}{p}{\F}; \\
    \seq{C}{1,2,,k} & \in \ms{q}{m}{\F}; \\
    \seq{a}{1,2,,k} & \in \F.
  \end{align*}
  Then
  \begin{align*}
    A \pa{\sum_{i = 1}^k a_i B_i} & = \sum_{i = 1}^k a_i A B_i, \\
    \pa{\sum_{i = 1}^k a_i C_i} A & = \sum_{i = 1}^k a_i C_i A.
  \end{align*}
\end{cor}

\begin{proof}[\pf{2.3.5}]
  We have
  \begin{align*}
    A \pa{\sum_{i = 1}^k a_i B_i} & = \sum_{i = 1}^k (A a_i B_i)  &  & \text{(by \cref{2.12}(a))} \\
                                  & = \sum_{i = 1}^k (a_i A B_i), &  & \text{(by \cref{2.12}(b))} \\
    \pa{\sum_{i = 1}^k a_i C_i} A & = \sum_{i = 1}^k (a_i C_i A). &  & \text{(by \cref{2.12}(a))}
  \end{align*}
\end{proof}

\begin{defn}\label{2.3.6}
  For an \(A \in \ms{n}{n}{\F}\), we define \(A^1 = A\), \(A^2 = AA\), \(A^3 = A^2 A\), and, in general, \(A^k = A^{k - 1} A\) for \(k = 2, 3, \dots\).
  We define \(A^0 = I_n\).
\end{defn}

\begin{eg}\label{2.3.7}
  If
  \[
    A = \begin{pmatrix}
      0 & 0 \\
      1 & 0
    \end{pmatrix},
  \]
  then \(A^2 = \zm\) (the zero matrix) even though \(A \neq \zm\).
  Thus the cancellation property for multiplication in fields is not valid for matrices.
  To see why, assume that the cancellation law is valid.
  Then, from \(A \cdot A = A^2 = \zm = A \cdot \zm\), we would conclude that \(A = \zm\), which is false.
\end{eg}

\begin{thm}\label{2.13}
  Let \(A \in \MS\) and \(B \in \ms{n}{p}{\F}\).
  For each \(j\) (\(1 \leq j \leq p\)) let \(u_j\) and \(v_j\) denote the \(j\)th columns of \(AB\) and \(B\), respectively.
  Then
  \begin{enumerate}
    \item \(u_j = A v_j\).
    \item \(v_j = B e_j\), where \(e_j\) is the \(j\)th standard vector of \(\vs{F}^p\).
  \end{enumerate}
\end{thm}

\begin{proof}[\pf{2.13}(a)]
  We have
  \begin{align*}
    u_j & = \begin{pmatrix}
              (AB)_{1 j} \\
              (AB)_{2 j} \\
              \vdots     \\
              (AB)_{m j}
            \end{pmatrix}                 &  & \text{(by \cref{2.2.4})}   \\
        & = \begin{pmatrix}
              \sum_{k = 1}^n A_{1 k} B_{k j} \\
              \sum_{k = 1}^n A_{2 k} B_{k j} \\
              \vdots                         \\
              \sum_{k = 1}^n A_{m k} B_{k j}
            \end{pmatrix} &  & \text{(by \cref{2.3.1})}                   \\
        & = A \begin{pmatrix}
                B_{1 j} \\
                B_{2 j} \\
                \vdots  \\
                B_{n j}
              \end{pmatrix}               &  & \text{(by \cref{2.3.1})}   \\
        & = A v_j.                          &  & \text{(by \cref{2.2.4})}
  \end{align*}
\end{proof}

\begin{proof}[\pf{2.13}(b)]
  We have
  \begin{align*}
    v_j & = \begin{pmatrix}
              B_{1 j} \\
              B_{2 j} \\
              \vdots  \\
              B_{m j}
            \end{pmatrix}                     &  & \text{(by \cref{2.2.4})}   \\
        & = \begin{pmatrix}
              (B I_p)_{1 j} \\
              (B I_p)_{2 j} \\
              \vdots        \\
              (B I_p)_{m j}
            \end{pmatrix}                     &  & \text{(by \cref{2.12}(c))} \\
        & = \begin{pmatrix}
              \sum_{k = 1}^p B_{1 k} (I_p)_{k j} \\
              \sum_{k = 1}^p B_{2 k} (I_p)_{k j} \\
              \vdots                             \\
              \sum_{k = 1}^p B_{m k} (I_p)_{k j}
            \end{pmatrix} &  & \text{(by \cref{2.3.1})}                       \\
        & = B \begin{pmatrix}
                (I_p)_{1 j} \\
                (I_p)_{2 j} \\
                \vdots      \\
                (I_p)_{p j}
              \end{pmatrix}                   &  & \text{(by \cref{2.3.1})}   \\
        & = B e_j.                              &  & \text{(by \cref{2.3.4})}
  \end{align*}
\end{proof}

\begin{note}
  It follows (see \cref{ex:2.3.14}) from \cref{2.13} that column \(j\) of \(AB\) is a linear combination of the columns of \(A\) with the coefficients in the linear combination being the entries of column \(j\) of \(B\).
  An analogous result holds for rows;
  that is, row \(i\) of \(AB\) is a linear combination of the rows of \(B\) with the coefficients in the linear combination being the entries of row \(i\) of \(A\).
\end{note}

\begin{thm}\label{2.14}
  Let \(\V\) and \(\W\) be finite-dimensional vector spaces over \(\F\) having ordered bases \(\beta\) and \(\gamma\) over \(\F\), respectively, and let \(\T \in \ls(\V, \W)\).
  Then, for each \(u \in \V\), we have
  \[
    [\T(u)]_{\gamma} = [\T]_{\beta}^{\gamma} [u]_{\beta}.
  \]
\end{thm}

\begin{proof}[\pf{2.14}]
  Fix \(u \in \V\), and define the linear transformations \(f : \F \to \V\) by \(f(a) = au\) and \(g : \F \to \W\) by \(g(a) = a \T(u)\) for all \(a \in \F\).
  Let \(\alpha = \set{1}\) be the standard ordered basis for \(\F\).
  Notice that \(g = \T f\).
  Identifying column vectors as matrices and using \cref{2.11}, we obtain
  \[
    [\T(u)]_{\gamma} = [g(1)]_{\gamma} = [g]_{\alpha}^{\gamma} = [\T f]_{\alpha}^{\gamma} = [\T]_{\beta}^{\gamma} [f]_{\alpha}^{\beta} = [\T]_{\beta}^{\gamma} [f(1)]_{\beta} = [\T]_{\beta}^{\gamma} [u]_{\beta}.
  \]
\end{proof}

\begin{defn}\label{2.3.8}
  Let \(A \in \ms{m}{n}{\F}\).
  We denote by \(\L_A\) the mapping \(\L_A : \vs{F}^n \to \vs{F}^m\) defined by \(\L_A(x) = Ax\) (the matrix product of \(A\) and \(x\)) for each column vector \(x \in \vs{F}^n\).
  We call \(\L_A\) a \textbf{left-multiplication transformation}.
\end{defn}

\begin{thm}\label{2.15}
  Let \(A \in \MS\).
  Then the left-multiplication transformation \(\L_A : \vs{F}^n \to \vs{F}^m\) is linear.
  Furthermore, if \(B \in \MS\) and \(\beta\) and \(\gamma\) are the standard ordered bases for \(\vs{F}^n\) and \(\vs{F}^m\) over \(\F\), respectively, then we have the following properties.
  \begin{enumerate}
    \item \([\L_A]_{\beta}^{\gamma} = A\).
    \item \(\L_A = \L_B\) iff \(A = B\).
    \item \(\L_{A + B} = \L_A + \L_B\) and \(\L_{aA} = a \L_A\) for all \(a \in \F\).
    \item If \(\T : \vs{F}^n \to \vs{F}^m\) is linear, then there exists a unique \(C \in \MS\) such that \(\T = \L_C\).
          In fact, \(C = [\T]_{\beta}^{\gamma}\).
    \item If \(E \in \ms{n}{p}{\F}\), then \(\L_{AE} = \L_A \L_E\).
    \item If \(m = n\), then \(\L_{I_n} = \IT[\vs{F}^n]\).
  \end{enumerate}
\end{thm}

\begin{proof}[\pf{2.15}(a)]
  The fact that \(\L_A\) is linear follows immediately from \cref{2.12}(a)(b).
  By \cref{2.2.4} the \(j\)th column of \([\L_A]_{\beta}^{\gamma}\) is equal to \(\L_A(e_j)\).
  However by \cref{2.3.8} we have \(\L_A(e_j) = A e_j\), which is also the \(j\)th column of \(A\) by \cref{2.13}(b).
  So \([\L_A]_{\beta}^{\gamma} = A\).
\end{proof}

\begin{proof}[\pf{2.15}(b)]
  If \(\L_A = \L_B\), then we may use \cref{2.15}(a) to write \(A = [\L_A]_{\beta}^{\gamma} = [\L_B]_{\beta}^{\gamma} = B\).
  Hence \(A = B\).
  The proof of the converse is trivial.
\end{proof}

\begin{proof}[\pf{2.15}(c)]
  For all \(x \in \vs{F}^n\), we have
  \begin{align*}
    \L_{A + B}(x) & = (A + B) x         &  & \text{(by \cref{2.3.8})}   \\
                  & = Ax + Bx           &  & \text{(by \cref{2.12}(a))} \\
                  & = \L_A(x) + \L_B(x) &  & \text{(by \cref{2.3.8})}   \\
                  & = (\L_A + \L_B)(x)  &  & \text{(by \cref{2.2.5})}
  \end{align*}
  and
  \begin{align*}
    \L_{aA}(x) & = (aA) x       &  & \text{(by \cref{2.3.8})}   \\
               & = a (Ax)       &  & \text{(by \cref{2.12}(b))} \\
               & = a \L_A(x)    &  & \text{(by \cref{2.3.8})}   \\
               & = (a \L_A)(x). &  & \text{(by \cref{2.2.5})}
  \end{align*}
  Thus \(\L_{A + B} = \L_A + \L_B\) and \(\L_{aA} = a \L_A\).
\end{proof}

\begin{proof}[\pf{2.15}(d)]
  Let \(C = [\T]_{\beta}^{\gamma}\).
  By \cref{2.14} we have \([\T(x)]_{\gamma} = [\T]_{\beta}^{\gamma} [x]_{\beta}\), or \(\T(x) = Cx = \L_C(x)\) for all \(x \in \vs{F}^n\).
  So \(\T = \L_C\).
  The uniqueness of \(C\) follows from \cref{2.15}(b).
\end{proof}

\begin{proof}[\pf{2.15}(e)]
  For any \(j\) (\(1 \leq j \leq p\)), we may apply \cref{2.13} several times to note that \((AE) e_j\) is the \(j\)th column of \(AE\) and that the \(j\)th column of \(AE\) is also equal to \(A (E e_j)\).
  So \((AE) e_j = A (Ee_j)\).
  Thus
  \[
    \L_{AE}(e_j) = (AE) e_j = A (E e_j) = \L_A(E e_j) = \L_A(L_E(e_j)).
  \]
  Hence \(\L_{AE} = \L_A \L_E\) by \cref{2.1.13}.
\end{proof}

\begin{proof}[\pf{2.15}(f)]
  For all \(x \in \vs{F}^n\), we have
  \begin{align*}
    \L_{I_n}(x) & = I_n x              &  & \text{(by \cref{2.3.8})}   \\
                & = x                  &  & \text{(by \cref{2.12}(c))} \\
                & = \IT_{\vs{F}^n}(x). &  & \text{(by \cref{2.1.9})}
  \end{align*}
  Thus \(\L_{I_n} = \IT_{\vs{F}^n}\).
\end{proof}

\begin{thm}\label{2.16}
  Let \(A\), \(B\), and \(C\) be matrices such that \(A (BC)\) is defined.
  Then \((AB) C\) is also defined and \(A (BC) = (AB) C\);
  that is, matrix multiplication is associative.
\end{thm}

\begin{proof}[\pf{2.16}]
  Since \(A (BC)\) is defined, by \cref{2.3.1} we can let \(A \in \MS\) and \(BC \in \ms{n}{p}{\F}\) such that \(A (BC) \in \ms{m}{p}{\F}\).
  Since \(BC\) is also defined, by \cref{2.3.1} again we can let \(B \in \ms{n}{k}{\F}\) and \(C \in \ms{k}{p}{\F}\).
  Then we have
  \begin{align*}
             & \begin{dcases}
                 A \in \MS           \\
                 B \in \ms{n}{k}{\F} \\
                 C \in \ms{k}{p}{\F}
               \end{dcases}                                    \\
    \implies & \begin{dcases}
                 AB \in \ms{m}{k}{\F} \\
                 C \in \ms{k}{p}{\F}
               \end{dcases}  &  & \text{(by \cref{2.3.1})}            \\
    \implies & (AB) C \in \ms{m}{p}{\F} &  & \text{(by \cref{2.3.1})}
  \end{align*}
  and thus \((AB) C\) is defined.

  Using \cref{2.15}(e) and the associativity of functional composition, we have
  \[
    \L_{A (BC)} = \L_A \L_{BC} = \L_A (\L_B \L_C) = (\L_A \L_B) \L_C = \L_{AB} \L_C = \L_{(AB) C}.
  \]
  So by \cref{2.15}(b), it follows that \(A (BC) = (AB) C\).
\end{proof}

\begin{defn}\label{2.3.9}
  An \textbf{incidence matrix} is a square matrix in which all the entries are either zero or one and, for convenience, all the diagonal entries are zero.
  If we have a relationship on a set of \(n\) objects that we denote by \(1, 2, \dots, n\), then we define the associated incidence matrix \(A\) by \(A_{i j} = 1\) if \(i\) is related to \(j\), and \(A_{i j} = 0\) otherwise.
\end{defn}

\begin{eg}\label{2.3.10}
  A maximal collection of three or more people with the property that any two can send to each other is called a \textbf{clique}.
  The problem of determining cliques is difficult, but there is a simple method for determining if someone belongs to a clique.
  If we define a new matrix \(B\) by \(B_{i j} = 1\) if \(i\) and \(j\) can send to each other, and \(B_{i j} = 0\) otherwise, then person \(i\) belongs to a clique iff \((B^3)_{i i} > 0\).
\end{eg}

\begin{proof}[\pf{2.3.10}]
  Suppose that \(B \in \ms{n}{n}{\F}\).
  We know that person \(i\) belongs to a clique iff there exists at least two other people \(j, k\) such that \(i\) can send to \(j\), \(j\) can send to \(k\) and \(k\) can send back to \(i\).
  In other words \(B_{i j} = B_{j k} = B_{k i} = 1\).
  Thus we have
  \begin{align*}
         & \exists j, k \in \set{1, \dots, n} \setminus \set{i} : B_{i j} = B_{j k} = B_{k i} = 1           &  & \text{(by \cref{2.3.10})} \\
    \iff & \exists j, k \in \set{1, \dots, n} \setminus \set{i} : B_{i j} B_{j k} B_{k i} = 1                                              \\
    \iff & (B^3)_{i i} = (BBB)_{i i}                                                                        &  & \text{(by \cref{2.3.6})}  \\
         & = \sum_{j = 1}^n B_{i j} (BB)_{j i} = \sum_{j = 1}^n B_{i j} \pa{\sum_{k = 1}^n B_{j k} B_{k i}} &  & \text{(by \cref{2.3.1})}  \\
         & = \sum_{j = 1}^n \sum_{k = 1}^n (B_{i j} B_{j k} B_{k i}) > 0.                                   &  & \text{(by \cref{1.2.1})}
  \end{align*}
\end{proof}

\begin{eg}\label{2.3.11}
  A relation among a group of people is called a \textbf{dominance relation} if the associated incidence matrix \(A\) has the property that for all distinct pairs \(i\) and \(j\), \(A_{i j} = 1\) iff \(A_{j i} = 0\), that is, given any two people, exactly one of them dominates the other.
  Since \(A\) is an incidence matrix, \(A_{i i} = 0\) for all \(i\).
  For such a relation, it can be shown that the matrix \(A + A^2\) has a row [column] in which each entry is positive except for the diagonal entry.
  In other words, there is at least one person who dominates [is dominated by] all others in one or two stages.
  In fact, it can be shown that any person who dominates [is dominated by] the greatest number of people in the first stage has this property.
\end{eg}

\begin{proof}[\pf{2.3.11}]
  Let \(A \in \ms{n}{n}{\F}\).
  First observe that
  \begin{align*}
    (A + A^2)_{i j} & = A_{i j} + (A^2)_{i j}                     &  & \text{(by \cref{1.2.9})} \\
                    & = A_{i j} + (AA)_{i j}                      &  & \text{(by \cref{2.3.6})} \\
                    & = A_{i j} + \sum_{k = 1}^n A_{i k} A_{k j}. &  & \text{(by \cref{2.3.1})}
  \end{align*}
  We see that all diagonal entries of \(A + A^2\) is \(0\) since
  \begin{align*}
             & \forall (i, j) \in \set{1, \dots, n}^2, (A_{i j} = 0) \lor (A_{j i} = 0)      &  & \text{(by \cref{2.3.11})}     \\
    \implies & \forall (i, j) \in \set{1, \dots, n}^2, A_{i j} A_{j i} = 0                                                      \\
    \implies & \forall i \in \set{1, \dots, n}, \sum_{j = 1}^n A_{i j} A_{j i} = 0                                              \\
    \implies & \forall i \in \set{1, \dots, n}, A_{i i} + \sum_{j = 1}^n A_{i j} A_{j i} = 0 &  & \text{(by \cref{2.3.9})}      \\
    \implies & \forall i \in \set{1, \dots, n}, (A + A^2)_{i i} = 0.                         &  & \text{(from the proof above)}
  \end{align*}

  Next we show that there exists an \(i \in \set{1, \dots, n}\) such that each entry in the \(i\)-th row of \(A + A^2\) is positive except for the diagonal entry.
  Let \(v_1 \in \set{1, \dots, n}\).
  If each entry in the \(v_1\)-th row of \(A + A^2\) is positive except for the diagonal entry, then we are done.
  If not, then there exists a \(v_2 \in \set{1, \dots, n} \setminus \set{v_1}\) such that \((A + A^2)_{v_1 v_2} = 0\).
  From the proof above we see that \(A_{v_1 v_2} = 0\) and thus by \cref{2.3.11} we must have \(A_{v_2 v_1} = 1\).
  We claim that if \(k \in \set{1, \dots, n} \setminus \set{v_1, v_2}\) such that \(A_{v_1 k} = 1\), then \(A_{v_2 k} = 1\).
  Otherwise we would have
  \begin{align*}
             & A_{v_2 k} = 0                                               \\
    \implies & A_{k v_2} = 1            &  & \text{(by \cref{2.3.11})}     \\
    \implies & A_{v_1 k} A_{k v_2} = 1                                     \\
    \implies & (A + A^2)_{v_1 v_2} > 0, &  & \text{(from the proof above)}
  \end{align*}
  a contradiction.
  From the claim above we see that the \(v_2\)-th row of \(A + A^2\) has greater number of positive entries than the \(v_1\)-th row of \(A + A^2\) (at least larger than one since \(A_{v_2 v_1} = 1\)).
  Now we can ask if each entry in the \(v_2\)-th row of \(A + A^2\) is positive except for the diagonal entry.
  If yes, then we are done.
  If not, we can find a \(v_3 \in \set{1, \dots, n} \setminus \set{v_1, v_2}\) such that \((A + A^2)_{v_2 v_3} = 0\).
  Using similar arguments as above we see that the \(v_3\)-th row of \(A + A^2\) has greater number of positive entries than the \(v_2\)-th row of \(A + A^2\) (at least larger than one since \(A_{v_3 v_2} = 1\)).
  Since there are only finitely many rows in \(A + A^2\), we see that by continuing the above process we can find an \(i \in \set{1, \dots, n}\) such that each entry in the \(i\)-th row of \(A + A^2\) is positive except for the diagonal entry.

  Finally we show that if \(i\) is the person who dominates the most in \(A\), then each entry in the \(i\)-th row of \(A + A^2\) is positive except for the diagonal entry.
  Suppose for sake of contradiction that there exists a \(j \in \set{1, \dots, n} \setminus \set{i}\) such that \((A + A^2)_{i j} = 0\).
  This means
  \begin{align*}
             & A_{i j} + \sum_{k = 1}^n A_{i k} A_{k j} = 0                                                            \\
    \implies & \begin{dcases}
                 A_{i j} = 0 \\
                 \sum_{k = 1}^n A_{i k} A_{k j} = 0
               \end{dcases}                                       &  & \text{(by \cref{2.3.11})}                       \\
    \implies & \forall k \in \set{1, \dots, n},                                                                        \\
             & (A_{i k} = 0 \land A_{k j} = 1) \lor (A_{i k} = 1 \land A_{k j} = 0)     &  & \text{(by \cref{2.3.9})}  \\
    \implies & \forall k \in \set{1, \dots, n},                                                                        \\
             & (A_{i k} = 0 \land A_{k j} = 1 \land A_{k i} = 1 \land A_{j k} = 0)                                     \\
             & \lor (A_{i k} = 1 \land A_{k j} = 0 \land A_{k i} = 0 \land A_{j k} = 1) &  & \text{(by \cref{2.3.11})} \\
    \implies & \sum_{k = 1}^n A_{i k} = \sum_{k = 1}^n A_{j k} > 0.                     &  & \text{(by \cref{2.3.11})}
  \end{align*}
  But this contradicts to the fact that \(i\) is the person who dominates the most in \(A\).
  Thus such \(j\) does not exist and each entry in the \(i\)-th row of \(A + A^2\) is positive except for the diagonal entry.
  One can replace the role of \(i\)-th row with \(j\)-th column and derive similar arguments.
\end{proof}

\exercisesection

\setcounter{ex}{9}
\begin{ex}\label{ex:2.3.10}
  Let \(A \in \ms{n}{n}{\F}\).
  Prove that \(A\) is a diagonal matrix iff \(A_{i j} = \delta_{i j} A_{i j}\) for all \(i\) and \(j\).
\end{ex}

\begin{proof}[\pf{ex:2.3.10}]
  We have
  \begin{align*}
         & A \text{ is diagonal matrix}                                                                                              \\
    \iff & A_{i j} = 0 \text{ for all } i, j \in \set{1, 2, \dots, n} \text{ and } i \neq j            &  & \text{(by \cref{1.3.8})} \\
    \iff & A_{i j} = \delta_{i j} \text{ for all } i, j \in \set{1, 2, \dots, n} \text{ and } i \neq j &  & \text{(by \cref{2.3.4})} \\
    \iff & A_{i j} = \delta_{i j} A_{i j} \text{ for all } i, j \in \set{1, 2, \dots, n}.              &  & \text{(by \cref{2.3.4})}
  \end{align*}
\end{proof}

\begin{ex}\label{ex:2.3.11}
  Let \(\V\) be a vector space over \(\F\), and let \(\T : \V \to \V\) be linear.
  Prove that \(\T^2 = \zT\) iff \(\rg{\T} \subseteq \ns{\T}\).
\end{ex}

\begin{proof}[\pf{ex:2.3.11}]
  We have
  \begin{align*}
         & \T^2 = \zT                                                                \\
    \iff & \forall x \in \V, \T(\T(x)) = \zT(x) = \zv &  & \text{(by \cref{2.1.9})}  \\
    \iff & \forall x \in \V, \T(x) \in \ns{\T}        &  & \text{(by \cref{2.1.10})} \\
    \iff & \rg{\T} \subseteq \ns{\T}.                 &  & \text{(by \cref{2.1.10})}
  \end{align*}
\end{proof}

\begin{ex}\label{ex:2.3.12}
  Let \(\V\), \(\W\), and \(\vs{Z}\) be vector spaces over \(\F\), and let \(\T \in \ls(\V, \W)\) and \(\U \in \ls(\W, \vs{Z})\).
  \begin{enumerate}
    \item Prove that if \(\U \T\) is one-to-one, then \(\T\) is one-to-one.
          Must \(\U\) also be one-to-one?
    \item Prove that if \(\U \T\) is onto, then \(\U\) is onto.
          Must \(\T\) also be onto?
    \item Prove that if \(\U\) and \(\T\) are one-to-one and onto, then \(\U \T\) is also.
  \end{enumerate}
\end{ex}

\begin{proof}[\pf{ex:2.3.12}(a)]
  Let \(x, y \in \V\) such that \(x \neq y\).
  Then we have
  \begin{align*}
             & (\U \T)(x) \neq (\U \T)(y) &  & \text{(\(\U \T\) is one-to-one)}            \\
    \implies & \U(\T(x)) \neq \U(\T(y))                                                    \\
    \implies & \T(x) \neq \T(y)           &  & \text{(this is the definition of function)} \\
    \implies & \T \text{ is one-to-one}.
  \end{align*}
  From the proof above we see that is doesn't matter whether \(\U\) is one-to-one or not.
\end{proof}

\begin{proof}[\pf{ex:2.3.12}(b)]
  Let \(z \in \vs{Z}\).
  Then we have
  \begin{align*}
             & \exists x \in \V : (\U \T)(x) = z                &  & \text{(\(\U \T\) is onto)} \\
    \implies & \exists (x, y) \in \V \times \W : \begin{dcases}
                                                   \T(x) = y \\
                                                   \U(\T(x)) = \U(y) = z
                                                 \end{dcases} &  & \text{(\(\U \T\) is onto)}   \\
    \implies & \U \text{ is onto}.
  \end{align*}
  From the proof above we see that is doesn't matter whether \(\T\) is onto or not.
\end{proof}

\begin{proof}[\pf{ex:2.3.12}(c)]
  First we show that \(\U \T\) is one-to-one.
  Let \(x, y \in \V\) such that \(x \neq y\).
  Then we have
  \begin{align*}
             & \T(x) \neq \T(y)             &  & \text{(\(\T\) is one-to-one)} \\
    \implies & \U(\T(x)) \neq \U(\T(y))     &  & \text{(\(\U\) is one-to-one)} \\
    \implies & \U \T \text{ is one-to-one}.
  \end{align*}

  Now we show that \(\U \T\) is onto.
  This is true since
  \begin{align*}
             & \begin{dcases}
                 \forall y \in \W, \exists x \in \V : \T(x) = y \\
                 \forall z \in \vs{Z}, \exists y \in \W : \U(y) = z
               \end{dcases}     &  & \text{(\(\U, \T\) are onto)}     \\
    \implies & \forall z \in \vs{Z}, \exists x \in \V : \U(\T(x)) = z \\
    \implies & \U \T \text{ is onto}.
  \end{align*}
\end{proof}

\begin{ex}\label{ex:2.3.13}
  Let \(A, B \in \ms{n}{n}{\F}\).
  Prove that \(\tr[AB] = \tr[BA]\) and \(\tr[A] = \tr[\tp{A}]\).
\end{ex}

\begin{proof}[\pf{ex:2.3.13}]
  We have
  \begin{align*}
    \tr[AB] & = \sum_{i = 1}^n (AB)_{i i}                          &  & \text{(by \cref{1.3.9})} \\
            & = \sum_{i = 1}^n \pa{\sum_{k = 1}^n A_{i k} B_{k i}} &  & \text{(by \cref{2.3.1})} \\
            & = \sum_{i = 1}^n \pa{\sum_{k = 1}^n B_{k i} A_{i k}} &  & \text{(by \cref{1.2.1})} \\
            & = \sum_{k = 1}^n \pa{\sum_{i = 1}^n B_{k i} A_{i k}} &  & \text{(by \cref{1.2.1})} \\
            & = \sum_{k = 1}^n (BA)_{k k}                          &  & \text{(by \cref{2.3.1})} \\
            & = \tr[BA]                                            &  & \text{(by \cref{1.3.9})}
  \end{align*}
  and
  \begin{align*}
    \tr[A] & = \sum_{i = 1}^n A_{i i}        &  & \text{(by \cref{1.3.9})} \\
           & = \sum_{i = 1}^n (\tp{A})_{i i} &  & \text{(by \cref{1.3.3})} \\
           & = \tr[\tp{A}].                  &  & \text{(by \cref{1.3.9})}
  \end{align*}
\end{proof}

\begin{ex}\label{ex:2.3.14}
  Assume the notation in \cref{2.13}.
  \begin{enumerate}
    \item Suppose that \(z\) is a (column) vector in \(\vs{F}^p\).
          Use \cref{2.13}(b) to prove that \(Bz\) is a linear combination of the columns of \(B\).
          In particular, if \(z = \tp{\tuple{a}{1,2,,p}}\), then show that
          \[
            Bz = \sum_{j = 1}^p a_j v_j.
          \]
    \item Extend (a) to prove that column \(j\) of \(AB\) is a linear combination of the columns of \(A\) with the coefficients in the linear combination being the entries of column \(j\) of \(B\).
    \item For any row vector \(w \in \vs{F}^m\), prove that \(wA\) is a linear combination of the rows of \(A\) with the coefficients in the linear combination being the coordinates of \(w\).
    \item Prove the analogous result to (b) about rows:
          Row \(i\) of \(AB\) is a linear combination of the rows of \(B\) with the coefficients in the linear combination being the entries of row \(i\) of \(A\).
  \end{enumerate}
\end{ex}

\begin{proof}[\pf{ex:2.3.14}(a)]
  We have
  \begin{align*}
    Bz & = \begin{pmatrix}
             (Bz)_{1 1} \\
             (Bz)_{2 1} \\
             \vdots     \\
             (Bz)_{n 1}
           \end{pmatrix} = \begin{pmatrix}
                             \sum_{k = 1}^p B_{1 k} z_{k 1} \\
                             \sum_{k = 1}^p B_{2 k} z_{k 1} \\
                             \vdots                         \\
                             \sum_{k = 1}^p B_{n k} z_{k 1}
                           \end{pmatrix}                      &  & \text{(by \cref{2.3.1})}                   \\
       & = \begin{pmatrix}
             \sum_{k = 1}^p B_{1 k} z_k \\
             \sum_{k = 1}^p B_{2 k} z_k \\
             \vdots                     \\
             \sum_{k = 1}^p B_{n k} z_k
           \end{pmatrix} = \begin{pmatrix}
                             \sum_{k = 1}^p B_{1 k} a_k \\
                             \sum_{k = 1}^p B_{2 k} a_k \\
                             \vdots                     \\
                             \sum_{k = 1}^p B_{n k} a_k
                           \end{pmatrix}                                                         \\
       & = \sum_{k = 1}^p \begin{pmatrix}
                            B_{1 k} a_k \\
                            B_{2 k} a_k \\
                            \vdots      \\
                            B_{n k} a_k
                          \end{pmatrix} = \sum_{k = 1}^p \pa{a_k \begin{pmatrix}
                                                                     B_{1 k} \\
                                                                     B_{2 k} \\
                                                                     \vdots  \\
                                                                     B_{n k}
                                                                   \end{pmatrix}} &  & \text{(by \cref{1.2.9})} \\
       & = \sum_{k = 1}^p a_k v_k.                              &  & \text{(by \cref{2.13})}
  \end{align*}
\end{proof}

\begin{proof}[\pf{ex:2.3.14}(b)]
  We have
  \begin{align*}
    u_j & = A v_j = A \begin{pmatrix}
                        B_{1 j} \\
                        B_{2 j} \\
                        \vdots  \\
                        B_{n j}
                      \end{pmatrix}              &  & \text{(by \cref{2.13}(a))}       \\
        & = \sum_{k = 1}^n B_{k j} \begin{pmatrix}
                                     A_{1 k} \\
                                     A_{2 k} \\
                                     \vdots  \\
                                     A_{n k}
                                   \end{pmatrix}. &  & \text{(by \cref{ex:2.3.14}(a))}
  \end{align*}
\end{proof}

\begin{proof}[\pf{ex:2.3.14}(c)]
  We have
  \begin{align*}
    wA & = \begin{pmatrix}
             (wA)_1 & (wA)_2 & \cdots & (wA)_n
           \end{pmatrix}                                                                         &  & \text{(by \cref{2.3.1})}                    \\
       & = \begin{pmatrix}
             (wA)_{1 1} & (wA)_{1 2} & \cdots & (wA)_{1 n}
           \end{pmatrix}                                                             &  & \text{(by \cref{2.3.1})}                                \\
       & = \begin{pmatrix}
             \sum_{k = 1}^m w_{1 k} A_{k 1} & \sum_{k = 1}^m w_{1 k} A_{k 2} & \cdots & \sum_{k = 1}^m w_{1 k} A_{k n}
           \end{pmatrix} &  & \text{(by \cref{2.3.1})}                              \\
       & = \sum_{k = 1}^m \begin{pmatrix}
                            w_{1 k} A_{k 1} & w_{1 k} A_{k 2} & \cdots & w_{1 k} A_{k n}
                          \end{pmatrix}                                              &  & \text{(by \cref{1.2.9})}                                \\
       & = \sum_{k = 1}^m \begin{pmatrix}
                            w_k A_{k 1} & w_k A_{k 2} & \cdots & w_k A_{k n}
                          \end{pmatrix}                                                          &  & \text{(by \cref{2.3.1})}                    \\
       & = \sum_{k = 1}^m \pa{w_k \begin{pmatrix}
                                      A_{k 1} & A_{k 2} & \cdots & A_{k n}
                                    \end{pmatrix}}.                                                                   &  & \text{(by \cref{1.2.9})} \\
  \end{align*}
\end{proof}

\begin{proof}[\pf{ex:2.3.14}(d)]
  We have
  \begin{align*}
     & \begin{pmatrix}
         (AB)_{i 1} & (AB)_{i 2} & \cdots & (AB)_{i p}
       \end{pmatrix}                                                                                            \\
     & = \begin{pmatrix}
           \sum_{k = 1}^n A_{i k} B_{k 1} & \sum_{k = 1}^n A_{i k} B_{k 2} & \cdots & \sum_{k = 1}^n A_{i k} B_{k p}
         \end{pmatrix} &  & \text{(by \cref{2.3.1})}                              \\
     & = \sum_{k = 1}^n \begin{pmatrix}
                          A_{i k} B_{k 1} & A_{i k} B_{k 2} & \cdots & A_{i k} B_{k p}
                        \end{pmatrix}                                              &  & \text{(by \cref{1.2.9})}                                \\
     & = \sum_{k = 1}^n \pa{A_{i k} \begin{pmatrix}
                                        B_{k 1} & B_{k 2} & \cdots & B_{k p}
                                      \end{pmatrix}}.                                                               &  & \text{(by \cref{1.2.9})}
  \end{align*}
\end{proof}

\begin{ex}\label{ex:2.3.15}
  Let \(M\) and \(A\) be matrices for which the product matrix \(MA\) is defined.
  If the \(j\)th column of \(A\) is a linear combination of a set of columns of \(A\), prove that the \(j\)th column of \(MA\) is a linear combination of the corresponding columns of \(MA\) with the same corresponding coefficients.
\end{ex}

\begin{proof}[\pf{ex:2.3.15}]
  Let \(M \in \MS\) and let \(A \in \ms{n}{p}{\F}\).
  For all \(i \in \set{1, 2, \dots, p}\) we define \(v_i\) to be the \(i\)th column of \(A\).
  By hypothesis we know that
  \[
    \exists \seq{c}{1,2,,p} \in \F : v_j = \sum_{i = 1}^p c_i v_i.
  \]
  Then we have
  \begin{align*}
    \begin{pmatrix}
      (MA)_{1 j} \\
      (MA)_{2 j} \\
      \vdots     \\
      (MA)_{p j}
    \end{pmatrix} & = M v_j                              &  & \text{(by \cref{2.13}(a))}   \\
                    & = M \pa{\sum_{i = 1}^p c_i v_i}                                      \\
                    & = \sum_{i = 1}^p M (c_i v_i)         &  & \text{(by \cref{2.12}(a))} \\
                    & = \sum_{i = 1}^p c_i (M v_i)         &  & \text{(by \cref{2.12}(b))} \\
                    & = \sum_{i = 1}^p c_i \begin{pmatrix}
                                             (MA)_{1 i} \\
                                             (MA)_{2 i} \\
                                             \vdots     \\
                                             (MA)_{p i}
                                           \end{pmatrix}. &  & \text{(by \cref{2.13}(b))}
  \end{align*}
\end{proof}

\begin{ex}\label{ex:2.3.16}
  Let \(V\) be a finite-dimensional vector space over \(\F\), and let \(\T : \V \to \V\) be linear.
  \begin{enumerate}
    \item If \(\rk{\T} = \rk{\T^2}\), prove that \(\rg{\T} \cap \ns{\T} = \set{\zv}\).
          Deduce that \(\V = \rg{\T} \oplus \ns{\T}\).
    \item Prove that \(\V = \rg{\T^k} \oplus \ns{\T^k}\) for some positive integer \(k\).
  \end{enumerate}
\end{ex}

\begin{proof}[\pf{ex:2.3.16}(a)]
  First observe that
  \begin{align*}
             & \begin{dcases}
                 \T(\V) \subseteq \V \\
                 \rk{\T^2} = \rk{\T}
               \end{dcases}                                                                                             \\
    \implies & \begin{dcases}
                 \rg{\T^2} = \T^{2}(\V) = \T(\T(\V)) \subseteq \T(\V) = \rg{\T} \\
                 \rk{\T^2} = \rk{\T}
               \end{dcases} &  & \text{(by \cref{2.1.10})}                                 \\
    \implies & \rg{\T^2} = \rg{\T}.                                                               &  & \text{(by \cref{1.11})}
  \end{align*}
  Let \(\beta = \set{\seq{v}{1,2,,n}}\) be a basis for \(\rg{\T}\) over \(\F\).
  Since \(\rg{\T^2} = \rg{\T}\), by \cref{2.2} we know that \(\T(\beta)\) is a basis for \(\rg{\T}\) over \(\F\).
  Let \(x \in \rg{\T} \cap \ns{\T}\).
  Since \(x \in \rg{\T}\), by \cref{1.6.1} we know that
  \[
    \exists \seq{a}{1,2,,n} \in \F : x = \sum_{i = 1}^n a_i v_i.
  \]
  Since \(x \in \ns{\T}\), we have
  \begin{align*}
             & \T(x) = \zv                                                   &  & \text{(by \cref{2.1.10})}   \\
    \implies & \T(\sum_{i = 1}^n a_i v_i) = \sum_{i = 1}^n a_i \T(v_i) = \zv &  & \text{(by \cref{2.1.2}(d))} \\
    \implies & \seq[=]{a}{1,2,,n} = 0                                        &  & \text{(by \cref{1.5.3})}    \\
    \implies & x = \zv.                                                      &  & \text{(by \cref{1.2}(a))}
  \end{align*}
  Thus \(\rg{\T} \cap \ns{\T} = \zv\).
  Since
  \begin{align*}
     & \dim(\rg{\T} + \ns{\T})                                                                        \\
     & = \dim(\rg{\T}) + \dim(\ns{\T}) - \dim(\rg{\T} \cap \ns{\T}) &  & \text{(by \cref{ex:1.6.29})} \\
     & = \dim(\rg{\T}) + \dim(\ns{\T})                              &  & \text{(by \cref{1.6.9})}     \\
     & = \rk{\T} + \nt{\T} = \dim(\V)                               &  & \text{(by \cref{2.3})}
  \end{align*}
  and by \cref{ex:1.3.23}(a) \(\rg{\T} + \ns{\T}\) is a subspace of \(\V\) over \(\F\), by \cref{1.11} we know that \(\V = \rg{\T} + \ns{\T}\).
  Thus by \cref{1.3.11} we know that \(\V = \rg{\T} \oplus \ns{\T}\).
\end{proof}

\begin{proof}[\pf{ex:2.3.16}(b)]
  First observe that
  \begin{align*}
             & \forall k \in \Z^+, \rg{\T^{k + 1}} = \T^{k + 1}(\V) \subseteq \T^k(\V) = \rg{\T^k} &  & \text{(by \cref{2.1.10})}   \\
    \implies & \forall k \in \Z^+, 0 \leq \rk{\T^{k + 1}} \leq \rk{\T^k} \leq \dim(\V).            &  & \text{(by \cref{1.11,2.3})}
  \end{align*}
  Since \(\V\) is finite-dimensional, we know that there must exists a \(k \in \Z^+\) such that \(\rk{\T^{k + 1}} = \rk{\T^k}\).
  By \cref{ex:2.3.16}(a) we see that \(\rg{\T^k} \cap \ns{\T^k} = \set{\zv}\) and \(\V = \rg{\T^k} \oplus \ns{\T^k}\).
\end{proof}

\begin{ex}\label{ex:2.3.17}
  Let \(\V\) be a vector space over \(\F\).
  Determine all linear transformations \(\T : \V \to \V\) such that \(\T = \T^2\).
\end{ex}

\begin{proof}[\pf{ex:2.3.17}]
  First observe that
  \[
    \forall x \in \V, x = \T(x) + (x - \T(x)).
  \]
  Suppose that \(\T = \T^2\).
  Then we have
  \begin{align*}
             & \forall x \in \V, \T(x) = \T(\T(x))                                                                                              \\
    \implies & \forall x \in \V,                                                                                                                \\
             & \begin{dcases}
                 T(x) \in \set{y \in \V : \T(y) = y} \\
                 \T(x - \T(x)) = \T(x) - \T(\T(x)) = \T(x) - \T(x) = \zv
               \end{dcases}                                       \\
    \implies & \forall x \in \V, \begin{dcases}
                                   T(x) \in \set{y \in \V : \T(y) = y} \\
                                   x - \T(x) \in \ns{\T}
                                 \end{dcases}                                                      &  & \text{(by \cref{2.1.10})}               \\
    \implies & \V \subseteq \set{y \in \V : \T(y) = y} + \ns{\T}                                           &  & \text{(by \cref{1.3.10})}       \\
    \implies & \V = \set{y \in \V : \T(y) = y} + \ns{\T}.                                                  &  & \text{(by \cref{ex:1.3.23}(a))}
  \end{align*}
  Thus
  \begin{align*}
             & \forall x \in \set{y \in \V : \T(y) = y} \cap \ns{\T}, \begin{dcases}
                                                                        \T(x) = x \\
                                                                        \T(x) = \zv
                                                                      \end{dcases} &  & \text{(by \cref{2.1.10})}         \\
    \implies & \forall x \in \set{y \in \V : \T(y) = y} \cap \ns{\T}, x = \zv                                             \\
    \implies & \set{y \in \V : \T(y) = y} \cap \ns{\T} = \set{\zv}                                                        \\
    \implies & \V = \set{y \in \V : \T(y) = y} \oplus \ns{\T}.                       &  & \text{(by \cref{ex:2.3.16}(a))}
  \end{align*}
\end{proof}

\begin{ex}\label{ex:2.3.18}
  Using only the definition of matrix multiplication, prove that multipli- cation of matrices is associative.
\end{ex}

\begin{proof}[\pf{ex:2.3.18}]
  Let \(A \in \MS\), \(B \in \ms{n}{p}{\F}\) and \(C \in \ms{p}{q}{\F}\).
  Then we have
  \begin{align*}
    ((AB) C)_{i j} & = \sum_{k_2 = 1}^p (AB)_{i k_2} C_{k_2 j}                                     &  & \text{(by \cref{2.3.1})} \\
                   & = \sum_{k_2 = 1}^p \pa{\pa{\sum_{k_1 = 1}^n A_{i k_1} B_{k_1 k_2}} C_{k_2 j}} &  & \text{(by \cref{2.3.1})} \\
                   & = \sum_{k_2 = 1}^p \pa{\sum_{k_1 = 1}^n (A_{i k_1} B_{k_1 k_2}) C_{k_2 j}}    &  & \text{(by \cref{1.2.1})} \\
                   & = \sum_{k_2 = 1}^p \pa{\sum_{k_1 = 1}^n A_{i k_1} (B_{k_1 k_2} C_{k_2 j})}    &  & \text{(by \cref{1.2.1})} \\
                   & = \sum_{k_1 = 1}^n \pa{\sum_{k_2 = 1}^p A_{i k_1} (B_{k_1 k_2} C_{k_2 j})}    &  & \text{(by \cref{1.2.1})} \\
                   & = \sum_{k_1 = 1}^n \pa{A_{i k_1} \pa{\sum_{k_2 = 1}^p B_{k_1 k_2} C_{k_2 j}}} &  & \text{(by \cref{1.2.1})} \\
                   & = \sum_{k_1 = 1}^n A_{i k_1} (BC)_{k_1 j}                                     &  & \text{(by \cref{2.3.1})} \\
                   & = (A (BC))_{i j}                                                              &  & \text{(by \cref{2.3.1})}
  \end{align*}
  and thus by \cref{1.2.9} \((AB) C = A (BC)\).
\end{proof}

\setcounter{ex}{22}
\begin{ex}\label{ex:2.3.23}
  Let \(A \in \ms{n}{n}{\F}\) be an incidence matrix that corresponds to a dominance relation.
  Determine the number of nonzero entries of \(A\).
\end{ex}

\begin{proof}[\pf{ex:2.3.23}]
  By \cref{2.3.11} we see that the number of nonzero entries is
  \[
    \frac{n (n + 1)}{2} - n = \frac{n (n - 1)}{2}.
  \]
\end{proof}

\section{Invertibility and Isomorphisms}\label{sec:2.4}

\begin{defn}\label{2.4.1}
  Let \(\V\) and \(\W\) be vector spaces over \(\F\), and let \(\T : \V \to \W\) be linear.
  A function \(\U : \W \to \V\) is said to be an \textbf{inverse} of \(\T\) if \(\T \U = \IT[\W]\) and \(\U \T = \IT[\V]\).
  If \(\T\) has an inverse, then \(\T\) is said to be \textbf{invertible}.
  If \(\T\) is invertible, then the inverse of \(\T\) is unique and is denoted by \(\T^{-1}\).
  The following facts hold for invertible functions T and U.
  \begin{itemize}
    \item \((\T \U)^{-1} = \U^{-1} \T^{-1}\).
    \item \((\T^{-1})^{-1} = \T\);
          in particular, \(\T^{-1}\) is invertible.
  \end{itemize}
  We often use the fact that a function is invertible iff it is both one-to-one and onto.
  We can therefore restate \cref{2.5} as follows.
  \begin{itemize}
    \item Let \(\T : \V \to \W\) be a linear transformation, where \(\V\) and \(\W\) are finite-dimensional spaces over \(\F\) of equal dimension.
          Then \(\T\) is invertible if and only if \(\rk{\T} = \dim(\V)\).
  \end{itemize}
\end{defn}

\begin{thm}\label{2.17}
  Let \(\V\) and \(\W\) be vector spaces over \(\F\), and let \(\T : \V \to \W\) be linear and invertible.
  Then \(\T^{-1} : \W \to \V\) is linear.
\end{thm}

\begin{proof}[\pf{2.17}]
  Let \(y_1, y_2 \in \W\) and \(c \in \F\).
  Since \(\T\) is onto and one-to-one, there exist unique vectors \(x_1\) and \(x_2\) such that \(\T(x_1) = y_1\) and \(\T(x_2) = y_2\).
  Thus \(x_1 = \T^{-1}(y_1)\) and \(x_2 = \T^{-1}(y_2)\);
  so
  \begin{align*}
    \T^{-1}(c y_1 + y_2) & = \T^{-1}(c \T(x_1) + \T(x_2))   &  & \text{(by \cref{2.4.1})}    \\
                         & = \T^{-1}(\T(c x_1 + x_2))       &  & \text{(by \cref{2.1.2}(b))} \\
                         & = c x_1 + x_2                    &  & \text{(by \cref{2.4.1})}    \\
                         & = c \T^{-1}(x_1) + \T^{-1}(x_2). &  & \text{(by \cref{2.4.1})}
  \end{align*}
\end{proof}

\begin{cor}\label{2.4.2}
  If \(\T\) is a linear transformation between vector spaces of equal (finite) dimension, then the conditions of being invertible, one-to-one, and onto are all equivalent.
\end{cor}

\begin{proof}[\pf{2.4.2}]
  By \cref{2.5} we see that this is true.
\end{proof}

\begin{defn}\label{2.4.3}
  Let \(A \in \ms{n}{n}{\F}\).
  Then \(A\) is \textbf{invertible} if there exists an \(B \in \ms{n}{n}{\F}\) such that \(AB = BA = I_n\).
\end{defn}

\begin{cor}\label{2.4.4}
  If \(A\) is invertible, then the matrix \(B\) such that \(AB = BA = I\) is unique.
  The matrix \(B\) is called the \textbf{inverse} of \(A\) and is denoted by \(A^{-1}\).
\end{cor}

\begin{proof}[\pf{2.4.4}]
  If \(C\) were another such matrix, then
  \[
    C = CI = C(AB) = (CA)B = IB = B.
  \]
  Thus \(B\) is unique.
\end{proof}

\begin{lem}\label{2.4.5}
  Let \(\T\) be an invertible linear transformation from \(\V\) to \(\W\).
  Then \(\V\) is finite-dimensional iff \(\W\) is finite-dimensional.
  In this case, \(\dim(\V) = \dim(\W)\).
\end{lem}

\begin{proof}[\pf{2.4.5}]
  Suppose that \(\V\) is finite-dimensional.
  Let \(\beta = \set{\seq{x}{1,,n}}\) be a basis for \(\V\) over \(\F\).
  By \cref{2.2} \(\T(\beta)\) spans \(\rg{\T} = \W\);
  hence \(\W\) is finite-dimensional by \cref{1.9}.
  Conversely, if \(\W\) is finite-dimensional, then so is \(\V\) by a similar argument, using \(\T^{-1}\).

  Now suppose that \(\V\) and \(\W\) are finite-dimensional.
  Because \(\T\) is one-to-one and onto, we have
  \[
    \nt{\T} = 0 \quad \text{and} \quad \rk{\T} = \dim(\rg{\T}) = \dim(\W).
  \]
  So by the dimension theorem (\cref{2.3}), it follows that \(\dim(\V) = \dim(\W)\).
\end{proof}

\begin{thm}\label{2.18}
  Let \(\V\) and \(\W\) be finite-dimensional vector spaces over \(\F\) with ordered bases \(\beta\) and \(\gamma\) over \(\F\), respectively.
  Let \(\T : \V \to \W\) be linear.
  Then \(\T\) is invertible iff \([\T]_{\beta}^{\gamma}\) is invertible.
  Furthermore, \([\T^{-1}]_{\gamma}^{\beta} = ([\T]_{\beta}^{\gamma})^{-1}\).
\end{thm}

\begin{proof}[\pf{2.18}]
  Suppose that \(\T\) is invertible.
  By \cref{2.4.5}, we have \(\dim(\V) = \dim(\W)\).
  Let \(n = \dim(\V)\).
  So \([\T]_{\beta}^{\gamma} \in \ms{n}{n}{\F}\).
  Now \(\T^{-1} : \W \to \V\) satisfies \(\T \T^{-1} = \IT[\W]\) and \(\T^{-1} \T = \IT[\V]\).
  Thus
  \[
    I_n = [\IT[\V]]_{\beta} = [\T^{-1} \T]_{\beta} = [\T^{-1}]_{\gamma}^{\beta} [\T]_{\beta}^{\gamma}.
  \]
  Similarly, \([\T]_{\beta}^{\gamma} [\T^{-1}]_{\gamma}^{\beta} = I_n\).
  So \([\T]_{\beta}^{\gamma}\) is invertible and \(\pa{[\T]_{\beta}^{\gamma}}^{-1} = [\T^{-1}]_{\gamma}^{\beta}\).

  Now suppose that \(A = [\T]_{\beta}^{\gamma}\) is invertible.
  Then there exists an \(B \in \ms{n}{n}{\F}\) such that \(AB = BA = I_n\).
  By \cref{2.6} there exists \(U \in \ls(\W, \V)\) such that
  \[
    \U(w_j) = \sum_{i = 1}^n B_{i j} v_i \quad \text{for } j \in \set{1, \dots, n},
  \]
  where \(\gamma = \set{\seq{w}{1,,n}}\) and \(\beta = \set{\seq{v}{1,,n}}\).
  It follows that \([\U]_{\gamma}^{\beta} = B\).
  To show that \(\U = \T^{-1}\), observe that
  \[
    [\U \T]_{\beta} = [\U]_{\gamma}^{\beta} [\T]_{\beta}^{\gamma} = BA = I_n = [\IT[\V]]_{\beta}
  \]
  by \cref{2.11}.
  So \(\U \T = \IT[\V]\), and similarly, \(\T \U = \IT[\W]\).
\end{proof}

\begin{cor}\label{2.4.6}
  Let \(\V\) be a finite-dimensional vector space over \(\F\) with an ordered basis \(\beta\), and let \(\T : \V \to \V\) be linear.
  Then \(\T\) is invertible iff \([\T]_{\beta}\) is invertible.
  Furthermore, \([\T^{-1}]_{\beta} = ([\T]_{\beta})^{-1}\).
\end{cor}

\begin{proof}[\pf{2.4.6}]
  This is done by \cref{2.18}.
\end{proof}

\begin{cor}\label{2.4.7}
  Let \(A \in \ms{n}{n}{\F}\).
  Then \(A\) is invertible iff \(\L_A\) is invertible.
  Furthermore, \((\L_A)^{-1} = \L_{A^{-1}}\).
\end{cor}

\begin{proof}[\pf{2.4.7}]
  By \cref{2.15} we know that \(\L_A\) is linear and \([\L_A]_{\beta} = A\) where \(\beta\) is the standard ordered basis for \(\vs{F}^n\) over \(\F\).
  Thus by \cref{2.4.7} \(A\) is invertible iff \([\L_A]_{\beta}\) is invertible iff \(\L_A\) is invertible.
  And we have \([\L_A^{-1}]_{\beta} = ([\L_A]_{\beta})^{-1} = A^{-1} = [\L_{A^{-1}}]_{\beta}\).
  By \cref{2.1.13} this means \((\L_A)^{-1} = \L_{A^{-1}}\).
\end{proof}

\begin{defn}\label{2.4.8}
  Let \(\V\) and \(\W\) be vector spaces over \(\F\).
  We say that \(\V\) is \textbf{isomorphic} to \(\W\) if there exists a linear transformation \(\T : \V \to \W\) that is invertible.
  Such a linear transformation is called an \textbf{isomorphism} from \(\V\) onto \(\W\).
\end{defn}

\begin{note}
  Since ``is isomorphic to'' is an equivalence relation.
  So we need only say that \(\V\) and \(\W\) are isomorphic.
\end{note}

\begin{thm}\label{2.19}
  Let \(\V\) and \(\W\) be finite-dimensional vector spaces (over the same field).
  Then \(\V\) is isomorphic to \(\W\) iff \(\dim(\V) = \dim(\W)\).
\end{thm}

\begin{proof}[\pf{2.19}]
  Suppose that \(\V\) is isomorphic to \(\W\) and that \(\T : \V \to \W\) is an isomorphism from \(\V\) to \(\W\).
  By \cref{2.4.5} we have that \(\dim(\V) = \dim(\W)\).

  Now suppose that \(\dim(\V) = \dim(\W)\), and let \(\beta = \set{\seq{v}{1,,n}}\) and \(\gamma = \set{\seq{w}{1,,n}}\) be bases for \(\V\) and \(\W\) over \(\F\), respectively.
  By \cref{2.6} there exists \(\T : \V \to \W\) such that \(\T\) is linear and \(\T(v_i) = w_i\) for \(i = 1, 2, \dots, n\).
  Using \cref{2.2} we have
  \[
    \rg{\T} = \spn{\T(\beta)} = \spn{\gamma} = \W.
  \]
  So \(\T\) is onto.
  From \cref{2.5} we have that \(\T\) is also one-to-one.
  Hence \(\T\) is an isomorphism.
\end{proof}

\begin{note}
  By \cref{2.4.5} if \(\V\) and \(\W\) are isomorphic, then either both of \(\V\) and \(\W\) are finite-dimensional or both are infinite-dimensional.
\end{note}

\begin{cor}\label{2.4.9}
  Let \(\V\) be a vector space over \(\F\).
  Then \(\V\) is isomorphic to \(\vs{F}^n\) iff \(\dim(\V) = n\).
\end{cor}

\begin{proof}[\pf{2.4.9}]
  We have
  \begin{align*}
         & \V \text{ is isomorphic to } \vs{F}^n &  & \text{(by \cref{2.19})}   \\
    \iff & \dim(\V) = \dim(\vs{F}^n) = n.        &  & \text{(by \cref{1.6.10})}
  \end{align*}
\end{proof}

\begin{thm}\label{2.20}
  Let \(\V\) and \(\W\) be finite-dimensional vector spaces over \(\F\) of dimensions \(n\) and \(m\), respectively, and let \(\beta\) and \(\gamma\) be ordered bases for \(\V\) and \(\W\) over \(\F\), respectively.
  Then the function \(\Phi : \ls(\V, \W) \to \MS\), defined by \(\Phi(\T) = [\T]_{\beta}^{\gamma}\) for \(\T \in \ls(\V, \W)\), is an isomorphism.
\end{thm}

\begin{proof}[\pf{2.20}]
  By \cref{2.8} \(\Phi\) is linear.
  Hence we must show that \(\Phi\) is one-to-one and onto.
  This is accomplished if we show that for every \(A \in \MS\), there exists a unique linear transformation \(\T : \V \to \W\) such that \(\Phi(\T) = A\).
  Let \(\beta = \set{\seq{v}{1,,n}}\), \(\gamma = \set{\seq{w}{1,,m}}\), and let \(A \in \MS\).
  By \cref{2.6} there exists a unique linear transformation \(\T : \V \to \W\) such that
  \[
    \T(v_j) = \sum_{i = 1}^m A_{i j} w_i \quad \text{for } 1 \leq j \leq n.
  \]
  But this means that \([\T]_{\beta}^{\gamma} = A\), or \(\Phi(\T) = A\).
  Thus \(\Phi\) is an isomorphism.
\end{proof}

\begin{cor}\label{2.4.10}
  Let \(\V\) and \(\W\) be finite-dimensional vector spaces over \(\F\) of dimensions \(n\) and \(m\), respectively.
  Then \(\ls(\V, \W)\) is finite-dimensional of dimension \(mn\).
\end{cor}

\begin{proof}[\pf{2.4.10}]
  The proof follows from \cref{2.20} and \cref{2.19} and the fact that \(\dim(\MS) = mn\).
\end{proof}

\section{The Change of Coordinate Matrix}\label{sec:2.5}

\begin{note}
	Geometrically, the change of variable
	\[
		\begin{pmatrix}
			x \\
			y
		\end{pmatrix} \to \begin{pmatrix}
			x' \\
			y'
		\end{pmatrix}
	\]
	is a change in the way that the position of a point \(P\) in the plane is described.
	This is done by introducing a new frame of reference, an \(x' y'\)-coordinate system with coordinate axes rotated from the original \(xy\)-coordinate axes.
\end{note}

\begin{thm}\label{2.22}
	Let \(\beta\) and \(\beta'\) be two ordered bases over \(\F\) for a finite-dimensional vector space \(\V\) over \(\F\), and let \(Q = [\IT[\V]]_{\beta'}^{\beta}\).
	Then
	\begin{enumerate}
		\item \(Q\) is invertible.
		\item For any \(v \in \V\), \([v]_{\beta} = Q [v]_{\beta'}\).
	\end{enumerate}
\end{thm}

\begin{proof}[\pf{2.22}(a)]
	Since \(\IT[\V]\) is invertible, \(Q\) is invertible by \cref{2.18}.
\end{proof}

\begin{proof}[\pf{2.22}(b)]
	For any \(v \in \V\),
	\[
		[v]_{\beta} = [\IT[\V](v)]_{\beta} = [\IT[\V]]_{\beta'}^{\beta} [v]_{\beta'} = Q [v]_{\beta'}
	\]
	by \cref{2.14}.
\end{proof}

\begin{defn}\label{2.5.1}
	The matrix \(Q = [\IT[\V]]_{\beta'}^{\beta}\) defined in \cref{2.22} is called a \textbf{change of coordinate matrix}.
	Because of \cref{2.22}(b), we say that \(Q\) \textbf{changes \(\beta'\)-coordinates into \(\beta\)-coordinates}.
	Observe that if \(\beta = \set{\seq{x}{1,2,,n}}\) and \(\beta' = \set{x_1', x_2', \dots, x_n'}\), then
	\[
		x_j' = \sum_{i = 1}^n Q_{i j} x_i
	\]
	for \(j \in \set{1, 2, \dots, n}\);
	that is, the \(j\)th column of \(Q\) is \([x_j']_{\beta}\).
\end{defn}

\begin{note}
	If \(Q\) changes \(\beta'\)-coordinates into \(\beta\)-coordinates, then \(Q^{-1}\) changes \(\beta\)-coordinates into \(\beta'\)-coordinates
	(See \cref{ex:2.5.11}).
\end{note}

\begin{defn}\label{2.5.2}
	For the remainder of this section, we consider only linear transformations that map a vector space \(\V\) over \(\F\) into itself.
	Such a linear transformation is called a \textbf{linear operator} on \(\V\) over \(\F\).
	Suppose now that \(\T\) is a linear operator on a finite-dimensional vector space \(\V\) over \(\F\) and that \(\beta\) and \(\beta'\) are ordered bases for \(\V\) over \(\F\).
	Then \(\V\) can be represented by the matrices \([\T]_{\beta}\) and \([\T]_{\beta'}\)
	(See \cref{2.2.4}).
\end{defn}

\begin{thm}\label{2.23}
	Let \(\T\) be a linear operator on a finite-dimensional vector space \(\V\) over \(\F\), and let \(\beta\) and \(\beta'\) be ordered bases for \(\V\) over \(\F\).
	Suppose that \(Q\) is the change of coordinate matrix that changes \(\beta'\)-coordinates into \(\beta\)-coordinates.
	Then
	\[
		[\T]_{\beta'} = Q^{-1} [\T]_{\beta} Q.
	\]
\end{thm}

\begin{proof}[\pf{2.23}]
	Let \(\IT[\V]\) be the identity transformation on \(\V\).
	Then \(\T = \IT[\V] \T = \T \IT[\V]\);
	hence, by \cref{2.11},
	\[
		Q [\T]_{\beta'} = [\IT[\V]]_{\beta'}^{\beta} [\T]_{\beta'}^{\beta'} = [\IT[\V] \T]_{\beta'}^{\beta} = [\T \IT[\V]]_{\beta'}^{\beta} = [\T]_{\beta}^{\beta} [\IT[\V]]_{\beta'}^{\beta} = [\T] Q.
	\]
	Therefore \([\T]_{\beta'} = Q^{-1} [\T]_{\beta} Q\).
\end{proof}

\begin{cor}\label{2.5.3}
	Let \(A \in \ms[n][n][\F]\), and let \(\gamma\) be an ordered basis for \(\vs{F}^n\) over \(\F\).
	Then \([\L_A]_{\gamma} = Q^{-1} A Q\), where \(Q \in \ms[n][n][\F]\) and the \(j\)th column of \(Q\) is the \(j\)th vector of \(\gamma\).
\end{cor}

\begin{proof}[\pf{2.5.3}]
	Let \(\gamma = \set{\seq{v}{1,,n}}\) and let \(\beta = \set{\seq{e}{1,,n}}\) be the standard ordered basis for \(\vs{F}^n\) over \(\F\).
	Then we have
	\begin{align*}
		         & \forall j \in \set{1, \dots, n}, v_j = \begin{pmatrix}
			                                                  Q_{1 j} \\
			                                                  \vdots  \\
			                                                  Q_{n j}
		                                                  \end{pmatrix} = \sum_{i = 1}^n Q_{i j} e_i &  & \by{1.6.3} \\
		\implies & Q = [\IT_{\vs{F}^n}]_{\gamma}^{\beta}                  &  & \by{2.5.1}                            \\
		\implies & [\L_A]_{\gamma} = Q^{-1} [\L_A]_{\beta} Q              &  & \by{2.23}                             \\
		         & = Q^{-1} A Q.                                          &  & \by{2.15}[a]
	\end{align*}
\end{proof}

\begin{defn}\label{2.5.4}
	Let \(A, B \in \ms[n][n][\F]\).
	We say that \(B\) is \textbf{similar} to \(A\) if there exists an invertible matrix \(Q\) such that \(B = Q^{-1} A Q\).
\end{defn}

\begin{note}
	Observe that the relation of similarity is an equivalence relation
	(see \cref{ex:2.5.9}).
	So we need only say that \(A\) and \(B\) are similar.
\end{note}

\begin{note}
	In term of \cref{2.5.4}, \cref{2.23} can be stated as follows:
	If \(\T\) is a linear operator on a finite-dimensional vector space \(\V\) over \(\F\), and if \(\beta\) and \(\beta'\) are any ordered bases for \(\V\) over \(\F\), then \([\T]_{\beta}\) and \([\T]_{\beta'}\) are similar.
\end{note}

\begin{note}
	\cref{2.23} can be generalized to allow \(\T \in \ls(\V, \W)\), where \(\V, \W\) are vector spaces over \(\F\) and \(\V\) is distinct from \(\W\).
	In this case, we can change bases in \(\V\) as well as in \(\W\)
	(see \cref{ex:2.5.8}).
\end{note}

\exercisesection

\setcounter{ex}{7}
\begin{ex}\label{ex:2.5.8}
	Prove the following generalization of \cref{2.23}.
	Let \(\T \in \ls(\V, \W)\) where \(\V, \W\) are finite-dimensional vector spaces over \(\F\).
	Let \(\beta\) and \(\beta'\) be ordered bases for \(\V\) over \(\F\), and let \(\gamma\) and \(\gamma'\) be ordered bases for \(\W\) over \(\F\).
	Then \([\T]_{\beta'}^{\gamma'} = P^{-1} [\T]_{\beta}^{\gamma} Q\), where \(Q\) is the matrix that changes \(\beta'\)-coordinates into \(\beta\)-coordinates and \(P\) is the matrix that changes \(\gamma'\)-coordinates into \(\gamma\)-coordinates.
\end{ex}

\begin{proof}[\pf{ex:2.5.8}]
	We have
	\begin{align*}
		P [\T]_{\beta'}^{\gamma'} & = [\IT[\W]]_{\gamma'}^{\gamma} [\T]_{\beta'}^{\gamma'} &  & \by{2.5.1} \\
		                          & = [\IT[\W] \T]_{\beta'}^{\gamma}                       &  & \by{2.11}  \\
		                          & = [\T]_{\beta'}^{\gamma}                               &  & \by{2.1.9} \\
		                          & = [\T \IT[\V]]_{\beta'}^{\gamma}                       &  & \by{2.1.9} \\
		                          & = [\T]_{\beta}^{\gamma} [\IT[\V]]_{\beta'}^{\beta}     &  & \by{2.11}  \\
		                          & = [\T]_{\beta}^{\gamma} Q                              &  & \by{2.5.1}
	\end{align*}
	and thus
	\begin{align*}
		         & P [\T]_{\beta'}^{\gamma'} = [\T]_{\beta}^{\gamma} Q                                 \\
		\implies & P^{-1} P [\T]_{\beta'}^{\gamma'} = P^{-1} [\T]_{\beta}^{\gamma} Q &  & \by{2.22}[a] \\
		\implies & [\T]_{\beta'}^{\gamma'} = P^{-1} [\T]_{\beta}^{\gamma} Q.         &  & \by{2.4.3}
	\end{align*}
\end{proof}

\begin{ex}\label{ex:2.5.9}
	Prove that ``is similar to'' is an equivalence relation on \(\ms[n][n][\F]\).
\end{ex}

\begin{proof}[\pf{ex:2.5.9}]
	Let \(A, B, C \in \ms[n][n][\F]\).
	\begin{description}
		\item[For reflexive:]
			We have
			\[
				A = I_n A I_n = I_n^{-1} A I_n.
			\]
		\item[For symmetric:]
			Suppose that \(B = Q^{-1} A Q\).
			Let \(P = Q^{-1}\).
			Then we have
			\begin{align*}
				     & B = Q^{-1} A Q                                                   \\
				\iff & Q B Q^{-1} = Q Q^{-1} A Q Q^{-1} = I_n A I_n = A &  & \by{2.4.3} \\
				\iff & P^{-1} B P = A.                                  &  & \by{2.4.3}
			\end{align*}
		\item[For transitive:]
			Suppose that \(B = Q^{-1} A Q\) and \(C = P^{-1} B P\).
			Then we have
			\begin{align*}
				C & = P^{-1} B P                             \\
				  & = P^{-1} Q^{-1} A Q P                    \\
				  & = (QP)^{-1} A (QP).   &  & \by{ex:2.4.4}
			\end{align*}
	\end{description}
	From all cases above we see that ``is similar to'' defined in \cref{2.5.4} is an equivalence relation on \(\ms[n][n][\F]\).
\end{proof}

\begin{ex}\label{ex:2.5.10}
	Prove that if \(A, B \in \ms[n][n][\F]\) and \(A, B\) are similar, then \(\tr[A] = \tr[B]\).
\end{ex}

\begin{proof}[\pf{ex:2.5.10}]
	By \cref{2.5.4} we know that there exists a \(Q \in \ms[n][n][\F]\) such that \(B = Q^{-1} A Q\).
	Thus we have
	\begin{align*}
		\tr[B] & = \tr[Q^{-1} A Q]                       \\
		       & = \tr[(Q^{-1} A) Q] &  & \by{2.16}      \\
		       & = \tr[Q (Q^{-1} A)] &  & \by{ex:2.3.13} \\
		       & = \tr[(Q Q^{-1}) A] &  & \by{2.16}      \\
		       & = \tr[I_n A]        &  & \by{2.4.3}     \\
		       & = \tr[A].           &  & \by{2.4.3}
	\end{align*}
\end{proof}

\begin{ex}\label{ex:2.5.11}
	Let \(\V\) be a finite-dimensional vector space over \(\F\) with ordered bases \(\alpha\), \(\beta\), and \(\gamma\).
	\begin{enumerate}
		\item Prove that if \(Q\) and \(R\) are the change of coordinate matrices that change \(\alpha\)-coordinates into \(\beta\)-coordinates and \(\beta\)-coordinates into \(\gamma\)-coordinates, respectively, then \(RQ\) is the change of coordinate matrix that changes \(\alpha\)-coordinates into \(\gamma\)-coordinates.
		\item Prove that if \(Q\) changes \(\alpha\)-coordinates into \(\beta\)-coordinates, then \(Q^{-1}\) changes \(\beta\)-coordinates into \(\alpha\)-coordinates.
	\end{enumerate}
\end{ex}

\begin{proof}[\pf{ex:2.5.11}(a)]
	We have
	\begin{align*}
		QR & = [\IT[\V]]_{\alpha}^{\beta} [\IT[\V]]_{\beta}^{\gamma} &  & \by{2.5.1} \\
		   & = [\IT[\V] \IT[\V]]_{\alpha}^{\gamma}                   &  & \by{2.11}  \\
		   & = [\IT[\V]]_{\alpha}^{\gamma}                           &  & \by{2.1.9}
	\end{align*}
	and thus by \cref{2.5.1} \(QR\) is the change of coordinate matrix that changes \(\alpha\)-coordinates into \(\gamma\)-coordinates.
\end{proof}

\begin{proof}[\pf{ex:2.5.11}(b)]
	We have
	\begin{align*}
		Q^{-1} & = \pa{[\IT[\V]]_{\alpha}^{\beta}}^{-1} &  & \by{2.5.1} \\
		       & = [\IT[\V]^{-1}]_{\beta}^{\alpha}      &  & \by{2.18}  \\
		       & = [\IT[\V]]_{\beta}^{\alpha}.          &  & \by{2.1.9}
	\end{align*}
\end{proof}

\setcounter{ex}{12}
\begin{ex}\label{ex:2.5.13}
	Let \(\V\) be a finite-dimensional vector space over a field \(\F\), and let \(\beta = \set{\seq{x}{1,2,,n}}\) be an ordered basis for \(\V\) over \(\F\).
	Let \(Q \in \ms[n][n][\F]\) and \(Q\) is invertible.
	Define
	\[
		x_j' = \sum_{i = 1}^n Q_{i j} x_i \quad \text{for } 1 \leq j \leq n,
	\]
	and set \(\beta' = \set{x_1', x_2', \dots, x_n'}\).
	Prove that \(\beta'\) is a basis for \(\V\) over \(\F\) and hence that \(Q\) is the change of coordinate matrix changing \(\beta'\)-coordinates into \(\beta\)-coordinates.
\end{ex}

\begin{proof}[\pf{ex:2.5.13}]
	Let \(a = \tp{\tuple{a}{1,,n}} \in \vs{F}^n\) such that
	\[
		\sum_{j = 1}^n a_j x_j' = \zv.
	\]
	Since
	\begin{align*}
		         & \sum_{j = 1}^n a_j x_j' = \sum_{j = 1}^n \pa{a_j \sum_{i = 1}^n Q_{i j} x_i}                                                             \\
		         & = \sum_{j = 1}^n \pa{\sum_{i = 1}^n a_j Q_{i j} x_i} = \sum_{i = 1}^n \pa{\sum_{j = 1}^n a_j Q_{i j}} x_i = \zv &  & \by{1.2.1}          \\
		\implies & \forall i \in \set{1, \dots, n}, \sum_{j = 1}^n a_j Q_{i j} = 0                                                 &  & \by{1.6.1}          \\
		\implies & Qa = \begin{pmatrix}
			                Q_{1 1} & \cdots & Q_{1 n} \\
			                \vdots  & \ddots & \vdots  \\
			                Q_{n 1} & \cdots & Q_{n n}
		                \end{pmatrix} \cdot \begin{pmatrix}
			                                    a_1    \\
			                                    \vdots \\
			                                    a_n
		                                    \end{pmatrix} = \zm                                                                             &  & \by{2.3.1} \\
		\implies & a = \zv,                                                                                                        &  & \by{ex:2.4.6}
	\end{align*}
	by \cref{1.5.3} we know that \(\beta'\) is linearly independent.
	Thus by \cref{1.6.15}(b) \(\beta'\) is a basis for \(\V\) over \(\F\).
	We conclude by \cref{2.5.1} that \(Q\) is the change of coordinate matrix changin \(\beta'\)-coordinates into \(\beta\)-coordinates.
\end{proof}

\begin{ex}\label{ex:2.5.14}
	Prove the converse of \cref{ex:2.5.8}:
	If \(A, B \in \ms\), and if there exist invertible matrices \(P \in \ms[m][m][\F]\) and \(Q \in \ms[n][n][\F]\) such that \(B = P^{-1} A Q\), then there exist an \(n\)-dimensional vector space \(\V\) and an \(m\)-dimensional vector space \(\W\) (both over \(\F\)), ordered bases \(\beta, \beta'\) for \(\V\) and \(\gamma, \gamma'\) for \(\W\) (all over \(\F\)), and a \(\T \in \ls(\V, \W)\) such that
	\[
		A = [\T]_{\beta}^{\gamma} \quad \text{and} \quad B = [\T]_{\beta'}^{\gamma'}.
	\]
\end{ex}

\begin{proof}[\pf{ex:2.5.14}]
	Let \(\beta\) and \(\gamma\) be the standard ordered bases for \(\vs{F}^n\) and \(\vs{F}^m\) over \(\F\), respectively.
	We define \(\beta'\) and \(\gamma'\) as follow:
	\begin{align*}
		 & \beta' = \set{\sum_{i = 1}^n \pa{Q^{-1}}_{i j} e_i : (j \in \set{1, \dots, n}) \land (e_i \in \beta)};   \\
		 & \gamma' = \set{\sum_{i = 1}^n \pa{P}^{-1}_{i j} e_i : (j \in \set{1, \dots, m}) \land (e_i \in \gamma)}.
	\end{align*}
	By \cref{ex:2.5.13} we see that \(\beta'\) and \(\gamma'\) are bases of \(\vs{F}^n\) and \(\vs{F}^m\) over \(\F\), respectively.
	By \cref{2.5.1} we see that \(Q^{-1}\) changes \(\beta\)-coordinates into \(\beta'\)-coordinates and \(P^{-1}\) changes \(\gamma\)-coordinates into \(\gamma'\)-coordinates.
	By \cref{ex:2.5.11}(b) we see that \(Q\) changes \(\beta'\)-coordinates into \(\beta\)-coordinates.
	If we set \(\V = \vs{F}^n\), \(\W = \vs{F}^m\) and \(\T = \L_A\), then we have
	\begin{align*}
		B & = P^{-1} A Q                                                                                  &  & \text{(by hypothesis)}         \\
		  & = [\IT[\vs{F}^m]]_{\gamma}^{\gamma'} A [\IT[\vs{F}^n]]_{\beta'}^{\beta}                       &  & \text{(from the proofs above)} \\
		  & = [\IT[\vs{F}^m]]_{\gamma}^{\gamma'} [\L_A]_{\beta}^{\gamma} [\IT[\vs{F}^n]]_{\beta'}^{\beta} &  & \by{2.15}[a]                   \\
		  & = [\IT[\vs{F}^m]]_{\gamma}^{\gamma'} [\T]_{\beta}^{\gamma} [\IT[\vs{F}^n]]_{\beta'}^{\beta}                                       \\
		  & = [\IT[\vs{F}^m] \T \IT[\vs{F}^n]]_{\beta'}^{\gamma'}                                         &  & \by{2.11}                      \\
		  & = [\T]_{\beta'}^{\gamma'}.                                                                    &  & \by{2.1.9}
	\end{align*}
\end{proof}

\section{Dual Spaces}\label{sec:2.6}

\begin{defn}\label{2.6.1}
  In this section, we are concerned exclusively with linear transformations from a vector space \(\V\) into its field of scalars \(\F\), which is itself a vector space of dimension \(1\) over \(\F\).
  Such a linear transformation is called a \textbf{linear functional} on \(\V\).
\end{defn}

\begin{eg}\label{2.6.2}
  Let \(\V\) be the vector space of continuous real-valued functions on the interval \([0, 2\pi]\).
  Fix a function \(g \in \V\).
  The function \(h : \V \to \R\) defined by
  \[
    h(x) = \frac{1}{2\pi} \int_{0}^{2\pi} x(t) g(t) \; dt
  \]
  is a linear functional on \(\V\).
  In the cases that \(g(t)\) equals \(\sin(nt)\) or \(\cos(nt)\), \(h(x)\) is often called the \textbf{\(n\)th Fourier coefficient of \(x\)}.
\end{eg}

\begin{proof}[\pf{2.6.2}]
  Let \(f_1, f_2 \in \V\) and let \(c \in \R\).
  Then we have
  \begin{align*}
    h(cf_1 + f_2) & = \frac{1}{2\pi} \int_{0}^{2\pi} (cf_1 + f_2)(t) g(t) \; dt                                           &  & \text{(by \cref{2.6.2})} \\
                  & = \frac{1}{2\pi} \int_{0}^{2\pi} cf_1(t) g(t) + f_2(t) g(t) \; dt                                                                   \\
                  & = \frac{c}{2\pi} \int_{0}^{2\pi} f_1(t) g(t) \; dt + \frac{1}{2\pi} \int_{0}^{2\pi} f_2(t) g(t) \; dt                               \\
                  & = c h(f_1) + h(f_2)                                                                                   &  & \text{(by \cref{2.6.2})}
  \end{align*}
  and thus by \cref{2.1.2}(b) \(h \in \ls(\V, \F)\).
\end{proof}

\begin{eg}\label{2.6.3}
  The trace function \(\tr : \ms{n}{n}{\F} \to \F\) is a linear functional.
\end{eg}

\begin{proof}[\pf{2.6.3}]
  By \cref{ex:1.3.6} we see that this is true.
\end{proof}

\begin{eg}\label{2.6.4}
  Let \(\V\) be a finite-dimensional vector space over \(\F\), and let \(\beta = \set{\seq{x}{1,,n}}\) be an ordered basis for \(\V\) over \(\F\).
  For each \(i \in \set{1, \dots, n}\), define \(f_i(x) = a_i\), where
  \[
    [x]_{\beta} = \begin{pmatrix}
      a_1    \\
      \vdots \\
      a_n
    \end{pmatrix}
  \]
  is the coordinate vector of \(x\) relative to \(\beta\).
  Then \(f_i\) is a linear functional on \(\V\) called the \textbf{\(i\)th coordinate function with respect to the basis \(\beta\)}.
  Note that \(f_i(x_j) = \delta_{i j}\), where \(\delta_{i j}\) is the Kronecker delta.
  These linear functionals play an important role in the theory of dual spaces (see \cref{2.24}).
\end{eg}

\begin{proof}[\pf{2.6.4}]
  Let \(a, b \in \V\) and let \(c \in \F\).
  By \cref{1.8} there exist some \(\seq{a}{1,,n}, \seq{b}{1,,n} \in \F\) such that
  \[
    a = \sum_{j = 1}^n a_j x_j \quad \text{and} \quad b = \sum_{j = 1}^n b_j x_j.
  \]
  Then we have
  \begin{align*}
    f_i(ca + b) & = f_i\pa{c \pa{\sum_{j = 1}^n a_j x_j} + \sum_{j = 1}^n b_j x_j}                               \\
                & = f_i\pa{\sum_{j = 1}^n (c a_j + b_j) x_j}                       &  & \text{(by \cref{1.2.1})} \\
                & = c a_i + b_i                                                    &  & \text{(by \cref{2.6.4})} \\
                & = c f_i(a) + f_i(b)                                              &  & \text{(by \cref{2.6.4})}
  \end{align*}
  and thus by \cref{2.1.2}(b) \(f_i \in \ls(\V, \F)\).
\end{proof}

\begin{defn}\label{2.6.5}
  For a vector space \(\V\) over \(\F\), we define the \textbf{dual space} of \(\V\) to be the vector space \(\ls(\V, \F)\), denoted by \(\V^*\).
  Thus \(\V^*\) is the vector space consisting of all linear functionals on \(\V\) with the operations of addition and scalar multiplication as defined in \cref{sec:2.2}.
  Note that if \(\V\) is finite-dimensional, then by the \cref{2.4.10}
  \[
    \dim(\V^*) = \dim(\ls(\V, \F)) = \dim(\V) \cdot \dim(\F) = \dim(\V).
  \]
  Hence by \cref{2.19} \(\V\) and \(\V^*\) are isomorphic.
  We also define the \textbf{double dual} \(\V^{**}\) of \(\V\) to be the dual of \(\V^*\).
  In \cref{2.26}, we show, in fact, that there is a natural identification of \(\V\) and \(\V^{**}\) in the case that \(\V\) is finite-dimensional.
\end{defn}

\begin{thm}\label{2.24}
  Suppose that \(\V\) is a finite-dimensional vector space over \(\F\) with the ordered basis \(\beta = \set{\seq{x}{1,,n}}\).
  Let \(f_i\) (\(1 \leq i \leq n\)) be the \(i\)th coordinate function with respect to \(\beta\) as defined in \cref{2.6.4}, and let \(\beta^* = \set{\seq{f}{1,,n}}\).
  Then \(\beta^*\) is an ordered basis for \(\V^*\), and, for any \(g \in \V^*\), we have
  \[
    g = \sum_{i = 1}^n g(x_i) f_i.
  \]
\end{thm}

\begin{proof}[\pf{2.24}]
  Let \(g \in \V^*\).
  Since \(\dim(\V^*) = n\), we need only show that
  \[
    g = \sum_{i = 1}^n g(x_i) f_i,
  \]
  from which it follows that \(\beta^*\) generates \(\V^*\), and hence is a basis by \cref{1.6.15}(a).
  Let
  \[
    h = \sum_{i = 1}^n g(x_i) f_i.
  \]
  For \(1 \leq j \leq n\), we have
  \begin{align*}
    h(x_j) & = \pa{\sum_{i = 1}^n g(x_i) f_i}(x_j)                               \\
           & = \sum_{i = 1}^n g(x_i) f_i(x_j)      &  & \text{(by \cref{2.2.5})} \\
           & = \sum_{i = 1}^n g(x_i) \delta_{i j}  &  & \text{(by \cref{2.6.4})} \\
           & = g(x_j).
  \end{align*}
  Therefore \(g = h\) by \cref{2.1.13}.
\end{proof}

\begin{defn}\label{2.6.6}
  Using the notation of \cref{2.24}, we call the ordered basis \(\beta^* = \set{\seq{f}{1,,n}}\) of \(\V^*\) over \(\F\) that satisfies \(f_i(x_j) = \delta_{i j}\) (\(1 \leq i, j \leq n\)) the \textbf{dual basis} of \(\beta\).
\end{defn}

\begin{note}
  We now assume that \(\V\) and \(\W\) are finite-dimensional vector spaces over \(\F\) with ordered bases \(\beta\) and \(\gamma\) over \(\F\), respectively.
  In \cref{sec:2.4}, we proved that there is a one-to-one correspondence between linear transformations \(\T : \V \to \W\) and \(m \times n\) matrices (over \(\F\)) via the correspondence \(\T \leftrightarrow [\T]_{\beta}^{\gamma}\).
  For a matrix of the form \(A = [\T]_{\beta}^{\gamma}\), the question arises as to whether or not there exists a linear transformation \(\U\) associated with \(\T\) in some natural way such that \(\U\) may be represented in some basis as \(\tp{A}\).
  Of course, if \(m \neq n\), it would be impossible for \(\U\) to be a linear transformation from \(\V\) into \(\W\).
  We now answer this question by applying what we have already learned about dual spaces.
\end{note}

\begin{thm}\label{2.25}
  Let \(\V\) and \(\W\) be finite-dimensional vector spaces over \(\F\) with ordered bases \(\beta\) and \(\gamma\), respectively.
  For any \(\T \in \ls(\V, \W)\), the mapping \(\tp{\T} : \W^* \to \V^*\) defined by \(\tp{\T}(g) = g \T\) for all \(g \in \W^*\) is a linear transformation with the property that \([\tp{\T}]_{\gamma^*}^{\beta^*} = \tp{\pa{[\T]_{\beta}^{\gamma}}}\).
\end{thm}

\begin{proof}[\pf{2.25}]
  For \(g \in \W^*\), it is clear that \(\tp{\T}(g) = g \T\) is a linear functional on \(\V\) and hence is in \(\V^*\) (see \cref{2.9}).
  Thus \(\tp{\T}\) maps \(\W^*\) into \(\V^*\).
  Now we show that \(\tp{\T} \in \ls(\W^*, \V^*)\).
  Let \(x, y \in \W^*\) and let \(c \in \F\).
  Since
  \begin{align*}
    \tp{\T}(cx + y) & = (cx + y) \T                &  & \text{(by \cref{2.25})}  \\
                    & = c x \T + y \T              &  & \text{(by \cref{2.2.5})} \\
                    & = c \tp{\T}(x) + \tp{\T}(y), &  & \text{(by \cref{2.25})}
  \end{align*}
  by \cref{2.1.2} we see that \(\tp{\T} \in \ls(\W^*, \V^*)\).

  To complete the proof, let \(\beta = \set{\seq{x}{1,,n}}\) and \(\gamma = \set{\seq{y}{1,,m}}\) with dual bases \(\beta^* = \set{\seq{f}{1,,n}}\) and \(\gamma^* = \set{\seq{g}{1,,m}}\), respectively.
  For convenience, let \(A = [\T]_{\beta}^{\gamma}\).
  To find the \(j\)th column of \([\tp{\T}]_{\gamma^*}^{\beta^*}\), we begin by expressing \(\tp{\T}(g_j)\) as a linear combination of the vectors of \(\beta^*\).
  By \cref{2.24} we have
  \[
    \tp{\T}(g_j) = g_j \T = \sum_{i = 1}^n (g_j \T)(x_i) f_i.
  \]
  So the row \(i\), column \(j\) entry of \([\tp{\T}]_{\gamma^*}^{\beta^*}\) is
  \begin{align*}
    (g_j \T)(x_i) & = g_j(\T(x_i))                                                         \\
                  & = g_j \pa{\sum_{k = 1}^m A_{k i} y_k} &  & \text{(by \cref{2.2.4})}    \\
                  & = \sum_{k = 1}^m A_{k i} g_j(y_k)     &  & \text{(by \cref{2.1.2}(d))} \\
                  & = \sum_{k = 1}^m A_{k i} \delta_{j k} &  & \text{(by \cref{2.6.4})}    \\
                  & = A_{j i}.
  \end{align*}
  Hence \([\tp{\T}]_{\gamma^*}^{\beta^*} = \tp{A}\).
\end{proof}

\begin{note}
  The linear transformation \(\tp{\T}\) defined in \cref{2.25} is called the \textbf{transpose} of \(\T\).
  It is clear that \(\tp{\T}\) is the unique linear transformation \(\U\) such that \([\U]_{\gamma^*}^{\beta^*} = \tp{\pa{[\T]_{\beta}^{\gamma}}}\)
  (see \cref{2.1.13}).
\end{note}

\begin{note}
  We now concern ourselves with demonstrating that any finite-dimensional vector space \(\V\) over \(\F\) \textbf{can be identified} in a natural way with its double dual \(\V^{**}\).
  There is, in fact, an isomorphism between \(\V\) and \(\V^{**}\) that does not depend on any choice of bases for the two vector spaces.
\end{note}

\begin{defn}\label{2.6.7}
  For a vector \(x \in \V\), we define \(\widehat{x} : \V^* \to \F\) by \(\widehat{x}(f) = f(x)\) for every \(f \in \V^*\).
  It is easy to verify that \(\widehat{x}\) is a linear functional on \(\V^*\), so \(\widehat{x} \in \V^{**}\).
  The correspondence \(x \leftrightarrow \widehat{x}\) allows us to define the desired isomorphism between \(\V\) and \(\V^{**}\).
\end{defn}

\begin{lem}\label{2.6.8}
  Let \(\V\) be a finite-dimensional vector space over \(\F\), and let \(x \in \V\).
  If \(\widehat{x}(f) = 0\) for all \(f \in \V^*\), then \(x = \zv\).
\end{lem}

\begin{proof}[\pf{2.6.8}]
  Let \(x \neq \zv\).
  We show that there exists \(f \in \V^*\) such that \(\widehat{x}(f) \neq 0\).
  Choose an ordered basis \(\beta = \set{\seq{x}{1,,n}}\) for \(\V\) over \(\F\) such that \(x_1 = x\).
  Let \(\set{\seq{f}{1,,n}}\) be the dual basis of \(\beta\).
  Then \(f_1(x_1) = 1 \neq 0\).
  Let \(f = f_1\).
\end{proof}

\begin{thm}\label{2.26}
  Let \(\V\) be a finite-dimensional vector space over \(\F\), and define \(\psi : \V \to \V^{**}\) by \(\psi(x) = \widehat{x}\).
  Then \(\psi\) is an isomorphism.
\end{thm}

\begin{proof}[\pf{2.26}]
  \begin{description}
    \item[\(\psi\) is linear:]
      Let \(x, y \in \V\) and \(c \in \F\).
      For \(f \in \V^*\), we have
      \begin{align*}
        \psi(cx + y)(f) & = f(cx + y)                         &  & \text{(by \cref{2.6.7})} \\
                        & = cf(x) + f(y)                      &  & \text{(by \cref{2.6.5})} \\
                        & = c \widehat{x}(f) + \widehat{y}(f) &  & \text{(by \cref{2.6.7})} \\
                        & = (c \widehat{x} + \widehat{y})(f). &  & \text{(by \cref{2.2.5})}
      \end{align*}
      Therefore
      \[
        \psi(cx + y) = c \widehat{x} + \widehat{y} = c \psi(x) + \psi(y).
      \]
    \item[\(\psi\) is one-to-one:]
      Suppose that \(\psi(x)\) is the zero functional on \(\V^*\) for some \(x \in \V\).
      Then \(\widehat{x}(f) = 0\) for every \(f \in \V^*\).
      By \cref{2.6.8} we conclude that \(x = \zv\).
      Since \(\ns{\psi} = \set{\zv}\) and \(\psi \in \ls(\V, \V^{**})\), by \cref{2.4} we know that \(\psi\) is one-to-one.
    \item[\(\psi\) is an isomorphism:]
      Since \(\psi \in \ls(\V, \V^{**})\), \(\psi\) is one-to-one and \(\dim(\V) = \dim(\V^{**})\), by \cref{2.5} and \cref{2.4.8} we know that \(\psi\) is isomorphism.
  \end{description}
\end{proof}

\begin{cor}\label{2.6.9}
  Let \(\V\) be a finite-dimensional vector space over \(\F\) with dual space \(\V^*\).
  Then every ordered basis for \(\V^*\) over \(\F\) is the dual basis for some basis for \(\V\) over \(\F\).
\end{cor}

\begin{proof}[\pf{2.6.9}]
  Let \(\set{\seq{f}{1,,n}}\) be an ordered basis for \(\V^*\) over \(\F\).
  We may combine \cref{2.24,2.26} to conclude that for this basis for \(\V^*\) over \(\F\), there exists a dual basis \(\set{\seq{\widehat{x}}{1,,n}}\) in \(\V^{**}\), that is,
  \begin{align*}
    \delta_{i j} & = \widehat{x}_i(f_j) &  & \text{(by \cref{2.24,2.26})} \\
                 & = f_j(x_i)           &  & \text{(by \cref{2.6.7})}
  \end{align*}
  for all \(i\) and \(j\).
  Thus by \cref{2.24} \(\set{\seq{f}{1,,n}}\) is the dual basis of \(\set{\seq{x}{1,,n}}\).
\end{proof}

\begin{note}
  Although many of the ideas of \cref{sec:2.6}, (e.g., the existence of a dual space), can be extended to the case where \(\V\) is not finite-dimensional, only a finite-dimensional vector space is isomorphic to its double dual via the map \(x \mapsto \widehat{x}\).
  In fact, for infinite-dimensional vector spaces, no two of \(\V\), \(\V^*\), and \(\V^{**}\) are isomorphic.
\end{note}

\exercisesection

\setcounter{ex}{7}
\begin{ex}\label{ex:2.6.8}
  Show that every plane through the origin in \(\R^3\) may be identified with the null space of a vector in \((\R^3)^*\).
  State an analogous result for \(\R^2\).
\end{ex}

\begin{proof}[\pf{ex:2.6.8}]
  First we prove the statement in the case of \(\R^3\).
  Let \((a, b, c) \in \R^3\) and let \(P\) be the following plane through the origin
  \[
    P = \set{(x, y, z) \in \R^3 : ax + by + cz = 0}.
  \]
  Let \(\T : \R^3 \to \R\) be the function \(\T(x, y, z) = ax + by + cz\).
  Clearly \(\T \in (\R^3)^*\).
  Then we have
  \begin{align*}
    \ns{\T} & = \set{(x, y, z) \in \R^3 : \T(x, y, z) = 0}       &  & \text{(by \cref{2.1.10})} \\
            & = \set{(x, y, z) \in \R^3 : ax + by + cz = 0} = P.
  \end{align*}

  Now we prove the statement in the case of \(\R^2\).
  Let \((a, b) \in \R^2\) and let \(P\) be the following plane through the origin
  \[
    P = \set{(x, y) \in \R^2 : ax + by = 0}.
  \]
  Let \(\T : \R^2 \to \R\) be the function \(\T(x, y) = ax + by\).
  Clearly \(\T \in (\R^2)^*\).
  Then we have
  \begin{align*}
    \ns{\T} & = \set{(x, y) \in \R^2 : \T(x, y) = 0}     &  & \text{(by \cref{2.1.10})} \\
            & = \set{(x, y) \in \R^2 : ax + by = 0} = P.
  \end{align*}
\end{proof}

\begin{ex}\label{ex:2.6.9}
  Prove that a function \(\T : \vs{F}^n \to \vs{F}^m\) is linear iff there exist \(\seq{f}{1,,m} \in (\vs{F}^n)^*\) such that \(\T(x) = (f_1(x), f_2(x), \dots, f_m(x))\) for all \(x \in \vs{F}^n\).
\end{ex}

\begin{proof}[\pf{ex:2.6.9}]
  First suppose that \(\T \in \ls(\vs{F}^n, \vs{F}^m)\).
  Let \(\gamma\) be the standard ordered basis for \(\vs{F}^m\) over \(\F\) and let \(\set{\seq{g}{1,,m}}\) be the dual basis of \(\gamma\).
  For each \(i \in \set{1, \dots, m}\), we define \(f_i : \vs{F}^n \to \F\) as follow:
  \[
    f_i = \tp{\T}(g_i) = g_i \T.
  \]
  By \cref{2.9,2.24} we see that \(f_i \in \ls(\vs{F}^n, \F)\).
  Thus by \cref{2.6.1} \(f_i \in (\vs{F}^n)^*\) and we have
  \begin{align*}
    \forall x \in \vs{F}^n, \T(x) & = [\T(x)]_{\gamma}                &  & \text{(by \cref{2.2.3})} \\
                                  & = \begin{pmatrix}
                                        g_1(\T(x)) & \cdots & g_m(\T(x))
                                      \end{pmatrix} &  & \text{(by \cref{2.6.4})}                   \\
                                  & = \begin{pmatrix}
                                        f_1(x) & \cdots & f_m(x)
                                      \end{pmatrix}.
  \end{align*}

  Now suppose that \(\T : \vs{F}^n \to \vs{F}^m\) is a function and there exist \(\seq{f}{1,,m} \in (\vs{F}^n)^*\) such that \(\T(x) = (f_1(x), \dots, f_m(x))\) for all \(x \in \vs{F}^n\).
  Let \(x, y \in \vs{F}^n\) and let \(c \in \F\).
  Then we have
  \begin{align*}
    \T(cx + y) & = \begin{pmatrix}
                     f_1(cx + y) & \cdots & f_m(cx + y)
                   \end{pmatrix}             \\
               & = \begin{pmatrix}
                     c f_1(x) + f_1(y) & \cdots & c f_m(x) + f_m(y)
                   \end{pmatrix} &  & \text{(by \cref{2.1.2}(b))} \\
               & = c \T(x) + \T(y)
  \end{align*}
  and thus by \cref{2.1.2}(b) \(\T \in \ls(\vs{F}^n, \vs{F}^m)\).
\end{proof}

\begin{ex}\label{ex:2.6.10}
  Let \(\seq{c}{0,,n}\) be distinct scalars in \(\F\).
  \begin{enumerate}
    \item For \(0 \leq i \leq n\), define \(f_i \in (\ps[n]{\F})^*\) by \(f_i(p) = p(c_i)\).
          Prove that \(\set{\seq{f}{0,,n}}\) is a basis for \((\ps[n]{\F})^*\).
    \item Use \cref{2.6.9} and (a) to show that there exist unique polynomials \(\seq{p}{0,,n}\) such that \(p_i(c_j) = \delta_{i j}\) for \(0 \leq i \leq n\).
          These polynomials are the Lagrange polynomials defined in \cref{1.6.20}.
    \item For any scalars \(\seq{a}{0,,n} \in \F\) (not necessarily distinct), deduce that there exists a unique polynomial \(q(x)\) of degree at most \(n\) such that \(q(c_i) = a_i\) for \(0 \leq i \leq n\).
          In fact,
          \[
            q(x) = \sum_{i = 0}^n a_i p_i(x).
          \]
    \item Deduce the Lagrange interpolation formula:
          \[
            p(x) = \sum_{i = 0}^n p(c_i) p_i(x)
          \]
          for any \(p \in \ps[n]{\F}\).
    \item Prove that
          \[
            \int_a^b p(t) \; dt = \sum_{i = 0}^n p(c_i) d_i,
          \]
          where
          \[
            d_i = \int_a^b p_i(t) \; dt.
          \]
          Suppose now that
          \[
            c_i = a + \frac{i (b - a)}{n} \quad \text{for } i \in \set{0, \dots, n}.
          \]
          For \(n = 1\), the preceding result yields the trapezoidal rule for evaluating the definite integral of a polynomial.
          For \(n = 2\), this result yields Simpson's rule for evaluating the definite integral of a polynomial.
  \end{enumerate}
\end{ex}

\begin{proof}[\pf{ex:2.6.10}(a)]
  Let \(\seq{a}{0,,n} \in \F\) such that
  \[
    \sum_{i = 0}^n a_i f_i = \zv.
  \]
  For each \(j \in \set{0, \dots, n}\), we define \(p_j \in \ps[n]{\F}\) as follow:
  \[
    \forall x \in \F, p_j(x) = \prod_{\substack{i = 0 \\ i \neq j}}^n (x - c_i).
  \]
  Then we have
  \begin{align*}
             & p_j(c_i) \neq 0 \iff j = i                                                                                                    \\
    \implies & \forall i \in \set{0, \dots, n}, \pa{\sum_{i = 0}^n a_i f_i}(p_j) = \sum_{i = 0}^n a_i f_i(p_j) &  & \text{(by \cref{2.2.5})} \\
             & = \sum_{i = 0}^n a_i p_j(c_i) = a_j p_j(c_j) = 0                                                                              \\
    \implies & \forall i \in \set{0, \dots, n}, a_j = 0.                                                       &  & (p_j(c_j) \neq 0)
  \end{align*}
  Thus by \cref{1.5.3} \(\set{\seq{f}{0,,n}}\) is linearly independent.
  Since
  \begin{align*}
    \dim((\ps[n]{\F})^*) & = \dim(\ls(\ps[n]{\F}, \F))       &  & \text{(by \cref{2.6.5})}  \\
                         & = \dim(\ps[n]{\F}) \cdot \dim(\F) &  & \text{(by \cref{2.4.10})} \\
                         & = n + 1                           &  & \text{(by \cref{1.6.12})} \\
                         & = \#(\set{\seq{f}{0,,n}}),
  \end{align*}
  by \cref{1.6.15}(b) we know that \(\set{\seq{f}{0,,n}}\) is a basis for \((\ps[n]{\F})^*\).
\end{proof}

\begin{proof}[\pf{ex:2.6.10}(b)]
  By \cref{2.6.9} there exists a set \(\beta = \set{\seq{p}{0,,n}}\) such that \(\beta\) is an ordered basis for \(\ps[n]{\F}\) over \(\F\) and the set \(\beta^* = \set{\seq{f}{0,,n}}\) defined in \cref{ex:2.6.10}(a) is the dual basis of \(\beta\).
  Then we have
  \begin{align*}
             & \forall i, j \in \set{0, \dots, n}, f_j(p_i) = \delta_{j i} = \delta_{i j} &  & \text{(by \cref{2.6.6})}        \\
    \implies & \forall i, j \in \set{0, \dots, n}, p_i(c_j) = \delta_{i j}.               &  & \text{(by \cref{ex:2.6.10}(a))}
  \end{align*}
  Now we show that \(\beta\) is unique.
  Suppose for sake of contradiction that there exists a set \(\gamma = \set{\seq{q}{0,,n}}\) such that \(\beta \neq \gamma\), \(q_i(c_j) = \delta_{i j}\) for \(i \in \set{0, \dots, n}\), \(\gamma\) is an ordered basis for \(\ps[n]{\F}\) over \(\F\) and \(\beta^*\) is also the dual basis for \(\gamma\).
  Since \(q_0 \in \ps[n]{\F}\), by \cref{1.6.1} we know that there exist some \(\seq{a}{0,,n} \in \F\) such that
  \[
    q_0 = \sum_{i = 0}^n a_i p_i.
  \]
  But
  \begin{align*}
    \forall j \in \set{0, \dots, n}, q_0(c_j) & = \delta_{0 j}                                                        \\
                                              & = \pa{\sum_{i = 0}^n a_i p_i}(c_j) &  & \text{(by \cref{2.2.5})}      \\
                                              & = \sum_{i = 0}^n a_i p_i(c_j)                                         \\
                                              & = \sum_{i = 0}^n a_i \delta_{i j}  &  & \text{(from the proof above)} \\
                                              & = a_j
  \end{align*}
  implies \(a_0 = 1\) and \(a_j = 0\) for all \(j \in \set{0, \dots, n} \setminus \set{0}\).
  Thus we have \(q_0 = p_0\).
  Using similar argument we see that \(q_i = p_i\) for all \(i \in \set{0, \dots, n}\), a contradiction.
  Thus \(\beta\) is unique.
\end{proof}

\begin{proof}[\pf{ex:2.6.10}(c)]
  Defined \(\seq{p}{0,,n} \in \ps[n]{\F}\) as in \cref{ex:2.6.10}(b).
  For any \(\seq{a}{0,,n} \in \F\), we can define \(q \in \ps[n]{\F}\) as follow:
  \[
    q = \sum_{j = 0}^n a_j p_j.
  \]
  Note that by \cref{ex:2.6.10}(b) \(q\) is unique.
  Then we have
  \[
    \forall i \in \set{0, \dots, n}, q(c_i) = \pa{\sum_{j = 0}^n a_j p_j}(c_i) = \sum_{j = 0}^n a_j p_j(c_i) = \sum_{j = 0}^n a_j \delta_{j i} = a_i
  \]
  and thus
  \[
    q = \sum_{j = 0}^n q(c_j) p_j.
  \]
\end{proof}

\begin{proof}[\pf{ex:2.6.10}(d)]
  By \cref{ex:2.6.10}(c) we see that
  \[
    \forall p \in \ps[n]{\F}, p = \sum_{i = 0}^n p(c_i) p_i.
  \]
\end{proof}

\begin{proof}[\pf{ex:2.6.10}(e)]
  By \cref{ex:2.6.10}(d) we have
  \begin{align*}
    \forall p \in \ps[n]{\F}, \int_a^b p(t) \; dt & = \int_a^b \pa{\sum_{i = 0}^n p(c_i) p_i(t)} \; dt \\
                                                  & = \sum_{i = 0}^n \pa{p(c_i) \int_a^b p_i(t) \; dt} \\
                                                  & = \sum_{i = 0}^n p(c_i) d_i.
  \end{align*}
\end{proof}

\begin{ex}\label{ex:2.6.11}
  Let \(\V\) and \(\W\) be finite-dimensional vector spaces over \(\F\), and let \(\psi_1\) and \(\psi_2\) be the isomorphisms between \(\V\) and \(\V^{**}\) and \(\W\) and \(\W^{**}\), respectively, as defined in \cref{2.26}.
  Let \(\T \in \ls(\V, \W)\), and define \(\tp{\tp{\T}} = \tp{\pa{\tp{\T}}}\).
  Prove that \(\psi_2 \T = \tp{\tp{\T}} \psi_1\).
\end{ex}

\begin{proof}[\pf{ex:2.6.11}]
  By \cref{2.25} we see that \(\tp{\tp{\T}} \in \ls(\V^{**}, \W^{**})\).
  Let \(x \in \V\) and let \(y \in \W^*\).
  Then we have
  \begin{align*}
    \pa{\pa{\tp{\tp{\T}} \psi_1}(x)}(y) & = \pa{\tp{\tp{\T}}(\psi_1(x))}(y)                                 \\
                                        & = \pa{\tp{\tp{\T}}(\widehat{x})}(y) &  & \text{(by \cref{2.26})}  \\
                                        & = \pa{\widehat{x} \tp{\T}}(y)       &  & \text{(by \cref{2.25})}  \\
                                        & = \widehat{x}\pa{\tp{\T}(y)}                                      \\
                                        & = \widehat{x}(y \T)                 &  & \text{(by \cref{2.26})}  \\
                                        & = (y \T)(x)                         &  & \text{(by \cref{2.6.7})} \\
                                        & = y(\T(x))                                                        \\
                                        & = \widehat{\T(x)}(y)                &  & \text{(by \cref{2.6.7})} \\
                                        & = \psi_2(\T(x))(y)                  &  & \text{(by \cref{2.26})}  \\
                                        & = ((\psi_2 \T)(x))(y).
  \end{align*}
  Thus \(\tp{\tp{\T}} \psi_1 = \psi_2 \T\).
\end{proof}

\begin{ex}\label{ex:2.6.12}
  Let \(\V\) be a finite-dimensional vector space over \(\F\) with the ordered basis \(\beta\) over \(\F\).
  Prove that \(\psi(\beta) = \beta^{**}\), where \(\psi\) is defined in \cref{2.26}.
\end{ex}

\begin{proof}[\pf{ex:2.6.12}]
  Let \(\beta = \set{\seq{x}{1,,n}}\) and let \(\beta^* = \set{\seq{f}{1,,n}}\) be the dual basis of \(\beta\).
  By \cref{2.26} we have \(\psi(\beta) = \set{\seq{\widehat{x}}{1,,n}}\).
  Since
  \begin{align*}
    \forall i, j \in \set{1, \dots, n}, \widehat{x}_i(f_j) & = f_j(x_i)      &  & \text{(by \cref{2.6.7})} \\
                                                           & = \delta_{i j}, &  & \text{(by \cref{2.6.6})}
  \end{align*}
  by \cref{2.6.6} we see that \(\beta^{**} = \set{\seq{\widehat{x}}{1,,n}}\).
\end{proof}

\begin{defn}\label{2.6.10}
  Let \(\V\) denotes a finite-dimensional vector space over \(\F\).
  For every subset \(S\) of \(\V\), define the \textbf{annihilator} \(S^0\) of \(S\) as
  \[
    S^0 = \set{f \in \V^* : f(x) = 0 \text{ for all } x \in S}.
  \]
\end{defn}

\begin{ex}\label{ex:2.6.13}
  Let \(\V\) be a vector space over \(\F\) and let \(S \subseteq \V\).
  \begin{enumerate}
    \item Prove that \(S^0\) is a subspace of \(\V^*\) over \(\F\).
    \item If \(\W\) is a subspace of \(\V\) over \(\F\) and \(x \notin \W\), prove that there exists \(f \in \W^0\) such that \(f(x) \neq 0\).
    \item Prove that \((S^0)^0 = \spn{\psi(S)}\), where \(\psi\) is defined as in \cref{2.26}.
    \item For subspaces \(\W_1\) and \(\W_2\) of \(\V\) over \(\F\), prove that \(\W_1 = \W_2\) iff \(\W_1^0 = \W_2^0\).
    \item For subspaces \(W_1\) and \(\W_2\) of \(\V\) over \(\F\), show that \((\W_1 + \W_2)^0 = \W_1^0 \cap \W_2^0\).
  \end{enumerate}
\end{ex}

\begin{proof}[\pf{ex:2.6.13}(a)]
  Let \(f, g \in S^0\) and let \(c \in \F\).
  Since
  \begin{align*}
    \forall x \in S, (cf + g)(x) & = cf(x) + g(x) &  & \text{(by \cref{2.2.5})}  \\
                                 & = c0 + 0       &  & \text{(by \cref{2.6.10})} \\
                                 & = 0,
  \end{align*}
  by \cref{2.6.10} we see that \(cf + g \in S^0\).
  Let \(\zT \in \V^*\) be the zero transformation of \(\V^*\).
  Since
  \begin{align*}
             & \forall x \in \V, \zT(x) = 0 &  & \text{(by \cref{2.1.9})}  \\
    \implies & \forall x \in S, \zT(x) = 0  &  & (S \subseteq \V)          \\
    \implies & \zT \in S^0,                 &  & \text{(by \cref{2.6.10})}
  \end{align*}
  by \cref{ex:1.3.18} we see that \(S^0\) is a subspace of \(\V^*\).
\end{proof}

\begin{proof}[\pf{ex:2.6.13}(b)]
  Let \(\beta_{\W}\) be a basis for \(\W\) over \(\F\).
  Since \(x \notin \W\), by \cref{1.7} we know that \(\beta_{\W} \cup \set{x}\) is linearly independent.
  By \cref{1.6.19} \(\beta_{\W} \cup \set{x}\) can be extended to a basis \(\beta\) for \(\V\) over \(\F\).
  By \cref{2.6} there exists a \(f \in \V^*\) such that
  \[
    \forall v \in \beta, f(v) = \begin{dcases}
      0 & \text{if } v \neq x \\
      1 & \text{if } v = x
    \end{dcases}.
  \]
  Then we have
  \begin{align*}
             & \W = \spn{\beta_{\W}}                          &  & \text{(by \cref{1.6.1})}  \\
    \implies & f(\W) = \rg{f} = \spn{f(\beta_{\W})} = \set{0} &  & \text{(by \cref{2.2})}    \\
    \implies & f \in \W^0.                                    &  & \text{(by \cref{2.6.10})}
  \end{align*}
\end{proof}

\begin{proof}[\pf{ex:2.6.13}(c)]
  Since
  \begin{align*}
         & f \in S^0                                                        \\
    \iff & \forall x \in S, f(x) = 0       &  & \text{(by \cref{2.6.10})}   \\
    \iff & \forall x \in \spn{S}, f(x) = 0 &  & \text{(by \cref{2.1.2}(d))} \\
    \iff & f \in (\spn{S})^0,              &  & \text{(by \cref{2.6.10})}
  \end{align*}
  we have \(S^0 = (\spn{S})^0\).
  Since \(\psi \in \ls(\V, \V^{**})\), by \cref{2.2} we know that \(\psi(\spn{S}) = \spn{\psi(S)}\).
  Thus to prove that \((S^0)^0 = \spn{\psi(S)}\), we can instead prove that \(((\spn{S})^0)^0 = \psi(\spn{S})\).
  By \cref{ex:2.6.13}(a) we see that \(S^0 \subseteq \V^*\), thus by \cref{2.6.10} we have
  \[
    (S^0)^0 = ((\spn{S})^0)^0 = \set{f \in \V^{**} : f((\spn{S})^0) = \set{0}}.
  \]

  First we show that \(((\spn{S})^0)^0 \subseteq \psi(\spn{S})\).
  Let \(f \in ((\spn{S})^0)^0\).
  Since \(f \in \V^{**}\), by \cref{2.26} we know that there exists an \(s \in \V\) such that \(\psi(s) = \widehat{s} = f\).
  Then we have
  \begin{align*}
             & f = \widehat{s}                                                                                  \\
    \implies & \forall g \in (\spn{S})^0, f(g) = 0 = \widehat{s}(g) = g(s) &  & \text{(by \cref{2.6.7})}        \\
    \implies & s \in \spn{S}                                               &  & \text{(by \cref{ex:2.6.13}(b))} \\
    \implies & f = \widehat{s} = \psi(s) \in \psi(\spn{S}).
  \end{align*}
  Since \(f\) is arbitrary, we see that \(((\spn{S})^0)^0 \subseteq \psi(\spn{S})\).

  Now we show that \(\psi(\spn{S}) \subseteq ((\spn{S})^0)^0\).
  Let \(f \in \psi(\spn{S})\).
  By \cref{2.26} we know that there exists an \(s \in \spn{S}\) such that \(\psi(s) = \widehat{s} = f\).
  Then we have
  \begin{align*}
             & f = \widehat{s}                                                                       \\
    \implies & \forall g \in \V^*, f(g) = \widehat{s}(g) = g(s) &  & \text{(by \cref{2.6.7})}        \\
    \implies & \forall g \in (\spn{S})^0, f(g) = g(s)           &  & \text{(by \cref{ex:2.6.13}(a))} \\
             & = 0                                              &  & \text{(by \cref{2.6.10})}       \\
    \implies & f = \widehat{s} = \psi(s) \in ((\spn{S})^0)^0.   &  & \text{(by \cref{2.6.10})}
  \end{align*}
  Since \(f\) is arbitrary, we see that \(\psi(\spn{S}) \subseteq ((\spn{S})^0)^0\).
\end{proof}

\begin{proof}[\pf{ex:2.6.13}(d)]
  We have
  \begin{align*}
             & \W_1 = \W_2                                                                    \\
    \implies & \W_1^0 = \set{f \in \V^* : f(\W_1) = \set{0}}   &  & \text{(by \cref{2.6.10})} \\
             & = \set{f \in \V^* : f(\W_2) = \set{0}} = \W_2^0 &  & \text{(by \cref{2.6.10})} \\
  \end{align*}
  and
  \begin{align*}
             & \W_1^0 = \W_2^0                                                                                                  \\
    \implies & \set{f \in \V^* : f(\W_1) = \set{0}} = \set{f \in \V^* : f(\W_2) = \set{0}} &  & \text{(by \cref{2.6.10})}       \\
    \implies & \W_1 = \W_2.                                                                &  & \text{(by \cref{ex:2.6.13}(b))}
  \end{align*}
  Thus \(\W_1 = \W_2 \iff \W_1^0 = \W_2^0\).
\end{proof}

\begin{proof}[\pf{ex:2.6.13}(e)]
  Since
  \begin{align*}
         & f \in (\W_1 + \W_2)^0                                                                  \\
    \iff & \forall x \in \W_1 + \W_2, f(x) = 0                     &  & \text{(by \cref{2.6.10})} \\
    \iff & \forall (x_1, x_2) \in \W_1 \times \W_2, \begin{dcases}
                                                      x_1 + \zv_{\V} \in \W_1 + \W_2 \\
                                                      \zv_{\V} + x_2 \in \W_1 + \W_2 \\
                                                      x_1 + x_2 \in \W_1 + \W_2      \\
                                                      f(x_1) = f(x_2) = f(x_1 + x_2) = 0
                                                    \end{dcases} &  & \text{(by \cref{1.3.10})}   \\
    \iff & \begin{dcases}
             f \in \W_1^0 \\
             f \in \W_2^0
           \end{dcases}                                         &  & \text{(by \cref{2.6.10})}    \\
    \iff & f \in \W_1^0 \cap \W_2^0,
  \end{align*}
  we know that \((\W_1 + \W_2)^0 = \W_1^0 \cap \W_2^0\).
\end{proof}

\begin{ex}\label{ex:2.6.14}
  Prove that if \(\V\) is a finite-dimensional vector space over \(\F\) and \(\W\) is a subspace of \(\V\) over \(\F\), then \(\dim(\W) + \dim(\W^0) = \dim(\V)\).
\end{ex}

\begin{proof}[\pf{ex:2.6.14}]
  Let \(\dim(\V) = n\) and let \(\beta_{\W} = \set{\seq{x}{1,,k}}\) be an ordered basis for \(\W\) over \(\F\).
  By \cref{1.6.19} we can extend \(\beta_{\W}\) to an ordered basis \(\beta = \set{\seq{x}{1,,k,k+1,,n}}\) for \(\V\) over \(\F\).
  By \cref{2.6.6} we denote the dual basis of \(\beta\) as \(\beta^* = \set{\seq{f}{1,,n}}\).
  If we can show that the set \(\gamma = \set{\seq{f}{k+1,,n}}\) is a basis for \(\W^0\) over \(\F\), then we have
  \[
    \dim(\W) + \dim(\W^0) = k + (n - k) = n = \dim(\V).
  \]
  By \cref{2.24} we know that \(\beta^*\) is a basis for \(\V^*\) over \(\F\), thus \(\gamma\) is linearly independent and by \cref{1.6.1} we only need to show that \(\spn{\gamma} = \W^0\).

  First we show that \(\spn{\gamma} \subseteq \W^0\).
  By \cref{2.24} we have \(\spn{\gamma} \subseteq \V^*\).
  Let \(w \in \W\).
  By \cref{1.6.1} we know that
  \[
    \exists \seq{a}{1,,k} \in \F : w = \sum_{j = 1}^k a_j x_j.
  \]
  Thus we have
  \begin{align*}
             & \forall g \in \spn{\gamma}, \exists \seq{b}{1,,n-k} \in \F : g = \sum_{i = 1}^{n - k} b_i f_{i + k}  &  & \text{(by \cref{1.4.3})}    \\
    \implies & \forall g \in \spn{\gamma}, g(w) = \pa{\sum_{i = 1}^{n - k} b_i f_{i + k}}(w)                                                         \\
             & = \sum_{i = 1}^{n - k} b_i f_{i + k}(w) = \sum_{i = 1}^{n - k} \sum_{j = 1}^k b_i a_j f_{i + k}(x_j) &  & \text{(by \cref{2.1.2}(d))} \\
             & = 0                                                                                                  &  & \text{(by \cref{2.6.6})}    \\
    \implies & \forall g \in \spn{\gamma}, g(\W) = \set{0}                                                                                           \\
    \implies & \forall g \in \spn{\gamma}, g \in \W^0                                                               &  & \text{(by \cref{2.6.10})}   \\
    \implies & \spn{\gamma} \subseteq \W^0.
  \end{align*}

  Now we show that \(\W^0 \subseteq \spn{\gamma}\).
  This is true since
  \begin{align*}
             & \W^0 \subseteq \V^*                                         &  & \text{(by \cref{2.6.10})} \\
    \implies & \W^0 \subseteq \spn{\beta^*}                                &  & \text{(by \cref{2.24})}   \\
    \implies & \forall g \in \W^0, g = \sum_{i = 1}^n g(x_i) f_i           &  & \text{(by \cref{2.24})}   \\
             & = \sum_{i = 1}^k g(x_i) f_i + \sum_{i = k + 1}^n g(x_i) f_i                                \\
             & = 0 + \sum_{i = k + 1}^n g(x_i) f_i                         &  & \text{(by \cref{2.6.10})} \\
    \implies & \forall g \in \W^0, g \in \spn{\gamma}                                                     \\
    \implies & \W^0 \subseteq \spn{\gamma}.
  \end{align*}
  Thus we have \(\W^0 = \spn{\gamma}\).
\end{proof}

\begin{ex}\label{ex:2.6.15}
  Suppose that \(\V, \W\) are finite-dimensional vector spaces over \(\F\) and that \(T \in \ls(\V, \W)\).
  Prove that \(\ns{\tp{\T}} = (\rg{\T})^0\).
\end{ex}

\begin{proof}[\pf{ex:2.6.15}]
  We have
  \begin{align*}
         & x \in \ns{\tp{\T}}                                                           \\
    \iff & \tp{\T}(x) = 0_{\V^*}                         &  & \text{(by \cref{2.1.10})} \\
    \iff & x \T = 0_{\V^*}                               &  & \text{(by \cref{2.25})}   \\
    \iff & (x \T)(\V) = x(\T(\V)) = x(\rg{\T}) = \set{0} &  & \text{(by \cref{2.1.10})} \\
    \iff & x \in (\rg{\T})^0                             &  & \text{(by \cref{2.6.10})}
  \end{align*}
  and thus \(\ns{\tp{\T}} = (\rg{\T})^0\).
\end{proof}

\begin{ex}\label{ex:2.6.16}
  Use \cref{ex:2.6.14,ex:2.6.15} to deduce that \(\rk{\L_{\tp{A}}} = \rk{\L_A}\) for any \(A \in \MS\).
\end{ex}

\begin{proof}[\pf{ex:2.6.16}]
  Let \(\beta\) and \(\gamma\) be ordered bases for \(\vs{F}^n\) and \(\vs{F}^m\) over \(\F\), respectively.
  Let \(\beta^*\) and \(\gamma^*\) be the dual bases of \(\beta\) and \(\gamma\), respectively.
  Since
  \begin{align*}
    [\tp{\pa{\L_A}}]_{\gamma^*}^{\beta^*} & = \tp{\pa{[\L_A]_{\beta}^{\gamma}}} &  & \text{(by \cref{2.25})}    \\
                                          & = \tp{A}                            &  & \text{(by \cref{2.15}(a))} \\
                                          & = [\L_{\tp{A}}]_{\gamma}^{\beta},   &  & \text{(by \cref{2.15}(a))}
  \end{align*}
  we have
  \begin{align*}
    \rk{\L_{\tp{A}}} & = \rk{\tp{\pa{\L_A}}}                                     &  & \text{(by \cref{2.15}(b))}   \\
                     & = \dim\pa{\pa{\vs{F}^m}^*} - \nt{\tp{\pa{\L_A}}}          &  & \text{(by \cref{2.3})}       \\
                     & = \dim\pa{\pa{\vs{F}^m}^*} - \dim\pa{\ns{\tp{\pa{\L_A}}}} &  & \text{(by \cref{2.1.12})}    \\
                     & = \dim\pa{\vs{F}^m} - \dim\pa{\ns{\tp{\pa{\L_A}}}}        &  & \text{(by \cref{2.6.5})}     \\
                     & = \dim\pa{\vs{F}^m} - \dim\pa{\pa{\rg{\L_A}}^0}           &  & \text{(by \cref{ex:2.6.15})} \\
                     & = \dim\pa{\rg{\L_A}}                                      &  & \text{(by \cref{ex:2.6.14})} \\
                     & = \rk{\L_A}.                                              &  & \text{(by \cref{2.1.12})}
  \end{align*}
\end{proof}

\begin{ex}\label{ex:2.6.17}
  Let \(\V\) be a vector space over \(\F\), let \(\T \in \ls(\V)\) and let \(\W\) be a subspace of \(\V\) over \(\F\).
  Prove that \(\W\) is \(\T\)-invariant (as defined in \cref{2.1.15}) iff \(\W^0\) is \(\tp{\T}\)-invariant.
\end{ex}

\begin{proof}[\pf{ex:2.6.17}]
  We have
  \begin{align*}
         & \W^0 \text{ is } \tp{\T} \text{-invariant}                                          \\
    \iff & \tp{\T}(\W^0) \subseteq \W^0                   &  & \text{(by \cref{2.1.15})}       \\
    \iff & \forall f \in \W^0, \tp{\T}(f) = f \T \in \W^0 &  & \text{(by \cref{2.25})}         \\
    \iff & \forall f \in \W^0, f(\T(\W)) = \set{0}        &  & \text{(by \cref{2.6.10})}       \\
    \iff & \T(\W) \subseteq \W                            &  & \text{(by \cref{ex:2.6.13}(b))} \\
    \iff & \W \text{ is } \T \text{-invariant}.           &  & \text{(by \cref{2.1.15})}
  \end{align*}
\end{proof}

\begin{ex}\label{ex:2.6.18}
  Let \(\V\) be a nonzero vector space over a field \(\F\), and let \(S\) be a basis for \(\V\) over \(\F\).
  (By \cref{1.7.10}, every vector space has a basis.)
  Let \(\Phi : \V^* \to \FS\) be the mapping defined by \(\Phi(f) = f_S\), the restriction of \(f\) to \(S\).
  Prove that \(\Phi\) is an isomorphism.
\end{ex}

\begin{proof}[\pf{ex:2.6.18}]
  \begin{description}
    \item[For linearity:]
      Let \(f, g \in \V^*\) and let \(c \in \F\).
      Then we have
      \[
        \Phi(cf + g) = (cf + g)_S = c f_S + g_S = c \Phi(f) + \Phi(g)
      \]\
      and thus by \cref{2.1.2}(b) \(\Phi \in \ls(\V^*, \FS)\).
    \item[For invertibility:]
      By \cref{ex:2.1.34} we see that \(\Phi\) is one-to-one and onto.
  \end{description}
\end{proof}

\begin{ex}\label{ex:2.6.19}
  Let \(\V\) be a nonzero vector space over \(\F\), and let \(\W\) be a proper subspace of \(\V\) over \(\F\)
  (i.e., \(\W \neq \V\)).
  Prove that there exists a nonzero linear functional \(f \in \V^*\) such that \(f(x) = 0\) for all \(x \in \W\).
\end{ex}

\begin{proof}[\pf{ex:2.6.19}]
  Let \(\beta_{\W}\) be a basis for \(\W\) over \(\F\).
  By \cref{1.12,1.13} we can extend \(\beta_{\W}\) to a basis \(\beta\) for \(\V\) over \(\F\).
  Then by \cref{ex:2.1.34} there exists a \(f \in \V^*\) such that
  \[
    \forall v \in \beta, f(v) = \begin{dcases}
      0 & \text{if } v \in \beta_{\W}                 \\
      1 & \text{if } v \in \beta \setminus \beta_{\W}
    \end{dcases}.
  \]
  Note that \(\W \neq \V\) implies \(\beta \setminus \beta_{\W} \neq \varnothing\) and thus \(f\) is not a zero transformation.
\end{proof}

\begin{ex}\label{ex:2.6.20}
  Let \(\V\) and \(\W\) be nonzero vector spaces over the same field \(\F\), and let \(\T \in \ls(\V, \W)\).
  \begin{enumerate}
    \item Prove that \(\T\) is onto iff \(\tp{\T}\) is one-to-one.
    \item Prove that \(\tp{\T}\) is onto iff \(\T\) is one-to-one.
  \end{enumerate}
\end{ex}

\begin{proof}[\pf{ex:2.6.20}(a)]
  First suppose that \(\T\) is onto.
  Let \(f \in \ns{\tp{\T}}\).
  Since
  \begin{align*}
             & \tp{\T}(f) = 0_{\V^*}          &  & \text{(by \cref{2.1.10})} \\
    \implies & f \T = 0_{\V^*}                &  & \text{(by \cref{2.25})}   \\
    \implies & f(\T(\V)) = f(\W) = \set{0}    &  & \text{(\(f\) is onto)}    \\
    \implies & f = 0_{\W^*}                   &  & \text{(by \cref{2.1.9})}  \\
    \implies & \ns{\tp{\T}} = \set{0_{\W^*}}, &  & \text{(by \cref{2.1.10})}
  \end{align*}
  by \cref{2.4} we know that \(\tp{\T}\) is one-to-one.

  Now suppose that \(\tp{\T}\) is one-to-one.
  Suppose for sake of contradiction that \(\T\) is not onto.
  By \cref{ex:2.1.33} we know that \(\rg{\T} \subsetneq \W\).
  By \cref{ex:2.6.19} there exists a \(f \in \W^*\) such that \(f(\rg{\T}) = \set{0}\) and \(f(\W) \neq \set{0}\).
  Let \(g \in \W^*\) be the zero transformation.
  Then we have
  \begin{align*}
    \forall x \in \V, \pa{\tp{\T}(g)}(x) & = (g \T)(x)          &  & \text{(by \cref{2.25})} \\
                                         & = g(\T(x))                                        \\
                                         & = 0                                               \\
                                         & = f(\T(x))                                        \\
                                         & = (f \T)(x)                                       \\
                                         & = \pa{\tp{\T}(f)}(x) &  & \text{(by \cref{2.25})}
  \end{align*}
  and thus \(\tp{\T}(g) = \tp{\T}(f)\).
  But \(\tp{\T}\) is one-to-one implies \(f = g\), a contradiction.
  Thus \(\T\) is onto.
\end{proof}

\begin{proof}[\pf{ex:2.6.20}(b)]
  First suppose that \(\tp{\T}\) is onto.
  Suppose for sake of contradiction that \(\T\) is not one-to-one.
  By \cref{2.4} there exists an \(x \in \V \setminus \set{\zv_{\V}}\) such that \(\T(x) = \zv_{\W}\).
  By \cref{ex:2.1.34} there exists a \(f \in \V^*\) such that \(f(x) \neq 0\).
  Since \(f \in \V^*\) and \(\tp{\T}\) is onto, there exists a \(g \in \W^*\) such that \(f = \tp{\T}(g)\).
  But this means
  \begin{align*}
    f(x) & = \pa{\tp{\T}(g)}(x)                                  \\
         & = (g \T)(x)          &  & \text{(by \cref{2.25})}     \\
         & = g(\T(x))                                            \\
         & = g(\zv_{\W})        &  & \text{(by hypothesis)}      \\
         & = 0,                 &  & \text{(by \cref{2.1.2}(a))}
  \end{align*}
  a contradiction.
  Thus \(\T\) is one-to-one.

  Now suppose that \(\T\) is one-to-one.
  Let \(f \in \V^*\) and let \(\beta\) be a basis for \(\V\) over \(\F\).
  Since \(\T\) is one-to-one, by \cref{ex:2.1.14}(a) we know that \(\T(\beta)\) is linearly independent.
  By \cref{1.12,1.13} we can extend \(\T(\beta)\) to a basis \(\gamma\) for \(\W\) over \(\F\).
  If we define \(h \in \fs(\gamma, \F)\) by setting
  \[
    \forall w \in \gamma, h(w) = \begin{dcases}
      f(v) & \text{if } (w \in \T(\beta)) \land (\exists! v \in \beta : \T(v) = w) \\
      0    & \text{if } w \in \gamma \setminus \T(\beta)
    \end{dcases},
  \]
  then by \cref{ex:2.6.18} we know that there exists a \(g \in \W^*\) such that \(\Phi(g) = g_{\gamma} = h\).
  Since
  \begin{align*}
    \forall v \in \beta, f(v) & = h(\T(v))            &  & \text{(by definition)}       \\
                              & = (h \T)(v)                                             \\
                              & = (g \T)(v)           &  & \text{(by \cref{ex:2.6.18})} \\
                              & = \pa{\tp{\T}(g)}(v), &  & \text{(by \cref{2.25})}
  \end{align*}
  by \cref{ex:2.1.34} we know that \(f = \tp{\T}(g)\).
  Since \(f\) is arbitrary, we know that \(\tp{\T}\) is onto.
\end{proof}

\section{Homogeneous Linear Differential Equations with Constant Coefficients}\label{sec:2.7}

\begin{defn}\label{2.7.1}
  A \textbf{differential equation} in an unknown function \(y = y(t)\) is an equation involving \(y\), \(t\), and derivatives of \(y\).
  If the differential equation is of the form
  \begin{equation}\label{eq:2.7.1}
    a_n y^{(n)} + a_{n - 1} y^{(n - 1)} + \cdots + a_1 y^{(1)} + a_0 y = f,
  \end{equation}
  where \(\seq{a}{0,,n}\) and \(f\) are functions of \(t\) and \(y^{(k)}\) denotes the \(k\)th derivative of \(y\), then the equation is said to be \textbf{linear}.
  The functions \(a_i\) are called the \textbf{coefficients} of the differential equation \cref{eq:2.7.1}.
  When \(f\) is identically zero, \cref{eq:2.7.1} is called \textbf{homogeneous}.

  If \(a_n \neq 0\), we say that differential equation \cref{eq:2.7.1} is of \textbf{order \(n\)}.
  In this case, we divide both sides by \(a_n\) to obtain a new, but equivalent, equation
  \[
    y^{(n)} + b_{n - 1} y^{(n - 1)} + \cdots + b_1 y^{(1)} + b_0 y = \zv,
  \]
  where \(b_i = a_i / a_n\) for \(i \in \set{0, \dots, n - 1}\).
  Because of this observation, we always assume that the coefficient \(a_n\) in \cref{eq:2.7.1} is \(1\).

  A \textbf{solution} to \cref{eq:2.7.1} is a function that when substituted for \(y\) reduces \cref{eq:2.7.1} to an identity.
\end{defn}

\begin{defn}\label{2.7.2}
  In our study of differential equations, it is useful to regard solutions as complex-valued functions of a real variable even though the solutions that are meaningful to us in a physical sense are real-valued.
  The convenience of this viewpoint will become clear later.
  Thus we are concerned with the vector space \(\fs(\R, \C)\).
  In order to consider complex-valued functions of a real variable as solutions to differential equations, we must define what it means to differentiate such functions.
  Given a complex-valued function \(x \in \fs(\R, \C)\) of a real variable \(t\), there exist unique real-valued functions \(x_1\) and \(x_2\) of \(t\), such that
  \[
    x(t) = x_1(t) + i x_2(t) \quad \text{for} \quad t \in \R,
  \]
  where \(i\) is the imaginary number such that \(i^2 = -1\).
  We call \(x_1\) the \textbf{real part} and \(x_2\) the \textbf{imaginary part} of \(x\).
\end{defn}

\begin{defn}\label{2.7.3}
  Given a function \(x \in \fs(\R, \C)\) with real part \(x_1\) and imaginary part \(x_2\), we say that \(x\) is \textbf{differentiable} if \(x_1\) and \(x_2\) are differentiable.
  If \(x\) is differentiable, we define the derivative \(x'\) of \(x\) by
  \[
    x' = x_1' + i x_2'.
  \]
\end{defn}

\begin{thm}\label{2.27}
  Any solution to a homogeneous linear differential equation with constant coefficients has derivatives of all orders;
  that is, if \(x\) is a solution to such an equation, then \(x^{(k)}\) exists for every positive integer \(k\).
\end{thm}

\begin{proof}[\pf{2.27}]
  Let
  \[
    y^{(n)} + a_{n - 1} y^{(n - 1)} + \cdots + a_1 y^{(1)} + a_0 y = 0
  \]
  be a homogeneous linear differential equation of order \(n\) with constant coefficients.
  Clearly \(y^{(k)}\) exists for all \(k \in \set{0, \dots, n}\).
  Now we prove that \(y^{(k)}\) exists for all \(k \in \N \setminus \set{0, \dots, n}\).
  We can rewrite the equation above as
  \[
    y^{(n)} = -a_{n - 1} y^{(n - 1)} - \cdots - a_1 y^{(1)} - a_0 y.
  \]
  Since \(y^{(n)}\) exists, each term on the right hand side of the above equation can be differentiated at least one more time.
  Thus \(y^{(n + 1)}\) must exist and is equal to the follow:
  \[
    y^{(n + 1)} = (y^{(n)})' = -a_{n - 1} y^{(n)} - \cdots - a_1 y^{(2)} - a_0 y^{(1)}.
  \]
  But again \(y^{(n + 1)}\) exists, therefore each term on the right hand side of the above equation can be differentiated at least one more time.
  Thus \(y^{(n + 2)}\) must exist and is equal to the follow:
  \[
    y^{(n + 2)} = (y^{(n + 1)})' = -a_{n - 1} y^{(n + 1)} - \cdots - a_1 y^{(3)} - a_0 y^{(2)}.
  \]
  In general we see that for all \(k \in \N\), we have
  \[
    y^{(n + k)} = (y^{(n)})^{(k)} = -a_{n - 1} y^{(n - 1 + k)} - \cdots - a_1 y^{(1 + k)} - a_0 y^{(k)}.
  \]
\end{proof}

\begin{defn}\label{2.7.4}
  We use \(\cfs[\infty](\R, \C)\) to denote the set of all functions in \(\fs(\R, \C)\) that have derivatives of all orders.
  By \cref{ex:2.7.5} \(\cfs[\infty](\R, \C)\) is a subspace of \(\fs(\R, \C)\) over \(\C\) and hence is a vector space over \(\C\).
\end{defn}

\begin{defn}\label{2.7.5}
  For \(x \in \cfs[\infty](\R, \C)\), the derivative \(x'\) of \(x\) also lies in \(\cfs[\infty](\R, \C)\).
  We can use the derivative operation to define a mapping \(\Dop : \cfs[\infty](\R, \C) \to \cfs[\infty](\R, \C)\) by
  \[
    \Dop(x) = x' \quad \text{for } x \in \cfs[\infty](\R, \C).
  \]
  It is easy to show that \(\Dop\) is a linear operator.
  More generally, consider any polynomial over \(\C\) of the form
  \[
    p(t) = a_n t^n + a_{n - 1} t^{n - 1} + \cdots + a_1 t + a_0.
  \]
  If we define
  \[
    p(\Dop) = a_n \Dop^n + a_{n - 1} \Dop^{n - 1} + \cdots + a_1 \Dop + a_0 \IT,
  \]
  then \(p(\Dop)\) is a linear operator on \(\cfs[\infty](\R, \C)\).
  (See \cref{e.0.7,ex:2.7.6}.)

  For any polynomial \(p\) over \(\C\) of positive degree, \(p(\Dop)\) is called a \textbf{differential operator}.
  The \textbf{order} of the differential operator \(p(\Dop)\) is the degree of the polynomial \(p\).

  Differential operators are useful since they provide us with a means of reformulating a differential equation in the context of linear algebra.
  Any homogeneous linear differential equation with constant coefficients,
  \[
    y^{(n)} + a_{n - 1} y^{(n - 1)} + \cdots + a_1 y^{(1)} + a_0 y = \zv,
  \]
  can be rewritten using differential operators as
  \[
    \pa{\Dop^{(n)} + a_{n - 1} \Dop^{(n - 1)} + \cdots + a_1 \Dop^{(1)} + a_0 \IT}(y) = \zv.
  \]
  Given the differential equation above, the complex polynomial
  \[
    p(t) = t^n + a_{n - 1} t^{n - 1} + \cdots + a_1 t + a_0
  \]
  is called the \textbf{auxiliary polynomial} associated with the equation.
  Any homogeneous linear differential equation with constant coefficients can be rewritten as
  \[
    p(\Dop)(y) = \zv,
  \]
  where \(p(t)\) is the auxiliary polynomial associated with the equation.
\end{defn}

\begin{thm}\label{2.28}
  The set of all solutions to a homogeneous linear differential equation with constant coefficients coincides with the null space of \(p(\Dop)\), where \(p(t)\) is the auxiliary polynomial associated with the equation.
\end{thm}

\begin{proof}[\pf{2.28}]
  Let
  \[
    y^{(n)} + a_{n - 1} y^{(n - 1)} + \cdots + a_1 y^{(1)} + a_0 y = \zv
  \]
  be a homogeneous linear differential equation where \(\seq{a}{0,,n-1} \in \C\).
  By \cref{2.7.5}
  \[
    p(t) = t^n + a_{n - 1} t^{n - 1} + \cdots + a_1 t + a_0
  \]
  is the auxiliary polynomial associated with the above homogeneous linear differential equation.
  Then we have
  \begin{align*}
         & x \in \cfs[\infty](\R, \C) \text{ is a solution of }                                  \\
         & y^{(n)} + a_{n - 1} y^{(n - 1)} + \cdots + a_1 y^{(1)} + a_0 y = \zv                  \\
    \iff & x^{(n)} + a_{n - 1} x^{(n - 1)} + \cdots + a_1 x^{(1)} + a_0 x = \zv &  & \by{2.7.1}  \\
    \iff & p(\Dop)(x) = \zv                                                     &  & \by{2.7.5}  \\
    \iff & x \in \ns{p(\Dop)}                                                   &  & \by{2.1.10}
  \end{align*}
  and thus \(\set{x \in \cfs[\infty](\R, \C) : x^{(n)} + a_{n - 1} x^{(n - 1)} + \cdots + a_1 x^{(1)} + a_0 x = \zv} = \ns{p(\Dop)}\).
\end{proof}

\begin{cor}\label{2.7.6}
  The set of all solutions to a homogeneous linear differential equation with constant coefficients is a subspace of \(\cfs[\infty](\R, \C)\).
\end{cor}

\begin{proof}[\pf{2.7.6}]
  By \cref{2.1,2.28} we see that this is true.
\end{proof}

\begin{defn}\label{2.7.7}
  In view of \cref{2.7.6}, we call the set of solutions to a homogeneous linear differential equation with constant coefficients the \textbf{solution space} of the equation.
  A practical way of describing such a space is in terms of a basis.
\end{defn}

\begin{defn}\label{2.7.8}
  Let \(c = a + ib\) be a complex number with real part \(a\) and imaginary part \(b\).
  Define
  \[
    e^c = e^a (\cos(b) + i \sin(b)).
  \]
  The special case
  \[
    e^{ib} = \cos(b) + i \sin(b)
  \]
  is called \textbf{Euler's formula}.
  Clearly, if \(c\) is real (\(b = 0\)), then we obtain the usual result:
  \(e^c = e^a\).
  We can show by the use of trigonometric identities that
  \[
    e^{c + d} = e^c e^d \quad \text{and} \quad e^{-c} = \dfrac{1}{e^c}
  \]
  for any complex number \(c\) and \(d\).
\end{defn}

\begin{defn}\label{2.7.9}
  A function \(f : \R \to \C\) defined by \(f(t) = e^{ct}\) for a fixed complex number \(c\) is called an \textbf{exponential function}.
\end{defn}

\begin{thm}\label{2.29}
  For any exponential function \(f(t) = e^{ct}\), \(f'(t) = c e^{ct}\).
\end{thm}

\begin{proof}[\pf{2.29}]
  We have
  \begin{align*}
    f'(t) & = (e^{ct})'                                                                                                 \\
          & = (e^{\Re(ct) + i \Im(ct)})'                                                      &  & \by{2.7.2}           \\
          & = (e^{\Re(ct)} \cdot (\cos(\Im(ct)) + i \cdot \sin(\Im(ct))))'                    &  & \by{2.7.8}           \\
          & = (e^{\Re(ct)} \cdot \cos(\Im(ct)) + i \cdot e^{\Re(ct)} \cdot \sin(\Im(ct)))'                              \\
          & = (e^{\Re(ct)} \cdot \cos(\Im(ct)))' + i \cdot (e^{\Re(ct)} \cdot \sin(\Im(ct)))' &  & \by{2.7.3}           \\
          & = e^{\Re(ct)} \cdot (\Re(ct))' \cdot \cos(\Im(ct))                                                          \\
          & \quad - e^{\Re(ct)} \cdot \sin(\Im(ct)) \cdot (\Im(ct))'                                                    \\
          & \quad + i \cdot e^{\Re(ct)} \cdot (\Re(ct))' \cdot \sin(\Im(ct))                                            \\
          & \quad + i \cdot e^{\Re(ct)} \cdot \cos(\Im(ct)) \cdot (\Im(ct))'                                            \\
          & = e^{\Re(ct)} \cdot \Re(c) \cdot \cos(\Im(ct))                                    &  & \by{2.7.3}           \\
          & \quad - e^{\Re(ct)} \cdot \sin(\Im(ct)) \cdot \Im(c)                              &  & (\Re(ct) = t \Re(c)) \\
          & \quad + i \cdot e^{\Re(ct)} \cdot \Re(c) \cdot \sin(\Im(ct))                      &  & (\Im(ct) = t \Im(c)) \\
          & \quad + i \cdot e^{\Re(ct)} \cdot \cos(\Im(ct)) \cdot \Im(c)                                                \\
          & = e^{\Re(ct)} \cdot \Re(c) \cdot (\cos(\Im(ct)) + i \sin(\Im(ct)))                                          \\
          & \quad + e^{\Re(ct)} \cdot \Im(c) \cdot (i \cos(\Im(ct)) - \sin(\Im(ct)))                                    \\
          & = e^{\Re(ct)} \cdot \Re(c) \cdot (\cos(\Im(ct)) + i \sin(\Im(ct)))                                          \\
          & \quad + e^{\Re(ct)} \cdot i \cdot \Im(c) \cdot (\cos(\Im(ct)) + i \sin(\Im(ct)))                            \\
          & = e^{\Re(ct)} \cdot (\Re(c) + i \Im(c)) \cdot (\cos(\Im(ct)) + i \sin(\Im(ct)))                             \\
          & = c \cdot e^{ct}.                                                                 &  & \by{2.7.8}
  \end{align*}
\end{proof}

\begin{thm}\label{2.30}
  The solution space for
  \[
    y' + a_0 y = \zv
  \]
  (homogeneous linear differential equation with order \(1\)) is of dimension \(1\) and has \(\set{e^{-a_0 t}}\) as a basis.
\end{thm}

\begin{proof}[\pf{2.30}]
  Since
  \begin{align*}
    (e^{-a_0 t})' + a_0 e^{-a_0 t} & = -a_0 e^{-a_0 t} + a_0 e^{-a_0 t} &  & \by{2.29} \\
                                   & = 0,
  \end{align*}
  we know that \(e^{-a_0 t}\) is a solution of \(y' + a_0 y = \zv\).
  Suppose that \(x(t)\) is any solution to \(y' + a_0 y = \zv\).
  Then
  \[
    \forall t \in \R, x'(t) = -a_0 x(t).
  \]
  Define
  \[
    \forall t \in \R, z(t) = e^{a_0 t} x(t).
  \]
  Differentiating \(z\) yields
  \[
    \forall t \in \R, z'(t) = (e^{a_0 t})' x(t) + e^{a_0 t} x'(t) = a_0 e^{a_0 t} x(t) - a_0 e^{a_0 t} x(t) = 0.
  \]
  (Notice that the familiar product rule for differentiation holds for complex-valued functions of a real variable.
  A justification of this involves a lengthy, although direct, computation.)

  Since \(z'\) is identically zero, \(z\) is a constant function.
  (Again, this fact, well known for real-valued functions, is also true for complex-valued functions.
  The proof, which relies on the real case, involves looking separately at the real and imaginary parts of \(z\).)
  Thus there exists a complex number \(k\) such that
  \[
    \forall t \in \R, z(t) = e^{a_0 t} x(t) = k
  \]
  So
  \[
    x(t) = k e^{-a_0 t}.
  \]
  We conclude that any solution to \(y' + a_0 y = \zv\) is a linear combination of \(e^{-a_0 t}\).
\end{proof}

\begin{cor}\label{2.7.10}
  For any complex number \(c\), the null space of the differential operator \(\Dop - c \IT\) has \(\set{e^{ct}}\) as a basis.
\end{cor}

\begin{proof}[\pf{2.7.10}]
  This is simply another way of stating \cref{2.30}.
\end{proof}

\begin{thm}\label{2.31}
  Let \(p\) be the auxiliary polynomial for a homogeneous linear differential equation with constant coefficients.
  For any complex number \(c\), if \(c\) is a zero of \(p\), then \(e^{ct}\) is a solution to the differential equation.
\end{thm}

\begin{proof}[\pf{2.31}]
  Given an \(n\)th order homogeneous linear differential equation with constant coefficients,
  \[
    y^{(n)} + a_{n - 1} y^{(n - 1)} + \cdots + a_1 y^{(1)} + a_0 y = \zv,
  \]
  its auxiliary polynomial
  \[
    p(t) = t^n + a_{n - 1} t^{n - 1} + \cdots + a_1 t + a_0
  \]
  factors into a product of polynomials of degree \(1\), that is,
  \[
    p(t) = (t - c_1) (t - c_2) \cdots (t - c_n),
  \]
  where \(\seq{c}{1,,n}\) are (not necessarily distinct) complex numbers.
  (This follows from the fundamental theorem of algebra, see \cref{d.0.7}.)
  Thus
  \[
    p(\Dop) = (\Dop - c_1 \IT) (\Dop - c_2 \IT) \cdots (\Dop - c_n \IT).
  \]
  The operators \(\Dop - c_i \IT\) commute, and so, by \cref{ex:2.7.9}, we have that
  \[
    \ns{\Dop - c_i \IT} \subseteq \ns{p(\Dop)} \quad \text{for all } i \in \set{1, \dots, n}.
  \]
  Since \(\ns{p(\Dop)}\) coincides with the solution space of the given differential equation, we conclude by \cref{2.7.10} that \cref{2.31} is true.
\end{proof}

\begin{lem}\label{2.7.11}
  The differential operator \(\Dop - c \IT : \cfs[\infty](\R, \C) \to \cfs[\infty](\R, \C)\) is onto for any complex number \(c\).
\end{lem}

\begin{proof}[\pf{2.7.11}]
  Let \(v \in \cfs[\infty](\R, \C)\).
  We wish to find a \(u \in \cfs[\infty](\R, \C)\) such that \((\Dop - c \IT) u = v\).
  Let \(w(t) = v(t) e^{-ct}\) for \(t \in \R\).
  Clearly, \(w \in \cfs[\infty](\R, \C)\) because both \(v\) and \(e^{-ct}\) lie in \(\cfs[\infty](\R, \C)\).
  Let \(w_1\) and \(w_2\) be the real and imaginary parts of \(w\).
  Then \(w_1\) and \(w_2\) are continuous because they are differentiable.
  Hence they have antiderivatives, say, \(W_1\) and \(W_2\), respectively.
  Let \(W : \R \to \C\) be defined by
  \[
    \forall t \in \R, W(t) = W_1(t) + i W_2(t).
  \]
  Then \(W \in \cfs[\infty](\R, \C)\), and the real and imaginary parts of \(W\) are \(W_1\) and \(W_2\), respectively.
  Furthermore, \(W' = w\).
  Finally, let \(u : \R \to \C\) be defined by \(u(t) = W(t) e^{ct}\) for \(t \in \R\).
  Clearly \(u \in \cfs[\infty](\R, \C)\), and since
  \begin{align*}
    (\Dop - c \IT) u(t) & = u'(t) - cu(t)                                               \\
                        & = W'(t) e^{ct} + W(t) c e^{ct} - c W(t) e^{ct} &  & \by{2.29} \\
                        & = w(t) e^{ct}                                                 \\
                        & = v(t) e^{-ct} e^{ct}                                         \\
                        & = v(t),
  \end{align*}
  we have \((\Dop - c \IT) u = v\).
\end{proof}

\begin{lem}\label{2.7.12}
  Let \(\V\) be a vector space over \(\F\), and suppose that \(\T\) and \(\U\) are linear operators on \(\V\) such that \(\U\) is onto and the null spaces of \(\T\) and \(\U\) are finite-dimensional.
  Then the null space of \(\T \U\) is finite-dimensional, and
  \[
    \dim(\ns{\T \U}) = \dim(\ns{\T}) + \dim(\ns{\U}).
  \]
\end{lem}

\begin{proof}[\pf{2.7.12}]
  Let \(p = \dim(\ns{\T})\), \(q = \dim(\ns{\U})\), and \(\set{\seq{u}{1,,p}}\) and \(\set{\seq{v}{1,,q}}\) be bases for \(\ns{\T}\) and \(\ns{\U}\) over \(\F\), respectively.
  Since \(\U\) is onto, we can choose for each \(i \in \set{1, \dots, p}\) a vector \(w_i \in \V\) such that \(\U(w_i) = u_i\).
  Note that the \(w_i\)'s are distinct.
  Furthermore, for any \(i\) and \(j\), \(w_i \neq v_j\), for otherwise \(u_i = \U(w_i) = \U(v_j) = \zv\) --- a contradiction.
  Hence the set
  \[
    \beta = \set{\seq{w}{1,,p}, \seq{v}{1,,q}}
  \]
  contains \(p + q\) distinct vectors.
  To complete the proof of the lemma, it suffices to show that \(\beta\) is a basis for \(\ns{\T \U}\) over \(\F\).

  We first show that \(\beta\) generates \(\ns{\T \U}\).
  Since for any \(w_i\) and \(v_j\) in \(\beta\), \(\T \U(w_i) = \T(u_i) = \zv\) and \(\T \U(v_j) = \T(\zv) = \zv\), it follows that \(\beta \subseteq \ns{\T \U}\).
  Now suppose that \(v \in \ns{\T \U}\).
  Then \(\zv = \T \U(v) = \T(\U(v))\).
  Thus \(\U(v) \in \ns{\T}\).
  So there exist scalars \(\seq{a}{1,,p} \in \F\) such that
  \begin{align*}
    \U(v) & = \seq[+]{a,u}{1,,p}                 \\
          & = a_1 \U(w_1) + \cdots + a_p \U(w_p) \\
          & = \U(\seq[+]{a,w}{1,,p}).
  \end{align*}
  Hence
  \[
    \U(v - (\seq[+]{a,w}{1,,p})) = \zv.
  \]
  Consequently, \(v - (\seq[+]{a,w}{1,,p})\) lies in \(\ns{\U}\).
  It follows that there exist scalars \(\seq{b}{1,,q} \in \F\) such that
  \[
    v - (\seq[+]{a,w}{1,,p}) = \seq[+]{b,v}{1,,q}
  \]
  or
  \[
    v = \seq[+]{a,w}{1,,p} + \seq[+]{b,v}{1,,q}.
  \]
  Therefore \(\beta\) spans \(\ns{\T \U}\).

  To prove that \(\beta\) is linearly independent, let \(\seq{a}{1,,p}, \seq{b}{1,,q} \in \F\) be any scalars such that
  \[
    \seq[+]{a,w}{1,,p} + \seq[+]{b,v}{1,,q} = \zv.
  \]
  Applying \(\U\) to both sides of the above equation, we obtain
  \[
    \seq[+]{a,u}{1,,p} = \zv.
  \]
  Since \(\set{\seq{u}{1,,p}}\) is linearly independent, the \(a_i\)'s are all zero.
  Thus the original equation reduces to
  \[
    \seq[+]{b,v}{1,,q} = \zv.
  \]
  Again, the linear independence of \(\set{\seq{v}{1,,q}}\) implies that the \(b_i\)'s are all zero.
  We conclude that \(\beta\) is a basis for \(\ns{\T \U}\) over \(\F\).
  Hence \(\ns{\T \U}\) is finite-dimensional, and \(\dim(\ns{\T \U}) = p + q = \dim(\ns{\T}) + \dim(\ns{\U})\).
\end{proof}

\begin{thm}\label{2.32}
  For any differential operator \(p(\Dop)\) of order \(n\), the null space of \(p(\Dop)\) is an \(n\)-dimensional subspace of \(\cfs[\infty](\R, \C)\).
\end{thm}

\begin{proof}[\pf{2.32}]
  The proof is by mathematical induction on the order of the differential operator \(p(\Dop)\).
  The first-order case coincides with \cref{2.30}.
  For some integer \(n > 1\), suppose that \cref{2.32} holds for any differential operator of order less than \(n\), and consider a differential operator \(p(\Dop)\) of order \(n\).
  The polynomial \(p\) can be factored into a product of two polynomials as follows:
  \[
    \forall t \in \R, p(t) = q(t) (t - c),
  \]
  where \(q\) is a polynomial of degree \(n - 1\) and \(c\) is a complex number.
  Thus the given differential operator may be rewritten as
  \[
    p(\Dop) = q(\Dop) (\Dop - c \IT).
  \]
  Now, by \cref{2.7.11}, \(\Dop - c \IT\) is onto, and by \cref{2.7.10}, \(\dim(\ns{\Dop - c \IT}) = 1\).
  Also, by the induction hypothesis, \(\dim(\ns{q(\Dop)}) = n - 1\).
  Thus, by \cref{2.7.12} we conclude that
  \[
    \dim(\ns{p(\Dop)}) = \dim(\ns{q(\Dop)}) + \dim(\ns{\Dop - c \IT}) = (n - 1) + 1 = n.
  \]
\end{proof}

\begin{cor}\label{2.7.13}
  The solution space of any \(n\)th-order homogeneous linear differential equation with constant coefficients is an \(n\)-dimensional subspace of \(\cfs[\infty](\R, \C)\).
\end{cor}

\begin{proof}[\pf{2.7.13}]
  By \cref{2.28,2.32} we conclude that \cref{2.7.13} is true.
\end{proof}

\begin{note}
  \cref{2.7.13} reduces the problem of finding all solutions to an \(n\)th-order homogeneous linear differential equation with constant coefficients to finding a set of \(n\) linearly independent solutions to the equation.
  By the results of \cref{ch:1}, any such set must be a basis for the solution space.
\end{note}

\begin{thm}\label{2.33}
  Given \(n\) distinct complex numbers \(\seq{c}{1,,n}\), the set of exponential functions \(\set{e^{c_1 t}, \dots, e^{c_n t}}\) is linearly independent.
\end{thm}

\begin{proof}[\pf{2.33}]
  We use induction on \(n\).
  For \(n = 1\), let \(b_1, c_1 \in \C\) such that
  \[
    \forall t \in \R, b_1 e^{c_1 t} = 0.
  \]
  By substituting \(t\) with \(0\) we have \(b_1 e^{c_1 0} = b_1 e^0 = b_1 = 0\).
  Thus by \cref{1.5.3} we know that the set \(\set{e^{c_1 t}}\) is linearly independent and the base case holds.
  Suppose inductively that \cref{2.33} is true for some \(n \geq 1\).
  We want to show that it is also true for \(n + 1\).
  Let \(\seq{c}{1,,n+1} \in \C\) be distinct and let \(\seq{b}{1,,n+1} \in \C\) such that
  \[
    \forall t \in \R, \sum_{i = 1}^{n + 1} b_i e^{c_i t} = 0.
  \]
  By applying the operator \(\Dop - c_{n + 1} \IT\) we have
  \begin{align*}
             & \forall t \in \R, \sum_{i = 1}^{n + 1} b_i c_i e^{c_i t} - \sum_{i = 1}^{n + 1} b_i c_{n + 1} e^{c_i t} = 0                                             \\
    \implies & \forall t \in \R, \sum_{i = 1}^n b_i c_i e^{c_i t} + b_{n + 1} c_{n + 1} e^{c_{n + 1} t}                                                                \\
             & - \sum_{i = 1}^n b_i c_{n + 1} e^{c_i t} - b_{n + 1} c_{n + 1} e^{c_{n + 1} t} = 0                                                                      \\
    \implies & \forall t \in \R, \sum_{i = 1}^n b_i c_i e^{c_i t} - \sum_{i = 1}^n b_i c_{n + 1} e^{c_i t} = 0                                                         \\
    \implies & \forall t \in \R, \sum_{i = 1}^n b_i (c_i - c_{n + 1}) e^{c_i t} = 0                                                                                    \\
    \implies & \seq[=]{b}{1,,n} = 0                                                                                        &  & \text{(by induction hypothesis)}       \\
    \implies & \forall t \in \R, \sum_{t = 1}^{n + 1} b_i e^{c_i t} = b_{n + 1} e^{c_{n + 1} t} = 0                                                                    \\
    \implies & b_{n + 1} = 0.                                                                                              &  & \text{(substituting \(t\) with \(0\))}
  \end{align*}
  Thus by \cref{1.5.3} we know that the set \(\set{e^{c_1 t}, \dots, e^{c_{n + 1} t}}\) is linearly independent and this closes the induction.
\end{proof}

\begin{cor}\label{2.7.14}
  For any \(n\)th-order homogeneous linear differential equation with constant coefficients, if the auxiliary polynomial has \(n\) distinct zeros \(\seq{c}{1,,n}\), then \(\set{e^{c_1 t}, \dots, e^{c_n t}}\) is a basis for the solution space of the differential equation.
\end{cor}

\begin{proof}[\pf{2.7.14}]
  By \cref{2.7.13,2.33} we see that this is true.
\end{proof}

\begin{lem}\label{2.7.15}
  For a given complex number \(c\) and positive integer \(n\), suppose that \((t - c)^n\) is the auxiliary polynomial of a homogeneous linear differential equation with constant coefficients.
  Then the set
  \[
    \beta = \set{e^{ct}, t e^{ct}, \dots, t^{n - 1} e^{ct}}
  \]
  is a basis for the solution space of the equation.
\end{lem}

\begin{proof}[\pf{2.7.15}]
  Since the solution space is \(n\)-dimensional, we only need to show that \(\beta\) is linearly independent and lies in the solution space.
  First, observe that for any positive integer \(k\),
  \[
    (\Dop - c \IT)(t^k e^{ct}) = k t^{k - 1} e^{ct} + c t^k e^{ct} - c t^k e^{ct} = k t^{k - 1} e^{ct}.
  \]
  Hence for \(k < n\),
  \[
    (\Dop - c \IT)^n (t^k e^{ct}) = \zv.
  \]
  It follows that \(\beta\) is a subset of the solution space.

  We next show that \(\beta\) is linearly independent.
  Consider any linear combination of vectors in \(\beta\) such that
  \[
    b_0 e^{ct} + b_1 t e^{ct} + \cdots + b_{n - 1} t^{n - 1} e^{ct} = \zv
  \]
  for some scalars \(\seq{b}{0,,n-1}\).
  Dividing by \(e^{ct}\) in the above equation, we obtain
  \[
    b_0 + b_1 t + \cdots + b_{n - 1} t^{n - 1} = \zv.
  \]
  Thus the left side of the above equation must be the zero polynomial function.
  We conclude that the coefficients \(\seq{b}{0,,n-1}\) are all zero.
  So \(\beta\) is linearly independent and hence is a basis for the solution space.
\end{proof}

\begin{thm}\label{2.34}
  Given a homogeneous linear differential equation with constant coefficients and auxiliary polynomial
  \[
    p(\Dop) = (t - c_1)^{n_1} (t - c_2)^{n_2} \cdots (t - c_k)^{n_k}
  \]
  where \(\seq{n}{1,,k}\) are positive integers and \(\seq{c}{1,,k}\) are distinct complex numbers, the following set is a basis for the solution space of the equation:
  \[
    \beta = \set{e^{c_1 t}, t e^{c_1 t}, \dots, t^{n_1 - 1} e^{c_1 t}, \dots, e^{c_k t}, t e^{c_k t}, \dots, t^{n_k - 1} e^{c_k t}}.
  \]
\end{thm}

\begin{proof}[\pf{2.34}]
  For any \(k \in \Z^+\), let \(\beta_k = \set{e^{c_k t}, \dots, t^{n_k - 1} e^{c_k t}}\).
  Observe that \(\beta = \bigcup_{i = 1}^k \beta_i\) and \(\#(\beta) = \seq[+]{n}{1,,k} = \sum_{i = 1}^k \#(\beta_i)\).
  Thus by \cref{2.7.13} we only need to show that \(\bigcup_{i = 1}^k \beta_i \subseteq \ns{p(\Dop)}\) and \(\bigcup_{i = 1}^k \beta_i\) is linearly independent.

  First we show that \(\bigcup_{i = 1}^k \beta_i \subseteq \ns{p(\Dop)}\).
  This is true since
  \begin{align*}
             & \forall i \in \set{1, \dots, k}, \beta_i \subseteq \ns{(\Dop - c_i \IT)^{n_i}} &  & \by{2.7.14}   \\
    \implies & \forall i \in \set{1, \dots, k}, \beta_i \subseteq \ns{p(\Dop)}                &  & \by{ex:2.7.9} \\
    \implies & \bigcup_{i = 1}^k \beta_i \subseteq \ns{p(\Dop)}.
  \end{align*}

  Now we show that \(\bigcup_{i = 1}^k \beta_i\) is linearly independent.
  We use induction on \(k\).
  For \(k = 1\), we have
  \[
    p(\Dop) = (t - c_1)^{n_1} \quad \text{and} \quad \beta_1 = \set{e^{c_1 t}, t e^{c_1 t}, \dots, t^{n_1 - 1} e^{c_1 t}}.
  \]
  By \cref{2.7.15} we know that \(\beta_1\) is linearly independent and thus the base case holds.
  Suppose inductively that \(\bigcup_{i = 1}^k \beta_i\) is linearly independent for some \(k \geq 1\).
  We need to show that when \(\bigcup_{i = 1}^{k + 1} \beta_i\) is also linearly independent.
  For each \(i \in \set{1, \dots, k + 1}\) and \(j \in \set{0, \dots, n_i - 1}\), let \(b_{i j} \in \C\) such that
  \[
    \sum_{i = 1}^{k + 1} \sum_{j = 0}^{n_i - 1} b_{i j} t^j e^{c_i t} = \zv.
  \]
  Observe that
  \begin{align*}
             & (\Dop - c_{k + 1} \IT)(t^j e^{c_i t}) = j t^{j - 1} e^{c_i t} + (c_i - c_{k + 1}) t^j e^{c_i t}                                                                                                             \\
    \implies & (\Dop - c_{k + 1} \IT)^{n_{k + 1}} \pa{t^j e^{c_i t}} = \sum_{q = 0}^{\min(n_{k + 1} - 1, j)} \binom{n_{k + 1}}{q} (c_i - c_{k + 1})^{n_{k + 1} - q} \dfrac{j!}{(j - q)!} t^{j - q} e^{c_i t}               \\
    \implies & \zv = (\Dop - c_{k + 1} \IT)^{n_{k + 1}} \pa{\sum_{i = 1}^{k + 1} \sum_{j = 0}^{n_i - 1} b_{i j} t^j e^{c_i t}} = \sum_{i = 1}^{k + 1} \sum_{j = 0}^{n_i - 1} b_{i j} (\Dop - c_{k + 1} \IT)(t^j e^{c_i t}) \\
             & = \sum_{i = 1}^{k + 1} \sum_{j = 0}^{n_i - 1} \sum_{q = 0}^{\min(n_{k + 1} - 1, j)} b_{i j} \binom{n_{k + 1}}{q} (c_i - c_{k + 1})^{n_{k + 1} - q} \dfrac{j!}{(j - q)!} t^{j - q} e^{c_i t}                 \\
             & = \sum_{i = 1}^k \sum_{j = 0}^{n_i - 1} \sum_{q = 0}^{\min(n_{k + 1} - 1, j)} b_{i j} \binom{n_{k + 1}}{q} (c_i - c_{k + 1})^{n_{k + 1} - q} \dfrac{j!}{(j - q)!} t^{j - q} e^{c_i t}.
  \end{align*}
  Note that the last line follow since \(q \leq n_{k + 1} - 1 < n_{k + 1}\).
  Since \(c_i - c_{k + 1} \neq 0\) for all \(i \in \set{1, \dots, k}\) and factorials are positive, by induction hypothesis we must have \(b_{i j} = 0\) for all \(i \in \set{1, \dots, k}\) and \(j \in \set{0, \dots n_i - 1}\).
  Thus we have
  \[
    \zv = \sum_{i = 1}^{k + 1} \sum_{j = 0}^{n_i - 1} b_{i j} t^j e^{c_i t} = \sum_{j = 0}^{n_{k + 1} - 1} b_{(k + 1) j} t^j e^{c_{k + 1} t}.
  \]
  By \cref{2.7.15} we know that \(b_{(k + 1) j} = 0\) for all \(j \in \set{0, \dots, n_{k + 1} - 1}\).
  Thus by \cref{1.5.3} \(\bigcup_{i = 1}^{k + 1} \beta_i\) is linearly independent and this closes the induction.
\end{proof}

\exercisesection

\setcounter{ex}{4}
\begin{ex}\label{ex:2.7.5}
  Show that \(\cfs[\infty](\R, \C)\) is a subspace of \(\fs(\R, \C)\).
\end{ex}

\begin{proof}[\pf{ex:2.7.5}]
  Clearly we have \(\cfs[\infty](\R, \C) \subseteq \fs(\R, \C)\).
  Since the zero function \(\zv\) is continously differentiable and \(\zv' = \zv\), we know that \(\zv \in \cfs[\infty](\R, \C)\).
  Thus by \cref{ex:1.3.18} we only need to show that \(cf + g \in \cfs[\infty](\R, \C)\) for any \(f, g \in \cfs[\infty](\R, \C)\) and \(c \in \C\).
  Since
  \[
    \forall k \in \N, (cf + g)^{(k)} = c f^{(k)} + g^{(k)},
  \]
  we see that \(cf + g \in \cfs[\infty](\R, \C)\).
\end{proof}

\begin{ex}\label{ex:2.7.6}
  \begin{enumerate}
    \item Show that \(\Dop : \cfs[\infty](\R, \C) \to \cfs[\infty](\R, \C)\) is a linear operator.
    \item Show that any differential operator is a linear operator on \(\cfs[\infty](\R, \C)\).
  \end{enumerate}
\end{ex}

\begin{proof}[\pf{ex:2.7.6}(a)]
  Let \(f, g \in \cfs[\infty](\R, \C)\) and let \(c \in \C\).
  Since
  \begin{align*}
    \Dop(cf + g) & = (cf + g)'            &  & \by{2.7.5} \\
                 & = c f' + g'                            \\
                 & = c \Dop(f) + \Dop(g), &  & \by{2.7.5}
  \end{align*}
  by \cref{2.1.2}(b) we know that \(\Dop \in \ls(\cfs[\infty](\R, \C))\).
\end{proof}

\begin{proof}[\pf{ex:2.7.6}(b)]
  Let
  \[
    p(\Dop) = a_n \Dop^n + \cdots + a_1 \Dop + a_0 \IT
  \]
  be a differential operator where \(\seq{a}{0,,n} \in \C\).
  Let \(f, g \in \cfs[\infty](\R, \C)\) and let \(c \in \C\).
  Since
  \begin{align*}
     & p(\Dop)(cf + g)                                                                                                       \\
     & = \pa{a_n \Dop^n + \cdots + a_1 \Dop + a_0 \IT}(cf + g)                                               &  & \by{2.7.5} \\
     & = a_n \Dop^n(cf + g) + \cdots + a_1 \Dop(cf + g) + a_0 \IT(cf + g)                                    &  & \by{2.2.5} \\
     & = a_n (cf + g)^{(n)} + \cdots + a_1 (cf + g)^{(1)} + a_0 (cf + g)                                                     \\
     & = c a_n f^{(n)} + a_n g^{(n)} + \cdots + c a_1 f^{(1)} + a_1 g^{(1)} + c a_0 f + a_0 g                                \\
     & = c \pa{a_n f^{(n)} + \cdots + a_1 f^{(1)} + a_0 f} + \pa{a_n g^{(n)} + \cdots + a_1 g^{(1)} + a_0 g}                 \\
     & = c p(\Dop)(f) + p(\Dop)(g),                                                                          &  & \by{2.7.5}
  \end{align*}
  by \cref{2.1.2}(b) we know that \(p(\Dop) \in \ls(\cfs[\infty](\R, \C))\).
\end{proof}

\begin{ex}\label{ex:2.7.7}
  Prove that if \(\set{x, y}\) is a basis for a vector space \(\V\) over \(\C\), then so is
  \[
    \set{\dfrac{1}{2} (x + y), \dfrac{1}{2i} (x - y)}.
  \]
\end{ex}

\begin{proof}[\pf{ex:2.7.7}]
  Let \(v \in \V\).
  By \cref{1.6.1} there exist \(\seq{c}{1,2} \in \C\) such that \(v = c_1 x + c_2 y\).
  Then we have
  \begin{align*}
    v & = c_1 x + c_2 y                                                                          \\
      & = \dfrac{c_1}{2} x + \dfrac{c_2}{2} y + \dfrac{c_1 i}{2 i} x - \dfrac{-c_2 i}{2 i} y     \\
      & \quad + \dfrac{c_2}{2} x + \dfrac{c_1}{2} y - \dfrac{c_2 i}{2 i} x - \dfrac{c_1 i}{2i} y \\
      & = \pa{\dfrac{c_1 + c_2}{2}} (x + y) + i \pa{\dfrac{c_1 - c_2}{2i}} (x - y).
  \end{align*}
  Thus by \cref{1.6.15}(a) we know that \(\set{\dfrac{1}{2} (x + y), \dfrac{1}{2i} (x - y)}\) is a basis for \(\V\) over \(\C\).
\end{proof}

\begin{ex}\label{ex:2.7.8}
  Consider a second-order homogeneous linear differential equation with constant coefficients in which the auxiliary polynomial has distinct conjugate complex roots \(a + ib\) and \(a - ib\), where \(a, b \in \R\).
  Show that \(\set{e^{at} \cos(bt), e^{at} \sin(bt)}\) is a basis for the solution space.
\end{ex}

\begin{proof}[\pf{ex:2.7.8}]
  By \cref{2.7.14} we know that \(\set{e^{(a + ib) t}, e^{(a - ib) t}}\) is a basis for the solution space.
  Since
  \begin{align*}
    e^{(a + ib) t} & = e^{at + ibt}                                     \\
                   & = e^{at} (\cos(bt) + i \sin(bt))   &  & \by{2.7.8} \\
    e^{(a - ib) t} & = e^{at - ibt}                                     \\
                   & = e^{at} (\cos(-bt) + i \sin(-bt)) &  & \by{2.7.8} \\
                   & = e^{at} (\cos(bt) - i \sin(bt))
  \end{align*}
  By \cref{ex:2.7.7} we know that the set
  \[
    \set{\dfrac{1}{2} (e^{(a + ib) t} + e^{(a - ib) t}), \dfrac{1}{2i} (e^{(a + ib) t} - e^{(a - ib) t})} = \set{e^{at} \cos(bt), e^{at} \sin(bt)}
  \]
  is also a basis for the solution space.
\end{proof}

\begin{ex}\label{ex:2.7.9}
  Suppose that \(\set{\seq{\U}{1,,n}}\) is a collection of pairwise commutative linear operators on a vector space \(\V\) over \(\F\)
  (i.e., operators such that \(\U_i \U_j = \U_j \U_i\) for all \(i, j \in \set{1, \dots, n}\)).
  Prove that, for any \(i \in \set{1, \dots, n}\),
  \[
    \ns{\U_i} \subseteq \ns{\U_1 \cdots \U_n}.
  \]
\end{ex}

\begin{proof}[\pf{ex:2.7.9}]
  Let \(i \in \set{1, \dots, n}\) and let \(x \in \ns{\U_i}\).
  Then we have
  \begin{align*}
             & \U_i(x) = \zv                                                                   &  & \by{2.1.10}                   \\
    \implies & (\U_1 \cdots \U_n)(x) = (\U_1 \cdots \U_{i - 1} \U_{i + 1} \cdots \U_n \U_i)(x) &  & \text{(pairwise commutative)} \\
             & = (\U_1 \cdots \U_{i - 1} \U_{i + 1} \cdots \U_n)(\U_i(x))                                                         \\
             & = (\U_1 \cdots \U_{i - 1} \U_{i + 1} \cdots \U_n)(\zv)                                                             \\
             & = \zv                                                                           &  & \by{2.1.2}[a]                 \\
    \implies & x \in \ns{\U_1 \cdots \U_n}                                                     &  & \by{2.1.10}
  \end{align*}
  and thus \(\ns{\U_i} \subseteq \ns{\U_1 \cdots \U_n}\).
\end{proof}

\setcounter{ex}{11}
\begin{ex}\label{ex:2.7.12}
  Let \(\V\) be the solution space of an \(n\)th-order homogeneous linear differential equation with constant coefficients having auxiliary polynomial \(p\).
  Prove that if \(p(t)\) = \(g(t) h(t)\) for all \(t \in \R\), where \(g\) and \(h\) are polynomials of positive degree, then
  \[
    \ns{h(\Dop)} = \rg{g(\Dop_{\V})} = g(\Dop)(\V),
  \]
  where \(\Dop_{\V} : \V \to \V\) is defined by \(\Dop_{\V}(x) = x'\) for \(x \in \V\).
\end{ex}

\begin{proof}[\pf{ex:2.7.12}]
  By \cref{2.28} we have \(\V = \ns{p(\Dop)}\).
  By \cref{e.0.7} we have \(\rg{g(\Dop_{\V})} = g(\Dop)(\V) = g(\Dop)(\ns{p(\Dop)})\).
  Since
  \begin{align*}
             & y \in g(\Dop)(\ns{p(\Dop)})                                                    \\
    \implies & \exists x \in \ns{p(\Dop)} : g(\Dop)(x) = y                                    \\
    \implies & \exists x \in \ns{p(\Dop)} : h(\Dop)(y) = h(\Dop)(g(\Dop)(x))                  \\
             & = (h(\Dop) g(\Dop))(x) = p(\Dop)(x) = \zv                     &  & \by{2.1.10} \\
    \implies & y \in \ns{h(\Dop)},                                           &  & \by{2.1.10}
  \end{align*}
  we know that \(g(\Dop)(\ns{p(\Dop)}) \subseteq \ns{h(\Dop)}\).
  If we can show that \(\dim(g(\Dop)(\ns{p(\Dop)})) = \dim(\ns{h(\Dop)})\), then by \cref{1.11} we can prove that \(g(\Dop)(\ns{p(\Dop)}) = \ns{h(\Dop)}\).
  This is true since
  \begin{align*}
     & \dim(g(\Dop)(\ns{p(\Dop)}))                                                     \\
     & = \rk{g(\Dop_{\V})}                                            &  & \by{2.1.10} \\
     & = \dim(\ns{p(\Dop)}) - \nt{g(\Dop_{\V})}                       &  & \by{2.3}    \\
     & = (\text{order of } p(\Dop)) - (\text{order of } g(\Dop_{\V})) &  & \by{2.32}   \\
     & = (\text{degree of } p) - (\text{degree of } g)                                 \\
     & = \text{order of } h(\Dop)                                                      \\
     & = \dim(\ns{h(\Dop)}).                                          &  & \by{2.32}
  \end{align*}
\end{proof}

\begin{defn}\label{2.7.16}
  A differential equation
  \[
    y^{(n)} + a_{n - 1} y^{(n - 1)} + \cdots + a_1 y^{(1)} + a_0 y = x
  \]
  is called a \textbf{nonhomogeneous} linear differential equation with constant coefficients if the \(a_i\)'s are constant and \(x\) is a function that is not identically zero.
\end{defn}

\begin{ex}\label{ex:2.7.13}
  Let
  \[
    y^{(n)} + a_{n - 1} y^{(n - 1)} + \cdots + a_1 y^{(1)} + a_0 y = x
  \]
  be a nonhomogeneous linear differential equation with constant coefficients.
  \begin{enumerate}
    \item Prove that for any \(x \in \cfs[\infty](\R, \C)\) there exists \(y \in \cfs[\infty](\R, \C)\) such that \(y\) is a solution to the differential equation.
    \item Let \(\V\) be the solution space for the homogeneous linear equation
          \[
            y^{(n)} + a_{n - 1} y^{(n - 1)} + \cdots + a_1 y^{(1)} + a_0 y = \zv.
          \]
          Prove that if \(z\) is any solution to the associated nonhomogeneous linear differential equation, then the set of all solutions to the nonhomogeneous linear differential equation is
          \[
            \set{z + y : y \in \V}.
          \]
  \end{enumerate}
\end{ex}

\begin{proof}[\pf{ex:2.7.13}(a)]
  By \cref{d.0.7} there exist some \(\seq{c}{1,,n} \in \C\) (not necessarily distinct) such that
  \[
    y^{(n)} + a_{n - 1} y^{(n - 1)} + \cdots + a_1 y^{(1)} + a_0 y = \pa{\prod_{i = 1}^n (\Dop - c_i \IT)}(y).
  \]
  Then we have
  \begin{align*}
             & \forall i \in \set{1, \dots, n}, \Dop - c_i \IT \text{ is onto}                                                     &  & \by{2.7.11} \\
    \implies & \prod_{i = 1}^n (\Dop - c_i \IT) = (\Dop - c_n \IT)(\Dop - c_{n - 1} \IT( \cdots (\Dop - c_1 \IT))) \text{ is onto}                  \\
    \implies & y^{(n)} + a_{n - 1} y^{(n - 1)} + \cdots + a_1 y^{(1)} + a_0 y \text{ is onto}.
  \end{align*}
\end{proof}

\begin{proof}[\pf{ex:2.7.13}(b)]
  We have
  \begin{align*}
             & \begin{dcases}
                 z^{(n)} + a_{n - 1} z^{(n - 1)} + \cdots + a_1 z^{(1)} + a_0 z = x \\
                 y^{(n)} + a_{n - 1} y^{(n - 1)} + \cdots + a_1 y^{(1)} + a_0 y = \zv
               \end{dcases}                                                        \\
    \implies & z^{(n)} + y^{(n)} + a_{n - 1} (z^{(n - 1)} + y^{(n - 1)}) + \cdots + a_1 (z^{(1)} + y^{(1)}) + a_0 (z^{(0)} + y^{(0)}) = x \\
    \implies & (z + y)^{(n)} + a_{n - 1} (z + y)^{(n - 1)} + \cdots + a_1 (z + y)^{(1)} + a_0 (z + y) = x.
  \end{align*}
  Thus \(z + y\) is a solution to the nonhomogeneous linear differential equation with constant coefficients.
\end{proof}

\begin{ex}\label{ex:2.7.14}
  Given any \(n\)th-order homogeneous linear differential equation with constant coefficients, prove that, for any solution \(x\), if there exists a \(t_0 \in \R\) such that \(x(t_0) = x'(t_0) = \cdots = x^{(n - 1)}(t_0) = 0\), then \(x = 0\) (the zero function).
\end{ex}

\begin{proof}[\pf{ex:2.7.14}]
  Let \(p\) be the auxiliary polynomial of the homogeneous linear differential equation with order \(n\).
  We use induction on \(n\).
  For \(n = 1\), we have
  \begin{align*}
             & \exists c \in \C : \forall t \in \R, p(t) = t - c       &  & \by{d.0.7}         \\
    \implies & \exists c \in \C : e^{ct} \in \ns{p(\Dop)}              &  & \by{2.31}          \\
    \implies & \exists c, k \in \C : \forall t \in \R, x(t) = k e^{ct} &  & \by{2.7.13}        \\
    \implies & \exists c, k \in \C : \begin{dcases}
                                       0 = x(t_0) = k e^{c t_0} \\
                                       0 = x'(t_0) = k c e^{c t_0}
                                     \end{dcases}                                \\
    \implies & x(t_0) = 0                                              &  & (e^{c t_0} \neq 0)
  \end{align*}
  and thus the base case holds.
  Suppose inductively that \cref{ex:2.7.14} is true for some \(n \geq 1\).
  We need to show that for \(n + 1\) \cref{ex:2.7.14} is also true.
  Let \(p\) be of order \(n + 1\).
  By \cref{d.0.7} we know that there exist a polynomial \(q\) with order \(n\) and a \(c \in \C\) such that
  \[
    \forall t \in \R, p(t) = q(t) (t - c).
  \]
  Let \(z = q(\Dop)(x)\).
  Since
  \begin{align*}
    (\Dop - c \IT)(z) & = (\Dop - c \IT)(q(\Dop)(x))                                   \\
                      & = 0                          &  & \text{(\(x\) is a solution)}
  \end{align*}
  we know that \(z \in \ns{\Dop - c \IT}\).
  Since
  \begin{align*}
    z(t_0) & = (q(\Dop)(x))(t_0)                                                  \\
           & = 0                        &  & (x(t_0) = \cdots = x^{(n)}(t_0) = 0) \\
           & = ((\Dop - c \IT)(z))(t_0) &  & (z \in \ns{\Dop - c \IT})            \\
           & = z'(t_0) - c z(t_0)                                                 \\
           & = z'(t_0)                  &  & (z(t_0) = 0)
  \end{align*}
  and \(\Dop - c \IT\) has order \(1\), by induction hypothesis we know that \(z = \zv\).
  Thus we have \(z = q(\Dop)(x) = \zv\).
  Since \(q(\Dop)\) has order \(n\), by induction hypothesis we know that \(x = \zv\).
\end{proof}

\begin{ex}\label{ex:2.7.15}
  Let \(\V\) be the solution space of an \(n\)th-order homogeneous linear differential equation with constant coefficients.
  Fix \(t_0 \in \R\), and define a mapping \(\Phi : \V \to \C^n\) by
  \[
    \Phi(x) = \begin{pmatrix}
      x(t_0)  \\
      x'(t_0) \\
      \vdots  \\
      x^{(n - 1)}(t_0)
    \end{pmatrix} \text{ for each } x \in \V.
  \]
  \begin{enumerate}
    \item Prove that \(\Phi\) is linear and its null space is the zero subspace of \(\V\).
          Deduce that \(\Phi\) is an isomorphism.
    \item Prove the following:
          For any \(n\)th-order homogeneous linear differential equation with constant coefficients, any \(t_0 \in \R\), and any complex numbers \(\seq{c}{0,,n-1}\) (not necessarily distinct), there exists exactly one solution, \(x\), to the given differential equation such that \(x(t_0) = c_0\) and \(x^{(k)}(t_0) = c_k\) for \(k \in \set{1, \dots, n - 1}\).
  \end{enumerate}
\end{ex}

\begin{proof}[\pf{ex:2.7.15}(a)]
  Let \(x, y \in \V\) and let \(c \in \C\).
  Since
  \begin{align*}
    \Phi(cx + y) & = \begin{pmatrix}
                       (cx + y)(t_0)  \\
                       (cx + y)'(t_0) \\
                       \vdots         \\
                       (cx + y)^{(n - 1)}(t_0)
                     \end{pmatrix}              \\
                 & = \begin{pmatrix}
                       cx(t_0) + y(t_0)   \\
                       cx'(t_0) + y'(t_0) \\
                       \vdots             \\
                       cx^{(n - 1)}(t_0) + y^{(n - 1)}(t_0)
                     \end{pmatrix} \\
                 & = c \begin{pmatrix}
                         x(t_0)  \\
                         x'(t_0) \\
                         \vdots  \\
                         x^{(n - 1)}(t_0)
                       \end{pmatrix} + \begin{pmatrix}
                                         y(t_0)  \\
                                         y'(t_0) \\
                                         \vdots  \\
                                         y^{(n - 1)}(t_0)
                                       \end{pmatrix}   \\
                 & = c \Phi(x) + \Phi(y),
  \end{align*}
  by \cref{2.1.2}(b) we know that \(\Phi \in \ls(\V, \C^n)\).
  By \cref{2.1.2}(a) and \cref{ex:2.7.14} we know that \(\Phi(x) = \zv_{\C^n}\) iff \(x\) is the zero function, thus by \cref{2.4} we know that \(\Phi\) is one-to-one.
  By \cref{2.32} we know that \(\dim(\V) = n = \dim(\C^n)\), thus by \cref{2.5,2.4.8} we know that \(\Phi\) is an isomorphism.
\end{proof}

\begin{proof}[\pf{ex:2.7.15}(b)]
  Since \(\Phi\) is an isomorphism (by \cref{ex:2.7.15}(a)), we know that there exists an \(x \in \V\) such that
  \[
    \Phi(x) = \begin{pmatrix}
      x(t_0)  \\
      x'(t_0) \\
      \vdots  \\
      x^{(n - 1)}(t_0)
    \end{pmatrix} = \begin{pmatrix}
      c_0    \\
      c_1    \\
      \vdots \\
      c_{n - 1}
    \end{pmatrix}.
  \]
\end{proof}

\begin{ex}[Pendular Motion]\label{ex:2.7.16}
  It is well known that the motion of a pendulum is approximated by the differential equation
  \[
    \theta'' + \dfrac{g}{l} \theta = \zv,
  \]
  where \(\theta(t)\) is the angle in radians that the pendulum makes with a vertical line at time \(t\), interpreted so that \(\theta\) is positive if the pendulum is to the right and negative if the pendulum is to the left of the vertical line as viewed by the reader.
  Here \(l\) is the length of the pendulum and \(g\) is the magnitude of acceleration due to gravity.
  The variable \(t\) and constants \(l\) and \(g\) must be in compatible units
  (e.g., \(t\) in seconds, \(l\) in meters, and \(g\) in meters per second per second).
  \begin{enumerate}
    \item Express an arbitrary solution to this equation as a linear combination of two real-valued solutions.
    \item Find the unique solution to the equation that satisfies the conditions
          \[
            \theta(0) = \theta_0 > 0 \quad \text{and} \quad \theta'(0) = 0.
          \]
          (The significance of these conditions is that at time \(t = 0\) the pendulum is released from a position displaced from the vertical by \(\theta_0\).)
    \item Prove that it takes \(2 \pi \sqrt{l / g}\) units of time for the pendulum to make one circuit back and forth.
          (This time is called the \textbf{period} of the pendulum.)
  \end{enumerate}
\end{ex}

\begin{proof}[\pf{ex:2.7.16}(a)]
  Since
  \[
    t^2 + \dfrac{g}{l} = \pa{t - i \sqrt{\dfrac{g}{l}}} \pa{t + i \sqrt{\dfrac{g}{l}}} = 0 \implies t = \pm i \sqrt{\dfrac{g}{l}},
  \]
  by \cref{2.7.14} we know that
  \begin{align*}
    \forall t \in \R, \theta(t) & = a e^{i \sqrt{\dfrac{g}{l}} t} + b e^{- i \sqrt{\dfrac{g}{l}} t}                        \\
                                & = a \cos\pa{\sqrt{\dfrac{g}{l}} t} + b \sin\pa{\sqrt{\dfrac{g}{l}} t} &  & \by{ex:2.7.8}
  \end{align*}
  for some \(a, b \in \C\).
\end{proof}

\begin{proof}[\pf{ex:2.7.16}(b)]
  Since
  \begin{align*}
    \theta_0 & = \theta(0)                                                                                                                                                   \\
             & = a \cos\pa{\sqrt{\dfrac{g}{l}} 0} + b \sin\pa{\sqrt{\dfrac{g}{l}} 0}                                                                  &  & \by{ex:2.7.16}[a] \\
             & = a                                                                                                                                                           \\
    0        & = \theta'(0)                                                                                                                                                  \\
             & = -a \cdot \sin\pa{\sqrt{\dfrac{g}{l}} 0} \cdot \sqrt{\dfrac{g}{l}} + b \cdot \cos\pa{\sqrt{\dfrac{g}{l}} 0} \cdot \sqrt{\dfrac{g}{l}} &  & \by{ex:2.7.16}[a] \\
             & = b \sqrt{\dfrac{g}{l}},
  \end{align*}
  we know that \(a = \theta_0\) and \(b = 0\).
  Thus we have
  \[
    \forall t \in \R, \theta(t) = \theta_0 \cos\pa{\sqrt{\dfrac{g}{l}} t}
  \]
\end{proof}

\begin{proof}[\pf{ex:2.7.16}(c)]
  Since
  \begin{align*}
    \theta(0) & = a \cos\pa{\sqrt{\dfrac{g}{l}} 0} + b \sin\pa{\sqrt{\dfrac{g}{l}} 0}                                                              &  & \by{ex:2.7.16}[a] \\
              & = a \cos(0) + b \sin(0)                                                                                                                                   \\
              & = a \cos(2 \pi) + b \sin(2 \pi)                                                                                                                           \\
              & = a \cos\pa{\sqrt{\dfrac{g}{l}} \cdot 2 \pi \sqrt{\dfrac{l}{g}}} + b \sin\pa{\sqrt{\dfrac{g}{l}}  \cdot 2 \pi \sqrt{\dfrac{l}{g}}}                        \\
              & = \theta\pa{2 \pi \sqrt{\dfrac{l}{g}}},                                                                                            &  & \by{ex:2.7.16}[a]
  \end{align*}
  we know that the period of \(\theta\) is \(2 \pi \sqrt{\dfrac{l}{g}}\).
\end{proof}

\begin{ex}[Periodic Motion of a Spring without Damping]\label{ex:2.7.17}
  Find the general solution to
  \[
    y'' + \dfrac{k}{m} y = \zv,
  \]
  which describes the periodic motion of a spring, ignoring frictional forces.
\end{ex}

\begin{proof}[\pf{ex:2.7.17}]
  Since
  \[
    t^2 + \dfrac{k}{m} = \pa{t - i \sqrt{\dfrac{k}{m}}} \pa{t + i \sqrt{\dfrac{k}{m}}} = 0 \implies t = \pm i \sqrt{\dfrac{k}{m}},
  \]
  we know that
  \begin{align*}
    \forall t \in \R, y(t) & = a e^{i \sqrt{\dfrac{k}{m}} t} + b e^{-i \sqrt{\dfrac{k}{m}} t}      &  & \by{2.7.14}   \\
                           & = a \cos\pa{\sqrt{\dfrac{k}{m}} t} + b \sin\pa{\sqrt{\dfrac{k}{m}} t} &  & \by{ex:2.7.8}
  \end{align*}
  for some \(a, b \in \C\).
\end{proof}

\begin{ex}[Periodic Motion of a Spring with Damping]\label{ex:2.7.18}
  The ideal periodic motion described by solutions to \cref{ex:2.7.17} is due to the ignoring of frictional forces.
  In reality, however, there is a frictional force acting on the motion that is proportional to the speed of motion, but that acts in the opposite direction.
  The modification of \cref{ex:2.7.17} to account for the frictional force, called the \emph{damping force}, is given by
  \[
    m y'' + r y' + ky = \zv,
  \]
  where \(r > 0\) is the proportionality constant.
  \begin{enumerate}
    \item Find the general solution to this equation.
    \item Find the unique solution in (a) that satisfies the initial conditions \(y(0) = 0\) and \(y'(0) = v_0\), the initial velocity.
    \item For \(y(t)\) as in (b), show that the amplitude of the oscillation decreases to zero;
          that is, prove that \(\Lim_{t \to \infty} y(t) = 0\).
  \end{enumerate}
\end{ex}

\begin{proof}[\pf{ex:2.7.18}(a)]
  Since
  \[
    t^2 + \dfrac{r}{m} t + \dfrac{k}{m} = 0 \implies t = \dfrac{-r \pm \sqrt{r^2 - 4mk}}{2m},
  \]
  by \cref{2.7.14} we know that
  \[
    \forall t \in \R, y(t) = a e^{\pa{\dfrac{-r + \sqrt{r^2 - 4mk}}{2m}} t} + b e^{\pa{\dfrac{-r - \sqrt{r^2 - 4mk}}{2m}} t}
  \]
  for some \(a, b \in \C\).
\end{proof}

\begin{proof}[\pf{ex:2.7.18}(b)]
  Since
  \begin{align*}
    0   & = y(0)                                                                                                        \\
        & = a + b                                                                                &  & \by{ex:2.7.18}[a] \\
    v_0 & = y'(0)                                                                                                       \\
        & = a \pa{\dfrac{-r + \sqrt{r^2 - 4mk}}{2m}} + b \pa{\dfrac{-r - \sqrt{r^2 - 4mk}}{2m}},
  \end{align*}
  we have
  \[
    a = \dfrac{v_0 m}{\sqrt{r^2 - 4mk}} \quad \text{and} \quad b = \dfrac{-v_0 m}{\sqrt{r^2 - 4mk}}
  \]
  and thus
  \[
    \forall y \in \R, y(t) = \dfrac{v_0 m}{\sqrt{r^2 - 4mk}} e^{\pa{\dfrac{-r + \sqrt{r^2 - 4mk}}{2m}} t} - \dfrac{v_0 m}{\sqrt{r^2 - 4mk}} e^{\pa{\dfrac{-r - \sqrt{r^2 - 4mk}}{2m}} t}.
  \]
\end{proof}

\begin{proof}[\pf{ex:2.7.18}(c)]
  We split into two cases:
  \begin{itemize}
    \item If \(r^2 - 4mk \geq 0\), then we have
          \begin{align*}
                     & r^2 > r^2 - 4mk                                                                                                                         &  & (mk > 0)          \\
            \implies & r > \sqrt{r^2 - 4mk} \geq -\sqrt{r^2 - 4mk}                                                                                             &  & (r > 0)           \\
            \implies & 0 > -r + \sqrt{r^2 - 4mk} \geq -r - \sqrt{r^2 - 4mk}                                                                                                           \\
            \implies & 0 > \dfrac{-r + \sqrt{r^2 - 4mk}}{2m} \geq \dfrac{-r - \sqrt{r^2 - 4mk}}{2m}                                                            &  & (m > 0)           \\
            \implies & \Lim_{t \to \infty} e^{\pa{\dfrac{-r + \sqrt{r^2 - 4mk}}{2m}} t} = \Lim_{t \to \infty} e^{\pa{\dfrac{-r - \sqrt{r^2 - 4mk}}{2m}} t} = 0                        \\
            \implies & \Lim_{t \to \infty} y(t) = 0.                                                                                                           &  & \by{ex:2.7.18}[b]
          \end{align*}
    \item If \(r^2 - 4mk < 0\), then we have
          \begin{align*}
             & \abs{e^{\pa{\dfrac{-r + \sqrt{r^2 - 4mk}}{2m}} t}}                                                                                              \\
             & = \abs{e^{\dfrac{-rt}{2m}} \pa{\cos\pa{\dfrac{\sqrt{r^2 - 4mk}}{2m} t} + i \sin\pa{\dfrac{\sqrt{r^2 - 4mk}}{2m} t}}} &  & \by{2.7.8}            \\
             & \leq \abs{e^{\dfrac{-rt}{2m}}}                                                                                       &  & (\cos^2 + \sin^2 = 1)
          \end{align*}
          and
          \begin{align*}
             & \abs{e^{\pa{\dfrac{-r - \sqrt{r^2 - 4mk}}{2m}} t}}                                                                                                \\
             & = \abs{e^{\dfrac{-rt}{2m}} \pa{\cos\pa{\dfrac{-\sqrt{r^2 - 4mk}}{2m} t} + i \sin\pa{\dfrac{-\sqrt{r^2 - 4mk}}{2m} t}}} &  & \by{2.7.8}            \\
             & \leq \abs{e^{\dfrac{-rt}{2m}}}.                                                                                        &  & (\cos^2 + \sin^2 = 1)
          \end{align*}
          Thus
          \begin{align*}
                     & \Lim_{t \to \infty} \abs{e^{\pa{\dfrac{-r - \sqrt{r^2 - 4mk}}{2m}} t}} = \Lim_{t \to \infty} \abs{e^{\pa{\dfrac{-r - \sqrt{r^2 - 4mk}}{2m}} t}} = 0 \\
            \implies & \Lim_{t \to \infty} e^{\pa{\dfrac{-r - \sqrt{r^2 - 4mk}}{2m}} t} = \Lim_{t \to \infty} e^{\pa{\dfrac{-r - \sqrt{r^2 - 4mk}}{2m}} t} = 0             \\
            \implies & \Lim_{t \to \infty} y(t) = 0.
          \end{align*}
  \end{itemize}
  From all cases above we conclude that \(\Lim_{t \to \infty} y(t) = 0\).
\end{proof}

\begin{ex}\label{ex:2.7.19}
  In our study of differential equations, we have regarded solutions as complex-valued functions even though functions that are useful in describing physical motion are real-valued.
  Justify this approach.
\end{ex}

\begin{proof}[\pf{ex:2.7.19}]
  Every solution to a homogeneous linear differential equation with constant coefficients is in the space of \(\cfs[\infty](\R, \C)\), and we know that \(\cfs[\infty](\R, \R) \subseteq \cfs[\infty](\R, \C)\).
  Therefore, if we are only interesting with real-valued solutions, we can just pick them from the solution space.
\end{proof}



%------------------------------------------------------------------------------
% Back matters.
%------------------------------------------------------------------------------

\backmatter

\end{document}

% We use chapter structure.
\documentclass[12pt,oneside]{book}

%==============================================================================
% Preamble.
%==============================================================================

% Correctly showing characters outside ASCII.
\usepackage[T1]{fontenc}
% File is written and read with utf8 encoding.
\usepackage[utf8]{inputenc}
% Set paging layout.
\usepackage[margin=1.2in]{geometry}
% Including `amsfonts'.  Must be loaded before `mathtools'.
\usepackage{amssymb}
% Including `amsmath' and fixing bugs for `amsmath'.
\usepackage{mathtools}
% Must be loaded after `amsmath' and `mathtools'.
\usepackage{amsthm}
% Define page header and footer layout.
\usepackage{fancyhdr}
% LaTex 3 new command tools.
\usepackage{xparse}

% Use `fancyhdr' page style.  All page style settings must be called only after
% this command.  See `fancyhdr' for details.
\pagestyle{fancy}

% Make header height enough to fit chapter / section titles.
\setlength{\headheight}{15pt}
% Change chapter and section marks' formatting.  Note that I use `\markright'
% to make sure header use chapter information when section information is not
% available (this is need for appendix).
\renewcommand{\chaptermark}[1]{\markright{\textsf{\small Chap. \thechapter \quad #1}}}
\renewcommand{\sectionmark}[1]{\markright{\textsf{\small Sec. \thesection \quad #1}}}
% Cleanup page header settings.
\fancyhead{}
% Put page number on the left of page header.
\fancyhead[L]{\textbf{\textsf{\small \thepage}}}
% Put chapter / section information on the right of page header.
\fancyhead[R]{\rightmark}
% Cleanup page footer settings.
\fancyfoot{}

% Reset `plain' page style which is used for the first page of each chapter.
% I set this to make page number style consistent.
\fancypagestyle{plain}{%
  \fancyhf{}% clear all fields
  \fancyfoot[C]{\textbf{\textsf{\small \thepage}}}%
  \renewcommand{\headrulewidth}{0pt}%
}

% Automatically adjust character spacing at margins.
\usepackage{microtype}
% Provide further utilities and fix bugs for `enumerate', `itemize' and
% `description'.
\usepackage{enumitem}
% Provide better quoting environment.
\usepackage{dirtytalk}
% Parsing list inside `\newcommand'.
\usepackage{listofitems}
% Nice looking if-then-else structure with comparison functionality.
\usepackage{ifthen}
% Automatically add hyperlinks to labels/refs.  Must be loaded after all
% packages above and before `cleveref'.  Recommend to use with `natbib' when
% you need bibtex.
\usepackage{hyperref}

\hypersetup{       % This macro come with `hyperref'.
	colorlinks=true, % Color hyperlinks.
	linkcolor=blue,  % Color local hyperlinks with blue.
	urlcolor=cyan,   % Color url links with cyan.
}

% Must be loaded after `hyperref'.  We always capitalize each cross-references'
% type name.  See `cleveref' for details.
\usepackage[capitalize]{cleveref}

% Allow page break in the middle of multi-line equations.
\allowdisplaybreaks

%------------------------------------------------------------------------------
% Define environments.
%------------------------------------------------------------------------------

% Text inside the body of theorem-like environments are set to Roman font.
% theorem-like environments share their counters, counters follow section and
% reset in every sections (except for theorems, theorems counters are reset for
% each chapter).  Theorems and exercises has their owned counter.  Notes do not
% use counter.  See `amsthm' for details.
\theoremstyle{definition}
\newtheorem{ax}{Ax.}[section]
\newtheorem{cor}[ax]{Cor.}
\newtheorem{defn}[ax]{Def.}
\newtheorem{eg}[ax]{E.g.}
\newtheorem{ex}{Ex.}[section]
\newtheorem{lem}[ax]{Lem.}
\newtheorem{prop}[ax]{Prop.}
\newtheorem{thm}{Thm.}[chapter]
\newtheorem*{note}{Note}

% Define plural form for theorem-like environments.  This is needed when we
% call `cref' with multiple references.  See `cleveref' for details.
\crefname{ax}{Ax.}{Ax.}
\crefname{cor}{Cor.}{Cor.}
\crefname{defn}{Def.}{Def.}
\crefname{eg}{E.g.}{E.g.}
\crefname{ex}{Ex.}{Ex.}
\crefname{lem}{Lem.}{Lem.}
\crefname{note}{Note}{Note}
\crefname{prop}{Prop.}{Prop.}
\crefname{section}{Sec.}{Sec.}
\crefname{chapter}{Ch.}{Ch.}
\crefname{thm}{Thm.}{Thm.}

% Proof environments reference text.
\NewDocumentCommand{\pf}{m}{%
	Proof of \cref{#1}%
}
% Proof statements reference text.
\NewDocumentCommand{\byOptionalArgumentProcess}{m}{(#1)}
\NewDocumentCommand{\by}{m >{\SplitList{,}} o}{%
	\IfNoValueTF{#2}{%
		\text{(by \cref{#1})}%
	}{%
		\text{(by \cref{#1}\ProcessList{#2}{\byOptionalArgumentProcess})}%
	}%
}

% In `enumerate' enviroments, items' label are alphabets and surrounded by
% parentheses.  See `enumitem' for details.
\renewcommand{\labelenumi}{\textnormal{(}\alph{enumi}\textnormal{)}}

% Formatting equations tag appearence.  See `mathtools' for details.
\renewcommand{\theequation}{\thechapter.\thesection.\arabic{equation}}
\numberwithin{equation}{section}

%------------------------------------------------------------------------------
% Define operators and symbols.
%------------------------------------------------------------------------------

% Define common operators with paired of delimiters.  Always use star versions
% of these operator to automatically adjust height.  See `mathtools' for
% details.

% Absolute value.
\DeclarePairedDelimiter{\absTmp}{\lvert}{\rvert}
\NewDocumentCommand{\abs}{m}{\absTmp*{#1}}
% Ceiling.
\DeclarePairedDelimiter{\ceilTmp}{\lceil}{\rceil}
\NewDocumentCommand{\ceil}{m}{\ceilTmp*{#1}}
% Floor.
\DeclarePairedDelimiter{\floorTmp}{\lfloor}{\rfloor}
\NewDocumentCommand{\floor}{m}{\floorTmp*{#1}}
% Evaluate.
\DeclarePairedDelimiter{\evalTmp}{.}{\rvert}
\NewDocumentCommand{\eval}{m}{\evalTmp*{#1}}
% Inner product.
\DeclarePairedDelimiter{\innTmp}{\langle}{\rangle}
\NewDocumentCommand{\inn}{m}{\innTmp*{#1}}
% Norm.
\DeclarePairedDelimiter{\normTmp}{\lVert}{\rVert}
\NewDocumentCommand{\norm}{m}{\normTmp*{#1}}
% Parenthese.
\DeclarePairedDelimiter{\paTmp}{\lparen}{\rparen}
\NewDocumentCommand{\pa}{m}{\paTmp*{#1}}
% Bracket.
\DeclarePairedDelimiter{\brTmp}{\lbrack}{\rbrack}
\NewDocumentCommand{\br}{m}{\brTmp*{#1}}
% Brace.
\DeclarePairedDelimiter{\BTmp}{\lbrace}{\rbrace}
\NewDocumentCommand{\B}{m}{\BTmp*{#1}}
% Set.
\NewDocumentCommand{\set}{m}{\B{#1}}

% Define common symbols.  See `amsmath' section 9.2 for details.

% Fields.
\NewDocumentCommand{\field}{m}{\mathit{#1}}
% General field.
\NewDocumentCommand{\F}{}{\field{F}}
% Complex number field.
\NewDocumentCommand{\C}{}{\mathbb{C}}
% Natural number field.
\NewDocumentCommand{\N}{}{\mathbb{N}}
% Rational number field.
\NewDocumentCommand{\Q}{}{\mathbb{Q}}
% Real number field.
\NewDocumentCommand{\R}{}{\mathbb{R}}
% Integer number field.
\NewDocumentCommand{\Z}{}{\mathbb{Z}}

% Complex conjugate.
\NewDocumentCommand{\conj}{m}{\overline{#1}}

% Vector space.
\NewDocumentCommand{\vs}{m}{\mathsf{#1}}
% General vector space.
\NewDocumentCommand{\V}{}{\vs{V}}
\NewDocumentCommand{\W}{}{\vs{W}}
% Collection of vector space.
\NewDocumentCommand{\cvs}{}{\mathcal{C}}

% Metric space with shape #1 x #2 over field #3.
\NewDocumentCommand{\ms}{O{m}O{n}O{\F}}{\vs{M}_{{#1} \times {#2}}\pa{#3}}

% Function space.
\NewDocumentCommand{\fs}{}{\mathcal{F}}
% General function space.
\NewDocumentCommand{\FS}{}{\fs(S, \F)}

% Continuous function space with continuous #1-th derivative.
\NewDocumentCommand{\cfs}{o}{%
	\IfNoValueTF{#1}{%
		\vs{C}%
	}{%
		\vs{C}^{#1}%
	}%
}

% Polynomial spaces of degree #1 over field #2.
% #1 (optional) is the degree of polynomial.
% #2 is field.
%
% For example:
% If we use \(\ps{\F}\), we will get
%
%    \vs{P}(\F)
%
% If we use \(\ps[n]{\F}\), we will get
%
%    \vs{P}_{n}(\F)
%
\NewDocumentCommand{\ps}{om}{%
	\IfNoValueTF{#1}{%
		\vs{P}\pa{#2}%
	}{%
		\vs{P}_{#1}\pa{#2}%
	}%
}

% Derivative operation.
\NewDocumentCommand{\Dop}{}{\mathsf{D}}

% Sequence.
% #1 is a join operator (defaults to comma).
% #2 is a comma-separated list of sequence symbols.
% #3 is a comma-separated list of sequence index.
%
% For example:
% If we use \(\seq{a}{1,2,3}\), we will get
%
%    a_{1}, a_{2}, a_{3}
%
% If we use \(\seq[+]{a,b}{1,2,3}\), we will get
%
%    a_{1} b_{1} + a_{2} b_{2} + a_{3} b_{3}
%
% If we use \(\seq{a}{1}\), we will get
%
%    a_{1}
%
% If we use \(\seq{a}{1,,n}\), we will get
%
%    a_{1}, a_{2}, \dots, a_{n}
%
\NewDocumentCommand{\seq}{O{,} mm}{%
	\setsepchar{,}%                              List items are separated by comma.
	\readlist\SeqSymbols{#2}%                    Define macro `\SeqSymbols' using #1.
	\readlist\SeqIndices{#3}%                    Define macro `\SeqIndices' using #2.
	\foreachitem\SeqIndex\in\SeqIndices{%        Loop over indices.
		\ifthenelse{%                              Output join operator if `\SeqIndex' is not the first index.
			\equal{\SeqIndexcnt}{1}
		}{%
		}{%
			#1
		}%
		%--------------------------------------------------------------------------
		\ifthenelse{%                              Output dots if `\SeqIndex' is empty.
			\equal{\SeqIndex}{}%
		}%
		{%
			\ifthenelse{%                            If #1 is `,' use `\dots', otherwise use `\cdots'.
				\equal{#1}{,}%
			}%
			{\dots}%
			{\cdots}%
		}%
		{%
			\foreachitem\SeqSymbol\in\SeqSymbols{%   Loop over symbols.
				{\SeqSymbol}_{\SeqIndex}%              Output symbols with index.
			}%
		}%
	}%
}

% Tuple.
\NewDocumentCommand{\tuple}{mm}{\pa{\seq{#1}{#2}}}

% 0 vector.
\NewDocumentCommand{\zv}{}{\mathit{0}}
% 0 metric.
\NewDocumentCommand{\zm}{}{\mathit{O}}

% Transpose of matrix #1.
\NewDocumentCommand{\tp}{m}{{#1}^{t}}

% Trace of matrix #1.
\NewDocumentCommand{\tr}{o}{%
	\IfNoValueTF{#1}{%
		\operatorname{tr}%
	}{%
		\operatorname{tr}\pa{#1}%
	}%
}

% Span of a set.
% We cannot use `\span' since it is a tex primitive.
\NewDocumentCommand{\spn}{m}{\operatorname{span}\pa{#1}}

% Linear transformation.
\NewDocumentCommand{\lt}{m}{\mathsf{#1}}
% Linear transformation T.
\NewDocumentCommand{\T}{}{\lt{T}}
% Linear transformation U.
\NewDocumentCommand{\U}{}{\lt{U}}
% Identity transformation I.
\NewDocumentCommand{\IT}{o}{%
	\IfNoValueTF{#1}{%
		\lt{I}%
	}{%
		\lt{I}_{#1}%
	}%
}
% Zero transformation T_0.
\NewDocumentCommand{\zT}{}{\T_0}
% Left-multiplication transformation L_A.
\RenewDocumentCommand{\L}{}{\lt{L}}

% Null space N(T).
\NewDocumentCommand{\ns}{m}{\mathsf{N}\pa{#1}}

% Range R(T).
\NewDocumentCommand{\rg}{m}{\mathsf{R}\pa{#1}}

% Nullity nullity(T).
\NewDocumentCommand{\nt}{m}{\operatorname{nullity}\pa{#1}}

% Rank rank(T).
\NewDocumentCommand{\rk}{m}{\operatorname{rank}\pa{#1}}

% Vector spaces of linear transformations.
% For example:
% 	\ls(\V) or \ls(\V, \W)
\NewDocumentCommand{\ls}{}{\mathcal{L}}

% Make limit always in display mode.
\NewDocumentCommand{\Lim}{}{\lim\limits}

% Formatting exercises section.
\NewDocumentCommand{\exercisesection}{}{
	\begin{center}
		\textbf{EXERCISES}
	\end{center}
}

%==============================================================================
% Document.
%==============================================================================

\begin{document}

%------------------------------------------------------------------------------
% Front matters.
%------------------------------------------------------------------------------

\frontmatter

% Author informations.
\title{Linear Algebra}
\author{ProFatXuanAll}
\maketitle

% Table of contents.
\tableofcontents

% Notations explanation.
\chapter*{Notations}

Almost all statements (defintions, examples, theorems, etc.) in this book are wrapped inside environments.
The following table provide the name of each environment and their meanings.

\begin{table}[h]
  \centering
  \begin{tabular}{|c|c|}
    \hline
    Environment         & Meaning     \\
    \hline
    \namecref{1.1.1}    & Axiom       \\
    \hline
    \namecref{ch:1}     & Chapter     \\
    \hline
    \namecref{1.2.14}   & Corollary   \\
    \hline
    \namecref{1.2.1}    & Definition  \\
    \hline
    \namecref{1.2.4}    & Example     \\
    \hline
    \namecref{ex:1.2.8} & Exercise    \\
    \hline
    \namecref{2.4.5}    & Lemma       \\
    \hline
    Note                & Note        \\
    \hline
    \namecref{2.1.2}    & Proposition \\
    \hline
    \namecref{sec:1.1}  & Section     \\
    \hline
    \namecref{1.1}      & Theorem     \\
    \hline
  \end{tabular}
\end{table}


\addcontentsline{toc}{chapter}{Notations}

%------------------------------------------------------------------------------
% Main matters.
%------------------------------------------------------------------------------

\mainmatter

% All chapters are in separated files.  We include them here.
\chapter{Vector Spaces}\label{ch:1}

% All sections are in separated files.  We include them here.
\section{Introduction}\label{sec:1.1}

\begin{note}
    Experiments show that if two like quantities act together, their effect is predictable.
    In this case, the vectors used to represent these quantities can be combined to form a resultant vector that represents the combined effects of the original quantities.
    This resultant vector is called the \emph{sum} of the original vectors, and the rule for their combination is called the \emph{parallelogram law}.
\end{note}

\begin{axiom}[Parallelogram Law for Vector Addition]\label{ax:1.1.1}
    The sum of two vectors \(x\) and \(y\) that act at the same point \(P\) is the vector beginning at \(P\) that is represented by the diagonal of parallelogram having \(x\) and \(y\) as adjacent sides.
\end{axiom}

\begin{note}
    Since a vector beginning at the origin is completely determined by its endpoint, we sometimes refer to \emph{the point \(x\)} rather than \emph{the endpoint of the vector \(x\)} if \(x\) is a vector emanating from the origin.
\end{note}

\begin{note}
    Besides the operation of vector addition, there is another natural operation that can be performed on vectors
    --- the length of a vector may be magnified or contracted.
    This operation, called \emph{scalar multiplication}, consists of multiplying the vector by a real number.
    If the vector \(x\) is represented by an arrow, then for any real number \(t\), the vector \(tx\) is represented by an arrow in the same direction if \(t \geq 0\) and in the opposite direction if \(t < 0\).
    The length of the arrow \(tx\) is \(\abs*{t}\) times the length of the arrow \(x\).
    Two nonzero vectors \(x\) and \(y\) are called \textbf{parallel} if \(y = tx\) for some nonzero real number \(t\).
    (Thus nonzero vectors having the same or opposite directions are parallel.)
\end{note}
\section{Vector Spaces}\label{sec:1.2}

\begin{defn}\label{1.2.1}
    A \textbf{vector space} (or \textbf{linear space}) \(\V\) over a field \(\F\) consists of a set on which two operations (called \textbf{addition} and \textbf{scalar multiplication}, respectively) are defined so that for each pair of elements \(x\), \(y\) in \(\V\) there is a unique element \(x + y\) in \(\V\), and for each element \(a\) in \(\F\) and each element \(x\) in \(\V\) there is a unique element \(ax\) in \(\V\), such that the following conditions hold.
    \begin{enumerate}[label=(VS \arabic*), ref=VS \arabic*]
        \item\label{vs1}
        For all \(x, y\) in \(\V\), \(x + y = y + x\)
        (commutativity of addition).
        \item\label{vs2}
        For all \(x, y, z\) in \(\V\), \(\p{x + y} + z = x + \p{y + z}\)
        (associativity of addition).
        \item\label{vs3}
        There exists an element in \(\V\) denoted by \(0\) such that \(x + 0 = x\) for each \(x\) in \(\V\).
        \item\label{vs4}
        For each element \(x\) in \(\V\) there exists an element \(y\) in \(\V\) such that \(x + y = 0\).
        \item\label{vs5}
        For each element \(x\) in \(\V\), \(1x = x\).
        \item\label{vs6}
        For each pair of elements \(a, b\) in \(\F\) and each element \(x\) in \(\V\), \(\p{ab} x = a \p{bx}\).
        \item\label{vs7}
        For each element \(a\) in \(\F\) and each pair of elements \(x, y\) in \(\V\), \(a \p{x + y} = ax + ay\).
        \item\label{vs8}
        For each pair of elements \(a, b\) in \(\F\) and each element \(x\) in \(\V\), \(\p{a + b} x = ax + bx\).
    \end{enumerate}
    The elements \(x + y\) and \(ax\) are called the \textbf{sum} of \(x\) and \(y\) and the \textbf{product} of \(a\) and \(x\), respectively.
\end{defn}

\begin{defn}\label{1.2.2}
    The elements of the field \(\F\) are called \textbf{scalars} and the elements of the vector space \(\V\) are called \textbf{vectors}.
\end{defn}

\begin{note}
    A vector space is frequently discussed in the text without explicitly mentioning its field of scalars.
    The reader is cautioned to remember, however, that \emph{every vector space is regarded as a vector space over a given field, which is denoted by \(\F\)}.
    Occasionally we restrict our attention to the fields of real and complex numbers, which are denoted \(\R\) and \(\C\), respectively.
\end{note}

\begin{note}
    \ref{vs2} permits us to unambiguously define the addition of any finite number of vectors
    (without the use of parentheses).
\end{note}

\begin{defn}\label{1.2.3}
    An object of the form \(\tp{a}{n}\), where the entries \(a_{1}, a_{2}, \dots, a_{n}\) are elements of a field \(\F\), is called an \textbf{\(n\)-tuple} with entries from \(\F\).
    The elements \(a_{1}, a_{2}, \dots, a_{n}\) are called the \textbf{entries} or \textbf{components} of the \(n\)-tuple.
    Two \(n\)-tuples \(\tp{a}{n}\) and \(\tp{b}{n}\) with entries from a field \(\F\) are called \textbf{equal} if \(a_i = b_i\) for \(i = 1, 2, \dots, n\).
\end{defn}

\begin{eg}\label{1.2.4}
    The set of all \(n\)-tuples with entries from a field \(\F\) is denoted by \(\vs{F}^{n}\).
    This set is a vector space over \(\F\) with the operations of coordinatewise addition and scalar multiplication;
    that is, if \(u = \tp{a}{n} \in \vs{F}^{n}\), \(v = \tp{b}{n} \in \vs{F}^{n}\), and \(c \in \F\), then
    \[
        u + v = (a_{1} + b_{1}, a_{2} + b_{2}, \dots, a_{n} + b_{n}) \quad \text{ and } \quad cu = \tp{ca}{n}.
    \]
\end{eg}

\begin{proof}
    Clearly we have
    \[
        \forall u, v \in \vs{F}^{n}, u + v \in \vs{F}^{n}
    \]
    and
    \[
        \begin{dcases}
            \forall u \in \vs{F}^{n} \\
            \forall c \in \F
        \end{dcases}, cu \in \vs{F}^{n}.
    \]

    Let \(I_{n} = \set{i \in \N : 1 \leq i \leq n}\).
    First we show that addition and scalar multiplication in \(\vs{F}^{n}\) over \(\F\) are unique.
    Suppose that \(u, u', v, v' \in \vs{F}^{n}\) such that \(u = u'\) and \(v = v'\).
    Then we have
    \begin{align*}
                 & \p{u = u'} \land \p{v = v'}                                                                                 \\
        \implies & \forall i \in I_{n}, \begin{dcases}
            u_{i} = u_{i}' \\
            v_{i} = v_{i}'
        \end{dcases}       &  & \text{(this is the definition of \(\vs{F}^{n}\))} \\
        \implies & \forall i \in I_{n}, u_{i} + v_{i} = u_{i}' + v_{i}' &  & \text{(\(\F\) is a field)}                        \\
        \implies & u + v = u' + v'                                      &  & \text{(this is the definition of \(\vs{F}^{n}\))}
    \end{align*}
    and thus the addition in \(\vs{F}^{n}\) over \(\F\) is unique.
    Now suppose that \(u, u' \in \vs{F}^{n}\) and \(c, c' \in \F\) such that \(u = u'\) and \(c = c'\).
    Then we have
    \begin{align*}
                 & \p{u = u'} \land \p{c = c'}                                                                     \\
        \implies & \begin{dcases}
            \forall i \in I_{n}, u_{i} = u_{i}' \\
            c = c'
        \end{dcases}                &  & \text{(this is the definition of \(\vs{F}^{n}\))} \\
        \implies & \forall i \in I_{n}, c u_{i} = c' u_{i}' &  & \text{(\(\F\) is a field)}                        \\
        \implies & cu = c' u'                               &  & \text{(this is the definition of \(\vs{F}^{n}\))}
    \end{align*}
    and thus the scalar multiplication in \(\vs{F}^{n}\) over \(\F\) is unique.

    Now we show that \ref{vs1}--\ref{vs8} holds for \cref{1.2.4}.
    \begin{description}
        \item[For \ref{vs1}:]
            For all \(x, y \in \vs{F}^{n}\), we have
            \begin{align*}
                x + y & = \p{x_{1} + y_{1}, x_{2} + y_{2}, \dots, x_{n} + y_{n}} &  & \text{(by \cref{1.2.4})}   \\
                      & = \p{y_{1} + x_{1}, y_{2} + x_{2}, \dots, y_{n} + x_{n}} &  & \text{(\(\F\) is a field)} \\
                      & = y + x.                                                 &  & \text{(by \cref{1.2.4})}
            \end{align*}
        \item[For \ref{vs2}:]
            For all \(x, y, z \in \vs{F}^{n}\), we have
            \begin{align*}
                 & \p{x + y} + z                                                                                                                \\
                 & = \p{x_{1} + y_{1}, x_{2} + y_{2}, \dots, x_{n} + y_{n}} + z                                 &  & \text{(by \cref{1.2.4})}   \\
                 & = \p{\p{x_{1} + y_{1}} + z_{1}, \p{x_{2} + y_{2}} + z_{2}, \dots, \p{x_{n} + y_{n}} + z_{n}} &  & \text{(by \cref{1.2.4})}   \\
                 & = \p{x_{1} + \p{y_{1} + z_{1}}, x_{2} + \p{y_{2} + z_{2}}, \dots, x_{n} + \p{y_{n} + z_{n}}} &  & \text{(\(\F\) is a field)} \\
                 & = x + \p{y_{1} + z_{1}, y_{2} + z_{2}, \dots, y_{n} + z_{n}}                                 &  & \text{(by \cref{1.2.4})}   \\
                 & = x + \p{y + z}.                                                                             &  & \text{(by \cref{1.2.4})}
            \end{align*}
        \item[For \ref{vs3}:]
            Since \(\F\) is a field, we know that \(0 \in \F\) and thus \(\p{0, \dots, 0} \in \vs{F}^{n}\).
            Then for all \(x \in \vs{F}^{n}\), we have
            \begin{align*}
                x + \p{0, \dots, 0} & = \p{x_{1} + 0, x_{2} + 0, \dots, x_{n} + 0} &  & \text{(by \cref{1.2.4})}   \\
                                    & = \tp{x}{n}                                  &  & \text{(\(\F\) is a field)} \\
                                    & = x.
            \end{align*}
            We denote \(\zv = \p{0, \dots, 0}\).
        \item[For \ref{vs4}:]
            For all \(x \in \vs{F}^{n}\), we have
            \begin{align*}
                         & \forall i \in I_{n}, x_{i} \in \F                                                                                 \\
                \implies & \forall i \in I_{n}, \exists y_{i} \in \F : x_{i} + y_{i} = 0                  &  & \text{(\(\F\) is a field)}    \\
                \implies & \exists y_{1}, y_{2}, \dots, y_{n} \in \F :                                                                       \\
                         & \p{x_{1} + y_{1}, x_{2} + y_{2}, \dots, x_{n} + y_{n}} = \p{0, \dots, 0} = \zv                                    \\
                \implies & \exists y \in \vs{F}^{n} : x + y = \zv.                                        &  & \text{(from the proof above)}
            \end{align*}
        \item[For \ref{vs5}:]
            Since \(\F\) is a field, we know that \(1 \in \F\).
            Then for all \(x \in \vs{F}^{n}\), we have
            \begin{align*}
                1x & = \p{1 x_{1}, 1 x_{2}, \dots, 1 x_{n}} &  & \text{(by \cref{1.2.4})}   \\
                   & = \tp{x}{n}                            &  & \text{(\(\F\) is a field)} \\
                   & = x.
            \end{align*}
        \item[For \ref{vs6}:]
            For all \(a, b \in \F\) and \(x \in \vs{F}^{n}\), we have
            \begin{align*}
                \p{ab} x & = \p{\p{ab} x_{1}, \p{ab} x_{2}, \dots, \p{ab} x_{n}}    &  & \text{(by \cref{1.2.4})}   \\
                         & = \p{a \p{b x_{1}}, a \p{b x_{2}}, \dots, a \p{b x_{n}}} &  & \text{(\(\F\) is a field)} \\
                         & = a \p{b x_{1}, b x_{2}, \dots, b x_{n}}                 &  & \text{(by \cref{1.2.4})}   \\
                         & = a \p{bx}.                                              &  & \text{(by \cref{1.2.4})}
            \end{align*}
        \item[For \ref{vs7}:]
            For all \(a \in \F\) and \(x, y \in \vs{F}^{n}\), we have
            \begin{align*}
                a \p{x + y} & = a \p{x_{1} + y_{1}, x_{2} + y_{2}, \dots, x_{n} + y_{n}}                    &  & \text{(by \cref{1.2.4})}   \\
                            & = \p{a \p{x_{1} + y_{1}}, a \p{x_{2} + y_{2}}, \dots, a \p{x_{n} + y_{n}}}    &  & \text{(by \cref{1.2.4})}   \\
                            & = \p{a x_{1} + a y_{1}, a x_{2} + a y_{2}, \dots, a x_{n} + a y_{n}}          &  & \text{(\(\F\) is a field)} \\
                            & = \p{a x_{1}, a x_{2}, \dots, a x_{n}} + \p{a y_{1}, a y_{2}, \dots, a y_{n}} &  & \text{(by \cref{1.2.4})}   \\
                            & = a x + a y.                                                                  &  & \text{(by \cref{1.2.4})}
            \end{align*}
        \item[For \ref{vs8}:]
            For all \(a, b \in \F\) and \(x \in \vs{F}^{n}\), we have
            \begin{align*}
                \p{a + b} x & = \p{\p{a + b} x_{1}, \p{a + b} x_{2}, \dots, \p{a + b} x_{n}}                &  & \text{(by \cref{1.2.4})}   \\
                            & = \p{a x_{1} + b x_{1}, a x_{2} + b x_{2}, \dots, a x_{n} + b x_{n}}          &  & \text{(\(\F\) is a field)} \\
                            & = \p{a x_{1}, a x_{2}, \dots, a x_{n}} + \p{b x_{1}, b x_{2}, \dots, b x_{n}} &  & \text{(by \cref{1.2.4})}   \\
                            & = a x + b x.                                                                  &  & \text{(by \cref{1.2.4})}
            \end{align*}
    \end{description}
    From all proofs above we conclude by \cref{1.2.1} that \cref{1.2.4} is indeed a vector space.
\end{proof}

\section{Subspaces}\label{sec:1.3}

\begin{defn}\label{1.3.1}
  A subset \(\W\) of a vector space \(\V\) over a field \(\F\) is called a \textbf{subspace} of \(\V\) if \(\W\) is a vector space over \(\F\) with the operations of addition and scalar multiplication defined on \(\V\).
\end{defn}

\begin{eg}\label{1.3.2}
  In any vector space \(\V\), note that \(\V\) and \(\set{\zv}\) are subspaces.
  The latter is called the \textbf{zero subspace} of \(\V\).
\end{eg}

\begin{proof}
  Since \(\V \subseteq \V\) and \(\V\) is a vector space over \(\F\) with the operations of addition and scalar multiplication defined on \(\V\), by \cref{1.3.1} we know that \(\V\) is a subspace of \(\V\).
  Since \(\zv \in \V\) (by \ref{vs3}), we know that \(\set{\zv} \subseteq \V\).
  Thus by \cref{ex:1.2.11} and \cref{1.3.1} \(\set{\zv}\) is a subspace of \(\V\).
\end{proof}
\section{Linear Combinations and Systems of Linear Equations}\label{sec:1.4}

\begin{defn}\label{1.4.1}
  Let \(\V\) be a vector space over \(\F\) and \(S\) a nonempty subset of \(\V\).
  A vector \(v \in \V\) is called a \textbf{linear combination} of vectors of \(S\) if there exist a finite number of vectors \(\seq{u}{1,2,,n}\) in \(S\) and scalars \(\seq{a}{1,2,,n}\) in \(\F\) such that \(v = \seq[+]{a,u}{1,2,,n}\).
  In this case we also say that \(v\) is a linear combination of \(\seq{u}{1,2,,n}\) and call \(\seq{a}{1,2,,n}\) the \textbf{coefficients} of the linear combination.
\end{defn}

\begin{eg}\label{1.4.2}
  Observe that in any vector space \(\V\), \(0v = \zv\) for each \(v \in \V\).
  Thus the zero vector is a linear combination of any nonempty subset of \(\V\).
\end{eg}

\begin{defn}\label{1.4.3}
  Let \(S\) be a nonempty subset of a vector space \(\V\) over \(\F\).
  The \textbf{span} of \(S\), denoted \(\spn{S}\), is the set consisting of all linear combinations of the vectors in \(S\).
  For convenience, we define \(\spn{\varnothing} = \set{\zv}\).
\end{defn}

\begin{thm}\label{1.5}
  The span of any subset \(S\) of a vector space \(\V\) over \(\F\) is a subspace of \(\V\) over \(\F\).
  Moreover, any subspace of \(\V\) over \(\F\) that contains \(S\) must also contain the span of \(S\).
\end{thm}

\begin{proof}
  This result is immediate if \(S = \varnothing\) because \(\spn{\varnothing} = \set{\zv}\), which is a subspace that is contained in any subspace of \(\V\) over \(\F\)
  (see \cref{1.3.2}).

  If \(S \neq \varnothing\), then \(S\) contains a vector \(z\).
  So \(0z = \zv\) is in \(\spn{S}\).
  Let \(x, y \in \spn{S}\).
  Then there exist vectors \(\seq{u}{1,2,,m}\), \(\seq{v}{1,2,,n}\) in \(S\) and scalars \(\seq{a}{1,2,,m}\), \(\seq{b}{1,2,,n}\) in \(\F\) such that
  \[
    x = \seq[+]{a,u}{1,2,,m} \quad \text{and} \quad y = \seq[+]{b,v}{1,2,,n}
  \]
  Then
  \[
    x + y = \seq[+]{a,u}{1,2,,m} + \seq[+]{b,v}{1,2,,n}
  \]
  and, for any scalar \(c \in \F\),
  \[
    cx = (ca_1) u_1 + (ca_2) u_2 + \cdots + (ca_m) u_m
  \]
  are clearly linear combinations of the vectors in \(S\);
  so \(x + y\) and \(cx\) are in \(\spn{S}\).
  Thus \(\spn{S}\) is a subspace of \(\V\) over \(\F\) (by \cref{1.3}).

  Now let \(\W\) denote any subspace of \(\V\) over \(\F\) that contains \(S\).
  If \(w \in \spn{S}\), then \(w\) has the form \(w = \seq[+]{c,w}{1,2,,k}\) for some vectors \(\seq{w}{1,2,,k}\) in \(S\) and some scalars \(\seq{c}{1,2,,k}\).
  Since \(S \subseteq \W\), we have \(\seq{w}{1,2,,k} \in \W\).
  Therefore \(w = \seq[+]{c,w}{1,2,,k}\) is in \(\W\) by \cref{ex:1.3.20} of \cref{sec:1.3}.
  Because \(w\), an arbitrary vector in \(\spn{S}\), belongs to \(\W\), it follows that \(\spn{S} \subseteq \W\).
\end{proof}

\begin{defn}\label{1.4.4}
  A subset \(S\) of a vector space \(\V\) over \(\F\) \textbf{generates} (or \textbf{spans}) \(\V\) if \(\spn{S} = \V\).
  In this case, we also say that the vectors of \(S\) generate (or span) \(\V\).
\end{defn}

\begin{note}
  Usually there are many different subsets that generate a subspace \(\W\).
  It is natural to seek a subset of \(\W\) that generates \(\W\) and is as small as possible.
\end{note}

\section{Linear Dependence and Linear Independence}\label{sec:1.5}

\begin{note}
  Suppose that \(\V\) is a vector space over an infinite field and that \(\W\) is a subspace of \(\V\).
  Unless \(\W\) is the zero subspace, \(\W\) is an infinite set.
  It is desirable to find a ``small'' finite subset \(S\) that generates \(\W\) because we can then describe each vector in \(\W\) as a linear combination of the finite number of vectors in \(S\).
  Indeed, the smaller that \(S\) is, the fewer computations that are required to represent vectors in \(\W\).

  Checking that some vector in \(S\) is a linear combination of the other vectors in \(S\) could require that we solve several different systems of linear equations before we determine which, if any, vectors in \(S\) is a linear combination of the others.

  Because some vector in \(S\) is a linear combination of the others, the zero vector can be expressed as a linear combination of the vectors in \(S\) using coefficients that are not all zero.
  The converse of this statement is also true:
  If the zero vector can be written as a linear combination of the vectors in \(S\) in which not all the coefficients are zero, then some vector in \(S\) is a linear combination of the others.
  Thus, rather than asking whether some vector in \(S\) is a linear combination of the other vectors in \(S\), it is more efficient to ask whether the zero vector can be expressed as a linear combination of the vectors in \(S\) with coefficients that are not all zero.
\end{note}

\begin{defn}\label{1.5.1}
  A subset \(S\) of a vector space \(\V\) over \(\F\) is called \textbf{linearly dependent} if there exist a finite number of distinct vectors \(\seq{u}{1,,n}\) in \(S\) and scalars \(\seq{a}{1,,n}\) in \(\F\), not all zero, such that
  \[
    \seq[+]{a,u}{1,,n} = \zv.
  \]
  In this case we also say that the vectors of \(S\) are linearly dependent.
\end{defn}

\begin{defn}\label{1.5.2}
  For any vectors \(\seq{u}{1,,n}\), we have \(\seq[+]{a,u}{1,,n} = \zv\) if \(\seq[=]{a}{1,,n} = 0\).
  We call this the \textbf{trivial representation} of \(\zv\) as a linear combination of \(\seq{u}{1,,n}\).
  Thus, for a set to be linearly dependent, there must exist a nontrivial representation of \(\zv\) as a linear combination of vectors in the set.
  Consequently, any subset of a vector space that contains the zero vector is linearly dependent, because \(\zv = 1 \cdot \zv\) is a nontrivial representation of \(\zv\) as a linear combination of vectors in the set.
\end{defn}

\begin{defn}\label{1.5.3}
  A subset \(S\) of a vector space over \(\F\) that is not linearly dependent is called \textbf{linearly independent}.
  As before, we also say that the vectors of \(S\) are linearly independent.
\end{defn}

\begin{eg}\label{1.5.4}
  The following facts about linearly independent sets are true in any vector space.
  \begin{enumerate}
    \item The empty set is linearly independent, for linearly dependent sets must be nonempty.
    \item A set consisting of a single nonzero vector is linearly independent.
          For if \(\set{u}\) is linearly dependent, then \(au = \zv\) for some nonzero scalar \(a\).
          Thus
          \[
            u = a^{-1} (au) = a^{-1} \zv = \zv.
          \]
    \item A set is linearly independent iff the only representations of \(\zv\) as linear combinations of its vectors are trivial representations.
  \end{enumerate}
\end{eg}

\begin{eg}\label{1.5.5}
  For \(k = 0, 1, \dots, n\) let \(p_k(x) = x^k + x^{k + 1} + \cdots + x^n\).
  The set
  \[
    \set{p_0(x), p_1(x), \dots, p_n(x)}
  \]
  is linearly independent in \(\ps[n]{\F}\).
  For if
  \[
    a_0 p_0(x) + a_1 p_1(x) + \cdots + a_n p_n(x) = \zv
  \]
  for some scalars \(\seq{a}{0,1,,n}\), then
  \[
    a_0 + (a_0 + a_1) x + (a_0 + a_1 + a_2) x^2 + \cdots + (a_0 + a_1 + \cdots + a_n) x^n = \zv.
  \]
  By equating the coefficients of \(x^k\) on both sides of this equation for \(k = 0, 1, \dots, n\), we obtain
  \[
    \begin{matrix*}[l]
      & \seq[+]{a}{0}      & = 0 \\
      & \seq[+]{a}{0,1}    & = 0 \\
      & \seq[+]{a}{0,1,2}  & = 0 \\
      & \vdots & \\
      & \seq[+]{a}{0,1,2,,n} & = 0
    \end{matrix*}
  \]
  Clearly the only solution to this system of linear equations is \(\seq[=]{a}{0,1,,n} = 0\).
\end{eg}

\begin{thm}\label{1.6}
  Let \(\V\) be a vector space over \(\F\), and let \(S_1 \subseteq S_2 \subseteq \V\).
  If \(S_1\) is linearly dependent, then \(S_2\) is linearly dependent.
\end{thm}

\begin{proof}[\pf{1.6}]
  We have
  \begin{align*}
             & \begin{dcases}
                 S_1 \subseteq S_2 \\
                 S_1 \text{ is linearly dependent}
               \end{dcases}                               \\
    \implies & \begin{dcases}
                 \exists \seq{u}{1,,n} \in S_1 \subseteq S_2 \\
                 \exists \seq{a}{1,,n} \in \F
               \end{dcases} :                    \\
             & \begin{dcases}
                 \seq[+]{a,u}{1,,n} = \zv \\
                 \lnot (\seq[=]{a}{1,,n} = 0)
               \end{dcases}                   &  & \by{1.5.1}                 \\
    \implies & S_2 \text{ is linearly dependent}.             &  & \by{1.5.1}
  \end{align*}
\end{proof}

\begin{cor}\label{1.5.6}
  Let \(\V\) be a vector space, and let \(S_1 \subseteq S_2 \subseteq \V\).
  If \(S_2\) is linearly independent, then \(S_1\) is linearly independent.
\end{cor}

\begin{proof}[\pf{1.5.6}]
  Suppose for sake of contradiction that \(S_1\) is linearly dependent.
  But by \cref{1.6} \(S_1 \subseteq S_2\) implies \(S_2\) is linearly dependent, which contradicts to the fact that \(S_2\) is linearly independent.
  Thus \(S_1\) is linearly independent.
\end{proof}

\begin{note}
  Earlier in this section, we noted that the issue of whether \(S\) is the smallest generating set for its span is related to the question of whether some vector in \(S\) is a linear combination of the other vectors in \(S\).
  Thus the issue of whether \(S\) is the smallest generating set for its span is related to the question of whether \(S\) is linearly dependent.

  More generally, suppose that \(S\) is any linearly dependent set containing two or more vectors.
  Then some vector \(v \in S\) can be written as a linear combination of the other vectors in \(S\), and the subset obtained by removing \(v\) from \(S\) has the same span as \(S\).
  It follows that \emph{if no proper subset of \(S\) generates the span of \(S\), then \(S\) must be linearly independent.}
\end{note}

\begin{thm}\label{1.7}
  Let \(S\) be a linearly independent subset of a vector space \(\V\) over \(\F\), and let \(v\) be a vector in \(\V\) that is not in \(S\).
  Then \(S \cup \set{v}\) is linearly dependent iff \(v \in \spn{S}\).
\end{thm}

\begin{proof}[\pf{1.7}]
  By \cref{1.5.2} we see that when \(v = \zv\) the statement holds.
  So suppose that \(v \neq \zv\).

  If \(S \cup \set{v}\) is linearly dependent, then there are vectors \\
  \(\seq{u}{1,,n}\) in \(S\) such that \(a_0 v + \seq[+]{a,u}{1,,n} = \zv\) for some nonzero scalars \(\seq{a}{0,1,2,,n}\) in \(\F\).
  We claim that \(a_0 \neq 0\).
  So suppose for sake of contradiction that \(a_0 = 0\).
  Because \(S\) is linearly independent, we know that
  \begin{align*}
             & a_0 v + \seq[+]{a,u}{1,,n}                            \\
             & = 0v + \seq[+]{a,u}{1,,n}                             \\
             & = \zv + \seq[+]{a,u}{1,,n} &  & \by{1.2}[a]           \\
             & = \seq[+]{a,u}{1,,n}       &  & \text{(by \ref{vs3})} \\
             & = \zv                                                 \\
    \implies & \seq[=]{a}{1,,n} = 0.      &  & \by{1.5.3}
  \end{align*}
  But this means \(\seq[=]{a}{0,1,2,,n} = 0\), a contradiction.
  Thus we must have \(a_0 \neq 0\), and so
  \[
    v = a_0^{-1} (-\seq[-]{a,u}{1,,n}) = -(a_0^{-1} a_1) u_1 - (a_0^{-1} a_2) u_2 - \cdots - (a_0^{-1} a_n) u_n.
  \]
  Since \(v\) is a linear combination of \(\seq{u}{1,,n}\), which are in \(S\), we have \(v \in \spn{S}\).

  Conversely, let \(v \in \spn{S}\).
  Then there exist vectors \(\seq{v}{1,,m}\) in \(S\) and scalars \(\seq{b}{1,,m}\) such that \(v = \seq[+]{b,v}{1,,m}\).
  Hence
  \[
    \zv = \seq[+]{b,v}{1,,m} + (-1)v.
  \]
  Since \(v \neq v_i\) for \(i = 1, 2, \dots, m\), the coefficient of \(v\) in this linear combination is nonzero, and so the set \(\set{\seq{v}{1,,m}, v}\) is linearly dependent.
  Therefore \(S \cup \set{v}\) is linearly dependent by \cref{1.6}.
\end{proof}

\exercisesection

\setcounter{ex}{3}
\begin{ex}\label{ex:1.5.4}
  In \(\vs{F}^n\), let \(e_j\) denote the vector whose \(j\)th coordinate is \(1\) and whose other coordinates are \(0\).
  Prove that \(\set{\seq{e}{1,,n}}\) is linearly independent.
\end{ex}

\begin{proof}[\pf{ex:1.5.4}]
  Let \(\seq{a}{1,,n} \in \F\).
  Since
  \begin{align*}
             & \seq[+]{a,e}{1,,n} = \zv                          \\
    \implies & \begin{dcases}
                 a_1 1 + a_2 0 + \cdots + a_n 0 = 0 \\
                 a_1 0 + a_2 1 + \cdots + a_n 0 = 0 \\
                 \vdots                             \\
                 a_1 0 + a_2 0 + \cdots + a_n 1 = 0
               \end{dcases}             &  & \by{1.2.4}          \\
    \implies & \forall i \in \set{1, \dots, n}, a_i 1 = a_i = 0,
  \end{align*}
  by \cref{1.5.3} we know that \(\set{\seq{e}{1,,n}}\) is linearly independent.
\end{proof}

\begin{ex}\label{ex:1.5.5}
  Show that the set \(\set{1, x, x^2, \dots, x^n}\) is linearly independent in \(\ps[n]{\F}\).
\end{ex}

\begin{proof}[\pf{ex:1.5.5}]
  Let \(\seq{a}{0,1,2,,n} \in \F\).
  Since
  \begin{align*}
             & a_0 + a_1 x^1 + a_2 x^2 + \cdots + a_n x^n = \zv = 0 + 0x + 0x^2 + \cdots + 0x^n \\
    \implies & \seq[=]{a}{0,1,2,,n} = 0,
  \end{align*}
  by \cref{1.5.3} we know that \(\set{1, x, x^2, \dots, x^n}\) is linearly independent.
\end{proof}

\begin{ex}\label{ex:1.5.6}
  In \(\ms\), let \(E^{i j}\) denote the matrix whose only nonzero entry is \(1\) in the \(i\)th row and \(j\)th column.
  Prove that \(\set{E^{i j} : 1 \leq i \leq m, 1 \leq j \leq n}\) is linearly independent.
\end{ex}

\begin{proof}[\pf{ex:1.5.6}]
  Let \(a_{i j} \in \F\) where \(1 \leq i \leq m\) and \(1 \leq j \leq n\).
  Since
  \begin{align*}
             & \sum_{i = 1}^m \sum_{j = 1}^{n} a_{i j} E^{i j} = \zm                 \\
    \implies & \begin{pmatrix}
                 a_{1 1} 1 & a_{1 2} 1 & \cdots & a_{1 n} 1 \\
                 a_{2 1} 1 & a_{2 2} 1 & \cdots & a_{2 n} 1 \\
                 \vdots    & \vdots    & \ddots & \vdots    \\
                 a_{m 1} 1 & a_{m 2} 1 & \cdots & a_{m n} 1
               \end{pmatrix}                            \\
             & = \begin{pmatrix}
                   a_{1 1} & a_{1 2} & \cdots & a_{1 n} \\
                   a_{2 1} & a_{2 2} & \cdots & a_{2 n} \\
                   \vdots  & \vdots  & \ddots & \vdots  \\
                   a_{m 1} & a_{m 2} & \cdots & a_{m n}
                 \end{pmatrix} = \zm                                \\
    \implies & a_{i j} = 0,                                          &  & \by{1.2.8}
  \end{align*}
  by \cref{1.5.3} we know that \(\set{E^{i j} : 1 \leq i \leq m, 1 \leq j \leq n}\) is linearly independent.
\end{proof}

\setcounter{ex}{7}
\begin{ex}\label{ex:1.5.8}
  Let \(S = \set{(1, 1, 0), (1, 0, 1), (0, 1, 1)}\) be a subset of the vector space \(\vs{F}^3\) over \(\F\).
  \begin{enumerate}
    \item Prove that if \(\F = \R\), then \(S\) is linearly independent.
    \item Prove that if \(\F\) has characteristic \(2\), then \(S\) is linearly dependent.
  \end{enumerate}
\end{ex}

\begin{proof}[\pf{ex:1.5.8}(a)]
  Let \(\seq{a}{1,2,3} \in \R\).
  Since
  \begin{align*}
             & a_1 (1, 1, 0) + a_2 (1, 0, 1) + a_3 (0, 1, 1) = (0, 0, 0) \\
    \implies & \begin{dcases}
                 a_1 1 + a_2 1 + a_3 0 = 0 \\
                 a_1 1 + a_2 0 + a_3 1 = 0 \\
                 a_1 0 + a_2 1 + a_3 1 = 0
               \end{dcases}                              &  & \by{1.2.4} \\
    \implies & \seq[=]{a}{1,2,3} = 0,
  \end{align*}
  by \cref{1.5.3} we know that \(S\) is linearly independent.
\end{proof}

\begin{proof}[\pf{ex:1.5.8}(b)]
  Observe that
  \begin{align*}
    (1, 1, 0) + (1, 0, 1) + (0, 1, 1) & = (1 + 1, 1 + 0, 0 + 1) + (0, 1, 1) \\
                                      & = (0, 1, 1) + (0, 1, 1)             \\
                                      & = (0 + 0, 1 + 1, 1 + 1)             \\
                                      & = (0, 0, 0).
  \end{align*}
  Thus by \cref{1.5.1} \(S\) is linearly dependent.
\end{proof}

\begin{ex}\label{ex:1.5.9}
  Let \(u\) and \(v\) be distinct vectors in a vector space \(\V\) over \(\F\).
  Show that \(\set{u, v}\) is linearly dependent iff \(u\) or \(v\) is a multiple of the other.
\end{ex}

\begin{proof}[\pf{ex:1.5.9}]
  We have
  \begin{align*}
         & \set{u, v} \text{ is linearly dependent}                        \\
    \iff & \exists a, b \in \F : \begin{dcases}
                                   au + bv = \zv \\
                                   \lnot (a = b = 0)
                                 \end{dcases}         &  & \by{1.5.1}      \\
    \iff & \exists a, b \in \F : \begin{dcases}
                                   au = -bv \\
                                   \lnot (a = b = 0)
                                 \end{dcases}                          \\
    \iff & \exists a, b \in \F : \begin{dcases}
                                   u = -\dfrac{b}{a} v & \text{if } a \neq 0 \\
                                   v = -\dfrac{a}{b} u & \text{if } b \neq 0
                                 \end{dcases} \\
    \iff & \exists c \in \F : u = cv.
  \end{align*}
\end{proof}

\setcounter{ex}{10}
\begin{ex}\label{ex:1.5.11}
  Let \(S = \set{\seq{u}{1,,n}}\) be a linearly independent subset of a vector space \(\V\) over the field \(\Z_2\).
  How many vectors are there in \(\spn{S}\)?
  Justify your answer.
\end{ex}

\begin{proof}[\pf{ex:1.5.11}]
  We have
  \[
    \forall v \in \spn{S}, \exists \seq{a}{1,,n} \in \Z_2 : \seq[+]{a,u}{1,,n} = v.
  \]
  Since \(a_i \in \Z_2\) for all \(i \in \set{1, \dots, n}\), \(a_i\) has only two choices, which are \(0\) or \(1\).
  Since there are \(n\) variables with \(2\) choices for each, there are \(2^n\) vectors in \(\spn{S}\).
\end{proof}

\setcounter{ex}{12}
\begin{ex}\label{ex:1.5.13}
  Let \(\V\) be a vector space over a field of characteristic not equal to two.
  \begin{enumerate}
    \item Let \(u\) and \(v\) be distinct vectors in \(\V\).
          Prove that \(\set{u,v}\) is linearly independent iff \(\set{u + v, u - v}\) is linearly independent.
    \item Let \(u, v\), and \(w\) be distinct vectors in \(\V\).
          Prove that \(\set{u, v, w}\) is linearly independent iff \(\set{u + v, u + w, v + w}\) is linearly independent.
  \end{enumerate}
\end{ex}

\begin{proof}[\pf{ex:1.5.13}(a)]
  We have
  \begin{align*}
         & \set{u, v} \text{ is linearly independent}                          \\
    \iff & \forall a, b \in \F,                                                \\
         & [au + bv = \zv \implies a = b = 0]                  &  & \by{1.5.3} \\
    \iff & \forall a, b \in \F,                                                \\
         & [a(u + v) + b(u - v) = (a + b) u + (a - b) v                        \\
         & = \zv \implies a + b = a - b = 0]                                   \\
    \iff & \set{u + v, u - v} \text{ is linearly independent}. &  & \by{1.5.3}
  \end{align*}
\end{proof}

\begin{proof}[\pf{ex:1.5.13}(b)]
  We have
  \begin{align*}
         & \set{u, v, w} \text{ is linearly independent}                              \\
    \iff & \forall a, b, c \in \F,                                                    \\
         & [au + bv + cw = \zv \implies a = b = c = 0]                &  & \by{1.5.3} \\
    \iff & \forall a, b \in \F,                                                       \\
         & [a(u + v) + b(u + w) + c(v + w)                                            \\
         & = (a + b) u + (a + c) v + (b + c) w                                        \\
         & = \zv \implies a + b = a + c = b + c = 0]                                  \\
    \iff & \set{u + v, u + w, v + w} \text{ is linearly independent}. &  & \by{1.5.3}
  \end{align*}
\end{proof}

\begin{ex}\label{ex:1.5.14}
  Prove that a set \(S\) is linearly dependent iff \(S = \set{\zv}\) or there exist distinct vectors \(v, \seq{u}{1,,n}\) in \(S\) such that \(v\) is a linear combination of \(\seq{u}{1,,n}\).
\end{ex}

\begin{proof}[\pf{ex:1.5.14}]
  By \cref{1.5.4}(a) we know that \(\varnothing\) is linearly independent, so \(S\) has at least one element.
  First suppose that \(S\) has only one element.
  Then by \cref{1.5.4}(b) we know that \(S\) is linearly dependent iff \(S = \set{\zv}\).

  Now suppose that \(S\) has more than one element.
  Then we have
  \begin{align*}
         & S \text{ is linearly dependent}                                                      \\
    \iff & \begin{dcases}
             \exists v \in S                               \\
             \exists \seq{u}{1,,n} \in S \setminus \set{v} \\
             \exists a_0 \in \F \setminus \set{0}          \\
             \exists \seq{a}{1,,n} \in \F
           \end{dcases} :                                        \\
         & a_0 v + \seq[+]{a,u}{1,,n} = \zv                                     &  & \by{1.5.1} \\
    \iff & \begin{dcases}
             \exists v \in S                               \\
             \exists \seq{u}{1,,n} \in S \setminus \set{v} \\
             \exists a_0 \in \F \setminus \set{0}          \\
             \exists \seq{a}{1,,n} \in \F
           \end{dcases} :                                        \\
         & v = -a_0^{-1} a_1 u_1 - a_0^{-1} a_2 u_2 - \cdots - a_0^{-1} a_n u_n                 \\
    \iff & \exists v, \seq{u}{1,,n} \in S :                                                     \\
         & v \text{ is a linear combination of } \seq{u}{1,,n}.                 &  & \by{1.4.1}
  \end{align*}
  Combine all proofs above we conclude that \(S\) is linearly dependent iff \(S = \set{\zv}\) or there exist distinct vectors \(v, \seq{u}{1,,n}\) in \(S\) such that \(v\) is a linear combination of \(\seq{u}{1,,n}\).
\end{proof}

\begin{ex}\label{ex:1.5.15}
  Let \(S = \set{\seq{u}{1,,n}}\) be a finite set of vectors.
  Prove that \(S\) is linearly dependent iff \(u_1 = \zv\) or \(u_{k + 1} \in \spn{\set{\seq{u}{1,,k}}}\) for some \(k\) (\(1 \leq k < n\)).
\end{ex}

\begin{proof}[\pf{ex:1.5.15}]
  By \cref{1.5.1} and \cref{1.5.2} we know that if \(u_1 = \zv\) or \(u_{k + 1} \in \spn{\set{\seq{u}{1,,k}}}\) for some \(k\) (\(1 \leq k < n\)), then \(S\) is linearly dependent.
  So we only need to show the converse is also true.

  Let \(S = \set{\seq{u}{1,,n}}\) be linearly dependent.
  Suppose for sake of contradiction that \(u_1 \neq \zv\) and
  \[
    \forall k \in \set{1, \dots, n - 1}, u_{k + 1} \notin \spn{\set{\seq{u}{1,,k}}}.
  \]
  By \cref{1.4.2} we know that \(u_k \neq \zv\) for all \(k \in \set{1, \dots, n}\).
  But then we have
  \begin{align*}
             & u_n \notin \spn{\set{\seq{u}{1,,n - 1}}}                                \\
    \implies & \forall \seq{a}{1,,n - 1} \in \F,                                       \\
             & \seq[+]{a,u}{1,,n - 1} \neq u_n                         &  & \by{1.4.3} \\
    \implies & \begin{dcases}
                 \forall \seq{a}{1,,n - 1} \in \F \\
                 \forall a_n \in \F \setminus \set{0}
               \end{dcases},                                     \\
             & \seq[+]{a,u}{1,,n - 1} \neq a_n u_n                                     \\
    \implies & \begin{dcases}
                 \forall \seq{a}{1,,n - 1} \in \F \\
                 \forall a_n \in \F \setminus \set{0}
               \end{dcases},                                     \\
             & \seq[+]{a,u}{1,,n} \neq \zv                                             \\
    \implies & \forall \seq{a}{1,,n} \in \F, [\seq[+]{a,u}{1,,n} = \zv                 \\
             & \iff \seq[=]{a}{1,,n} = 0]                                              \\
    \implies & S \text{ is linearly independent},                      &  & \by{1.5.3}
  \end{align*}
  a contradiction.
  Thus the converse must be true.
\end{proof}

\begin{ex}\label{ex:1.5.16}
  Prove that a set \(S\) of vectors is linearly independent iff each finite subset of \(S\) is linearly independent.
\end{ex}

\begin{proof}[\pf{ex:1.5.16}]
  First suppose that \(S\) is linearly independent.
  Then by \cref{1.5.6} we know that every finite subset of \(S\) is linearly independent.

  Now suppose that every finite subset of \(S\) is linearly independent.
  Suppose for sake of contradiction that \(S\) is linearly dependent.
  Then by \cref{ex:1.5.14} we know that \(S = \set{\zv}\) or there exist \(v, \seq{u}{1,,n} \in S\) such that \(v\) is a linear combination of \(\seq{u}{1,,n}\).
  Clearly \(S \neq \set{\zv}\) since \(\set{\zv}\) is a finite subset of \(S\) but \(\set{\zv}\) is linearly dependent.
  So we must have \(v\) as a linear combination of \(\seq{u}{1,,n}\).
  Since the set \(\set{v,\seq{u}{1,,n}}\) is finite, by hypothese it is linearly independent.
  But this contradicts to the fact that \(v\) is a linear combination of \(\seq{u}{1,,n}\).
  Thus \(S\) is linearly independent.
\end{proof}

\begin{ex}\label{ex:1.5.17}
  Let \(M \in \ms[n][n][\F]\) be a square upper triangular matrix (as defined in \cref{ex:1.3.12}) with nonzero diagonal entries.
  Prove that the columns of \(M\) are linearly independent.
\end{ex}

\begin{proof}[\pf{ex:1.5.17}]
  The columns of \(M\) are vectors in \(\vs{F}^n\), thus we can write the \(i\)th column (\(1 \leq i \leq n\)) of \(M\) as
  \[
    \begin{pmatrix}
      M_{1 i} \\
      M_{2 i} \\
      \vdots  \\
      M_{n i}
    \end{pmatrix} = M_{1 i} \cdot e_1 + M_{2 i} \cdot e_2 + \cdots + M_{n i} \cdot e_n = \sum_{j = 1}^n M_{j i} \cdot e_j,
  \]
  where \(\seq{e}{1,,n}\) are defined as in \cref{ex:1.5.4}.
  We now show that the set of column vectors of \(M\) is linearly independent.
  Let \(\seq{a}{1,,n} \in \F\).
  Since
  \begin{align*}
             & \sum_{i = 1}^n \br{a_i \cdot \pa{\sum_{j = 1}^n M_{j i} \cdot e_j}} = \zv                                                        \\
    \implies & \sum_{i = 1}^n \pa{\sum_{j = 1}^n a_i \cdot M_{j i} \cdot e_j} = \zv                                                             \\
    \implies & \sum_{j = 1}^n \pa{\sum_{i = 1}^n a_i \cdot M_{j i} \cdot e_j} = \zv      &  & \text{(by \ref{vs1} and \ref{vs2})}               \\
    \implies & \sum_{j = 1}^n \pa{\sum_{i = 1}^n a_i \cdot M_{j i}} \cdot e_j = \zv                                                             \\
    \implies & \forall j \in \set{1, \dots, n}, \sum_{i = 1}^n a_i \cdot M_{j i} = 0     &  & \by{ex:1.5.4}                                     \\
    \implies & \forall j \in \set{1, \dots, n}, \sum_{i = j}^n a_i \cdot M_{j i} = 0     &  & \by{ex:1.3.12}                                    \\
    \implies & \seq[=]{a}{1,,n} = 0,                                                     &  & (\forall j \in \set{1, \dots, n}, M_{j j} \neq 0)
  \end{align*}
  by \cref{1.5.3} we know that the column vectors of \(M\) is linearly independent.
\end{proof}

\begin{ex}\label{ex:1.5.18}
  Let \(S\) be a set of nonzero polynomials in \(\ps{\F}\) such that no two have the same degree.
  Prove that \(S\) is linearly independent.
\end{ex}

\begin{proof}[\pf{ex:1.5.18}]
  Suppose for sake of contradiction that \(S\) is linearly dependent.
  Then by \cref{ex:1.5.14} we know that
  \[
    \begin{dcases}
      \exists g, \seq{f}{1,,n} \in S \\
      \exists \seq{a}{1,,n} \in \F
    \end{dcases} : \begin{dcases}
      g = \seq[+]{a,f}{1,,n} \\
      \lnot (\seq[=]{a}{1,,n} = 0)
    \end{dcases}.
  \]
  Let \(\deg(h)\) be the degree of any function \(h \in \ps{\F}\) and let \(m = \deg(g)\).
  Then by \cref{1.2.11} we have
  \[
    \exists \seq{c}{0,1,,m} \in \F : \begin{dcases}
      g(x) = \sum_{j = 0}^m c_j x^j \\
      c_m \neq 0
    \end{dcases}.
  \]
  By hypothese we know that
  \[
    \forall i \in \set{1, \dots, n}, \deg(f_i) \neq m.
  \]
  But by \cref{ex:1.5.5} we have
  \[
    c_m x^m = \sum_{i = 1}^n a_i \cdot (0x^m) = 0 \implies c_m = 0,
  \]
  a contradiction.
  Thus \(S\) is linearly independent.
\end{proof}

\begin{ex}\label{ex:1.5.19}
  Prove that if \(\set{\seq{A}{1,,k}}\) is a linearly independent subset of \(\ms[n][n][\F]\), then \(\set{\tp{A}_1, \tp{A}_2, \dots, \tp{A}_k}\) is also linearly independent.
\end{ex}

\begin{proof}[\pf{ex:1.5.19}]
  Let \(i, j \in \set{1, \dots, n}\).
  Then we have
  \begin{align*}
             & \set{\seq{A}{1,,k}} \text{ is linearly independent}                                                         \\
    \implies & \br{\forall \seq{c}{1,,k} \in \F, \sum_{p = 1}^k c_p A_p = \zm \implies c_p = 0}            &  & \by{1.5.3} \\
    \implies & \br{\forall \seq{c}{1,,k} \in \F, \sum_{p = 1}^k c_p (A_p)_{i j} = 0 \implies c_p = 0}      &  & \by{1.2.9} \\
    \implies & \br{\forall \seq{c}{1,,k} \in \F, \sum_{p = 1}^k c_p \tp{(A_p)}_{j i} = 0 \implies c_p = 0} &  & \by{1.3.3} \\
    \implies & \br{\forall \seq{c}{1,,k} \in \F, \sum_{p = 1}^k c_p \tp{(A_p)} = \zm \implies c_p = 0}     &  & \by{1.2.9} \\
    \implies & \set{\tp{A}_1, \tp{A}_2, \dots, \tp{A}_k} \text{ is linearly independent}.                  &  & \by{1.5.3}
  \end{align*}
\end{proof}

\begin{ex}\label{ex:1.5.20}
  Let \(f, g \in \fs(\R, \R)\) be the functions defined by \(f(t) = e^{rt}\) and \(g(t) = e^{st}\), where \(r \neq s\).
  Prove that \(f\) and \(g\) are linearly independent in \(\fs(\R, \R)\).
\end{ex}

\begin{proof}[\pf{ex:1.5.20}]
  Suppose for sake of contradiction that \(f, g\) are linearly dependent.
  Then by \cref{1.5.1} we have
  \[
    \exists a, b \in \R : \begin{dcases}
      af + bg = \zv \\
      \lnot (a = b = 0)
    \end{dcases}.
  \]
  But this means
  \begin{align*}
             & \forall t \in \R, ae^{rt} + be^{st} = 0                                                         \\
    \implies & \forall t \in \R, a = -b e^{(s - r) t}                                                          \\
    \implies & a = b = 0,                              &  & \text{(\(t \mapsto e^{(s - r) t}\) is one-to-one)}
  \end{align*}
  a contradiction.
  Thus \(f, g\) are linearly independent.
\end{proof}

\section{Bases and Dimension}\label{sec:1.6}

\begin{defn}\label{1.6.1}
  A \textbf{basis} \(\beta\) for a vector space \(\V\) over \(\F\) is a linearly independent subset of \(\V\) that generates \(\V\).
  If \(\beta\) is a basis for \(\V\), we also say that the vectors of \(\beta\) form a basis for \(\V\).
\end{defn}

\begin{eg}\label{1.6.2}
  Recalling that \(\spn{\varnothing} = \set{\zv}\) and \(\varnothing\) is linearly independent, we see that \(\varnothing\) is a basis for the zero vector space.
\end{eg}

\begin{eg}\label{1.6.3}
  In \(\vs{F}^n\), let \(e_1 = (1, 0, 0, \dots, 0)\), \(e_2 = (0, 1, 0, \dots, 0)\), \dots, \(e_n = (0, 0, \dots, 0, 1)\);
  \(\set{\seq{e}{1,2,,n}}\) is readily seen to be a basis for \(\vs{F}^n\) and is called the \textbf{standard basis} for \(\vs{F}^n\).
\end{eg}

\begin{proof}[\pf{1.6.3}]
  By \cref{ex:1.5.4} we know that \(\set{\seq{e}{1,2,,n}}\) is linearly independent.
  By \cref{ex:1.4.7} we know that \(\vs{F}^n = \spn{\set{\seq{e}{1,2,,n}}}\).
  Thus by \cref{1.6.1} \(\set{\seq{e}{1,2,,n}}\) is a basis for \(\vs{F}^n\) over \(\F\).
\end{proof}

\begin{eg}\label{1.6.4}
  In \(\MS\), let \(E^{i j}\) denote the matrix whose only nonzero entry is a \(1\) in the \(i\)th row and \(j\)th column.
  Then \(\set{E^{i j} : 1 \leq i \leq m, 1 \leq j \leq n}\) is a basis for \(\MS\).
\end{eg}

\begin{proof}[\pf{1.6.4}]
  By \cref{ex:1.5.6} we know that \(\set{E^{i j} : 1 \leq i \leq m, 1 \leq j \leq n}\) is linearly independent.
  Since
  \begin{align*}
             & \forall A \in \MS, A = \begin{pmatrix}
      A_{1 1} & A_{1 2} & \cdots & A_{1 n} \\
      A_{2 1} & A_{2 2} & \cdots & A_{2 n} \\
      \vdots  & \vdots  & \ddots & \vdots  \\
      A_{m 1} & A_{m 2} & \cdots & A_{m n}
    \end{pmatrix}                                                              \\
             & = \sum_{i = 1}^m \sum_{j = 1}^n A_{i j} E^{i j}                                 &  & \text{(by \cref{1.2.9})} \\
    \implies & \forall A \in \MS, A \in \spn{\set{E^{i j} : 1 \leq i \leq m, 1 \leq j \leq n}} &  & \text{(by \cref{1.4.3})} \\
    \implies & \MS = \spn{\set{E^{i j} : 1 \leq i \leq m, 1 \leq j \leq n}},                   &  & \text{(by \cref{1.5})}
  \end{align*}
  by \cref{1.6.1} we know that \(\set{E^{i j} : 1 \leq i \leq m, 1 \leq j \leq n}\) is a basis for \(\MS\) over \(\F\).
\end{proof}

\begin{eg}\label{1.6.5}
  In \(\ps[n]{\F}\) the set \(\set{1, x, x^2, \dots, x^n}\) is a basis.
  We call this basis the \textbf{standard basis} for \(\ps[n]{\F}\).
\end{eg}

\begin{proof}[\pf{1.6.5}]
  By \cref{ex:1.5.5} we know that \(\set{1, x, x^2, \dots, x^n}\) is linearly independent.
  By \cref{ex:1.4.8} we know that \(\ps[n]{\F} = \spn{\set{1, x, x^2, \dots, x^n}}\).
  Thus by \cref{1.6.1} \(\set{1, x, x^2, \dots, x^n}\) is a basis for \(\ps[n]{\F}\) over \(\F\).
\end{proof}

\begin{eg}\label{1.6.6}
  In \(\ps{\F}\) the set \(\set{1, x, x^2, \dots}\) is a basis.
\end{eg}

\begin{proof}[\pf{1.6.6}]
  Suppose for sake of contradiction that \(\set{1, x, x^2, \dots}\) is not a basis for \(\ps{\F}\).
  Then by \cref{1.6.1} we can split into two cases:
  \begin{itemize}
    \item If \(\set{1, x, x^2, \dots}\) is linearly dependent, then by \cref{ex:1.5.14} we have
          \[
            \begin{dcases}
              \exists x^n \in \set{1, x, x^2, \dots} \\
              \exists \set{\seq{a}{0,1,2,}} \subseteq \F
            \end{dcases} : x^n = \sum_{i \in \N : i \neq n} a_i x^i.
          \]
          By setting \(a_n = -1\) we have
          \begin{align*}
                     & \sum_{i \in \N} a_i x^i = \zv = \sum_{i \in \N} 0x^i                                \\
            \implies & \seq[=]{a}{0,1,2,} = 0.                              &  & \text{(by \cref{1.2.11})}
          \end{align*}
          But this means \(a_n = 0\), a contradiction.
    \item If \(\ps{\F} \neq \spn{\set{1, x, x^2, \dots}}\), then by \cref{1.4.3} we have
          \[
            \exists f \in \ps{\F} : \forall \set{\seq{a}{0,1,2,}} \subseteq \F, f(x) \neq \sum_{i \in \N} a_i x^i.
          \]
          Let \(m\) be the degree of \(f\).
          Then by \cref{1.2.11} we have
          \[
            \exists \seq{c}{0,1,,m} \in \F : f(x) = c_0 + c_1 x + \cdots + c_m x^m.
          \]
          But by setting
          \[
            \begin{dcases}
              a_i = c_i & \text{if } i \leq m \\
              a_i = 0   & \text{if } i > m
            \end{dcases}
          \]
          we have \(f(x) = \sum_{i \in \N} a_i x^i\), a contradiction.
  \end{itemize}
  From all cases above we derived contradictions.
  Thus \(\set{1, x, x^2, \dots}\) is a basis for \(\ps{\F}\).
\end{proof}

\begin{note}
  Observe that \cref{1.6.6} shows that a basis need not be finite.
  In fact, later in \cref{sec:1.6} it is shown that no basis for \(\ps{\F}\) can be finite.
  Hence not every vector space has a finite basis.
\end{note}

\begin{thm}\label{1.8}
  Let \(\V\) be a vector space over \(\F\) and \(\beta = \set{\seq{u}{1,2,,n}}\) be a subset of \(\V\).
  Then \(\beta\) is a basis for \(\V\) if and only if each \(v \in \V\) can be uniquely expressed as a linear combination of vectors of \(\beta\), that is, can be expressed in the form
  \[
    v = \seq[+]{a,u}{1,2,,n}
  \]
  for unique scalars \(\seq{a}{1,2,,n} \in \F\).
\end{thm}

\begin{proof}[\pf{1.8}]
  First suppose that \(\beta\) be a basis for \(\V\).
  If \(v \in V\), then \(v \in \spn{\beta}\) because \(\spn{\beta} = \V\).
  Thus \(v\) is a linear combination of the vectors of \(\beta\).
  Suppose that
  \[
    v = \seq[+]{a,u}{1,2,,n} \quad \text{and} \quad v = \seq[+]{b,u}{1,2,,n}
  \]
  are two such representations of \(v\).
  Subtracting the second equation from the first gives
  \[
    \zv = (a_1 - b_1) u_1 + (a_2 - b_2) u_2 + \cdots + (a_n - b_n) u_n.
  \]
  Since \(\beta\) is linearly independent, it follows that \(a_1 - b_1 = a_2 - b_2 = \cdots = a_n - b_n = 0\).
  Hence \(a_1 = b_1, a_2 = b_2, \dots, a_n = b_n\), and so \(v\) is uniquely expressible as a linear combination of the vectors of \(\beta\).

  Now suppose that each \(v \in \V\) can be uniquely expressed as a linear combination of vectors of \(\beta\).
  By \cref{1.4.3} and \cref{1.5} this means \(\V = \spn{\beta}\).
  Thus to show that \(\beta\) is a basis for \(\V\), by \cref{1.6.1} we need to show that \(\beta\) is linearly independent.
  This is true since
  \begin{align*}
             & \zv \in \V                                             &  & \text{(by \ref{vs3})}    \\
    \implies & \exists! \seq{a}{1,2,,n} \in \F :                                                    \\
             & \zv = \seq[+]{a,u}{1,2,,n} = \seq[+]{0u}{1,2,,n}       &  & \text{(by hypothesis)}   \\
    \implies & \seq[=]{a}{1,2,,n} = 0                                                               \\
    \implies & \set{\seq{u}{1,2,,n}} \text{ is linearly independent}. &  & \text{(by \cref{1.5.3})}
  \end{align*}
\end{proof}

\begin{note}
  \cref{1.8} shows that if the vectors \(\seq{u}{1,2,,n}\) form a basis for a vector space \(\V\), then every vector in \(\V\) can be uniquely expressed in the form
  \[
    v = \seq[+]{a,u}{1,2,,n}
  \]
  for appropriately chosen scalars \(\seq{a}{1,2,,n}\).
  Thus \(v\) determines a unique \(n\)-tuple of scalars \(\tuple{a}{1,2,,n}\) and, conversely, each \(n\)-tuple of scalars determines a unique vector \(v \in \V\) by using the entries of the \(n\)-tuple as the coefficients of a linear combination of \(\seq{u}{1,2,,n}\).
  This fact suggests that \(\V\) is like the vector space \(\vs{F}^n\), where \(n\) is the number of vectors in the basis for \(\V\).
  We see in \cref{sec:2.4} that this is indeed the case.
\end{note}

\begin{thm}\label{1.9}
  If a vector space \(\V\) over \(\F\) is generated by a finite set \(S\), then some subset of \(S\) is a basis for \(\V\).
  Hence \(\V\) has a finite basis.
\end{thm}

\begin{proof}[\pf{1.9}]
  If \(S = \varnothing\) or \(S = \set{\zv}\), then \(\V = \set{\zv}\) and \(\varnothing\) is a subset of \(S\) that is a basis for \(\V\).
  Otherwise \(S\) contains a nonzero vector \(u_1\).
  By \cref{1.5.4}(b), \(\set{u_1}\) is a linearly independent set.
  Continue, if possible, choosing vectors \(\seq{u}{2,,k}\) in \(S\) such that \(\set{\seq{u}{1,2,,k}}\) is linearly independent.
  Since \(S\) is a finite set, we must eventually reach a stage at which \(\beta = \set{\seq{u}{1,2,,k}}\) is a linearly independent subset of \(S\), but adjoining to \(\beta\) any vector in \(S \setminus \beta\) produces a linearly dependent set.
  We claim that \(\beta\) is a basis for \(\V\).
  Because \(\beta\) is linearly independent by construction, it suffices to show that \(\beta\) spans \(\V\).
  By \cref{1.5} we need to show that \(S \subseteq \spn{\beta}\).
  Let \(v \in S\).
  If \(v \in \beta\), then clearly \(v \in \spn{\beta}\).
  Otherwise, if \(v \notin \beta\), then the preceding construction shows that \(\beta \cup \set{v}\) is linearly dependent.
  So \(v \in \spn{\beta}\) by \cref{1.7}.
  Thus \(S \subseteq \spn{\beta}\).
\end{proof}

\begin{note}
  Because of the method by which the basis \(\beta\) was obtained in the proof of \cref{1.9}, this theorem is often remembered as saying that \emph{a finite spanning set for \(\V\) can be reduced to a basis for \(\V\).}
\end{note}

\begin{thm}[Replacement Theorem]\label{1.10}
  Let \(\V\) be a vector space over \(\F\) that is generated by a set \(G\) containing exactly \(n\) vectors, and let \(L\) be a linearly independent subset of \(\V\) containing exactly \(m\) vectors.
  Then \(m \leq n\) and there exists a subset \(H\) of \(G\) containing exactly \(n - m\) vectors such that \(L \cup H\) generates \(\V\).
\end{thm}

\begin{proof}[\pf{1.10}]
  The proof is by mathematical induction on \(m\).
  The induction begins with \(m = 0\);
  for in this case \(L = \varnothing\), and so taking \(H = G\) gives the desired result.

  Now suppose that the theorem is true for some integer \(m \geq 0\).
  We prove that the theorem is true for \(m + 1\).
  Let \(L = \set{\seq{v}{1,2,,m + 1}}\) be a linearly independent subset of \(\V\) consisting of \(m + 1\) vectors.
  By \cref{1.5.6} \(\set{\seq{v}{1,2,,m}} \subseteq L\) is linearly independent, and so we may apply the induction hypothesis to conclude that \(m \leq n\) and that there is a subset \(\set{\seq{u}{1,2,,n - m}}\) of \(G\) such that \(\set{\seq{v}{1,2,,m}} \cup \set{\seq{u}{1,2,,n - m}}\) generates \(\V\).
  Thus there exist scalars \(\seq{a}{1,2,,m}, \seq{b}{1,2,,n - m} \in \F\) such that
  \begin{equation}\label{eq:1.6.1}
    \seq[+]{a,v}{1,2,,m} + \seq[+]{b,u}{1,2,,n - m} = v_{m + 1}
  \end{equation}
  Note that \(n - m > 0\), otherwise \(n - m = 0\) implies \(v_{m + 1}\) is a linear combination of \(\seq{v}{1,2,,m}\), which by \cref{1.7} contradicts the assumption that \(L\) is linearly independent.
  Hence \(n > m\);
  that is, \(n \geq m + 1\).
  Moreover, some \(b_i\), say \(b_1\), is nonzero, for otherwise we obtain the same contradiction.
  Solving \cref{eq:1.6.1} for \(u_1\) gives
  \begin{multline*}
    u_1 = (-b_1^{-1} a_1) v_1 + (-b_1^{-1} a_2) v_2 + \cdots + (-b_1^{-1} a_m) v_m + (b_1^{-1}) v_{m + 1} \\
    + (-b_1^{-1} b_2) u_2 + \cdots + (-b_1^{-1} b_{n - m}) u_{n - m}.
  \end{multline*}
  Let \(H = \set{\seq{u}{2,,n - m}}\).
  Then \(u_1 \in \spn{L \cup H}\), and because \(\seq{v}{1,2,,m}\), \\
  \(\seq{u}{2,,n - m}\) are clearly in \(\spn{L \cup H}\), it follows that
  \[
    \set{\seq{v}{1,2,,m}, \seq{u}{1,2,,n - m}} \subseteq \spn{L \cup H}.
  \]
  Because \(\set{\seq{v}{1,2,,m}, \seq{u}{1,2,n - m}}\) generates \(\V\), \cref{1.5} implies that \\
  \(\spn{L \cup H} = \V\).
  Since \(H\) is a subset of \(G\) that contains \((n - m) - 1 = n - (m + 1)\) vectors, the theorem is true for \(m + 1\).
  This completes the induction.
\end{proof}

\begin{cor}\label{1.6.7}
  Let \(\V\) be a vector space over \(\F\) having a finite basis.
  Then every basis for \(\V\) contains the same number of vectors.
\end{cor}

\begin{proof}[\pf{1.6.7}]
  Suppose that \(\beta\) is a finite basis for \(\V\) that contains exactly \(n\) vectors, and let \(\gamma\) be any other basis for \(\V\).
  If \(\gamma\) contains more than \(n\) vectors, then we can select a subset \(S\) of \(\gamma\) containing exactly \(n + 1\) vectors.
  Since \(S\) is linearly independent and \(\beta\) generates \(\V\), the replacement theorem (\cref{1.10}) implies that \(n + 1 \leq n\), a contradiction.
  Therefore \(\gamma\) is finite, and the number \(m\) of vectors in \(\gamma\) satisfies \(m \leq n\).
  Reversing the roles of \(\beta\) and \(\gamma\) and arguing as above, we obtain \(n \leq m\).
  Hence \(m = n\).
\end{proof}

\begin{note}
  If a vector space has a finite basis, \cref{1.6.7} asserts that the number of vectors in \emph{any} basis for \(\V\) is an intrinsic property of \(\V\).
\end{note}

\begin{defn}\label{1.6.8}
  A vector space is called \textbf{finite-dimensional} if it has a basis consisting of a finite number of vectors.
  The unique number of vectors in each basis for \(\V\) is called the \textbf{dimension} of \(\V\) and is denoted by \(\dim(\V)\).
  A vector space that is not finite-dimensional is called \textbf{infinite-dimensional}.
\end{defn}

\begin{eg}\label{1.6.9}
  The vector space \(\set{\zv}\) has dimension zero.
\end{eg}

\begin{proof}[\pf{1.6.9}]
  By \cref{1.6.2} and \cref{1.6.7} we are done.
\end{proof}

\begin{eg}\label{1.6.10}
  The vector space \(\vs{F}^n\) has dimension \(n\).
\end{eg}

\begin{proof}[\pf{1.6.10}]
  By \cref{1.6.3} and \cref{1.6.7} we are done.
\end{proof}

\begin{eg}\label{1.6.11}
  The vector space \(\MS\) has dimension \(mn\).
\end{eg}

\begin{proof}[\pf{1.6.11}]
  By \cref{1.6.4} and \cref{1.6.7} we are done.
\end{proof}

\begin{eg}\label{1.6.12}
  The vector space \(\ps[n]{\F}\) has dimension \(n + 1\).
\end{eg}

\begin{proof}[\pf{1.6.12}]
  By \cref{1.6.5} and \cref{1.6.7} we are done.
\end{proof}

\begin{eg}\label{1.6.13}
  Over the field of complex numbers, the vector space of complex numbers has dimension \(1\).
  (A basis is \(\set{1}\).)
\end{eg}

\begin{proof}[\pf{1.6.13}]
  We have
  \begin{align*}
             & \forall c \in \C, c = c \cdot 1                               \\
    \implies & \spn{\set{1}} = \C              &  & \text{(by \cref{1.5})}   \\
    \implies & \#\pa{\set{1}} = 1 = \dim(\C).  &  & \text{(by \cref{1.6.8})}
  \end{align*}
\end{proof}

\begin{eg}\label{1.6.14}
  Over the field of real numbers, the vector space of complex numbers has dimension \(2\).
  (A basis is \(\set{1, i}\).)
\end{eg}

\begin{proof}[\pf{1.6.14}]
  We have
  \begin{align*}
             & \forall c \in \C, c = \Re(c) + i \Im(c) = \Re(c) \cdot 1 + \Im(c) \cdot i &  & (\Re(c), \Im(c) \in \R)  \\
    \implies & \spn{\set{1, i}} = \C                                                     &  & \text{(by \cref{1.5})}   \\
    \implies & \#\pa{\set{1, i}} = 2 = \dim(\C).                                         &  & \text{(by \cref{1.6.8})}
  \end{align*}
\end{proof}

\begin{note}
  From \cref{1.6.13} and \cref{1.6.14} we see that the dimension of a vector space depends on its field of scalars.
\end{note}

\begin{note}
  In the terminology of dimension, the first conclusion in the replacement theorem states that if \(\V\) is a finite-dimensional vector space over \(\F\), then no linearly independent subset of \(\V\) can contain more than \(\dim(\V)\) vectors.
  From this fact it follows that the vector space \(\ps{\F}\) over \(\F\) is infinite-dimensional because it has an infinite linearly independent set, namely \(\set{1, x, x^2, \dots}\).
  This set is, in fact, a basis for \(\ps{\F}\).
  Yet nothing that we have proved in this section guarantees an infinite-dimensional vector space must have a basis.
  In \cref{sec:1.7} it is shown, however, that \emph{every vector space has a basis}.
\end{note}

\begin{cor}\label{1.6.15}
  Let \(\V\) be a vector space over \(\F\) with dimension \(n\).
  \begin{enumerate}
    \item Any finite generating set for \(\V\) contains at least \(n\) vectors, and a generating set for \(\V\) that contains exactly \(n\) vectors is a basis for \(\V\).
    \item Any linearly independent subset of \(\V\) that contains exactly \(n\) vectors is a basis for \(\V\).
    \item Every linearly independent subset of \(\V\) can be extended to a basis for \(\V\).
  \end{enumerate}
\end{cor}

\begin{proof}[\pf{1.6.15}]
  Let \(\beta\) be a basis for \(\V\).
  \begin{enumerate}
    \item Let \(G\) be a finite generating set for \(\V\).
          By \cref{1.9} some subset \(H\) of \(G\) is a basis for \(\V\).
          \cref{1.6.7} implies that \(H\) contains exactly \(n\) vectors.
          Since a subset of \(G\) contains \(n\) vectors, \(G\) must contain at least \(n\) vectors.
          Moreover, if \(G\) contains exactly \(n\) vectors, then we must have \(H = G\), so that \(G\) is a basis for \(\V\).
    \item Let \(L\) be a linearly independent subset of \(\V\) containing exactly \(n\) vectors.
          It follows from the replacement theorem that there is a subset \(H\) of \(\beta\) containing \(n - n = 0\) vectors such that \(L \cup H\) generates \(\V\).
          Thus \(H = \varnothing\), and \(L\) generates \(\V\).
          Since \(L\) is also linearly independent, \(L\) is a basis for \(\V\).
    \item If \(L\) is a linearly independent subset of \(\V\) containing \(m\) vectors, then the replacement theorem asserts that there is a subset \(H\) of \(\beta\) containing exactly \(n - m\) vectors such that \(L \cup H\) generates \(\V\).
          Now \(L \cup H\) contains at most \(n\) vectors;
          therefore (a) implies that \(L \cup H\) contains exactly \(n\) vectors and that \(L \cup H\) is a basis for \(\V\).
  \end{enumerate}
\end{proof}

\begin{eg}\label{1.6.16}
  For \(k = 0, 1, \dots, n\), let \(p_k(x) = x^k + x^{k + 1} + \cdots + x^n\).
  It follows from \cref{1.5.5} and \cref{1.6.15}(b) that
  \[
    \set{p_0(x), p_1(x), \dots, p_n(x)}
  \]
  is a basis for \(\ps[n]{\F}\).
\end{eg}

\begin{thm}\label{1.11}
  Let \(\W\) be a subspace of a finite-dimensional vector space \(\V\) over \(\F\).
  Then \(\W\) is finite-dimensional and \(\dim(\W) \leq \dim(\V)\).
  Moreover, if \(\dim(\W) = \dim(\V)\), then \(\V = \W\).
\end{thm}

\begin{proof}[\pf{1.11}]
  Let \(\dim(\V) = n\).
  If \(\W = \set{\zv}\), then \(\W\) is finite-dimensional and \(\dim(\W) = 0 \leq n\).
  Otherwise, \(\W\) contains a nonzero vector \(x_1\);
  so \(\set{x_1}\) is a linearly independent set.
  Continue choosing vectors, \(\seq{x}{1,2,,k}\) in \(\W\) such that \(\set{\seq{x}{1,2,,k}}\) is linearly independent.
  Since no linearly independent subset of \(\V\) can contain more than \(n\) vectors, this process must stop at a stage where \(k \leq n\) and \(\set{\seq{x}{1,2,,k}}\) is linearly independent but adjoining any other vector from \(\W\) produces a linearly dependent set.
  \cref{1.7} implies that \(\set{\seq{x}{1,2,,k}}\) generates \(\W\), and hence it is a basis for \(\W\).
  Therefore \(\dim(\W) = k \leq n\).

  If \(\dim(\W) = n\), then a basis for \(\W\) is a linearly independent subset of \(\V\) containing \(n\) vectors.
  But \cref{1.6.15}(b) implies that this basis for \(\W\) is also a basis for \(\V\);
  so \(\W = \V\).
\end{proof}

\begin{eg}\label{1.6.17}
  The set of diagonal \(n \times n\) matrices is a subspace \(\W\) of \(\ms{n}{n}{\F}\)
  (see \cref{1.3.8}).
  A basis for \(\W\) is
  \[
    \set{E^{1 1}, E^{2 2}, \dots, E^{n n}},
  \]
  where \(E^{i j}\) is the matrix in which the only nonzero entry is a \(1\) in the \(i\)th row and \(j\)th column.
  Thus \(\dim(\W) = n\).
\end{eg}

\begin{proof}[\pf{1.6.17}]
  By \cref{ex:1.5.6} we know that \(\set{E^{1 1}, E^{2 2}, \dots, E^{n n}}\) is linearly independent.
  Since \(\W = \spn{\set{E^{1 1}, E^{2 2}, \dots, E^{n n}}}\), by \cref{1.6.15}(a) we know that \(\dim(\W) \leq n\).
  By \cref{1.6.15}(c) we also know that \(\dim(\W) \geq n\).
  Thus we have \(\dim(\W) = n\).
\end{proof}

\begin{eg}\label{1.6.18}
  The set of symmetric \(n \times n\) matrices is a subspace \(\W\) of \(\ms{n}{n}{\F}\) over \(\F\).
  A basis for \(\W\) is
  \[
    \set{A^{i j} : 1 \leq i \leq j \leq n}
  \]
  where \(A^{i j}\) is the \(n \times n\) matrix having \(1\) in the \(i\)th row and \(j\)th column, \(1\) in the \(j\)th row and \(i\)th column, and \(0\) elsewhere.
  It follows that
  \[
    \dim(\W) = n + (n - 1) + \cdots + 1 = \frac{1}{2} n(n + 1).
  \]
\end{eg}

\begin{proof}[\pf{1.6.18}]
  By \cref{ex:1.5.6} we see that each \(A^{i j}\) can only express as \(E^{i j} + E^{j i}\).
  Thus \(\set{A^{i j} : 1 \leq i \leq j \leq n}\) is linearly independent and by \cref{1.6.15}(c)we have  \(\dim(\W) \geq \#\pa{\set{A^{i j} : 1 \leq i \leq j \leq n}}\).
  Since
  \begin{align*}
             & \forall M \in \W, M_{i j} = M_{j i} = M_{i j} \cdot 1                                                 \\
    \implies & \forall M \in \W, M = \sum_{i = 1}^n \sum_{j = i}^n M_{i j} A^{i j} &  & \text{(by \cref{1.2.9})}     \\
    \implies & \W = \spn{\set{A^{i j} : 1 \leq i \leq j \leq n}}                   &  & \text{(by \cref{1.5})}       \\
    \implies & \dim(\W) \leq \#\pa{\set{A^{i j} : 1 \leq i \leq j \leq n}},        &  & \text{(by \cref{1.6.15}(a))}
  \end{align*}
  we have \(\dim(\W) = \#\pa{\set{A^{i j} : 1 \leq i \leq j \leq n}} = \frac{1}{2} n(n + 1)\).
\end{proof}

\begin{cor}\label{1.6.19}
  If \(\W\) is a subspace of a finite-dimensional vector space \(\V\) over \(\F\), then any basis for \(\W\) can be extended to a basis for \(\V\).
\end{cor}

\begin{proof}[\pf{1.6.19}]
  Let \(S\) be a basis for \(\W\).
  Because \(S\) is a linearly independent subset of \(\V\), \cref{1.6.15}(c) guarantees that \(S\) can be extended to a basis for \(\V\).
\end{proof}

\begin{defn}[The Lagrange Interpolation Formula]\label{1.6.20}
  Let \(\seq{c}{0,1,,n}\) be distinct scalars in an infinite field \(\F\).
  The polynomials \(f_0(x), f_1(x), \dots, f_n(x)\) defined by
  \[
    f_i(x) = \frac{(x - c_0) \cdots (x - c_{i - 1}) (x - c_{i + 1}) \cdots (x - c_n)}{(c_i - c_0) \cdots (c_i - c_{i - 1}) (c_i - c_{i + 1}) \cdots (c_i - c_n)} = \prod_{\substack{k = 0 \\ k \neq i}}^n \frac{x - c_k}{c_i - c_k}
  \]
  are called the \textbf{Lagrange polynomials} (associated with \(\seq{c}{0,1,,n}\)).
  Note that each \(f_i(x)\) is a polynomial of degree \(n\) and hence is in \(\ps[n]{\F}\).
  By regarding \(f_i(x)\) as a polynomial function \(f_i : \F \to \F\), we see that
  \begin{equation}\label{eq:1.6.2}
    f_i(c_j) = \begin{dcases}
      0 & \text{if } i \neq j \\
      1 & \text{if } i = j
    \end{dcases}.
  \end{equation}

  This property of Lagrange polynomials can be used to show that \(\beta = \set{\seq{f}{0,1,,n}}\) is a linearly independent subset of \(\ps[n]{\F}\).
  Suppose that
  \[
    \sum_{i = 0}^n a_i f_i = \zv \quad \text{for some scalars } \seq{a}{0,1,,n},
  \]
  where \(\zv\) denotes the zero function.
  Then
  \[
    \sum_{i = 0}^n a_i f_i(c_j) = 0 \quad \text{for } j = 0, 1, \dots, n.
  \]
  But also
  \[
    \sum_{i = 0}^n a_i f_i(c_j) = a_j
  \]
  by \cref{eq:1.6.2}.
  Hence \(a_j = 0\) for \(j = 0, 1, \dots, n\);
  so \(\beta\) is linearly independent.
  Since the dimension of \(\ps[n]{\F}\) is \(n + 1\), it follows from \cref{1.6.15} that \(\beta\) is a basis for \(\ps[n]{\F}\).

  Because \(\beta\) is a basis for \(\ps[n]{\F}\), every polynomial function \(g\) in \(\ps[n]{\F}\) is a linear combination of polynomial functions of \(\beta\), say,
  \[
    g = \sum_{i = 0}^n b_i f_i.
  \]
  It follows that
  \[
    g(c_j) = \sum_{i = 0}^n b_i f_i(c_j) = b_j;
  \]
  so
  \[
    g = \sum_{i = 0}^n g(c_i) f_i
  \]
  is the unique representation of \(g\) as a linear combination of elements of \(\beta\).
  This representation is called the \textbf{Lagrange interpolation formula}.
  Notice that the preceding argument shows that if \(\seq{b}{0,1,,n}\) are any \(n + 1\) scalars in \(\F\) (not necessarily distinct), then the polynomial function
  \[
    g = \sum_{i = 0}^n b_i f_i
  \]
  is the unique polynomial in \(\ps[n]{\F}\) such that \(g(c_j) = b_j\).
  Thus we have found the unique polynomial of degree not exceeding \(n\) that has specified values \(b_j\) at given points \(c_j\) in its domain (\(j = 0, 1, \dots, n\)).

  An important consequence of the Lagrange interpolation formula is the following result:
  If \(f \in \ps[n]{\F}\) and \(f(c_i) = 0\) for \(n + 1\) distinct scalars \(\seq{c}{0,1,,n}\) in \(\F\), then \(f\) is the zero function.
\end{defn}

\exercisesection

\setcounter{ex}{10}
\begin{ex}\label{ex:1.6.11}
  Let \(u\) and \(v\) be distinct vectors of a vector space \(\V\) over \(\F\).
  Show that if \(\set{u, v}\) is a basis for \(\V\) and \(a\) and \(b\) are nonzero scalars, then both \(\set{u + v, au}\) and \(\set{au, bv}\) are also bases for \(\V\).
\end{ex}

\begin{proof}[\pf{ex:1.6.11}]
  Let \(c_1, c_2 \in \F\).
  Since
  \begin{align*}
             & c_1 (u + v) + c_2 (au) = \zv                                \\
    \implies & (c_1 + c_2 a) u + c_1 v = \zv &  & \text{(by \cref{1.2.1})} \\
    \implies & \begin{dcases}
      c_1 + c_2 a = 0 \\
      c_1 = 0
    \end{dcases}    &  & \text{(by \cref{1.5.3})} \\
    \implies & \begin{dcases}
      c_2 a = 0 \\
      c_1 = 0
    \end{dcases}                                  \\
    \implies & c_1 = c_2 = 0                 &  & (a \neq 0)
  \end{align*}
  and
  \begin{align*}
             & c_1 (au) + c_2 (bv) = \zv                                 \\
    \implies & (c_1 a) u + (c_2 b) v = \zv &  & \text{(by \cref{1.2.1})} \\
    \implies & \begin{dcases}
      c_1 a = 0 \\
      c_2 b = 0
    \end{dcases}  &  & \text{(by \cref{1.5.3})} \\
    \implies & c_1 = c_2 = 0,              &  & (a \neq 0 \neq b)
  \end{align*}
  by \cref{1.5.3} we know that \(\set{u + v, au}\) and \(\set{au, bv}\) are linearly independent.
  Since
  \[
    \#(\set{u + v, au}) = \#(\set{au, bv}) = 2 = \#(\set{u, v}),
  \]
  by \cref{1.6.15}(a) we know that \(\set{u + v, au}\) and \(\set{au, bv}\) are basis for \(\V\).
\end{proof}

\begin{ex}\label{ex:1.6.12}
  Let \(u, v\), and \(w\) be distinct vectors of a vector space \(\V\) over \(\F\).
  Show that if \(\set{u, v, w}\) is a basis for \(\V\), then \(\set{u + v + w, v + w, w}\) is also a basis for \(\V\).
\end{ex}

\begin{proof}[\pf{ex:1.6.12}]
  Let \(a, b, c \in \F\).
  Since
  \begin{align*}
             & a(u + v + w) + b(v + w) + cw = \zv                               \\
    \implies & au + (a + b)v + (a + b + c)w = \zv &  & \text{(by \cref{1.2.1})} \\
    \implies & \begin{dcases}
      a = 0     \\
      a + b = 0 \\
      a + b + c = 0
    \end{dcases}         &  & \text{(by \cref{1.5.3})} \\
    \implies & a = b = c = 0,
  \end{align*}
  by \cref{1.5.3} we know that \(\set{u + v + w, v + w, w}\) is linearly independent.
  Since
  \[
    \#(\set{u + v + w, v + w, w}) = 3 = \#(\set{u, v, w}),
  \]
  by \cref{1.6.15}(a) we know that \(\set{u + v + w, v + w, w}\) is a basis for \(\V\).
\end{proof}

\setcounter{ex}{14}
\begin{ex}\label{ex:1.6.15}
  The set of all \(n \times n\) matrices having trace equal to zero is a subspace \(\W\) of \(\ms{n}{n}{\F}\) (see \cref{1.3.9}).
  Find a basis for \(\W\).
  What is the dimension of \(\W\)?
\end{ex}

\begin{proof}[\pf{ex:1.6.15}]
  Let \(E^{i j} \in \ms{n}{n}{\F}\) be matrix defined as in \cref{ex:1.5.6} and let \(\beta\) be the set
  \[
    \beta = \set{E^{i j} : i, j \in \set{1, \dots, n}, i \neq j} \cup \set{E^{i i} - E^{1 1} : 2 \leq i \leq n}.
  \]
  Observe that
  \begin{align*}
             & \forall A \in \W, \begin{dcases}
      \tr{A} = 0 \\
      A = \sum_{i = 1}^n \sum_{j = 1}^n A_{i j} E^{i j}
    \end{dcases}                   &  & \text{(by \cref{ex:1.5.6})} \\
    \implies & \forall A \in \W, \begin{dcases}
      A_{1 1} + A_{2 2} + \cdots + A_{n n} = 0 \\
      A = \sum_{i = 1}^n \sum_{j = 1}^n A_{i j} E^{i j}
    \end{dcases}                   &  & \text{(by \cref{1.3.9})}    \\
    \implies & \forall A \in \W, \begin{dcases}
      A_{2 2} + \cdots + A_{n n} = -A_{1 1}        \\
      A = \pa{\sum_{i = 1}^n \sum_{\substack{j = 1 \\ j \neq i}}^n A_{i j} E^{i j}} + \pa{\sum_{i = 1}^n A_{i i} E^{i i}}
    \end{dcases}                                                    \\
    \implies & \forall A \in \W, A = \pa{\sum_{i = 1}^n \sum_{\substack{j = 1                                  \\ j \neq i}}^n A_{i j} E^{i j}} + \pa{\sum_{i = 2}^n A_{i i} (E^{i i} - E^{1 1})} \\
    \implies & \forall A \in \W, A \in \spn{\beta}                            &  & \text{(by \cref{1.4.3})}    \\
    \implies & \W \subseteq \spn{\beta}.
  \end{align*}
  Since
  \[
    \tr{E^{i j}} = 0 \quad \forall i, j \in \set{1, \dots, n} \text{ and } i \neq j
  \]
  and
  \[
    \tr{E^{i i} - E^{1 1}} = 1 + (-1) = 0 \quad \forall i \in \set{2, \dots, n},
  \]
  we know that \(\beta \subseteq \W\).
  Thus by \cref{1.5} we have \(\W = \spn{\beta}\).
  By \cref{ex:1.5.6} we know that \(\beta\) is linearly independent, thus by \cref{1.6.1} \(\beta\) is a basis for \(\W\), and by \cref{1.6.8} we know that \(\dim(\W) = n^2 - 1\).
\end{proof}

\begin{ex}\label{ex:1.6.16}
  The set of all upper triangular \(n \times n\) matrices is a subspace \(\W\) of \(\ms{n}{n}{\F}\) (see \cref{ex:1.3.12}).
  Find a basis for \(\W\).
  What is the dimension of \(\W\)?
\end{ex}

\begin{proof}[\pf{ex:1.6.16}]
  Let \(E^{i j} \in \ms{n}{n}{\F}\) be matrix defined as in \cref{ex:1.5.6} and let \(\beta\) be the set
  \[
    \beta = \set{E^{i j} : i, j \in \set{1, \dots, n}, i \leq j}.
  \]
  Clearly \(\beta \subseteq \W\).
  Since
  \begin{align*}
             & \forall A \in \W, A = \sum_{i = 1}^n \sum_{j = 1}^n A_{i j} E^{i j} = \sum_{i = 1}^n \sum_{j = i}^n A_{i j} E^{i j} &  & \text{(by \cref{ex:1.3.12})} \\
    \implies & \forall A \in \W, A \in \spn{\beta}                                                                                 &  & \text{(by \cref{1.4.3})}     \\
    \implies & \W = \spn{\beta}                                                                                                    &  & \text{(by \cref{1.5})}
  \end{align*}
  and \(\beta\) is linearly independent (by \cref{ex:1.5.6}), by \cref{1.6.1} we know that \(\beta\) is a basis for \(\W\).
  Thus by \cref{1.6.8} we have \(\dim(\W) = \frac{1}{2} n(n + 1)\).
\end{proof}

\begin{ex}\label{ex:1.6.17}
  The set of all skew-symmetric \(n \times n\) matrices is a subspace \(\W\) of \(\ms{n}{n}{\F}\) (see \cref{ex:1.3.28}).
  Find a basis for \(\W\).
  What is the dimension of \(\W\)?
\end{ex}

\begin{proof}[\pf{ex:1.6.17}]
  Let \(E^{i j} \in \ms{n}{n}{\F}\) be matrix defined as in \cref{ex:1.5.6} and let \(\beta\) be the set
  \[
    \beta = \set{E^{i j} - E^{j i} : i, j \in \set{1, \dots, n}, i < j}.
  \]
  By \cref{ex:1.3.28} we know that \(\beta \subseteq \W\).
  Since
  \begin{align*}
             & \forall A \in \W, \tp{A} = -A                                                  &  & \text{(by \cref{ex:1.3.28})} \\
    \implies & \forall A \in \W, A_{j i} = -A_{i j} \text{ where } i, j \in \set{1, \dots, n} &  & \text{(by \cref{1.3.3})}     \\
    \implies & \forall A \in \W, \begin{dcases}
      A_{i j} = -A_{j i} & \text{if } i \neq j \\
      A_{i j} = 0        & \text{if } i = j
    \end{dcases}                                                                     \\
    \implies & \forall A \in \W, A = \sum_{i = 1}^n \sum_{j = 1}^n A_{i j} E^{i j}                                              \\
             & = \sum_{i = 1}^n \sum_{j = 1}^{i - 1} (A_{i j} E^{i j} + A_{j i} E^{j i})                                        \\
             & = \sum_{i = 1}^n \sum_{j = 1}^{i - 1} (A_{i j} E^{i j} - A_{i j} E^{j i})                                        \\
             & = \sum_{i = 1}^n \sum_{j = 1}^{i - 1} A_{i j} (E^{i j} - E^{j i})                                                \\
    \implies & \forall A \in \W, A \in \spn{\beta}                                            &  & \text{(by \cref{1.4.3})}     \\
    \implies & \W = \spn{\beta}                                                               &  & \text{(by \cref{1.5})}
  \end{align*}
  and \(\beta\) is linearly independent (by \cref{ex:1.5.6}), by \cref{1.6.1} we know that \(\beta\) is a basis for \(\W\).
  Thus by \cref{1.6.8} we have \(\dim(\W) = \frac{1}{2} n(n - 1)\).
\end{proof}

\begin{ex}\label{ex:1.6.18}
  Find a basis for the vector space in \cref{1.2.13}.
  Justify your answer.
\end{ex}

\begin{proof}[\pf{ex:1.6.18}]
  Let \(\V\) be the vector space in \cref{1.2.13} over field \(\F\).
  We claim that the set
  \[
    \beta = \set{\set{e_n^i} : e_n^i = 0 \text{ when } n \neq i ; e_n^i = 1 \text{ when } n = i}
  \]
  is a basis for \(\V\).
  Clearly \(\beta \subseteq \V\).
  Since
  \begin{align*}
             & \forall \set{a_n} \in \V, \set{a_n} = \sum_{i \in \N} a_n \set{e_n^i} &  & \text{(by \cref{1.2.13})} \\
    \implies & \V = \spn{\beta}                                                      &  & \text{(by \cref{1.5})}
  \end{align*}
  and \(\beta\) is linearly independent (for obvious reason), by \cref{1.6.1} we know that \(\beta\) is a basis for \(\V\).
\end{proof}

\setcounter{ex}{19}
\begin{ex}\label{ex:1.6.20}
  Let \(\V\) be a vector space over \(\F\) having dimension \(n\), and let \(S\) be a subset of \(\V\) that generates \(\V\).
  \begin{enumerate}
    \item Prove that there is a subset of \(S\) that is a basis for \(\V\).
          (Be careful not to assume that \(S\) is finite.)
    \item Prove that \(S\) contains at least \(n\) vectors.
  \end{enumerate}
\end{ex}

\begin{proof}[\pf{ex:1.6.20}(a)]
  Note that the hypothesis of \cref{ex:1.6.20}(a) is different from \cref{1.9} that in \cref{1.9} \(S\) is assumed to be finite.
  Suppose for sake of contradiction that such subset does not exist.
  By \cref{1.6.1} this means every subset of \(S\) must either be linearly dependent or cannot generate \(\V\).
  Since \(S\) is a subset of \(S\) and \(\spn{S} = \V\), we know that \(S\) must be linearly dependent.

  Now we let \(\beta_0\) be a finite, linearly independent subset of \(S\).
  From previous paragraph we know that \(\beta_0 \neq S\) and \(\V \neq \spn{\beta_0}\).
  By \cref{1.10} we know that \(\#(\beta_0) \leq n\).
  Then there exists a \(v_1 \in S\) such that \(v_1 \cup \beta_0\) is linearly independent.
  If such \(v_1\) does not exist, then we would have \(S \subseteq \spn{\beta_0}\), which implies \(\spn{\beta_0} = \V\), a contradiction.
  So such \(v_1\) exists and we let \(\beta_1 = \beta_0 \cup v_1\).
  Again we must have \(\beta_1 \neq S\), \(\V \neq \spn{\beta_1}\) and \(\#(\beta_1) \leq n\).
  Using the same argument as above there must exist a \(v_2 \in S\) such that \(v_2 \cup \beta_1\) is linearly independent.
  Continue this definition we can define \(\beta_n = \set{\seq{v}{1,2,,n}}\).
  But by \cref{1.6.15}(b) we know that \(\beta_n\) must be a basis for \(\V\), a contradiction.
  Thus there must exists a subset of \(S\) which is a basis for \(\V\).
\end{proof}

\begin{proof}[\pf{ex:1.6.20}(b)]
  From \cref{ex:1.6.20}(a) we know that there exists a subset of \(S\) which is a basis for \(\V\).
  Thus by \cref{1.6.15}(a) \(S\) must has at least \(n\) vectors.
\end{proof}

\begin{ex}\label{ex:1.6.21}
  Prove that a vector space is infinite-dimensional if and only if it contains an infinite linearly independent subset.
\end{ex}

\begin{proof}[\pf{ex:1.6.21}]
  Let \(\V\) be a vector spaces over \(\F\).
  Then
  \begin{align*}
         & \V \text{ is infinite-dimensional}                                                    \\
    \iff & \V \text{ has an infinite basis}                        &  & \text{(by \cref{1.6.8})} \\
    \iff & \V \text{ has an infinite linearly independent subset}. &  & \text{(by \cref{1.6.1})}
  \end{align*}
\end{proof}

\begin{ex}\label{ex:1.6.22}
  Let \(\W_1\) and \(\W_2\) be subspaces of a finite-dimensional vector space \(\V\) over \(\F\).
  Determine necessary and sufficient conditions on \(\W_1\) and \(\W_2\) so that \(\dim(\W_1 \cap \W_2) = \dim(\W_1)\).
\end{ex}

\begin{proof}[\pf{ex:1.6.22}]
  We have
  \begin{align*}
         & \dim(\W_1 \cap \W_2) = \dim(\W_1)                                                     \\
    \iff & \begin{dcases}
      \exists \beta_1 \subseteq \W_1 \cap \W_2 \\
      \exists \beta_2 \subseteq \W_1
    \end{dcases} : \begin{dcases}
      \beta_1 \text{ is a basis for } \W_1 \cap \W_2 \\
      \beta_2 \text{ is a basis for } \W_1           \\
      \#(\beta_1) = \#(\beta_2)
    \end{dcases} &  & \text{(by \cref{1.6.8})} \\
    \iff & \W_1 \cap \W_2 = \W_1                                   &  & \text{(by \cref{1.11})}  \\
    \iff & \W_1 \subseteq \W_2.
  \end{align*}
\end{proof}

\begin{ex}\label{ex:1.6.23}
  Let \(\seq{v}{1,2,,k}, v\) be vectors in a vector space \(\V\), and define \(W_1 = \spn{\set{\seq{v}{1,2,,k}}}\), and \(\W_2 = \spn{\set{\seq{v}{1,2,,k}, v}}\).
  \begin{enumerate}
    \item Find necessary and sufficient conditions on \(v\) such that \(\dim(\W_1) = \dim(\W_2)\).
    \item State and prove a relationship involving \(\dim(\W_1)\) and \(\dim(\W_2)\) in the case that \(\dim(\W_1) \neq \dim(\W_2)\).
  \end{enumerate}
\end{ex}

\begin{proof}[\pf{ex:1.6.23}(a)]
  We have
  \begin{align*}
         & \begin{dcases}
      \W_1 \subseteq \W_2 \\
      \dim(\W_1) = \dim(\W_2)
    \end{dcases}                                  &  & \text{(by \cref{ex:1.4.13})} \\
    \iff & \W_1 = \W_2                                                 &  & \text{(by \cref{1.11})}      \\
    \iff & v \in \spn{\set{\seq{v}{1,2,,k}}}                           &  & \text{(by \cref{1.4.3})}     \\
    \iff & v \cup \set{\seq{v}{1,2,,k}} \text{ is linearly dependent}. &  & \text{(by \cref{1.7})}
  \end{align*}
\end{proof}

\begin{proof}[\pf{ex:1.6.23}(b)]
  By \cref{1.11} we have
  \[
    \begin{dcases}
      \W_1 \subseteq \W_2 \\
      \W_1 \neq \W_2
    \end{dcases} \implies \dim(\W_1) \neq \dim(\W_2).
  \]
\end{proof}

\begin{ex}\label{ex:1.6.24}
  Let \(f(x)\) be a polynomial of degree n in \(\ps[n]{\R}\).
  Prove that for any \(g(x) \in \ps[n]{\R}\) there exist scalars \(\seq{c}{0,1,,n}\) such that
  \[
    g(x) = c_0 f(x) + c_1 f'(x) + c_2 f''(x) + \cdots + c_n f^{(n)}(x),
  \]
  where \(f^{(n)}(x)\) denotes the \(n\)th derivative of \(f(x)\).
\end{ex}

\begin{proof}[\pf{ex:1.6.24}]
  We denote \(f^{(0)} = f\).
  Since \(f^{(i)}(x)\) has degree \(n - i\) for all \(i = 0, 1, \dots, n\), we know that the set
  \[
    \beta = \set{f^{(i)} : i = 0, 1, \dots, n}
  \]
  is linearly independent.
  Since \(\#(\beta) = n + 1\), by \cref{1.6.15}(b) we know that \(\beta\) is a basis for \(\ps[n]{\R}\), thus there exist \(\seq{c}{0,1,,n}\) such that \(g = \sum_{i = 0}^n c_i f^{(i)}\).
\end{proof}

\begin{ex}\label{ex:1.6.25}
  If \(\V\) and \(\W\) are vector spaces over \(\F\) of dimensions \(m\) and \(n\), determine the dimension of \(\V \times \W\) (see \cref{ex:1.2.21}).
\end{ex}

\begin{proof}[\pf{ex:1.6.25}]
  Let \(\beta_v = \set{\seq{v}{1,2,,m}}, \beta_w = \set{\seq{w}{1,2,n}}\) be a basis for \(\V, \W\), respectively.
  Let \(\zv_v, \zv_w\) be the zero vectors of \(\V, \W\), respectively.
  Then we claim the set
  \[
    \beta = \pa{\beta_v \times \set{\zv_w}} \cup \pa{\set{\zv_v} \times \beta_w}
  \]
  is a basis for \(\V \times \W\).
  Clearly \(\beta \subseteq \V \times \W\).
  Since
  \begin{align*}
             & \forall \seq{a}{1,2,,m + n} \in \F, \sum_{i = 1}^m a_i (v_i, \zv_w) + \sum_{i = 1}^n a_{m + i} (\zv_v, w_i)                                   \\
             & = (\sum_{i = 1}^m a_i v_i, \sum_{i = 1}^n a_{m + i} w_i) = (\zv_v, \zv_w)                                   &  & \text{(by \cref{ex:1.2.21})} \\
    \implies & \begin{dcases}
      \sum_{i = 1}^m a_i v_i = \zv_v \\
      \sum_{i = 1}^n a_{m + i} w_i = \zv_w
    \end{dcases}                                                                                                                    \\
    \implies & \seq[=]{a}{1,2,,m + n} = 0,
  \end{align*}
  by \cref{1.5.3} we know that \(\beta\) is linearly independent.
  Since
  \begin{align*}
             & \forall (v, w) \in \V \times \W, \exists \seq{a}{1,2,,m + n} \in \F :                                       \\
             & (v, w) = (\sum_{i = 1}^m a_i v_i, \sum_{i = 1}^n a_{m + i} w_i)                                             \\
             & = \sum_{i = 1}^m a_i (v_i, \zv_w) + \sum_{i = 1}^n a_{m + i} (\zv_v, w_i) &  & \text{(by \cref{ex:1.2.21})} \\
    \implies & \forall (v, w) \in \V \times \W, (v, w) \in \spn{\beta}                                                     \\
    \implies & \V \times \W = \spn{\beta},                                               &  & \text{(by \cref{1.5})}
  \end{align*}
  by \cref{1.6.1} we know that \(\beta\) is a basis for \(\V \times \W\) and \(\dim(\V \times \W) = m + n\).
\end{proof}

\begin{ex}\label{ex:1.6.26}
  For a fixed \(a \in \R\), determine the dimension of the subspace of \(\ps[n]{\R}\) defined by \(\set{f \in \ps[n]{\R} : f(a) = 0}\).
\end{ex}

\begin{proof}[\pf{ex:1.6.26}]
  Since the set
  \[
    \beta = \set{x - a, x^2 - a^2, \dots, x^n - a^n} \subseteq \set{f \in \ps[n]{\R} : f(a) = 0} \subseteq \ps[n]{\R}
  \]
  is linearly independent (by \cref{ex:1.5.5}) and
  \begin{align*}
             & \forall g \in \set{f \in \ps[n]{\R} : f(a) = 0}, \exists \seq{c}{1,2,,n} \in \R :                               \\
             & \forall x \in \R, g(x) = c_1 (x - a) + c_2 (x^2 - a^2) + \cdots + c_n (x^n - a^n)                               \\
    \implies & \forall g \in \set{f \in \ps[n]{\R} : f(a) = 0}, g \in \spn{\beta}                &  & \text{(by \cref{1.4.3})} \\
    \implies & \set{f \in \ps[n]{\R} : f(a) = 0} = \spn{\beta},                                  &  & \text{(by \cref{1.5})}
  \end{align*}
  by \cref{1.6.1} we know that \(\beta\) is a basis for \(\set{f \in \ps[n]{\R} : f(a) = 0}\).
  Thus by \cref{1.6.8} we have \(\dim(\set{f \in \ps[n]{\R} : f(a) = 0}) = \#(\beta) = n\).
\end{proof}

\begin{ex}\label{ex:1.6.27}
  Let \(\W_1\) and \(\W_2\) be the subspaces of \(\ps{\F}\) defined in \cref{ex:1.3.25}.
  Determine the dimensions of the subspaces \(\W_1 \cap \ps[n]{\F}\) and \(\W_2 \cap \ps[n]{\F}\).
\end{ex}

\begin{proof}[\pf{ex:1.6.27}]
  First suppose that \(n\) is odd.
  We define
  \begin{align*}
    \beta_1 & = \set{1, x^2, x^4 \dots, x^{n - 1}} \\
    \beta_2 & = \set{x, x^3, x^5 \dots, x^n}.
  \end{align*}
  Then we have
  \begin{align*}
             & \begin{dcases}
      \beta_1 \text{ is linearly independent} \\
      \beta_2 \text{ is linearly independent} \\
      \W_1 = \spn{\beta_1}                    \\
      \W_2 = \spn{\beta_2}
    \end{dcases}  &  & \text{(by \cref{ex:1.3.25})} \\
    \implies & \begin{dcases}
      \beta_1 \text{ is a basis for } \W_1 \\
      \beta_2 \text{ is a basis for } \W_2
    \end{dcases}  &  & \text{(by \cref{1.6.1})}     \\
    \implies & \begin{dcases}
      \dim(\W_1) = \#(\beta_1) = \frac{n - 1}{2} \\
      \dim(\W_2) = \#(\beta_2) = \frac{n - 1}{2}
    \end{dcases}. &  & \text{(by \cref{1.6.8})}
  \end{align*}
  Using similar arguments as above we can show that when \(n\) is even we have
  \[
    \begin{dcases}
      \dim(\W_1) = \#(\beta_1) = \frac{n}{2} + 1 \\
      \dim(\W_2) = \#(\beta_2) = \frac{n}{2}
    \end{dcases}.
  \]
\end{proof}

\begin{ex}\label{ex:1.6.28}
  Let \(\V\) be a finite-dimensional vector space over \(\C\) with dimension \(n\).
  Prove that if \(\V\) is now regarded as a vector space over \(\R\), then \(\dim(\V) = 2n\).
\end{ex}

\begin{proof}[\pf{ex:1.6.28}]
  Let \(\beta = \set{\seq{v}{1,2,,n}}\) be a basis for \(\V\) over \(\C\).
  Since
  \begin{align*}
             & \spn{\beta} = \V                                                           &  & \text{(by \cref{1.6.1})}    \\
    \implies & \forall v \in \V, \exists \seq{a}{1,2,,n} \in \C :                                                          \\
             & v = \sum_{j = 1}^n a_j v_j = \sum_{j = 1}^n \pa{\Re(a_j) + i \Im(a_j)} v_j                                  \\
             & = \sum_{j = 1}^n \Re(a_j) (1 v_j) + \sum_{j = 1}^n \Im(a_j) (i v_j)        &  & \text{(by \cref{1.4.3})}    \\
    \implies & \forall v \in \V, \exists \seq{b}{1,2,,2n} \in \R :                                                         \\
             & v = \sum_{j = 1}^n b_j v_j + \sum_{j = 1}^n b_{n + j} (i v_j)              &  & (\Re(a_j), \Im(a_j) \in \R) \\
    \implies & \spn{\set{\seq{v}{1,2,,n}, \seq{iv}{1,2,,n}}} = \V                         &  & \text{(by \cref{1.4.3})}
  \end{align*}
  and
  \begin{align*}
             & \forall \seq{b}{1,2,,2n} \in \R, \sum_{j = 1}^n b_j v_j + \sum_{j = 1}^n b_{n + j} (i v_j) = \zv                                            \\
    \implies & \sum_{j = 1}^n (b_j + i b_{n + j}) v_j = \zv                                                                                                \\
    \implies & b_1 + i b_{n + 1} = b_2 + i b_{n + 2} = \cdots = b_n + i b_{2n} = 0                              &  & \text{(by \cref{1.5.3})}              \\
    \implies & \seq[=]{b}{1,2,,2n} = 0,                                                                         &  & (\set{\seq{b}{1,2,,2n}} \subseteq \R)
  \end{align*}
  by \cref{1.6.1} we know that \(\set{\seq{v}{1,2,,n}, \seq{iv}{1,2,,n}}\) is a basis for \(\V\) over \(\R\) with dimension \(2n\).
\end{proof}

\begin{ex}\label{ex:1.6.29}
  \quad
  \begin{enumerate}
    \item Prove that if \(\W_1\) and \(\W_2\) are finite-dimensional subspaces of a vector space \(\V\) over \(\F\), then the subspace \(\W_1 + \W_2\) is finite-dimensional, and \(\dim(\W_1 + \W_2) = \dim(\W_1) + \dim(\W_2) - \dim(\W_1 \cap \W_2)\).
    \item Let \(\W_1\) and \(\W_2\) be finite-dimensional subspaces of a vector space \(\V\) over \(\F\), and let \(\V = \W_1 + \W_2\).
          Deduce that \(\V\) is the direct sum of \(\W_1\) and \(\W_2\) if and only if \(\dim(\V) = \dim(\W_1) + \dim(\W_2)\).
  \end{enumerate}
\end{ex}

\begin{proof}[\pf{ex:1.6.29}(a)]
  If \(\dim(\W_1) = \dim(\W_2) = 0\), then we have
  \[
    \dim(\W_1 + \W_2) = \dim(\set{\zv}) = 0 = 0 + 0 - 0 = \dim(\W_1) + \dim(\W_2) - \dim(\W_1 \cap \W_2).
  \]
  So suppose that \(\dim(\W_1) > 0\) or \(\dim(\W_2) > 0\).

  By \cref{1.4} we know that \(\W_1 \cap \W_2\) is a subspace of \(\W_1\) and \(\W_2\) over \(\F\).
  Since \(\W_1\) and \(\W_2\) are finite-dimensional, by \cref{1.11} we know that \(\W_1 \cap \W_2\) is finite-dimensional.

  Let \(\beta = \set{\seq{u}{1,2,,k}}\) be a basis for \(\W_1 \cap \W_2\) over \(\F\).
  Note that \(k \geq 0\) and when \(k = 0\) we have \(\beta = \varnothing\).
  By \cref{1.6.15}(c) we can extend \(\beta\) to another set \(\beta_1 = \set{\seq{u}{1,2,,k}, \seq{v}{1,2,,m}}\) which is basis for \(\W_1\) over \(\F\).
  Using similar argument we can extend \(\beta\) to \(\beta_2 = \set{\seq{u}{1,2,,k}, \seq{w}{1,2,,p}}\) which is a basis for \(\W_2\) over \(\F\).
  Note that \(m, p \geq 0\) and \(\max(m, p) \geq 1\).
  Since
  \begin{align*}
             & \begin{dcases}
      \forall v \in \beta_1, v + \zv \in \W_1 + \W_2 \\
      \forall w \in \beta_2, \zv + w \in \W_1 + \W_2
    \end{dcases}                 &  & \text{(by \cref{1.3.10})} \\
    \implies & \beta_1 \cup \beta_2 \subseteq \W_1 + \W_2
  \end{align*}
  and
  \begin{align*}
             & \forall x \in \W_1 + \W_2, \exists (y, z) \in \W_1 \times \W_2 : x = y + z                                                          &  & \text{(by \cref{1.3.10})} \\
    \implies & \forall x \in \W_1 + \W_2, \begin{dcases}
      \exists \seq{a}{1,2,,k + m} \in \F \\
      \exists \seq{b}{1,2,,k + p} \in \F
    \end{dcases} :                                                                                                            \\
             & x = \pa{\sum_{i = 1}^k a_i u_i + \sum_{i = k + 1}^{k + m} a_i v_i} + \pa{\sum_{i = 1}^k b_i u_i + \sum_{i = k + 1}^{k + p} b_i w_i} &  & \text{(by \cref{1.6.1})}  \\
             & = \sum_{i = 1}^k (a_i + b_i) u_i + \sum_{i = k + 1}^{k + m} a_i v_i + \sum_{i = k + 1}^{k + p} b_i w_i                              &  & \text{(by \cref{1.2.1})}  \\
    \implies & \forall x \in \W_1 + \W_2, \exists \seq{a}{1,2,,k + m + p} \in \F :                                                                                                \\
             & x = \sum_{i = 1}^k a_i u_i + \sum_{i = k + 1}^{k + m} a_i v_i + \sum_{i = k + m + 1}^{k + m + p} a_i w_i                                                           \\
    \implies & \forall x \in \W_1 + \W_2, x \in \spn{\beta_1 \cup \beta_2}                                                                         &  & \text{(by \cref{1.4.3})}  \\
    \implies & \W_1 + \W_2 = \spn{\beta_1 \cup \beta_2},                                                                                           &  & \text{(by \cref{1.5})}
  \end{align*}
  by \cref{1.9} we know that \(\dim(\W_1 + \W_2) \leq \#(\beta_1 \cup \beta_2)\) and there exist some \(S \subseteq \beta_1 \cup \beta_2\) such that \(S\) is a basis for \(\W_1 + \W_2\) over \(\F\).
  We now claim that \(\beta_1 \cup \beta_2\) is a basis for \(\W_1 + \W_2\) over \(\F\).
  By \cref{1.6.1} we need to show that \(\beta_1 \cup \beta_2\) is linearly independent.
  Since
  \begin{align*}
             & \beta_1 \cap \beta_2 = \beta                                     \\
    \implies & \begin{dcases}
      \spn{\beta_1 \setminus \beta} \cap \beta_2 = \varnothing \\
      \spn{\beta_2 \setminus \beta} \cap \beta_1 = \varnothing
    \end{dcases}   &  & (\spn{\beta} = \W_1 \cap \W_2) \\
    \implies & \begin{dcases}
      \spn{\beta_1} \cap \beta_2 = \varnothing \\
      \spn{\beta_2} \cap \beta_1 = \varnothing
    \end{dcases},
  \end{align*}
  we know that
  \begin{align*}
             & \forall \seq{a}{1,2,,k + m + p} \in \F,                                                                                                  \\
             & \sum_{i = 1}^k a_i u_i + \sum_{i = k + 1}^{k + m} a_i v_i + \sum_{i = k + m + 1}^{k + m + p} a_i w_i = \zv                               \\
    \implies & \sum_{i = 1}^k a_i u_i + \sum_{i = k + 1}^{k + m} a_i v_i = \sum_{i = k + m + 1}^{k + m + p} -a_i w_i      &  & \text{(by \cref{1.2.1})} \\
    \implies & \seq[=]{a}{1,2,,k + m + p} = 0.                                                                            &  & \text{(by \cref{1.5.3})}
  \end{align*}
  Thus by \cref{1.5.3} \(\beta_1 \cup \beta_2\) is linearly independent and by \cref{1.6.1} \(\beta_1 \cup \beta_2\) is a basis for \(\W_1 + \W_2\).
  Then we have
  \[
    \dim(\W_1 + \W_2) = k + m + p = (k + m) + (k + p) - k = \dim(\W_1) + \dim(\W_2) - \dim(\W_1 \cap \W_2).
  \]
\end{proof}

\begin{proof}[\pf{ex:1.6.29}(b)]
  We have
  \begin{align*}
         & \V = \W_1 \oplus \W_2                                                                        \\
    \iff & \begin{dcases}
      \V = \W_1 + \W_2 \\
      \W_1 \cap \W_2 = \varnothing
    \end{dcases}                             &  & \text{(by \cref{1.3.11})}       \\
    \iff & \begin{dcases}
      \V = \W_1 + \W_2 \\
      \dim(\W_1 \cap \W_2) = 0
    \end{dcases}                             &  & \text{(by \cref{1.6.2})}        \\
    \iff & \dim(\V) = \dim(\W_1 + \W_2) = \dim(\W_1) + \dim(\W_2). &  & \text{(by \cref{ex:1.6.29}(a))}
  \end{align*}
\end{proof}

\setcounter{ex}{30}
\begin{ex}\label{ex:1.6.31}
  Let \(\W_1\) and \(\W_2\) be subspaces of a vector space \(\V\) over \(\F\) having dimensions \(m\) and \(n\), respectively, where \(m \geq n\).
  \begin{enumerate}
    \item Prove that \(\dim(\W_1 \cap \W_2) \leq n\).
    \item Prove that \(\dim(\W_1 + \W_2) \leq m + n\).
  \end{enumerate}
\end{ex}

\begin{proof}[\pf{ex:1.6.31}(a)]
  We have
  \begin{align*}
             & \W_1 \cap \W_2 \subseteq \W_2                                          \\
    \implies & \dim(\W_1 \cap \W_2) \leq \dim(\W_2) = n. &  & \text{(by \cref{1.11})}
  \end{align*}
\end{proof}

\begin{proof}[\pf{ex:1.6.31}(b)]
  We have
  \begin{align*}
    \dim(\W_1 + \W_2) & = \dim(\W_1) + \dim(\W_2) - \dim(\W_1 \cap \W_2) &  & \text{(by \cref{ex:1.6.29}(a))} \\
                      & = m + n - \dim(\W_1 \cap \W_2)                                                        \\
                      & \leq m + n.
  \end{align*}
\end{proof}

\setcounter{ex}{32}
\begin{ex}\label{ex:1.6.33}
  \quad
  \begin{enumerate}
    \item Let \(\W_1\) and \(\W_2\) be subspaces of a vector space \(\V\) over \(\F\) such that \(\V = \W_1 \oplus \W_2\).
          If \(\beta_1\) and \(\beta_2\) are bases for \(\W_1\) and \(\W_2\) over \(\F\), respectively, show that \(\beta_1 \cap \beta_2 = \varnothing\) and \(\beta_1 \cup \beta_2\) is a basis for \(\V\) over \(\F\).
    \item Conversely, let \(\beta_1\) and \(\beta_2\) be disjoint bases for subspaces \(\W_1\) and \(\W_2\), respectively, of a vector space \(\V\) over \(\F\).
          Prove that if \(\beta_1 \cup \beta_2\) is a basis for \(\V\) over \(\F\), then \(\V = \W_1 \oplus \W_2\).
  \end{enumerate}
\end{ex}

\begin{proof}[\pf{ex:1.6.33}(a)]
  First observe that
  \begin{align*}
             & \V = \W_1 \oplus \W_2                                                                                              \\
    \implies & \V = \W_1 + \W_2                                                                    &  & \text{(by \cref{1.3.11})} \\
    \implies & \forall v \in \V, \exists (x, y) \in \W_1 \times \W_2 : v = x + y                                                  \\
    \implies & \forall v \in \V, \exists (x, y) \in \spn{\beta_1} \times \spn{\beta_2} : v = x + y &  & \text{(by \cref{1.6.1})}  \\
    \implies & \forall v \in \V, v \in \spn{\beta_1 \cup \beta_2}                                  &  & \text{(by \cref{1.2.1})}  \\
    \implies & \V = \spn{\beta_1 \cup \beta_2}.                                                    &  & \text{(by \cref{1.5})}
  \end{align*}
  Since
  \begin{align*}
             & V = \W_1 \oplus \W_2                                                                             \\
    \implies & \W_1 \cap \W_2 = \set{\zv}                                  &  & \text{(by \cref{1.3.11})}       \\
    \implies & \beta_1 \cap \beta_2 = \varnothing                                                               \\
    \implies & \dim(\V) = \dim(\W_1 \oplus \W_2) = \dim(\W_1) + \dim(\W_2) &  & \text{(by \cref{ex:1.6.29}(b))} \\
             & = \#(\beta_1) + \#(\beta_2) = \#(\beta_1 \cup \beta_2),
  \end{align*}
  by \cref{1.6.15}(a) we know that \(\beta_1 \cup \beta_2\) is a basis for \(\V\) over \(\F\).
\end{proof}

\begin{proof}[\pf{ex:1.6.33}(b)]
  Since
  \begin{align*}
             & \begin{dcases}
      \beta_1 \cap \beta_2 = \varnothing \\
      \spn{\beta_1 \cup \beta_2} = \V
    \end{dcases}                                                                                           \\
    \implies & \forall v \in \V, v \in \spn{\beta_1 \cup \beta_2}                                                                    \\
    \implies & \forall v \in \V, \exists (x, y) \in \spn{\beta_1} \times \spn{\beta_2} : v = x + y &  & \text{(by \cref{1.2.1})}     \\
    \implies & \forall v \in \V, v \in \W_1 + \W_2                                                 &  & \text{(by \cref{1.3.10})}    \\
    \implies & \V \subseteq \W_1 + \W_2                                                            &  & \text{(by \cref{1.5})}       \\
    \implies & \V = \W_1 + \W_2                                                                    &  & \text{(by \cref{ex:1.3.23})}
  \end{align*}
  and
  \begin{align*}
             & \begin{dcases}
      \beta_1 \cap \beta_2 = \varnothing \\
      \beta_1 \cup \beta_2 \text{ is a basis for } \V \text{ over } \F
    \end{dcases}                                                \\
    \implies & \begin{dcases}
      \spn{\beta_1} \cap \beta_2 = \varnothing \\
      \spn{\beta_2} \cap \beta_1 = \varnothing
    \end{dcases}                  &  & \text{(by \cref{1.7})}   \\
    \implies & \spn{\beta_1} \cap \spn{\beta_2} = \set{\zv} &  & \text{(by \cref{1.5.3})} \\
    \implies & \W_1 \cap \W_2 = \set{\zv},
  \end{align*}
  by \cref{1.3.11} we know that \(\V = \W_1 \oplus \W_2\).
\end{proof}


\chapter{Linear Transformations and Matrices}\label{ch:2}

% All sections are in separated files.  We include them here.
\section{Linear Transformations, Null Spaces and Ranges}\label{sec:2.1}

\begin{defn}\label{2.1.1}
  Let \(\V\) and \(\W\) be vector spaces over \(\F\).
  We call a function \(\T : \V \to \W\) a \textbf{linear transformation from \(\V\) to \(\W\)} if, for all \(x, y \in \V\) and \(c \in \F\), we have
  \begin{enumerate}
    \item \(\T(x + y) = \T(x) + \T(y)\) and
    \item \(\T(cx) = c\T(x)\).
  \end{enumerate}
  We often simply call \(\T\) \textbf{linear}.
\end{defn}

\begin{note}
  If the underlying field \(\F\) is the field of rational numbers, then \cref{2.1.1}(a) implies \cref{2.1.1}(b) (see \cref{ex:2.1.37}), but, in general \cref{2.1.1}(a)(b) are logically independent.
\end{note}

\begin{prop}\label{2.1.2}
  Let \(\V\) and \(\W\) be vector spaces over \(\F\) and let \(\T : \V \to \W\) be a function.
  \begin{enumerate}
    \item If \(\T\) is linear, then \(\T(\zv_{\V}) = \zv_{\W}\).
    \item \(\T\) is linear if and only if \(\T(cx + y) = c\T(x) + \T(y)\) for all \(x, y \in \V\) and \(c \in \F\).
    \item If \(\T\) is linear, then \(\T(x - y) = \T(x) - \T(y)\) for all \(x, y \in \V\).
    \item \(\T\) is linear if and only if, for \(\seq{x}{1,2,,n} \in \V\) and \(\seq{a}{1,2,,n} \in F\), we have
          \[
            \T\pa{\sum_{i = 1}^n a_i x_i} = \sum_{i = 1}^n a_i \T(x_i).
          \]
  \end{enumerate}
\end{prop}

\begin{proof}[\pf{2.1.2}(a)]
  We have
  \begin{align*}
             & \T \text{ is linear}                                                                   \\
    \implies & \T(\zv_{\V}) + \T(\zv_{\V}) = \T(\zv_{\V} + \zv_{\V}) &  & \text{(by \cref{2.1.1}(a))} \\
             & = \T(\zv_{\V}) = \T(\zv_{\V}) + \zv_{\W}              &  & \text{(by \ref{vs3})}       \\
    \implies & \T(\zv_{\V}) = \zv_{\W}.                              &  & \text{(by \cref{1.1})}
  \end{align*}
\end{proof}

\begin{proof}[\pf{2.1.2}(b)]
  We have
  \begin{align*}
             & \T \text{ is linear}                                                                                  \\
    \implies & \begin{dcases}
      \forall x, y \in \V \\
      \forall c \in \F
    \end{dcases}, \begin{dcases}
      \T(x + y) = \T(x) + \T(y) \\
      \T(cx) = c\T(x)
    \end{dcases}                    &  & \text{(by \cref{2.1.1})} \\
    \implies & \begin{dcases}
      \forall x, y \in \V \\
      \forall c \in \F
    \end{dcases}, \T(cx + y) = \T(cx) + \T(y) = c\T(x) + \T(y) &  & \text{(by \cref{2.1.1})}
  \end{align*}
  and
  \begin{align*}
             & \begin{dcases}
      \forall x, y \in \V \\
      \forall c \in \F
    \end{dcases}, \T(cx + y) = c\T(x) + \T(y)                                  \\
    \implies & \begin{dcases}
      \forall x, y \in \V \\
      \forall c \in \F
    \end{dcases}, \begin{dcases}
      \T(x + y) = \T(x) + \T(y)             & \text{if } c = 0        \\
      \T(cx + \zv_{\V}) = c\T(x) + \zv_{\W} & \text{if } y = \zv_{\V}
    \end{dcases}  &  & \text{(by \cref{2.1.2}(a))} \\
    \implies & \begin{dcases}
      \forall x, y \in \V \\
      \forall c \in \F
    \end{dcases}, \begin{dcases}
      \T(x + y) = \T(x) + \T(y) & \text{if } c = 0        \\
      \T(cx) = c\T(x)           & \text{if } y = \zv_{\V}
    \end{dcases} &  & \text{(by \ref{vs3})}       \\
    \implies & \T \text{ is linear}.                                  &  & \text{(by \cref{2.1.1})}
  \end{align*}
  Thus
  \[
    \T \text{ is linear} \iff \begin{dcases}
      \forall x, y \in \V \\
      \forall c \in \F
    \end{dcases}, \T(cx + y) = c\T(x) + \T(y).
  \]
\end{proof}

\begin{proof}[\pf{2.1.2}(c)]
  For all \(x, y \in \V\), we have
  \begin{align*}
    \T(x - y) & = \T(x + (-1)y)      &  & \text{(by \cref{1.2}(b))}   \\
              & = \T(x) + \T((-1)y)  &  & \text{(by \cref{2.1.1}(a))} \\
              & = \T(x) + (-1) \T(y) &  & \text{(by \cref{2.1.1}(b))} \\
              & = \T(x) - \T(y).     &  & \text{(by \cref{1.2}(b))}
  \end{align*}
\end{proof}

\begin{proof}[\pf{2.1.2}(d)]
  We have
  \begin{align*}
             & \T \text{ is linear}                                                                                                  \\
    \implies & \begin{dcases}
      \forall \seq{x}{1,2,,n} \in \V \\
      \forall \seq{a}{1,2,,n} \in \F
    \end{dcases},                                                                                           \\
             & \T\pa{\sum_{i = 1}^n a_i x_i} = \sum_{i = 1}^n \T(a_i x_i) = \sum_{i = 1}^n a_i \T(x_i) &  & \text{(by \cref{2.1.1})}
  \end{align*}
  and
  \begin{align*}
             & \begin{dcases}
      \forall \seq{x}{1,2,,n} \in \V \\
      \forall \seq{a}{1,2,,n} \in \F
    \end{dcases},                                                                 \\
             & \T\pa{\sum_{i = 1}^n a_i x_i} = \sum_{i = 1}^n a_i \T(x_i) &  & \text{(by \cref{2.1.1})}    \\
    \implies & \begin{dcases}
      \forall x, y \in \V \\
      \forall c \in \F
    \end{dcases},                                                                 \\
             & \T(cx + 1y) = c\T(x) + 1\T(y) = c\T(x) + \T(y)             &  & \text{(by \ref{vs5})}       \\
    \implies & \T \text{ is linear}.                                      &  & \text{(by \cref{2.1.2}(b))}
  \end{align*}
  Thus
  \[
    \T \text{ is linear} \iff \begin{dcases}
      \forall \seq{x}{1,2,,n} \in \V \\
      \forall \seq{a}{1,2,,n} \in \F
    \end{dcases}, \T\pa{\sum_{i = 1}^n a_i x_i} = \sum_{i = 1}^n a_i \T(x_i).
  \]
\end{proof}

\begin{note}
  We generally use \cref{2.1.2}(b) to prove that a given transformation is linear.
\end{note}

\begin{eg}\label{2.1.3}
  For any angle \(\theta\), define \(\T_{\theta} : \R^2 \to \R^2\) by the rule: \(\T_{\theta}(a_1, a_2)\) is the vector obtained by rotating \((a_1, a_2)\) counterclockwise by \(\theta\) if \((a_1, a_2) \neq (0, 0)\), and \(\T_{\theta}(0, 0) = (0, 0)\).
  Then \(\T_{\theta} : \R^2 \to \R^2\) is a linear transformation that is called the \textbf{rotation by \(\theta\)}.

  We determine an explicit formula for \(\T_{\theta}\).
  Fix a nonzero vector \((a_1, a_2) \in \R^2\).
  Let \(\alpha\) be the angle that \((a_1, a_2)\) makes with the positive \(x\)-axis, and let \(r = \sqrt{a_1^2 +a_2^2}\).
  Then \(a_1 = r \cos(\alpha)\) and \(a_2 = r \sin(\alpha)\).
  Also, \(\T_{\theta}(a_1, a_2)\) has length \(r\) and makes an angle \(\alpha + \theta\) with the positive \(x\)-axis.
  It follows that
  \begin{align*}
    \T_{\theta}(a_1, a_2) & = (r \cos(\alpha + \theta), r \sin(\alpha + \theta))                                                                     \\
                          & = (r \cos(\alpha) \cos(\theta) - r \sin(\alpha) \sin(\theta), r \cos(\alpha) \sin(\theta) + r \sin(\alpha) \cos(\theta)) \\
                          & = (a_1 \cos(\theta) - a_2 \sin(\theta), a_1 \sin(\theta) + a_2 \cos(\theta)).
  \end{align*}
  Finally, observe that this same formula is valid for \((a_1 ,a_2) = (0, 0)\).
  It is now easy to show that \(\T_{\theta}\) is linear.
\end{eg}

\begin{proof}[\pf{2.1.3}]
  For all \(x, y \in \R^2\) and \(c \in \R\), we have
  \begin{align*}
    \T_{\theta}(cx + y) & = \T_{\theta}(cx_1 + y_1, cx_2 + y_2)                                              &  & \text{(by \cref{1.2.4})} \\
                        & = ((cx_1 + y_1) \cos(\theta) - (cx_2 + y_2)\sin(\theta),                                                         \\
                        & \quad (cx_1 + y_1) \sin(\theta) + (cx_2 + y_2) \cos(\theta))                       &  & \text{(by \cref{2.1.3})} \\
                        & = c (x_1 \cos(\theta) - x_2 \sin(\theta), x_1 \sin(\theta) + x_2 \cos(\theta))     &  & \text{(by \cref{1.2.1})} \\
                        & \quad + (y_1 \cos(\theta) - y_2 \sin(\theta), y_1 \sin(\theta) + y_2 \cos(\theta))                               \\
                        & = c\T(x_1, x_2) + \T(y_1, y_2)                                                     &  & \text{(by \cref{2.1.3})} \\
                        & = c\T_{\theta}(x) + \T_{\theta}(y).                                                &  & \text{(by \cref{1.2.4})}
  \end{align*}
  Thus by \cref{2.1.2}(b) \(\T_{\theta}\) is linear.
\end{proof}

\begin{eg}\label{2.1.4}
  Define \(\T : \R^2 \to \R^2\) by \(\T(a_1, a_2) = (a_1, -a_2)\).
  \(\T\) is called the \textbf{reflection about the \(x\)-axis} and \(\T\) is linear.
\end{eg}

\begin{proof}[\pf{2.1.4}]
  For all \(x, y \in \R^2\) and \(c \in \R\), we have
  \begin{align*}
    \T(cx + y) & = \T(cx_1 + y_1, cx_2 + y_2)   &  & \text{(by \cref{1.2.4})} \\
               & = (cx_1 + y_1, -cx_2 - y_2)    &  & \text{(by \cref{2.1.4})} \\
               & = c(x_1, -x_2) + (y_1, -y_2)   &  & \text{(by \cref{1.2.1})} \\
               & = c\T(x_1, x_2) + \T(y_1, y_2) &  & \text{(by \cref{2.1.4})} \\
               & = c\T(x) + \T(y).              &  & \text{(by \cref{1.2.4})}
  \end{align*}
  Thus by \cref{2.1.2}(b) \(\T\) is linear.
\end{proof}

\begin{eg}\label{2.1.5}
  Define \(\T : \R^2 \to \R^2\) by \(\T(a_1, a_2) = (a_1, 0)\).
  \(\T\) is called the \textbf{projection on the \(x\)-axis} and \(\T\) is linear.
\end{eg}

\begin{proof}[\pf{2.1.5}]
  For all \(x, y \in \R^2\) and \(c \in \R\), we have
  \begin{align*}
    \T(cx + y) & = \T(cx_1 + y_1, cx_2 + y_2)   &  & \text{(by \cref{1.2.4})} \\
               & = (cx_1 + y_1, 0)              &  & \text{(by \cref{2.1.5})} \\
               & = c(x_1, 0) + (y_1, 0)         &  & \text{(by \cref{1.2.1})} \\
               & = c\T(x_1, x_2) + \T(y_1, y_2) &  & \text{(by \cref{2.1.5})} \\
               & = c\T(x) + \T(y).              &  & \text{(by \cref{1.2.4})}
  \end{align*}
  Thus by \cref{2.1.2}(b) \(\T\) is linear.
\end{proof}

\begin{eg}\label{2.1.6}
  Define \(\T : \ms{m}{n}{\F} \to \ms{n}{m}{\F}\) by \(\T(A) = \tp{A}\), where \(\tp{A}\) is the transpose of \(A\), defined in \cref{1.3.3}.
  Then \(\T\) is linear.
\end{eg}

\begin{proof}[\pf{2.1.6}]
  By \cref{ex:1.3.3} and \cref{2.1.2}(b) we see that \(\T\) is linear.
\end{proof}

\begin{eg}\label{2.1.7}
  Define \(\T : \ps{\R} \to \ps{\R}\) by \(\T(f) = f'\), where \(f'\) denotes the derivative of \(f\).
  Then \(\T\) is linear.
\end{eg}

\begin{proof}[\pf{2.1.7}]
  Let \(g, h \in \ps{\R}\) and \(a \in \R\).
  Now
  \[
    \T(ag + h) = (ag + h)' = ag' + h' = a\T(g) + \T(h).
  \]
  So by \cref{2.1.2}(b) \(\T\) is linear.
\end{proof}

\begin{eg}\label{2.1.8}
  Let \(\V = \cfs{\R}\), the vector space of continuous real-valued functions on \(\R\).
  Let \(a, b \in \R\), \(a < b\).
  Define \(\T : \V \to \R\) by
  \[
    \T(f) = \int_a^b f(t) \; dt
  \]
  for all \(f \in \V\).
  Then \(\T\) is linear.
\end{eg}

\begin{proof}[\pf{2.1.8}]
  Let \(g, h \in \cfs{\R}\) and \(a \in \R\).
  Now
  \begin{align*}
    \T(cg + h) & = \int_a^b (cg + h)(t) \; dt                  &  & \text{(by \cref{2.1.8})}  \\
               & = \int_a^b cg(t) + h(t) \; dt                 &  & \text{(by \cref{1.2.10})} \\
               & = c \int_a^b g(t) \; dt + \int_a^b h(t) \; dt                                \\
               & = c \T(g) + \T(h).                            &  & \text{(by \cref{2.1.8})}
  \end{align*}
  So by \cref{2.1.2}(b) \(\T\) is linear.
\end{proof}

\begin{eg}\label{2.1.9}
  For vector spaces \(\V\) and \(\W\) over \(\F\), we define the \textbf{identity transformation} \(\IT[\V] : \V \to \V\) by \(\IT[\V](x) = x\) for all \(x \in \V\) and the \textbf{zero transformation} \(\zT : \V \to \W\) by \(\zT(x) = \zv_{\W}\) for all \(x \in \V\).
  It is clear that both of these transformations are linear.
  We often write \(\IT\) instead of \(\IT[\V]\).
\end{eg}

\begin{proof}[\pf{2.1.9}]
  For all \(x, y \in \V\) and \(c \in \F\), we have
  \begin{align*}
    \IT[\V](cx + y) & = cx + y                    &  & \text{(by \cref{2.1.9})} \\
                    & = c \IT[\V](x) + \IT[\V](y) &  & \text{(by \cref{2.1.9})}
  \end{align*}
  and
  \begin{align*}
    \zT(cx + y) & = \zv_{\W}              &  & \text{(by \cref{2.1.9})}  \\
                & = c \zv_{\W}            &  & \text{(by \cref{1.2}(c))} \\
                & = c \zv_{\W} + \zv_{\W} &  & \text{(by \ref{vs3})}     \\
                & = c\T(x) + \T(y).       &  & \text{(by \cref{2.1.9})}
  \end{align*}
  Thus by \cref{2.1.2}(b) \(\IT[\V], \zT\) are linear.
\end{proof}

\begin{defn}\label{2.1.10}
  Let \(\V\) and \(\W\) be vector spaces over \(\F\), and let \(\T : \V \to \W\) be linear.
  We define the \textbf{null space} (or \textbf{kernel}) \(\ns{\T}\) of \(\T\) to be the set of all vectors \(x\) in \(\V\) such that \(\T(x) = \zv_{\W}\);
  that is, \(\ns{\T} = \set{x \in \V : \T(x) = \zv_{\W}}\).

  We define the \textbf{range} (or \textbf{image}) \(\rg{\T}\) of \(\T\) to be the subset of \(\W\) consisting of all images (under \(\T\)) of vectors in \(\V\);
  that is, \(\rg{\T} = \set{\T(x) : x \in V}\).
\end{defn}

\begin{eg}\label{2.1.11}
  Let \(\V\) and \(\W\) be vector spaces over \(\F\), and let \(\IT : \V \to \V\) and \(\zT : \V \to \W\) be the identity and zero transformations, respectively.
  Then \(\ns{\IT} = \set{\zv_{\V}}\), \(\rg{\IT} = \V\), \(\ns{\zT} = \V\), and \(\rg{\zT} = \set{\zv_{\W}}\).
\end{eg}

\begin{thm}\label{2.1}
  Let \(\V\) and \(\W\) be vector spaces over \(\F\) and \(\T : \V \to \W\) be linear.
  Then \(\ns{\T}\) and \(\rg{\T}\) are subspaces of \(\V\) and \(\W\) over \(\F\), respectively.
\end{thm}

\begin{proof}[\pf{2.1}]
  To clarify the notation, we use the symbols \(\zv_{\V}\) and \(\zv_{\W}\) to denote the zero vectors of \(\V\) and \(\W\), respectively.

  Since \(\T(\zv_{\V}) = \zv_{\W}\), we have that \(\zv_{\V} \in \ns{\T}\).
  Let \(x, y \in \ns{\T}\) and \(c \in \F\).
  Then \(\T(x + y) = \T(x) + \T(y) = \zv_{\W} +\zv_{\W} = \zv_{\W}\), and \(\T(cx) = c \T(x) = c \zv_{\W} = \zv_{\W}\).
  Hence \(x + y \in \ns{\T}\) and \(cx \in \ns{\T}\), so that \(\ns{\T}\) is a subspace of \(\V\) over \(\F\) (see \cref{1.3}).

  Because \(\T(\zv_{\V}) = \zv_{\W}\), we have that \(\zv_{\W} \in \rg{\T}\).
  Now let \(x, y \in \rg{\T}\) and \(c \in \F\).
  Then there exist \(v\) and \(w\) in \(\V\) such that \(\T(v) = x\) and \(\T(w) = y\).
  So \(\T(v + w) = \T(v) + \T(w) = x + y\), and \(\T(cv) = c \T(v) = cx\).
  Thus \(x + y \in \rg{\T}\) and \(cx \in \rg{\T}\), so \(\rg{\T}\) is a subspace of \(\W\) over \(\F\) (see \cref{1.3}).
\end{proof}

\begin{thm}\label{2.2}
  Let \(\V\) and \(\W\) be vector spaces over \(\F\), and let \(\T : \V \to \W\) be linear.
  If \(\beta = \set{\seq{v}{1,2,,n}}\) is a basis for \(\V\) over \(\F\), then
  \[
    \rg{\T} = \spn{\T(\beta)} = \spn{\set{\T(v_1), \T(v_2), \dots, \T(v_n)}}.
  \]
\end{thm}

\begin{proof}[\pf{2.2}]
  Clearly \(\T(v_i) \in \rg{\T}\) for each \(i\).
  Because \(\rg{\T}\) is a subspace, \(\rg{\T}\) contains \(\spn{\set{\T(v_1), \T(v_2), \dots, \T(v_n)}} = \spn{\T(\beta)}\) by \cref{1.5}.

  Now suppose that \(w \in \rg{\T}\).
  Then \(w = \T(v)\) for some \(v \in \V\).
  Because \(\beta\) is a basis for \(\V\) over \(\F\), we have
  \[
    v = \sum_{i = 1}^n a_i v_i \quad \text{for some } \seq{a}{1,2,,n} \in \F.
  \]
  Since \(\T\) is linear, it follows that
  \[
    w = \T(v) = \sum_{i = 1}^n a_i \T(v_i) \in \spn{T(\beta)}.
  \]
  So \(\rg{\T}\) is contained in \(\spn{\T(\beta)}\).
\end{proof}

\begin{note}
  It should be noted that \cref{2.2} is true if \(\beta\) is infinite.
  (See \cref{ex:2.1.33}.)
\end{note}

\begin{defn}\label{2.1.12}
  Let \(\V\) and \(\W\) be vector spaces over \(\F\), and let \(\T : \V \to \W\) be linear.
  If \(\ns{\T}\) and \(\rg{\T}\) are finite-dimensional, then we define the \textbf{nullity} of \(\T\), denoted \(\nt{\T}\), and the \textbf{rank} of \(\T\), denoted \(\rk{\T}\), to be the dimensions of \(\ns{\T}\) and \(\rg{\T}\), respectively.
\end{defn}

\begin{note}
  Reflecting on the action of a linear transformation, we see intuitively that the larger the nullity, the smaller the rank.
  In other words, the more vectors that are carried into \(\zv\), the smaller the range.
  The same heuristic reasoning tells us that the larger the rank, the smaller the nullity.
\end{note}

\begin{thm}[Dimension Theorem]\label{2.3}
  Let \(\V\) and \(\W\) be vector spaces over \(\F\), and let \(\T : \V \to \W\) be linear.
  If \(\V\) is finite-dimensional, then
  \[
    \nt{\T} + \rk{\T} = \dim(\V).
  \]
\end{thm}

\begin{proof}[\pf{2.3}]
  Suppose that \(\dim(\V) = n\), \(\dim(\ns{\T}) = k\), and \(\set{\seq{v}{1,2,,k}}\) is a basis for \(\ns{\T}\) over \(\F\).
  By the \cref{1.6.19} we may extend \(\set{\seq{v}{1,2,,k}}\) to a basis \(\beta = \set{\seq{v}{1,2,,n}}\) for \(\V\) over \(\F\).
  We claim that \(S = \set{\T(v_{k + 1}), \T(v_{k + 2}), \dots, \T(v_n)}\) is a basis for \(\rg{\T}\) over \(\F\).

  First we prove that \(S\) generates \(\rg{\T}\).
  Using \cref{2.2} and the fact that \(\T(v_i) = \zv\) for \(1 \leq i \leq k\), we have
  \begin{align*}
    \rg{\T} & = \spn{\set{\T(v_1), \T(v_2), \dots, \T(v_n)}}             \\
            & = \spn{\set{\T(v_{k + 1}), \T(v_{k + 2}), \dots, \T(v_n)}} \\
            & = \spn{S}.
  \end{align*}
  Now we prove that \(S\) is linearly independent.
  Suppose that
  \[
    \sum_{i = k + 1}^n b_i \T(v_i) = \zv \quad \text{for } \seq{b}{k + 1,k + 2,,n} \in \F.
  \]
  Using the fact that \(\T\) is linear, we have
  \[
    \T\pa{\sum_{i = k + 1}^n b_i v_i} = \zv.
  \]
  So
  \[
    \sum_{i = k + 1}^n b_i v_i \in \ns{\T}.
  \]
  Hence there exist \(\seq{c}{1,2,,k} \in \F\) such that
  \[
    \sum_{i = k + 1}^n b_i v_i = \sum_{i = 1}^k c_i v_i \quad \text{or} \quad \sum_{i = 1}^k (-c_i) v_i + \sum_{i = k + 1}^n b_i v_i = \zv.
  \]
  Since \(\beta\) is a basis for \(\V\) over \(\F\), we have \(b_i = 0\) for all \(i\).
  Hence \(S\) is linearly independent.
  Notice that this argument also shows that \(\T(v_{k + 1}), \T(v_{k + 2}), \dots, \T(v_n)\) are distinct;
  therefore \(\rk{\T} = n - k\).
\end{proof}

\begin{thm}\label{2.4}
  Let \(\V\) and \(\W\) be vector spaces over \(\F\), and let \(\T : \V \to \W\) be linear.
  Then \(\T\) is one-to-one if and only if \(\ns{\T} = \set{\zv_{\V}}\).
\end{thm}

\begin{proof}[\pf{2.4}]
  Suppose that \(\T\) is one-to-one and \(x \in \ns{\T}\).
  Then \(\T(x) = \zv_{\W} = \T(\zv_{\V})\).
  Since \(\T\) is one-to-one, we have \(x = \zv_{\V}\).
  Hence \(\ns{\T} = \set{\zv_{\V}}\).

  Now assume that \(\ns{\T} = \set{\zv_{\V}}\), and suppose that \(\T(x) = \T(y)\).
  Then \(\zv_{\W} = \T(x) - \T(y) = \T(x - y)\) by \cref{2.1.2}(c).
  Therefore \(x - y \in \ns{\T} = \set{\zv_{\V}}\).
  So \(x - y = \zv_{\V}\), or \(x = y\).
  This means that \(\T\) is one-to-one.
\end{proof}

\begin{thm}\label{2.5}
  Let \(\V\) and \(\W\) be vector spaces over \(\F\) of equal (finite) dimension, and let \(\T : \V \to \W\) be linear.
  Then the following are equivalent.
  \begin{enumerate}
    \item \(\T\) is one-to-one.
    \item \(\T\) is onto.
    \item \(\rk{\T} = \dim(\V)\)
  \end{enumerate}
\end{thm}

\begin{proof}[\pf{2.5}]
  From the dimension theorem (\cref{2.3}), we have
  \[
    \nt{\T} + \rk{\T} = \dim(\V).
  \]
  Now, with the use of \cref{2.4}, we have that \(\T\) is one-to-one if and only if \(\ns{\T} = \set{\zv_{\V}}\), if and only if \(\nt{\T} = 0\), if and only if \(\rk{\T} = \dim(\V)\), if and only if \(\rk{\T} = \dim(\W)\), and if and only if \(\dim(\rg{\T}) = \dim(\W)\).
  By \cref{1.11}, this equality is equivalent to \(\rg{\T} = \W\), the definition of \(\T\) being onto.
\end{proof}

\begin{note}
  We note that if \(\V\) is not finite-dimensional and \(\T : \V \to \V\) is linear, then it does \emph{not} follow that one-to-one and onto are equivalent.

  The linearity of \(\T\) in \cref{2.4} and \cref{2.5} is essential, for it is easy to construct examples of functions from \(\R\) into \(\R\) that are not one-to-one, but are onto, and vice versa.
\end{note}

\begin{thm}\label{2.6}
  Let \(\V\) and \(\W\) be vector spaces over \(\F\), and suppose that \(\set{\seq{v}{1,2,,n}}\) is a basis for \(\V\) over \(\F\).
  For \(\seq{w}{1,2,,n}\) in \(\W\), there exists exactly one linear transformation \(\T : \V \to \W\) such that \(\T(v_i) = w_i\) for \(i = 1, 2, \dots, n\).
\end{thm}

\begin{proof}[\pf{2.6}]
  Let \(x \in \V\).
  Then
  \[
    x = \sum_{i = 1}^n a_i v_i
  \]
  where \(\seq{a}{1,2,,n}\) are unique scalars.
  Define
  \[
    \T : \V \to \W \quad \text{by} \quad \T(x) = \sum_{i = 1}^n a_i w_i.
  \]
  \begin{enumerate}
    \item \(\T\) is linear:
          Suppose that \(u, v \in \V\) and \(d \in \F\).
          Then we may write
          \[
            u = \sum_{i = 1}^n b_i v_i \quad \text{and} \quad v = \sum_{i = 1}^n c_i v_i
          \]
          for some scalars \(\seq{b}{1,2,,n}, \seq{c}{1,2,,n}\).
          Thus
          \[
            du + v = \sum_{i = 1}^n (db_i + c_i) v_i.
          \]
          So
          \[
            \T(du + v) = \sum_{i = 1}^n (db_i + c_i) w_i = d \sum_{i = 1}^n b_i w_i + \sum_{i = 1}^n c_i w_i = d \T(u) + \T(v).
          \]
    \item Clearly
          \[
            \T(v_i) = w_i \quad \text{for } i = 1, 2, \dots, n.
          \]
    \item \(\T\) is unique:
          Suppose that \(\U : \V \to \W\) is linear and \(\U(v_i) = w_i\) for \(i = 1, 2, \dots, n\).
          Then for \(x \in \V\) with
          \[
            x = \sum_{i = 1}^n a_i v_i,
          \]
          we have
          \[
            \U(x) = \sum_{i = 1}^n a_i \U(v_i) = \sum_{i = 1}^n a_i w_i = \T(x).
          \]
          Hence \(\U = \T\).
  \end{enumerate}
\end{proof}

\begin{cor}\label{2.1.13}
  Let \(\V\) and \(\W\) be vector spaces over \(\F\), and suppose that \(\V\) has a finite basis \(\set{\seq{v}{1,2,,n}}\) over \(\F\).
  If \(\U, \T : \V \to \W\) are linear and \(\U(v_i) = \T(v_i)\) for \(i = 1, 2, \dots, n\), then \(\U = \T\).
\end{cor}

\begin{proof}[\pf{2.1.13}]
  Since \(\U(v_i) = \T(v_i)\) for all \(i = 1, 2, \dots, n\), by \cref{2.6} we know that \(\U = \T\).
\end{proof}

\exercisesection

\setcounter{ex}{5}
\begin{ex}\label{ex:2.1.6}
  Define \(\T : \ms{n}{n}{\F} \to \F\) by \(\T(A) = \tr[A]\).
  Prove that \(\T\) is a linear transformation, and find bases for both \(\ns{\T}\) and \(\rg{\T}\) over \(\F\).
  Then compute the nullity and rank of \(\T\), and verify the dimension theorem.
  Finally, use the appropriate theorems in \cref{sec:2.1} to determine whether \(\T\) is one-to-one or onto.
\end{ex}

\begin{proof}[\pf{ex:2.1.6}]
  Let \(A, B \in \ms{n}{n}{\F}\) and let \(c \in \F\).
  First we show that \(\T\) is linear.
  Since
  \begin{align*}
    \T(cA + B) & = \tr[cA + B]                                       &  & \text{(by \cref{ex:2.1.6})}  \\
               & = \sum_{i = 1}^n \pa{cA + B}_{i i}                  &  & \text{(by \cref{1.3.9})}     \\
               & = \sum_{i = 1}^n \pa{cA_{i i} + B_{i i}}            &  & \text{(by \cref{1.2.9})}     \\
               & = c \sum_{i = 1}^n A_{i i} + \sum_{i = 1}^n B_{i i} &  & (A_{i i}, B_{i i}, c \in \F) \\
               & = c \tr[A] + \tr[B]                                 &  & \text{(by \cref{1.3.9})}     \\
               & = c \T(A) + \T(B),                                  &  & \text{(by \cref{ex:2.1.6})}
  \end{align*}
  by \cref{2.1.2}(b) we know that \(\T\) is linear.

  Next we find a basis for \(\ns{\T}\) over \(\F\).
  Let \(\W\) and \(\beta\) be the sets defined in \cref{ex:1.6.15}.
  From \cref{ex:1.6.15} we see that \(\W = \ns{\T}\) and \(\beta\) is a basis for \(\ns{\T}\) over \(\F\).
  Thus we have \(\dim(\ns{\T}) = \nt{\T} = n^2 - 1\).

  Next we find a basis for \(\rg{\T}\) over \(\F\).
  Since \(\tr[A] \in \F\) for all \(A \in \ms{n}{n}{\F}\), we know that \(\tr[\ms{n}{n}{\F}] \subseteq \F\).
  Since
  \[
    \forall c \in \F, \tr\begin{pmatrix}
      c      & 0      & \cdots & 0      \\
      0      & 0      & \cdots & 0      \\
      \vdots & \vdots & \ddots & \vdots \\
      0      & 0      & \cdots & 0
    \end{pmatrix} = c,
  \]
  we know that \(\F \subseteq \tr[\ms{n}{n}{\F}]\).
  Thus we have \(\rg{\T} = \T(\ms{n}{n}{\F}) = \tr[\ms{n}{n}{\F}] = \F\), and \(\dim(\rg{\T}) = \rk{\T} = 1\).

  From the proofs above we see that
  \[
    \dim(\ms{n}{n}{\F}) = n^2 = (n^2 - 1) + 1 = \nt{\T} + \rk{\T},
  \]
  thus the dimension theorem (\cref{2.3}) holds.
  Since \(\ns{\T} \neq \set{\zm}\), by \cref{2.4} we know that \(\T\) is not one-to-one.
  Since \(\rg{\T} = \F\), we know that \(\T\) is onto.
\end{proof}

\setcounter{ex}{12}
\begin{ex}\label{ex:2.1.13}
  Let \(\V\) and \(\W\) be vector spaces over \(\F\), let \(\T : \V \to \W\) be linear, and let \(\set{\seq{w}{1,2,,k}}\) be a linearly independent subset of \(\rg{\T}\).
  Prove that if \(S = \set{\seq{v}{1,2,,k}}\) is chosen so that \(\T(v_i) = w_i\) for \(i = 1, 2, \dots, k\), then \(S\) is linearly independent.
\end{ex}

\begin{proof}[\pf{ex:2.1.13}]
  Let \(\seq{a}{1,2,,k} \in \F\).
  Since
  \begin{align*}
             & \sum_{i = 1}^k a_i v_i = \zv_{\V}                                         \\
    \implies & \T\pa{\sum_{i = 1}^k a_i v_i} = \zv_{\W} &  & \text{(by \cref{2.1.2}(a))} \\
    \implies & \sum_{i = 1}^k a_i \T(v_i) = \zv_{\W}    &  & \text{(by \cref{2.1.2}(d))} \\
    \implies & \sum_{i = 1}^k a_i w_i = \zv_{\W}                                         \\
    \implies & \seq[=]{a}{1,2,,k} = 0,                  &  & \text{(by \cref{1.5.3})}
  \end{align*}
  by \cref{1.5.3} we know that \(\set{\seq{v}{1,2,,k}}\) is linearly independent.
\end{proof}

\begin{ex}\label{ex:2.1.14}
  Let \(\V\) and \(\W\) be vector spaces over \(\F\) and \(\T : \V \to \W\) be linear.
  \begin{enumerate}
    \item Prove that \(\T\) is one-to-one if and only if \(\T\) carries linearly independent subsets of \(\V\) onto linearly independent subsets of \(\W\).
    \item Suppose that \(\T\) is one-to-one and that \(S\) is a subset of \(\V\).
          Prove that \(S\) is linearly independent if and only if \(\T(S)\) is linearly independent.
    \item Suppose \(\beta = \set{\seq{v}{1,2,,n}}\) is a basis for \(\V\) over \(\F\) and \(\T\) is one-to-one and onto.
          Prove that \(\T(\beta) = \set{\T(v_1), \T(v_2), \dots, \T(v_n)}\) is a basis for \(\W\) over \(\F\).
  \end{enumerate}
\end{ex}

\begin{proof}[\pf{ex:2.1.14}(a)]
  First suppose that \(\T\) is one-to-one.
  Let \(S\) be a linearly independent subset of \(\V\).
  Note that \(\V\) can be infinite-dimensional and thus \(S\) can be infinite.
  Since
  \begin{align*}
             & \begin{dcases}
      \forall \seq{w}{1,2,,k} \in \T(S) \\
      \forall \seq{a}{1,2,,k} \in \F
    \end{dcases}, \sum_{i = 1}^k a_i w_i = \zv_{\W}                                    \\
    \implies & \exists \seq{v}{1,2,,k} \in S :                                                                  \\
             & \begin{dcases}
      \forall i \in \set{1, 2, \dots, k}, \T(v_i) = w_i \\
      \sum_{i = 1}^k a_i w_i = \sum_{i = 1}^k a_i \T(v_i) = \T\pa{\sum_{i = 1}^k a_i v_i} = \zv_{\W} = \T(\zv_{\V})
    \end{dcases}                                    &  & \text{(by \cref{2.1.2})}      \\
    \implies & \sum_{i = 1}^k a_i v_i = \zv_{\V}                             &  & \text{(\(\T\) is one-to-one)} \\
    \implies & \seq[=]{a}{1,2,,k} = 0,                                       &  & \text{(by \cref{1.5.3})}
  \end{align*}
  by \cref{1.5.3} we know that \(\T(S)\) is linearly independent.
  Since \(S\) is arbitrary, we conclude that \(\T\) carries linearly independent subsets of \(\V\) onto linearly independent subsets of \(\W\).

  Now suppose that \(\T\) carries linearly independent subsets of \(\V\) onto linearly independent subsets of \(\W\).
  Since
  \begin{align*}
             & \forall x, y \in \V, x \neq y                                                    \\
    \implies & x - y \neq \zv_{\V}                                                              \\
    \implies & \set{x - y} \text{ is linearly independent}     &  & \text{(by \cref{1.5.4}(b))} \\
    \implies & \set{\T(x - y)} \text{ is linearly independent} &  & \text{(by hypothesis)}      \\
    \implies & \T(x - y) \neq \zv_{\W}                         &  & \text{(by \cref{1.5.2})}    \\
    \implies & \T(x) - \T(y) \neq \zv_{\W}                     &  & \text{(by \cref{2.1.2}(c))} \\
    \implies & \T(x) \neq \T(y),
  \end{align*}
  we know that \(\T\) is one-to-one.
  From all proofs above we conclude that \(\T\) is one-to-one if and only if \(\T\) carries linearly independent subsets of \(\V\) onto linearly independent subsets of \(\W\).
\end{proof}

\begin{proof}[\pf{ex:2.1.14}(b)]
  By \cref{ex:2.1.14}(a) and \cref{ex:2.1.13} we are done.
\end{proof}

\begin{proof}[\pf{ex:2.1.14}(c)]
  Since \(\T\) is one-to-one, by \cref{ex:2.1.14}(b) we know that \(\T(\beta)\) is linearly independent.
  Since \(\T\) is onto, by \cref{2.2} we know that \(\spn{\T(\beta)} = \rg{\T} = \W\).
  Thus by \cref{1.6.1} \(\T(\beta)\) is a basis for \(\W\) over \(\F\).
\end{proof}

\begin{ex}\label{ex:2.1.15}
  Define
  \[
    \T : \ps{\R} \to \ps{\R} \quad \text{by} \quad \T(f) = \int_0^x f(t) \; dt.
  \]
  Prove that \(\T\) is linear and one-to-one, but not onto.
\end{ex}

\begin{proof}[\pf{ex:2.1.15}]
  First we show that \(\T\) is linear.
  Let \(f, g \in \ps{\R}\) and let \(c \in \R\).
  Since
  \begin{align*}
    \T(cf + g) & = \int_0^x (cf + g)(t) \; dt                  &  & \text{(by \cref{ex:2.1.15})} \\
               & = \int_0^x cf(t) + g(t) \; dt                 &  & \text{(by \cref{1.2.10})}    \\
               & = c \int_0^x f(t) \; dt + \int_0^x g(t) \; dt                                   \\
               & = c \T(f) + \T(g),
  \end{align*}
  by \cref{2.1.2}(b) we know that \(\T\) is linear.

  Next we show that \(\T\) is one-to-one.
  Let \(\zv : \R \to \R\) denote the zero function.
  Since
  \[
    \forall f \in \ps{\R}, \int_0^x f(t) \; dt = \zv \implies f = \zv \implies \ns{\T} = \set{\zv},
  \]
  by \cref{2.4} we know that \(\T\) is one-to-one.

  Now we show that \(\T\) is not onto.
  Let \(c \in \F \setminus \set{0}\).
  Since \(c \in \ps{\R}\) and no polynomial function has indefinite integral equals to \(c\), we know that \(\T\) is not onto.
\end{proof}

\begin{ex}\label{ex:2.1.16}
  Let \(\T : \ps{\R} \to \ps{\R}\) be defined by \(\T(f) = f'\).
  Recall that \(\T\) is linear (\cref{2.1.7}).
  Prove that \(\T\) is onto, but not one-to-one.
\end{ex}

\begin{proof}[\pf{ex:2.1.16}]
  First we show that \(\T\) is onto.
  Let \(f \in \ps{\R}\).
  Since
  \begin{align*}
             & f \in \ps{\R}                                                          \\
    \implies & \exists \seq{a}{0,1,,n} \in \R : f(x) = a_0 + a_1 x + \cdots + a_n x^n \\
    \implies & \begin{dcases}
      x \mapsto a_0 x + \frac{a_1}{2} x^2 + \cdots + \frac{a_n}{n + 1} x^{n + 1} \in \ps{\R} \\
      (x \mapsto a_0 x + \frac{a_1}{2} x^2 + \cdots + \frac{a_n}{n + 1} x^{n + 1})' = f
    \end{dcases}
  \end{align*}
  and \(f\) is arbitrary, we know that \(\T\) is onto.

  Now we show that \(\T\) is not one-to-one.
  Observe that
  \begin{align*}
     & x \mapsto x \in \ps{\R};          \\
     & x \mapsto x + 1 \in \ps{\R};      \\
     & (x \mapsto x)' = 1;               \\
     & (x \mapsto x + 1)' = 1;           \\
     & x \mapsto x \neq x \mapsto x + 1.
  \end{align*}
  Thus \(\T\) is not one-to-one.
\end{proof}

\begin{ex}\label{ex:2.1.17}
  Let \(\V\) and \(\W\) be finite-dimensional vector spaces over \(\F\) and \(\T : \V \to \W\) be linear.
  \begin{enumerate}
    \item Prove that if \(\dim(\V) < \dim(\W)\), then \(\T\) cannot be onto.
    \item Prove that if \(\dim(\V) > \dim(\W)\), then \(\T\) cannot be one-to-one.
  \end{enumerate}
\end{ex}

\begin{proof}[\pf{ex:2.1.17}(a)]
  Suppose that \(\dim(\V) < \dim(\W)\).
  Suppose for sake of contradiction that \(\T\) is onto.
  Let \(\beta_{\W} = \set{\seq{w}{1,2,,n}}\) be a basis for \(\W\) over \(\F\).
  Since \(\T\) is onto, there exists a set \(\beta_{\V} = \set{\seq{v}{1,2,,n}}\) such that \(\T(v_i) = w_i\) for all \(i \in \set{1, 2, \dots, n}\).
  But by \cref{ex:2.1.13} we know that \(\beta_{\V}\) is linearly independent, by \cref{1.6.8} this means \(\dim(\V) \geq \dim(\W)\), a contradiction.
  Thus \(\T\) cannot be onto.
\end{proof}

\begin{proof}[\pf{ex:2.1.17}(b)]
  Suppose that \(\dim(\V) > \dim(\W)\).
  Suppose for sake of contradiction that \(\T\) is one-to-one.
  Since \(\T\) is one-to-one, by \cref{2.4} we know that \(\ns{\T} = \set{\zv_{\V}}\).
  Thus we have
  \begin{align*}
    \dim(\V) & = \rk{\T} + \nt{\T} &  & \text{(by \cref{2.3})}    \\
             & = \rk{\T} + 0       &  & \text{(by \cref{1.6.9})}  \\
             & = \rk{\T}                                          \\
             & = \dim(\rg{\T})     &  & \text{(by \cref{2.1.12})} \\
             & \leq \dim(\W).      &  & \text{(by \cref{1.11})}
  \end{align*}
  But this contradict to the fact that \(\dim(\V) > \dim(\W)\).
  Thus \(\T\) cannot be one-to-one.
\end{proof}

\setcounter{ex}{19}
\begin{ex}\label{ex:2.1.20}
  Let \(\V\) and \(\W\) be vector spaces over \(\F\) with subspaces \(\V_1\) and \(\W_1\) over \(\F\), respectively.
  If \(\T : \V \to \W\) is linear, prove that \(\T(\V_1)\) is a subspace of \(\W\) over \(\F\) and that \(\set{x \in \V : \T(x) \in \W_1}\) is a subspace of \(\V\) over \(\F\).
\end{ex}

\begin{proof}[\pf{ex:2.1.20}]
  First we show that \(\T(\V_1)\) is a subspace of \(\W\) over \(\F\).
  Let \(w_1, w_2 \in \T(\V_1)\) and let \(c \in \F\).
  Since
  \begin{align*}
             & \zv_{\V} \in \V_1                    &  & \text{(by \cref{1.3}(a))}   \\
    \implies & \T(\zv_{\V}) = \zv_{\W} \in \T(\V_1) &  & \text{(by \cref{2.1.2}(a))}
  \end{align*}
  and
  \begin{align*}
             & w_1, w_2 \in \T(\V_1)                                                                    \\
    \implies & \exists v_1, v_2 \in \V_1 : \begin{dcases}
      \T(v_1) = w_1 \\
      \T(v_2) = w_2
    \end{dcases}                                   \\
    \implies & \begin{dcases}
      v_1 + v_2 \in \V_1 \\
      c v_1 \in \V_1
    \end{dcases}                             &  & \text{(by \cref{1.3}(b)(c))} \\
    \implies & \begin{dcases}
      w_1 + w_2 = \T(v_1) + \T(v_2) = \T(v_1 + v_2) \in \T(\V_1) \\
      c w_1 = c \T(v_1) = \T(c v_1) \in \T(\V_1)
    \end{dcases},                            &  & \text{(by \cref{2.1.1})}
  \end{align*}
  by \cref{1.3} we know that \(\T(\V_1)\) is a subspace of \(\W\) over \(\F\).

  Now we show that \(\V' = \set{x \in \V : \T(x) \in \W_1}\) is a subspace of \(\V\) over \(\F\).
  Let \(v_1, v_2 \in \V'\) and let \(c \in \F\).
  Since
  \begin{align*}
             & \zv_{\V} \in \V                  &  & \text{(by \cref{1.3}(a))}   \\
    \implies & \T(\zv_{\V}) = \zv_{\W} \in \W   &  & \text{(by \cref{2.1.2}(a))} \\
    \implies & \T(\zv_{\V}) = \zv_{\W} \in \W_1 &  & \text{(by \cref{1.3}(a))}   \\
    \implies & \zv_{\V} \in \V'
  \end{align*}
  and
  \begin{align*}
             & v_1, v_2 \in \V'                                              \\
    \implies & \T(v_1), \T(v_2) \in \W_1                                     \\
    \implies & \begin{dcases}
      \T(v_1) + \T(v_2) \in \W_1 \\
      c \T(v_1) \in \W_1
    \end{dcases}  &  & \text{(by \cref{1.3}(b)(c))} \\
    \implies & \begin{dcases}
      \T(v_1) + \T(v_2) = \T(v_1 + v_2) \in \W_1 \\
      c \T(v_1) = \T(c v_1) \in \W_1
    \end{dcases}  &  & \text{(by \cref{2.1.1})}     \\
    \implies & \begin{dcases}
      v_1 + v_2 \in \V' \\
      c v_1 \in \V'
    \end{dcases},
  \end{align*}
  by \cref{1.3} we know that \(\V'\) is a subspace of \(\V\) over \(\F\).
\end{proof}

\begin{ex}\label{ex:2.1.21}
  Let \(\V\) be the vector space of sequences over \(\F\) described in \cref{1.2.13}.
  Define the functions \(\T, \U : \V \to \V\) by
  \[
    \T(\seq{a}{1,2,}) = \tuple{a}{2,3,} \quad \text{and} \quad \U(\seq{a}{1,2,}) = (0, \seq{a}{1,2,}).
  \]
  \(\T\) and \(\U\) are called the \textbf{left shift} and \textbf{right shift} operators on \(\V\), respectively.
  \begin{enumerate}
    \item Prove that \(\T\) and \(\U\) are linear.
    \item Prove that \(\T\) is onto, but not one-to-one.
    \item Prove that \(\U\) is one-to-one, but not onto.
  \end{enumerate}
\end{ex}

\begin{proof}[\pf{ex:2.1.21}(a)]
  Let \(\set{a_n}, \set{b_n} \in \V\) and let \(t \in \F\).
  Since
  \begin{align*}
    \T(c \set{a_n} + \set{b_n}) & = \T(\set{c a_n + b_n})                   &  & \text{(by \cref{1.2.13})}    \\
                                & = \T(c a_1 + b_1, c a_2 + b_2, \dots)                                       \\
                                & = (c a_2 + b_2, c a_3 + b_3, \dots)       &  & \text{(by \cref{ex:2.1.21})} \\
                                & = c \tuple{a}{2,3,} + \tuple{b}{2,3,}     &  & \text{(by \cref{1.2.13})}    \\
                                & = c \T(\seq{a}{1,2,}) + \T(\seq{b}{1,2,}) &  & \text{(by \cref{ex:2.1.21})} \\
                                & = c \T(\set{a_n}) + \T(\set{b_n}),
  \end{align*}
  by \cref{2.1.2}(b) we know that \(\T\) is linear.
  Since
  \begin{align*}
    \U(c \set{a_n} + \set{b_n}) & = \U(\set{c a_n + b_n})                     &  & \text{(by \cref{1.2.13})}    \\
                                & = \U(c a_1 + b_1, c a_2 + b_2, \dots)                                         \\
                                & = (0, c a_1 + b_1, c a_2 + b_2, \dots)      &  & \text{(by \cref{ex:2.1.21})} \\
                                & = c (0, \seq{a}{1,2,}) + (0, \seq{b}{1,2,}) &  & \text{(by \cref{1.2.13})}    \\
                                & = c \U(\seq{a}{1,2,}) + \U(\seq{b}{1,2,})   &  & \text{(by \cref{ex:2.1.21})} \\
                                & = c \U(\set{a_n}) + \U(\set{b_n}),
  \end{align*}
  by \cref{2.1.2}(b) we know that \(\U\) is linear.
\end{proof}

\begin{proof}[\pf{ex:2.1.21}(b)]
  Since
  \[
    \forall \set{a_n} \in \V, \T(0, \seq{a}{1,2,}) = \tuple{a}{1,2,},
  \]
  we know that \(\T\) is onto.
  Since
  \[
    \forall \set{a_n} \in \V, \T(0, \seq{a}{1,2,}) = \tuple{a}{1,2,} = \T(1, \seq{a}{1,2,}),
  \]
  we know that \(\T\) is not one-to-one.
\end{proof}

\begin{proof}[\pf{ex:2.1.21}(c)]
  Since
  \begin{align*}
             & \forall \set{a_n}, \set{b_n} \in \V, \set{a_n} \neq \set{b_n} \\
    \implies & \tuple{a}{1,2,} \neq \tuple{b}{1,2,}                          \\
    \implies & (0, \seq{a}{1,2,}) \neq (0, \seq{b}{1,2,})                    \\
    \implies & \U(\seq{a}{1,2,}) \neq \U(\seq{b}{1,2,}),
  \end{align*}
  we know that \(\U\) is one-to-one.
  Since
  \[
    \forall \set{a_n} \in \V, (1, \seq{a}{1,2,}) \notin \T(\V),
  \]
  we know that \(\U\) is not onto.
\end{proof}

\begin{ex}\label{ex:2.1.22}
  Let \(\T: \vs{F}^n \to \F\) be linear.
  Show that
  \[
    \forall x = \tuple{x}{1,2,,n} \in \vs{F}^n, \exists \seq{a}{1,2,,n} \in \F : \T(\seq{x}{1,2,,n}) = \sum_{i = 1}^n a_i x_i.
  \]
  State and prove an analogous result for \(\T : \vs{F}^n \to \vs{F}^m\).
\end{ex}

\begin{proof}[\pf{ex:2.1.22}]
  For all \(i \in \set{1, \dots, n}\), defined \(e_i \in \vs{F}^n\) as in \cref{1.6.3}.
  We claim that if \(\T : \vs{F}^n \to \vs{F}^m\) is linear, then
  \[
    \forall x \in \vs{F}^n, \exists \seq{a}{1,2,,n} \in \vs{F}^m : \T(x) = \sum_{i = 1}^n x_i a_i.
  \]
  Since
  \begin{align*}
             & \vs{F}^n = \spn{\seq{e}{1,2,,n}}                                       &  & \text{(by \cref{1.6.3})}    \\
    \implies & \forall x \in \vs{F}^n, x = \tuple{x}{1,2,,n} = \sum_{i = 1}^n x_i e_i &  & \text{(by \cref{1.4.3})}    \\
    \implies & \forall x \in \vs{F}^n, \T(x) = \sum_{i = 1}^n x_i \T(e_i),            &  & \text{(by \cref{2.1.2}(d))}
  \end{align*}
  by setting \(\T(e_i) = a_i\) for all \(i \in \set{1, 2, \dots, n}\) we see that our claim is true.
  In particular, when \(m = 1\) we see that
  \[
    \forall x \in \vs{F}^n, \exists \seq{a}{1,2,,n} \in \F : \T(x) = \sum_{i = 1}^n x_i a_i = \sum_{i = 1}^n a_i x_i.
  \]
\end{proof}

\begin{defn}\label{2.1.14}
  Let \(\V\) be a vector space over \(\F\) and \(\W_1\) and \(\W_2\) be subspaces of \(\V\) over \(\F\) such that \(V = \W_1 \oplus \W_2\).
  (See \cref{1.3.11}.)
  A function \(\T : \V \to \V\) is called a \textbf{projection on \(\W_1\) along \(\W_2\)} if, for \(x = x_1 + x_2\) with \(x_1 \in \W_1\) and \(x_2 \in \W_2\), we have \(\T(x) = x_1\).
\end{defn}

\setcounter{ex}{25}
\begin{ex}\label{ex:2.1.26}
  Using the notation in \cref{2.1.14}, assume that \(\T : \V \to \V\) is the projection on \(\W_1\) along \(\W_2\).
  \begin{enumerate}
    \item Prove that \(\T\) is linear and \(\W_1 = \set{x \in \V : \T(x) = x}\).
    \item Prove that \(\W_1 = \rg{\T}\) and \(\W_2 = \ns{\T}\).
    \item Describe \(\T\) if \(\W_1 = \V\).
    \item Describe \(\T\) if \(\W_1\) is the zero subspace.
  \end{enumerate}
\end{ex}

\begin{proof}[\pf{ex:2.1.26}(a)]
  First we show that \(\T\) is linear.
  Let \(x, y \in \V\) and let \(c \in \F\).
  Since
  \begin{align*}
             & \V = \W_1 \oplus \W_2                                                                                           \\
    \implies & \exists (x_1, x_2), (y_1, y_2) \in \W_1 \times \W_2 : \begin{dcases}
      x = x_1 + x_2 \\
      y = y_1 + y_2
    \end{dcases} &  & \text{(by \cref{1.3.11})} \\
    \implies & \T(cx + y) = \T(c (x_1 + x_2) + y_1 + y_2)                                                                      \\
             & = \T(c x_1 + y_1 + c x_2 + y_2) = c x_1 + y_1 = c \T(x) + \T(y),                 &  & \text{(by \cref{2.1.14})}
  \end{align*}
  by \cref{2.1.2}(b) we know that \(\T\) is linear.

  Now we show that \(\W_1 = \set{x \in \V : \T(x) = x}\).
  Since
  \begin{align*}
             & \zv \in \W_2                                &  & \text{(by \cref{1.3}(a))} \\
    \implies & \forall x \in \W_1, x = x + \zv             &  & \text{(by \ref{vs3})}     \\
    \implies & \forall x \in \W_1, \T(x) = \T(x + \zv) = x &  & \text{(by \cref{2.1.14})} \\
    \implies & \W_1 \subseteq \set{x \in \V : \T(x) = x}
  \end{align*}
  and
  \begin{align*}
             & \forall x \in \V, \T(x) = x                                               \\
    \implies & x = \T(x) \in \W_1                         &  & \text{(by \cref{2.1.14})} \\
    \implies & \set{x \in \V : \T(x) = x} \subseteq \W_1,
  \end{align*}
  we know that \(\W_1 = \set{x \in \V : \T(x) = x}\).
\end{proof}

\begin{proof}[\pf{ex:2.1.26}(b)]
  First we show that \(\W_1 = \rg{\T}\).
  Since
  \begin{align*}
             & \W_1 = \set{x \in \V : \T(x) = x} &  & \text{(by \cref{ex:2.1.26}(a))} \\
    \implies & \W_1 \subseteq \rg{\T}            &  & \text{(by \cref{2.1.10})}
  \end{align*}
  and
  \begin{align*}
             & \forall x \in \V, \T(x) \in \W_1 &  & \text{(by \cref{2.1.14})} \\
    \implies & \rg{\T} \subseteq \W_1,
  \end{align*}
  we know that \(\W_1 = \rg{\T}\).

  Now we show that \(\W_2 = \ns{\T}\).
  Since
  \begin{align*}
             & \zv \in \W_1                    &  & \text{(by \cref{1.3}(a))} \\
    \implies & \forall x \in \W_2, x = \zv + x &  & \text{(by \ref{vs3})}     \\
    \implies & \forall x \in \W_2, \T(x) = \zv &  & \text{(by \cref{2.1.14})} \\
    \implies & \W_2 \subseteq \ns{\T}          &  & \text{(by \cref{2.1.10})}
  \end{align*}
  and
  \begin{align*}
             & \ns{\T} \subseteq \V                                                           &  & \text{(by \cref{2.1.10})} \\
    \implies & \forall x \in \ns{\T}, \exists (x_1, x_2) \in \W_1 \times \W_2 : x = x_1 + x_2 &  & \text{(by \cref{1.3.11})} \\
    \implies & \forall x \in \ns{\T}, \exists (x_1, x_2) \in \W_1 \times \W_2 :                                              \\
             & \T(x) = \T(x_1 + x_2) = \zv = x_1                                              &  & \text{(by \cref{2.1.14})} \\
    \implies & \forall x \in \ns{\T}, \exists (x_1, x_2) \in \W_1 \times \W_2 :                                              \\
             & x = x_1 + x_2 = \zv + x_2 = x_2 \in \W_2                                       &  & \text{(by \ref{vs3})}     \\
    \implies & \ns{\T} \subseteq \W_2,
  \end{align*}
  we know that \(\W_2 = \ns{\T}\).
\end{proof}

\begin{proof}[\pf{ex:2.1.26}(c)]
  We have
  \begin{align*}
             & \begin{dcases}
      \V = \W_1 \oplus \W_2 \\
      \V = \W_1
    \end{dcases}                                                            \\
    \implies & \W_1 \cap \W_2 = \V \cap \W_2 = \W_2 = \set{\zv} &  & \text{(by \cref{1.3.11})}       \\
    \implies & \ns{\T} = \W_2 = \set{\zv}                       &  & \text{(by \cref{ex:2.1.26}(b))} \\
    \implies & \T \text{ is one-to-one}                         &  & \text{(by \cref{2.4})}          \\
    \implies & \T \text{ is onto}.                              &  & \text{(by \cref{2.5})}
  \end{align*}
\end{proof}

\begin{proof}[\pf{ex:2.1.26}(d)]
  We have
  \begin{align*}
             & \W_1 = \set{\zv}                                                            \\
    \implies & \V = \W_1 \oplus \W_2 = \W_2           &  & \text{(by \cref{1.3.11})}       \\
             & = \ns{\T}                              &  & \text{(by \cref{ex:2.1.26}(b))} \\
    \implies & \forall x \in \V, \T(x) = \zv          &  & \text{(by \cref{2.1.10})}       \\
    \implies & \T \text{ is the zero transformation}. &  & \text{(by \cref{2.1.9})}
  \end{align*}
\end{proof}

\begin{ex}\label{ex:2.1.27}
  Suppose that \(\W\) is a subspace of a finite-dimensional vector space \(\V\) over \(\F\).
  \begin{enumerate}
    \item Prove that there exists a subspace \(\W'\) and a function \(\T : \V \to \V\) such that \(\T\) is a projection on \(\W\) along \(\W'\).
    \item Give an example of a subspace \(\W\) of a vector space \(\V\) over \(\F\) such that there are two projections on \(\W\) along two (distinct) subspaces.
  \end{enumerate}
\end{ex}

\begin{proof}[\pf{ex:2.1.27}(a)]
  Let \(\beta_{\W} = \set{\seq{v}{1,2,,k}}\) be a basis for \(\V\) over \(\F\).
  By \cref{1.6.19} we can extend \(\beta_{\W}\) to \(\beta = \set{\seq{v}{1,2,,n}}\) such that \(\beta\) is a basis for \(\V\) over \(\F\).
  Let \(\beta_{\W'} = \beta \setminus \beta_{\W}\) and let \(\W' = \spn{\beta_{\W'}}\).
  By \cref{1.5} we know that \(\W'\) is a subspace of \(\V\) over \(\F\).

  First we claim that \(\W \cap \W' = \set{\zv}\).
  Let \(v \in \W \cap \W'\).
  Then we have
  \begin{align*}
             & v \in \W \cap \W'                                                                                                                \\
    \implies & \exists \seq{a}{1,2,,k,k + 1,,n} \in \F : v = \sum_{i = 1}^k a_i v_i = \sum_{i = k + 1}^n a_i v_i &  & \text{(by \cref{1.4.3})}  \\
    \implies & \exists \seq{a}{1,2,,k,k + 1,,n} \in \F :                                                                                        \\
             & \sum_{i = 1}^k a_i v_i + \sum_{i = k + 1}^n (-a_i) v_i = \zv                                      &  & \text{(by \cref{1.2.1})}  \\
    \implies & \seq[=]{a}{1,2,,n} = 0                                                                            &  & \text{(by \cref{1.5.3})}  \\
    \implies & v = \zv                                                                                           &  & \text{(by \cref{1.2}(a))} \\
    \implies & \W \cap \W' \subseteq \set{\zv}                                                                                                  \\
    \implies & \W \cap \W' = \set{\zv}.                                                                          &  & \text{(by \cref{1.3}(a))}
  \end{align*}

  Next we claim that \(\V = \W \oplus \W'\).
  Since
  \begin{align*}
             & \forall v \in \V, \exists \seq{a}{1,2,,n} \in \F : v = \sum_{i = 1}^n a_i v_i &  & \text{(by \cref{1.6.1})}        \\
             & = \sum_{i = 1}^k a_i v_i + \sum_{i = k + 1}^n a_i v_i                         &  & \text{(by \cref{1.2.1})}        \\
    \implies & \forall v \in \V, \exists (u, u') \in \W \times \W' : v = u + u'              &  & \text{(by \cref{1.4.3})}        \\
    \implies & \V \subseteq \W + \W'                                                         &  & \text{(by \cref{1.3.10})}       \\
    \implies & \V = \W + \W'                                                                 &  & \text{(by \cref{ex:1.3.23}(a))}
  \end{align*}
  and \(\W \cap \W' = \set{\zv}\), by \cref{1.3.11} we know that \(\V = \W \oplus \W'\).

  Since \(\beta\) is a basis for \(\V\) over \(\F\), by \cref{1.8} we know that
  \[
    \forall v \in \V, \exists \seq{a}{1,2,,n} \in \F : v = \sum_{i = 1}^n a_i v_i.
  \]
  Now we define a function \(\T\) as follow:
  \[
    \forall v \in \V, \T(v) = \sum_{i = 1}^k a_i v_i.
  \]
  We claim that \(\T\) is a projection on \(\W\) along \(\W'\).
  Since
  \begin{align*}
             & \forall v \in \V, \T(v) = \T\pa{\sum_{i = 1}^n a_i v_i}                                                             \\
             & = \T\pa{\sum_{i = 1}^k a_i v_i + \sum_{i = k + 1}^n a_i v_i} = \sum_{i = 1}^k a_i v_i                               \\
    \implies & \forall v \in \V, \exists (u, u') \in \W \times \W' : \T(v) = \T(u + u') = u,         &  & \text{(by \cref{1.4.3})}
  \end{align*}
  by \cref{2.1.14} we know that \(\T\) is a projection on \(\W\) along \(\W'\).
\end{proof}

\begin{proof}[\pf{ex:2.1.27}(b)]
  See Exercise 2.1.24.
\end{proof}

\begin{defn}\label{2.1.15}
  Let \(\V\) be a vector space over \(\F\), and let \(\T : \V \to \V\) be linear.
  A subspace \(\W\) of \(\V\) over \(\F\) is said to be \textbf{\(\T\)-invariant} if \(\T(x) \in \W\) for every \(x \in \W\), that is, \(\T(\W) \subseteq \W\).
  If \(\W\) is \(\T\)-invariant, we define the \textbf{restriction of \(\T\) on \(\W\)} to be the function \(\T_{\W} : \W \to \W\) defined by \(\T_{\W}(x) = \T(x)\) for all \(x \in \W\).
\end{defn}

\cref{ex:2.1.28,ex:2.1.29,ex:2.1.30,ex:2.1.31,ex:2.1.32} assume that \(\W\) is a subspace of a vector space \(\V\) over \(\F\) and that \(\T : \V \to \V\) is linear.

\begin{ex}\label{ex:2.1.28}
  Prove that the subspaces \(\set{\zv}\), \(\V\), \(\rg{\T}\), and \(\ns{\T}\) are all \(\T\)-invariant.
\end{ex}

\begin{proof}[\pf{ex:2.1.28}]
  First we show that \(\set{\zv}\) is \(\T\)-invariant.
  By \cref{2.1.2}(a) we know that \(\T(\set{\zv}) = \set{\zv} \subseteq \set{\zv}\), thus by \cref{2.1.15} \(\set{\zv}\) is \(\T\)-invariant.

  Next we show that \(\V\) is \(\T\)-invariant.
  Obviously \(\T(\V) \subseteq \T(\V)\), thus by \cref{2.1.15} \(\V\) is \(\T\)-invariant.

  Next we show that \(\rg{\T}\) is \(\T\)-invariant.
  Since
  \begin{align*}
             & \forall y \in \rg{\T}, y \in \V          &  & (\T : \V \to \V)          \\
    \implies & \forall y \in \rg{\T}, \T(y) \in \rg{\T} &  & \text{(by \cref{2.1.10})} \\
    \implies & \T(\rg{\T}) \subseteq \rg{\T},
  \end{align*}
  by \cref{2.1.15} we know that \(\rg{\T}\) is \(\T\)-invariant.

  Finally we show that \(\ns{\T}\) is \(\T\)-invariant.
  Since
  \begin{align*}
             & \forall x \in \ns{\T}, \T(x) = \zv                 &  & \text{(by \cref{2.1.10})} \\
    \implies & \T(\ns{\T}) \subseteq \set{\zv} \subseteq \ns{\T}, &  & \text{(by \cref{1.3}(a))}
  \end{align*}
  by \cref{2.1.15} we know that \(\ns{\T}\) is \(\T\)-invariant.
\end{proof}

\begin{ex}\label{ex:2.1.29}
  If \(\W\) is \(\T\)-invariant, prove that \(\T_{\W}\) is linear.
\end{ex}

\begin{proof}[\pf{ex:2.1.29}]
  Let \(x, y \in \W\) and let \(c \in \F\).
  Since
  \begin{align*}
             & cx + y \in \W                &  & \text{(by \cref{1.3}(b)(c))} \\
    \implies & \T_{\W}(cx + y) = \T(cx + y) &  & \text{(by \cref{2.1.15})}    \\
             & = c \T(x) + \T(y)            &  & \text{(by \cref{2.1.2}(b))}  \\
             & = c \T_{\W}(x) + \T_{\W}(y), &  & \text{(by \cref{2.1.15})}
  \end{align*}
  by \cref{2.1.2}(b) we know that \(\T_{\W}\) is linear.
\end{proof}

\begin{ex}\label{ex:2.1.30}
  Suppose that \(\T\) is a projection on \(\W\) along some subspace \(\W'\).
  Prove that \(\W\) is \(\T\)-invariant and that \(\T_{\W} = \IT[\W]\).
\end{ex}

\begin{proof}[\pf{ex:2.1.30}]
  We have
  \begin{align*}
             & \T \text{ is a projection on } \W \text{ along } \W'                                                                 \\
    \implies & \begin{dcases}
      \V = \W \oplus \W' \\
      \forall x \in \V, \exists (x_1, x_2) \in \W \times \W' : \T(x) = \T(x_1 + x_2) = x_1
    \end{dcases}                                                           &  & \text{(by \cref{2.1.14})} \\
    \implies & \forall x \in \W, \exists (x, \zv) \in \W \times \W' : \T(x) = \T(x + \zv) = x \in \W &  & \text{(by \cref{1.3}(a))} \\
    \implies & \T(\W) \subseteq \W                                                                                                  \\
    \implies & \W \text{ is } \T\text{-invariant}                                                    &  & \text{(by \cref{2.1.15})}
  \end{align*}
  and
  \begin{align*}
             & \forall x \in \W, \T(x) = x                                     \\
    \implies & \forall x \in \W, \T_{\W}(x) = x &  & \text{(by \cref{2.1.15})} \\
    \implies & \T_{\W} = \IT[\W].               &  & \text{(by \cref{2.1.9})}
  \end{align*}
\end{proof}

\begin{ex}\label{ex:2.1.31}
  Suppose that \(\V = \rg{\T} \oplus \W\) and \(\W\) is \(\T\)-invariant.
  (See \cref{1.3.11}.)
  \begin{enumerate}
    \item Prove that \(\W \subseteq \ns{\T}\).
    \item Show that if \(\V\) is finite-dimensional, then \(\W = \ns{\T}\).
    \item Show by example that the conclusion of (b) is not necessarily true if \(\V\) is not finite-dimensional.
  \end{enumerate}
\end{ex}

\begin{proof}[\pf{ex:2.1.31}(a)]
  We have
  \begin{align*}
             & \begin{dcases}
      \V = \rg{\T} \oplus \W \\
      \T(\W) \subseteq \W
    \end{dcases}                               &  & \text{(by \cref{2.1.15})} \\
    \implies & \rg{\T} \cap \T(\W) \subseteq \rg{\T} \cap \W = \set{\zv} &  & \text{(by \cref{1.3.11})} \\
    \implies & \T(\W) = \rg{\T} \cap \T(\W) \subseteq \set{\zv}          &  & \text{(by \cref{2.1.10})} \\
    \implies & \forall x \in \W, \T(x) = \zv                                                            \\
    \implies & \W \subseteq \ns{\T}.                                     &  & \text{(by \cref{2.1.10})}
  \end{align*}
\end{proof}

\begin{proof}[\pf{ex:2.1.31}(b)]
  Since \(\V = \rg{\T} \oplus \W\), by \cref{ex:1.6.29}(b) we know that \(\dim(\V) = \rk{\T} + \dim(\W)\).
  By dimension theorem (\cref{2.3}) we know that \(\ns{\T} = \dim(\V) - \rk{\T} = \dim(\W)\).
  By \cref{ex:2.1.31}(a) we know that \(\W \subseteq \ns{\T}\), thus by \cref{1.11} we have \(\W = \ns{\T}\).
\end{proof}

\begin{proof}[\pf{ex:2.1.31}(c)]
  Let \(\V\) and \(\T\) defined as in \cref{ex:2.1.21}.
  Let \(\W = \set{(0, 0, \dots)}\).
  By \cref{ex:2.1.21}(b) we know that \(\V = \rg{\T} = \rg{\T} \oplus \W\).
  But then we have
  \begin{align*}
             & (1, 0, 0, \dots) \in \V                                                     \\
    \implies & \T(1, 0, 0, \dots) = (0, 0, \dots)        &  & \text{(by \cref{ex:2.1.21})} \\
    \implies & (1, 0, 0, \dots) \in \ns{\T} \setminus \W                                   \\
    \implies & \ns{\T} \neq \W.
  \end{align*}
\end{proof}

\begin{ex}\label{ex:2.1.32}
  Suppose that \(\W\) is \(\T\)-invariant.
  Prove that \(\ns{\T_{\W}} = \ns{\T} \cap \W\) and \(\rg{\T_{\W}} = \T(\W)\).
\end{ex}

\begin{proof}[\pf{ex:2.1.32}]
  First we show that \(\ns{\T_{\W}} = \ns{\T} \cap \W\).
  Since
  \begin{align*}
         & x \in \ns{\T_{\W}}                                         \\
    \iff & \begin{dcases}
      x \in \W \\
      \T_{\W}(x) = \zv
    \end{dcases} &  & \text{(by \cref{2.1.10})} \\
    \iff & \begin{dcases}
      x \in \W \\
      \T(x) = \zv
    \end{dcases} &  & \text{(by \cref{2.1.15})} \\
    \iff & \begin{dcases}
      x \in \W \\
      x \in \ns{\T}
    \end{dcases} &  & \text{(by \cref{2.1.15})} \\
    \iff & x \in \ns{\T} \cap \W,
  \end{align*}
  we know that \(\ns{\T_{\W}} = \ns{\T} \cap \W\).

  Now we show that \(\rg{\T_{\W}} = \T(\W)\).
  This is true since
  \begin{align*}
    \T(\W) & = \T_{\W}(\W)   &  & \text{(by \cref{2.1.15})} \\
           & = \rg{\T_{\W}}. &  & \text{(by \cref{2.1.10})}
  \end{align*}
\end{proof}

\begin{ex}\label{ex:2.1.33}
  Prove \cref{2.2} for the case that \(\beta\) is infinite, that is, \(\rg{\T} = \spn{\set{\T(v) : v \in \beta}}\).
\end{ex}

\begin{proof}[\pf{ex:2.1.33}]
  Clearly we have \(\set{\T(v) : v \in \beta} \subseteq \rg{\T}\).
  By \cref{2.1} we know that \(\rg{\T}\) is a vector space over \(\F\), thus by \cref{1.5} we have
  \[
    \spn{\set{\T(v) : v \in \beta}} \subseteq \rg{\T}.
  \]
  Let \(x \in \V\).
  Since \(\beta\) is a basis for \(\V\) over \(\F\), by \cref{1.6.1} there exists some \(\seq{v}{1,2,,n} \in \beta\) such that \(x \in \spn{\set{\seq{v}{1,2,,n}}}\).
  In particular, there exist some \(\seq{a}{1,2,,n} \in \F\) such that \(x = \sum_{i = 1}^n a_i v_i\).
  Since \(\T\) is linear, by \cref{2.1.2}(d) we know that
  \[
    \T(x) = \T\pa{\sum_{i = 1}^n a_i v_i} = \sum_{i = 1}^n a_i \T(v_i) \in \spn{\set{\T(v) : v \in \beta}}.
  \]
  This means \(\rg{\T} \subseteq \spn{\set{\T(v) : v \in \beta}}\), thus we have \(\rg{\T} = \spn{\set{\T(v) : v \in \beta}}\).
\end{proof}

\begin{ex}\label{ex:2.1.34}
  Prove the following generalization of \cref{2.6}:
  Let \(\V\) and \(\W\) be vector spaces over a common field \(\F\), and let \(\beta\) be a basis for \(\V\) over \(\F\).
  Then for any function \(f : \beta \to \W\) there exists exactly one linear transformation \(\T : \V \to \W\) such that \(\T(x) = f(x)\) for all \(x \in \beta\).
\end{ex}

\begin{proof}[\pf{ex:2.1.34}]
  Note that \(\V\) can be infinite-dimensional.
  Let \(x \in \V\).
  Since \(\beta\) is a basis for \(\V\) over \(\F\), there exists some \(v_1^x, v_2^x, \dots, v_{n^x}^x \in \beta\) and \(a_1^x, a_2^x, \dots, a_{n^x}^x \in \F\) such that \(x = \sum_{i = 1}^{n^x} a_i x_i\).
  Here we use \(v_i^x, a_i^x\) and \(n^x\) to emphasis the choice of \(v_i^x, a_i^x, n^x\) depends on \(x\).
  Now we define \(\T : \V \to \W\) as follow:
  \[
    \forall x \in \V, \T(x) = \sum_{i = 1}^n a_i^x f(v_i^x).
  \]
  \begin{itemize}
    \item \(\T\) is linear:
          Suppose that \(x, y \in \V\) and \(c \in \F\).
          Then we may write
          \[
            x = \sum_{i = 1}^{n^x} a_i^x v_i^x \quad \text{and} \quad y = \sum_{i = 1}^{n^y} a_i^y v_i^y.
          \]
          Since \(\set{v_1^x, \dots, v_{n^x}^x}\) and \(\set{v_1^y, \dots, v_{n^y}^y}\) are finite, we know that \(\gamma = \set{v_1^x, \dots, v_{n^x}^x} \cup \set{v_1^y, \dots, v_{n^y}^y}\) is also finite.
          Let \(n = \#(\gamma)\).
          By relabeling vectors in \(\gamma\) as \(v_1, \dots v_n\) we have
          \[
            x = \sum_{i = 1}^n a_i^x v_i \quad \text{and} \quad y = \sum_{i = 1}^n a_i^y v_i
          \]
          where \(a_i^x = 0\) if \(v_i \notin \set{v_1^x, \dots, v_{n^x}^x}\) and \(a_i^y = 0\) if \(v_i \notin \set{v_1^y, \dots, v_{n^y}^y}\).
          Thus
          \[
            cx + y = \sum_{i = 1}^n c a_i^x v_i + \sum_{i = 1}^n a_i^y v_i = \sum_{i = 1}^n (c a_i^x + a_i^y) v_i.
          \]
          So
          \[
            \T(cx + y) = \sum_{i = 1}^n (c a_i^x + a_i^y) f(v_i) = c \sum_{i = 1}^n a_i^x f(v_i) + \sum_{i = 1}^n a_i^y f(v_i) = c \T(x) + \T(y).
          \]
    \item Clearly
          \[
            \forall v \in \beta, \T(v) = f(v).
          \]
    \item \(\T\) is unique:
          Suppose that \(\U : \V \to \W\) is linear and \(\U(v) = f(v)\) for all \(v \in \beta\).
          Then for \(x \in \V\) with
          \[
            x = \sum_{i = 1}^n a_i^x v_i^x,
          \]
          we have
          \[
            \U(x) = \sum_{i = 1}^n a_i^x \U(v_i^x) = \sum_{i = 1}^n a_i^x f(v_i^x) = \T(x).
          \]
          Hence \(\U = \T\).
  \end{itemize}
\end{proof}

\begin{ex}\label{ex:2.1.35}
  Let \(\V\) be a finite-dimensional vector space over \(\F\) and \(\T : \V \to \V\) be linear.
  \begin{enumerate}
    \item Suppose that \(\V = \rg{\T} + \ns{\T}\).
          Prove that \(\V = \rg{\T} \oplus \ns{\T}\).
    \item Suppose that \(\rg{\T} \cap \ns{\T} = \set{\zv}\).
          Prove that \(\V = \rg{\T} \oplus \ns{\T}\).
  \end{enumerate}
\end{ex}

\begin{proof}[\pf{ex:2.1.35}(a)]
  We have
  \begin{align*}
             & \begin{dcases}
      \V = \rg{\T} + \ns{\T} \\
      \dim(\V) = \rk{\T} + \nt{\T}
    \end{dcases}  &  & \text{(by \cref{2.3})}          \\
    \implies & \V = \rg{\T} \oplus \ns{\T}. &  & \text{(by \cref{ex:1.6.29}(b))}
  \end{align*}
\end{proof}

\begin{proof}[\pf{ex:2.1.35}(b)]
  By \cref{ex:1.3.23}(a) we know that \(\rg{\T} + \ns{\T} \subseteq \V\).
  Since
  \begin{align*}
    \dim(\rg{\T} + \ns{\T}) & = \rk{\T} + \ns{\T} - \dim(\rg{\T} \cap \ns{\T}) &  & \text{(by \cref{ex:1.6.29}(a))} \\
                            & = \rk{\T} + \ns{\T} - 0                          &  & \text{(by hypothesis)}          \\
                            & = \dim(\V),                                      &  & \text{(by \cref{2.3})}
  \end{align*}
  by \cref{1.11} we know that \(\V = \rg{\T} + \ns{\T}\).
  Thus by \cref{ex:2.1.35}(a) we know that \(\V = \rg{\T} \oplus \ns{\T}\).
\end{proof}

\begin{ex}\label{ex:2.1.36}
  Let \(\V\) and \(\T\) be as defined in \cref{ex:2.1.21}.
  \begin{enumerate}
    \item Prove that \(\V = \rg{\T} + \ns{\T}\), but \(\V\) is not a direct sum of these two spaces.
          Thus the result of \cref{ex:2.1.35}(a) cannot be proved without assuming that \(\V\) is finite-dimensional.
    \item Find a linear operator \(\T_1\) on \(\V\) such that \(\rg{\T_1} \cap \ns{\T_1} = \set{\zv}\) but \(\V\) is not a direct sum of \(\rg{\T_1}\) and \(\ns{\T_1}\).
          Conclude that \(\V\) being finite-dimensional is also essential in \cref{ex:2.1.35}(b).
  \end{enumerate}
\end{ex}

\begin{proof}[\pf{ex:2.1.36}(a)]
  We have
  \begin{align*}
             & \T \text{ is onto}                                                   &  & \text{(by \cref{ex:2.1.21}(b))} \\
    \implies & \V = \rg{\T}                                                         &  & \text{(by \cref{2.1.10})}       \\
    \implies & \rg{\T} + \ns{\T} \subseteq \V = \rg{\T} \subseteq \rg{\T} + \ns{\T} &  & \text{(by \cref{ex:1.3.23}(a))} \\
    \implies & \V = \rg{\T} + \ns{\T}.
  \end{align*}
  Since \(\T(1, 0, 0, \dots) = (0, 0, \dots)\), we know that \((1, 0, 0, \dots) \in \ns{\T}\).
  Thus \((1, 0, 0, \dots) \in \rg{\T} \cap \ns{\T}\) and \(\rg{\T} \cap \ns{\T} \neq \set{\zv}\).
  By \cref{1.3.11} this means \(\V\) is not a direct sum of \(\rg{\T}\) and \(\ns{\T}\).
\end{proof}

\begin{proof}[\pf{ex:2.1.36}(b)]
  Define \(\U\) as in \cref{ex:2.1.21}.
  By \cref{ex:2.1.21}(c) we know that \(\U\) is one-to-one, thus by \cref{2.4} we know that \(\ns{\U} = \set{\zv}\) and therefore \(\rg{\U} \cap \ns{\U} = \set{\zv}\).
  By \cref{ex:2.1.21}(c) we know that \(\U\) is not onto, thus \(\V \neq \rg{\U} \cup \ns{\U}\).
  By \cref{1.3.11} this means \(\V\) is not a direct sum of \(\rg{\U}\) and \(\ns{\U}\).
\end{proof}

\begin{ex}\label{ex:2.1.37}
  A function \(\T : \V \to \W\) between vector spaces \(\V\) and \(\W\) over \(\F\) is called \textbf{additive} if \(\T(x + y) = \T(x) + \T(y)\) for all \(x, y \in \V\).
  Prove that if \(\V\) and \(\W\) are vector spaces over the field of rational numbers \(\Q\), then any additive function from \(\V\) into \(\W\) is a linear transformation.
\end{ex}

\begin{proof}[\pf{ex:2.1.37}]
  First observe that for all \(b \in \Z^+\) we have
  \begin{align*}
             & \forall x \in \V, \T(x) = \T\pa{b \frac{1}{b} x} = \T\pa{\sum_{i = 1}^b \frac{1}{b} x} &  & \text{(by \cref{1.2.1})}     \\
             & = \sum_{i = 1}^b \T\pa{\frac{1}{b} x} = b \T\pa{\frac{1}{b} x}                         &  & \text{(by \cref{ex:2.1.37})} \\
    \implies & \forall x \in \V, \frac{1}{b} \T(x) = \T\pa{\frac{1}{b} x}.                            &  & \text{(by \cref{1.2.1})}
  \end{align*}
  Let \(q \in \Q\).
  Then there exists some \((a, b) \in \Z \times \Z^+\) such that \(q = a / b\).
  Now we split into three cases:
  \begin{itemize}
    \item If \(a = 0\), then we have
          \begin{align*}
                     & \forall x \in \V, \T\pa{\frac{0}{b} x} = \T(0x) = \T(0x + \zv_{\V}) &  & \text{(by \ref{vs3})}        \\
                     & = \T(0x + 0x)                                                       &  & \text{(by \cref{1.2}(a))}    \\
                     & = \T(0x) + \T(0x) = \T(0x) + \T\pa{\frac{0}{b} x}                   &  & \text{(by \cref{ex:2.1.37})} \\
                     & = \T(0x) + \zv_{\W}                                                 &  & \text{(by \ref{vs3})}        \\
                     & = \T(0x) + 0 \T(x) = \T(0x) + \frac{0}{b} \T(x)                     &  & \text{(by \cref{1.2}(a))}    \\
            \implies & \forall x \in \V, \T\pa{\frac{0}{b}x} = \frac{0}{b} \T(x).          &  & \text{(by \cref{1.1})}
          \end{align*}
    \item If \(a > 0\), then we have
          \begin{align*}
            \forall x \in \V, \frac{a}{b} \T(x) & = \sum_{i = 1}^a \frac{1}{b} \T(x)    &  & \text{(by \cref{1.2.1})}      \\
                                                & = \sum_{i = 1}^a \T\pa{\frac{1}{b} x} &  & \text{(from the proof above)} \\
                                                & = \T\pa{\sum_{i = 1}^a \frac{1}{b} x} &  & \text{(by \cref{ex:2.1.37})}  \\
                                                & = \T\pa{\frac{a}{b} x}.               &  & \text{(by \cref{1.2.1})}
          \end{align*}
    \item If \(a < 0\), then we have
          \begin{align*}
                     & \forall x \in \V, \T\pa{\frac{a}{b} x} + \T\pa{\frac{-a}{b} x} = \T\pa{\frac{a}{b} x + \frac{-a}{b} x} &  & \text{(by \cref{ex:2.1.37})}  \\
                     & = \T(\zv_{\V})                                                                                         &  & \text{(by \cref{1.2.1})}      \\
                     & = \zv_{\W}                                                                                             &  & \text{(from the proof above)} \\
            \implies & \forall x \in \V, \T\pa{\frac{a}{b} x} + \frac{-a}{b} \T(x) = \zv_{\W}                                 &  & \text{(from the proof above)} \\
            \implies & \forall x \in \V, \T\pa{\frac{a}{b} x} = \frac{a}{b} \T(x).
          \end{align*}
  \end{itemize}
  From all cases above we conclude that \(\T(qx) = q \T(x)\).
  Since \(q\) is arbitrary and \(\T\) is additive, by \cref{2.1.1} we know that \(\T\) is linear.
\end{proof}

\section{The Matrix Representation of a Linear Transformation}\label{sec:2.2}

\begin{note}
  In \cref{sec:2.2}, we embark on one of the most useful approaches to the analysis of a linear transformation on a finite-dimensional vector space:
  the representation of a linear transformation by a matrix.
  In fact, we develop a one-to-one correspondence between matrices and linear transformations that allows us to utilize properties of one to study properties of the other.
\end{note}

\begin{defn}\label{2.2.1}
  Let \(\V\) be a finite-dimensional vector space over \(\F\).
  An \textbf{ordered basis} for \(\V\) over \(\F\) is a basis for \(\V\) over \(\F\) endowed with a specific order;
  that is, an ordered basis for \(\V\) over \(\F\) is a finite sequence of linearly independent vectors in \(\V\) that generates \(\V\).
\end{defn}

\begin{defn}\label{2.2.2}
  For the vector space \(\vs{F}^n\), we call \(\set{\seq{e}{1,2,,n}}\) the \textbf{standard ordered basis} for \(\vs{F}^n\) over \(\F\).
  Similarly, for the vector space \(\ps[n]{\F}\), we call \(\set{1, x, \dots, x^n}\) the \textbf{standard ordered basis} for \(\ps[n]{\F}\) over \(\F\).
\end{defn}

\begin{defn}\label{2.2.3}
  Let \(\beta = \set{\seq{u}{1,2,,n}}\) be an ordered basis for a finite-dimensional vector space \(\V\) over \(\F\).
  For \(x \in \V\), let \(\seq{a}{1,2,,n} \in \F\) be the unique scalars such that
  \[
    x = \sum_{i = 1}^n a_i u_i.
  \]
  We define the \textbf{coordinate vector of \(x\) relative to \(\beta\)}, denoted \([x]_{\beta}\), by
  \[
    [x]_{\beta} = \begin{pmatrix}
      a_1    \\
      a_2    \\
      \vdots \\
      a_n
    \end{pmatrix}.
  \]
\end{defn}

\begin{defn}\label{2.2.4}
  Suppose that \(\V\) and \(\W\) are finite-dimensional vector spaces over \(\F\) with ordered bases \(\beta = \set{\seq{v}{1,2,,n}}\) and \(\gamma = \set{\seq{w}{1,2,,m}}\) over \(\F\), respectively.
  Let \(\T : \V \to \W\) be linear.
  Then for each \(j\), \(1 \leq j \leq n\), there exist unique scalars \(a_{i j} \in \F\), \(1 \leq i \leq m\), such that
  \[
    \T(v_j) = \sum_{i = 1}^m a_{i j} w_i \quad \text{for } 1 \leq j \leq n.
  \]
  We call the \(m \times n\) matrix \(A\) defined by \(A_{i j} = a_{i j}\) the \textbf{matrix representation of \(\T\) in the ordered bases \(\beta\) and \(\gamma\)} and write \(A = [\T]_{\beta}^{\gamma}\).
  If \(\V = \W\) and \(\beta = \gamma\), then we write \(A = [\T]_{\beta}\).
\end{defn}

\begin{note}
  The \(j\)th column of \(A\) is simply \([\T(v_j)]_{\gamma}\).
  Also observe that if \(\U : \V \to \W\) is a linear transformation such that \([\U]_{\beta}^{\gamma} = [\T]_{\beta}^{\gamma}\), then \(\U = \T\) by \cref{2.1.13}.
\end{note}

\begin{defn}\label{2.2.5}
  Let \(\T, \U : \V \to \W\) be arbitrary functions, where \(\V\) and \(\W\) are vector spaces over \(\F\), and let \(a \in \F\).
  We define \(\T + \U : \V \to \W\) by \((\T + \U)(x) = \T(x) + \U(x)\) for all \(x \in \V\), and \(a \T: \V \to \W\) by \((a \T)(x) = a \T(x)\) for all \(x \in \V\).
\end{defn}

\begin{thm}\label{2.7}
  Let \(\V\) and \(\W\) be vector spaces over a field \(\F\), and let \(\T, \U : \V \to \W\) be linear.
  \begin{enumerate}
    \item For all \(a \in \F\), \(a \T + \U\) is linear.
    \item Using the operations of addition and scalar multiplication in \cref{2.2.5}, the collection of all linear transformations from \(\V\) to \(\W\) is a vector space over \(\F\).
  \end{enumerate}
\end{thm}

\begin{proof}[\pf{2.7}(a)]
  Let \(x, y \in \V\) and \(c \in \F\).
  Then
  \begin{align*}
    (a \T + \U)(cx + y) & = a \T(cx + y) + \U(cx + y)            &  & \text{(by \cref{2.2.5})}    \\
                        & = a(\T(cx + y)) + c \U(x) + \U(y)      &  & \text{(by \cref{2.1.2}(b))} \\
                        & = a(c \T(x) + \T(y)) + c \U(x) + \U(y) &  & \text{(by \cref{2.1.2}(b))} \\
                        & = ac \T(x) + c \U(x) + a \T(y) + \U(y) &  & \text{(by \cref{1.2.1})}    \\
                        & = c (a \T + \U)(x) + (a \T + \U)(y).   &  & \text{(by \cref{2.2.5})}
  \end{align*}
  So \(a \T + \U\) is linear.
\end{proof}

\begin{proof}[\pf{2.7}(b)]
  Let \(\ls(\V, \W)\) be the set of all linear transformation from \(\V\) to \(\W\).
  First we show that \ref{vs1}--\ref{vs8} is true.
  Let \(f, g, h \in \ls(\V, \W)\) and let \(a, b \in \F\).
  Now we split into eight cases:
  \begin{description}
    \item[For \ref{vs1}:] We have
      \begin{align*}
        \forall x \in \V, (f + g)(x) & = f(x) + g(x) &  & \text{(by \cref{2.2.5})} \\
                                     & = g(x) + f(x) &  & \text{(by \ref{vs1})}    \\
                                     & = (g + f)(x)  &  & \text{(by \cref{2.2.5})}
      \end{align*}
      and thus \(f + g = g + f\).
    \item[For \ref{vs2}:] We have
      \begin{align*}
        \forall x \in \V, ((f + g) + h)(x) & = (f + g)(x) + h(x)    &  & \text{(by \cref{2.2.5})} \\
                                           & = (f(x) + g(x)) + h(x) &  & \text{(by \cref{2.2.5})} \\
                                           & = f(x) + (g(x) + h(x)) &  & \text{(by \ref{vs2})}    \\
                                           & = f(x) + (g + h)(x)    &  & \text{(by \cref{2.2.5})} \\
                                           & = (f + (g + h))(x)     &  & \text{(by \cref{2.2.5})}
      \end{align*}
      and thus \((f + g) + h = f + (g + h)\).
    \item[For \ref{vs3}:] We have
      \begin{align*}
        \forall x \in \V, (f + \zT)(x) & = f(x) + \zT(x)   &  & \text{(by \cref{2.2.5})} \\
                                       & = f(x) + \zv_{\W} &  & \text{(by \cref{2.1.9})} \\
                                       & = f(x)            &  & \text{(by \ref{vs3})}
      \end{align*}
      and thus \(f + \zT = f\).
    \item[For \ref{vs4}:] We have
      \begin{align*}
        \forall x \in \V, (f + ((-1)f))(x) & = f(x) + ((-1)f)(x) &  & \text{(by \cref{2.2.5})} \\
                                           & = f(x) + (-1)f(x)   &  & \text{(by \cref{2.2.5})} \\
                                           & = \zv_{\W}          &  & \text{(by \ref{vs4})}    \\
                                           & = \zT(x)            &  & \text{(by \cref{2.1.9})}
      \end{align*}
      and thus \(f + (-1)f = \zT\).
    \item[For \ref{vs5}:] We have
      \begin{align*}
        \forall x \in \V, (1f)(x) & = 1f(x) &  & \text{(by \cref{2.2.5})} \\
                                  & = f(x)  &  & \text{(by \ref{vs5})}
      \end{align*}
      and thus \(1f = f\).
    \item[For \ref{vs6}:] We have
      \begin{align*}
        \forall x \in \V, ((ab)f)(x) & = (ab) f(x)   &  & \text{(by \cref{2.2.5})} \\
                                     & = a (bf(x))   &  & \text{(by \ref{vs6})}    \\
                                     & = a ((bf)(x)) &  & \text{(by \cref{2.2.5})} \\
                                     & = (a (bf))(x) &  & \text{(by \cref{2.2.5})}
      \end{align*}
      and thus \((ab)f = a (bf)\).
    \item[For \ref{vs7}:] We have
      \begin{align*}
        \forall x \in \V, (a(f + g))(x) & = a((f + g)(x))     &  & \text{(by \cref{2.2.5})} \\
                                        & = a(f(x) + g(x))    &  & \text{(by \cref{2.2.5})} \\
                                        & = af(x) + ag(x)     &  & \text{(by \ref{vs7})}    \\
                                        & = (af)(x) + (ag)(x) &  & \text{(by \cref{2.2.5})} \\
                                        & = (af + ag)(x)      &  & \text{(by \cref{2.2.5})}
      \end{align*}
      and thus \(a(f + g) = af + ag\).
    \item[For \ref{vs8}:] We have
      \begin{align*}
        \forall x \in \V, ((a + b)f)(x) & = (a + b) f(x)      &  & \text{(by \cref{2.2.5})} \\
                                        & = af(x) + bf(x)     &  & \text{(by \ref{vs8})}    \\
                                        & = (af)(x) + (bf)(x) &  & \text{(by \cref{2.2.5})} \\
                                        & = (af + bf)(x)      &  & \text{(by \cref{2.2.5})}
      \end{align*}
      and thus \((a + b)f = af + bf\).
  \end{description}
  From the proofs above we see that \ref{vs1}--\ref{vs8} is true.

  By \cref{2.7}(a) and the proofs above we see that
  \[
    \forall \T, \U \in \ls(\V, \W), \T + \U = 1 \T + \U \in \ls(\V, \W)
  \]
  and
  \[
    \forall (c, \T) \in \F \times \ls(\V, \W), c \T = c \T + \zT \in \ls(\V, \W).
  \]
  Thus by \cref{1.2.1} \(\ls(\V, \W)\) is a vector space over \(\F\).
\end{proof}

\begin{defn}\label{2.2.6}
  Let \(\V\) and \(\W\) be vector spaces over \(\F\).
  We denote the vector space of all linear transformations from \(\V\) into \(\W\) by \(\ls(\V, \W)\).
  In the case that \(\V = \W\), we write \(\ls(\V)\) instead of \(\ls(\V, \W)\).
\end{defn}

\begin{thm}\label{2.8}
  Let \(\V\) and \(\W\) be finite-dimensional vector spaces over \(\F\) with ordered bases \(\beta\) and \(\gamma\) over \(\F\), respectively, and let \(\T, \U : \V \to \W\) be linear transformations.
  Then
  \begin{enumerate}
    \item \([\T + \U]_{\beta}^{\gamma} = [\T]_{\beta}^{\gamma} + [\U]_{\beta}^{\gamma}\) and
    \item \([c \T]_{\beta}^{\gamma} = c [\T]_{\beta}^{\gamma}\) for all scalars \(c \in \F\).
  \end{enumerate}
\end{thm}

\begin{proof}[\pf{2.8}]
  Let \(\beta = \set{\seq{v}{1,2,,n}}\) and \(\gamma = \set{\seq{w}{1,2,,m}}\).
  There exist unique scalars \(a_{i j}\) and \(b_{i j}\) (\(1 \leq i \leq m\), \(1 \leq j \leq n\)) such that
  \[
    \T(v_j) = \sum_{i = 1}^m a_{i j} w_i \quad \text{and} \quad \U(v_j) = \sum_{i = 1}^m b_{i j} w_i \quad \text{for } 1 \leq j \leq n.
  \]
  Hence
  \begin{align*}
    (\T + \U)(v_j) & = \sum_{i = 1}^m (a_{i j} + b_{i j}) w_i \\
    (c \T)(v_j)    & = \sum_{i = 1}^m (c a_{i j}) w_i.
  \end{align*}
  Thus
  \begin{align*}
    ([\T + \U]_{\beta}^{\gamma})_{i j} & = a_{i j} + b_{i j} = ([\T]_{\beta}^{\gamma} + [\U]_{\beta}^{\gamma})_{i j} \\
    ([c \T]_{\beta}^{\gamma})_{i j}    & = c a_{i j} = c ([\T]_{\beta}^{\gamma})_{i j}.
  \end{align*}
\end{proof}

\exercisesection

\setcounter{ex}{7}
\begin{ex}\label{ex:2.2.8}
  Let \(\V\) be an \(n\)-dimensional vector space over \(\F\) with an ordered basis \(\beta\) over \(\F\).
  Define \(\T : \V \to \vs{F}^n\) by \(\T(x) = [x]_{\beta}\).
  Prove that \(\T\) is linear.
\end{ex}

\begin{proof}[\pf{ex:2.2.8}]
  Let \(\beta = \set{\seq{v}{1,2,,n}}\), let \(x, y \in \V\) and let \(c \in \F\).
  Since
  \begin{align*}
             & \exists \seq{a}{1,2,,n}, \seq{b}{1,2,,n} \in \F : \begin{dcases}
                                                                   x = \sum_{i = 1}^n a_i v_i \\
                                                                   y = \sum_{i = 1}^n b_i v_i
                                                                 \end{dcases} &  & \text{(by \cref{1.8})}       \\
    \implies & cx + y = \sum_{i = 1}^n (ca_i + b_i) v_i                         &  & \text{(by \cref{1.2.1})}   \\
    \implies & \T(cx + y) = [cx + y]_{\beta} = \begin{pmatrix}
                                                 ca_1 + b_1 \\
                                                 \vdots     \\
                                                 ca_n + b_n
                                               \end{pmatrix}                  &  & \text{(by \cref{2.2.3})}     \\
             & = c \begin{pmatrix}
                     a_1    \\
                     \vdots \\
                     a_n
                   \end{pmatrix} + \begin{pmatrix}
                                     b_1    \\
                                     \vdots \\
                                     b_n
                                   \end{pmatrix}                                  &  & \text{(by \cref{1.2.9})} \\
             & = c [x]_{\beta} + [y]_{\beta} = c \T(x) + \T(y),                 &  & \text{(by \cref{2.2.3})}
  \end{align*}
  by \cref{2.1.2}(b) we know that \(\T\) is linear.
\end{proof}

\begin{ex}\label{ex:2.2.9}
  Let \(\V\) be the vector space of complex numbers \(\C\) over the field \(\R\).
  Define \(\T : \V \to \V\) by \(\T(z) = \overline{z}\), where \(\overline{z}\) is the complex conjugate of \(z\).
  Prove that \(\T\) is linear, and compute \([\T]_{\beta}\), where \(\beta = \set{1, i}\).
  (Recall by \cref{ex:2.1.38} that \(\T\) is not linear if \(\V\) is regarded as a vector space over the field \(\C\).)
\end{ex}

\begin{proof}[\pf{ex:2.2.9}]
  Let \(x, y \in \C\) and let \(c \in \R\).
  Since
  \begin{align*}
    \T(cx + y) & = \overline{cx + y}                              \\
               & = \overline{cx} + \overline{y}                   \\
               & = \overline{c} \cdot \overline{x} + \overline{y} \\
               & = c \overline{x} + \overline{y}                  \\
               & = c \T(x) + \T(y),
  \end{align*}
  by \cref{2.1.2}(b) we know that \(\T\) is linear.
  Since
  \begin{align*}
    \T(1) & = 1 = 1 \cdot 1 + 0 \cdot i,     \\
    \T(i) & = -i = 0 \cdot 1 + (-1) \cdot i,
  \end{align*}
  by \cref{2.2.4} we know that
  \[
    [\T]_{\beta} = \begin{pmatrix}
      1 & 0  \\
      0 & -1
    \end{pmatrix}.
  \]
\end{proof}

\begin{ex}\label{ex:2.2.10}
  Let \(\V\) be a vector space over \(\F\) with the ordered basis \(\beta = \set{\seq{v}{1,2,,n}}\) over \(\F\).
  Define \(v_0 = 0\).
  By \cref{2.6} there exists a \(\T \in \ls(\V)\) such that \(\T(v_j) = v_j + v_{j - 1}\) for \(j = 1, 2, \dots, n\).
  Compute \([\T]_{\beta}\).
\end{ex}

\begin{proof}[\pf{ex:2.2.10}]
  By \cref{2.2.4,ex:1.5.6} we have
  \[
    [\T]_{\beta} = \begin{pmatrix}
      1      & 1      & 0      & \cdots & 0      & 0      & 0      \\
      0      & 1      & 1      & \cdots & 0      & 0      & 0      \\
      0      & 0      & 1      & \cdots & 0      & 0      & 0      \\
      \vdots & \vdots & \vdots & \ddots & \vdots & \vdots & \vdots \\
      0      & 0      & 0      & \cdots & 1      & 1      & 0      \\
      0      & 0      & 0      & \cdots & 0      & 1      & 1      \\
      0      & 0      & 0      & \cdots & 0      & 0      & 1
    \end{pmatrix} = \sum_{i = 1}^n E^{i i} + \sum_{i = 2}^n E^{(i - 1) i}.
  \]
\end{proof}

\begin{ex}\label{ex:2.2.11}
  Let \(\V\) be an \(n\)-dimensional vector space over \(\F\), and let \(\T \in \ls(\V)\).
  Suppose that \(\W\) is a \(\T\)-invariant subspace of \(\V\) over \(\F\) (see \cref{2.1.15}) having dimension \(k\).
  Show that there is a basis \(\beta\) for \(\V\) over \(\F\) such that \([\T]_{\beta}\) has the form
  \[
    \begin{pmatrix}
      A   & B \\
      \zm & C
    \end{pmatrix},
  \]
  where \(A\) is a \(k \times k\) matrix and \(\zm\) is the \((n - k) \times k\) zero matrix.
\end{ex}

\begin{proof}[\pf{ex:2.2.11}]
  Let \(\beta_{\W} = \set{\seq{v}{1,2,,k}}\) be a basis for \(\W\) over \(\F\).
  By \cref{1.6.19} we can extend \(\beta_{\W}\) to \(\beta = \set{\seq{v}{1,2,,n}}\) such that \(\beta\) is a basis for \(\V\) over \(\F\).
  Since \(\W\) is \(\T\)-invariant, we know that
  \begin{align*}
             & \T(\W) \subseteq \W                                                                                               &  & \text{(by \cref{2.1.15})} \\
    \implies & \forall v_j \in \beta_{\W}, \T(v_j) \in \W = \spn{\beta_{\W}}                                                     &  & \text{(by \cref{1.6.1})}  \\
    \implies & \forall v_j \in \beta_{\W}, \exists a_{1 j}, a_{2 j}, \dots, a_{k j} \in \F :                                                                    \\
             & \T(v_j) = \sum_{i = 1}^k a_{i j} v_i = \sum_{i = 1}^k a_{i j} v_i + \sum_{i = k + 1}^n 0 v_i                      &  & \text{(by \cref{1.2}(a))} \\
    \implies & \forall v_j \in \beta_{\W}, \exists a_{1 j}, a_{2 j}, \dots, a_{k j} \in \F : [\T(v_j)]_{\beta} = \begin{pmatrix}
                                                                                                                   a_{1 j} \\
                                                                                                                   \vdots  \\
                                                                                                                   a_{k j} \\
                                                                                                                   0       \\
                                                                                                                   \vdots  \\
                                                                                                                   0
                                                                                                                 \end{pmatrix}. &  & \text{(by \cref{2.2.3})}
  \end{align*}
  By setting
  \begin{align*}
    A & = \begin{pmatrix}
            a_{1 1} & \cdots & a_{1 k} \\
            \vdots  & \ddots & \vdots  \\
            a_{k 1} & \cdots & a_{k k}
          \end{pmatrix} \in \ms{k}{k}{\F}                                                                    \\
    B & = \begin{pmatrix}
            ([\T(v_{k + 1})]_{\beta})_1 & \cdots & ([\T(v_n)]_{\beta})_1 \\
            \vdots                      & \ddots & \vdots                \\
            ([\T(v_{k + 1})]_{\beta})_k & \cdots & ([\T(v_n)]_{\beta})_k
          \end{pmatrix} \in \ms{k}{(n - k)}{\F}             \\
    C & = \begin{pmatrix}
            ([\T(v_{k + 1})]_{\beta})_{k + 1} & \cdots & ([\T(v_n)]_{\beta})_{k + 1} \\
            \vdots                            & \ddots & \vdots                      \\
            ([\T(v_{k + 1})]_{\beta})_n       & \cdots & ([\T(v_n)]_{\beta})_n
          \end{pmatrix} \in \ms{(n - k)}{(n - k)}{\F}
  \end{align*}
  we have
  \begin{align*}
    [\T]_{\beta} & = \begin{pmatrix}
                       ([\T(v_1)]_{\beta})_1 & \cdots & ([\T(v_n)]_{\beta})_1   \\
                       \vdots                & \ddots & \vdots                  \\
                       ([\T(v_1)]_{\beta})_n & \cdots & ([\T(v_n)]_{\beta})_{n}
                     \end{pmatrix}                                        &  & \text{(by \cref{2.2.4})}                                        \\
                 & = \begin{pmatrix}
                       a_{1 1} & \cdots & a_{1 k} & ([\T(v_{k + 1})]_{\beta})_1       & \cdots & ([\T(v_n)]_{\beta})_1       \\
                       \vdots  & \ddots & \vdots  & \vdots                            & \ddots & \vdots                      \\
                       a_{k 1} & \cdots & a_{k k} & ([\T(v_{k + 1})]_{\beta})_k       & \cdots & ([\T(v_n)]_{\beta})_k       \\
                       0       & \cdots & 0       & ([\T(v_{k + 1})]_{\beta})_{k + 1} & \cdots & ([\T(v_n)]_{\beta})_{k + 1} \\
                       \vdots  & \ddots & \vdots  & \vdots                            & \ddots & \vdots                      \\
                       0       & \cdots & 0       & ([\T(v_{k + 1})]_{\beta})_{n}     & \cdots & ([\T(v_n)]_{\beta})_{n}
                     \end{pmatrix} \\
                 & = \begin{pmatrix}
                       A   & B \\
                       \zm & C
                     \end{pmatrix}.
  \end{align*}
\end{proof}

\begin{ex}\label{ex:2.2.12}
  Let \(\V\) be a finite-dimensional vector space over \(\F\) and \(\T\) be the projection on \(\W\) along \(\W'\), where \(\W\) and \(\W'\) are subspaces of \(\V\) over \(\F\).
  (See \cref{2.1.14}.)
  Find an ordered basis \(\beta\) for \(\V\) over \(\F\) such that \([\T]_{\beta}\) is a diagonal matrix.
\end{ex}

\begin{proof}[\pf{ex:2.2.12}]
  Let \(\beta_{\W} = \set{\seq{v}{1,2,,k}}\) be a basis for \(\W\) over \(\F\).
  By \cref{1.6.19} we can extend \(\beta_{\W}\) to \(\beta = \set{\seq{v}{1,2,,n}}\) such that \(\beta\) is a basis for \(\V\) over \(\F\).
  Since
  \begin{align*}
             & \begin{dcases}
                 \forall v_j \in \beta_{\W}, v_j = v_j + \zv \in \W + \W' \\
                 \forall v_j \in \beta \setminus \beta_{\W}, v_j = \zv + v_j \in \W + \W'
               \end{dcases}                         &  & \text{(by \cref{1.3.10})}                         \\
    \implies & \begin{dcases}
                 \forall v_j \in \beta_{\W}, \T(v_j) = v_j \\
                 \forall v_j \in \beta \setminus \beta_{\W}, \T(v_j) = \zv
               \end{dcases}                                        &  & \text{(by \cref{2.1.14})}              \\
    \implies & \begin{dcases}
                 \forall v_j \in \beta_{\W}, [\T(v_j)]_{\beta} = e_j \in \vs{F}^n \\
                 \forall v_j \in \beta \setminus \beta_{\W}, [\T(v_j)]_{\beta} = \zv \in \vs{F}^n
               \end{dcases} &  & \text{(by \cref{2.2.2})} \\
    \implies & [\T]_{\beta} = \begin{pmatrix}
                                e_1 & \cdots & e_k & \zv & \cdots & \zv
                              \end{pmatrix}                                          \\
             & = \begin{pmatrix}
                   1      & 0      & \cdots & 0      & 0      & \cdots & 0      \\
                   0      & 1      & \cdots & 0      & 0      & \cdots & 0      \\
                   \vdots & \vdots & \ddots & \vdots & 0      & \cdots & 0      \\
                   0      & 0      & \cdots & 1      & 0      & \cdots & 0      \\
                   0      & 0      & \cdots & 0      & 0      & \cdots & 0      \\
                   \vdots & \vdots & \ddots & \vdots & \vdots & \ddots & \vdots \\
                   0      & 0      & \cdots & 0      & 0      & \cdots & 0
                 \end{pmatrix},
  \end{align*}
  by \cref{1.3.8} we know that \([\T]_{\beta}\) is a diagonal matrix.
\end{proof}

\begin{ex}\label{ex:2.2.13}
  Let \(\V\) and \(\W\) be vector spaces over \(\F\), and let \(\T\) and \(\U\) be nonzero linear transformations from \(\V\) into \(\W\).
  If \(\rg{\T} \cap \rg{\U} = \set{\zv}\), prove that \(\set{\T, \U}\) is a linearly independent subset of \(\ls(\V, \W)\).
\end{ex}

\begin{proof}[\pf{ex:2.2.13}]
  Let \(\zT : \V \to \W\) be the zero transformation in \(\ls(\V, \W)\).
  Suppose for sake of contradiction that \(\set{\T, \U}\) is linearly dependent.
  Since
  \begin{align*}
             & \set{\T} \text{ is linearly independent}        &  & \text{(by \cref{1.5.4}(b))} \\
    \implies & \U \in \spn{\set{\T}}                           &  & \text{(by \cref{1.7})}      \\
    \implies & \exists c \in \F \setminus \set{0} : \U = c \T, &  & \text{(by \cref{1.4.3})}
  \end{align*}
  by fixing such \(c\) we know that
  \begin{align*}
             & \T \neq \zT                                                                             \\
    \implies & \exists x \in \V : \T(x) \neq \zv_{\W}                                                  \\
    \implies & \exists x \in \V : \U(\frac{1}{c} x) = \frac{1}{c} \U(x) &  & \text{(by \cref{2.1.1})}  \\
             & = \frac{1}{c} (c \T(x)) = \T(x) \neq \zv_{\W}                                           \\
    \implies & \rg{\U} \cap \rg{\T} \neq \set{\zv}.                     &  & \text{(by \cref{2.1.10})}
  \end{align*}
  But this contradicts to the fact that \(\rg{\U} \cap \rg{\T} = \set{\zv}\).
  Thus \(\set{\T, \U}\) is linearly independent.
\end{proof}

\begin{ex}\label{ex:2.2.14}
  Let \(f \in \ps{\R}\), and for \(j \geq 1\) define \(\T_j(f) = f^{(j)}\), where \(f^{(j)}\) is the \(j\)th derivative of \(f\).
  Prove that the set \(\set{\seq{\T}{1,2,,n}}\) is a linearly independent subset of \(\ls(\ps{\R})\) for any positive integer \(n\).
\end{ex}

\begin{proof}[\pf{ex:2.2.14}]
  First we show that \(\T_i \in \ls(\ps{\R})\) for all \(i \in \Z^+\).
  Let \(f, g \in \ps{\R}\) and let \(c \in \R\).
  Since
  \[
    \forall i \in \Z^+, \begin{dcases}
      \rg{\T_i} \subseteq \ps{\R} \\
      \T_i(cf + g) = (cf + g)^{(i)} = c f^{(i)} + g^{(i)} = c \T_i(f) + \T_i(g)
    \end{dcases},
  \]
  by \cref{2.1.2}(b) we know that \(\T_i \in \ls(\ps{\R})\) for all \(i \in \Z^+\).

  Now we show that \(\set{\seq{\T}{1,2,,n}}\) is linearly independent.
  Let \(\seq{b}{1,2,,n} \in \R\) such that
  \[
    \sum_{i = 1}^n b_i \T_i = \zT
  \]
  where \(\zT\) is the zero transformation of \(\ls(\ps{\R})\).
  Since
  \begin{align*}
             & \sum_{i = 1}^n b_i \T_i(x^n) = \sum_{i = 1}^n \pa{b_i \cdot \pa{\prod_{k = 0}^{i - 1} (n - k)} x^{n - i}} = 0                               \\
    \implies & \forall i \in \set{1, \dots, n}, b_i \cdot \prod_{k = 0}^{i - 1} (n - k) = 0                                  &  & \text{(by \cref{1.6.5})} \\
    \implies & b_i = 0,                                                                                                      &  & (n - k > 0)
  \end{align*}
  by \cref{1.5.3} we know that \(\set{\seq{\T}{1,2,,n}}\) is linearly independent.
\end{proof}

\begin{ex}\label{ex:2.2.15}
  Let \(\V\) and \(\W\) be vector spaces over \(\F\), and let \(S\) be a subset of \(\V\).
  Define \(S^0 = \set{\T \in \ls(\V, \W) : \T(x) = 0 \text{ for all } x \in S}\).
  Prove the following statements.
  \begin{enumerate}
    \item \(S^0\) is a subspace of \(\ls(\V, \W)\) over \(\F\).
    \item If \(S_1\) and \(S_2\) are subsets of \(V\) and \(S_1 \subseteq S_2\), then \(S_2^0 \subseteq S_1^0\).
    \item If \(\V_1\) and \(\V_2\) are subspaces of \(\V\) over \(\F\), then \((\V_1 + \V_2)^0 = \V_1^0 \cap \V_2^0\).
  \end{enumerate}
\end{ex}

\begin{proof}[\pf{ex:2.2.15}(a)]
  Let \(\zT : \V \to \W\) be the zero transformation in \(\ls(\V, \W)\).
  Since
  \begin{align*}
             & \forall x \in \V, \zT(x) = \zv_{\W} &  & \text{(by \cref{2.1.9})}     \\
    \implies & \forall x \in S, \zT(x) = \zv_{\W}                                    \\
    \implies & \zT \in S^0                         &  & \text{(by \cref{ex:2.2.15})}
  \end{align*}
  and
  \begin{align*}
             & \begin{dcases}
                 \forall \T, \U \in S^0 \\
                 \forall c \in \F       \\
                 \forall x \in S
               \end{dcases}, (c\T + \U)(x) = c \T(x) + \U(x) = \zv_{\W} &  & \text{(by \cref{2.2.5})}     \\
    \implies & \begin{dcases}
                 \forall \T, \U \in S^0 \\
                 \forall c \in \F
               \end{dcases}, c\T + \U \in S^0,                          &  & \text{(by \cref{ex:2.2.15})}
  \end{align*}
  by \cref{ex:1.3.18} we know that \(S^0\) is a subspace of \(\ls(\V, \W)\) over \(\F\).
\end{proof}

\begin{proof}[\pf{ex:2.2.15}(b)]
  Let \(\T \in S_2^0\).
  Since
  \begin{align*}
             & \T \in S_2^0                                                          \\
    \implies & \forall x \in S_2, \T(x) = \zv_{\W} &  & \text{(by \cref{ex:2.2.15})} \\
    \implies & \forall x \in S_1, \T(x) = \zv_{\W} &  & (S_1 \subseteq S_2)          \\
    \implies & \T \in S_1^0                        &  & \text{(by \cref{ex:2.2.15})}
  \end{align*}
  and \(\T\) is arbitrary, we know that \(S_2^0 \subseteq S_1^0\).
\end{proof}

\begin{proof}[\pf{ex:2.2.15}(c)]
  Since
  \begin{align*}
         & \T \in (\V_1 + \V_2)^0                                                                    \\
    \iff & \forall x \in \V_1 + \V_2, \T(x) = \zv_{\W}             &  & \text{(by \cref{ex:2.2.15})} \\
    \iff & \forall (x_1, x_2) \in \V_1 \times \V_2, \begin{dcases}
                                                      x_1 + \zv_{\V} \in \V_1 + \V_2 \\
                                                      \zv_{\V} + x_2 \in \V_1 + \V_2 \\
                                                      x_1 + x_2 \in \V_1 + \V_2      \\
                                                      \T(x_1) = \T(x_2) = \T(x_1 + x_2) = \zv_{\W}
                                                    \end{dcases} &  & \text{(by \cref{1.3.10})}      \\
    \iff & \begin{dcases}
             \T \in \V_1^0 \\
             \T \in \V_2^0
           \end{dcases}                                        &  & \text{(by \cref{ex:2.2.15})}     \\
    \iff & \T \in \V_1^0 \cap \V_2^0,
  \end{align*}
  we know that \((\V_1 + \V_2)^0 = \V_1^0 \cap \V_2^0\).
\end{proof}

\begin{ex}\label{ex:2.2.16}
  Let \(\V\) and \(\W\) be vector spaces over \(\F\) such that \(\dim(\V) = \dim(\W)\), and let \(\T \in \ls(\V, \W)\).
  Show that there exist ordered bases \(\beta\) and \(\gamma\) for \(\V\) and \(\W\) over \(\F\), respectively, such that \([\T]_{\beta}^{\gamma}\) is a diagonal matrix.
\end{ex}

\begin{proof}[\pf{ex:2.2.16}]
  If \(\T\) is the zero transformation, then for arbitrary bases \(\beta\) and \(\gamma\) for \(\V\) and \(\W\) over \(\F\), respectively, we have
  \[
    [\T]_{\beta}^{\gamma} = \zm
  \]
  which by \cref{1.3.8} is a diagonal matrix.
  So suppose that \(\T\) is not the zero transformation.
  Let \(\beta\) be a basis for \(\V\) over \(\F\).
  By \cref{2.2} we know that \(\spn{\T(\beta)} = \rg{\T}\).
  Since \(\T\) is not the zero transformation, we know that
  \begin{align*}
             & \exists x \in \V : \T(x) \neq \zv_{\W}                &  & \text{(by \cref{2.1.9})}  \\
    \implies & 1 \leq \rk{\T} \leq \dim(\W)                          &  & \text{(by \cref{2.1.12})} \\
    \implies & \exists \gamma_1 \subseteq \T(\beta) : \begin{dcases}
                                                        \gamma_1 \text{ is linearly independent} \\
                                                        \spn{\gamma_1} = \rg{\T}
                                                      \end{dcases}. &  & \text{(by \cref{1.6.8})}
  \end{align*}
  Fix such \(\gamma_1\).
  By \cref{2.3} we know that \(\#(\gamma_1) \leq \#(\beta_1)\) for all \(\beta_1 \subseteq \beta\) and \(\T(\beta_1) = \gamma_1\).
  Thus we can define \(\beta_1 \subseteq \beta\) such that \(\T(\beta_1) = \gamma_1\) and \(\#(\beta_1) = \#(\gamma_1)\).
  Let \(k = \#(\beta_1)\) and we can write \(\beta_1 = \set{\seq{v}{1,2,,k}}\) and \(\gamma_1 = \set{\seq{w}{1,2,,k}}\) such that
  \[
    \forall i \in \set{1, \dots, k}, \T(v_i) = w_i.
  \]
  If we let \(\beta\) follow this ordered and we extend \(\gamma_1\) to a basis \(\gamma\) for \(\W\) over \(\F\) (this can be done by \cref{1.6.19}), then we see that (by setting \(n = \dim(\V)\) and \(\beta = \set{\seq{v}{1,2,,n}}\))
  \begin{align*}
             & \forall v_i \in \beta, [\T(v_i)]_{\gamma} = \begin{dcases}
                                                             \zv_{\W}           & \text{if } i \notin \set{1, \dots, k} \\
                                                             e_i \in \vs{F}^{n} & \text{if } i \in \set{1, \dots, k}
                                                           \end{dcases}   &  & \text{(by \cref{2.2.3})} \\
    \implies & [\T]_{\beta}^{\gamma} = \begin{pmatrix}
                                         1      & 0      & \cdots & 0      & 0      & \cdots & 0      \\
                                         0      & 1      & \cdots & 0      & 0      & \cdots & 0      \\
                                         \vdots & \vdots & \ddots & \vdots & \vdots & \ddots & \vdots \\
                                         0      & 0      & \cdots & 1      & 0      & \cdots & 0      \\
                                         0      & 0      & \cdots & 0      & 0      & \cdots & 0      \\
                                         \vdots & \vdots & \ddots & \vdots & \vdots & \ddots & 0      \\
                                         0      & 0      & \cdots & 0      & 0      & \cdots & 0
                                       \end{pmatrix} &  & \text{(by \cref{2.2.4})}
  \end{align*}
  and by \cref{1.3.8} \([\T]_{\beta}^{\gamma}\) is a diagonal matrix.
\end{proof}

\section{Composition of Linear Transformations and Matrix Multiplication}\label{sec:2.3}

\begin{note}
  We use the more convenient notation of \(\U \T\) rather than \(\U \circ \T\) for the composite of linear transformations \(\U\) and \(\T\).
\end{note}

\begin{thm}\label{2.9}
  Let \(\V\), \(\W\), and \(\vs{Z}\) be vector spaces over the same field \(\F\), and let \(\T \in \ls(\V, \W)\) and \(\U \in \ls(\W, \vs{Z})\).
  Then \(\U \T \in \ls(\V, \vs{Z})\).
\end{thm}

\begin{proof}[\pf{2.9}]
  Let \(x, y \in \V\) and \(a \in \F\).
  Then
  \begin{align*}
    \U \T(ax + y) & = \U(\T(ax + y))                                            \\
                  & = \U(a \T(x) + \T(y))      &  & \text{(by \cref{2.1.2}(b))} \\
                  & = a \U(\T(x)) + \U(\T(y))  &  & \text{(by \cref{2.1.2}(b))} \\
                  & = a (\U \T)(x) + \U \T(y).
  \end{align*}
\end{proof}

\begin{thm}\label{2.10}
  Let \(\V, \W, \vs{X}, \vs{Y}\) be vector spaces over \(\F\).
  Let \(\lt{S}, \lt{S}_1, \lt{S}_2 \in \ls(\vs{X}, \vs{Y})\), let \(\T \in \ls(\W, \vs{X})\) and let \(\U, \U_1, \U_2 \in \ls(\V, \W)\).
  Then
  \begin{enumerate}
    \item \(\T(\U_1 + \U_2) = \T \U_1 + \T \U_2\) and \((\lt{S}_1 + \lt{S}_2) \T = \lt{S}_1 \T + \lt{S}_2 \T\).
    \item \(\lt{S} (\T \U) = (\lt{S} \T) \U\).
    \item \(\T \IT[\W] = \IT[\vs{X}] \T = \T\).
    \item \(a(\T \U) = (a \T) \U = \T (a \U)\) for all scalars \(a \in \F\).
  \end{enumerate}
\end{thm}

\begin{proof}[\pf{2.10}(a)]
  For all \(x \in \V\), we have
  \begin{align*}
    (\T(\U_1 + \U_2))(x) & = \T((\U_1 + \U_2)(x))                                      \\
                         & = \T(\U_1(x) + \U_2(x))       &  & \text{(by \cref{2.2.5})} \\
                         & = \T(\U_1(x)) + \T(\U_2(x))   &  & \text{(by \cref{2.1.1})} \\
                         & = (\T \U_1)(x) + (\T \U_2)(x)                               \\
                         & = (\T \U_1 + \T \U_2)(x).     &  & \text{(by \cref{2.2.5})}
  \end{align*}
  Thus \(\T(\U_1 + \U_2) = \T \U_1 + \T \U_2\).
  For all \(x \in \W\), we have
  \begin{align*}
    ((\vs{S}_1 + \vs{S}_2) \T)(x) & = (\vs{S}_1 + \vs{S}_2)(\T(x))                                      \\
                                  & = \vs{S}_1(\T(x)) + \vs{S}_2(\T(x))   &  & \text{(by \cref{2.2.5})} \\
                                  & = (\vs{S}_1 \T)(x) + (\vs{S}_2 \T)(x)                               \\
                                  & = (\vs{S}_1 \T + \vs{S}_2 \T)(x).     &  & \text{(by \cref{2.2.5})}
  \end{align*}
  Thus \((\vs{S}_1 + \vs{S}_2) \T = \vs{S}_1 \T + \vs{S}_2 \T\).
\end{proof}

\begin{proof}[\pf{2.10}(b)]
  For all \(x \in \V\), we have
  \begin{align*}
    (\lt{S} (\T \U))(x) & = \lt{S}((\T \U)(x))  \\
                        & = \lt{S}(\T(\U(x)))   \\
                        & = (\lt{S} \T)(\U(x))  \\
                        & = ((\lt{S} \T) \U)(x)
  \end{align*}
  and thus \(\lt{S} (\T \U) = (\lt{S} \T) \U\).
\end{proof}

\begin{proof}[\pf{2.10}(c)]
  For all \(x \in \V\), we have
  \begin{align*}
    (\T \IT[\W])(x) & = \T(\IT[\W](x))                                    \\
                    & = \T(x)               &  & \text{(by \cref{2.1.9})} \\
                    & = \IT[\vs{X}](\T(x))  &  & \text{(by \cref{2.1.9})} \\
                    & = (\IT[\vs{X}] \T)(x)
  \end{align*}
  and thus \(\T \IT[\W] = \T = \IT[\vs{X}] \T\).
\end{proof}

\begin{proof}[\pf{2.10}(d)]
  For all \(x \in \V\), we have
  \begin{align*}
    (a (\T \U))(x) & = a (\T \U)(x)                                 \\
                   & = a \T(\U(x))                                  \\
                   & = (a \T)(\U(x))  &  & \text{(by \cref{2.2.5})} \\
                   & = ((a \T) \U)(x)                               \\
                   & = \T(a \U(x))    &  & \text{(by \cref{2.1.1})} \\
                   & = \T((a \U)(x))                                \\
                   & = (\T (a \U))(x)
  \end{align*}
  and thus \(a (\T \U) = (a \T) \U = \T (a \U)\).
\end{proof}

\begin{defn}\label{2.3.1}
  Let \(A \in \MS\) matrix and \(B \in \ms{n}{p}{\F}\).
  We define the \textbf{product} of \(A\) and \(B\), denoted \(AB\), to be the \(m \times p\) matrix such that
  \[
    (AB)_{i j} = \sum_{k = 1}^n A_{i k} B_{k j} \quad \text{for } 1 \leq i \leq m, 1 \leq j \leq p.
  \]
  Note that \((AB)_{i j}\) is the sum of products of corresponding entries from the \(i\)th row of \(A\) and the \(j\)th column of \(B\).
\end{defn}

\begin{note}
  The reader should observe that in order for the product \(AB\) to be defined, there are restrictions regarding the relative sizes of \(A\) and \(B\).
  The following mnemonic device is helpful:
  ``\((m \times n) \cdot (n \times p) = (m \times p)\)'';
  that is, in order for the product \(AB\) to be defined, the two ``inner'' dimensions must be equal, and the two ``outer'' dimensions yield the size of the product.
\end{note}

\begin{note}
  As in the case with composition of functions, we have that matrix multiplication is not commutative. Consider the following two products:
  \[
    \begin{pmatrix}
      1 & 1 \\
      0 & 0
    \end{pmatrix} \begin{pmatrix}
      0 & 1 \\
      1 & 0
    \end{pmatrix} = \begin{pmatrix}
      1 & 1 \\
      0 & 0
    \end{pmatrix} \quad \text{and} \quad \begin{pmatrix}
      0 & 1 \\
      1 & 0
    \end{pmatrix} \begin{pmatrix}
      1 & 1 \\
      0 & 0
    \end{pmatrix} = \begin{pmatrix}
      0 & 0 \\
      1 & 1
    \end{pmatrix}.
  \]
  Hence we see that even if both of the matrix products \(AB\) and \(BA\) are defined, it need not be true that \(AB = BA\).
\end{note}

\begin{eg}\label{2.3.2}
  If \(A \in \MS\) and \(B \in \ms{n}{p}{\F}\), then \(\tp{(AB)} = \tp{B} \tp{A}\).
\end{eg}

\begin{proof}[\pf{2.3.2}]
  Since
  \[
    \tp{(AB)}_{i j} = (AB)_{j i} = \sum_{k = 1}^n A_{j k} B_{k i}
  \]
  and
  \[
    (\tp{B} \tp{A})_{i j} = \sum_{k = 1}^n \tp{B}_{i k} \tp{A}_{k j} = \sum_{k = 1}^n B_{k i} A_{j k},
  \]
  we are finished.
  Therefore the transpose of a product is the product of the transposes in the \emph{opposite order}.
\end{proof}

\begin{thm}\label{2.11}
  Let \(\V\), \(\W\), and \(\vs{Z}\) be finite-dimensional vector spaces over \(\F\) with ordered bases \(\alpha\), \(\beta\), and \(\gamma\) over \(\F\), respectively.
  Let \(\T \in \ls(\V, \W)\) and \(\U \in \ls(\W, \vs{Z})\).
  Then
  \[
    [\U \T]_{\alpha}^{\gamma} = [\U]_{\beta}^{\gamma} [\T]_{\alpha}^{\beta}.
  \]
\end{thm}

\begin{proof}[\pf{2.11}]
  Define
  \begin{align*}
    \alpha & = \set{\seq{v}{1,2,,n}}; \\
    \beta  & = \set{\seq{w}{1,2,,m}}; \\
    \gamma & = \set{\seq{z}{1,2,,p}}.
  \end{align*}
  For \(1 \leq j \leq n\), we have
  \begin{align*}
    (\U \T)(v_j) & = \sum_{i = 1}^p ([\U \T]_{\alpha}^{\gamma})_{i j} \cdot z_i                                                      &  & \text{(by \cref{2.2.4})}    \\
                 & = \U(\T(v_j))                                                                                                                                      \\
                 & = \U\pa{\sum_{k = 1}^m ([\T]_{\alpha}^{\beta})_{k j} \cdot w_k}                                                   &  & \text{(by \cref{2.2.4})}    \\
                 & = \sum_{k = 1}^m ([\T]_{\alpha}^{\beta})_{k j} \cdot \U(w_k)                                                      &  & \text{(by \cref{2.1.2}(d))} \\
                 & = \sum_{k = 1}^m ([\T]_{\alpha}^{\beta})_{k j} \cdot \pa{\sum_{i = 1}^p ([\U]_{\beta}^{\gamma})_{i k} \cdot z_i}  &  & \text{(by \cref{2.2.4})}    \\
                 & = \sum_{k = 1}^m \sum_{i = 1}^p \pa{([\T]_{\alpha}^{\beta})_{k j}  \cdot ([\U]_{\beta}^{\gamma})_{i k} \cdot z_i} &  & \text{(by \cref{1.2.1})}    \\
                 & = \sum_{k = 1}^m \sum_{i = 1}^p \pa{([\U]_{\beta}^{\gamma})_{i k} \cdot ([\T]_{\alpha}^{\beta})_{k j} \cdot z_i}  &  & \text{(by \cref{1.2.1})}    \\
                 & = \sum_{i = 1}^p \sum_{k = 1}^m \pa{([\U]_{\beta}^{\gamma})_{i k} \cdot ([\T]_{\alpha}^{\beta})_{k j} \cdot z_i}  &  & \text{(by \cref{1.2.1})}    \\
                 & = \sum_{i = 1}^p \pa{\sum_{k = 1}^m ([\U]_{\beta}^{\gamma})_{i k} \cdot ([\T]_{\alpha}^{\beta})_{k j}} \cdot z_i  &  & \text{(by \cref{1.2.1})}    \\
                 & = \sum_{i = 1}^p \pa{\pa{[\U]_{\beta}^{\gamma} [\T]_{\alpha}^{\beta}}_{i j} \cdot z_i}.                           &  & \text{(by \cref{2.3.1})}
  \end{align*}
  Thus by \cref{1.8} we see that \([\U \T]_{\alpha}^{\gamma} = [\U]_{\beta}^{\gamma} [\T]_{\alpha}^{\beta}\).
\end{proof}

\begin{cor}\label{2.3.3}
  Let \(\V\) be a finite-dimensional vector space over \(\F\) with an ordered basis \(\beta\).
  Let \(\T, \U \in \ls(\V)\).
  Then \([\U \T]_{\beta} = [\U]_{\beta} [\T]_{\beta}\).
\end{cor}

\begin{proof}[\pf{2.3.3}]
  We have
  \begin{align*}
    [\U \T]_{\beta} & = [\U \T]_{\beta}^{\beta}                   &  & \text{(by \cref{2.2.4})} \\
                    & = [\U]_{\beta}^{\beta} [\T]_{\beta}^{\beta} &  & \text{(by \cref{2.11})}  \\
                    & = [\U]_{\beta} [\T]_{\beta}.                &  & \text{(by \cref{2.2.4})}
  \end{align*}
\end{proof}

\begin{defn}\label{2.3.4}
  We define the \textbf{Kronecker delta} \(\delta_{i j}\) by \(\delta_{i j} = 1\) if \(i = j\) and \(\delta_{i j} = 0\) if \(i \neq j\).
  The \(n \times n\) \textbf{identity matrix} \(I_n\) is defined by \((I_n)_{i j} = \delta_{i j}\).
\end{defn}

\begin{thm}\label{2.12}
  Let \(A \in \MS\), let \(B, C \in \ms{n}{p}{\F}\) and let \(D, E \in \ms{q}{m}{\F}\).
  Then
  \begin{enumerate}
    \item \(A (B + C) = AB + AC\) and \((D + E) A = DA + EA\).
    \item \(a (AB) = (aA) B = A (aB)\) for any \(a \in \F\).
    \item \(I_m A = A = A I_n\).
    \item If \(\V\) is an \(n\)-dimensional vector space over \(\F\) with an ordered basis \(\beta\), then \([\IT[\V]]_{\beta} = I_n\).
  \end{enumerate}
\end{thm}

\begin{proof}[\pf{2.12}(a)]
  We have
  \begin{align*}
    (A (B + C))_{i j} & = \sum_{k = 1}^n A_{i k} (B + C)_{k j}                            &  & \text{(by \cref{2.3.1})}           \\
                      & = \sum_{k = 1}^n A_{i k} (B_{k j} + C_{k j})                      &  & \text{(by \cref{1.2.9})}           \\
                      & = \sum_{k = 1}^n (A_{i k} B_{k j} + A_{i k} C_{k j})              &  & (A_{i k}, B_{k j}, C_{k j} \in \F) \\
                      & = \sum_{k = 1}^n A_{i k} B_{k j} + \sum_{k = 1}^n A_{i k} C_{k j} &  & (A_{i k}, B_{k j}, C_{k j} \in \F) \\
                      & = (AB)_{i j} + (AC)_{i j}                                         &  & \text{(by \cref{2.3.1})}           \\
                      & = (AB + AC)_{i j}                                                 &  & \text{(by \cref{1.2.9})}
  \end{align*}
  and
  \begin{align*}
    ((D + E) A)_{i j} & = \sum_{k = 1}^m (D + E)_{i k} A_{k j}                            &  & \text{(by \cref{2.3.1})}           \\
                      & = \sum_{k = 1}^m (D_{i k} + E_{i k}) A_{k j}                      &  & \text{(by \cref{1.2.9})}           \\
                      & = \sum_{k = 1}^m (D_{i k} A_{k j} + E_{i k} A_{k j})              &  & (A_{k j}, D_{i k}, E_{i k} \in \F) \\
                      & = \sum_{k = 1}^m D_{i k} A_{k j} + \sum_{k = 1}^m E_{i k} A_{k j} &  & (A_{k j}, D_{i k}, E_{i k} \in \F) \\
                      & = (DA)_{i j} + (EA)_{i j}                                         &  & \text{(by \cref{2.3.1})}           \\
                      & = (DA + EA)_{i j}.                                                &  & \text{(by \cref{1.2.9})}
  \end{align*}
  Thus by \cref{1.2.8} \(A (B + C) = AB + AC\) and \((D + E) A = DA + EA\).
\end{proof}

\begin{proof}[\pf{2.12}(b)]
  We have
  \begin{align*}
    (a (AB))_{i j} & = a (AB)_{i j}                          &  & \text{(by \cref{1.2.9})} \\
                   & = a \pa{\sum_{k = 1}^n A_{i k} B_{k j}} &  & \text{(by \cref{2.3.1})} \\
                   & = \sum_{k = 1}^n (a A_{i k}) B_{k j}    &  & \text{(by \cref{1.2.9})} \\
                   & = \sum_{k = 1}^n (a A)_{i k} B_{k j}    &  & \text{(by \cref{1.2.9})} \\
                   & = ((aA) B)_{i j}                        &  & \text{(by \cref{2.3.1})} \\
                   & = \sum_{k = 1}^n A_{i k} (a B_{k j})    &  & \text{(by \cref{1.2.9})} \\
                   & = \sum_{k = 1}^n A_{i k} (a B)_{k j}    &  & \text{(by \cref{1.2.9})} \\
                   & = (A (aB))_{i j}                        &  & \text{(by \cref{2.3.1})}
  \end{align*}
  thus by \cref{1.2.8} \(a (AB) = (aA) B = A (aB)\).
\end{proof}

\begin{proof}[\pf{2.12}(c)]
  We have
  \begin{align*}
    (I_m A)_{i j} & = \sum_{k = 1}^m (I_m)_{i k} A_{k j}  &  & \text{(by \cref{2.3.1})} \\
                  & = \sum_{k = 1}^m \delta_{i k} A_{k j} &  & \text{(by \cref{2.3.4})} \\
                  & = A_{i j}                             &  & \text{(by \cref{2.3.4})} \\
                  & = \sum_{k = 1}^n A_{i k} \delta_{k j} &  & \text{(by \cref{2.3.4})} \\
                  & = \sum_{k = 1}^n A_{i k} (I_n)_{k j}  &  & \text{(by \cref{2.3.4})} \\
                  & = (A I_n)_{i j}                       &  & \text{(by \cref{2.3.1})}
  \end{align*}
  thus by \cref{1.2.8} \(I_m A = A = A I_n\).
\end{proof}

\begin{proof}[\pf{2.12}(d)]
  Let \(\beta = \set{\seq{v}{1,2,,n}}\).
  For all \(j \in \set{1, 2, \dots, n}\), we have
  \begin{align*}
             & \IT[\V](v_j) = v_j                                                                          &  & \text{(by \cref{2.1.9})} \\
             & = \sum_{i = 1}^n ([\IT[\V]]_{\beta})_{i j} v_i                                              &  & \text{(by \cref{2.2.4})} \\
    \implies & \forall i \in \set{1, 2, \dots, n}, ([\IT[\V]]_{\beta})_{i j} = \delta_{i j} = (I_n)_{i j}. &  & \text{(by \cref{2.3.4})}
  \end{align*}
  Thus by \cref{1.2.8} \([\IT[\V]]_{\beta} = I_n\).
\end{proof}

\begin{note}
  \cref{2.12} provides analogs of (a), (c), and (d) of \cref{2.10}.
  \cref{2.10}(b) has its analog in \cref{2.16}.
  Observe also that part (c) of \cref{2.12} illustrates that the identity matrix acts as a multiplicative identity in \(\ms{n}{n}{\F}\).
  When the context is clear, we sometimes omit the subscript \(n\) from \(I_n\).
\end{note}

\begin{cor}\label{2.3.5}
  Let
  \begin{align*}
    A               & \in \MS;           \\
    \seq{B}{1,2,,k} & \in \ms{n}{p}{\F}; \\
    \seq{C}{1,2,,k} & \in \ms{q}{m}{\F}; \\
    \seq{a}{1,2,,k} & \in \F.
  \end{align*}
  Then
  \begin{align*}
    A \pa{\sum_{i = 1}^k a_i B_i} & = \sum_{i = 1}^k a_i A B_i, \\
    \pa{\sum_{i = 1}^k a_i C_i} A & = \sum_{i = 1}^k a_i C_i A.
  \end{align*}
\end{cor}

\begin{proof}[\pf{2.3.5}]
  We have
  \begin{align*}
    A \pa{\sum_{i = 1}^k a_i B_i} & = \sum_{i = 1}^k (A a_i B_i)  &  & \text{(by \cref{2.12}(a))} \\
                                  & = \sum_{i = 1}^k (a_i A B_i), &  & \text{(by \cref{2.12}(b))} \\
    \pa{\sum_{i = 1}^k a_i C_i} A & = \sum_{i = 1}^k (a_i C_i A). &  & \text{(by \cref{2.12}(a))}
  \end{align*}
\end{proof}

\begin{defn}\label{2.3.6}
  For an \(A \in \ms{n}{n}{\F}\), we define \(A^1 = A\), \(A^2 = AA\), \(A^3 = A^2 A\), and, in general, \(A^k = A^{k - 1} A\) for \(k = 2, 3, \dots\).
  We define \(A^0 = I_n\).
\end{defn}

\begin{eg}\label{2.3.7}
  If
  \[
    A = \begin{pmatrix}
      0 & 0 \\
      1 & 0
    \end{pmatrix},
  \]
  then \(A^2 = \zm\) (the zero matrix) even though \(A \neq \zm\).
  Thus the cancellation property for multiplication in fields is not valid for matrices.
  To see why, assume that the cancellation law is valid.
  Then, from \(A \cdot A = A^2 = \zm = A \cdot \zm\), we would conclude that \(A = \zm\), which is false.
\end{eg}

\begin{thm}\label{2.13}
  Let \(A \in \MS\) and \(B \in \ms{n}{p}{\F}\).
  For each \(j\) (\(1 \leq j \leq p\)) let \(u_j\) and \(v_j\) denote the \(j\)th columns of \(AB\) and \(B\), respectively.
  Then
  \begin{enumerate}
    \item \(u_j = A v_j\).
    \item \(v_j = B e_j\), where \(e_j\) is the \(j\)th standard vector of \(\vs{F}^p\).
  \end{enumerate}
\end{thm}

\begin{proof}[\pf{2.13}(a)]
  We have
  \begin{align*}
    u_j & = \begin{pmatrix}
              (AB)_{1 j} \\
              (AB)_{2 j} \\
              \vdots     \\
              (AB)_{m j}
            \end{pmatrix}                 &  & \text{(by \cref{2.2.4})}   \\
        & = \begin{pmatrix}
              \sum_{k = 1}^n A_{1 k} B_{k j} \\
              \sum_{k = 1}^n A_{2 k} B_{k j} \\
              \vdots                         \\
              \sum_{k = 1}^n A_{m k} B_{k j}
            \end{pmatrix} &  & \text{(by \cref{2.3.1})}                   \\
        & = A \begin{pmatrix}
                B_{1 j} \\
                B_{2 j} \\
                \vdots  \\
                B_{n j}
              \end{pmatrix}               &  & \text{(by \cref{2.3.1})}   \\
        & = A v_j.                          &  & \text{(by \cref{2.2.4})}
  \end{align*}
\end{proof}

\begin{proof}[\pf{2.13}(b)]
  We have
  \begin{align*}
    v_j & = \begin{pmatrix}
              B_{1 j} \\
              B_{2 j} \\
              \vdots  \\
              B_{m j}
            \end{pmatrix}                     &  & \text{(by \cref{2.2.4})}   \\
        & = \begin{pmatrix}
              (B I_p)_{1 j} \\
              (B I_p)_{2 j} \\
              \vdots        \\
              (B I_p)_{m j}
            \end{pmatrix}                     &  & \text{(by \cref{2.12}(c))} \\
        & = \begin{pmatrix}
              \sum_{k = 1}^p B_{1 k} (I_p)_{k j} \\
              \sum_{k = 1}^p B_{2 k} (I_p)_{k j} \\
              \vdots                             \\
              \sum_{k = 1}^p B_{m k} (I_p)_{k j}
            \end{pmatrix} &  & \text{(by \cref{2.3.1})}                       \\
        & = B \begin{pmatrix}
                (I_p)_{1 j} \\
                (I_p)_{2 j} \\
                \vdots      \\
                (I_p)_{p j}
              \end{pmatrix}                   &  & \text{(by \cref{2.3.1})}   \\
        & = B e_j.                              &  & \text{(by \cref{2.3.4})}
  \end{align*}
\end{proof}

\begin{note}
  It follows (see \cref{ex:2.3.14}) from \cref{2.13} that column \(j\) of \(AB\) is a linear combination of the columns of \(A\) with the coefficients in the linear combination being the entries of column \(j\) of \(B\).
  An analogous result holds for rows;
  that is, row \(i\) of \(AB\) is a linear combination of the rows of \(B\) with the coefficients in the linear combination being the entries of row \(i\) of \(A\).
\end{note}

\begin{thm}\label{2.14}
  Let \(\V\) and \(\W\) be finite-dimensional vector spaces over \(\F\) having ordered bases \(\beta\) and \(\gamma\) over \(\F\), respectively, and let \(\T \in \ls(\V, \W)\).
  Then, for each \(u \in \V\), we have
  \[
    [\T(u)]_{\gamma} = [\T]_{\beta}^{\gamma} [u]_{\beta}.
  \]
\end{thm}

\begin{proof}[\pf{2.14}]
  Fix \(u \in \V\), and define the linear transformations \(f : \F \to \V\) by \(f(a) = au\) and \(g : \F \to \W\) by \(g(a) = a \T(u)\) for all \(a \in \F\).
  Let \(\alpha = \set{1}\) be the standard ordered basis for \(\F\).
  Notice that \(g = \T f\).
  Identifying column vectors as matrices and using \cref{2.11}, we obtain
  \[
    [\T(u)]_{\gamma} = [g(1)]_{\gamma} = [g]_{\alpha}^{\gamma} = [\T f]_{\alpha}^{\gamma} = [\T]_{\beta}^{\gamma} [f]_{\alpha}^{\beta} = [\T]_{\beta}^{\gamma} [f(1)]_{\beta} = [\T]_{\beta}^{\gamma} [u]_{\beta}.
  \]
\end{proof}

\begin{defn}\label{2.3.8}
  Let \(A \in \ms{m}{n}{\F}\).
  We denote by \(\L_A\) the mapping \(\L_A : \vs{F}^n \to \vs{F}^m\) defined by \(\L_A(x) = Ax\) (the matrix product of \(A\) and \(x\)) for each column vector \(x \in \vs{F}^n\).
  We call \(\L_A\) a \textbf{left-multiplication transformation}.
\end{defn}

\begin{thm}\label{2.15}
  Let \(A \in \MS\).
  Then the left-multiplication transformation \(\L_A : \vs{F}^n \to \vs{F}^m\) is linear.
  Furthermore, if \(B \in \MS\) and \(\beta\) and \(\gamma\) are the standard ordered bases for \(\vs{F}^n\) and \(\vs{F}^m\) over \(\F\), respectively, then we have the following properties.
  \begin{enumerate}
    \item \([\L_A]_{\beta}^{\gamma} = A\).
    \item \(\L_A = \L_B\) iff \(A = B\).
    \item \(\L_{A + B} = \L_A + \L_B\) and \(\L_{aA} = a \L_A\) for all \(a \in \F\).
    \item If \(\T : \vs{F}^n \to \vs{F}^m\) is linear, then there exists a unique \(C \in \MS\) such that \(\T = \L_C\).
          In fact, \(C = [\T]_{\beta}^{\gamma}\).
    \item If \(E \in \ms{n}{p}{\F}\), then \(\L_{AE} = \L_A \L_E\).
    \item If \(m = n\), then \(\L_{I_n} = \IT[\vs{F}^n]\).
  \end{enumerate}
\end{thm}

\begin{proof}[\pf{2.15}(a)]
  The fact that \(\L_A\) is linear follows immediately from \cref{2.12}(a)(b).
  By \cref{2.2.4} the \(j\)th column of \([\L_A]_{\beta}^{\gamma}\) is equal to \(\L_A(e_j)\).
  However by \cref{2.3.8} we have \(\L_A(e_j) = A e_j\), which is also the \(j\)th column of \(A\) by \cref{2.13}(b).
  So \([\L_A]_{\beta}^{\gamma} = A\).
\end{proof}

\begin{proof}[\pf{2.15}(b)]
  If \(\L_A = \L_B\), then we may use \cref{2.15}(a) to write \(A = [\L_A]_{\beta}^{\gamma} = [\L_B]_{\beta}^{\gamma} = B\).
  Hence \(A = B\).
  The proof of the converse is trivial.
\end{proof}

\begin{proof}[\pf{2.15}(c)]
  For all \(x \in \vs{F}^n\), we have
  \begin{align*}
    \L_{A + B}(x) & = (A + B) x         &  & \text{(by \cref{2.3.8})}   \\
                  & = Ax + Bx           &  & \text{(by \cref{2.12}(a))} \\
                  & = \L_A(x) + \L_B(x) &  & \text{(by \cref{2.3.8})}   \\
                  & = (\L_A + \L_B)(x)  &  & \text{(by \cref{2.2.5})}
  \end{align*}
  and
  \begin{align*}
    \L_{aA}(x) & = (aA) x       &  & \text{(by \cref{2.3.8})}   \\
               & = a (Ax)       &  & \text{(by \cref{2.12}(b))} \\
               & = a \L_A(x)    &  & \text{(by \cref{2.3.8})}   \\
               & = (a \L_A)(x). &  & \text{(by \cref{2.2.5})}
  \end{align*}
  Thus \(\L_{A + B} = \L_A + \L_B\) and \(\L_{aA} = a \L_A\).
\end{proof}

\begin{proof}[\pf{2.15}(d)]
  Let \(C = [\T]_{\beta}^{\gamma}\).
  By \cref{2.14} we have \([\T(x)]_{\gamma} = [\T]_{\beta}^{\gamma} [x]_{\beta}\), or \(\T(x) = Cx = \L_C(x)\) for all \(x \in \vs{F}^n\).
  So \(\T = \L_C\).
  The uniqueness of \(C\) follows from \cref{2.15}(b).
\end{proof}

\begin{proof}[\pf{2.15}(e)]
  For any \(j\) (\(1 \leq j \leq p\)), we may apply \cref{2.13} several times to note that \((AE) e_j\) is the \(j\)th column of \(AE\) and that the \(j\)th column of \(AE\) is also equal to \(A (E e_j)\).
  So \((AE) e_j = A (Ee_j)\).
  Thus
  \[
    \L_{AE}(e_j) = (AE) e_j = A (E e_j) = \L_A(E e_j) = \L_A(L_E(e_j)).
  \]
  Hence \(\L_{AE} = \L_A \L_E\) by \cref{2.1.13}.
\end{proof}

\begin{proof}[\pf{2.15}(f)]
  For all \(x \in \vs{F}^n\), we have
  \begin{align*}
    \L_{I_n}(x) & = I_n x              &  & \text{(by \cref{2.3.8})}   \\
                & = x                  &  & \text{(by \cref{2.12}(c))} \\
                & = \IT_{\vs{F}^n}(x). &  & \text{(by \cref{2.1.9})}
  \end{align*}
  Thus \(\L_{I_n} = \IT_{\vs{F}^n}\).
\end{proof}

\begin{thm}\label{2.16}
  Let \(A\), \(B\), and \(C\) be matrices such that \(A (BC)\) is defined.
  Then \((AB) C\) is also defined and \(A (BC) = (AB) C\);
  that is, matrix multiplication is associative.
\end{thm}

\begin{proof}[\pf{2.16}]
  Since \(A (BC)\) is defined, by \cref{2.3.1} we can let \(A \in \MS\) and \(BC \in \ms{n}{p}{\F}\) such that \(A (BC) \in \ms{m}{p}{\F}\).
  Since \(BC\) is also defined, by \cref{2.3.1} again we can let \(B \in \ms{n}{k}{\F}\) and \(C \in \ms{k}{p}{\F}\).
  Then we have
  \begin{align*}
             & \begin{dcases}
                 A \in \MS           \\
                 B \in \ms{n}{k}{\F} \\
                 C \in \ms{k}{p}{\F}
               \end{dcases}                                    \\
    \implies & \begin{dcases}
                 AB \in \ms{m}{k}{\F} \\
                 C \in \ms{k}{p}{\F}
               \end{dcases}  &  & \text{(by \cref{2.3.1})}            \\
    \implies & (AB) C \in \ms{m}{p}{\F} &  & \text{(by \cref{2.3.1})}
  \end{align*}
  and thus \((AB) C\) is defined.

  Using \cref{2.15}(e) and the associativity of functional composition, we have
  \[
    \L_{A (BC)} = \L_A \L_{BC} = \L_A (\L_B \L_C) = (\L_A \L_B) \L_C = \L_{AB} \L_C = \L_{(AB) C}.
  \]
  So by \cref{2.15}(b), it follows that \(A (BC) = (AB) C\).
\end{proof}

\begin{defn}\label{2.3.9}
  An \textbf{incidence matrix} is a square matrix in which all the entries are either zero or one and, for convenience, all the diagonal entries are zero.
  If we have a relationship on a set of \(n\) objects that we denote by \(1, 2, \dots, n\), then we define the associated incidence matrix \(A\) by \(A_{i j} = 1\) if \(i\) is related to \(j\), and \(A_{i j} = 0\) otherwise.
\end{defn}

\begin{eg}\label{2.3.10}
  A maximal collection of three or more people with the property that any two can send to each other is called a \textbf{clique}.
  The problem of determining cliques is difficult, but there is a simple method for determining if someone belongs to a clique.
  If we define a new matrix \(B\) by \(B_{i j} = 1\) if \(i\) and \(j\) can send to each other, and \(B_{i j} = 0\) otherwise, then person \(i\) belongs to a clique iff \((B^3)_{i i} > 0\).
\end{eg}

\begin{proof}[\pf{2.3.10}]
  Suppose that \(B \in \ms{n}{n}{\F}\).
  We know that person \(i\) belongs to a clique iff there exists at least two other people \(j, k\) such that \(i\) can send to \(j\), \(j\) can send to \(k\) and \(k\) can send back to \(i\).
  In other words \(B_{i j} = B_{j k} = B_{k i} = 1\).
  Thus we have
  \begin{align*}
         & \exists j, k \in \set{1, \dots, n} \setminus \set{i} : B_{i j} = B_{j k} = B_{k i} = 1           &  & \text{(by \cref{2.3.10})} \\
    \iff & \exists j, k \in \set{1, \dots, n} \setminus \set{i} : B_{i j} B_{j k} B_{k i} = 1                                              \\
    \iff & (B^3)_{i i} = (BBB)_{i i}                                                                        &  & \text{(by \cref{2.3.6})}  \\
         & = \sum_{j = 1}^n B_{i j} (BB)_{j i} = \sum_{j = 1}^n B_{i j} \pa{\sum_{k = 1}^n B_{j k} B_{k i}} &  & \text{(by \cref{2.3.1})}  \\
         & = \sum_{j = 1}^n \sum_{k = 1}^n (B_{i j} B_{j k} B_{k i}) > 0.                                   &  & \text{(by \cref{1.2.1})}
  \end{align*}
\end{proof}

\begin{eg}\label{2.3.11}
  A relation among a group of people is called a \textbf{dominance relation} if the associated incidence matrix \(A\) has the property that for all distinct pairs \(i\) and \(j\), \(A_{i j} = 1\) iff \(A_{j i} = 0\), that is, given any two people, exactly one of them dominates the other.
  Since \(A\) is an incidence matrix, \(A_{i i} = 0\) for all \(i\).
  For such a relation, it can be shown that the matrix \(A + A^2\) has a row [column] in which each entry is positive except for the diagonal entry.
  In other words, there is at least one person who dominates [is dominated by] all others in one or two stages.
  In fact, it can be shown that any person who dominates [is dominated by] the greatest number of people in the first stage has this property.
\end{eg}

\begin{proof}[\pf{2.3.11}]
  Let \(A \in \ms{n}{n}{\F}\).
  First observe that
  \begin{align*}
    (A + A^2)_{i j} & = A_{i j} + (A^2)_{i j}                     &  & \text{(by \cref{1.2.9})} \\
                    & = A_{i j} + (AA)_{i j}                      &  & \text{(by \cref{2.3.6})} \\
                    & = A_{i j} + \sum_{k = 1}^n A_{i k} A_{k j}. &  & \text{(by \cref{2.3.1})}
  \end{align*}
  We see that all diagonal entries of \(A + A^2\) is \(0\) since
  \begin{align*}
             & \forall (i, j) \in \set{1, \dots, n}^2, (A_{i j} = 0) \lor (A_{j i} = 0)      &  & \text{(by \cref{2.3.11})}     \\
    \implies & \forall (i, j) \in \set{1, \dots, n}^2, A_{i j} A_{j i} = 0                                                      \\
    \implies & \forall i \in \set{1, \dots, n}, \sum_{j = 1}^n A_{i j} A_{j i} = 0                                              \\
    \implies & \forall i \in \set{1, \dots, n}, A_{i i} + \sum_{j = 1}^n A_{i j} A_{j i} = 0 &  & \text{(by \cref{2.3.9})}      \\
    \implies & \forall i \in \set{1, \dots, n}, (A + A^2)_{i i} = 0.                         &  & \text{(from the proof above)}
  \end{align*}

  Next we show that there exists an \(i \in \set{1, \dots, n}\) such that each entry in the \(i\)-th row of \(A + A^2\) is positive except for the diagonal entry.
  Let \(v_1 \in \set{1, \dots, n}\).
  If each entry in the \(v_1\)-th row of \(A + A^2\) is positive except for the diagonal entry, then we are done.
  If not, then there exists a \(v_2 \in \set{1, \dots, n} \setminus \set{v_1}\) such that \((A + A^2)_{v_1 v_2} = 0\).
  From the proof above we see that \(A_{v_1 v_2} = 0\) and thus by \cref{2.3.11} we must have \(A_{v_2 v_1} = 1\).
  We claim that if \(k \in \set{1, \dots, n} \setminus \set{v_1, v_2}\) such that \(A_{v_1 k} = 1\), then \(A_{v_2 k} = 1\).
  Otherwise we would have
  \begin{align*}
             & A_{v_2 k} = 0                                               \\
    \implies & A_{k v_2} = 1            &  & \text{(by \cref{2.3.11})}     \\
    \implies & A_{v_1 k} A_{k v_2} = 1                                     \\
    \implies & (A + A^2)_{v_1 v_2} > 0, &  & \text{(from the proof above)}
  \end{align*}
  a contradiction.
  From the claim above we see that the \(v_2\)-th row of \(A + A^2\) has greater number of positive entries than the \(v_1\)-th row of \(A + A^2\) (at least larger than one since \(A_{v_2 v_1} = 1\)).
  Now we can ask if each entry in the \(v_2\)-th row of \(A + A^2\) is positive except for the diagonal entry.
  If yes, then we are done.
  If not, we can find a \(v_3 \in \set{1, \dots, n} \setminus \set{v_1, v_2}\) such that \((A + A^2)_{v_2 v_3} = 0\).
  Using similar arguments as above we see that the \(v_3\)-th row of \(A + A^2\) has greater number of positive entries than the \(v_2\)-th row of \(A + A^2\) (at least larger than one since \(A_{v_3 v_2} = 1\)).
  Since there are only finitely many rows in \(A + A^2\), we see that by continuing the above process we can find an \(i \in \set{1, \dots, n}\) such that each entry in the \(i\)-th row of \(A + A^2\) is positive except for the diagonal entry.

  Finally we show that if \(i\) is the person who dominates the most in \(A\), then each entry in the \(i\)-th row of \(A + A^2\) is positive except for the diagonal entry.
  Suppose for sake of contradiction that there exists a \(j \in \set{1, \dots, n} \setminus \set{i}\) such that \((A + A^2)_{i j} = 0\).
  This means
  \begin{align*}
             & A_{i j} + \sum_{k = 1}^n A_{i k} A_{k j} = 0                                                            \\
    \implies & \begin{dcases}
                 A_{i j} = 0 \\
                 \sum_{k = 1}^n A_{i k} A_{k j} = 0
               \end{dcases}                                       &  & \text{(by \cref{2.3.11})}                       \\
    \implies & \forall k \in \set{1, \dots, n},                                                                        \\
             & (A_{i k} = 0 \land A_{k j} = 1) \lor (A_{i k} = 1 \land A_{k j} = 0)     &  & \text{(by \cref{2.3.9})}  \\
    \implies & \forall k \in \set{1, \dots, n},                                                                        \\
             & (A_{i k} = 0 \land A_{k j} = 1 \land A_{k i} = 1 \land A_{j k} = 0)                                     \\
             & \lor (A_{i k} = 1 \land A_{k j} = 0 \land A_{k i} = 0 \land A_{j k} = 1) &  & \text{(by \cref{2.3.11})} \\
    \implies & \sum_{k = 1}^n A_{i k} = \sum_{k = 1}^n A_{j k} > 0.                     &  & \text{(by \cref{2.3.11})}
  \end{align*}
  But this contradicts to the fact that \(i\) is the person who dominates the most in \(A\).
  Thus such \(j\) does not exist and each entry in the \(i\)-th row of \(A + A^2\) is positive except for the diagonal entry.
  One can replace the role of \(i\)-th row with \(j\)-th column and derive similar arguments.
\end{proof}

\exercisesection

\setcounter{ex}{9}
\begin{ex}\label{ex:2.3.10}
  Let \(A \in \ms{n}{n}{\F}\).
  Prove that \(A\) is a diagonal matrix iff \(A_{i j} = \delta_{i j} A_{i j}\) for all \(i\) and \(j\).
\end{ex}

\begin{proof}[\pf{ex:2.3.10}]
  We have
  \begin{align*}
         & A \text{ is diagonal matrix}                                                                                              \\
    \iff & A_{i j} = 0 \text{ for all } i, j \in \set{1, 2, \dots, n} \text{ and } i \neq j            &  & \text{(by \cref{1.3.8})} \\
    \iff & A_{i j} = \delta_{i j} \text{ for all } i, j \in \set{1, 2, \dots, n} \text{ and } i \neq j &  & \text{(by \cref{2.3.4})} \\
    \iff & A_{i j} = \delta_{i j} A_{i j} \text{ for all } i, j \in \set{1, 2, \dots, n}.              &  & \text{(by \cref{2.3.4})}
  \end{align*}
\end{proof}

\begin{ex}\label{ex:2.3.11}
  Let \(\V\) be a vector space over \(\F\), and let \(\T : \V \to \V\) be linear.
  Prove that \(\T^2 = \zT\) iff \(\rg{\T} \subseteq \ns{\T}\).
\end{ex}

\begin{proof}[\pf{ex:2.3.11}]
  We have
  \begin{align*}
         & \T^2 = \zT                                                                \\
    \iff & \forall x \in \V, \T(\T(x)) = \zT(x) = \zv &  & \text{(by \cref{2.1.9})}  \\
    \iff & \forall x \in \V, \T(x) \in \ns{\T}        &  & \text{(by \cref{2.1.10})} \\
    \iff & \rg{\T} \subseteq \ns{\T}.                 &  & \text{(by \cref{2.1.10})}
  \end{align*}
\end{proof}

\begin{ex}\label{ex:2.3.12}
  Let \(\V\), \(\W\), and \(\vs{Z}\) be vector spaces over \(\F\), and let \(\T \in \ls(\V, \W)\) and \(\U \in \ls(\W, \vs{Z})\).
  \begin{enumerate}
    \item Prove that if \(\U \T\) is one-to-one, then \(\T\) is one-to-one.
          Must \(\U\) also be one-to-one?
    \item Prove that if \(\U \T\) is onto, then \(\U\) is onto.
          Must \(\T\) also be onto?
    \item Prove that if \(\U\) and \(\T\) are one-to-one and onto, then \(\U \T\) is also.
  \end{enumerate}
\end{ex}

\begin{proof}[\pf{ex:2.3.12}(a)]
  Let \(x, y \in \V\) such that \(x \neq y\).
  Then we have
  \begin{align*}
             & (\U \T)(x) \neq (\U \T)(y) &  & \text{(\(\U \T\) is one-to-one)}            \\
    \implies & \U(\T(x)) \neq \U(\T(y))                                                    \\
    \implies & \T(x) \neq \T(y)           &  & \text{(this is the definition of function)} \\
    \implies & \T \text{ is one-to-one}.
  \end{align*}
  From the proof above we see that is doesn't matter whether \(\U\) is one-to-one or not.
\end{proof}

\begin{proof}[\pf{ex:2.3.12}(b)]
  Let \(z \in \vs{Z}\).
  Then we have
  \begin{align*}
             & \exists x \in \V : (\U \T)(x) = z                &  & \text{(\(\U \T\) is onto)} \\
    \implies & \exists (x, y) \in \V \times \W : \begin{dcases}
                                                   \T(x) = y \\
                                                   \U(\T(x)) = \U(y) = z
                                                 \end{dcases} &  & \text{(\(\U \T\) is onto)}   \\
    \implies & \U \text{ is onto}.
  \end{align*}
  From the proof above we see that is doesn't matter whether \(\T\) is onto or not.
\end{proof}

\begin{proof}[\pf{ex:2.3.12}(c)]
  First we show that \(\U \T\) is one-to-one.
  Let \(x, y \in \V\) such that \(x \neq y\).
  Then we have
  \begin{align*}
             & \T(x) \neq \T(y)             &  & \text{(\(\T\) is one-to-one)} \\
    \implies & \U(\T(x)) \neq \U(\T(y))     &  & \text{(\(\U\) is one-to-one)} \\
    \implies & \U \T \text{ is one-to-one}.
  \end{align*}

  Now we show that \(\U \T\) is onto.
  This is true since
  \begin{align*}
             & \begin{dcases}
                 \forall y \in \W, \exists x \in \V : \T(x) = y \\
                 \forall z \in \vs{Z}, \exists y \in \W : \U(y) = z
               \end{dcases}     &  & \text{(\(\U, \T\) are onto)}     \\
    \implies & \forall z \in \vs{Z}, \exists x \in \V : \U(\T(x)) = z \\
    \implies & \U \T \text{ is onto}.
  \end{align*}
\end{proof}

\begin{ex}\label{ex:2.3.13}
  Let \(A, B \in \ms{n}{n}{\F}\).
  Prove that \(\tr[AB] = \tr[BA]\) and \(\tr[A] = \tr[\tp{A}]\).
\end{ex}

\begin{proof}[\pf{ex:2.3.13}]
  We have
  \begin{align*}
    \tr[AB] & = \sum_{i = 1}^n (AB)_{i i}                          &  & \text{(by \cref{1.3.9})} \\
            & = \sum_{i = 1}^n \pa{\sum_{k = 1}^n A_{i k} B_{k i}} &  & \text{(by \cref{2.3.1})} \\
            & = \sum_{i = 1}^n \pa{\sum_{k = 1}^n B_{k i} A_{i k}} &  & \text{(by \cref{1.2.1})} \\
            & = \sum_{k = 1}^n \pa{\sum_{i = 1}^n B_{k i} A_{i k}} &  & \text{(by \cref{1.2.1})} \\
            & = \sum_{k = 1}^n (BA)_{k k}                          &  & \text{(by \cref{2.3.1})} \\
            & = \tr[BA]                                            &  & \text{(by \cref{1.3.9})}
  \end{align*}
  and
  \begin{align*}
    \tr[A] & = \sum_{i = 1}^n A_{i i}        &  & \text{(by \cref{1.3.9})} \\
           & = \sum_{i = 1}^n (\tp{A})_{i i} &  & \text{(by \cref{1.3.3})} \\
           & = \tr[\tp{A}].                  &  & \text{(by \cref{1.3.9})}
  \end{align*}
\end{proof}

\begin{ex}\label{ex:2.3.14}
  Assume the notation in \cref{2.13}.
  \begin{enumerate}
    \item Suppose that \(z\) is a (column) vector in \(\vs{F}^p\).
          Use \cref{2.13}(b) to prove that \(Bz\) is a linear combination of the columns of \(B\).
          In particular, if \(z = \tp{\tuple{a}{1,2,,p}}\), then show that
          \[
            Bz = \sum_{j = 1}^p a_j v_j.
          \]
    \item Extend (a) to prove that column \(j\) of \(AB\) is a linear combination of the columns of \(A\) with the coefficients in the linear combination being the entries of column \(j\) of \(B\).
    \item For any row vector \(w \in \vs{F}^m\), prove that \(wA\) is a linear combination of the rows of \(A\) with the coefficients in the linear combination being the coordinates of \(w\).
    \item Prove the analogous result to (b) about rows:
          Row \(i\) of \(AB\) is a linear combination of the rows of \(B\) with the coefficients in the linear combination being the entries of row \(i\) of \(A\).
  \end{enumerate}
\end{ex}

\begin{proof}[\pf{ex:2.3.14}(a)]
  We have
  \begin{align*}
    Bz & = \begin{pmatrix}
             (Bz)_{1 1} \\
             (Bz)_{2 1} \\
             \vdots     \\
             (Bz)_{n 1}
           \end{pmatrix} = \begin{pmatrix}
                             \sum_{k = 1}^p B_{1 k} z_{k 1} \\
                             \sum_{k = 1}^p B_{2 k} z_{k 1} \\
                             \vdots                         \\
                             \sum_{k = 1}^p B_{n k} z_{k 1}
                           \end{pmatrix}                      &  & \text{(by \cref{2.3.1})}                   \\
       & = \begin{pmatrix}
             \sum_{k = 1}^p B_{1 k} z_k \\
             \sum_{k = 1}^p B_{2 k} z_k \\
             \vdots                     \\
             \sum_{k = 1}^p B_{n k} z_k
           \end{pmatrix} = \begin{pmatrix}
                             \sum_{k = 1}^p B_{1 k} a_k \\
                             \sum_{k = 1}^p B_{2 k} a_k \\
                             \vdots                     \\
                             \sum_{k = 1}^p B_{n k} a_k
                           \end{pmatrix}                                                         \\
       & = \sum_{k = 1}^p \begin{pmatrix}
                            B_{1 k} a_k \\
                            B_{2 k} a_k \\
                            \vdots      \\
                            B_{n k} a_k
                          \end{pmatrix} = \sum_{k = 1}^p \pa{a_k \begin{pmatrix}
                                                                     B_{1 k} \\
                                                                     B_{2 k} \\
                                                                     \vdots  \\
                                                                     B_{n k}
                                                                   \end{pmatrix}} &  & \text{(by \cref{1.2.9})} \\
       & = \sum_{k = 1}^p a_k v_k.                              &  & \text{(by \cref{2.13})}
  \end{align*}
\end{proof}

\begin{proof}[\pf{ex:2.3.14}(b)]
  We have
  \begin{align*}
    u_j & = A v_j = A \begin{pmatrix}
                        B_{1 j} \\
                        B_{2 j} \\
                        \vdots  \\
                        B_{n j}
                      \end{pmatrix}              &  & \text{(by \cref{2.13}(a))}       \\
        & = \sum_{k = 1}^n B_{k j} \begin{pmatrix}
                                     A_{1 k} \\
                                     A_{2 k} \\
                                     \vdots  \\
                                     A_{n k}
                                   \end{pmatrix}. &  & \text{(by \cref{ex:2.3.14}(a))}
  \end{align*}
\end{proof}

\begin{proof}[\pf{ex:2.3.14}(c)]
  We have
  \begin{align*}
    wA & = \begin{pmatrix}
             (wA)_1 & (wA)_2 & \cdots & (wA)_n
           \end{pmatrix}                                                                         &  & \text{(by \cref{2.3.1})}                    \\
       & = \begin{pmatrix}
             (wA)_{1 1} & (wA)_{1 2} & \cdots & (wA)_{1 n}
           \end{pmatrix}                                                             &  & \text{(by \cref{2.3.1})}                                \\
       & = \begin{pmatrix}
             \sum_{k = 1}^m w_{1 k} A_{k 1} & \sum_{k = 1}^m w_{1 k} A_{k 2} & \cdots & \sum_{k = 1}^m w_{1 k} A_{k n}
           \end{pmatrix} &  & \text{(by \cref{2.3.1})}                              \\
       & = \sum_{k = 1}^m \begin{pmatrix}
                            w_{1 k} A_{k 1} & w_{1 k} A_{k 2} & \cdots & w_{1 k} A_{k n}
                          \end{pmatrix}                                              &  & \text{(by \cref{1.2.9})}                                \\
       & = \sum_{k = 1}^m \begin{pmatrix}
                            w_k A_{k 1} & w_k A_{k 2} & \cdots & w_k A_{k n}
                          \end{pmatrix}                                                          &  & \text{(by \cref{2.3.1})}                    \\
       & = \sum_{k = 1}^m \pa{w_k \begin{pmatrix}
                                      A_{k 1} & A_{k 2} & \cdots & A_{k n}
                                    \end{pmatrix}}.                                                                   &  & \text{(by \cref{1.2.9})} \\
  \end{align*}
\end{proof}

\begin{proof}[\pf{ex:2.3.14}(d)]
  We have
  \begin{align*}
     & \begin{pmatrix}
         (AB)_{i 1} & (AB)_{i 2} & \cdots & (AB)_{i p}
       \end{pmatrix}                                                                                            \\
     & = \begin{pmatrix}
           \sum_{k = 1}^n A_{i k} B_{k 1} & \sum_{k = 1}^n A_{i k} B_{k 2} & \cdots & \sum_{k = 1}^n A_{i k} B_{k p}
         \end{pmatrix} &  & \text{(by \cref{2.3.1})}                              \\
     & = \sum_{k = 1}^n \begin{pmatrix}
                          A_{i k} B_{k 1} & A_{i k} B_{k 2} & \cdots & A_{i k} B_{k p}
                        \end{pmatrix}                                              &  & \text{(by \cref{1.2.9})}                                \\
     & = \sum_{k = 1}^n \pa{A_{i k} \begin{pmatrix}
                                        B_{k 1} & B_{k 2} & \cdots & B_{k p}
                                      \end{pmatrix}}.                                                               &  & \text{(by \cref{1.2.9})}
  \end{align*}
\end{proof}

\begin{ex}\label{ex:2.3.15}
  Let \(M\) and \(A\) be matrices for which the product matrix \(MA\) is defined.
  If the \(j\)th column of \(A\) is a linear combination of a set of columns of \(A\), prove that the \(j\)th column of \(MA\) is a linear combination of the corresponding columns of \(MA\) with the same corresponding coefficients.
\end{ex}

\begin{proof}[\pf{ex:2.3.15}]
  Let \(M \in \MS\) and let \(A \in \ms{n}{p}{\F}\).
  For all \(i \in \set{1, 2, \dots, p}\) we define \(v_i\) to be the \(i\)th column of \(A\).
  By hypothesis we know that
  \[
    \exists \seq{c}{1,2,,p} \in \F : v_j = \sum_{i = 1}^p c_i v_i.
  \]
  Then we have
  \begin{align*}
    \begin{pmatrix}
      (MA)_{1 j} \\
      (MA)_{2 j} \\
      \vdots     \\
      (MA)_{p j}
    \end{pmatrix} & = M v_j                              &  & \text{(by \cref{2.13}(a))}   \\
                    & = M \pa{\sum_{i = 1}^p c_i v_i}                                      \\
                    & = \sum_{i = 1}^p M (c_i v_i)         &  & \text{(by \cref{2.12}(a))} \\
                    & = \sum_{i = 1}^p c_i (M v_i)         &  & \text{(by \cref{2.12}(b))} \\
                    & = \sum_{i = 1}^p c_i \begin{pmatrix}
                                             (MA)_{1 i} \\
                                             (MA)_{2 i} \\
                                             \vdots     \\
                                             (MA)_{p i}
                                           \end{pmatrix}. &  & \text{(by \cref{2.13}(b))}
  \end{align*}
\end{proof}

\begin{ex}\label{ex:2.3.16}
  Let \(V\) be a finite-dimensional vector space over \(\F\), and let \(\T : \V \to \V\) be linear.
  \begin{enumerate}
    \item If \(\rk{\T} = \rk{\T^2}\), prove that \(\rg{\T} \cap \ns{\T} = \set{\zv}\).
          Deduce that \(\V = \rg{\T} \oplus \ns{\T}\).
    \item Prove that \(\V = \rg{\T^k} \oplus \ns{\T^k}\) for some positive integer \(k\).
  \end{enumerate}
\end{ex}

\begin{proof}[\pf{ex:2.3.16}(a)]
  First observe that
  \begin{align*}
             & \begin{dcases}
                 \T(\V) \subseteq \V \\
                 \rk{\T^2} = \rk{\T}
               \end{dcases}                                                                                             \\
    \implies & \begin{dcases}
                 \rg{\T^2} = \T^{2}(\V) = \T(\T(\V)) \subseteq \T(\V) = \rg{\T} \\
                 \rk{\T^2} = \rk{\T}
               \end{dcases} &  & \text{(by \cref{2.1.10})}                                 \\
    \implies & \rg{\T^2} = \rg{\T}.                                                               &  & \text{(by \cref{1.11})}
  \end{align*}
  Let \(\beta = \set{\seq{v}{1,2,,n}}\) be a basis for \(\rg{\T}\) over \(\F\).
  Since \(\rg{\T^2} = \rg{\T}\), by \cref{2.2} we know that \(\T(\beta)\) is a basis for \(\rg{\T}\) over \(\F\).
  Let \(x \in \rg{\T} \cap \ns{\T}\).
  Since \(x \in \rg{\T}\), by \cref{1.6.1} we know that
  \[
    \exists \seq{a}{1,2,,n} \in \F : x = \sum_{i = 1}^n a_i v_i.
  \]
  Since \(x \in \ns{\T}\), we have
  \begin{align*}
             & \T(x) = \zv                                                   &  & \text{(by \cref{2.1.10})}   \\
    \implies & \T(\sum_{i = 1}^n a_i v_i) = \sum_{i = 1}^n a_i \T(v_i) = \zv &  & \text{(by \cref{2.1.2}(d))} \\
    \implies & \seq[=]{a}{1,2,,n} = 0                                        &  & \text{(by \cref{1.5.3})}    \\
    \implies & x = \zv.                                                      &  & \text{(by \cref{1.2}(a))}
  \end{align*}
  Thus \(\rg{\T} \cap \ns{\T} = \zv\).
  Since
  \begin{align*}
     & \dim(\rg{\T} + \ns{\T})                                                                        \\
     & = \dim(\rg{\T}) + \dim(\ns{\T}) - \dim(\rg{\T} \cap \ns{\T}) &  & \text{(by \cref{ex:1.6.29})} \\
     & = \dim(\rg{\T}) + \dim(\ns{\T})                              &  & \text{(by \cref{1.6.9})}     \\
     & = \rk{\T} + \nt{\T} = \dim(\V)                               &  & \text{(by \cref{2.3})}
  \end{align*}
  and by \cref{ex:1.3.23}(a) \(\rg{\T} + \ns{\T}\) is a subspace of \(\V\) over \(\F\), by \cref{1.11} we know that \(\V = \rg{\T} + \ns{\T}\).
  Thus by \cref{1.3.11} we know that \(\V = \rg{\T} \oplus \ns{\T}\).
\end{proof}

\begin{proof}[\pf{ex:2.3.16}(b)]
  First observe that
  \begin{align*}
             & \forall k \in \Z^+, \rg{\T^{k + 1}} = \T^{k + 1}(\V) \subseteq \T^k(\V) = \rg{\T^k} &  & \text{(by \cref{2.1.10})}   \\
    \implies & \forall k \in \Z^+, 0 \leq \rk{\T^{k + 1}} \leq \rk{\T^k} \leq \dim(\V).            &  & \text{(by \cref{1.11,2.3})}
  \end{align*}
  Since \(\V\) is finite-dimensional, we know that there must exists a \(k \in \Z^+\) such that \(\rk{\T^{k + 1}} = \rk{\T^k}\).
  By \cref{ex:2.3.16}(a) we see that \(\rg{\T^k} \cap \ns{\T^k} = \set{\zv}\) and \(\V = \rg{\T^k} \oplus \ns{\T^k}\).
\end{proof}

\begin{ex}\label{ex:2.3.17}
  Let \(\V\) be a vector space over \(\F\).
  Determine all linear transformations \(\T : \V \to \V\) such that \(\T = \T^2\).
\end{ex}

\begin{proof}[\pf{ex:2.3.17}]
  First observe that
  \[
    \forall x \in \V, x = \T(x) + (x - \T(x)).
  \]
  Suppose that \(\T = \T^2\).
  Then we have
  \begin{align*}
             & \forall x \in \V, \T(x) = \T(\T(x))                                                                                              \\
    \implies & \forall x \in \V,                                                                                                                \\
             & \begin{dcases}
                 T(x) \in \set{y \in \V : \T(y) = y} \\
                 \T(x - \T(x)) = \T(x) - \T(\T(x)) = \T(x) - \T(x) = \zv
               \end{dcases}                                       \\
    \implies & \forall x \in \V, \begin{dcases}
                                   T(x) \in \set{y \in \V : \T(y) = y} \\
                                   x - \T(x) \in \ns{\T}
                                 \end{dcases}                                                      &  & \text{(by \cref{2.1.10})}               \\
    \implies & \V \subseteq \set{y \in \V : \T(y) = y} + \ns{\T}                                           &  & \text{(by \cref{1.3.10})}       \\
    \implies & \V = \set{y \in \V : \T(y) = y} + \ns{\T}.                                                  &  & \text{(by \cref{ex:1.3.23}(a))}
  \end{align*}
  Thus
  \begin{align*}
             & \forall x \in \set{y \in \V : \T(y) = y} \cap \ns{\T}, \begin{dcases}
                                                                        \T(x) = x \\
                                                                        \T(x) = \zv
                                                                      \end{dcases} &  & \text{(by \cref{2.1.10})}         \\
    \implies & \forall x \in \set{y \in \V : \T(y) = y} \cap \ns{\T}, x = \zv                                             \\
    \implies & \set{y \in \V : \T(y) = y} \cap \ns{\T} = \set{\zv}                                                        \\
    \implies & \V = \set{y \in \V : \T(y) = y} \oplus \ns{\T}.                       &  & \text{(by \cref{ex:2.3.16}(a))}
  \end{align*}
\end{proof}

\begin{ex}\label{ex:2.3.18}
  Using only the definition of matrix multiplication, prove that multipli- cation of matrices is associative.
\end{ex}

\begin{proof}[\pf{ex:2.3.18}]
  Let \(A \in \MS\), \(B \in \ms{n}{p}{\F}\) and \(C \in \ms{p}{q}{\F}\).
  Then we have
  \begin{align*}
    ((AB) C)_{i j} & = \sum_{k_2 = 1}^p (AB)_{i k_2} C_{k_2 j}                                     &  & \text{(by \cref{2.3.1})} \\
                   & = \sum_{k_2 = 1}^p \pa{\pa{\sum_{k_1 = 1}^n A_{i k_1} B_{k_1 k_2}} C_{k_2 j}} &  & \text{(by \cref{2.3.1})} \\
                   & = \sum_{k_2 = 1}^p \pa{\sum_{k_1 = 1}^n (A_{i k_1} B_{k_1 k_2}) C_{k_2 j}}    &  & \text{(by \cref{1.2.1})} \\
                   & = \sum_{k_2 = 1}^p \pa{\sum_{k_1 = 1}^n A_{i k_1} (B_{k_1 k_2} C_{k_2 j})}    &  & \text{(by \cref{1.2.1})} \\
                   & = \sum_{k_1 = 1}^n \pa{\sum_{k_2 = 1}^p A_{i k_1} (B_{k_1 k_2} C_{k_2 j})}    &  & \text{(by \cref{1.2.1})} \\
                   & = \sum_{k_1 = 1}^n \pa{A_{i k_1} \pa{\sum_{k_2 = 1}^p B_{k_1 k_2} C_{k_2 j}}} &  & \text{(by \cref{1.2.1})} \\
                   & = \sum_{k_1 = 1}^n A_{i k_1} (BC)_{k_1 j}                                     &  & \text{(by \cref{2.3.1})} \\
                   & = (A (BC))_{i j}                                                              &  & \text{(by \cref{2.3.1})}
  \end{align*}
  and thus by \cref{1.2.9} \((AB) C = A (BC)\).
\end{proof}

\setcounter{ex}{22}
\begin{ex}\label{ex:2.3.23}
  Let \(A \in \ms{n}{n}{\F}\) be an incidence matrix that corresponds to a dominance relation.
  Determine the number of nonzero entries of \(A\).
\end{ex}

\begin{proof}[\pf{ex:2.3.23}]
  By \cref{2.3.11} we see that the number of nonzero entries is
  \[
    \frac{n (n + 1)}{2} - n = \frac{n (n - 1)}{2}.
  \]
\end{proof}

\section{Invertibility and Isomorphisms}\label{sec:2.4}

\begin{defn}\label{2.4.1}
  Let \(\V\) and \(\W\) be vector spaces over \(\F\), and let \(\T : \V \to \W\) be linear.
  A function \(\U : \W \to \V\) is said to be an \textbf{inverse} of \(\T\) if \(\T \U = \IT[\W]\) and \(\U \T = \IT[\V]\).
  If \(\T\) has an inverse, then \(\T\) is said to be \textbf{invertible}.
  If \(\T\) is invertible, then the inverse of \(\T\) is unique and is denoted by \(\T^{-1}\).
  The following facts hold for invertible functions T and U.
  \begin{itemize}
    \item \((\T \U)^{-1} = \U^{-1} \T^{-1}\).
    \item \((\T^{-1})^{-1} = \T\);
          in particular, \(\T^{-1}\) is invertible.
  \end{itemize}
  We often use the fact that a function is invertible iff it is both one-to-one and onto.
  We can therefore restate \cref{2.5} as follows.
  \begin{itemize}
    \item Let \(\T : \V \to \W\) be a linear transformation, where \(\V\) and \(\W\) are finite-dimensional spaces over \(\F\) of equal dimension.
          Then \(\T\) is invertible if and only if \(\rk{\T} = \dim(\V)\).
  \end{itemize}
\end{defn}

\begin{thm}\label{2.17}
  Let \(\V\) and \(\W\) be vector spaces over \(\F\), and let \(\T : \V \to \W\) be linear and invertible.
  Then \(\T^{-1} : \W \to \V\) is linear.
\end{thm}

\begin{proof}[\pf{2.17}]
  Let \(y_1, y_2 \in \W\) and \(c \in \F\).
  Since \(\T\) is onto and one-to-one, there exist unique vectors \(x_1\) and \(x_2\) such that \(\T(x_1) = y_1\) and \(\T(x_2) = y_2\).
  Thus \(x_1 = \T^{-1}(y_1)\) and \(x_2 = \T^{-1}(y_2)\);
  so
  \begin{align*}
    \T^{-1}(c y_1 + y_2) & = \T^{-1}(c \T(x_1) + \T(x_2))   &  & \text{(by \cref{2.4.1})}    \\
                         & = \T^{-1}(\T(c x_1 + x_2))       &  & \text{(by \cref{2.1.2}(b))} \\
                         & = c x_1 + x_2                    &  & \text{(by \cref{2.4.1})}    \\
                         & = c \T^{-1}(x_1) + \T^{-1}(x_2). &  & \text{(by \cref{2.4.1})}
  \end{align*}
\end{proof}

\begin{cor}\label{2.4.2}
  If \(\T\) is a linear transformation between vector spaces of equal (finite) dimension, then the conditions of being invertible, one-to-one, and onto are all equivalent.
\end{cor}

\begin{proof}[\pf{2.4.2}]
  By \cref{2.5} we see that this is true.
\end{proof}

\begin{defn}\label{2.4.3}
  Let \(A \in \ms{n}{n}{\F}\).
  Then \(A\) is \textbf{invertible} if there exists an \(B \in \ms{n}{n}{\F}\) such that \(AB = BA = I_n\).
\end{defn}

\begin{cor}\label{2.4.4}
  If \(A\) is invertible, then the matrix \(B\) such that \(AB = BA = I\) is unique.
  The matrix \(B\) is called the \textbf{inverse} of \(A\) and is denoted by \(A^{-1}\).
\end{cor}

\begin{proof}[\pf{2.4.4}]
  If \(C\) were another such matrix, then
  \[
    C = CI = C(AB) = (CA)B = IB = B.
  \]
  Thus \(B\) is unique.
\end{proof}

\begin{lem}\label{2.4.5}
  Let \(\T\) be an invertible linear transformation from \(\V\) to \(\W\).
  Then \(\V\) is finite-dimensional iff \(\W\) is finite-dimensional.
  In this case, \(\dim(\V) = \dim(\W)\).
\end{lem}

\begin{proof}[\pf{2.4.5}]
  Suppose that \(\V\) is finite-dimensional.
  Let \(\beta = \set{\seq{x}{1,,n}}\) be a basis for \(\V\) over \(\F\).
  By \cref{2.2} \(\T(\beta)\) spans \(\rg{\T} = \W\);
  hence \(\W\) is finite-dimensional by \cref{1.9}.
  Conversely, if \(\W\) is finite-dimensional, then so is \(\V\) by a similar argument, using \(\T^{-1}\).

  Now suppose that \(\V\) and \(\W\) are finite-dimensional.
  Because \(\T\) is one-to-one and onto, we have
  \[
    \nt{\T} = 0 \quad \text{and} \quad \rk{\T} = \dim(\rg{\T}) = \dim(\W).
  \]
  So by the dimension theorem (\cref{2.3}), it follows that \(\dim(\V) = \dim(\W)\).
\end{proof}

\begin{thm}\label{2.18}
  Let \(\V\) and \(\W\) be finite-dimensional vector spaces over \(\F\) with ordered bases \(\beta\) and \(\gamma\) over \(\F\), respectively.
  Let \(\T : \V \to \W\) be linear.
  Then \(\T\) is invertible iff \([\T]_{\beta}^{\gamma}\) is invertible.
  Furthermore, \([\T^{-1}]_{\gamma}^{\beta} = ([\T]_{\beta}^{\gamma})^{-1}\).
\end{thm}

\begin{proof}[\pf{2.18}]
  Suppose that \(\T\) is invertible.
  By \cref{2.4.5}, we have \(\dim(\V) = \dim(\W)\).
  Let \(n = \dim(\V)\).
  So \([\T]_{\beta}^{\gamma} \in \ms{n}{n}{\F}\).
  Now \(\T^{-1} : \W \to \V\) satisfies \(\T \T^{-1} = \IT[\W]\) and \(\T^{-1} \T = \IT[\V]\).
  Thus
  \[
    I_n = [\IT[\V]]_{\beta} = [\T^{-1} \T]_{\beta} = [\T^{-1}]_{\gamma}^{\beta} [\T]_{\beta}^{\gamma}.
  \]
  Similarly, \([\T]_{\beta}^{\gamma} [\T^{-1}]_{\gamma}^{\beta} = I_n\).
  So \([\T]_{\beta}^{\gamma}\) is invertible and \(\pa{[\T]_{\beta}^{\gamma}}^{-1} = [\T^{-1}]_{\gamma}^{\beta}\).

  Now suppose that \(A = [\T]_{\beta}^{\gamma}\) is invertible.
  Then there exists an \(B \in \ms{n}{n}{\F}\) such that \(AB = BA = I_n\).
  By \cref{2.6} there exists \(U \in \ls(\W, \V)\) such that
  \[
    \U(w_j) = \sum_{i = 1}^n B_{i j} v_i \quad \text{for } j \in \set{1, \dots, n},
  \]
  where \(\gamma = \set{\seq{w}{1,,n}}\) and \(\beta = \set{\seq{v}{1,,n}}\).
  It follows that \([\U]_{\gamma}^{\beta} = B\).
  To show that \(\U = \T^{-1}\), observe that
  \[
    [\U \T]_{\beta} = [\U]_{\gamma}^{\beta} [\T]_{\beta}^{\gamma} = BA = I_n = [\IT[\V]]_{\beta}
  \]
  by \cref{2.11}.
  So \(\U \T = \IT[\V]\), and similarly, \(\T \U = \IT[\W]\).
\end{proof}

\begin{cor}\label{2.4.6}
  Let \(\V\) be a finite-dimensional vector space over \(\F\) with an ordered basis \(\beta\), and let \(\T : \V \to \V\) be linear.
  Then \(\T\) is invertible iff \([\T]_{\beta}\) is invertible.
  Furthermore, \([\T^{-1}]_{\beta} = ([\T]_{\beta})^{-1}\).
\end{cor}

\begin{proof}[\pf{2.4.6}]
  This is done by \cref{2.18}.
\end{proof}

\begin{cor}\label{2.4.7}
  Let \(A \in \ms{n}{n}{\F}\).
  Then \(A\) is invertible iff \(\L_A\) is invertible.
  Furthermore, \((\L_A)^{-1} = \L_{A^{-1}}\).
\end{cor}

\begin{proof}[\pf{2.4.7}]
  By \cref{2.15} we know that \(\L_A\) is linear and \([\L_A]_{\beta} = A\) where \(\beta\) is the standard ordered basis for \(\vs{F}^n\) over \(\F\).
  Thus by \cref{2.4.7} \(A\) is invertible iff \([\L_A]_{\beta}\) is invertible iff \(\L_A\) is invertible.
  And we have \([\L_A^{-1}]_{\beta} = ([\L_A]_{\beta})^{-1} = A^{-1} = [\L_{A^{-1}}]_{\beta}\).
  By \cref{2.1.13} this means \((\L_A)^{-1} = \L_{A^{-1}}\).
\end{proof}

\begin{defn}\label{2.4.8}
  Let \(\V\) and \(\W\) be vector spaces over \(\F\).
  We say that \(\V\) is \textbf{isomorphic} to \(\W\) if there exists a linear transformation \(\T : \V \to \W\) that is invertible.
  Such a linear transformation is called an \textbf{isomorphism} from \(\V\) onto \(\W\).
\end{defn}

\begin{note}
  Since ``is isomorphic to'' is an equivalence relation.
  So we need only say that \(\V\) and \(\W\) are isomorphic.
\end{note}

\begin{thm}\label{2.19}
  Let \(\V\) and \(\W\) be finite-dimensional vector spaces (over the same field).
  Then \(\V\) is isomorphic to \(\W\) iff \(\dim(\V) = \dim(\W)\).
\end{thm}

\begin{proof}[\pf{2.19}]
  Suppose that \(\V\) is isomorphic to \(\W\) and that \(\T : \V \to \W\) is an isomorphism from \(\V\) to \(\W\).
  By \cref{2.4.5} we have that \(\dim(\V) = \dim(\W)\).

  Now suppose that \(\dim(\V) = \dim(\W)\), and let \(\beta = \set{\seq{v}{1,,n}}\) and \(\gamma = \set{\seq{w}{1,,n}}\) be bases for \(\V\) and \(\W\) over \(\F\), respectively.
  By \cref{2.6} there exists \(\T : \V \to \W\) such that \(\T\) is linear and \(\T(v_i) = w_i\) for \(i = 1, 2, \dots, n\).
  Using \cref{2.2} we have
  \[
    \rg{\T} = \spn{\T(\beta)} = \spn{\gamma} = \W.
  \]
  So \(\T\) is onto.
  From \cref{2.5} we have that \(\T\) is also one-to-one.
  Hence \(\T\) is an isomorphism.
\end{proof}

\begin{note}
  By \cref{2.4.5} if \(\V\) and \(\W\) are isomorphic, then either both of \(\V\) and \(\W\) are finite-dimensional or both are infinite-dimensional.
\end{note}

\begin{cor}\label{2.4.9}
  Let \(\V\) be a vector space over \(\F\).
  Then \(\V\) is isomorphic to \(\vs{F}^n\) iff \(\dim(\V) = n\).
\end{cor}

\begin{proof}[\pf{2.4.9}]
  We have
  \begin{align*}
         & \V \text{ is isomorphic to } \vs{F}^n &  & \text{(by \cref{2.19})}   \\
    \iff & \dim(\V) = \dim(\vs{F}^n) = n.        &  & \text{(by \cref{1.6.10})}
  \end{align*}
\end{proof}

\begin{thm}\label{2.20}
  Let \(\V\) and \(\W\) be finite-dimensional vector spaces over \(\F\) of dimensions \(n\) and \(m\), respectively, and let \(\beta\) and \(\gamma\) be ordered bases for \(\V\) and \(\W\) over \(\F\), respectively.
  Then the function \(\Phi : \ls(\V, \W) \to \MS\), defined by \(\Phi(\T) = [\T]_{\beta}^{\gamma}\) for \(\T \in \ls(\V, \W)\), is an isomorphism.
\end{thm}

\begin{proof}[\pf{2.20}]
  By \cref{2.8} \(\Phi\) is linear.
  Hence we must show that \(\Phi\) is one-to-one and onto.
  This is accomplished if we show that for every \(A \in \MS\), there exists a unique linear transformation \(\T : \V \to \W\) such that \(\Phi(\T) = A\).
  Let \(\beta = \set{\seq{v}{1,,n}}\), \(\gamma = \set{\seq{w}{1,,m}}\), and let \(A \in \MS\).
  By \cref{2.6} there exists a unique linear transformation \(\T : \V \to \W\) such that
  \[
    \T(v_j) = \sum_{i = 1}^m A_{i j} w_i \quad \text{for } 1 \leq j \leq n.
  \]
  But this means that \([\T]_{\beta}^{\gamma} = A\), or \(\Phi(\T) = A\).
  Thus \(\Phi\) is an isomorphism.
\end{proof}

\begin{cor}\label{2.4.10}
  Let \(\V\) and \(\W\) be finite-dimensional vector spaces over \(\F\) of dimensions \(n\) and \(m\), respectively.
  Then \(\ls(\V, \W)\) is finite-dimensional of dimension \(mn\).
\end{cor}

\begin{proof}[\pf{2.4.10}]
  The proof follows from \cref{2.20} and \cref{2.19} and the fact that \(\dim(\MS) = mn\).
\end{proof}

\section{The Change of Coordinate Matrix}\label{sec:2.5}

\begin{note}
	Geometrically, the change of variable
	\[
		\begin{pmatrix}
			x \\
			y
		\end{pmatrix} \to \begin{pmatrix}
			x' \\
			y'
		\end{pmatrix}
	\]
	is a change in the way that the position of a point \(P\) in the plane is described.
	This is done by introducing a new frame of reference, an \(x' y'\)-coordinate system with coordinate axes rotated from the original \(xy\)-coordinate axes.
\end{note}

\begin{thm}\label{2.22}
	Let \(\beta\) and \(\beta'\) be two ordered bases over \(\F\) for a finite-dimensional vector space \(\V\) over \(\F\), and let \(Q = [\IT[\V]]_{\beta'}^{\beta}\).
	Then
	\begin{enumerate}
		\item \(Q\) is invertible.
		\item For any \(v \in \V\), \([v]_{\beta} = Q [v]_{\beta'}\).
	\end{enumerate}
\end{thm}

\begin{proof}[\pf{2.22}(a)]
	Since \(\IT[\V]\) is invertible, \(Q\) is invertible by \cref{2.18}.
\end{proof}

\begin{proof}[\pf{2.22}(b)]
	For any \(v \in \V\),
	\[
		[v]_{\beta} = [\IT[\V](v)]_{\beta} = [\IT[\V]]_{\beta'}^{\beta} [v]_{\beta'} = Q [v]_{\beta'}
	\]
	by \cref{2.14}.
\end{proof}

\begin{defn}\label{2.5.1}
	The matrix \(Q = [\IT[\V]]_{\beta'}^{\beta}\) defined in \cref{2.22} is called a \textbf{change of coordinate matrix}.
	Because of \cref{2.22}(b), we say that \(Q\) \textbf{changes \(\beta'\)-coordinates into \(\beta\)-coordinates}.
	Observe that if \(\beta = \set{\seq{x}{1,2,,n}}\) and \(\beta' = \set{x_1', x_2', \dots, x_n'}\), then
	\[
		x_j' = \sum_{i = 1}^n Q_{i j} x_i
	\]
	for \(j \in \set{1, 2, \dots, n}\);
	that is, the \(j\)th column of \(Q\) is \([x_j']_{\beta}\).
\end{defn}

\begin{note}
	If \(Q\) changes \(\beta'\)-coordinates into \(\beta\)-coordinates, then \(Q^{-1}\) changes \(\beta\)-coordinates into \(\beta'\)-coordinates
	(See \cref{ex:2.5.11}).
\end{note}

\begin{defn}\label{2.5.2}
	For the remainder of this section, we consider only linear transformations that map a vector space \(\V\) over \(\F\) into itself.
	Such a linear transformation is called a \textbf{linear operator} on \(\V\) over \(\F\).
	Suppose now that \(\T\) is a linear operator on a finite-dimensional vector space \(\V\) over \(\F\) and that \(\beta\) and \(\beta'\) are ordered bases for \(\V\) over \(\F\).
	Then \(\V\) can be represented by the matrices \([\T]_{\beta}\) and \([\T]_{\beta'}\)
	(See \cref{2.2.4}).
\end{defn}

\begin{thm}\label{2.23}
	Let \(\T\) be a linear operator on a finite-dimensional vector space \(\V\) over \(\F\), and let \(\beta\) and \(\beta'\) be ordered bases for \(\V\) over \(\F\).
	Suppose that \(Q\) is the change of coordinate matrix that changes \(\beta'\)-coordinates into \(\beta\)-coordinates.
	Then
	\[
		[\T]_{\beta'} = Q^{-1} [\T]_{\beta} Q.
	\]
\end{thm}

\begin{proof}[\pf{2.23}]
	Let \(\IT[\V]\) be the identity transformation on \(\V\).
	Then \(\T = \IT[\V] \T = \T \IT[\V]\);
	hence, by \cref{2.11},
	\[
		Q [\T]_{\beta'} = [\IT[\V]]_{\beta'}^{\beta} [\T]_{\beta'}^{\beta'} = [\IT[\V] \T]_{\beta'}^{\beta} = [\T \IT[\V]]_{\beta'}^{\beta} = [\T]_{\beta}^{\beta} [\IT[\V]]_{\beta'}^{\beta} = [\T] Q.
	\]
	Therefore \([\T]_{\beta'} = Q^{-1} [\T]_{\beta} Q\).
\end{proof}

\begin{cor}\label{2.5.3}
	Let \(A \in \ms[n][n][\F]\), and let \(\gamma\) be an ordered basis for \(\vs{F}^n\) over \(\F\).
	Then \([\L_A]_{\gamma} = Q^{-1} A Q\), where \(Q \in \ms[n][n][\F]\) and the \(j\)th column of \(Q\) is the \(j\)th vector of \(\gamma\).
\end{cor}

\begin{proof}[\pf{2.5.3}]
	Let \(\gamma = \set{\seq{v}{1,,n}}\) and let \(\beta = \set{\seq{e}{1,,n}}\) be the standard ordered basis for \(\vs{F}^n\) over \(\F\).
	Then we have
	\begin{align*}
		         & \forall j \in \set{1, \dots, n}, v_j = \begin{pmatrix}
			                                                  Q_{1 j} \\
			                                                  \vdots  \\
			                                                  Q_{n j}
		                                                  \end{pmatrix} = \sum_{i = 1}^n Q_{i j} e_i &  & \by{1.6.3} \\
		\implies & Q = [\IT_{\vs{F}^n}]_{\gamma}^{\beta}                  &  & \by{2.5.1}                            \\
		\implies & [\L_A]_{\gamma} = Q^{-1} [\L_A]_{\beta} Q              &  & \by{2.23}                             \\
		         & = Q^{-1} A Q.                                          &  & \by{2.15}[a]
	\end{align*}
\end{proof}

\begin{defn}\label{2.5.4}
	Let \(A, B \in \ms[n][n][\F]\).
	We say that \(B\) is \textbf{similar} to \(A\) if there exists an invertible matrix \(Q\) such that \(B = Q^{-1} A Q\).
\end{defn}

\begin{note}
	Observe that the relation of similarity is an equivalence relation
	(see \cref{ex:2.5.9}).
	So we need only say that \(A\) and \(B\) are similar.
\end{note}

\begin{note}
	In term of \cref{2.5.4}, \cref{2.23} can be stated as follows:
	If \(\T\) is a linear operator on a finite-dimensional vector space \(\V\) over \(\F\), and if \(\beta\) and \(\beta'\) are any ordered bases for \(\V\) over \(\F\), then \([\T]_{\beta}\) and \([\T]_{\beta'}\) are similar.
\end{note}

\begin{note}
	\cref{2.23} can be generalized to allow \(\T \in \ls(\V, \W)\), where \(\V, \W\) are vector spaces over \(\F\) and \(\V\) is distinct from \(\W\).
	In this case, we can change bases in \(\V\) as well as in \(\W\)
	(see \cref{ex:2.5.8}).
\end{note}

\exercisesection

\setcounter{ex}{7}
\begin{ex}\label{ex:2.5.8}
	Prove the following generalization of \cref{2.23}.
	Let \(\T \in \ls(\V, \W)\) where \(\V, \W\) are finite-dimensional vector spaces over \(\F\).
	Let \(\beta\) and \(\beta'\) be ordered bases for \(\V\) over \(\F\), and let \(\gamma\) and \(\gamma'\) be ordered bases for \(\W\) over \(\F\).
	Then \([\T]_{\beta'}^{\gamma'} = P^{-1} [\T]_{\beta}^{\gamma} Q\), where \(Q\) is the matrix that changes \(\beta'\)-coordinates into \(\beta\)-coordinates and \(P\) is the matrix that changes \(\gamma'\)-coordinates into \(\gamma\)-coordinates.
\end{ex}

\begin{proof}[\pf{ex:2.5.8}]
	We have
	\begin{align*}
		P [\T]_{\beta'}^{\gamma'} & = [\IT[\W]]_{\gamma'}^{\gamma} [\T]_{\beta'}^{\gamma'} &  & \by{2.5.1} \\
		                          & = [\IT[\W] \T]_{\beta'}^{\gamma}                       &  & \by{2.11}  \\
		                          & = [\T]_{\beta'}^{\gamma}                               &  & \by{2.1.9} \\
		                          & = [\T \IT[\V]]_{\beta'}^{\gamma}                       &  & \by{2.1.9} \\
		                          & = [\T]_{\beta}^{\gamma} [\IT[\V]]_{\beta'}^{\beta}     &  & \by{2.11}  \\
		                          & = [\T]_{\beta}^{\gamma} Q                              &  & \by{2.5.1}
	\end{align*}
	and thus
	\begin{align*}
		         & P [\T]_{\beta'}^{\gamma'} = [\T]_{\beta}^{\gamma} Q                                 \\
		\implies & P^{-1} P [\T]_{\beta'}^{\gamma'} = P^{-1} [\T]_{\beta}^{\gamma} Q &  & \by{2.22}[a] \\
		\implies & [\T]_{\beta'}^{\gamma'} = P^{-1} [\T]_{\beta}^{\gamma} Q.         &  & \by{2.4.3}
	\end{align*}
\end{proof}

\begin{ex}\label{ex:2.5.9}
	Prove that ``is similar to'' is an equivalence relation on \(\ms[n][n][\F]\).
\end{ex}

\begin{proof}[\pf{ex:2.5.9}]
	Let \(A, B, C \in \ms[n][n][\F]\).
	\begin{description}
		\item[For reflexive:]
			We have
			\[
				A = I_n A I_n = I_n^{-1} A I_n.
			\]
		\item[For symmetric:]
			Suppose that \(B = Q^{-1} A Q\).
			Let \(P = Q^{-1}\).
			Then we have
			\begin{align*}
				     & B = Q^{-1} A Q                                                   \\
				\iff & Q B Q^{-1} = Q Q^{-1} A Q Q^{-1} = I_n A I_n = A &  & \by{2.4.3} \\
				\iff & P^{-1} B P = A.                                  &  & \by{2.4.3}
			\end{align*}
		\item[For transitive:]
			Suppose that \(B = Q^{-1} A Q\) and \(C = P^{-1} B P\).
			Then we have
			\begin{align*}
				C & = P^{-1} B P                             \\
				  & = P^{-1} Q^{-1} A Q P                    \\
				  & = (QP)^{-1} A (QP).   &  & \by{ex:2.4.4}
			\end{align*}
	\end{description}
	From all cases above we see that ``is similar to'' defined in \cref{2.5.4} is an equivalence relation on \(\ms[n][n][\F]\).
\end{proof}

\begin{ex}\label{ex:2.5.10}
	Prove that if \(A, B \in \ms[n][n][\F]\) and \(A, B\) are similar, then \(\tr[A] = \tr[B]\).
\end{ex}

\begin{proof}[\pf{ex:2.5.10}]
	By \cref{2.5.4} we know that there exists a \(Q \in \ms[n][n][\F]\) such that \(B = Q^{-1} A Q\).
	Thus we have
	\begin{align*}
		\tr[B] & = \tr[Q^{-1} A Q]                       \\
		       & = \tr[(Q^{-1} A) Q] &  & \by{2.16}      \\
		       & = \tr[Q (Q^{-1} A)] &  & \by{ex:2.3.13} \\
		       & = \tr[(Q Q^{-1}) A] &  & \by{2.16}      \\
		       & = \tr[I_n A]        &  & \by{2.4.3}     \\
		       & = \tr[A].           &  & \by{2.4.3}
	\end{align*}
\end{proof}

\begin{ex}\label{ex:2.5.11}
	Let \(\V\) be a finite-dimensional vector space over \(\F\) with ordered bases \(\alpha\), \(\beta\), and \(\gamma\).
	\begin{enumerate}
		\item Prove that if \(Q\) and \(R\) are the change of coordinate matrices that change \(\alpha\)-coordinates into \(\beta\)-coordinates and \(\beta\)-coordinates into \(\gamma\)-coordinates, respectively, then \(RQ\) is the change of coordinate matrix that changes \(\alpha\)-coordinates into \(\gamma\)-coordinates.
		\item Prove that if \(Q\) changes \(\alpha\)-coordinates into \(\beta\)-coordinates, then \(Q^{-1}\) changes \(\beta\)-coordinates into \(\alpha\)-coordinates.
	\end{enumerate}
\end{ex}

\begin{proof}[\pf{ex:2.5.11}(a)]
	We have
	\begin{align*}
		QR & = [\IT[\V]]_{\alpha}^{\beta} [\IT[\V]]_{\beta}^{\gamma} &  & \by{2.5.1} \\
		   & = [\IT[\V] \IT[\V]]_{\alpha}^{\gamma}                   &  & \by{2.11}  \\
		   & = [\IT[\V]]_{\alpha}^{\gamma}                           &  & \by{2.1.9}
	\end{align*}
	and thus by \cref{2.5.1} \(QR\) is the change of coordinate matrix that changes \(\alpha\)-coordinates into \(\gamma\)-coordinates.
\end{proof}

\begin{proof}[\pf{ex:2.5.11}(b)]
	We have
	\begin{align*}
		Q^{-1} & = \pa{[\IT[\V]]_{\alpha}^{\beta}}^{-1} &  & \by{2.5.1} \\
		       & = [\IT[\V]^{-1}]_{\beta}^{\alpha}      &  & \by{2.18}  \\
		       & = [\IT[\V]]_{\beta}^{\alpha}.          &  & \by{2.1.9}
	\end{align*}
\end{proof}

\setcounter{ex}{12}
\begin{ex}\label{ex:2.5.13}
	Let \(\V\) be a finite-dimensional vector space over a field \(\F\), and let \(\beta = \set{\seq{x}{1,2,,n}}\) be an ordered basis for \(\V\) over \(\F\).
	Let \(Q \in \ms[n][n][\F]\) and \(Q\) is invertible.
	Define
	\[
		x_j' = \sum_{i = 1}^n Q_{i j} x_i \quad \text{for } 1 \leq j \leq n,
	\]
	and set \(\beta' = \set{x_1', x_2', \dots, x_n'}\).
	Prove that \(\beta'\) is a basis for \(\V\) over \(\F\) and hence that \(Q\) is the change of coordinate matrix changing \(\beta'\)-coordinates into \(\beta\)-coordinates.
\end{ex}

\begin{proof}[\pf{ex:2.5.13}]
	Let \(a = \tp{\tuple{a}{1,,n}} \in \vs{F}^n\) such that
	\[
		\sum_{j = 1}^n a_j x_j' = \zv.
	\]
	Since
	\begin{align*}
		         & \sum_{j = 1}^n a_j x_j' = \sum_{j = 1}^n \pa{a_j \sum_{i = 1}^n Q_{i j} x_i}                                                             \\
		         & = \sum_{j = 1}^n \pa{\sum_{i = 1}^n a_j Q_{i j} x_i} = \sum_{i = 1}^n \pa{\sum_{j = 1}^n a_j Q_{i j}} x_i = \zv &  & \by{1.2.1}          \\
		\implies & \forall i \in \set{1, \dots, n}, \sum_{j = 1}^n a_j Q_{i j} = 0                                                 &  & \by{1.6.1}          \\
		\implies & Qa = \begin{pmatrix}
			                Q_{1 1} & \cdots & Q_{1 n} \\
			                \vdots  & \ddots & \vdots  \\
			                Q_{n 1} & \cdots & Q_{n n}
		                \end{pmatrix} \cdot \begin{pmatrix}
			                                    a_1    \\
			                                    \vdots \\
			                                    a_n
		                                    \end{pmatrix} = \zm                                                                             &  & \by{2.3.1} \\
		\implies & a = \zv,                                                                                                        &  & \by{ex:2.4.6}
	\end{align*}
	by \cref{1.5.3} we know that \(\beta'\) is linearly independent.
	Thus by \cref{1.6.15}(b) \(\beta'\) is a basis for \(\V\) over \(\F\).
	We conclude by \cref{2.5.1} that \(Q\) is the change of coordinate matrix changin \(\beta'\)-coordinates into \(\beta\)-coordinates.
\end{proof}

\begin{ex}\label{ex:2.5.14}
	Prove the converse of \cref{ex:2.5.8}:
	If \(A, B \in \ms\), and if there exist invertible matrices \(P \in \ms[m][m][\F]\) and \(Q \in \ms[n][n][\F]\) such that \(B = P^{-1} A Q\), then there exist an \(n\)-dimensional vector space \(\V\) and an \(m\)-dimensional vector space \(\W\) (both over \(\F\)), ordered bases \(\beta, \beta'\) for \(\V\) and \(\gamma, \gamma'\) for \(\W\) (all over \(\F\)), and a \(\T \in \ls(\V, \W)\) such that
	\[
		A = [\T]_{\beta}^{\gamma} \quad \text{and} \quad B = [\T]_{\beta'}^{\gamma'}.
	\]
\end{ex}

\begin{proof}[\pf{ex:2.5.14}]
	Let \(\beta\) and \(\gamma\) be the standard ordered bases for \(\vs{F}^n\) and \(\vs{F}^m\) over \(\F\), respectively.
	We define \(\beta'\) and \(\gamma'\) as follow:
	\begin{align*}
		 & \beta' = \set{\sum_{i = 1}^n \pa{Q^{-1}}_{i j} e_i : (j \in \set{1, \dots, n}) \land (e_i \in \beta)};   \\
		 & \gamma' = \set{\sum_{i = 1}^n \pa{P}^{-1}_{i j} e_i : (j \in \set{1, \dots, m}) \land (e_i \in \gamma)}.
	\end{align*}
	By \cref{ex:2.5.13} we see that \(\beta'\) and \(\gamma'\) are bases of \(\vs{F}^n\) and \(\vs{F}^m\) over \(\F\), respectively.
	By \cref{2.5.1} we see that \(Q^{-1}\) changes \(\beta\)-coordinates into \(\beta'\)-coordinates and \(P^{-1}\) changes \(\gamma\)-coordinates into \(\gamma'\)-coordinates.
	By \cref{ex:2.5.11}(b) we see that \(Q\) changes \(\beta'\)-coordinates into \(\beta\)-coordinates.
	If we set \(\V = \vs{F}^n\), \(\W = \vs{F}^m\) and \(\T = \L_A\), then we have
	\begin{align*}
		B & = P^{-1} A Q                                                                                  &  & \text{(by hypothesis)}         \\
		  & = [\IT[\vs{F}^m]]_{\gamma}^{\gamma'} A [\IT[\vs{F}^n]]_{\beta'}^{\beta}                       &  & \text{(from the proofs above)} \\
		  & = [\IT[\vs{F}^m]]_{\gamma}^{\gamma'} [\L_A]_{\beta}^{\gamma} [\IT[\vs{F}^n]]_{\beta'}^{\beta} &  & \by{2.15}[a]                   \\
		  & = [\IT[\vs{F}^m]]_{\gamma}^{\gamma'} [\T]_{\beta}^{\gamma} [\IT[\vs{F}^n]]_{\beta'}^{\beta}                                       \\
		  & = [\IT[\vs{F}^m] \T \IT[\vs{F}^n]]_{\beta'}^{\gamma'}                                         &  & \by{2.11}                      \\
		  & = [\T]_{\beta'}^{\gamma'}.                                                                    &  & \by{2.1.9}
	\end{align*}
\end{proof}

\section{Dual Spaces}\label{sec:2.6}

\begin{defn}\label{2.6.1}
  In this section, we are concerned exclusively with linear transformations from a vector space \(\V\) into its field of scalars \(\F\), which is itself a vector space of dimension \(1\) over \(\F\).
  Such a linear transformation is called a \textbf{linear functional} on \(\V\).
\end{defn}

\begin{eg}\label{2.6.2}
  Let \(\V\) be the vector space of continuous real-valued functions on the interval \([0, 2\pi]\).
  Fix a function \(g \in \V\).
  The function \(h : \V \to \R\) defined by
  \[
    h(x) = \frac{1}{2\pi} \int_{0}^{2\pi} x(t) g(t) \; dt
  \]
  is a linear functional on \(\V\).
  In the cases that \(g(t)\) equals \(\sin(nt)\) or \(\cos(nt)\), \(h(x)\) is often called the \textbf{\(n\)th Fourier coefficient of \(x\)}.
\end{eg}

\begin{proof}[\pf{2.6.2}]
  Let \(f_1, f_2 \in \V\) and let \(c \in \R\).
  Then we have
  \begin{align*}
    h(cf_1 + f_2) & = \frac{1}{2\pi} \int_{0}^{2\pi} (cf_1 + f_2)(t) g(t) \; dt                                           &  & \text{(by \cref{2.6.2})} \\
                  & = \frac{1}{2\pi} \int_{0}^{2\pi} cf_1(t) g(t) + f_2(t) g(t) \; dt                                                                   \\
                  & = \frac{c}{2\pi} \int_{0}^{2\pi} f_1(t) g(t) \; dt + \frac{1}{2\pi} \int_{0}^{2\pi} f_2(t) g(t) \; dt                               \\
                  & = c h(f_1) + h(f_2)                                                                                   &  & \text{(by \cref{2.6.2})}
  \end{align*}
  and thus by \cref{2.1.2}(b) \(h \in \ls(\V, \F)\).
\end{proof}

\begin{eg}\label{2.6.3}
  The trace function \(\tr : \ms{n}{n}{\F} \to \F\) is a linear functional.
\end{eg}

\begin{proof}[\pf{2.6.3}]
  By \cref{ex:1.3.6} we see that this is true.
\end{proof}

\begin{eg}\label{2.6.4}
  Let \(\V\) be a finite-dimensional vector space over \(\F\), and let \(\beta = \set{\seq{x}{1,,n}}\) be an ordered basis for \(\V\) over \(\F\).
  For each \(i \in \set{1, \dots, n}\), define \(f_i(x) = a_i\), where
  \[
    [x]_{\beta} = \begin{pmatrix}
      a_1    \\
      \vdots \\
      a_n
    \end{pmatrix}
  \]
  is the coordinate vector of \(x\) relative to \(\beta\).
  Then \(f_i\) is a linear functional on \(\V\) called the \textbf{\(i\)th coordinate function with respect to the basis \(\beta\)}.
  Note that \(f_i(x_j) = \delta_{i j}\), where \(\delta_{i j}\) is the Kronecker delta.
  These linear functionals play an important role in the theory of dual spaces (see \cref{2.24}).
\end{eg}

\begin{proof}[\pf{2.6.4}]
  Let \(a, b \in \V\) and let \(c \in \F\).
  By \cref{1.8} there exist some \(\seq{a}{1,,n}, \seq{b}{1,,n} \in \F\) such that
  \[
    a = \sum_{j = 1}^n a_j x_j \quad \text{and} \quad b = \sum_{j = 1}^n b_j x_j.
  \]
  Then we have
  \begin{align*}
    f_i(ca + b) & = f_i\pa{c \pa{\sum_{j = 1}^n a_j x_j} + \sum_{j = 1}^n b_j x_j}                               \\
                & = f_i\pa{\sum_{j = 1}^n (c a_j + b_j) x_j}                       &  & \text{(by \cref{1.2.1})} \\
                & = c a_i + b_i                                                    &  & \text{(by \cref{2.6.4})} \\
                & = c f_i(a) + f_i(b)                                              &  & \text{(by \cref{2.6.4})}
  \end{align*}
  and thus by \cref{2.1.2}(b) \(f_i \in \ls(\V, \F)\).
\end{proof}

\begin{defn}\label{2.6.5}
  For a vector space \(\V\) over \(\F\), we define the \textbf{dual space} of \(\V\) to be the vector space \(\ls(\V, \F)\), denoted by \(\V^*\).
  Thus \(\V^*\) is the vector space consisting of all linear functionals on \(\V\) with the operations of addition and scalar multiplication as defined in \cref{sec:2.2}.
  Note that if \(\V\) is finite-dimensional, then by the \cref{2.4.10}
  \[
    \dim(\V^*) = \dim(\ls(\V, \F)) = \dim(\V) \cdot \dim(\F) = \dim(\V).
  \]
  Hence by \cref{2.19} \(\V\) and \(\V^*\) are isomorphic.
  We also define the \textbf{double dual} \(\V^{**}\) of \(\V\) to be the dual of \(\V^*\).
  In \cref{2.26}, we show, in fact, that there is a natural identification of \(\V\) and \(\V^{**}\) in the case that \(\V\) is finite-dimensional.
\end{defn}

\begin{thm}\label{2.24}
  Suppose that \(\V\) is a finite-dimensional vector space over \(\F\) with the ordered basis \(\beta = \set{\seq{x}{1,,n}}\).
  Let \(f_i\) (\(1 \leq i \leq n\)) be the \(i\)th coordinate function with respect to \(\beta\) as defined in \cref{2.6.4}, and let \(\beta^* = \set{\seq{f}{1,,n}}\).
  Then \(\beta^*\) is an ordered basis for \(\V^*\), and, for any \(g \in \V^*\), we have
  \[
    g = \sum_{i = 1}^n g(x_i) f_i.
  \]
\end{thm}

\begin{proof}[\pf{2.24}]
  Let \(g \in \V^*\).
  Since \(\dim(\V^*) = n\), we need only show that
  \[
    g = \sum_{i = 1}^n g(x_i) f_i,
  \]
  from which it follows that \(\beta^*\) generates \(\V^*\), and hence is a basis by \cref{1.6.15}(a).
  Let
  \[
    h = \sum_{i = 1}^n g(x_i) f_i.
  \]
  For \(1 \leq j \leq n\), we have
  \begin{align*}
    h(x_j) & = \pa{\sum_{i = 1}^n g(x_i) f_i}(x_j)                               \\
           & = \sum_{i = 1}^n g(x_i) f_i(x_j)      &  & \text{(by \cref{2.2.5})} \\
           & = \sum_{i = 1}^n g(x_i) \delta_{i j}  &  & \text{(by \cref{2.6.4})} \\
           & = g(x_j).
  \end{align*}
  Therefore \(g = h\) by \cref{2.1.13}.
\end{proof}

\begin{defn}\label{2.6.6}
  Using the notation of \cref{2.24}, we call the ordered basis \(\beta^* = \set{\seq{f}{1,,n}}\) of \(\V^*\) over \(\F\) that satisfies \(f_i(x_j) = \delta_{i j}\) (\(1 \leq i, j \leq n\)) the \textbf{dual basis} of \(\beta\).
\end{defn}

\begin{note}
  We now assume that \(\V\) and \(\W\) are finite-dimensional vector spaces over \(\F\) with ordered bases \(\beta\) and \(\gamma\) over \(\F\), respectively.
  In \cref{sec:2.4}, we proved that there is a one-to-one correspondence between linear transformations \(\T : \V \to \W\) and \(m \times n\) matrices (over \(\F\)) via the correspondence \(\T \leftrightarrow [\T]_{\beta}^{\gamma}\).
  For a matrix of the form \(A = [\T]_{\beta}^{\gamma}\), the question arises as to whether or not there exists a linear transformation \(\U\) associated with \(\T\) in some natural way such that \(\U\) may be represented in some basis as \(\tp{A}\).
  Of course, if \(m \neq n\), it would be impossible for \(\U\) to be a linear transformation from \(\V\) into \(\W\).
  We now answer this question by applying what we have already learned about dual spaces.
\end{note}

\begin{thm}\label{2.25}
  Let \(\V\) and \(\W\) be finite-dimensional vector spaces over \(\F\) with ordered bases \(\beta\) and \(\gamma\), respectively.
  For any \(\T \in \ls(\V, \W)\), the mapping \(\tp{\T} : \W^* \to \V^*\) defined by \(\tp{\T}(g) = g \T\) for all \(g \in \W^*\) is a linear transformation with the property that \([\tp{\T}]_{\gamma^*}^{\beta^*} = \tp{\pa{[\T]_{\beta}^{\gamma}}}\).
\end{thm}

\begin{proof}[\pf{2.25}]
  For \(g \in \W^*\), it is clear that \(\tp{\T}(g) = g \T\) is a linear functional on \(\V\) and hence is in \(\V^*\) (see \cref{2.9}).
  Thus \(\tp{\T}\) maps \(\W^*\) into \(\V^*\).
  Now we show that \(\tp{\T} \in \ls(\W^*, \V^*)\).
  Let \(x, y \in \W^*\) and let \(c \in \F\).
  Since
  \begin{align*}
    \tp{\T}(cx + y) & = (cx + y) \T                &  & \text{(by \cref{2.25})}  \\
                    & = c x \T + y \T              &  & \text{(by \cref{2.2.5})} \\
                    & = c \tp{\T}(x) + \tp{\T}(y), &  & \text{(by \cref{2.25})}
  \end{align*}
  by \cref{2.1.2} we see that \(\tp{\T} \in \ls(\W^*, \V^*)\).

  To complete the proof, let \(\beta = \set{\seq{x}{1,,n}}\) and \(\gamma = \set{\seq{y}{1,,m}}\) with dual bases \(\beta^* = \set{\seq{f}{1,,n}}\) and \(\gamma^* = \set{\seq{g}{1,,m}}\), respectively.
  For convenience, let \(A = [\T]_{\beta}^{\gamma}\).
  To find the \(j\)th column of \([\tp{\T}]_{\gamma^*}^{\beta^*}\), we begin by expressing \(\tp{\T}(g_j)\) as a linear combination of the vectors of \(\beta^*\).
  By \cref{2.24} we have
  \[
    \tp{\T}(g_j) = g_j \T = \sum_{i = 1}^n (g_j \T)(x_i) f_i.
  \]
  So the row \(i\), column \(j\) entry of \([\tp{\T}]_{\gamma^*}^{\beta^*}\) is
  \begin{align*}
    (g_j \T)(x_i) & = g_j(\T(x_i))                                                         \\
                  & = g_j \pa{\sum_{k = 1}^m A_{k i} y_k} &  & \text{(by \cref{2.2.4})}    \\
                  & = \sum_{k = 1}^m A_{k i} g_j(y_k)     &  & \text{(by \cref{2.1.2}(d))} \\
                  & = \sum_{k = 1}^m A_{k i} \delta_{j k} &  & \text{(by \cref{2.6.4})}    \\
                  & = A_{j i}.
  \end{align*}
  Hence \([\tp{\T}]_{\gamma^*}^{\beta^*} = \tp{A}\).
\end{proof}

\begin{note}
  The linear transformation \(\tp{\T}\) defined in \cref{2.25} is called the \textbf{transpose} of \(\T\).
  It is clear that \(\tp{\T}\) is the unique linear transformation \(\U\) such that \([\U]_{\gamma^*}^{\beta^*} = \tp{\pa{[\T]_{\beta}^{\gamma}}}\)
  (see \cref{2.1.13}).
\end{note}

\begin{note}
  We now concern ourselves with demonstrating that any finite-dimensional vector space \(\V\) over \(\F\) \textbf{can be identified} in a natural way with its double dual \(\V^{**}\).
  There is, in fact, an isomorphism between \(\V\) and \(\V^{**}\) that does not depend on any choice of bases for the two vector spaces.
\end{note}

\begin{defn}\label{2.6.7}
  For a vector \(x \in \V\), we define \(\widehat{x} : \V^* \to \F\) by \(\widehat{x}(f) = f(x)\) for every \(f \in \V^*\).
  It is easy to verify that \(\widehat{x}\) is a linear functional on \(\V^*\), so \(\widehat{x} \in \V^{**}\).
  The correspondence \(x \leftrightarrow \widehat{x}\) allows us to define the desired isomorphism between \(\V\) and \(\V^{**}\).
\end{defn}

\begin{lem}\label{2.6.8}
  Let \(\V\) be a finite-dimensional vector space over \(\F\), and let \(x \in \V\).
  If \(\widehat{x}(f) = 0\) for all \(f \in \V^*\), then \(x = \zv\).
\end{lem}

\begin{proof}[\pf{2.6.8}]
  Let \(x \neq \zv\).
  We show that there exists \(f \in \V^*\) such that \(\widehat{x}(f) \neq 0\).
  Choose an ordered basis \(\beta = \set{\seq{x}{1,,n}}\) for \(\V\) over \(\F\) such that \(x_1 = x\).
  Let \(\set{\seq{f}{1,,n}}\) be the dual basis of \(\beta\).
  Then \(f_1(x_1) = 1 \neq 0\).
  Let \(f = f_1\).
\end{proof}

\begin{thm}\label{2.26}
  Let \(\V\) be a finite-dimensional vector space over \(\F\), and define \(\psi : \V \to \V^{**}\) by \(\psi(x) = \widehat{x}\).
  Then \(\psi\) is an isomorphism.
\end{thm}

\begin{proof}[\pf{2.26}]
  \begin{description}
    \item[\(\psi\) is linear:]
      Let \(x, y \in \V\) and \(c \in \F\).
      For \(f \in \V^*\), we have
      \begin{align*}
        \psi(cx + y)(f) & = f(cx + y)                         &  & \text{(by \cref{2.6.7})} \\
                        & = cf(x) + f(y)                      &  & \text{(by \cref{2.6.5})} \\
                        & = c \widehat{x}(f) + \widehat{y}(f) &  & \text{(by \cref{2.6.7})} \\
                        & = (c \widehat{x} + \widehat{y})(f). &  & \text{(by \cref{2.2.5})}
      \end{align*}
      Therefore
      \[
        \psi(cx + y) = c \widehat{x} + \widehat{y} = c \psi(x) + \psi(y).
      \]
    \item[\(\psi\) is one-to-one:]
      Suppose that \(\psi(x)\) is the zero functional on \(\V^*\) for some \(x \in \V\).
      Then \(\widehat{x}(f) = 0\) for every \(f \in \V^*\).
      By \cref{2.6.8} we conclude that \(x = \zv\).
      Since \(\ns{\psi} = \set{\zv}\) and \(\psi \in \ls(\V, \V^{**})\), by \cref{2.4} we know that \(\psi\) is one-to-one.
    \item[\(\psi\) is an isomorphism:]
      Since \(\psi \in \ls(\V, \V^{**})\), \(\psi\) is one-to-one and \(\dim(\V) = \dim(\V^{**})\), by \cref{2.5} and \cref{2.4.8} we know that \(\psi\) is isomorphism.
  \end{description}
\end{proof}

\begin{cor}\label{2.6.9}
  Let \(\V\) be a finite-dimensional vector space over \(\F\) with dual space \(\V^*\).
  Then every ordered basis for \(\V^*\) over \(\F\) is the dual basis for some basis for \(\V\) over \(\F\).
\end{cor}

\begin{proof}[\pf{2.6.9}]
  Let \(\set{\seq{f}{1,,n}}\) be an ordered basis for \(\V^*\) over \(\F\).
  We may combine \cref{2.24,2.26} to conclude that for this basis for \(\V^*\) over \(\F\), there exists a dual basis \(\set{\seq{\widehat{x}}{1,,n}}\) in \(\V^{**}\), that is,
  \begin{align*}
    \delta_{i j} & = \widehat{x}_i(f_j) &  & \text{(by \cref{2.24,2.26})} \\
                 & = f_j(x_i)           &  & \text{(by \cref{2.6.7})}
  \end{align*}
  for all \(i\) and \(j\).
  Thus by \cref{2.24} \(\set{\seq{f}{1,,n}}\) is the dual basis of \(\set{\seq{x}{1,,n}}\).
\end{proof}

\begin{note}
  Although many of the ideas of \cref{sec:2.6}, (e.g., the existence of a dual space), can be extended to the case where \(\V\) is not finite-dimensional, only a finite-dimensional vector space is isomorphic to its double dual via the map \(x \mapsto \widehat{x}\).
  In fact, for infinite-dimensional vector spaces, no two of \(\V\), \(\V^*\), and \(\V^{**}\) are isomorphic.
\end{note}

\exercisesection

\setcounter{ex}{7}
\begin{ex}\label{ex:2.6.8}
  Show that every plane through the origin in \(\R^3\) may be identified with the null space of a vector in \((\R^3)^*\).
  State an analogous result for \(\R^2\).
\end{ex}

\begin{proof}[\pf{ex:2.6.8}]
  First we prove the statement in the case of \(\R^3\).
  Let \((a, b, c) \in \R^3\) and let \(P\) be the following plane through the origin
  \[
    P = \set{(x, y, z) \in \R^3 : ax + by + cz = 0}.
  \]
  Let \(\T : \R^3 \to \R\) be the function \(\T(x, y, z) = ax + by + cz\).
  Clearly \(\T \in (\R^3)^*\).
  Then we have
  \begin{align*}
    \ns{\T} & = \set{(x, y, z) \in \R^3 : \T(x, y, z) = 0}       &  & \text{(by \cref{2.1.10})} \\
            & = \set{(x, y, z) \in \R^3 : ax + by + cz = 0} = P.
  \end{align*}

  Now we prove the statement in the case of \(\R^2\).
  Let \((a, b) \in \R^2\) and let \(P\) be the following plane through the origin
  \[
    P = \set{(x, y) \in \R^2 : ax + by = 0}.
  \]
  Let \(\T : \R^2 \to \R\) be the function \(\T(x, y) = ax + by\).
  Clearly \(\T \in (\R^2)^*\).
  Then we have
  \begin{align*}
    \ns{\T} & = \set{(x, y) \in \R^2 : \T(x, y) = 0}     &  & \text{(by \cref{2.1.10})} \\
            & = \set{(x, y) \in \R^2 : ax + by = 0} = P.
  \end{align*}
\end{proof}

\begin{ex}\label{ex:2.6.9}
  Prove that a function \(\T : \vs{F}^n \to \vs{F}^m\) is linear iff there exist \(\seq{f}{1,,m} \in (\vs{F}^n)^*\) such that \(\T(x) = (f_1(x), f_2(x), \dots, f_m(x))\) for all \(x \in \vs{F}^n\).
\end{ex}

\begin{proof}[\pf{ex:2.6.9}]
  First suppose that \(\T \in \ls(\vs{F}^n, \vs{F}^m)\).
  Let \(\gamma\) be the standard ordered basis for \(\vs{F}^m\) over \(\F\) and let \(\set{\seq{g}{1,,m}}\) be the dual basis of \(\gamma\).
  For each \(i \in \set{1, \dots, m}\), we define \(f_i : \vs{F}^n \to \F\) as follow:
  \[
    f_i = \tp{\T}(g_i) = g_i \T.
  \]
  By \cref{2.9,2.24} we see that \(f_i \in \ls(\vs{F}^n, \F)\).
  Thus by \cref{2.6.1} \(f_i \in (\vs{F}^n)^*\) and we have
  \begin{align*}
    \forall x \in \vs{F}^n, \T(x) & = [\T(x)]_{\gamma}                &  & \text{(by \cref{2.2.3})} \\
                                  & = \begin{pmatrix}
                                        g_1(\T(x)) & \cdots & g_m(\T(x))
                                      \end{pmatrix} &  & \text{(by \cref{2.6.4})}                   \\
                                  & = \begin{pmatrix}
                                        f_1(x) & \cdots & f_m(x)
                                      \end{pmatrix}.
  \end{align*}

  Now suppose that \(\T : \vs{F}^n \to \vs{F}^m\) is a function and there exist \(\seq{f}{1,,m} \in (\vs{F}^n)^*\) such that \(\T(x) = (f_1(x), \dots, f_m(x))\) for all \(x \in \vs{F}^n\).
  Let \(x, y \in \vs{F}^n\) and let \(c \in \F\).
  Then we have
  \begin{align*}
    \T(cx + y) & = \begin{pmatrix}
                     f_1(cx + y) & \cdots & f_m(cx + y)
                   \end{pmatrix}             \\
               & = \begin{pmatrix}
                     c f_1(x) + f_1(y) & \cdots & c f_m(x) + f_m(y)
                   \end{pmatrix} &  & \text{(by \cref{2.1.2}(b))} \\
               & = c \T(x) + \T(y)
  \end{align*}
  and thus by \cref{2.1.2}(b) \(\T \in \ls(\vs{F}^n, \vs{F}^m)\).
\end{proof}

\begin{ex}\label{ex:2.6.10}
  Let \(\seq{c}{0,,n}\) be distinct scalars in \(\F\).
  \begin{enumerate}
    \item For \(0 \leq i \leq n\), define \(f_i \in (\ps[n]{\F})^*\) by \(f_i(p) = p(c_i)\).
          Prove that \(\set{\seq{f}{0,,n}}\) is a basis for \((\ps[n]{\F})^*\).
    \item Use \cref{2.6.9} and (a) to show that there exist unique polynomials \(\seq{p}{0,,n}\) such that \(p_i(c_j) = \delta_{i j}\) for \(0 \leq i \leq n\).
          These polynomials are the Lagrange polynomials defined in \cref{1.6.20}.
    \item For any scalars \(\seq{a}{0,,n} \in \F\) (not necessarily distinct), deduce that there exists a unique polynomial \(q(x)\) of degree at most \(n\) such that \(q(c_i) = a_i\) for \(0 \leq i \leq n\).
          In fact,
          \[
            q(x) = \sum_{i = 0}^n a_i p_i(x).
          \]
    \item Deduce the Lagrange interpolation formula:
          \[
            p(x) = \sum_{i = 0}^n p(c_i) p_i(x)
          \]
          for any \(p \in \ps[n]{\F}\).
    \item Prove that
          \[
            \int_a^b p(t) \; dt = \sum_{i = 0}^n p(c_i) d_i,
          \]
          where
          \[
            d_i = \int_a^b p_i(t) \; dt.
          \]
          Suppose now that
          \[
            c_i = a + \frac{i (b - a)}{n} \quad \text{for } i \in \set{0, \dots, n}.
          \]
          For \(n = 1\), the preceding result yields the trapezoidal rule for evaluating the definite integral of a polynomial.
          For \(n = 2\), this result yields Simpson's rule for evaluating the definite integral of a polynomial.
  \end{enumerate}
\end{ex}

\begin{proof}[\pf{ex:2.6.10}(a)]
  Let \(\seq{a}{0,,n} \in \F\) such that
  \[
    \sum_{i = 0}^n a_i f_i = \zv.
  \]
  For each \(j \in \set{0, \dots, n}\), we define \(p_j \in \ps[n]{\F}\) as follow:
  \[
    \forall x \in \F, p_j(x) = \prod_{\substack{i = 0 \\ i \neq j}}^n (x - c_i).
  \]
  Then we have
  \begin{align*}
             & p_j(c_i) \neq 0 \iff j = i                                                                                                    \\
    \implies & \forall i \in \set{0, \dots, n}, \pa{\sum_{i = 0}^n a_i f_i}(p_j) = \sum_{i = 0}^n a_i f_i(p_j) &  & \text{(by \cref{2.2.5})} \\
             & = \sum_{i = 0}^n a_i p_j(c_i) = a_j p_j(c_j) = 0                                                                              \\
    \implies & \forall i \in \set{0, \dots, n}, a_j = 0.                                                       &  & (p_j(c_j) \neq 0)
  \end{align*}
  Thus by \cref{1.5.3} \(\set{\seq{f}{0,,n}}\) is linearly independent.
  Since
  \begin{align*}
    \dim((\ps[n]{\F})^*) & = \dim(\ls(\ps[n]{\F}, \F))       &  & \text{(by \cref{2.6.5})}  \\
                         & = \dim(\ps[n]{\F}) \cdot \dim(\F) &  & \text{(by \cref{2.4.10})} \\
                         & = n + 1                           &  & \text{(by \cref{1.6.12})} \\
                         & = \#(\set{\seq{f}{0,,n}}),
  \end{align*}
  by \cref{1.6.15}(b) we know that \(\set{\seq{f}{0,,n}}\) is a basis for \((\ps[n]{\F})^*\).
\end{proof}

\begin{proof}[\pf{ex:2.6.10}(b)]
  By \cref{2.6.9} there exists a set \(\beta = \set{\seq{p}{0,,n}}\) such that \(\beta\) is an ordered basis for \(\ps[n]{\F}\) over \(\F\) and the set \(\beta^* = \set{\seq{f}{0,,n}}\) defined in \cref{ex:2.6.10}(a) is the dual basis of \(\beta\).
  Then we have
  \begin{align*}
             & \forall i, j \in \set{0, \dots, n}, f_j(p_i) = \delta_{j i} = \delta_{i j} &  & \text{(by \cref{2.6.6})}        \\
    \implies & \forall i, j \in \set{0, \dots, n}, p_i(c_j) = \delta_{i j}.               &  & \text{(by \cref{ex:2.6.10}(a))}
  \end{align*}
  Now we show that \(\beta\) is unique.
  Suppose for sake of contradiction that there exists a set \(\gamma = \set{\seq{q}{0,,n}}\) such that \(\beta \neq \gamma\), \(q_i(c_j) = \delta_{i j}\) for \(i \in \set{0, \dots, n}\), \(\gamma\) is an ordered basis for \(\ps[n]{\F}\) over \(\F\) and \(\beta^*\) is also the dual basis for \(\gamma\).
  Since \(q_0 \in \ps[n]{\F}\), by \cref{1.6.1} we know that there exist some \(\seq{a}{0,,n} \in \F\) such that
  \[
    q_0 = \sum_{i = 0}^n a_i p_i.
  \]
  But
  \begin{align*}
    \forall j \in \set{0, \dots, n}, q_0(c_j) & = \delta_{0 j}                                                        \\
                                              & = \pa{\sum_{i = 0}^n a_i p_i}(c_j) &  & \text{(by \cref{2.2.5})}      \\
                                              & = \sum_{i = 0}^n a_i p_i(c_j)                                         \\
                                              & = \sum_{i = 0}^n a_i \delta_{i j}  &  & \text{(from the proof above)} \\
                                              & = a_j
  \end{align*}
  implies \(a_0 = 1\) and \(a_j = 0\) for all \(j \in \set{0, \dots, n} \setminus \set{0}\).
  Thus we have \(q_0 = p_0\).
  Using similar argument we see that \(q_i = p_i\) for all \(i \in \set{0, \dots, n}\), a contradiction.
  Thus \(\beta\) is unique.
\end{proof}

\begin{proof}[\pf{ex:2.6.10}(c)]
  Defined \(\seq{p}{0,,n} \in \ps[n]{\F}\) as in \cref{ex:2.6.10}(b).
  For any \(\seq{a}{0,,n} \in \F\), we can define \(q \in \ps[n]{\F}\) as follow:
  \[
    q = \sum_{j = 0}^n a_j p_j.
  \]
  Note that by \cref{ex:2.6.10}(b) \(q\) is unique.
  Then we have
  \[
    \forall i \in \set{0, \dots, n}, q(c_i) = \pa{\sum_{j = 0}^n a_j p_j}(c_i) = \sum_{j = 0}^n a_j p_j(c_i) = \sum_{j = 0}^n a_j \delta_{j i} = a_i
  \]
  and thus
  \[
    q = \sum_{j = 0}^n q(c_j) p_j.
  \]
\end{proof}

\begin{proof}[\pf{ex:2.6.10}(d)]
  By \cref{ex:2.6.10}(c) we see that
  \[
    \forall p \in \ps[n]{\F}, p = \sum_{i = 0}^n p(c_i) p_i.
  \]
\end{proof}

\begin{proof}[\pf{ex:2.6.10}(e)]
  By \cref{ex:2.6.10}(d) we have
  \begin{align*}
    \forall p \in \ps[n]{\F}, \int_a^b p(t) \; dt & = \int_a^b \pa{\sum_{i = 0}^n p(c_i) p_i(t)} \; dt \\
                                                  & = \sum_{i = 0}^n \pa{p(c_i) \int_a^b p_i(t) \; dt} \\
                                                  & = \sum_{i = 0}^n p(c_i) d_i.
  \end{align*}
\end{proof}

\begin{ex}\label{ex:2.6.11}
  Let \(\V\) and \(\W\) be finite-dimensional vector spaces over \(\F\), and let \(\psi_1\) and \(\psi_2\) be the isomorphisms between \(\V\) and \(\V^{**}\) and \(\W\) and \(\W^{**}\), respectively, as defined in \cref{2.26}.
  Let \(\T \in \ls(\V, \W)\), and define \(\tp{\tp{\T}} = \tp{\pa{\tp{\T}}}\).
  Prove that \(\psi_2 \T = \tp{\tp{\T}} \psi_1\).
\end{ex}

\begin{proof}[\pf{ex:2.6.11}]
  By \cref{2.25} we see that \(\tp{\tp{\T}} \in \ls(\V^{**}, \W^{**})\).
  Let \(x \in \V\) and let \(y \in \W^*\).
  Then we have
  \begin{align*}
    \pa{\pa{\tp{\tp{\T}} \psi_1}(x)}(y) & = \pa{\tp{\tp{\T}}(\psi_1(x))}(y)                                 \\
                                        & = \pa{\tp{\tp{\T}}(\widehat{x})}(y) &  & \text{(by \cref{2.26})}  \\
                                        & = \pa{\widehat{x} \tp{\T}}(y)       &  & \text{(by \cref{2.25})}  \\
                                        & = \widehat{x}\pa{\tp{\T}(y)}                                      \\
                                        & = \widehat{x}(y \T)                 &  & \text{(by \cref{2.26})}  \\
                                        & = (y \T)(x)                         &  & \text{(by \cref{2.6.7})} \\
                                        & = y(\T(x))                                                        \\
                                        & = \widehat{\T(x)}(y)                &  & \text{(by \cref{2.6.7})} \\
                                        & = \psi_2(\T(x))(y)                  &  & \text{(by \cref{2.26})}  \\
                                        & = ((\psi_2 \T)(x))(y).
  \end{align*}
  Thus \(\tp{\tp{\T}} \psi_1 = \psi_2 \T\).
\end{proof}

\begin{ex}\label{ex:2.6.12}
  Let \(\V\) be a finite-dimensional vector space over \(\F\) with the ordered basis \(\beta\) over \(\F\).
  Prove that \(\psi(\beta) = \beta^{**}\), where \(\psi\) is defined in \cref{2.26}.
\end{ex}

\begin{proof}[\pf{ex:2.6.12}]
  Let \(\beta = \set{\seq{x}{1,,n}}\) and let \(\beta^* = \set{\seq{f}{1,,n}}\) be the dual basis of \(\beta\).
  By \cref{2.26} we have \(\psi(\beta) = \set{\seq{\widehat{x}}{1,,n}}\).
  Since
  \begin{align*}
    \forall i, j \in \set{1, \dots, n}, \widehat{x}_i(f_j) & = f_j(x_i)      &  & \text{(by \cref{2.6.7})} \\
                                                           & = \delta_{i j}, &  & \text{(by \cref{2.6.6})}
  \end{align*}
  by \cref{2.6.6} we see that \(\beta^{**} = \set{\seq{\widehat{x}}{1,,n}}\).
\end{proof}

\begin{defn}\label{2.6.10}
  Let \(\V\) denotes a finite-dimensional vector space over \(\F\).
  For every subset \(S\) of \(\V\), define the \textbf{annihilator} \(S^0\) of \(S\) as
  \[
    S^0 = \set{f \in \V^* : f(x) = 0 \text{ for all } x \in S}.
  \]
\end{defn}

\begin{ex}\label{ex:2.6.13}
  Let \(\V\) be a vector space over \(\F\) and let \(S \subseteq \V\).
  \begin{enumerate}
    \item Prove that \(S^0\) is a subspace of \(\V^*\) over \(\F\).
    \item If \(\W\) is a subspace of \(\V\) over \(\F\) and \(x \notin \W\), prove that there exists \(f \in \W^0\) such that \(f(x) \neq 0\).
    \item Prove that \((S^0)^0 = \spn{\psi(S)}\), where \(\psi\) is defined as in \cref{2.26}.
    \item For subspaces \(\W_1\) and \(\W_2\) of \(\V\) over \(\F\), prove that \(\W_1 = \W_2\) iff \(\W_1^0 = \W_2^0\).
    \item For subspaces \(W_1\) and \(\W_2\) of \(\V\) over \(\F\), show that \((\W_1 + \W_2)^0 = \W_1^0 \cap \W_2^0\).
  \end{enumerate}
\end{ex}

\begin{proof}[\pf{ex:2.6.13}(a)]
  Let \(f, g \in S^0\) and let \(c \in \F\).
  Since
  \begin{align*}
    \forall x \in S, (cf + g)(x) & = cf(x) + g(x) &  & \text{(by \cref{2.2.5})}  \\
                                 & = c0 + 0       &  & \text{(by \cref{2.6.10})} \\
                                 & = 0,
  \end{align*}
  by \cref{2.6.10} we see that \(cf + g \in S^0\).
  Let \(\zT \in \V^*\) be the zero transformation of \(\V^*\).
  Since
  \begin{align*}
             & \forall x \in \V, \zT(x) = 0 &  & \text{(by \cref{2.1.9})}  \\
    \implies & \forall x \in S, \zT(x) = 0  &  & (S \subseteq \V)          \\
    \implies & \zT \in S^0,                 &  & \text{(by \cref{2.6.10})}
  \end{align*}
  by \cref{ex:1.3.18} we see that \(S^0\) is a subspace of \(\V^*\).
\end{proof}

\begin{proof}[\pf{ex:2.6.13}(b)]
  Let \(\beta_{\W}\) be a basis for \(\W\) over \(\F\).
  Since \(x \notin \W\), by \cref{1.7} we know that \(\beta_{\W} \cup \set{x}\) is linearly independent.
  By \cref{1.6.19} \(\beta_{\W} \cup \set{x}\) can be extended to a basis \(\beta\) for \(\V\) over \(\F\).
  By \cref{2.6} there exists a \(f \in \V^*\) such that
  \[
    \forall v \in \beta, f(v) = \begin{dcases}
      0 & \text{if } v \neq x \\
      1 & \text{if } v = x
    \end{dcases}.
  \]
  Then we have
  \begin{align*}
             & \W = \spn{\beta_{\W}}                          &  & \text{(by \cref{1.6.1})}  \\
    \implies & f(\W) = \rg{f} = \spn{f(\beta_{\W})} = \set{0} &  & \text{(by \cref{2.2})}    \\
    \implies & f \in \W^0.                                    &  & \text{(by \cref{2.6.10})}
  \end{align*}
\end{proof}

\begin{proof}[\pf{ex:2.6.13}(c)]
  Since
  \begin{align*}
         & f \in S^0                                                        \\
    \iff & \forall x \in S, f(x) = 0       &  & \text{(by \cref{2.6.10})}   \\
    \iff & \forall x \in \spn{S}, f(x) = 0 &  & \text{(by \cref{2.1.2}(d))} \\
    \iff & f \in (\spn{S})^0,              &  & \text{(by \cref{2.6.10})}
  \end{align*}
  we have \(S^0 = (\spn{S})^0\).
  Since \(\psi \in \ls(\V, \V^{**})\), by \cref{2.2} we know that \(\psi(\spn{S}) = \spn{\psi(S)}\).
  Thus to prove that \((S^0)^0 = \spn{\psi(S)}\), we can instead prove that \(((\spn{S})^0)^0 = \psi(\spn{S})\).
  By \cref{ex:2.6.13}(a) we see that \(S^0 \subseteq \V^*\), thus by \cref{2.6.10} we have
  \[
    (S^0)^0 = ((\spn{S})^0)^0 = \set{f \in \V^{**} : f((\spn{S})^0) = \set{0}}.
  \]

  First we show that \(((\spn{S})^0)^0 \subseteq \psi(\spn{S})\).
  Let \(f \in ((\spn{S})^0)^0\).
  Since \(f \in \V^{**}\), by \cref{2.26} we know that there exists an \(s \in \V\) such that \(\psi(s) = \widehat{s} = f\).
  Then we have
  \begin{align*}
             & f = \widehat{s}                                                                                  \\
    \implies & \forall g \in (\spn{S})^0, f(g) = 0 = \widehat{s}(g) = g(s) &  & \text{(by \cref{2.6.7})}        \\
    \implies & s \in \spn{S}                                               &  & \text{(by \cref{ex:2.6.13}(b))} \\
    \implies & f = \widehat{s} = \psi(s) \in \psi(\spn{S}).
  \end{align*}
  Since \(f\) is arbitrary, we see that \(((\spn{S})^0)^0 \subseteq \psi(\spn{S})\).

  Now we show that \(\psi(\spn{S}) \subseteq ((\spn{S})^0)^0\).
  Let \(f \in \psi(\spn{S})\).
  By \cref{2.26} we know that there exists an \(s \in \spn{S}\) such that \(\psi(s) = \widehat{s} = f\).
  Then we have
  \begin{align*}
             & f = \widehat{s}                                                                       \\
    \implies & \forall g \in \V^*, f(g) = \widehat{s}(g) = g(s) &  & \text{(by \cref{2.6.7})}        \\
    \implies & \forall g \in (\spn{S})^0, f(g) = g(s)           &  & \text{(by \cref{ex:2.6.13}(a))} \\
             & = 0                                              &  & \text{(by \cref{2.6.10})}       \\
    \implies & f = \widehat{s} = \psi(s) \in ((\spn{S})^0)^0.   &  & \text{(by \cref{2.6.10})}
  \end{align*}
  Since \(f\) is arbitrary, we see that \(\psi(\spn{S}) \subseteq ((\spn{S})^0)^0\).
\end{proof}

\begin{proof}[\pf{ex:2.6.13}(d)]
  We have
  \begin{align*}
             & \W_1 = \W_2                                                                    \\
    \implies & \W_1^0 = \set{f \in \V^* : f(\W_1) = \set{0}}   &  & \text{(by \cref{2.6.10})} \\
             & = \set{f \in \V^* : f(\W_2) = \set{0}} = \W_2^0 &  & \text{(by \cref{2.6.10})} \\
  \end{align*}
  and
  \begin{align*}
             & \W_1^0 = \W_2^0                                                                                                  \\
    \implies & \set{f \in \V^* : f(\W_1) = \set{0}} = \set{f \in \V^* : f(\W_2) = \set{0}} &  & \text{(by \cref{2.6.10})}       \\
    \implies & \W_1 = \W_2.                                                                &  & \text{(by \cref{ex:2.6.13}(b))}
  \end{align*}
  Thus \(\W_1 = \W_2 \iff \W_1^0 = \W_2^0\).
\end{proof}

\begin{proof}[\pf{ex:2.6.13}(e)]
  Since
  \begin{align*}
         & f \in (\W_1 + \W_2)^0                                                                  \\
    \iff & \forall x \in \W_1 + \W_2, f(x) = 0                     &  & \text{(by \cref{2.6.10})} \\
    \iff & \forall (x_1, x_2) \in \W_1 \times \W_2, \begin{dcases}
                                                      x_1 + \zv_{\V} \in \W_1 + \W_2 \\
                                                      \zv_{\V} + x_2 \in \W_1 + \W_2 \\
                                                      x_1 + x_2 \in \W_1 + \W_2      \\
                                                      f(x_1) = f(x_2) = f(x_1 + x_2) = 0
                                                    \end{dcases} &  & \text{(by \cref{1.3.10})}   \\
    \iff & \begin{dcases}
             f \in \W_1^0 \\
             f \in \W_2^0
           \end{dcases}                                         &  & \text{(by \cref{2.6.10})}    \\
    \iff & f \in \W_1^0 \cap \W_2^0,
  \end{align*}
  we know that \((\W_1 + \W_2)^0 = \W_1^0 \cap \W_2^0\).
\end{proof}

\begin{ex}\label{ex:2.6.14}
  Prove that if \(\V\) is a finite-dimensional vector space over \(\F\) and \(\W\) is a subspace of \(\V\) over \(\F\), then \(\dim(\W) + \dim(\W^0) = \dim(\V)\).
\end{ex}

\begin{proof}[\pf{ex:2.6.14}]
  Let \(\dim(\V) = n\) and let \(\beta_{\W} = \set{\seq{x}{1,,k}}\) be an ordered basis for \(\W\) over \(\F\).
  By \cref{1.6.19} we can extend \(\beta_{\W}\) to an ordered basis \(\beta = \set{\seq{x}{1,,k,k+1,,n}}\) for \(\V\) over \(\F\).
  By \cref{2.6.6} we denote the dual basis of \(\beta\) as \(\beta^* = \set{\seq{f}{1,,n}}\).
  If we can show that the set \(\gamma = \set{\seq{f}{k+1,,n}}\) is a basis for \(\W^0\) over \(\F\), then we have
  \[
    \dim(\W) + \dim(\W^0) = k + (n - k) = n = \dim(\V).
  \]
  By \cref{2.24} we know that \(\beta^*\) is a basis for \(\V^*\) over \(\F\), thus \(\gamma\) is linearly independent and by \cref{1.6.1} we only need to show that \(\spn{\gamma} = \W^0\).

  First we show that \(\spn{\gamma} \subseteq \W^0\).
  By \cref{2.24} we have \(\spn{\gamma} \subseteq \V^*\).
  Let \(w \in \W\).
  By \cref{1.6.1} we know that
  \[
    \exists \seq{a}{1,,k} \in \F : w = \sum_{j = 1}^k a_j x_j.
  \]
  Thus we have
  \begin{align*}
             & \forall g \in \spn{\gamma}, \exists \seq{b}{1,,n-k} \in \F : g = \sum_{i = 1}^{n - k} b_i f_{i + k}  &  & \text{(by \cref{1.4.3})}    \\
    \implies & \forall g \in \spn{\gamma}, g(w) = \pa{\sum_{i = 1}^{n - k} b_i f_{i + k}}(w)                                                         \\
             & = \sum_{i = 1}^{n - k} b_i f_{i + k}(w) = \sum_{i = 1}^{n - k} \sum_{j = 1}^k b_i a_j f_{i + k}(x_j) &  & \text{(by \cref{2.1.2}(d))} \\
             & = 0                                                                                                  &  & \text{(by \cref{2.6.6})}    \\
    \implies & \forall g \in \spn{\gamma}, g(\W) = \set{0}                                                                                           \\
    \implies & \forall g \in \spn{\gamma}, g \in \W^0                                                               &  & \text{(by \cref{2.6.10})}   \\
    \implies & \spn{\gamma} \subseteq \W^0.
  \end{align*}

  Now we show that \(\W^0 \subseteq \spn{\gamma}\).
  This is true since
  \begin{align*}
             & \W^0 \subseteq \V^*                                         &  & \text{(by \cref{2.6.10})} \\
    \implies & \W^0 \subseteq \spn{\beta^*}                                &  & \text{(by \cref{2.24})}   \\
    \implies & \forall g \in \W^0, g = \sum_{i = 1}^n g(x_i) f_i           &  & \text{(by \cref{2.24})}   \\
             & = \sum_{i = 1}^k g(x_i) f_i + \sum_{i = k + 1}^n g(x_i) f_i                                \\
             & = 0 + \sum_{i = k + 1}^n g(x_i) f_i                         &  & \text{(by \cref{2.6.10})} \\
    \implies & \forall g \in \W^0, g \in \spn{\gamma}                                                     \\
    \implies & \W^0 \subseteq \spn{\gamma}.
  \end{align*}
  Thus we have \(\W^0 = \spn{\gamma}\).
\end{proof}

\begin{ex}\label{ex:2.6.15}
  Suppose that \(\V, \W\) are finite-dimensional vector spaces over \(\F\) and that \(T \in \ls(\V, \W)\).
  Prove that \(\ns{\tp{\T}} = (\rg{\T})^0\).
\end{ex}

\begin{proof}[\pf{ex:2.6.15}]
  We have
  \begin{align*}
         & x \in \ns{\tp{\T}}                                                           \\
    \iff & \tp{\T}(x) = 0_{\V^*}                         &  & \text{(by \cref{2.1.10})} \\
    \iff & x \T = 0_{\V^*}                               &  & \text{(by \cref{2.25})}   \\
    \iff & (x \T)(\V) = x(\T(\V)) = x(\rg{\T}) = \set{0} &  & \text{(by \cref{2.1.10})} \\
    \iff & x \in (\rg{\T})^0                             &  & \text{(by \cref{2.6.10})}
  \end{align*}
  and thus \(\ns{\tp{\T}} = (\rg{\T})^0\).
\end{proof}

\begin{ex}\label{ex:2.6.16}
  Use \cref{ex:2.6.14,ex:2.6.15} to deduce that \(\rk{\L_{\tp{A}}} = \rk{\L_A}\) for any \(A \in \MS\).
\end{ex}

\begin{proof}[\pf{ex:2.6.16}]
  Let \(\beta\) and \(\gamma\) be ordered bases for \(\vs{F}^n\) and \(\vs{F}^m\) over \(\F\), respectively.
  Let \(\beta^*\) and \(\gamma^*\) be the dual bases of \(\beta\) and \(\gamma\), respectively.
  Since
  \begin{align*}
    [\tp{\pa{\L_A}}]_{\gamma^*}^{\beta^*} & = \tp{\pa{[\L_A]_{\beta}^{\gamma}}} &  & \text{(by \cref{2.25})}    \\
                                          & = \tp{A}                            &  & \text{(by \cref{2.15}(a))} \\
                                          & = [\L_{\tp{A}}]_{\gamma}^{\beta},   &  & \text{(by \cref{2.15}(a))}
  \end{align*}
  we have
  \begin{align*}
    \rk{\L_{\tp{A}}} & = \rk{\tp{\pa{\L_A}}}                                     &  & \text{(by \cref{2.15}(b))}   \\
                     & = \dim\pa{\pa{\vs{F}^m}^*} - \nt{\tp{\pa{\L_A}}}          &  & \text{(by \cref{2.3})}       \\
                     & = \dim\pa{\pa{\vs{F}^m}^*} - \dim\pa{\ns{\tp{\pa{\L_A}}}} &  & \text{(by \cref{2.1.12})}    \\
                     & = \dim\pa{\vs{F}^m} - \dim\pa{\ns{\tp{\pa{\L_A}}}}        &  & \text{(by \cref{2.6.5})}     \\
                     & = \dim\pa{\vs{F}^m} - \dim\pa{\pa{\rg{\L_A}}^0}           &  & \text{(by \cref{ex:2.6.15})} \\
                     & = \dim\pa{\rg{\L_A}}                                      &  & \text{(by \cref{ex:2.6.14})} \\
                     & = \rk{\L_A}.                                              &  & \text{(by \cref{2.1.12})}
  \end{align*}
\end{proof}

\begin{ex}\label{ex:2.6.17}
  Let \(\V\) be a vector space over \(\F\), let \(\T \in \ls(\V)\) and let \(\W\) be a subspace of \(\V\) over \(\F\).
  Prove that \(\W\) is \(\T\)-invariant (as defined in \cref{2.1.15}) iff \(\W^0\) is \(\tp{\T}\)-invariant.
\end{ex}

\begin{proof}[\pf{ex:2.6.17}]
  We have
  \begin{align*}
         & \W^0 \text{ is } \tp{\T} \text{-invariant}                                          \\
    \iff & \tp{\T}(\W^0) \subseteq \W^0                   &  & \text{(by \cref{2.1.15})}       \\
    \iff & \forall f \in \W^0, \tp{\T}(f) = f \T \in \W^0 &  & \text{(by \cref{2.25})}         \\
    \iff & \forall f \in \W^0, f(\T(\W)) = \set{0}        &  & \text{(by \cref{2.6.10})}       \\
    \iff & \T(\W) \subseteq \W                            &  & \text{(by \cref{ex:2.6.13}(b))} \\
    \iff & \W \text{ is } \T \text{-invariant}.           &  & \text{(by \cref{2.1.15})}
  \end{align*}
\end{proof}

\begin{ex}\label{ex:2.6.18}
  Let \(\V\) be a nonzero vector space over a field \(\F\), and let \(S\) be a basis for \(\V\) over \(\F\).
  (By \cref{1.7.10}, every vector space has a basis.)
  Let \(\Phi : \V^* \to \FS\) be the mapping defined by \(\Phi(f) = f_S\), the restriction of \(f\) to \(S\).
  Prove that \(\Phi\) is an isomorphism.
\end{ex}

\begin{proof}[\pf{ex:2.6.18}]
  \begin{description}
    \item[For linearity:]
      Let \(f, g \in \V^*\) and let \(c \in \F\).
      Then we have
      \[
        \Phi(cf + g) = (cf + g)_S = c f_S + g_S = c \Phi(f) + \Phi(g)
      \]\
      and thus by \cref{2.1.2}(b) \(\Phi \in \ls(\V^*, \FS)\).
    \item[For invertibility:]
      By \cref{ex:2.1.34} we see that \(\Phi\) is one-to-one and onto.
  \end{description}
\end{proof}

\begin{ex}\label{ex:2.6.19}
  Let \(\V\) be a nonzero vector space over \(\F\), and let \(\W\) be a proper subspace of \(\V\) over \(\F\)
  (i.e., \(\W \neq \V\)).
  Prove that there exists a nonzero linear functional \(f \in \V^*\) such that \(f(x) = 0\) for all \(x \in \W\).
\end{ex}

\begin{proof}[\pf{ex:2.6.19}]
  Let \(\beta_{\W}\) be a basis for \(\W\) over \(\F\).
  By \cref{1.12,1.13} we can extend \(\beta_{\W}\) to a basis \(\beta\) for \(\V\) over \(\F\).
  Then by \cref{ex:2.1.34} there exists a \(f \in \V^*\) such that
  \[
    \forall v \in \beta, f(v) = \begin{dcases}
      0 & \text{if } v \in \beta_{\W}                 \\
      1 & \text{if } v \in \beta \setminus \beta_{\W}
    \end{dcases}.
  \]
  Note that \(\W \neq \V\) implies \(\beta \setminus \beta_{\W} \neq \varnothing\) and thus \(f\) is not a zero transformation.
\end{proof}

\begin{ex}\label{ex:2.6.20}
  Let \(\V\) and \(\W\) be nonzero vector spaces over the same field \(\F\), and let \(\T \in \ls(\V, \W)\).
  \begin{enumerate}
    \item Prove that \(\T\) is onto iff \(\tp{\T}\) is one-to-one.
    \item Prove that \(\tp{\T}\) is onto iff \(\T\) is one-to-one.
  \end{enumerate}
\end{ex}

\begin{proof}[\pf{ex:2.6.20}(a)]
  First suppose that \(\T\) is onto.
  Let \(f \in \ns{\tp{\T}}\).
  Since
  \begin{align*}
             & \tp{\T}(f) = 0_{\V^*}          &  & \text{(by \cref{2.1.10})} \\
    \implies & f \T = 0_{\V^*}                &  & \text{(by \cref{2.25})}   \\
    \implies & f(\T(\V)) = f(\W) = \set{0}    &  & \text{(\(f\) is onto)}    \\
    \implies & f = 0_{\W^*}                   &  & \text{(by \cref{2.1.9})}  \\
    \implies & \ns{\tp{\T}} = \set{0_{\W^*}}, &  & \text{(by \cref{2.1.10})}
  \end{align*}
  by \cref{2.4} we know that \(\tp{\T}\) is one-to-one.

  Now suppose that \(\tp{\T}\) is one-to-one.
  Suppose for sake of contradiction that \(\T\) is not onto.
  By \cref{ex:2.1.33} we know that \(\rg{\T} \subsetneq \W\).
  By \cref{ex:2.6.19} there exists a \(f \in \W^*\) such that \(f(\rg{\T}) = \set{0}\) and \(f(\W) \neq \set{0}\).
  Let \(g \in \W^*\) be the zero transformation.
  Then we have
  \begin{align*}
    \forall x \in \V, \pa{\tp{\T}(g)}(x) & = (g \T)(x)          &  & \text{(by \cref{2.25})} \\
                                         & = g(\T(x))                                        \\
                                         & = 0                                               \\
                                         & = f(\T(x))                                        \\
                                         & = (f \T)(x)                                       \\
                                         & = \pa{\tp{\T}(f)}(x) &  & \text{(by \cref{2.25})}
  \end{align*}
  and thus \(\tp{\T}(g) = \tp{\T}(f)\).
  But \(\tp{\T}\) is one-to-one implies \(f = g\), a contradiction.
  Thus \(\T\) is onto.
\end{proof}

\begin{proof}[\pf{ex:2.6.20}(b)]
  First suppose that \(\tp{\T}\) is onto.
  Suppose for sake of contradiction that \(\T\) is not one-to-one.
  By \cref{2.4} there exists an \(x \in \V \setminus \set{\zv_{\V}}\) such that \(\T(x) = \zv_{\W}\).
  By \cref{ex:2.1.34} there exists a \(f \in \V^*\) such that \(f(x) \neq 0\).
  Since \(f \in \V^*\) and \(\tp{\T}\) is onto, there exists a \(g \in \W^*\) such that \(f = \tp{\T}(g)\).
  But this means
  \begin{align*}
    f(x) & = \pa{\tp{\T}(g)}(x)                                  \\
         & = (g \T)(x)          &  & \text{(by \cref{2.25})}     \\
         & = g(\T(x))                                            \\
         & = g(\zv_{\W})        &  & \text{(by hypothesis)}      \\
         & = 0,                 &  & \text{(by \cref{2.1.2}(a))}
  \end{align*}
  a contradiction.
  Thus \(\T\) is one-to-one.

  Now suppose that \(\T\) is one-to-one.
  Let \(f \in \V^*\) and let \(\beta\) be a basis for \(\V\) over \(\F\).
  Since \(\T\) is one-to-one, by \cref{ex:2.1.14}(a) we know that \(\T(\beta)\) is linearly independent.
  By \cref{1.12,1.13} we can extend \(\T(\beta)\) to a basis \(\gamma\) for \(\W\) over \(\F\).
  If we define \(h \in \fs(\gamma, \F)\) by setting
  \[
    \forall w \in \gamma, h(w) = \begin{dcases}
      f(v) & \text{if } (w \in \T(\beta)) \land (\exists! v \in \beta : \T(v) = w) \\
      0    & \text{if } w \in \gamma \setminus \T(\beta)
    \end{dcases},
  \]
  then by \cref{ex:2.6.18} we know that there exists a \(g \in \W^*\) such that \(\Phi(g) = g_{\gamma} = h\).
  Since
  \begin{align*}
    \forall v \in \beta, f(v) & = h(\T(v))            &  & \text{(by definition)}       \\
                              & = (h \T)(v)                                             \\
                              & = (g \T)(v)           &  & \text{(by \cref{ex:2.6.18})} \\
                              & = \pa{\tp{\T}(g)}(v), &  & \text{(by \cref{2.25})}
  \end{align*}
  by \cref{ex:2.1.34} we know that \(f = \tp{\T}(g)\).
  Since \(f\) is arbitrary, we know that \(\tp{\T}\) is onto.
\end{proof}

\section{Homogeneous Linear Differential Equations with Constant Coefficients}\label{sec:2.7}

\begin{defn}\label{2.7.1}
  A \textbf{differential equation} in an unknown function \(y = y(t)\) is an equation involving \(y\), \(t\), and derivatives of \(y\).
  If the differential equation is of the form
  \begin{equation}\label{eq:2.7.1}
    a_n y^{(n)} + a_{n - 1} y^{(n - 1)} + \cdots + a_1 y^{(1)} + a_0 y = f,
  \end{equation}
  where \(\seq{a}{0,,n}\) and \(f\) are functions of \(t\) and \(y^{(k)}\) denotes the \(k\)th derivative of \(y\), then the equation is said to be \textbf{linear}.
  The functions \(a_i\) are called the \textbf{coefficients} of the differential equation \cref{eq:2.7.1}.
  When \(f\) is identically zero, \cref{eq:2.7.1} is called \textbf{homogeneous}.

  If \(a_n \neq 0\), we say that differential equation \cref{eq:2.7.1} is of \textbf{order \(n\)}.
  In this case, we divide both sides by \(a_n\) to obtain a new, but equivalent, equation
  \[
    y^{(n)} + b_{n - 1} y^{(n - 1)} + \cdots + b_1 y^{(1)} + b_0 y = \zv,
  \]
  where \(b_i = a_i / a_n\) for \(i \in \set{0, \dots, n - 1}\).
  Because of this observation, we always assume that the coefficient \(a_n\) in \cref{eq:2.7.1} is \(1\).

  A \textbf{solution} to \cref{eq:2.7.1} is a function that when substituted for \(y\) reduces \cref{eq:2.7.1} to an identity.
\end{defn}

\begin{defn}\label{2.7.2}
  In our study of differential equations, it is useful to regard solutions as complex-valued functions of a real variable even though the solutions that are meaningful to us in a physical sense are real-valued.
  The convenience of this viewpoint will become clear later.
  Thus we are concerned with the vector space \(\fs(\R, \C)\).
  In order to consider complex-valued functions of a real variable as solutions to differential equations, we must define what it means to differentiate such functions.
  Given a complex-valued function \(x \in \fs(\R, \C)\) of a real variable \(t\), there exist unique real-valued functions \(x_1\) and \(x_2\) of \(t\), such that
  \[
    x(t) = x_1(t) + i x_2(t) \quad \text{for} \quad t \in \R,
  \]
  where \(i\) is the imaginary number such that \(i^2 = -1\).
  We call \(x_1\) the \textbf{real part} and \(x_2\) the \textbf{imaginary part} of \(x\).
\end{defn}

\begin{defn}\label{2.7.3}
  Given a function \(x \in \fs(\R, \C)\) with real part \(x_1\) and imaginary part \(x_2\), we say that \(x\) is \textbf{differentiable} if \(x_1\) and \(x_2\) are differentiable.
  If \(x\) is differentiable, we define the derivative \(x'\) of \(x\) by
  \[
    x' = x_1' + i x_2'.
  \]
\end{defn}

\begin{thm}\label{2.27}
  Any solution to a homogeneous linear differential equation with constant coefficients has derivatives of all orders;
  that is, if \(x\) is a solution to such an equation, then \(x^{(k)}\) exists for every positive integer \(k\).
\end{thm}

\begin{proof}[\pf{2.27}]
  Let
  \[
    y^{(n)} + a_{n - 1} y^{(n - 1)} + \cdots + a_1 y^{(1)} + a_0 y = 0
  \]
  be a homogeneous linear differential equation of order \(n\) with constant coefficients.
  Clearly \(y^{(k)}\) exists for all \(k \in \set{0, \dots, n}\).
  Now we prove that \(y^{(k)}\) exists for all \(k \in \N \setminus \set{0, \dots, n}\).
  We can rewrite the equation above as
  \[
    y^{(n)} = -a_{n - 1} y^{(n - 1)} - \cdots - a_1 y^{(1)} - a_0 y.
  \]
  Since \(y^{(n)}\) exists, each term on the right hand side of the above equation can be differentiated at least one more time.
  Thus \(y^{(n + 1)}\) must exist and is equal to the follow:
  \[
    y^{(n + 1)} = (y^{(n)})' = -a_{n - 1} y^{(n)} - \cdots - a_1 y^{(2)} - a_0 y^{(1)}.
  \]
  But again \(y^{(n + 1)}\) exists, therefore each term on the right hand side of the above equation can be differentiated at least one more time.
  Thus \(y^{(n + 2)}\) must exist and is equal to the follow:
  \[
    y^{(n + 2)} = (y^{(n + 1)})' = -a_{n - 1} y^{(n + 1)} - \cdots - a_1 y^{(3)} - a_0 y^{(2)}.
  \]
  In general we see that for all \(k \in \N\), we have
  \[
    y^{(n + k)} = (y^{(n)})^{(k)} = -a_{n - 1} y^{(n - 1 + k)} - \cdots - a_1 y^{(1 + k)} - a_0 y^{(k)}.
  \]
\end{proof}

\begin{defn}\label{2.7.4}
  We use \(\cfs[\infty](\R, \C)\) to denote the set of all functions in \(\fs(\R, \C)\) that have derivatives of all orders.
  By \cref{ex:2.7.5} \(\cfs[\infty](\R, \C)\) is a subspace of \(\fs(\R, \C)\) over \(\C\) and hence is a vector space over \(\C\).
\end{defn}

\begin{defn}\label{2.7.5}
  For \(x \in \cfs[\infty](\R, \C)\), the derivative \(x'\) of \(x\) also lies in \(\cfs[\infty](\R, \C)\).
  We can use the derivative operation to define a mapping \(\Dop : \cfs[\infty](\R, \C) \to \cfs[\infty](\R, \C)\) by
  \[
    \Dop(x) = x' \quad \text{for } x \in \cfs[\infty](\R, \C).
  \]
  It is easy to show that \(\Dop\) is a linear operator.
  More generally, consider any polynomial over \(\C\) of the form
  \[
    p(t) = a_n t^n + a_{n - 1} t^{n - 1} + \cdots + a_1 t + a_0.
  \]
  If we define
  \[
    p(\Dop) = a_n \Dop^n + a_{n - 1} \Dop^{n - 1} + \cdots + a_1 \Dop + a_0 \IT,
  \]
  then \(p(\Dop)\) is a linear operator on \(\cfs[\infty](\R, \C)\).
  (See \cref{e.0.7,ex:2.7.6}.)

  For any polynomial \(p\) over \(\C\) of positive degree, \(p(\Dop)\) is called a \textbf{differential operator}.
  The \textbf{order} of the differential operator \(p(\Dop)\) is the degree of the polynomial \(p\).

  Differential operators are useful since they provide us with a means of reformulating a differential equation in the context of linear algebra.
  Any homogeneous linear differential equation with constant coefficients,
  \[
    y^{(n)} + a_{n - 1} y^{(n - 1)} + \cdots + a_1 y^{(1)} + a_0 y = \zv,
  \]
  can be rewritten using differential operators as
  \[
    \pa{\Dop^{(n)} + a_{n - 1} \Dop^{(n - 1)} + \cdots + a_1 \Dop^{(1)} + a_0 \IT}(y) = \zv.
  \]
  Given the differential equation above, the complex polynomial
  \[
    p(t) = t^n + a_{n - 1} t^{n - 1} + \cdots + a_1 t + a_0
  \]
  is called the \textbf{auxiliary polynomial} associated with the equation.
  Any homogeneous linear differential equation with constant coefficients can be rewritten as
  \[
    p(\Dop)(y) = \zv,
  \]
  where \(p(t)\) is the auxiliary polynomial associated with the equation.
\end{defn}

\begin{thm}\label{2.28}
  The set of all solutions to a homogeneous linear differential equation with constant coefficients coincides with the null space of \(p(\Dop)\), where \(p(t)\) is the auxiliary polynomial associated with the equation.
\end{thm}

\begin{proof}[\pf{2.28}]
  Let
  \[
    y^{(n)} + a_{n - 1} y^{(n - 1)} + \cdots + a_1 y^{(1)} + a_0 y = \zv
  \]
  be a homogeneous linear differential equation where \(\seq{a}{0,,n-1} \in \C\).
  By \cref{2.7.5}
  \[
    p(t) = t^n + a_{n - 1} t^{n - 1} + \cdots + a_1 t + a_0
  \]
  is the auxiliary polynomial associated with the above homogeneous linear differential equation.
  Then we have
  \begin{align*}
         & x \in \cfs[\infty](\R, \C) \text{ is a solution of }                                  \\
         & y^{(n)} + a_{n - 1} y^{(n - 1)} + \cdots + a_1 y^{(1)} + a_0 y = \zv                  \\
    \iff & x^{(n)} + a_{n - 1} x^{(n - 1)} + \cdots + a_1 x^{(1)} + a_0 x = \zv &  & \by{2.7.1}  \\
    \iff & p(\Dop)(x) = \zv                                                     &  & \by{2.7.5}  \\
    \iff & x \in \ns{p(\Dop)}                                                   &  & \by{2.1.10}
  \end{align*}
  and thus \(\set{x \in \cfs[\infty](\R, \C) : x^{(n)} + a_{n - 1} x^{(n - 1)} + \cdots + a_1 x^{(1)} + a_0 x = \zv} = \ns{p(\Dop)}\).
\end{proof}

\begin{cor}\label{2.7.6}
  The set of all solutions to a homogeneous linear differential equation with constant coefficients is a subspace of \(\cfs[\infty](\R, \C)\).
\end{cor}

\begin{proof}[\pf{2.7.6}]
  By \cref{2.1,2.28} we see that this is true.
\end{proof}

\begin{defn}\label{2.7.7}
  In view of \cref{2.7.6}, we call the set of solutions to a homogeneous linear differential equation with constant coefficients the \textbf{solution space} of the equation.
  A practical way of describing such a space is in terms of a basis.
\end{defn}

\begin{defn}\label{2.7.8}
  Let \(c = a + ib\) be a complex number with real part \(a\) and imaginary part \(b\).
  Define
  \[
    e^c = e^a (\cos(b) + i \sin(b)).
  \]
  The special case
  \[
    e^{ib} = \cos(b) + i \sin(b)
  \]
  is called \textbf{Euler's formula}.
  Clearly, if \(c\) is real (\(b = 0\)), then we obtain the usual result:
  \(e^c = e^a\).
  We can show by the use of trigonometric identities that
  \[
    e^{c + d} = e^c e^d \quad \text{and} \quad e^{-c} = \dfrac{1}{e^c}
  \]
  for any complex number \(c\) and \(d\).
\end{defn}

\begin{defn}\label{2.7.9}
  A function \(f : \R \to \C\) defined by \(f(t) = e^{ct}\) for a fixed complex number \(c\) is called an \textbf{exponential function}.
\end{defn}

\begin{thm}\label{2.29}
  For any exponential function \(f(t) = e^{ct}\), \(f'(t) = c e^{ct}\).
\end{thm}

\begin{proof}[\pf{2.29}]
  We have
  \begin{align*}
    f'(t) & = (e^{ct})'                                                                                                 \\
          & = (e^{\Re(ct) + i \Im(ct)})'                                                      &  & \by{2.7.2}           \\
          & = (e^{\Re(ct)} \cdot (\cos(\Im(ct)) + i \cdot \sin(\Im(ct))))'                    &  & \by{2.7.8}           \\
          & = (e^{\Re(ct)} \cdot \cos(\Im(ct)) + i \cdot e^{\Re(ct)} \cdot \sin(\Im(ct)))'                              \\
          & = (e^{\Re(ct)} \cdot \cos(\Im(ct)))' + i \cdot (e^{\Re(ct)} \cdot \sin(\Im(ct)))' &  & \by{2.7.3}           \\
          & = e^{\Re(ct)} \cdot (\Re(ct))' \cdot \cos(\Im(ct))                                                          \\
          & \quad - e^{\Re(ct)} \cdot \sin(\Im(ct)) \cdot (\Im(ct))'                                                    \\
          & \quad + i \cdot e^{\Re(ct)} \cdot (\Re(ct))' \cdot \sin(\Im(ct))                                            \\
          & \quad + i \cdot e^{\Re(ct)} \cdot \cos(\Im(ct)) \cdot (\Im(ct))'                                            \\
          & = e^{\Re(ct)} \cdot \Re(c) \cdot \cos(\Im(ct))                                    &  & \by{2.7.3}           \\
          & \quad - e^{\Re(ct)} \cdot \sin(\Im(ct)) \cdot \Im(c)                              &  & (\Re(ct) = t \Re(c)) \\
          & \quad + i \cdot e^{\Re(ct)} \cdot \Re(c) \cdot \sin(\Im(ct))                      &  & (\Im(ct) = t \Im(c)) \\
          & \quad + i \cdot e^{\Re(ct)} \cdot \cos(\Im(ct)) \cdot \Im(c)                                                \\
          & = e^{\Re(ct)} \cdot \Re(c) \cdot (\cos(\Im(ct)) + i \sin(\Im(ct)))                                          \\
          & \quad + e^{\Re(ct)} \cdot \Im(c) \cdot (i \cos(\Im(ct)) - \sin(\Im(ct)))                                    \\
          & = e^{\Re(ct)} \cdot \Re(c) \cdot (\cos(\Im(ct)) + i \sin(\Im(ct)))                                          \\
          & \quad + e^{\Re(ct)} \cdot i \cdot \Im(c) \cdot (\cos(\Im(ct)) + i \sin(\Im(ct)))                            \\
          & = e^{\Re(ct)} \cdot (\Re(c) + i \Im(c)) \cdot (\cos(\Im(ct)) + i \sin(\Im(ct)))                             \\
          & = c \cdot e^{ct}.                                                                 &  & \by{2.7.8}
  \end{align*}
\end{proof}

\begin{thm}\label{2.30}
  The solution space for
  \[
    y' + a_0 y = \zv
  \]
  (homogeneous linear differential equation with order \(1\)) is of dimension \(1\) and has \(\set{e^{-a_0 t}}\) as a basis.
\end{thm}

\begin{proof}[\pf{2.30}]
  Since
  \begin{align*}
    (e^{-a_0 t})' + a_0 e^{-a_0 t} & = -a_0 e^{-a_0 t} + a_0 e^{-a_0 t} &  & \by{2.29} \\
                                   & = 0,
  \end{align*}
  we know that \(e^{-a_0 t}\) is a solution of \(y' + a_0 y = \zv\).
  Suppose that \(x(t)\) is any solution to \(y' + a_0 y = \zv\).
  Then
  \[
    \forall t \in \R, x'(t) = -a_0 x(t).
  \]
  Define
  \[
    \forall t \in \R, z(t) = e^{a_0 t} x(t).
  \]
  Differentiating \(z\) yields
  \[
    \forall t \in \R, z'(t) = (e^{a_0 t})' x(t) + e^{a_0 t} x'(t) = a_0 e^{a_0 t} x(t) - a_0 e^{a_0 t} x(t) = 0.
  \]
  (Notice that the familiar product rule for differentiation holds for complex-valued functions of a real variable.
  A justification of this involves a lengthy, although direct, computation.)

  Since \(z'\) is identically zero, \(z\) is a constant function.
  (Again, this fact, well known for real-valued functions, is also true for complex-valued functions.
  The proof, which relies on the real case, involves looking separately at the real and imaginary parts of \(z\).)
  Thus there exists a complex number \(k\) such that
  \[
    \forall t \in \R, z(t) = e^{a_0 t} x(t) = k
  \]
  So
  \[
    x(t) = k e^{-a_0 t}.
  \]
  We conclude that any solution to \(y' + a_0 y = \zv\) is a linear combination of \(e^{-a_0 t}\).
\end{proof}

\begin{cor}\label{2.7.10}
  For any complex number \(c\), the null space of the differential operator \(\Dop - c \IT\) has \(\set{e^{ct}}\) as a basis.
\end{cor}

\begin{proof}[\pf{2.7.10}]
  This is simply another way of stating \cref{2.30}.
\end{proof}

\begin{thm}\label{2.31}
  Let \(p\) be the auxiliary polynomial for a homogeneous linear differential equation with constant coefficients.
  For any complex number \(c\), if \(c\) is a zero of \(p\), then \(e^{ct}\) is a solution to the differential equation.
\end{thm}

\begin{proof}[\pf{2.31}]
  Given an \(n\)th order homogeneous linear differential equation with constant coefficients,
  \[
    y^{(n)} + a_{n - 1} y^{(n - 1)} + \cdots + a_1 y^{(1)} + a_0 y = \zv,
  \]
  its auxiliary polynomial
  \[
    p(t) = t^n + a_{n - 1} t^{n - 1} + \cdots + a_1 t + a_0
  \]
  factors into a product of polynomials of degree \(1\), that is,
  \[
    p(t) = (t - c_1) (t - c_2) \cdots (t - c_n),
  \]
  where \(\seq{c}{1,,n}\) are (not necessarily distinct) complex numbers.
  (This follows from the fundamental theorem of algebra, see \cref{d.0.7}.)
  Thus
  \[
    p(\Dop) = (\Dop - c_1 \IT) (\Dop - c_2 \IT) \cdots (\Dop - c_n \IT).
  \]
  The operators \(\Dop - c_i \IT\) commute, and so, by \cref{ex:2.7.9}, we have that
  \[
    \ns{\Dop - c_i \IT} \subseteq \ns{p(\Dop)} \quad \text{for all } i \in \set{1, \dots, n}.
  \]
  Since \(\ns{p(\Dop)}\) coincides with the solution space of the given differential equation, we conclude by \cref{2.7.10} that \cref{2.31} is true.
\end{proof}

\begin{lem}\label{2.7.11}
  The differential operator \(\Dop - c \IT : \cfs[\infty](\R, \C) \to \cfs[\infty](\R, \C)\) is onto for any complex number \(c\).
\end{lem}

\begin{proof}[\pf{2.7.11}]
  Let \(v \in \cfs[\infty](\R, \C)\).
  We wish to find a \(u \in \cfs[\infty](\R, \C)\) such that \((\Dop - c \IT) u = v\).
  Let \(w(t) = v(t) e^{-ct}\) for \(t \in \R\).
  Clearly, \(w \in \cfs[\infty](\R, \C)\) because both \(v\) and \(e^{-ct}\) lie in \(\cfs[\infty](\R, \C)\).
  Let \(w_1\) and \(w_2\) be the real and imaginary parts of \(w\).
  Then \(w_1\) and \(w_2\) are continuous because they are differentiable.
  Hence they have antiderivatives, say, \(W_1\) and \(W_2\), respectively.
  Let \(W : \R \to \C\) be defined by
  \[
    \forall t \in \R, W(t) = W_1(t) + i W_2(t).
  \]
  Then \(W \in \cfs[\infty](\R, \C)\), and the real and imaginary parts of \(W\) are \(W_1\) and \(W_2\), respectively.
  Furthermore, \(W' = w\).
  Finally, let \(u : \R \to \C\) be defined by \(u(t) = W(t) e^{ct}\) for \(t \in \R\).
  Clearly \(u \in \cfs[\infty](\R, \C)\), and since
  \begin{align*}
    (\Dop - c \IT) u(t) & = u'(t) - cu(t)                                               \\
                        & = W'(t) e^{ct} + W(t) c e^{ct} - c W(t) e^{ct} &  & \by{2.29} \\
                        & = w(t) e^{ct}                                                 \\
                        & = v(t) e^{-ct} e^{ct}                                         \\
                        & = v(t),
  \end{align*}
  we have \((\Dop - c \IT) u = v\).
\end{proof}

\begin{lem}\label{2.7.12}
  Let \(\V\) be a vector space over \(\F\), and suppose that \(\T\) and \(\U\) are linear operators on \(\V\) such that \(\U\) is onto and the null spaces of \(\T\) and \(\U\) are finite-dimensional.
  Then the null space of \(\T \U\) is finite-dimensional, and
  \[
    \dim(\ns{\T \U}) = \dim(\ns{\T}) + \dim(\ns{\U}).
  \]
\end{lem}

\begin{proof}[\pf{2.7.12}]
  Let \(p = \dim(\ns{\T})\), \(q = \dim(\ns{\U})\), and \(\set{\seq{u}{1,,p}}\) and \(\set{\seq{v}{1,,q}}\) be bases for \(\ns{\T}\) and \(\ns{\U}\) over \(\F\), respectively.
  Since \(\U\) is onto, we can choose for each \(i \in \set{1, \dots, p}\) a vector \(w_i \in \V\) such that \(\U(w_i) = u_i\).
  Note that the \(w_i\)'s are distinct.
  Furthermore, for any \(i\) and \(j\), \(w_i \neq v_j\), for otherwise \(u_i = \U(w_i) = \U(v_j) = \zv\) --- a contradiction.
  Hence the set
  \[
    \beta = \set{\seq{w}{1,,p}, \seq{v}{1,,q}}
  \]
  contains \(p + q\) distinct vectors.
  To complete the proof of the lemma, it suffices to show that \(\beta\) is a basis for \(\ns{\T \U}\) over \(\F\).

  We first show that \(\beta\) generates \(\ns{\T \U}\).
  Since for any \(w_i\) and \(v_j\) in \(\beta\), \(\T \U(w_i) = \T(u_i) = \zv\) and \(\T \U(v_j) = \T(\zv) = \zv\), it follows that \(\beta \subseteq \ns{\T \U}\).
  Now suppose that \(v \in \ns{\T \U}\).
  Then \(\zv = \T \U(v) = \T(\U(v))\).
  Thus \(\U(v) \in \ns{\T}\).
  So there exist scalars \(\seq{a}{1,,p} \in \F\) such that
  \begin{align*}
    \U(v) & = \seq[+]{a,u}{1,,p}                 \\
          & = a_1 \U(w_1) + \cdots + a_p \U(w_p) \\
          & = \U(\seq[+]{a,w}{1,,p}).
  \end{align*}
  Hence
  \[
    \U(v - (\seq[+]{a,w}{1,,p})) = \zv.
  \]
  Consequently, \(v - (\seq[+]{a,w}{1,,p})\) lies in \(\ns{\U}\).
  It follows that there exist scalars \(\seq{b}{1,,q} \in \F\) such that
  \[
    v - (\seq[+]{a,w}{1,,p}) = \seq[+]{b,v}{1,,q}
  \]
  or
  \[
    v = \seq[+]{a,w}{1,,p} + \seq[+]{b,v}{1,,q}.
  \]
  Therefore \(\beta\) spans \(\ns{\T \U}\).

  To prove that \(\beta\) is linearly independent, let \(\seq{a}{1,,p}, \seq{b}{1,,q} \in \F\) be any scalars such that
  \[
    \seq[+]{a,w}{1,,p} + \seq[+]{b,v}{1,,q} = \zv.
  \]
  Applying \(\U\) to both sides of the above equation, we obtain
  \[
    \seq[+]{a,u}{1,,p} = \zv.
  \]
  Since \(\set{\seq{u}{1,,p}}\) is linearly independent, the \(a_i\)'s are all zero.
  Thus the original equation reduces to
  \[
    \seq[+]{b,v}{1,,q} = \zv.
  \]
  Again, the linear independence of \(\set{\seq{v}{1,,q}}\) implies that the \(b_i\)'s are all zero.
  We conclude that \(\beta\) is a basis for \(\ns{\T \U}\) over \(\F\).
  Hence \(\ns{\T \U}\) is finite-dimensional, and \(\dim(\ns{\T \U}) = p + q = \dim(\ns{\T}) + \dim(\ns{\U})\).
\end{proof}

\begin{thm}\label{2.32}
  For any differential operator \(p(\Dop)\) of order \(n\), the null space of \(p(\Dop)\) is an \(n\)-dimensional subspace of \(\cfs[\infty](\R, \C)\).
\end{thm}

\begin{proof}[\pf{2.32}]
  The proof is by mathematical induction on the order of the differential operator \(p(\Dop)\).
  The first-order case coincides with \cref{2.30}.
  For some integer \(n > 1\), suppose that \cref{2.32} holds for any differential operator of order less than \(n\), and consider a differential operator \(p(\Dop)\) of order \(n\).
  The polynomial \(p\) can be factored into a product of two polynomials as follows:
  \[
    \forall t \in \R, p(t) = q(t) (t - c),
  \]
  where \(q\) is a polynomial of degree \(n - 1\) and \(c\) is a complex number.
  Thus the given differential operator may be rewritten as
  \[
    p(\Dop) = q(\Dop) (\Dop - c \IT).
  \]
  Now, by \cref{2.7.11}, \(\Dop - c \IT\) is onto, and by \cref{2.7.10}, \(\dim(\ns{\Dop - c \IT}) = 1\).
  Also, by the induction hypothesis, \(\dim(\ns{q(\Dop)}) = n - 1\).
  Thus, by \cref{2.7.12} we conclude that
  \[
    \dim(\ns{p(\Dop)}) = \dim(\ns{q(\Dop)}) + \dim(\ns{\Dop - c \IT}) = (n - 1) + 1 = n.
  \]
\end{proof}

\begin{cor}\label{2.7.13}
  The solution space of any \(n\)th-order homogeneous linear differential equation with constant coefficients is an \(n\)-dimensional subspace of \(\cfs[\infty](\R, \C)\).
\end{cor}

\begin{proof}[\pf{2.7.13}]
  By \cref{2.28,2.32} we conclude that \cref{2.7.13} is true.
\end{proof}

\begin{note}
  \cref{2.7.13} reduces the problem of finding all solutions to an \(n\)th-order homogeneous linear differential equation with constant coefficients to finding a set of \(n\) linearly independent solutions to the equation.
  By the results of \cref{ch:1}, any such set must be a basis for the solution space.
\end{note}

\begin{thm}\label{2.33}
  Given \(n\) distinct complex numbers \(\seq{c}{1,,n}\), the set of exponential functions \(\set{e^{c_1 t}, \dots, e^{c_n t}}\) is linearly independent.
\end{thm}

\begin{proof}[\pf{2.33}]
  We use induction on \(n\).
  For \(n = 1\), let \(b_1, c_1 \in \C\) such that
  \[
    \forall t \in \R, b_1 e^{c_1 t} = 0.
  \]
  By substituting \(t\) with \(0\) we have \(b_1 e^{c_1 0} = b_1 e^0 = b_1 = 0\).
  Thus by \cref{1.5.3} we know that the set \(\set{e^{c_1 t}}\) is linearly independent and the base case holds.
  Suppose inductively that \cref{2.33} is true for some \(n \geq 1\).
  We want to show that it is also true for \(n + 1\).
  Let \(\seq{c}{1,,n+1} \in \C\) be distinct and let \(\seq{b}{1,,n+1} \in \C\) such that
  \[
    \forall t \in \R, \sum_{i = 1}^{n + 1} b_i e^{c_i t} = 0.
  \]
  By applying the operator \(\Dop - c_{n + 1} \IT\) we have
  \begin{align*}
             & \forall t \in \R, \sum_{i = 1}^{n + 1} b_i c_i e^{c_i t} - \sum_{i = 1}^{n + 1} b_i c_{n + 1} e^{c_i t} = 0                                             \\
    \implies & \forall t \in \R, \sum_{i = 1}^n b_i c_i e^{c_i t} + b_{n + 1} c_{n + 1} e^{c_{n + 1} t}                                                                \\
             & - \sum_{i = 1}^n b_i c_{n + 1} e^{c_i t} - b_{n + 1} c_{n + 1} e^{c_{n + 1} t} = 0                                                                      \\
    \implies & \forall t \in \R, \sum_{i = 1}^n b_i c_i e^{c_i t} - \sum_{i = 1}^n b_i c_{n + 1} e^{c_i t} = 0                                                         \\
    \implies & \forall t \in \R, \sum_{i = 1}^n b_i (c_i - c_{n + 1}) e^{c_i t} = 0                                                                                    \\
    \implies & \seq[=]{b}{1,,n} = 0                                                                                        &  & \text{(by induction hypothesis)}       \\
    \implies & \forall t \in \R, \sum_{t = 1}^{n + 1} b_i e^{c_i t} = b_{n + 1} e^{c_{n + 1} t} = 0                                                                    \\
    \implies & b_{n + 1} = 0.                                                                                              &  & \text{(substituting \(t\) with \(0\))}
  \end{align*}
  Thus by \cref{1.5.3} we know that the set \(\set{e^{c_1 t}, \dots, e^{c_{n + 1} t}}\) is linearly independent and this closes the induction.
\end{proof}

\begin{cor}\label{2.7.14}
  For any \(n\)th-order homogeneous linear differential equation with constant coefficients, if the auxiliary polynomial has \(n\) distinct zeros \(\seq{c}{1,,n}\), then \(\set{e^{c_1 t}, \dots, e^{c_n t}}\) is a basis for the solution space of the differential equation.
\end{cor}

\begin{proof}[\pf{2.7.14}]
  By \cref{2.7.13,2.33} we see that this is true.
\end{proof}

\begin{lem}\label{2.7.15}
  For a given complex number \(c\) and positive integer \(n\), suppose that \((t - c)^n\) is the auxiliary polynomial of a homogeneous linear differential equation with constant coefficients.
  Then the set
  \[
    \beta = \set{e^{ct}, t e^{ct}, \dots, t^{n - 1} e^{ct}}
  \]
  is a basis for the solution space of the equation.
\end{lem}

\begin{proof}[\pf{2.7.15}]
  Since the solution space is \(n\)-dimensional, we only need to show that \(\beta\) is linearly independent and lies in the solution space.
  First, observe that for any positive integer \(k\),
  \[
    (\Dop - c \IT)(t^k e^{ct}) = k t^{k - 1} e^{ct} + c t^k e^{ct} - c t^k e^{ct} = k t^{k - 1} e^{ct}.
  \]
  Hence for \(k < n\),
  \[
    (\Dop - c \IT)^n (t^k e^{ct}) = \zv.
  \]
  It follows that \(\beta\) is a subset of the solution space.

  We next show that \(\beta\) is linearly independent.
  Consider any linear combination of vectors in \(\beta\) such that
  \[
    b_0 e^{ct} + b_1 t e^{ct} + \cdots + b_{n - 1} t^{n - 1} e^{ct} = \zv
  \]
  for some scalars \(\seq{b}{0,,n-1}\).
  Dividing by \(e^{ct}\) in the above equation, we obtain
  \[
    b_0 + b_1 t + \cdots + b_{n - 1} t^{n - 1} = \zv.
  \]
  Thus the left side of the above equation must be the zero polynomial function.
  We conclude that the coefficients \(\seq{b}{0,,n-1}\) are all zero.
  So \(\beta\) is linearly independent and hence is a basis for the solution space.
\end{proof}

\begin{thm}\label{2.34}
  Given a homogeneous linear differential equation with constant coefficients and auxiliary polynomial
  \[
    p(\Dop) = (t - c_1)^{n_1} (t - c_2)^{n_2} \cdots (t - c_k)^{n_k}
  \]
  where \(\seq{n}{1,,k}\) are positive integers and \(\seq{c}{1,,k}\) are distinct complex numbers, the following set is a basis for the solution space of the equation:
  \[
    \beta = \set{e^{c_1 t}, t e^{c_1 t}, \dots, t^{n_1 - 1} e^{c_1 t}, \dots, e^{c_k t}, t e^{c_k t}, \dots, t^{n_k - 1} e^{c_k t}}.
  \]
\end{thm}

\begin{proof}[\pf{2.34}]
  For any \(k \in \Z^+\), let \(\beta_k = \set{e^{c_k t}, \dots, t^{n_k - 1} e^{c_k t}}\).
  Observe that \(\beta = \bigcup_{i = 1}^k \beta_i\) and \(\#(\beta) = \seq[+]{n}{1,,k} = \sum_{i = 1}^k \#(\beta_i)\).
  Thus by \cref{2.7.13} we only need to show that \(\bigcup_{i = 1}^k \beta_i \subseteq \ns{p(\Dop)}\) and \(\bigcup_{i = 1}^k \beta_i\) is linearly independent.

  First we show that \(\bigcup_{i = 1}^k \beta_i \subseteq \ns{p(\Dop)}\).
  This is true since
  \begin{align*}
             & \forall i \in \set{1, \dots, k}, \beta_i \subseteq \ns{(\Dop - c_i \IT)^{n_i}} &  & \by{2.7.14}   \\
    \implies & \forall i \in \set{1, \dots, k}, \beta_i \subseteq \ns{p(\Dop)}                &  & \by{ex:2.7.9} \\
    \implies & \bigcup_{i = 1}^k \beta_i \subseteq \ns{p(\Dop)}.
  \end{align*}

  Now we show that \(\bigcup_{i = 1}^k \beta_i\) is linearly independent.
  We use induction on \(k\).
  For \(k = 1\), we have
  \[
    p(\Dop) = (t - c_1)^{n_1} \quad \text{and} \quad \beta_1 = \set{e^{c_1 t}, t e^{c_1 t}, \dots, t^{n_1 - 1} e^{c_1 t}}.
  \]
  By \cref{2.7.15} we know that \(\beta_1\) is linearly independent and thus the base case holds.
  Suppose inductively that \(\bigcup_{i = 1}^k \beta_i\) is linearly independent for some \(k \geq 1\).
  We need to show that when \(\bigcup_{i = 1}^{k + 1} \beta_i\) is also linearly independent.
  For each \(i \in \set{1, \dots, k + 1}\) and \(j \in \set{0, \dots, n_i - 1}\), let \(b_{i j} \in \C\) such that
  \[
    \sum_{i = 1}^{k + 1} \sum_{j = 0}^{n_i - 1} b_{i j} t^j e^{c_i t} = \zv.
  \]
  Observe that
  \begin{align*}
             & (\Dop - c_{k + 1} \IT)(t^j e^{c_i t}) = j t^{j - 1} e^{c_i t} + (c_i - c_{k + 1}) t^j e^{c_i t}                                                                                                             \\
    \implies & (\Dop - c_{k + 1} \IT)^{n_{k + 1}} \pa{t^j e^{c_i t}} = \sum_{q = 0}^{\min(n_{k + 1} - 1, j)} \binom{n_{k + 1}}{q} (c_i - c_{k + 1})^{n_{k + 1} - q} \dfrac{j!}{(j - q)!} t^{j - q} e^{c_i t}               \\
    \implies & \zv = (\Dop - c_{k + 1} \IT)^{n_{k + 1}} \pa{\sum_{i = 1}^{k + 1} \sum_{j = 0}^{n_i - 1} b_{i j} t^j e^{c_i t}} = \sum_{i = 1}^{k + 1} \sum_{j = 0}^{n_i - 1} b_{i j} (\Dop - c_{k + 1} \IT)(t^j e^{c_i t}) \\
             & = \sum_{i = 1}^{k + 1} \sum_{j = 0}^{n_i - 1} \sum_{q = 0}^{\min(n_{k + 1} - 1, j)} b_{i j} \binom{n_{k + 1}}{q} (c_i - c_{k + 1})^{n_{k + 1} - q} \dfrac{j!}{(j - q)!} t^{j - q} e^{c_i t}                 \\
             & = \sum_{i = 1}^k \sum_{j = 0}^{n_i - 1} \sum_{q = 0}^{\min(n_{k + 1} - 1, j)} b_{i j} \binom{n_{k + 1}}{q} (c_i - c_{k + 1})^{n_{k + 1} - q} \dfrac{j!}{(j - q)!} t^{j - q} e^{c_i t}.
  \end{align*}
  Note that the last line follow since \(q \leq n_{k + 1} - 1 < n_{k + 1}\).
  Since \(c_i - c_{k + 1} \neq 0\) for all \(i \in \set{1, \dots, k}\) and factorials are positive, by induction hypothesis we must have \(b_{i j} = 0\) for all \(i \in \set{1, \dots, k}\) and \(j \in \set{0, \dots n_i - 1}\).
  Thus we have
  \[
    \zv = \sum_{i = 1}^{k + 1} \sum_{j = 0}^{n_i - 1} b_{i j} t^j e^{c_i t} = \sum_{j = 0}^{n_{k + 1} - 1} b_{(k + 1) j} t^j e^{c_{k + 1} t}.
  \]
  By \cref{2.7.15} we know that \(b_{(k + 1) j} = 0\) for all \(j \in \set{0, \dots, n_{k + 1} - 1}\).
  Thus by \cref{1.5.3} \(\bigcup_{i = 1}^{k + 1} \beta_i\) is linearly independent and this closes the induction.
\end{proof}

\exercisesection

\setcounter{ex}{4}
\begin{ex}\label{ex:2.7.5}
  Show that \(\cfs[\infty](\R, \C)\) is a subspace of \(\fs(\R, \C)\).
\end{ex}

\begin{proof}[\pf{ex:2.7.5}]
  Clearly we have \(\cfs[\infty](\R, \C) \subseteq \fs(\R, \C)\).
  Since the zero function \(\zv\) is continously differentiable and \(\zv' = \zv\), we know that \(\zv \in \cfs[\infty](\R, \C)\).
  Thus by \cref{ex:1.3.18} we only need to show that \(cf + g \in \cfs[\infty](\R, \C)\) for any \(f, g \in \cfs[\infty](\R, \C)\) and \(c \in \C\).
  Since
  \[
    \forall k \in \N, (cf + g)^{(k)} = c f^{(k)} + g^{(k)},
  \]
  we see that \(cf + g \in \cfs[\infty](\R, \C)\).
\end{proof}

\begin{ex}\label{ex:2.7.6}
  \begin{enumerate}
    \item Show that \(\Dop : \cfs[\infty](\R, \C) \to \cfs[\infty](\R, \C)\) is a linear operator.
    \item Show that any differential operator is a linear operator on \(\cfs[\infty](\R, \C)\).
  \end{enumerate}
\end{ex}

\begin{proof}[\pf{ex:2.7.6}(a)]
  Let \(f, g \in \cfs[\infty](\R, \C)\) and let \(c \in \C\).
  Since
  \begin{align*}
    \Dop(cf + g) & = (cf + g)'            &  & \by{2.7.5} \\
                 & = c f' + g'                            \\
                 & = c \Dop(f) + \Dop(g), &  & \by{2.7.5}
  \end{align*}
  by \cref{2.1.2}(b) we know that \(\Dop \in \ls(\cfs[\infty](\R, \C))\).
\end{proof}

\begin{proof}[\pf{ex:2.7.6}(b)]
  Let
  \[
    p(\Dop) = a_n \Dop^n + \cdots + a_1 \Dop + a_0 \IT
  \]
  be a differential operator where \(\seq{a}{0,,n} \in \C\).
  Let \(f, g \in \cfs[\infty](\R, \C)\) and let \(c \in \C\).
  Since
  \begin{align*}
     & p(\Dop)(cf + g)                                                                                                       \\
     & = \pa{a_n \Dop^n + \cdots + a_1 \Dop + a_0 \IT}(cf + g)                                               &  & \by{2.7.5} \\
     & = a_n \Dop^n(cf + g) + \cdots + a_1 \Dop(cf + g) + a_0 \IT(cf + g)                                    &  & \by{2.2.5} \\
     & = a_n (cf + g)^{(n)} + \cdots + a_1 (cf + g)^{(1)} + a_0 (cf + g)                                                     \\
     & = c a_n f^{(n)} + a_n g^{(n)} + \cdots + c a_1 f^{(1)} + a_1 g^{(1)} + c a_0 f + a_0 g                                \\
     & = c \pa{a_n f^{(n)} + \cdots + a_1 f^{(1)} + a_0 f} + \pa{a_n g^{(n)} + \cdots + a_1 g^{(1)} + a_0 g}                 \\
     & = c p(\Dop)(f) + p(\Dop)(g),                                                                          &  & \by{2.7.5}
  \end{align*}
  by \cref{2.1.2}(b) we know that \(p(\Dop) \in \ls(\cfs[\infty](\R, \C))\).
\end{proof}

\begin{ex}\label{ex:2.7.7}
  Prove that if \(\set{x, y}\) is a basis for a vector space \(\V\) over \(\C\), then so is
  \[
    \set{\dfrac{1}{2} (x + y), \dfrac{1}{2i} (x - y)}.
  \]
\end{ex}

\begin{proof}[\pf{ex:2.7.7}]
  Let \(v \in \V\).
  By \cref{1.6.1} there exist \(\seq{c}{1,2} \in \C\) such that \(v = c_1 x + c_2 y\).
  Then we have
  \begin{align*}
    v & = c_1 x + c_2 y                                                                          \\
      & = \dfrac{c_1}{2} x + \dfrac{c_2}{2} y + \dfrac{c_1 i}{2 i} x - \dfrac{-c_2 i}{2 i} y     \\
      & \quad + \dfrac{c_2}{2} x + \dfrac{c_1}{2} y - \dfrac{c_2 i}{2 i} x - \dfrac{c_1 i}{2i} y \\
      & = \pa{\dfrac{c_1 + c_2}{2}} (x + y) + i \pa{\dfrac{c_1 - c_2}{2i}} (x - y).
  \end{align*}
  Thus by \cref{1.6.15}(a) we know that \(\set{\dfrac{1}{2} (x + y), \dfrac{1}{2i} (x - y)}\) is a basis for \(\V\) over \(\C\).
\end{proof}

\begin{ex}\label{ex:2.7.8}
  Consider a second-order homogeneous linear differential equation with constant coefficients in which the auxiliary polynomial has distinct conjugate complex roots \(a + ib\) and \(a - ib\), where \(a, b \in \R\).
  Show that \(\set{e^{at} \cos(bt), e^{at} \sin(bt)}\) is a basis for the solution space.
\end{ex}

\begin{proof}[\pf{ex:2.7.8}]
  By \cref{2.7.14} we know that \(\set{e^{(a + ib) t}, e^{(a - ib) t}}\) is a basis for the solution space.
  Since
  \begin{align*}
    e^{(a + ib) t} & = e^{at + ibt}                                     \\
                   & = e^{at} (\cos(bt) + i \sin(bt))   &  & \by{2.7.8} \\
    e^{(a - ib) t} & = e^{at - ibt}                                     \\
                   & = e^{at} (\cos(-bt) + i \sin(-bt)) &  & \by{2.7.8} \\
                   & = e^{at} (\cos(bt) - i \sin(bt))
  \end{align*}
  By \cref{ex:2.7.7} we know that the set
  \[
    \set{\dfrac{1}{2} (e^{(a + ib) t} + e^{(a - ib) t}), \dfrac{1}{2i} (e^{(a + ib) t} - e^{(a - ib) t})} = \set{e^{at} \cos(bt), e^{at} \sin(bt)}
  \]
  is also a basis for the solution space.
\end{proof}

\begin{ex}\label{ex:2.7.9}
  Suppose that \(\set{\seq{\U}{1,,n}}\) is a collection of pairwise commutative linear operators on a vector space \(\V\) over \(\F\)
  (i.e., operators such that \(\U_i \U_j = \U_j \U_i\) for all \(i, j \in \set{1, \dots, n}\)).
  Prove that, for any \(i \in \set{1, \dots, n}\),
  \[
    \ns{\U_i} \subseteq \ns{\U_1 \cdots \U_n}.
  \]
\end{ex}

\begin{proof}[\pf{ex:2.7.9}]
  Let \(i \in \set{1, \dots, n}\) and let \(x \in \ns{\U_i}\).
  Then we have
  \begin{align*}
             & \U_i(x) = \zv                                                                   &  & \by{2.1.10}                   \\
    \implies & (\U_1 \cdots \U_n)(x) = (\U_1 \cdots \U_{i - 1} \U_{i + 1} \cdots \U_n \U_i)(x) &  & \text{(pairwise commutative)} \\
             & = (\U_1 \cdots \U_{i - 1} \U_{i + 1} \cdots \U_n)(\U_i(x))                                                         \\
             & = (\U_1 \cdots \U_{i - 1} \U_{i + 1} \cdots \U_n)(\zv)                                                             \\
             & = \zv                                                                           &  & \by{2.1.2}[a]                 \\
    \implies & x \in \ns{\U_1 \cdots \U_n}                                                     &  & \by{2.1.10}
  \end{align*}
  and thus \(\ns{\U_i} \subseteq \ns{\U_1 \cdots \U_n}\).
\end{proof}

\setcounter{ex}{11}
\begin{ex}\label{ex:2.7.12}
  Let \(\V\) be the solution space of an \(n\)th-order homogeneous linear differential equation with constant coefficients having auxiliary polynomial \(p\).
  Prove that if \(p(t)\) = \(g(t) h(t)\) for all \(t \in \R\), where \(g\) and \(h\) are polynomials of positive degree, then
  \[
    \ns{h(\Dop)} = \rg{g(\Dop_{\V})} = g(\Dop)(\V),
  \]
  where \(\Dop_{\V} : \V \to \V\) is defined by \(\Dop_{\V}(x) = x'\) for \(x \in \V\).
\end{ex}

\begin{proof}[\pf{ex:2.7.12}]
  By \cref{2.28} we have \(\V = \ns{p(\Dop)}\).
  By \cref{e.0.7} we have \(\rg{g(\Dop_{\V})} = g(\Dop)(\V) = g(\Dop)(\ns{p(\Dop)})\).
  Since
  \begin{align*}
             & y \in g(\Dop)(\ns{p(\Dop)})                                                    \\
    \implies & \exists x \in \ns{p(\Dop)} : g(\Dop)(x) = y                                    \\
    \implies & \exists x \in \ns{p(\Dop)} : h(\Dop)(y) = h(\Dop)(g(\Dop)(x))                  \\
             & = (h(\Dop) g(\Dop))(x) = p(\Dop)(x) = \zv                     &  & \by{2.1.10} \\
    \implies & y \in \ns{h(\Dop)},                                           &  & \by{2.1.10}
  \end{align*}
  we know that \(g(\Dop)(\ns{p(\Dop)}) \subseteq \ns{h(\Dop)}\).
  If we can show that \(\dim(g(\Dop)(\ns{p(\Dop)})) = \dim(\ns{h(\Dop)})\), then by \cref{1.11} we can prove that \(g(\Dop)(\ns{p(\Dop)}) = \ns{h(\Dop)}\).
  This is true since
  \begin{align*}
     & \dim(g(\Dop)(\ns{p(\Dop)}))                                                     \\
     & = \rk{g(\Dop_{\V})}                                            &  & \by{2.1.10} \\
     & = \dim(\ns{p(\Dop)}) - \nt{g(\Dop_{\V})}                       &  & \by{2.3}    \\
     & = (\text{order of } p(\Dop)) - (\text{order of } g(\Dop_{\V})) &  & \by{2.32}   \\
     & = (\text{degree of } p) - (\text{degree of } g)                                 \\
     & = \text{order of } h(\Dop)                                                      \\
     & = \dim(\ns{h(\Dop)}).                                          &  & \by{2.32}
  \end{align*}
\end{proof}

\begin{defn}\label{2.7.16}
  A differential equation
  \[
    y^{(n)} + a_{n - 1} y^{(n - 1)} + \cdots + a_1 y^{(1)} + a_0 y = x
  \]
  is called a \textbf{nonhomogeneous} linear differential equation with constant coefficients if the \(a_i\)'s are constant and \(x\) is a function that is not identically zero.
\end{defn}

\begin{ex}\label{ex:2.7.13}
  Let
  \[
    y^{(n)} + a_{n - 1} y^{(n - 1)} + \cdots + a_1 y^{(1)} + a_0 y = x
  \]
  be a nonhomogeneous linear differential equation with constant coefficients.
  \begin{enumerate}
    \item Prove that for any \(x \in \cfs[\infty](\R, \C)\) there exists \(y \in \cfs[\infty](\R, \C)\) such that \(y\) is a solution to the differential equation.
    \item Let \(\V\) be the solution space for the homogeneous linear equation
          \[
            y^{(n)} + a_{n - 1} y^{(n - 1)} + \cdots + a_1 y^{(1)} + a_0 y = \zv.
          \]
          Prove that if \(z\) is any solution to the associated nonhomogeneous linear differential equation, then the set of all solutions to the nonhomogeneous linear differential equation is
          \[
            \set{z + y : y \in \V}.
          \]
  \end{enumerate}
\end{ex}

\begin{proof}[\pf{ex:2.7.13}(a)]
  By \cref{d.0.7} there exist some \(\seq{c}{1,,n} \in \C\) (not necessarily distinct) such that
  \[
    y^{(n)} + a_{n - 1} y^{(n - 1)} + \cdots + a_1 y^{(1)} + a_0 y = \pa{\prod_{i = 1}^n (\Dop - c_i \IT)}(y).
  \]
  Then we have
  \begin{align*}
             & \forall i \in \set{1, \dots, n}, \Dop - c_i \IT \text{ is onto}                                                     &  & \by{2.7.11} \\
    \implies & \prod_{i = 1}^n (\Dop - c_i \IT) = (\Dop - c_n \IT)(\Dop - c_{n - 1} \IT( \cdots (\Dop - c_1 \IT))) \text{ is onto}                  \\
    \implies & y^{(n)} + a_{n - 1} y^{(n - 1)} + \cdots + a_1 y^{(1)} + a_0 y \text{ is onto}.
  \end{align*}
\end{proof}

\begin{proof}[\pf{ex:2.7.13}(b)]
  We have
  \begin{align*}
             & \begin{dcases}
                 z^{(n)} + a_{n - 1} z^{(n - 1)} + \cdots + a_1 z^{(1)} + a_0 z = x \\
                 y^{(n)} + a_{n - 1} y^{(n - 1)} + \cdots + a_1 y^{(1)} + a_0 y = \zv
               \end{dcases}                                                        \\
    \implies & z^{(n)} + y^{(n)} + a_{n - 1} (z^{(n - 1)} + y^{(n - 1)}) + \cdots + a_1 (z^{(1)} + y^{(1)}) + a_0 (z^{(0)} + y^{(0)}) = x \\
    \implies & (z + y)^{(n)} + a_{n - 1} (z + y)^{(n - 1)} + \cdots + a_1 (z + y)^{(1)} + a_0 (z + y) = x.
  \end{align*}
  Thus \(z + y\) is a solution to the nonhomogeneous linear differential equation with constant coefficients.
\end{proof}

\begin{ex}\label{ex:2.7.14}
  Given any \(n\)th-order homogeneous linear differential equation with constant coefficients, prove that, for any solution \(x\), if there exists a \(t_0 \in \R\) such that \(x(t_0) = x'(t_0) = \cdots = x^{(n - 1)}(t_0) = 0\), then \(x = 0\) (the zero function).
\end{ex}

\begin{proof}[\pf{ex:2.7.14}]
  Let \(p\) be the auxiliary polynomial of the homogeneous linear differential equation with order \(n\).
  We use induction on \(n\).
  For \(n = 1\), we have
  \begin{align*}
             & \exists c \in \C : \forall t \in \R, p(t) = t - c       &  & \by{d.0.7}         \\
    \implies & \exists c \in \C : e^{ct} \in \ns{p(\Dop)}              &  & \by{2.31}          \\
    \implies & \exists c, k \in \C : \forall t \in \R, x(t) = k e^{ct} &  & \by{2.7.13}        \\
    \implies & \exists c, k \in \C : \begin{dcases}
                                       0 = x(t_0) = k e^{c t_0} \\
                                       0 = x'(t_0) = k c e^{c t_0}
                                     \end{dcases}                                \\
    \implies & x(t_0) = 0                                              &  & (e^{c t_0} \neq 0)
  \end{align*}
  and thus the base case holds.
  Suppose inductively that \cref{ex:2.7.14} is true for some \(n \geq 1\).
  We need to show that for \(n + 1\) \cref{ex:2.7.14} is also true.
  Let \(p\) be of order \(n + 1\).
  By \cref{d.0.7} we know that there exist a polynomial \(q\) with order \(n\) and a \(c \in \C\) such that
  \[
    \forall t \in \R, p(t) = q(t) (t - c).
  \]
  Let \(z = q(\Dop)(x)\).
  Since
  \begin{align*}
    (\Dop - c \IT)(z) & = (\Dop - c \IT)(q(\Dop)(x))                                   \\
                      & = 0                          &  & \text{(\(x\) is a solution)}
  \end{align*}
  we know that \(z \in \ns{\Dop - c \IT}\).
  Since
  \begin{align*}
    z(t_0) & = (q(\Dop)(x))(t_0)                                                  \\
           & = 0                        &  & (x(t_0) = \cdots = x^{(n)}(t_0) = 0) \\
           & = ((\Dop - c \IT)(z))(t_0) &  & (z \in \ns{\Dop - c \IT})            \\
           & = z'(t_0) - c z(t_0)                                                 \\
           & = z'(t_0)                  &  & (z(t_0) = 0)
  \end{align*}
  and \(\Dop - c \IT\) has order \(1\), by induction hypothesis we know that \(z = \zv\).
  Thus we have \(z = q(\Dop)(x) = \zv\).
  Since \(q(\Dop)\) has order \(n\), by induction hypothesis we know that \(x = \zv\).
\end{proof}

\begin{ex}\label{ex:2.7.15}
  Let \(\V\) be the solution space of an \(n\)th-order homogeneous linear differential equation with constant coefficients.
  Fix \(t_0 \in \R\), and define a mapping \(\Phi : \V \to \C^n\) by
  \[
    \Phi(x) = \begin{pmatrix}
      x(t_0)  \\
      x'(t_0) \\
      \vdots  \\
      x^{(n - 1)}(t_0)
    \end{pmatrix} \text{ for each } x \in \V.
  \]
  \begin{enumerate}
    \item Prove that \(\Phi\) is linear and its null space is the zero subspace of \(\V\).
          Deduce that \(\Phi\) is an isomorphism.
    \item Prove the following:
          For any \(n\)th-order homogeneous linear differential equation with constant coefficients, any \(t_0 \in \R\), and any complex numbers \(\seq{c}{0,,n-1}\) (not necessarily distinct), there exists exactly one solution, \(x\), to the given differential equation such that \(x(t_0) = c_0\) and \(x^{(k)}(t_0) = c_k\) for \(k \in \set{1, \dots, n - 1}\).
  \end{enumerate}
\end{ex}

\begin{proof}[\pf{ex:2.7.15}(a)]
  Let \(x, y \in \V\) and let \(c \in \C\).
  Since
  \begin{align*}
    \Phi(cx + y) & = \begin{pmatrix}
                       (cx + y)(t_0)  \\
                       (cx + y)'(t_0) \\
                       \vdots         \\
                       (cx + y)^{(n - 1)}(t_0)
                     \end{pmatrix}              \\
                 & = \begin{pmatrix}
                       cx(t_0) + y(t_0)   \\
                       cx'(t_0) + y'(t_0) \\
                       \vdots             \\
                       cx^{(n - 1)}(t_0) + y^{(n - 1)}(t_0)
                     \end{pmatrix} \\
                 & = c \begin{pmatrix}
                         x(t_0)  \\
                         x'(t_0) \\
                         \vdots  \\
                         x^{(n - 1)}(t_0)
                       \end{pmatrix} + \begin{pmatrix}
                                         y(t_0)  \\
                                         y'(t_0) \\
                                         \vdots  \\
                                         y^{(n - 1)}(t_0)
                                       \end{pmatrix}   \\
                 & = c \Phi(x) + \Phi(y),
  \end{align*}
  by \cref{2.1.2}(b) we know that \(\Phi \in \ls(\V, \C^n)\).
  By \cref{2.1.2}(a) and \cref{ex:2.7.14} we know that \(\Phi(x) = \zv_{\C^n}\) iff \(x\) is the zero function, thus by \cref{2.4} we know that \(\Phi\) is one-to-one.
  By \cref{2.32} we know that \(\dim(\V) = n = \dim(\C^n)\), thus by \cref{2.5,2.4.8} we know that \(\Phi\) is an isomorphism.
\end{proof}

\begin{proof}[\pf{ex:2.7.15}(b)]
  Since \(\Phi\) is an isomorphism (by \cref{ex:2.7.15}(a)), we know that there exists an \(x \in \V\) such that
  \[
    \Phi(x) = \begin{pmatrix}
      x(t_0)  \\
      x'(t_0) \\
      \vdots  \\
      x^{(n - 1)}(t_0)
    \end{pmatrix} = \begin{pmatrix}
      c_0    \\
      c_1    \\
      \vdots \\
      c_{n - 1}
    \end{pmatrix}.
  \]
\end{proof}

\begin{ex}[Pendular Motion]\label{ex:2.7.16}
  It is well known that the motion of a pendulum is approximated by the differential equation
  \[
    \theta'' + \dfrac{g}{l} \theta = \zv,
  \]
  where \(\theta(t)\) is the angle in radians that the pendulum makes with a vertical line at time \(t\), interpreted so that \(\theta\) is positive if the pendulum is to the right and negative if the pendulum is to the left of the vertical line as viewed by the reader.
  Here \(l\) is the length of the pendulum and \(g\) is the magnitude of acceleration due to gravity.
  The variable \(t\) and constants \(l\) and \(g\) must be in compatible units
  (e.g., \(t\) in seconds, \(l\) in meters, and \(g\) in meters per second per second).
  \begin{enumerate}
    \item Express an arbitrary solution to this equation as a linear combination of two real-valued solutions.
    \item Find the unique solution to the equation that satisfies the conditions
          \[
            \theta(0) = \theta_0 > 0 \quad \text{and} \quad \theta'(0) = 0.
          \]
          (The significance of these conditions is that at time \(t = 0\) the pendulum is released from a position displaced from the vertical by \(\theta_0\).)
    \item Prove that it takes \(2 \pi \sqrt{l / g}\) units of time for the pendulum to make one circuit back and forth.
          (This time is called the \textbf{period} of the pendulum.)
  \end{enumerate}
\end{ex}

\begin{proof}[\pf{ex:2.7.16}(a)]
  Since
  \[
    t^2 + \dfrac{g}{l} = \pa{t - i \sqrt{\dfrac{g}{l}}} \pa{t + i \sqrt{\dfrac{g}{l}}} = 0 \implies t = \pm i \sqrt{\dfrac{g}{l}},
  \]
  by \cref{2.7.14} we know that
  \begin{align*}
    \forall t \in \R, \theta(t) & = a e^{i \sqrt{\dfrac{g}{l}} t} + b e^{- i \sqrt{\dfrac{g}{l}} t}                        \\
                                & = a \cos\pa{\sqrt{\dfrac{g}{l}} t} + b \sin\pa{\sqrt{\dfrac{g}{l}} t} &  & \by{ex:2.7.8}
  \end{align*}
  for some \(a, b \in \C\).
\end{proof}

\begin{proof}[\pf{ex:2.7.16}(b)]
  Since
  \begin{align*}
    \theta_0 & = \theta(0)                                                                                                                                                   \\
             & = a \cos\pa{\sqrt{\dfrac{g}{l}} 0} + b \sin\pa{\sqrt{\dfrac{g}{l}} 0}                                                                  &  & \by{ex:2.7.16}[a] \\
             & = a                                                                                                                                                           \\
    0        & = \theta'(0)                                                                                                                                                  \\
             & = -a \cdot \sin\pa{\sqrt{\dfrac{g}{l}} 0} \cdot \sqrt{\dfrac{g}{l}} + b \cdot \cos\pa{\sqrt{\dfrac{g}{l}} 0} \cdot \sqrt{\dfrac{g}{l}} &  & \by{ex:2.7.16}[a] \\
             & = b \sqrt{\dfrac{g}{l}},
  \end{align*}
  we know that \(a = \theta_0\) and \(b = 0\).
  Thus we have
  \[
    \forall t \in \R, \theta(t) = \theta_0 \cos\pa{\sqrt{\dfrac{g}{l}} t}
  \]
\end{proof}

\begin{proof}[\pf{ex:2.7.16}(c)]
  Since
  \begin{align*}
    \theta(0) & = a \cos\pa{\sqrt{\dfrac{g}{l}} 0} + b \sin\pa{\sqrt{\dfrac{g}{l}} 0}                                                              &  & \by{ex:2.7.16}[a] \\
              & = a \cos(0) + b \sin(0)                                                                                                                                   \\
              & = a \cos(2 \pi) + b \sin(2 \pi)                                                                                                                           \\
              & = a \cos\pa{\sqrt{\dfrac{g}{l}} \cdot 2 \pi \sqrt{\dfrac{l}{g}}} + b \sin\pa{\sqrt{\dfrac{g}{l}}  \cdot 2 \pi \sqrt{\dfrac{l}{g}}}                        \\
              & = \theta\pa{2 \pi \sqrt{\dfrac{l}{g}}},                                                                                            &  & \by{ex:2.7.16}[a]
  \end{align*}
  we know that the period of \(\theta\) is \(2 \pi \sqrt{\dfrac{l}{g}}\).
\end{proof}

\begin{ex}[Periodic Motion of a Spring without Damping]\label{ex:2.7.17}
  Find the general solution to
  \[
    y'' + \dfrac{k}{m} y = \zv,
  \]
  which describes the periodic motion of a spring, ignoring frictional forces.
\end{ex}

\begin{proof}[\pf{ex:2.7.17}]
  Since
  \[
    t^2 + \dfrac{k}{m} = \pa{t - i \sqrt{\dfrac{k}{m}}} \pa{t + i \sqrt{\dfrac{k}{m}}} = 0 \implies t = \pm i \sqrt{\dfrac{k}{m}},
  \]
  we know that
  \begin{align*}
    \forall t \in \R, y(t) & = a e^{i \sqrt{\dfrac{k}{m}} t} + b e^{-i \sqrt{\dfrac{k}{m}} t}      &  & \by{2.7.14}   \\
                           & = a \cos\pa{\sqrt{\dfrac{k}{m}} t} + b \sin\pa{\sqrt{\dfrac{k}{m}} t} &  & \by{ex:2.7.8}
  \end{align*}
  for some \(a, b \in \C\).
\end{proof}

\begin{ex}[Periodic Motion of a Spring with Damping]\label{ex:2.7.18}
  The ideal periodic motion described by solutions to \cref{ex:2.7.17} is due to the ignoring of frictional forces.
  In reality, however, there is a frictional force acting on the motion that is proportional to the speed of motion, but that acts in the opposite direction.
  The modification of \cref{ex:2.7.17} to account for the frictional force, called the \emph{damping force}, is given by
  \[
    m y'' + r y' + ky = \zv,
  \]
  where \(r > 0\) is the proportionality constant.
  \begin{enumerate}
    \item Find the general solution to this equation.
    \item Find the unique solution in (a) that satisfies the initial conditions \(y(0) = 0\) and \(y'(0) = v_0\), the initial velocity.
    \item For \(y(t)\) as in (b), show that the amplitude of the oscillation decreases to zero;
          that is, prove that \(\Lim_{t \to \infty} y(t) = 0\).
  \end{enumerate}
\end{ex}

\begin{proof}[\pf{ex:2.7.18}(a)]
  Since
  \[
    t^2 + \dfrac{r}{m} t + \dfrac{k}{m} = 0 \implies t = \dfrac{-r \pm \sqrt{r^2 - 4mk}}{2m},
  \]
  by \cref{2.7.14} we know that
  \[
    \forall t \in \R, y(t) = a e^{\pa{\dfrac{-r + \sqrt{r^2 - 4mk}}{2m}} t} + b e^{\pa{\dfrac{-r - \sqrt{r^2 - 4mk}}{2m}} t}
  \]
  for some \(a, b \in \C\).
\end{proof}

\begin{proof}[\pf{ex:2.7.18}(b)]
  Since
  \begin{align*}
    0   & = y(0)                                                                                                        \\
        & = a + b                                                                                &  & \by{ex:2.7.18}[a] \\
    v_0 & = y'(0)                                                                                                       \\
        & = a \pa{\dfrac{-r + \sqrt{r^2 - 4mk}}{2m}} + b \pa{\dfrac{-r - \sqrt{r^2 - 4mk}}{2m}},
  \end{align*}
  we have
  \[
    a = \dfrac{v_0 m}{\sqrt{r^2 - 4mk}} \quad \text{and} \quad b = \dfrac{-v_0 m}{\sqrt{r^2 - 4mk}}
  \]
  and thus
  \[
    \forall y \in \R, y(t) = \dfrac{v_0 m}{\sqrt{r^2 - 4mk}} e^{\pa{\dfrac{-r + \sqrt{r^2 - 4mk}}{2m}} t} - \dfrac{v_0 m}{\sqrt{r^2 - 4mk}} e^{\pa{\dfrac{-r - \sqrt{r^2 - 4mk}}{2m}} t}.
  \]
\end{proof}

\begin{proof}[\pf{ex:2.7.18}(c)]
  We split into two cases:
  \begin{itemize}
    \item If \(r^2 - 4mk \geq 0\), then we have
          \begin{align*}
                     & r^2 > r^2 - 4mk                                                                                                                         &  & (mk > 0)          \\
            \implies & r > \sqrt{r^2 - 4mk} \geq -\sqrt{r^2 - 4mk}                                                                                             &  & (r > 0)           \\
            \implies & 0 > -r + \sqrt{r^2 - 4mk} \geq -r - \sqrt{r^2 - 4mk}                                                                                                           \\
            \implies & 0 > \dfrac{-r + \sqrt{r^2 - 4mk}}{2m} \geq \dfrac{-r - \sqrt{r^2 - 4mk}}{2m}                                                            &  & (m > 0)           \\
            \implies & \Lim_{t \to \infty} e^{\pa{\dfrac{-r + \sqrt{r^2 - 4mk}}{2m}} t} = \Lim_{t \to \infty} e^{\pa{\dfrac{-r - \sqrt{r^2 - 4mk}}{2m}} t} = 0                        \\
            \implies & \Lim_{t \to \infty} y(t) = 0.                                                                                                           &  & \by{ex:2.7.18}[b]
          \end{align*}
    \item If \(r^2 - 4mk < 0\), then we have
          \begin{align*}
             & \abs{e^{\pa{\dfrac{-r + \sqrt{r^2 - 4mk}}{2m}} t}}                                                                                              \\
             & = \abs{e^{\dfrac{-rt}{2m}} \pa{\cos\pa{\dfrac{\sqrt{r^2 - 4mk}}{2m} t} + i \sin\pa{\dfrac{\sqrt{r^2 - 4mk}}{2m} t}}} &  & \by{2.7.8}            \\
             & \leq \abs{e^{\dfrac{-rt}{2m}}}                                                                                       &  & (\cos^2 + \sin^2 = 1)
          \end{align*}
          and
          \begin{align*}
             & \abs{e^{\pa{\dfrac{-r - \sqrt{r^2 - 4mk}}{2m}} t}}                                                                                                \\
             & = \abs{e^{\dfrac{-rt}{2m}} \pa{\cos\pa{\dfrac{-\sqrt{r^2 - 4mk}}{2m} t} + i \sin\pa{\dfrac{-\sqrt{r^2 - 4mk}}{2m} t}}} &  & \by{2.7.8}            \\
             & \leq \abs{e^{\dfrac{-rt}{2m}}}.                                                                                        &  & (\cos^2 + \sin^2 = 1)
          \end{align*}
          Thus
          \begin{align*}
                     & \Lim_{t \to \infty} \abs{e^{\pa{\dfrac{-r - \sqrt{r^2 - 4mk}}{2m}} t}} = \Lim_{t \to \infty} \abs{e^{\pa{\dfrac{-r - \sqrt{r^2 - 4mk}}{2m}} t}} = 0 \\
            \implies & \Lim_{t \to \infty} e^{\pa{\dfrac{-r - \sqrt{r^2 - 4mk}}{2m}} t} = \Lim_{t \to \infty} e^{\pa{\dfrac{-r - \sqrt{r^2 - 4mk}}{2m}} t} = 0             \\
            \implies & \Lim_{t \to \infty} y(t) = 0.
          \end{align*}
  \end{itemize}
  From all cases above we conclude that \(\Lim_{t \to \infty} y(t) = 0\).
\end{proof}

\begin{ex}\label{ex:2.7.19}
  In our study of differential equations, we have regarded solutions as complex-valued functions even though functions that are useful in describing physical motion are real-valued.
  Justify this approach.
\end{ex}

\begin{proof}[\pf{ex:2.7.19}]
  Every solution to a homogeneous linear differential equation with constant coefficients is in the space of \(\cfs[\infty](\R, \C)\), and we know that \(\cfs[\infty](\R, \R) \subseteq \cfs[\infty](\R, \C)\).
  Therefore, if we are only interesting with real-valued solutions, we can just pick them from the solution space.
\end{proof}


\chapter{Elementary Matrix Operations and Systems of Linear Equations}\label{ch:3}

\chapter{Determinants}\label{ch:4}

% All sections are in separated files.  We include them here.
\section{Determinants of Order \textrm{2}}\label{sec:4.1}

\begin{defn}\label{4.1.1}
  If
  \[
    A = \begin{pmatrix}
      a & b \\
      c & d
    \end{pmatrix}
  \]
  is a \(2 \times 2\) matrix with entries from a field \(\F\), then we define the \textbf{determinant} of \(A\), denoted \(\det(A)\) or \(\abs{A}\), to be the scalar \(ad - bc\).
\end{defn}

\begin{note}
  There exist \(A, B \in \ms[2][2][\F]\) such that \(\det(A + B) \neq \det(A) + \det(B)\), the function \(\det : \ms[2][2][\F] \to \F\) is \emph{not} a linear transformation.
\end{note}

\begin{thm}\label{4.1}
  The function \(\det : \ms[2][2][\F] \to \F\) is a linear function of each row of a \(2 \times 2\) matrix when the other row is held fixed.
  That is, if \(u, v, w \in \vs{F}^2\) and \(k \in \F\), then
  \[
    \det\begin{pmatrix}
      u + kv \\
      w
    \end{pmatrix} = \det\begin{pmatrix}
      u \\
      w
    \end{pmatrix} + k \det\begin{pmatrix}
      v \\
      w
    \end{pmatrix}
  \]
  and
  \[
    \det\begin{pmatrix}
      w \\
      u + kv
    \end{pmatrix} = \det\begin{pmatrix}
      w \\
      u
    \end{pmatrix} + k \det\begin{pmatrix}
      w \\
      v
    \end{pmatrix}.
  \]
\end{thm}

\begin{proof}[\pf{4.1}]
  Let \(u = \tuple{a}{1,2}, v = \tuple{b}{1,2}, w = \tuple{c}{1,2} \in \vs{F}^2\) and let \(k \in \F\).
  Then
  \begin{align*}
    \det\begin{pmatrix}
          u \\
          w
        \end{pmatrix} + k \det\begin{pmatrix}
                                v \\
                                w
                              \end{pmatrix} & = \det\begin{pmatrix}
                                                      a_1 & a_2 \\
                                                      c_1 & c_2
                                                    \end{pmatrix} + k \det\begin{pmatrix}
                                                                            b_1 & b_2 \\
                                                                            c_1 & c_2
                                                                          \end{pmatrix}  \\
                                          & = (a_1 c_2 - a_2 c_1) + k (b_1 c_2 - b_2 c_1) \\
                                          & = (a_1 + k b_1) c_2 - (a_2 + k b_2) c_1       \\
                                          & = \det\begin{pmatrix}
                                                    a_1 + k b_1 & a_2 + k b_2 \\
                                                    c_1         & c_2
                                                  \end{pmatrix}               \\
                                          & = \det\begin{pmatrix}
                                                    u + kv \\
                                                    w
                                                  \end{pmatrix}.
  \end{align*}
  A similar calculation shows that
  \[
    \det\begin{pmatrix}
      w \\
      u
    \end{pmatrix} + k \det\begin{pmatrix}
      w \\
      v
    \end{pmatrix} = \det\begin{pmatrix}
      w \\
      u + kv
    \end{pmatrix}.
  \]
\end{proof}

\begin{thm}\label{4.2}
  Let \(A \in \ms[2][2][\F]\).
  Then the determinant of \(A\) is nonzero iff \(A\) is invertible.
  Moreover, if \(A\) is invertible, then
  \[
    A^{-1} = \dfrac{1}{\det(A)} \begin{pmatrix}
      A_{2 2}  & -A_{1 2} \\
      -A_{2 1} & A_{1 1}
    \end{pmatrix}.
  \]
\end{thm}

\begin{proof}[\pf{4.2}]
  If \(\det(A) \neq 0\), then we can define a matrix
  \[
    M = \dfrac{1}{\det(A)} \begin{pmatrix}
      A_{2 2}  & -A_{1 2} \\
      -A_{2 1} & A_{1 1}
    \end{pmatrix}.
  \]
  A straightforward calculation shows that \(AM = MA = I\), and so \(A\) is invertible and \(M = A^{-1}\).

  Conversely, suppose that \(A\) is invertible.
  \cref{3.2.2} shows that the rank of
  \[
    A = \begin{pmatrix}
      A_{1 1} & A_{1 2} \\
      A_{2 1} & A_{2 2}
    \end{pmatrix}
  \]
  must be \(2\).
  Hence \(A_{1 1} \neq 0\) or \(A_{2 1} \neq 0\).
  If \(A_{1 1} \neq 0\), add \(-A_{2 1} / A_{1 1}\) times row \(1\) of \(A\) to row \(2\) to obtain the matrix
  \[
    \begin{pmatrix}
      A_{1 1} & A_{1 2}                                    \\
      0       & A_{2 2} - \dfrac{A_{1 2} A_{2 1}}{A_{1 1}}
    \end{pmatrix}.
  \]
  Because elementary row operations are rank-preserving by \cref{3.2.3}, it follows that
  \[
    A_{2 2} - \dfrac{A_{1 2} A_{2 1}}{A_{1 1}} \neq 0.
  \]
  Therefore \(\det(A) = A_{1 1} A_{2 2} - A_{1 2} A_{2 1} \neq 0\).
  On the other hand, if \(A_{2 1} \neq 0\), we see that \(\det(A) \neq 0\) by adding \(-A_{1 1} / A_{2 1}\) times row \(2\) of \(A\) to row \(1\) and applying a similar argument.
  Thus, in either case, \(\det(A) \neq 0\).
\end{proof}

\begin{defn}\label{4.1.2}
  By the \textbf{angle} between two vectors in \(\R^2\), we mean the angle with measure \(\theta\) (\(0 \leq \theta < \pi\)) that is formed by the vectors having the same magnitude and direction as the given vectors but emanating from the origin.

  If \(\beta = \set{u, v}\) is an ordered basis for \(\R^2\), we define the \textbf{orientation} of \(\beta\) to be the real number
  \[
    \mathbf{O}\begin{pmatrix}
      u \\
      v
    \end{pmatrix} = \dfrac{\det\begin{pmatrix}
        u \\
        v
      \end{pmatrix}}{\abs{\det\begin{pmatrix}
          u \\
          v
        \end{pmatrix}}}.
  \]
  (The denominator of this fraction is nonzero by \cref{4.2}.)
  Clearly
  \[
    \mathbf{O}\begin{pmatrix}
      u \\
      v
    \end{pmatrix} = \pm 1.
  \]
  Notice that
  \[
    \mathbf{O}\begin{pmatrix}
      e_1 \\
      e_2
    \end{pmatrix} = 1 \quad \text{and} \quad \mathbf{O}\begin{pmatrix}
      e_1 \\
      -e_2
    \end{pmatrix} = -1.
  \]

  Recall that a coordinate system \(\set{u, v}\) is called \textbf{right-handed} if \(u\) can be rotated in a counterclockwise direction through an angle \(\theta\) (\(0 < \theta < \pi\)) to coincide with \(v\).
  Otherwise \(\set{u, v}\) is called a \textbf{left-handed} system.
  In general (see \cref{ex:4.1.12}),
  \[
    \mathbf{O}\begin{pmatrix}
      u \\
      v
    \end{pmatrix} = 1
  \]
  iff the ordered basis \(\set{u, v}\) forms a right-handed coordinate system.
  For convenience, we also define
  \[
    \mathbf{O}\begin{pmatrix}
      u \\
      v
    \end{pmatrix} = 1
  \]
  if \(\set{u, v}\) is linearly dependent.
\end{defn}

\begin{defn}\label{4.1.3}
  Any ordered set \(\set{u, v}\) in \(\R^2\) determines a parallelogram in the following manner.
  Regarding \(u\) and \(v\) as arrows emanating from the origin of \(\R^2\), we call the parallelogram having \(u\) and \(v\) as adjacent sides the \textbf{parallelogram determined by \(u\) and \(v\)}.
  Observe that if the set \(\set{u, v}\) is linearly dependent (i.e., if \(u\) and \(v\) are parallel), then the ``parallelogram'' determined by \(u\) and \(v\) is actually a line segment, which we consider to be a degenerate parallelogram having area zero.
\end{defn}

\begin{prop}\label{4.1.4}
  Let \(\set{u, v} \subseteq \R^2\).
  If we define
  \[
    \mathbf{A}\begin{pmatrix}
      u \\
      v
    \end{pmatrix}
  \]
  as the area of the parallelogram determined by \(u\) and \(v\), then
  \[
    \mathbf{A}\begin{pmatrix}
      u \\
      v
    \end{pmatrix} = \mathbf{O}\begin{pmatrix}
      u \\
      v
    \end{pmatrix} \cdot \det\begin{pmatrix}
      u \\
      v
    \end{pmatrix} = \abs{\det\begin{pmatrix}
        u \\
        v
      \end{pmatrix}}.
  \]
\end{prop}

\begin{proof}[\pf{4.1.4}]
  First, since
  \[
    \mathbf{O}\begin{pmatrix}
      u \\
      v
    \end{pmatrix} = \pm 1,
  \]
  we may multiply both sides of the desired equation by
  \[
    \mathbf{O}\begin{pmatrix}
      u \\
      v
    \end{pmatrix}
  \]
  to obtain the equivalent form
  \[
    \mathbf{O}\begin{pmatrix}
      u \\
      v
    \end{pmatrix} \cdot \mathbf{A}\begin{pmatrix}
      u \\
      v
    \end{pmatrix} = \det\begin{pmatrix}
      u \\
      v
    \end{pmatrix}.
  \]
  We establish this equation by verifying that the three conditions of \cref{ex:4.1.11} are satisfied by the function
  \[
    \delta\begin{pmatrix}
      u \\
      v
    \end{pmatrix} = \mathbf{O}\begin{pmatrix}
      u \\
      v
    \end{pmatrix} \cdot \mathbf{A}\begin{pmatrix}
      u \\
      v
    \end{pmatrix}.
  \]
  \begin{enumerate}
    \item We begin by showing that for any real number \(c\)
          \[
            \delta\begin{pmatrix}
              u \\
              cv
            \end{pmatrix} = c \cdot \delta\begin{pmatrix}
              u \\
              v
            \end{pmatrix}.
          \]
          Observe that this equation is valid if \(c = 0\) because
          \[
            \delta\begin{pmatrix}
              u \\
              cv
            \end{pmatrix} = \mathbf{O}\begin{pmatrix}
              u \\
              \zv
            \end{pmatrix} \cdot \mathbf{A}\begin{pmatrix}
              u \\
              \zv
            \end{pmatrix} = 1 \cdot 0 = 0.
          \]
          So assume that \(c \neq 0\).
          Regarding \(cv\) as the base of the parallelogram determined by \(u\) and \(cv\), we see that
          \[
            \mathbf{A}\begin{pmatrix}
              u \\
              cv
            \end{pmatrix} = \text{base } \times \text{ altitude} = \abs{c} (\text{length of } v) (\text{altitude}) = \abs{c} \cdot \mathbf{A}\begin{pmatrix}
              u \\
              v
            \end{pmatrix},
          \]
          since the altitude of the parallelogram determined by \(u\) and \(cv\) is the same as that in the parallelogram determined by \(u\) and \(v\).
          Hence
          \begin{align*}
            \delta\begin{pmatrix}
                    u \\
                    cv
                  \end{pmatrix} & = \mathbf{O}\begin{pmatrix}
                                                u \\
                                                cv
                                              \end{pmatrix} \cdot \mathbf{A}\begin{pmatrix}
                                                                              u \\
                                                                              cv
                                                                            \end{pmatrix}                                               \\
                                  & = \pa{\dfrac{c}{\abs{c}} \mathbf{O}\begin{pmatrix}
                                                                           u \\
                                                                           v
                                                                         \end{pmatrix}} \pa{\abs{c} \mathbf{A}\begin{pmatrix}
                                                                                                                u \\
                                                                                                                v
                                                                                                              \end{pmatrix}} &  & \by{4.1} \\
                                  & = c \cdot \mathbf{O}\begin{pmatrix}
                                                          u \\
                                                          v
                                                        \end{pmatrix} \cdot \mathbf{A}\begin{pmatrix}
                                                                                        u \\
                                                                                        v
                                                                                      \end{pmatrix}                                     \\
                                  & = c \cdot \delta\begin{pmatrix}
                                                      u \\
                                                      v
                                                    \end{pmatrix}.
          \end{align*}
          A similar argument shows that
          \[
            \delta\begin{pmatrix}
              cu \\
              v
            \end{pmatrix} = c \cdot \delta\begin{pmatrix}
              u \\
              v
            \end{pmatrix}.
          \]

          We next prove that
          \[
            \delta\begin{pmatrix}
              u \\
              au + bw
            \end{pmatrix} = b \cdot \delta\begin{pmatrix}
              u \\
              w
            \end{pmatrix}
          \]
          for any \(u, w \in \R^2\) and any real numbers \(a\) and \(b\).
          Because the parallelograms determined by \(u\) and \(w\) and by \(u\) and \(u + w\) have a common base \(u\) and the same altitude, it follows that
          \[
            \mathbf{A}\begin{pmatrix}
              u \\
              w
            \end{pmatrix} = \mathbf{A}\begin{pmatrix}
              u \\
              u + w
            \end{pmatrix}.
          \]
          If \(a = 0\), then
          \[
            \delta\begin{pmatrix}
              u \\
              au + bw
            \end{pmatrix} = \delta\begin{pmatrix}
              u \\
              bw
            \end{pmatrix} = b \cdot \delta\begin{pmatrix}
              u \\
              w
            \end{pmatrix}
          \]
          by the first paragraph of (a).
          Otherwise, if \(a \neq 0\), then
          \[
            \delta\begin{pmatrix}
              u \\
              au + bw
            \end{pmatrix} = a \cdot \delta\begin{pmatrix}
              u \\
              u + \dfrac{b}{a} w
            \end{pmatrix} = a \cdot \delta\begin{pmatrix}
              u \\
              \dfrac{b}{a} w
            \end{pmatrix} = b \cdot \delta\begin{pmatrix}
              u \\
              w
            \end{pmatrix}.
          \]
          So the desired conclusion is obtained in either case.

          We are now able to show that
          \[
            \delta\begin{pmatrix}
              u \\
              v_1 + v_2
            \end{pmatrix} = \delta\begin{pmatrix}
              u \\
              v_1
            \end{pmatrix} + \delta\begin{pmatrix}
              u \\
              v_2
            \end{pmatrix}
          \]
          for all \(u, v_1, v_2 \in \R^2\).
          Since the result is immediate if \(u = 0\), we assume that \(u \neq 0\).
          Choose any vector \(w \in \R^2\) such that \(\set{u, w}\) is linearly independent.
          Then for any vectors \(v_1, v_2 \in \R^2\) there exist scalars \(a_i\) and \(b_i\) such that \(v_i = a_i u + b_i w\) (\(i \in \set{1, 2}\)).
          Thus
          \begin{align*}
            \delta\begin{pmatrix}
                    u \\
                    v_1 + v_2
                  \end{pmatrix} & = \delta\begin{pmatrix}
                                            u \\
                                            (a_1 + a_2) u + (b_1 + b_2) w
                                          \end{pmatrix}           \\
                                  & = (b_1 + b_2) \delta\begin{pmatrix}
                                                          u \\
                                                          w
                                                        \end{pmatrix}           \\
                                  & = \delta\begin{pmatrix}
                                              u \\
                                              a_1 u + b_1 w
                                            \end{pmatrix} + \delta\begin{pmatrix}
                                                                    u \\
                                                                    a_2 u + b_2 w
                                                                  \end{pmatrix} \\
                                  & = \delta\begin{pmatrix}
                                              u \\
                                              v_1
                                            \end{pmatrix} + \delta\begin{pmatrix}
                                                                    u \\
                                                                    v_2
                                                                  \end{pmatrix}.
          \end{align*}
          A similar argument shows that
          \[
            \delta\begin{pmatrix}
              u_1 + u_2 \\
              v
            \end{pmatrix} = \delta\begin{pmatrix}
              u_1 \\
              v
            \end{pmatrix} + \delta\begin{pmatrix}
              u_2 \\
              v
            \end{pmatrix}
          \]
          for all \(u_1, u_2, v \in \R^2\).
    \item Since
          \[
            \mathbf{A}\begin{pmatrix}
              u \\
              u
            \end{pmatrix} = 0,
          \]
          it follows that
          \[
            \delta\begin{pmatrix}
              u \\
              u
            \end{pmatrix} = \mathbf{O}\begin{pmatrix}
              u \\
              u
            \end{pmatrix} \cdot \mathbf{A}\begin{pmatrix}
              u \\
              u
            \end{pmatrix} = 0
          \]
          for any \(u \in \R^2\).
    \item Because the parallelogram determined by \(e_1\) and \(e_2\) is the unit square,
          \[
            \delta\begin{pmatrix}
              e_1 \\
              e_2
            \end{pmatrix} = \mathbf{O}\begin{pmatrix}
              e_1 \\
              e_2
            \end{pmatrix} \cdot \mathbf{A}\begin{pmatrix}
              e_1 \\
              e_2
            \end{pmatrix} = 1 \cdot 1 = 1.
          \]
          Therefore \(\delta\) satisfies the three conditions of \cref{ex:4.1.11}, and hence \(\delta = \det\).
          So the area of the parallelogram determined by \(u\) and \(v\) equals
          \[
            \mathbf{O}\begin{pmatrix}
              u \\
              v
            \end{pmatrix} \cdot \det\begin{pmatrix}
              u \\
              v
            \end{pmatrix}.
          \]
  \end{enumerate}
\end{proof}

\exercisesection

\setcounter{ex}{4}
\begin{ex}\label{ex:4.1.5}
  Prove that if \(B\) is the matrix obtained by interchanging the rows of \(A \in \ms[2][2][\F]\), then \(\det(B) = -\det(A)\).
\end{ex}

\begin{proof}[\pf{ex:4.1.5}]
  We have
  \begin{align*}
    \det(B) & = \det\begin{pmatrix}
                      A_{2 1} & A_{2 2} \\
                      A_{1 1} & A_{1 2}
                    \end{pmatrix}                                \\
            & = A_{2 1} A_{1 2} - A_{2 2} A_{1 1}    &  & \by{4.1.1} \\
            & = -(A_{1 1} A_{2 2} - A_{1 2} A_{2 1})                 \\
            & = -\det(A).                            &  & \by{4.1.1}
  \end{align*}
\end{proof}

\begin{ex}\label{ex:4.1.6}
  Prove that if the two columns of \(A \in \ms[2][2][\F]\) are identical, then \(\det(A) = 0\).
\end{ex}

\begin{proof}[\pf{ex:4.1.6}]
  We have
  \begin{align*}
             & (A_{1 1}, A_{2 1}) = (A_{1 2}, A_{2 2})                                                              \\
    \implies & \det(A) = A_{1 1} A_{2 2} - A_{1 2} A_{2 1} = A_{1 1} A_{2 2} - A_{1 1} A_{2 2} = 0. &  & \by{4.1.1}
  \end{align*}
\end{proof}

\begin{ex}\label{ex:4.1.7}
  Prove that \(\det(\tp{A}) = \det(A)\) for any \(A \in \ms[2][2][\F]\).
\end{ex}

\begin{proof}[\pf{ex:4.1.7}]
  We have
  \begin{align*}
    \det(\tp{A}) & = \det\begin{pmatrix}
                           A_{1 1} & A_{2 1} \\
                           A_{1 2} & A_{2 2}
                         \end{pmatrix}               &  & \by{1.3.3}   \\
                 & = A_{1 1} A_{2 2} - A_{2 1} A_{1 2} &  & \by{4.1.1} \\
                 & = A_{1 1} A_{2 2} - A_{1 2} A_{2 1}                 \\
                 & = \det(A).                          &  & \by{4.1.1}
  \end{align*}
\end{proof}

\begin{ex}\label{ex:4.1.8}
  Prove that if \(A \in \ms[2][2][\F]\) is upper triangular, then \(\det(A)\) equals the product of the diagonal entries of \(A\).
\end{ex}

\begin{proof}[\pf{ex:4.1.8}]
  We have
  \begin{align*}
             & A_{2 1} = 0                                                    &  & \by{ex:1.3.12} \\
    \implies & \det(A) = A_{1 1} A_{2 2} - A_{1 2} A_{2 1} = A_{1 1} A_{2 2}. &  & \by{4.1.1}
  \end{align*}
\end{proof}

\begin{ex}\label{ex:4.1.9}
  Prove that \(\det(AB) = \det(A) \cdot \det(B)\) for any \(A, B \in \ms[2][2][\F]\).
\end{ex}

\begin{proof}[\pf{ex:4.1.9}]
  We have
  \begin{align*}
     & \det(AB)                                                                                                                        \\
     & = \det\begin{pmatrix}
               A_{1 1} B_{1 1} + A_{1 2} B_{2 1} & A_{1 1} B_{1 2} + A_{1 2} B_{2 2} \\
               A_{2 1} B_{1 1} + A_{2 2} B_{2 1} & A_{2 1} B_{1 2} + A_{2 2} B_{2 2}
             \end{pmatrix}                                        &  & \by{2.3.1}                                                     \\
     & = (A_{1 1} B_{1 1} + A_{1 2} B_{2 1}) (A_{2 1} B_{1 2} + A_{2 2} B_{2 2})                                                       \\
     & \quad - (A_{1 1} B_{1 2} + A_{1 2} B_{2 2}) (A_{2 1} B_{1 1} + A_{2 2} B_{2 1})                                 &  & \by{4.1.1} \\
     & = (A_{1 1} A_{2 2}) (B_{1 1} B_{2 2} - B_{1 2} B_{2 1}) - (A_{1 2} A_{2 1}) (B_{1 1} B_{2 2} - B_{1 2} B_{2 1})                 \\
     & = (A_{1 1} A_{2 2} - A_{1 2} A_{2 1}) (B_{1 1} B_{2 2} - B_{1 2} B_{2 1})                                                       \\
     & = \det(A) \det(B).                                                                                              &  & \by{4.1.1}
  \end{align*}
\end{proof}

\begin{ex}\label{ex:4.1.10}
  The \textbf{classical adjoint} of a matrix \(A \in \ms[2][2][\F]\) is the matrix
  \[
    C = \begin{pmatrix}
      A_{2 2}  & -A_{1 2} \\
      -A_{2 1} & A_{1 1}
    \end{pmatrix}.
  \]
  Prove that
  \begin{enumerate}
    \item \(CA = AC = \det(A) I\).
    \item \(\det(C) = \det(A)\).
    \item The classical adjoint of \(\tp{A}\) is \(\tp{C}\).
    \item If \(A\) is invertible, then \(A^{-1} = (\det(A))^{-1} C\).
  \end{enumerate}
\end{ex}

\begin{proof}[\pf{ex:4.1.10}(a)]
  We have
  \begin{align*}
    CA & = \begin{pmatrix}
             C_{1 1} A_{1 1} + C_{1 2} A_{2 1} & C_{1 1} A_{1 2} + C_{1 2} A_{2 2} \\
             C_{2 1} A_{1 1} + C_{2 2} A_{2 1} & C_{2 1} A_{1 2} + C_{2 2} A_{2 2}
           \end{pmatrix}   &  & \by{2.3.1}                    \\
       & = \begin{pmatrix}
             A_{2 2} A_{1 1} - A_{1 2} A_{2 1}  & A_{2 2} A_{1 2} - A_{1 2} A_{2 2}  \\
             -A_{2 1} A_{1 1} + A_{1 1} A_{2 1} & -A_{2 1} A_{1 2} + A_{1 1} A_{2 2}
           \end{pmatrix} &  & \by{ex:4.1.10}                  \\
       & = \begin{pmatrix}
             \det(A) & 0       \\
             0       & \det(A)
           \end{pmatrix}                                                       &  & \by{4.1.1}      \\
       & = \det(A) I                                                                &  & \by{1.2.9}
  \end{align*}
  and
  \begin{align*}
    AC & = \begin{pmatrix}
             A_{1 1} C_{1 1} + A_{1 2} C_{2 1} & A_{1 1} C_{1 2} + A_{1 2} C_{2 2} \\
             A_{2 1} C_{1 1} + A_{2 2} C_{2 1} & A_{2 1} C_{1 2} + A_{2 2} C_{2 2}
           \end{pmatrix}  &  & \by{2.3.1}                   \\
       & = \begin{pmatrix}
             A_{1 1} A_{2 2} - A_{1 2} A_{2 1} & -A_{1 1} A_{1 2} + A_{1 2} A_{1 1} \\
             A_{2 1} A_{2 2} - A_{2 2} A_{2 1} & -A_{2 1} A_{1 2} + A_{2 2} A_{1 1}
           \end{pmatrix} &  & \by{ex:4.1.10}                  \\
       & = \begin{pmatrix}
             \det(A) & 0       \\
             0       & \det(A)
           \end{pmatrix}                                                      &  & \by{4.1.1}      \\
       & = \det(A) I.                                                              &  & \by{1.2.9}
  \end{align*}
\end{proof}

\begin{proof}[\pf{ex:4.1.10}(b)]
  We have
  \begin{align*}
    \det(C) & = A_{2 2} A_{1 1} - (-A_{1 2}) (-A_{2 1}) &  & \by{4.1.1} \\
            & = A_{1 1} A_{2 2} - A_{1 2} A_{2 1}                       \\
            & = \det(A).                                &  & \by{4.1.1}
  \end{align*}
\end{proof}

\begin{proof}[\pf{ex:4.1.10}(c)]
  We have
  \begin{align*}
     & \text{the classical adjoint of } \tp{A}                     \\
     & = \begin{pmatrix}
           (\tp{A})_{2 2}  & -(\tp{A})_{1 2} \\
           -(\tp{A})_{2 1} & (\tp{A})_{1 1}
         \end{pmatrix}    &  & \by{ex:4.1.10}                      \\
     & = \begin{pmatrix}
           A_{2 2}  & -A_{2 1} \\
           -A_{1 2} & A_{1 1}
         \end{pmatrix}                  &  & \by{1.3.3}            \\
     & = \tp{C}.                               &  & \by{ex:4.1.10}
  \end{align*}
\end{proof}

\begin{proof}[\pf{ex:4.1.10}(d)]
  We have
  \begin{align*}
    \dfrac{1}{\det(A)} C & = \dfrac{1}{\det(A)} \begin{pmatrix}
                                                  A_{2 2}  & -A_{1 2} \\
                                                  -A_{2 1} & A_{1 1}
                                                \end{pmatrix} &  & \by{ex:4.1.10} \\
                         & = A^{-1}.                            &  & \by{4.2}
  \end{align*}
\end{proof}

\begin{ex}\label{ex:4.1.11}
  Let \(\delta : \ms[2][2][\F] \to \F\) be a function with the following three properties.
  \begin{enumerate}
    \item \(\delta\) is a linear function of each row of the matrix when the other row is held fixed.
    \item If the two rows of \(A \in \ms[2][2][\F]\) are identical, then \(\delta(A) = 0\).
    \item \(\delta(I_2) = 1\).
  \end{enumerate}
  Prove that \(\delta(A) = \det(A)\) for all \(A \in \ms[2][2][\F]\).
\end{ex}

\begin{proof}[\pf{ex:4.1.11}]
  Let \(A \in \ms[n][n][\F]\), let \(e_1 = (1, 0)\) and let \(e_2 = (0, 1)\).
  Since
  \begin{align*}
    0 & = \delta\begin{pmatrix}
                  e_1 + e_2 \\
                  e_1 + e_2
                \end{pmatrix}               &  & \by{ex:4.1.11}[b]                                                     \\
      & = \delta\begin{pmatrix}
                  e_1 \\
                  e_1 + e_2
                \end{pmatrix} + \delta\begin{pmatrix}
                                        e_2 \\
                                        e_1 + e_2
                                      \end{pmatrix} &  & \by{ex:4.1.11}[a]                                             \\
      & = \delta\begin{pmatrix}
                  e_1 \\
                  e_1
                \end{pmatrix} + \delta\begin{pmatrix}
                                        e_1 \\
                                        e_2
                                      \end{pmatrix} + \delta\begin{pmatrix}
                                                              e_2 \\
                                                              e_1
                                                            \end{pmatrix} + \delta\begin{pmatrix}
                                                                                    e_2 \\
                                                                                    e_2
                                                                                  \end{pmatrix} &  & \by{ex:4.1.11}[a] \\
      & = 0 + 1 + \delta\begin{pmatrix}
                          e_2 \\
                          e_1
                        \end{pmatrix} + 0,       &  & \by{ex:4.1.11}[b,c]
  \end{align*}
  we know that
  \[
    \delta\begin{pmatrix}
      e_2 \\
      e_1
    \end{pmatrix} = -1
  \]
  and thus
  \begin{align*}
    \delta(A) & = \delta\begin{pmatrix}
                          A_{1 1} & A_{1 2} \\
                          A_{2 1} & A_{2 2}
                        \end{pmatrix}                                                                                     \\
              & = \delta\begin{pmatrix}
                          A_{1 1} e_1 + A_{1 2} e_2 \\
                          A_{2 1} e_1 + A_{2 2} e_2
                        \end{pmatrix}                                                                             \\
              & = A_{1 1} \delta\begin{pmatrix}
                                  e_1 \\
                                  A_{2 1} e_1 + A_{2 2} e_2
                                \end{pmatrix} + A_{1 2} \delta\begin{pmatrix}
                                                                e_2 \\
                                                                A_{2 1} e_1 + A_{2 2} e_2
                                                              \end{pmatrix}             &  & \by{ex:4.1.11}[a]                \\
              & = A_{1 1} \pa{A_{2 1} \delta\begin{pmatrix}
                                                e_1 \\
                                                e_1
                                              \end{pmatrix} + A_{2 2} \delta\begin{pmatrix}
                                                                              e_1 \\
                                                                              e_2
                                                                            \end{pmatrix}}             &  & \by{ex:4.1.11}[a]   \\
              & \quad + A_{1 2} \pa{A_{2 1} \delta\begin{pmatrix}
                                                      e_2 \\
                                                      e_1
                                                    \end{pmatrix} + A_{2 2} \delta\begin{pmatrix}
                                                                                    e_2 \\
                                                                                    e_2
                                                                                  \end{pmatrix}}         &  & \by{ex:4.1.11}[a] \\
              & = A_{1 1} A_{2 2} \delta\begin{pmatrix}
                                          e_1 \\
                                          e_2
                                        \end{pmatrix} + A_{1 2} A_{2 1} \delta\begin{pmatrix}
                                                                                e_2 \\
                                                                                e_1
                                                                              \end{pmatrix}     &  & \by{ex:4.1.11}[b]        \\
              & = A_{1 1} A_{2 2} + A_{1 2} A_{2 1} \delta\begin{pmatrix}
                                                            e_2 \\
                                                            e_1
                                                          \end{pmatrix} &  & \by{ex:4.1.11}[c]                                \\
              & = A_{1 1} A_{2 2} - A_{1 2} A_{2 1}                       &  & \text{(from the proof above)}                  \\
              & = \det(A).                                                &  & \by{4.1.1}
  \end{align*}
  Since \(A\) is arbitrary, we conclude that \(\delta = \det\).
\end{proof}

\begin{ex}\label{ex:4.1.12}
  Let \(\set{u, v}\) be an ordered basis for \(\R^2\).
  Prove that
  \[
    \mathbf{O}\begin{pmatrix}
      u \\
      v
    \end{pmatrix} = 1
  \]
  iff \(\set{u, v}\) forms a right-handed coordinate system.
\end{ex}

\begin{proof}[\pf{ex:4.1.12}]
  First suppose that \(\set{u, v}\) forms a right-handed coordinate system.
  By \cref{4.1.2} there exists a \(\theta \in (0, \pi)\) such that by rotating \(u\) with angle \(\theta\) and scaling \(u\) with some \(t > 0\) we get \(v\).
  Note that \(\theta \neq 0\) since \(\set{u, v}\) is a basis for \(\R^2\) over \(\R\).
  By \cref{2.1.3} this means
  \[
    v = t (u_1 \cos(\theta) - u_2 \sin(\theta), u_1 \sin(\theta) + u_2 \cos(\theta)).
  \]
  Thus we have
  \begin{align*}
    \det\begin{pmatrix}
          u \\
          v
        \end{pmatrix} & = \det\begin{pmatrix}
                                u_1                                     & u_2                                     \\
                                t u_1 \cos(\theta) - t u_2 \sin(\theta) & t u_1 \sin(\theta) + t u_2 \cos(\theta)
                              \end{pmatrix}                           \\
                        & = t \det\begin{pmatrix}
                                    u_1                                 & u_2                                 \\
                                    u_1 \cos(\theta) - u_2 \sin(\theta) & u_1 \sin(\theta) + u_2 \cos(\theta)
                                  \end{pmatrix}         &  & \by{4.1}                               \\
                        & = t (u_1^2 \sin(\theta) + u_2^2 \sin(\theta))                                        &  & \by{4.1.1}            \\
                        & > 0.                                                                                 &  & (\theta \in (0, \pi))
  \end{align*}
  By \cref{4.1.2} we have
  \[
    \mathbf{O}\begin{pmatrix}
      u \\
      v
    \end{pmatrix} = \dfrac{\det\begin{pmatrix}
        u \\
        v
      \end{pmatrix}}{\abs{\det\begin{pmatrix}
          u \\
          v
        \end{pmatrix}}} = \dfrac{\det\begin{pmatrix}
        u \\
        v
      \end{pmatrix}}{\det\begin{pmatrix}
        u \\
        v
      \end{pmatrix}} = 1.
  \]

  Now suppose that
  \[
    \mathbf{O}\begin{pmatrix}
      u \\
      v
    \end{pmatrix} = 1.
  \]
  Since \(\set{u, v}\) is a basis for \(\R^2\), by rotating \(u\) with angle \(\theta > 0\) and scaling \(u\) with \(t > 0\) we get \(v\).
  By \cref{2.1.3} this means
  \[
    v = t (u_1 \cos(\theta) - u_2 \sin(\theta), u_1 \sin(\theta) + u_2 \cos(\theta)).
  \]
  Then we have
  \begin{align*}
             & \mathbf{O}\begin{pmatrix}
                           u \\
                           v
                         \end{pmatrix} = 1                                                        \\
    \implies & \det\begin{pmatrix}
                     u \\
                     v
                   \end{pmatrix} > 0                             &  & \by{4.1.2}                  \\
    \implies & u_1 v_2 - u_2 v_1 > 0                           &  & \by{4.1.1}                    \\
    \implies & t u_1^2 \sin(\theta) + t u_2^2 \sin(\theta) > 0 &  & \text{(from the proof above)} \\
    \implies & \sin(\theta) > 0                                &  & (t > 0)                       \\
    \implies & \theta \in (0, \pi)
  \end{align*}
  and thus by \cref{4.1.2} \(\set{u, v}\) forms a right-handed coordinate system.
\end{proof}


\chapter{Diagonalization}\label{ch:5}

% All sections are in separated files.  We include them here.
\section{Eigenvalues and Eigenvectors}\label{sec:5.1}

\begin{defn}\label{5.1.1}
  A linear operator \(\T\) on a finite-dimensional vector space \(\V\) over \(\F\) is called \textbf{diagonalizable} if there is an ordered basis \(\beta\) for \(\V\) over \(\F\) such that \([\T]_{\beta}\) is a diagonal matrix.
  A square matrix \(A\) is called \textbf{diagonalizable} if \(\L_A\) is diagonalizable.
\end{defn}

\begin{note}
  We want to determine when a linear operator \(\T\) on a finite-dimensional vector space \(\V\) over \(\F\) is diagonalizable and, if so, how to obtain an ordered basis \(\beta = \set{\seq{v}{1,,n}}\) for \(\V\) over \(\F\) such that \([\T]_{\beta}\) is a diagonal matrix.
  Note that, if \(D = [\T]_{\beta}\) is a diagonal matrix, then for each vector \(v_j \in \beta\), we have
  \[
    \T(v_j) = \sum_{i = 1}^n D_{i j} v_i = D_{j j} v_j = \lambda_j v_j,
  \]
  where \(\lambda_j = D_{j j}\).

  Conversely, if \(\beta = \set{\seq{v}{1,,n}}\) is an ordered basis for \(\V\) over \(\F\) such that \(\T(v_j) = \lambda_j v_j\) for some scalars \(\seq{\lambda}{1,,n}\), then clearly
  \[
    [\T]_{\beta} = \begin{pmatrix}
      \lambda_1 & 0         & \cdots & 0         \\
      0         & \lambda_2 & \cdots & 0         \\
      \vdots    & \vdots    &        & \vdots    \\
      0         & 0         & \cdots & \lambda_n
    \end{pmatrix}.
  \]
  In the preceding paragraph, each vector \(v\) in the basis \(\beta\) satisfies the condition that \(\T(v) = \lambda v\) for some scalar \(\lambda\).
  Moreover, because \(v\) lies in a basis, \(v\) is nonzero.
  These computations motivate \cref{5.1.2}.
\end{note}

\begin{defn}\label{5.1.2}
  Let \(\T\) be a linear operator on a vector space \(\V\) over \(\F\).
  A nonzero vector \(v \in \V\) is called an \textbf{eigenvector} of \(\T\) if there exists a scalar \(\lambda\) such that \(\T(v) = \lambda v\).
  The scalar \(\lambda\) is called the \textbf{eigenvalue} corresponding to the eigenvector \(v\).

  Let \(A \in \ms{n}{n}{\F}\).
  A nonzero vector \(v \in \vs{F}^n\) is called an \textbf{eigenvector} of \(A\) if \(v\) is an eigenvector of \(\L_A\);
  that is, if \(Av = \lambda v\) for some scalar \(\lambda\).
  The scalar \(\lambda\) is called the \textbf{eigenvalue} of \(A\) corresponding to the eigenvector \(v\).
\end{defn}

\begin{note}
  The words \emph{characteristic vector} and \emph{proper vector} are also used in place of eigenvector.
  The corresponding terms for eigenvalue are \emph{characteristic value} and \emph{proper value}.
\end{note}

\begin{thm}\label{5.1}
  A linear operator \(\T\) on a finite-dimensional vector space \(\V\) over \(\F\) is diagonalizable iff there exists an ordered basis \(\beta\) for \(\V\) over \(\F\) consisting of eigenvectors of \(\T\).
  Furthermore, if \(\T\) is diagonalizable, \(\beta = \set{\seq{v}{1,,n}}\) is an ordered basis of eigenvectors of \(\T\), and \(D = [\T]_{\beta}\), then \(D\) is a diagonal matrix and \(D_{j j}\) is the eigenvalue corresponding to \(v_j\) for \(j \in \set{1, \dots, n}\).
\end{thm}

\begin{proof}[\pf{5.1}]
  This is simply a restatement of \cref{5.1.1,5.1.2}.
\end{proof}

\begin{note}
  To \emph{diagonalize} a matrix or a linear operator is to find a basis of eigenvectors and the corresponding eigenvalues.
\end{note}

\begin{eg}\label{5.1.3}
  Let \(\T\) be the linear operator on \(\R^2\) that rotates each vector in the plane through an angle of \(\pi / 2\).
  It is clear geometrically that for any nonzero vector \(v\), the vectors \(v\) and \(\T(v)\) are not collinear;
  hence \(\T(v)\) is not a multiple of \(v\).
  Therefore \(\T\) has no eigenvectors and, consequently, no eigenvalues.
  Thus there exist operators (and matrices) with no eigenvalues or eigenvectors.
  Of course, such operators and matrices are not diagonalizable.
\end{eg}

\begin{eg}\label{5.1.4}
  Let \(\cfs[\infty](\R)\) denote the set of all functions \(f : \R \to \R\) having derivatives of all orders.
  (Thus \(\cfs[\infty](\R)\) includes the polynomial functions, the sine and cosine functions, the exponential functions, etc.)
  Clearly, \(\cfs[\infty](\R)\) is a subspace of the vector space \(\fs(\R, \R)\) of all functions from \(\R\) to \(\R\) as defined in \cref{1.2.10}.
  Let \(\T : \cfs[\infty](\R) \to \cfs[\infty](\R)\) be the function defined by \(\T(f) = f'\), the derivative of \(f\).
  It is easily verified that \(\T\) is a linear operator on \(\cfs[\infty](\R)\) (\cref{ex:2.7.6}).
  We determine the eigenvalues and eigenvectors of \(\T\).

  Suppose that \(f\) is an eigenvector of \(\T\) with corresponding eigenvalue \(\lambda\).
  Then \(f' = T(f) = \lambda f\).
  This is a first-order differential equation whose solutions are of the form \(f(t) = ce^{\lambda t}\) for some constant \(c\) (\cref{2.30}).
  Consequently, every real number \(\lambda\) is an eigenvalue of \(\T\), and \(\lambda\) corresponds to eigenvectors of the form \(ce^{\lambda t}\) for \(c \neq 0\).
  Note that for \(\lambda = 0\), the eigenvectors are the nonzero constant functions.
\end{eg}

\begin{thm}\label{5.2}
  Let \(A \in \ms{n}{n}{\F}\).
  Then a scalar \(\lambda\) is an eigenvalue of \(A\) iff \(\det(A - \lambda I_n) = 0\).
\end{thm}

\begin{proof}[\pf{5.2}]
  A scalar \(\lambda \in \F\) is an eigenvalue of \(A\) iff there exists a nonzero vector \(v \in \vs{F}^n\) such that \(Av = \lambda v\), that is, \((A - \lambda I_n)(v) = 0\).
  By \cref{2.5}, this is true iff \(A - \lambda I_n\) is not invertible.
  However, by \cref{4.3.1} this result is equivalent to the statement that \(\det(A - \lambda I_n) = 0\).
\end{proof}

\begin{defn}\label{5.1.5}
  Let \(A \in \ms{n}{n}{\F}\).
  The polynomial \(f(t) = \det(A - t I_n)\) is called the \textbf{characteristic polynomial} of \(A\).
\end{defn}

\begin{note}
  The entries of the matrix \(A - t I_n\) are not scalars in the field \(\F\).
  They are, however, scalars in another field \(F(t)\), the field of quotients of polynomials in \(t\) with coefficients from \(\F\).
  Consequently, any results proved about determinants in \cref{ch:4} remain valid in this context.
\end{note}

\begin{note}
  \cref{5.2} states that the eigenvalues of a matrix are the zeros of its characteristic polynomial.
  When determining the eigenvalues of a matrix or a linear operator, we normally compute its characteristic polynomial.
\end{note}

\begin{note}
  It is easily shown that similar matrices have the same characteristic polynomial (see \cref{ex:5.1.12}).
  This fact enables us to define the characteristic polynomial of a linear operator as in \cref{5.1.6}
\end{note}

\begin{defn}\label{5.1.6}
  Let \(\T\) be a linear operator on an \(n\)-dimensional vector space \(\V\) over \(\F\) with ordered basis \(\beta\).
  We define the \textbf{characteristic polynomial} \(f(t)\) of \(\T\) to be the characteristic polynomial of \(A = [\T]_{\beta}\).
  That is,
  \[
    f(t) = \det(A - t I_n).
  \]
\end{defn}

\begin{note}
  \cref{ex:5.1.12} shows that \cref{5.1.6} is independent of the choice of ordered basis \(\beta\).
  Thus if \(\T\) is a linear operator on a finite-dimensional vector space \(\V\) over \(\F\) and \(\beta\) is an ordered basis for \(\V\) over \(\F\), then \(\lambda\) is an eigenvalue of \(\T\) iff \(\lambda\) is an eigenvalue of \([\T]_{\beta}\).
  We often denote the characteristic polynomial of an operator \(\T\) by \(\det(\T - tI)\).
\end{note}

\begin{thm}\label{5.3}
  Let \(A \in \ms{n}{n}{\F}\).
  \begin{enumerate}
    \item The characteristic polynomial of \(A\) is a polynomial of degree \(n\) with leading coefficient \((-1)^n\).
    \item \(A\) has at most \(n\) distinct eigenvalues.
  \end{enumerate}
\end{thm}

\begin{proof}[\pf{5.3}]

\end{proof}

\begin{thm}\label{5.4}
  Let \(\T\) be a linear operator on a vector space \(\V\) over \(\F\), and let \(\lambda\) be an eigenvalue of \(\T\).
  A vector \(v \in \V\) is an eigenvector of \(\T\) corresponding to \(\lambda\) iff \(v \neq 0\) and \(v \in \ns{\T - \lambda \IT[\V]}\).
\end{thm}

\begin{proof}[\pf{5.4}]
  We have
  \begin{align*}
         & v \text{ is an eigenvector of } \T \text{ corresponds to eigenvalue } \lambda                             \\
    \iff & \begin{dcases}
             v \neq 0 \\
             \T(v) = \lambda v
           \end{dcases}                                                                &  & \text{(by \cref{5.1.2})} \\
    \iff & \begin{dcases}
             v \neq 0 \\
             (\T - \lambda \IT[\V])(v) = \T(v) - \lambda \IT[\V](v) = \T(v) - \lambda v = \zv
           \end{dcases} &  & \text{(by \cref{2.1.9})}                           \\
    \iff & \begin{dcases}
             v \neq 0 \\
             v \in \ns{\T - \lambda \IT[\V]}
           \end{dcases}.                                   &  & \text{(by \cref{2.1.10})}
  \end{align*}
\end{proof}

\exercisesection

\setcounter{ex}{5}
\begin{ex}\label{ex:5.1.6}
  Let \(\T\) be a linear operator on a finite-dimensional vector space \(\V\) over \(\F\), and let \(\beta\) be an ordered basis for \(\V\) over \(\F\).
  Prove that \(\lambda\) is an eigenvalue of \(\T\) iff \(\lambda\) is an eigenvalue of \([\T]_{\beta}\).
\end{ex}

\begin{proof}[\pf{ex:5.1.6}]
  We have
  \begin{align*}
         & \lambda \text{ is an eigenvalue of } \T                                                                             \\
    \iff & \exists v \in \V \setminus \set{\zv} : \T(v) = \lambda v                              &  & \text{(by \cref{5.1.2})} \\
    \iff & \exists v \in \V \setminus \set{\zv} : [\T(v)]_{\beta} = [\lambda v]_{\beta}          &  & \text{(by \cref{2.21})}  \\
    \iff & \exists v \in \V \setminus \set{\zv} : [\T]_{\beta} [v]_{\beta} = \lambda [v]_{\beta} &  & \text{(by \cref{2.14})}  \\
    \iff & \lambda \text{ is an eigenvalue of } [\T]_{\beta}.                                    &  & \text{(by \cref{5.1.2})}
  \end{align*}
\end{proof}

\begin{ex}\label{ex:5.1.12}
  \begin{enumerate}
    \item Prove that similar matrices have the same characteristic polynomial.
    \item Show that the definition of the characteristic polynomial of a linear operator on a finite-dimensional vector space \(\V\) over \(\F\) is independent of the choice of basis for \(\V\) over \(\F\).
  \end{enumerate}
\end{ex}

\begin{proof}[\pf{ex:5.1.12}(a)]
  Let \(A, B \in \ms{n}{n}{\F}\) such that \(A, B\) are similar.
  By \cref{2.5.4} there exists some \(Q \in \ms{n}{n}{\F}\) such that \(B = Q^{-1} A Q\).
  Then we have
  \begin{align*}
    \det(A - t I_n) & = \det(I_n) \det(A - t I_n)            &  & \text{(by \cref{4.2.3})} \\
                    & = \det(Q^{-1} Q) \det(A - t I_n)       &  & \text{(by \cref{2.4.3})} \\
                    & = \det(Q^{-1}) \det(Q) \det(A - t I_n) &  & \text{(by \cref{4.7})}   \\
                    & = \det(Q^{-1}) \det(A - t I_n) \det(Q)                               \\
                    & = \det(Q^{-1} (A - t I_n) Q)           &  & \text{(by \cref{4.7})}   \\
                    & = \det(Q^{-1} A Q - t Q^{-1} I_n Q)    &  & \text{(by \cref{2.3.5})} \\
                    & = \det(B - t I_n).                     &  & \text{(by \cref{2.4.3})}
  \end{align*}
\end{proof}

\begin{proof}[\pf{ex:5.1.12}(b)]
  Let \(\beta, \beta'\) be ordered basis for \(\V\) over \(\F\).
  By \cref{2.23} we know that \([\T]_{\beta}\) and \([\T]_{\beta'}\) are similar.
  Thus by \cref{ex:5.1.12}(a) we have \(\det([\T]_{\beta} - t I_n) = \det([\T]_{\beta'} - t I_n)\).
  We conclude that the definition of the characteristic polynomial of a linear operator on a finite-dimensional vector space \(\V\) over \(\F\) is independent of the choice of basis for \(\V\) over \(\F\).
\end{proof}

\section{Diagonalizability}\label{sec:5.2}

\begin{thm}\label{5.5}
  Let \(\T\) be a linear operator on a vector space \(\V\) over \(\F\), and let \(\seq{\lambda}{1,,k}\) be distinct eigenvalues of \(\T\).
  If \(\seq{v}{1,,k}\) are eigenvectors of \(\T\) such that \(\lambda_i\) corresponds to \(v_i\) for all \(i \in \set{1, \dots, k}\), then \(\set{\seq{v}{1,,k}}\) is linearly independent.
\end{thm}

\begin{proof}[\pf{5.5}]
  The proof is by mathematical induction on \(k\).
  Suppose that \(k = 1\).
  Then \(v_1 \neq \zv\) since \(v_1\) is an eigenvector, and hence \(\set{v_1}\) is linearly independent.
  Now assume that the theorem holds for \(k\) distinct eigenvalues, where \(k \geq 1\), and that we have \(k + 1\) eigenvectors \(\seq{v}{1,,k+1}\) corresponding to the distinct eigenvalues \(\seq{\lambda}{1,,k+1}\).
  We wish to show that \(\set{\seq{v}{1,,k+1}}\) is linearly independent.
  Suppose that \(\seq{a}{1,,k+1} \in \F\) such that
  \[
    \seq[+]{a,v}{1,2,,k+1} = \zv.
  \]
  Applying \(\T - \lambda_{k + 1} \IT[\V]\) to both sides of the above equation, we obtain
  \[
    a_1 (\lambda_1 - \lambda_{k + 1}) v_1 + a_2 (\lambda_2 - \lambda_{k + 1}) v_2 + \cdots + a_k (\lambda_k - \lambda_{k + 1}) v_k = \zv.
  \]
  By the induction hypothesis \(\set{\seq{v}{1,,k}}\) is linearly independent, and
  hence
  \[
    a_1 (\lambda_1 - \lambda_{k + 1}) = a_2 (\lambda_2 - \lambda_{k + 1}) = \cdots = a_k (\lambda_k - \lambda_{k + 1}) = 0.
  \]
  Since \(\seq{\lambda}{1,,k+1}\) are distinct, it follows that \(\lambda_i - \lambda_{k + 1} \neq 0\) for \(i \in \set{1, \dots, k}\).
  So \(\seq[=]{a}{1,,k} = 0\), and therefore \(\seq[+]{a,v}{1,,k+1} = \zv\) reduces to \(a_{k + 1} v_{k + 1} = \zv\).
  But \(v_{k + 1} \neq \zv\) and therefore \(a_{k + 1} = 0\).
  Consequently \(\seq[=]{a}{1,,k+1} = 0\), and it follows that \(\set{\seq{v}{1,,k+1}}\) is linearly independent.
\end{proof}

\begin{cor}\label{5.2.1}
  Let \(\T\) be a linear operator on an \(n\)-dimensional vector space \(\V\) over \(\F\).
  If \(\T\) has \(n\) distinct eigenvalues, then \(\T\) is diagonalizable.
\end{cor}

\begin{proof}[\pf{5.2.1}]
  Suppose that \(\T\) has \(n\) distinct eigenvalues \(\seq{\lambda}{1,,n}\).
  For each \(i \in \set{1, \dots, n}\) choose an eigenvector \(v_i\) corresponding to \(\lambda_i\).
  By \cref{5.5}, \(\set{\seq{v}{1,,n}}\) is linearly independent, and since \(\dim(\V) = n\), this set is a basis for \(\V\) over \(\F\).
  Thus by \cref{5.1} \(\T\) is diagonalizable.
\end{proof}

\begin{note}
  The converse of \cref{5.5} is false.
  That is, it is not true that if \(\T\) is diagonalizable, then it has \(n\) distinct eigenvalues.
  For example, the identity operator is diagonalizable even though it has only one eigenvalue, namely, \(\lambda = 1\).
\end{note}

\begin{defn}\label{5.2.2}
  A polynomial \(f\) in \(\ps{\F}\) \textbf{splits over} \(\F\) if there are scalars \(c, \seq{a}{1,,n}\) (not necessarily distinct) in \(\F\) such that
  \[
    f(t) = c (t - a_1) (t - a_2) \cdots (t - a_n).
  \]
  If \(f\) is the characteristic polynomial of a linear operator or a matrix over a field \(\F\), then the statement that \(f\) splits is understood to mean that it splits over \(\F\).
\end{defn}

\begin{thm}\label{5.6}
  The characteristic polynomial of any diagonalizable linear operator splits.
\end{thm}

\begin{proof}[\pf{5.6}]
  Let \(\T\) be a diagonalizable linear operator on the \(n\)-dimensional vector space \(\V\) over \(\F\), and let \(\beta\) be an ordered basis for \(\V\) over \(\F\) such that \([\T]_{\beta} = D\) is a diagonal matrix.
  Suppose that
  \[
    D = \begin{pmatrix}
      \lambda_1 & 0         & \cdots & 0         \\
      0         & \lambda_2 & \cdots & 0         \\
      \vdots    & \vdots    &        & \vdots    \\
      0         & 0         & \cdots & \lambda_n
    \end{pmatrix},
  \]
  and let \(f\) be the characteristic polynomial of \(\T\).
  Then
  \begin{align*}
    f(t) & = \det(D - t I_n)                                                &  & \text{(by \cref{5.1.6})}     \\
         & = \det\begin{pmatrix}
                   \lambda_1 - t & 0             & \cdots & 0             \\
                   0             & \lambda_2 - t & \cdots & 0             \\
                   \vdots        & \vdots        &        & \vdots        \\
                   0             & 0             & \cdots & \lambda_n - t
                 \end{pmatrix}                                       \\
         & = (\lambda_1 - t) (\lambda_2 - t) \cdots (\lambda_n - t)         &  & \text{(by \cref{ex:4.2.23})} \\
         & = (-1)^n (t - \lambda_1) (t - \lambda_2) \cdots (t - \lambda_n).
  \end{align*}
\end{proof}

\begin{note}
  From \cref{5.6}, it is clear that if \(\T\) is a diagonalizable linear operator on an \(n\)-dimensional vector space that fails to have distinct eigenvalues, then the characteristic polynomial of \(\T\) must have repeated zeros.

  The converse of \cref{5.6} is false;
  that is, the characteristic polynomial of \(\T\) may split, but \(\T\) need not be diagonalizable.
\end{note}

\begin{defn}\label{5.2.3}
  Let \(\lambda\) be an eigenvalue of a linear operator or matrix with characteristic polynomial \(f\).
  The \textbf{(algebraic) multiplicity} of \(\lambda\) is the largest positive integer \(k\) for which \((t - \lambda)^k\) is a factor of \(f\).
\end{defn}

\begin{note}
  If \(\T\) is a diagonalizable linear operator on a \(n\)-dimensional vector space \(\V\) over \(\F\), then there is an ordered basis \(\beta\) for \(\V\) over \(\F\) consisting of eigenvectors of \(\T\).
  We know from \cref{5.1} that \([\T]_{\beta}\) is a diagonal matrix in which the diagonal entries are the eigenvalues of \(\T\).
  Since the characteristic polynomial of \(\T\) is \(\det([\T]_{\beta} - t I_n)\), it is easily seen that each eigenvalue of \(\T\) must occur as a diagonal entry of \([\T]_{\beta}\) exactly as many times as its multiplicity.
  Hence \(\beta\) contains as many (linearly independent) eigenvectors corresponding to an eigenvalue as the multiplicity of that eigenvalue.
  So the number of linearly independent eigenvectors corresponding to a given eigenvalue is of interest in determining whether an operator can be diagonalized.
  Recalling from \cref{5.4} that the eigenvectors of \(\T\) corresponding to the eigenvalue \(\lambda\) are the nonzero vectors in the null space of \(\T - \lambda \IT[\V]\), we are led naturally to the study of this set.
\end{note}

\begin{defn}\label{5.2.4}
  Let \(\T\) be a linear operator on a vector space \(\V\) over \(\F\), and let \(\lambda\) be an eigenvalue of \(\T\).
  Define \(E_\lambda = \set{x \in \V : \T(x) = \lambda x} = \ns{\T - \lambda \IT[\V]}\).
  The set \(E_\lambda\) is called the \textbf{eigenspace} of \(\T\) corresponding to the eigenvalue \(\lambda\).
  Analogously, we define the \textbf{eigenspace} of a square matrix \(A\) to be the eigenspace of \(\L_A\).
\end{defn}

\begin{note}
  Clearly, \(E_{\lambda}\) is a subspace of \(\V\) over \(\F\) consisting of the zero vector and the eigenvectors of \(\T\) corresponding to the eigenvalue \(\lambda\).
  The maximum number of linearly independent eigenvectors of \(\T\) corresponding to the eigenvalue \(\lambda\) is therefore the dimension of \(E_{\lambda}\).
  \cref{5.7} relates this dimension to the multiplicity of \(\lambda\).
\end{note}

\begin{thm}\label{5.7}
  Let \(\T\) be a linear operator on a finite-dimensional vector space \(\V\) over \(\F\), and let \(\lambda\) be an eigenvalue of \(\T\) having multiplicity \(m\).
  Then \(1 \leq \dim(E_{\lambda}) \leq m\).
\end{thm}

\begin{proof}[\pf{5.7}]
  Choose an ordered basis \(\set{\seq{v}{1,,p}}\) for \(E_{\lambda}\), extend it to an ordered basis \(\beta = \set{\seq{v}{1,,p,p+1,,n}}\) for \(\V\) over \(\F\), and let \(A = [\T]_{\beta}\).
  Observe that for each \(i \in \set{1, \dots p}\), \(v_i\) is an eigenvector of \(\T\) corresponding to \(\lambda\), and therefore
  \[
    A = \begin{pmatrix}
      \lambda I_p & B \\
      \zm         & C
    \end{pmatrix}.
  \]
  By \cref{ex:4.3.21}, the characteristic polynomial of \(\T\) is
  \begin{align*}
    f(t) & = \det(A - t I_n)                               &  & \text{(by \cref{5.1.6})}     \\
         & = \det\begin{pmatrix}
                   (\lambda - t) I_p & B               \\
                   \zm               & C - t I_{n - p}
                 \end{pmatrix}                                         \\
         & = \det((\lambda - t) I_p) \det(C - t I_{n - p}) &  & \text{(by \cref{ex:4.3.21})} \\
         & = (\lambda - t)^p g(t),                         &  & \text{(by \cref{ex:4.2.23})}
  \end{align*}
  where \(g\) is a polynomial.
  Thus \((\lambda - t)^p\) is a factor of \(f\), and hence the multiplicity of \(\lambda\) is at least \(p\).
  But \(\dim(E_{\lambda}) = p\), and so \(\dim(E_{\lambda}) \leq m\).
\end{proof}

\begin{lem}\label{5.2.5}
  Let \(\T\) be a linear operator, and let \(\seq{\lambda}{1,,k}\) be distinct eigenvalues of \(\T\).
  For each \(i \in \set{1, \dots, k}\), let \(v_i \in E_{\lambda_i}\), the eigenspace corresponding to \(\lambda_i\).
  If
  \[
    \seq[+]{v}{1,,k} = \zv,
  \]
  then \(v_i = \zv\) for all \(i \in \set{1, \dots, k}\).
\end{lem}

\begin{proof}[\pf{5.2.5}]
  Suppose otherwise.
  By renumbering if necessary, suppose that, for \(m \in \set{1, \dots, k}\), we have \(v_i \neq \zv\) for \(i \in \set{1, \dots, m}\), and \(v_i = \zv\) for \(i \in \set{m + 1, \dots, k}\).
  Then, for each \(i \leq m\), \(v_i\) is an eigenvector of \(\T\) corresponding to \(\lambda_i\) and
  \[
    \seq[+]{v}{1,,m} = \zv.
  \]
  But this contradicts \cref{5.5}, which states that these \(v_i\)'s are linearly independent.
  We conclude, therefore, that \(v_i = \zv\) for all \(i \in \set{1, \dots, k}\).
\end{proof}

\begin{thm}\label{5.8}
  Let \(\T\) be a linear operator on a vector space \(\V\) over \(\F\), and let \(\seq{\lambda}{1,,k}\) be distinct eigenvalues of \(\T\).
  For each \(i \in \set{1, \dots, k}\), let \(S_i\) be a finite linearly independent subset of the eigenspace \(E_{\lambda_i}\).
  Then \(S = \bigcup_{i = 1}^k S_i\) is a linearly independent subset of \(\V\).
\end{thm}

\begin{proof}[\pf{5.8}]
  Suppose that for each \(i \in \set{1, \dots, k}\)
  \[
    S_i = \set{v_{i 1}, v_{i 2}, \dots v_{i n_i}}.
  \]
  Then \(S = \set{v_{i j} : (1 \leq j \leq n_i) \land (1 \leq i \leq k)}\).
  Consider any scalars \(\set{a_{i j}} \subseteq \F\) such that
  \[
    \sum_{i = 1}^k \sum_{j = 1}^{n_i} a_{i j} v_{i j} = \zv.
  \]
  For each \(i \in \set{1, \dots, k}\), let
  \[
    w_i = \sum_{j = 1}^{n_i} a_{i j} v_{i j}.
  \]
  Then \(w_i \in E_{\lambda_i}\) for each \(i \in \set{1, \dots, k}\), and \(\seq[+]{w}{1,,k} = \zv\).
  Therefore, by \cref{5.2.5}, \(w_i = \zv\) for all \(i \in \set{1, \dots, k}\).
  But each \(S_i\) is linearly independent, and hence \(a_{i j} = 0\) for all \(j \in \set{1, \dots, n_i}\).
  We conclude that \(S\) is linearly independent.
\end{proof}

\begin{thm}\label{5.9}
  Let \(\T\) be a linear operator on a finite-dimensional vector space \(\V\) over \(\F\) such that the characteristic polynomial of \(\T\) splits.
  Let \(\seq{\lambda}{1,,k}\) be the distinct eigenvalues of \(\T\).
  Then
  \begin{enumerate}
    \item \(\T\) is diagonalizable iff the multiplicity of \(\lambda_i\) is equal to \(\dim(E_{\lambda_i})\) for all \(i \in \set{1, \dots, k}\).
    \item If \(\T\) is diagonalizable and \(\beta_i\) is an ordered basis for \(E_{\lambda_i}\) for each \(i \in \set{1, \dots, k}\), then \(\beta = \bigcup_{i = 1}^k \beta_i\) is an ordered basis for \(\V\) over \(\F\) consisting of eigenvectors of \(\T\).
  \end{enumerate}
\end{thm}

\begin{proof}[\pf{5.9}]
  For each \(i \in \set{1, \dots, k}\), let \(m_i\) denote the multiplicity of \(\lambda_i\), \(d_i = \dim(E_{\lambda_i})\), and \(n = \dim(\V)\).

  First, suppose that \(\T\) is diagonalizable.
  Let \(\beta\) be a basis for \(\V\) over \(\F\) consisting of eigenvectors of \(\T\).
  For each \(i \in \set{1, \dots, k}\), let \(\beta_i = \beta \cap E_{\lambda_i}\), the set of vectors in \(\beta\) that are eigenvectors corresponding to \(\lambda_i\), and let \(n_i\) denote the number of vectors in \(\beta_i\).
  Then \(n_i \leq d_i\) for each \(i \in \set{1, \dots, k}\) because \(\beta_i\) is a linearly independent subset of a subspace of dimension \(d_i\), and \(d_i \leq m_i\) by \cref{5.7}.
  The \(n_i\)'s sum to \(n\) because \(\beta\) contains \(n\) vectors.
  The \(m_i\)'s also sum to \(n\) because the degree of the characteristic polynomial of \(\T\) is equal to the sum of the multiplicities of the eigenvalues.
  Thus
  \[
    n = \sum_{i = 1}^k n_i \leq \sum_{i = 1}^k d_i \leq \sum_{i = 1}^k m_i = n.
  \]
  It follows that
  \[
    \sum_{i = 1}^k (m_i - d_i) = 0.
  \]
  Since \((m_i - d_i) \geq 0\) for all \(i \in \set{1, \dots, k}\), we conclude that \(m_i = d_i\) for all \(i \in \set{1, \dots, k}\).

  Conversely, suppose that \(m_i = d_i\) for all \(i \in \set{1, \dots, k}\).
  We simultaneously show that \(\T\) is diagonalizable and prove (b).
  For each \(i \in \set{1, \dots, k}\), let \(\beta_i\) be an ordered basis for \(E_{\lambda_i}\), and let \(\beta = \bigcup_{i = 1}^k \beta_i\).
  By \cref{5.8}, \(\beta\) is linearly independent.
  Furthermore, since \(d_i = m_i\) for all \(i \in \set{1, \dots, k}\), \(\beta\) contains
  \[
    \sum_{i = 1}^k d_i = \sum_{i = 1}^k m_i = n
  \]
  vectors.
  Therefore \(\beta\) is an ordered basis for \(\V\) over \(\F\) consisting of eigenvectors of \(\V\), and we conclude that \(\T\) is diagonalizable.
\end{proof}

\begin{note}
  Let \(\T\) be a linear operator on an \(n\)-dimensional vector space \(\V\) over \(\F\).
  Then \(\T\) is diagonalizable iff both of the following conditions hold.
  \begin{itemize}
    \item The characteristic polynomial of \(\T\) splits.
    \item For each eigenvalue \(\lambda\) of \(\T\), the multiplicity of \(\lambda\) equals \(n - \rk{\T - \lambda \IT[\V]}\).
  \end{itemize}
  These same conditions can be used to test if a square matrix \(A\) is diagonalizable because diagonalizability of \(A\) is equivalent to diagonalizability of the operator \(\L_A\).

  If \(\T\) is a diagonalizable operator and \(\seq{\beta}{1,,k}\) are ordered bases for the eigenspaces of \(\T\), then the union \(\beta = \bigcup_{i = 1}^k \beta_i\) is an ordered basis for \(\V\) over \(\F\) consisting of eigenvectors of \(\T\), and hence \([\T]_{\beta}\) is a diagonal matrix.

  When testing \(\T\) for diagonalizability, it is usually easiest to choose a convenient basis \(\alpha\) for \(\V\) over \(\F\) and work with \(B = [\T]_{\alpha}\).
  If the characteristic polynomial of \(B\) splits, then use condition 2 above to check if the multiplicity of each \emph{repeated} eigenvalue of \(B\) equals \(n - \rk{B - \lambda I_n}\).
  (By \cref{5.7}, condition 2 is automatically satisfied for eigenvalues of multiplicity \(1\).)
  If so, then \(B\), and hence \(\T\), is diagonalizable.

  If \(\T\) is diagonalizable and a basis \(\beta\) for \(\V\) over \(\F\) consisting of eigenvectors of \(\T\) is desired, then we first find a basis for each eigenspace of \(B\).
  The union of these bases is a basis \(\gamma\) for \(\vs{F}^n\) consisting of eigenvectors of \(B\).
  Each vector in \(\gamma\) is the coordinate vector relative to \(\alpha\) of an eigenvector of \(\T\).
  The set consisting of these \(n\) eigenvectors of \(\T\) is the desired basis \(\beta\).

  Furthermore, if \(A\) is an \(n \times n\) diagonalizable matrix, we can use the \cref{2.5.3} to find an invertible \(n \times n\) matrix \(Q\) and a diagonal \(n \times n\) matrix \(D\) such that \(Q^{-1} A Q = D\).
  The matrix \(Q\) has as its columns the vectors in a basis of eigenvectors of \(A\), and \(D\) has as its \(j\)th diagonal entry the eigenvalue of \(A\) corresponding to the \(j\)th column of \(Q\).
\end{note}

\begin{defn}\label{5.2.6}
  Let \(\seq{\W}{1,,k}\) be subspaces of a vector space \(\V\) over \(\F\).
  We define the \textbf{sum} of these subspaces to be the set
  \[
    \set{\seq[+]{v}{1,,k} : v_i \in \W_i \text{ for } i \in \set{1, \dots, k}},
  \]
  which we denote by \(\seq[+]{\W}{1,,k}\) or \(\sum_{i = 1}^k \W_i\).
\end{defn}

\begin{defn}\label{5.2.7}
  Let \(\seq{\W}{1,,k}\) be subspaces of a vector space \(\V\) over \(\F\).
  We call \(\V\) the \textbf{direct sum} of the subspaces \(\seq{\W}{1,,k}\) and write \(\V = \seq[\oplus]{\W}{1,,k}\), if
  \[
    \V = \sum_{i = 1}^k \W_i
  \]
  and
  \[
    \W_j \cap \sum_{i \neq j} \W_i = \set{\zv} \quad \text{for each } j \in \set{1, \dots, k}.
  \]
\end{defn}

\begin{thm}\label{5.10}
  Let \(\seq{\W}{1,,k}\) be subspaces of a finite-dimensional vector space \(\V\) over \(\F\).
  The following conditions are equivalent.
  \begin{enumerate}
    \item \(\V = \seq[\oplus]{\W}{1,,k}\).
    \item \(\V = \sum_{i = 1}^k \W_i\) and, for any vectors \(\seq{v}{1,,k}\) such that \(v_i \in \W\) for \(i \in \set{1, \dots, k}\), if \(\seq[+]{v}{1,,k} = \zv\), then \(v_i = \zv\) for all \(i \in \set{1, \dots, k}\).
    \item Each vector \(v \in \V\) can be uniquely written as \(v = \seq[+]{v}{1,,k}\), where \(v_i \in \W_i\) for all \(i \in \set{1, \dots, k}\).
    \item For each \(i \in \set{1, \dots, k}\), if \(\gamma_i\) is an ordered basis for \(\W_i\) over \(\F\), then \(\bigcup_{i = 1}^k \gamma_i\) is an ordered basis for \(\V\) over \(\F\).
    \item For each \(i \in \set{1, \dots, k}\), there exists an ordered basis \(\gamma_i\) for \(\W_i\) over \(\F\) such that \(\bigcup_{i = 1}^k \gamma_i\) is an ordered basis for \(\V\) over \(\F\).
  \end{enumerate}
\end{thm}

\begin{proof}[\pf{5.10}]
  Assume (a).
  We prove (b).
  Clearly
  \[
    \V = \sum_{i = 1}^k \W_i.
  \]
  Now suppose that \(\seq{v}{1,,k}\) are vectors such that \(v_i \in \W_i\) for all \(i \in \set{1, \dots, k}\) and \(\seq[+]{v}{1,,k} = \zv\).
  Then for any \(j \in \set{1, \dots, k}\)
  \[
    -v_j = \sum_{\substack{i = 1 \\ i \neq j}}^k v_i \in \sum_{\substack{i = 1 \\ i \neq j}}^k \W_i.
  \]
  But \(-v_j \in \W_j\) and hence
  \[
    -v_j \in \W_j \cap \sum_{\substack{i = 1 \\ i \neq j}}^k \W_i = \set{\zv}.
  \]
  So \(v_j = \zv\), proving (b).

  Now assume (b).
  We prove (c).
  Let \(v \in \V\).
  By (b), there exist vectors \(\seq{v}{1,,k}\) such that \(v_i \in \W_i\) for all \(i \in \set{1, \dots, k}\) and \(v = \seq[+]{v}{1,,k}\).
  We must show that this representation is unique.
  Suppose also that \(v = \seq[+]{w}{1,,k}\), where \(w_i \in \W_i\) for all \(i \in \set{1, \dots, k}\).
  Then
  \[
    (v_1 - w_1) + \cdots + (v_k - w_k) = \zv.
  \]
  But \(v_i - w_i \in \W_i\) for all \(i \in \set{1, \dots, k}\), and therefore \(v_i - w_i = \zv\) for all \(i \in \set{1, \dots, k}\) by (b).
  Thus \(v_i = w_i\) for all \(i \in \set{1, \dots, k}\), proving the uniqueness of the representation.

  Now assume (c).
  We prove (d).
  For each \(i \in \set{1, \dots, k}\), let \(\gamma_i\) be an ordered basis for \(\W_i\) over \(\F\).
  Since
  \[
    \V = \sum_{i = 1}^k \W_i
  \]
  by (c), it follows that \(\bigcup_{i = 1}^k \gamma_i\) generates \(\V\).
  To show that this set is linearly independent, consider vectors \(v_{i j} \in \gamma_i\) (where \(j \in \set{1, \dots, m_i}\) and \(i \in \set{1, \dots, k}\)) and scalars \(a_{i j} \in \F\) such that
  \[
    \sum_{i = 1}^k \sum_{j = 1}^{m_i} a_{i j} v_{i j} = \zv.
  \]
  For each \(i \in \set{1, \dots, k}\), set
  \[
    w_i = \sum_{j = 1}^{m_i} a_{i j} v_{i j}.
  \]
  Then for each \(i \in \set{1, \dots, k}\), \(w_i \in \spn{\gamma_i} = \W_i\) and
  \[
    \seq[+]{w}{1,,k} = \sum_{i = 1}^k \sum_{j = 1}^{m_i} a_{i j} v_{i j} = \zv.
  \]
  Since \(\zv \in \W_i\) for each \(i \in \set{1, \dots, k}\) and \(\zv + \cdots + \zv = \seq[+]{w}{1,,k}\), (c) implies that \(w_i = \zv\) for all \(i \in \set{1, \dots, k}\).
  Thus
  \[
    \zv = w_i = \sum_{j = 1}^{m_i} a_{i j} v_{i j}
  \]
  for each \(i \in \set{1, \dots, k}\).
  But each \(\gamma_i\) is linearly independent, and hence \(a_{i j} = 0\) for all \(i \in \set{1, \dots, k}\) and \(j \in \set{1, \dots, m_i}\).
  Consequently \(\bigcup_{i = 1}^k \gamma_i\) is linearly independent and therefore is a basis for \(\V\) over \(\F\).

  Clearly (e) follows immediately from (d).

  Finally, we assume (e) and prove (a).
  For each \(i \in \set{1, \dots, k}\), let \(\gamma_i\) be an ordered basis for \(\W_i\) over \(\F\) such that \(\bigcup_{i = 1}^k \gamma_i\) is an ordered basis for \(\V\) over \(\F\).
  Then
  \[
    \V = \spn{\bigcup_{i = 1}^k \gamma_i} = \sum_{i = 1}^k \spn{\gamma_i} = \sum_{i = 1}^k \W_i
  \]
  by repeated applications of \cref{ex:1.4.14}.
  Fix \(j \in \set{1, \dots, k}\), and suppose that, for some nonzero vector \(v \in \V\),
  \[
    v \in \W_j \cap \sum_{\substack{i = 1 \\ i \neq j}}^k \W_i.
  \]
  Then
  \[
    v \in \W_j = \spn{\gamma_j} \quad \text{and} \quad v \in \sum_{\substack{i = 1 \\ i \neq j}}^k \W_i = \spn{\bigcup_{\substack{i = 1 \\ i \neq j}}^k \gamma_i}.
  \]
  Hence \(v\) is a nontrivial linear combination of both \(\gamma_j\) and \(\bigcup_{i \neq j} \gamma_i\), so that \(v\) can be expressed as a linear combination of \(\bigcup_{i = 1}^k \gamma_i\) in more than one way.
  But these representations contradict \cref{1.8}, and so we conclude that
  \[
    \W_j \cap \sum_{\substack{i = 1 \\ i \neq j}}^k \W_i = \set{\zv},
  \]
  proving (a).
\end{proof}

\begin{thm}\label{5.11}
  A linear operator \(\T\) on a finite-dimensional vector space \(\V\) over \(\F\) is diagonalizable iff \(\V\) is the direct sum of the eigenspaces of \(\T\).
\end{thm}

\begin{proof}[\pf{5.11}]
  Let \(\seq{\lambda}{1,,k}\) be the distinct eigenvalues of \(\T\).

  First suppose that \(\T\) is diagonalizable, and for each \(i \in \set{1, \dots, k}\) choose an ordered basis \(\gamma_i\) for the eigenspace \(E_{\lambda_i}\) over \(\F\).
  By \cref{5.9}, \(\bigcup_{i = 1}^k \gamma_i\) is a basis for \(\V\) over \(\F\), and hence \(\V\) is a direct sum of the \(E_{\lambda_i}\)'s by \cref{5.10}.

  Conversely, suppose that \(\V\) is a direct sum of the eigenspaces of \(\T\).
  For each \(i \in \set{1, \dots, k}\), choose an ordered basis \(\gamma_i\) for \(E_{\lambda_i}\) over \(\F\).
  By \cref{5.10}, the union \(\bigcup_{i = 1}^k \gamma_i\) is a basis for \(\V\) over \(\F\).
  Since this basis consists of eigenvectors of \(\T\), we conclude that \(\T\) is diagonalizable.
\end{proof}

\exercisesection

\setcounter{ex}{3}
\begin{ex}\label{ex:5.2.4}
  Prove the matrix version of \cref{5.2.1}:
  If \(A \in \ms{n}{n}{\F}\) has \(n\) distinct eigenvalues, then \(A\) is diagonalizable.
\end{ex}

\begin{proof}[\pf{ex:5.2.4}]
  We have
  \begin{align*}
             & A \in \ms{n}{n}{\F} \text{ has } n \text{ distinct  eigenvalues}                               \\
    \implies & \L_A \text{ has } n \text{ distinct  eigenvalues}                &  & \text{(by \cref{5.1.2})} \\
    \implies & \L_A \text{ is diagonalizable}                                   &  & \text{(by \cref{5.2.1})} \\
    \implies & A \text{ is diagonalizable}.                                     &  & \text{(by \cref{5.1.1})}
  \end{align*}
\end{proof}

\begin{ex}\label{ex:5.2.5}
  State and prove the matrix version of \cref{5.6}.
\end{ex}

\begin{proof}[\pf{ex:5.2.5}]
  We claim that the characteristic polynomial of any diagonalizable matrix splits.
  This is true since
  \begin{align*}
             & A \text{ is diagonalizable}                                                                 \\
    \implies & \L_A \text{ is diagonalizable}                                &  & \text{(by \cref{5.1.1})} \\
    \implies & \text{ the characteristic polynomial of } \L_A \text{ splits} &  & \text{(by \cref{5.6})}   \\
    \implies & \text{ the characteristic polynomial of } A \text{ splits}.   &  & \text{(by \cref{5.1.6})}
  \end{align*}
\end{proof}

\setcounter{ex}{7}
\begin{ex}\label{ex:5.2.8}
  Suppose that \(A \in \ms{n}{n}{\F}\) has two distinct eigenvalues, \(\lambda_1\) and \(\lambda_2\), and that \(\dim(E_{\lambda_1}) = n - 1\).
  Prove that \(A\) is diagonalizable.
\end{ex}

\begin{proof}[\pf{ex:5.2.8}]
  Let \(\beta_1, \beta_2\) be ordered bases for \(E_{\lambda_1}, E_{\lambda_2}\) over \(\F\), respectively.
  By \cref{5.8} we know that \(\beta = \beta_1 \cup \beta_2\) is linearly independent.
  Since
  \begin{align*}
             & \begin{dcases}
                 \dim(E_{\lambda_1}) = n - 1 \\
                 \dim(E_{\lambda_2}) \geq 1
               \end{dcases}                                    &  & \text{(by \cref{5.7})}       \\
    \implies & \begin{dcases}
                 \#(\beta_1) = n - 1 \\
                 1 \leq \#(\beta_2) \leq n
               \end{dcases}                                         &  & \text{(by \cref{1.11})} \\
    \implies & n = (n - 1) + 1 \leq \#(\beta_1) + \#(\beta_2) = \#(\beta) \leq n                 \\
    \implies & \#(\beta) = n,
  \end{align*}
  by \cref{1.6.15}(b) we know that \(\beta\) is an ordered basis for \(\vs{F}^n\) over \(\F\).
  Since \(\beta\) is consist of eigenvectors, by \cref{5.1.1} we know that \(A\) is diagonalizable.
\end{proof}

\begin{ex}\label{ex:5.2.9}
  Let \(\T\) be a linear operator on a \(n\)-dimensional vector space \(\V\) over \(\F\), and suppose there exists an ordered basis \(\beta\) for \(\V\) over \(\F\) such that \([\T]_{\beta}\) is an upper triangular matrix.
  \begin{enumerate}
    \item Prove that the characteristic polynomial for \(\T\) splits.
    \item State and prove an analogous result for matrices.
  \end{enumerate}
  The converse of (a) is treated in \cref{ex:5.4.32}.
\end{ex}

\begin{proof}[\pf{ex:5.2.9}(a)]
  Let \(A = [\T]_{\beta}\).
  Since \(A\) and \(t I_n\) are upper triangular matrices, by \cref{ex:1.3.12} we know that \(A - t I_n\) is also an upper triangular matrix.
  Thus by \cref{ex:4.2.23} we have \(\det(A - t I_n) = \prod_{i = 1}^n (A_{i i} - t) = (-1)^n \prod_{i = 1}^n (t - A_{i i})\).
  By \cref{5.2.2} this means the characteristic polynomial for \(\T\) splits.
\end{proof}

\begin{proof}[\pf{ex:5.2.9}(b)]
  Let \(A \in \ms{n}{n}{\F}\) be an upper triangular matrix.
  We claim that the characteristic polynomial for \(A\) splits.
  Let \(\beta\) be the standard ordered basis for \(\vs{F}^n\).
  By \cref{2.15}(a) we have \(A = [\L_A]_{\beta}\), thus by \cref{ex:5.2.9}(a) and \cref{5.1.6} the characteristic polynomial for \(A\) splits.
\end{proof}

\begin{ex}\label{ex:5.2.10}
  Let \(\T\) be a linear operator on a \(n\)-dimensional vector space \(\V\) over \(\F\) with the distinct eigenvalues \(\seq{\lambda}{1,,k}\) and corresponding multiplicities \(\seq{m}{1,,k}\).
  Suppose that \(\beta\) is a basis for \(\V\) over \(\F\) such that \([\T]_{\beta}\) is an upper triangular matrix.
  Prove that the diagonal entries of \([\T]_{\beta}\) are \(\seq{\lambda}{1,,k}\) and that each \(\lambda_i\) occurs \(m_i\) times for all \(i \in \set{1, \dots, k}\).
\end{ex}

\begin{proof}[\pf{ex:5.2.10}]
  By \cref{ex:5.2.9}(a) we know that
  \[
    \det([\T]_{\beta} - t I_n) = \prod_{j = 1}^n (([\T]_{\beta})_{j j} - t).
  \]
  Thus by \cref{5.2} the diagonal entries of \([\T]_{\beta}\) are \(\seq{\lambda}{1,,k}\) and by \cref{5.2.3} each \(\lambda_i\) occurs \(m_i\) times for all \(i \in \set{1, \dots, k}\).
\end{proof}

\begin{ex}\label{ex:5.2.11}
  Let \(A \in \ms{n}{n}{\F}\) such that \(A\) is similar to an upper triangular matrix and has the distinct eigenvalues \(\seq{\lambda}{1,,k}\) with corresponding multiplicities \(\seq{m}{1,,k}\).
  Prove the following statements.
  \begin{enumerate}
    \item \(\tr(A) = \sum_{i = 1}^k m_i \lambda_i\).
    \item \(\det(A) = \prod_{i = 1}^k \lambda_i^{m_i}\).
  \end{enumerate}
\end{ex}

\begin{proof}[\pf{ex:5.2.11}]
  By \cref{2.5.4} there exists a \(Q \in \ms{n}{n}{\F}\) such that \(Q^{-1} A Q\) is an upper triangular matrix.
  If we let \(D = Q^{-1} A Q\), then by \cref{ex:5.1.12} we know that \(D\) and \(A\) have the same eigenvalues.
  Now let \(\beta\) be the standard ordered basis for \(\vs{F}^n\) over \(\F\).
  By \cref{2.15}(a) we know that \(D = [\L_D]_{\beta}\) is an upper triangular matrix, thus by \cref{ex:5.2.10} the diagonal entries of \(D\) are \(\seq{\lambda}{1,,k}\) and that each \(\lambda_i\) occurs \(m_i\) times for all \(i \in \set{1, \dots, k}\).
  Then we have
  \begin{align*}
    \tr(A) & = \tr(D)                       &  & \text{(by \cref{ex:2.5.10})} \\
           & = \sum_{i = 1}^k m_i \lambda_i &  & \text{(by \cref{ex:5.2.10})}
  \end{align*}
  and
  \begin{align*}
    \det(A) & = \det(D)                          &  & \text{(by \cref{ex:4.3.15})} \\
            & = \prod_{i = 1}^k \lambda_i^{m_i}. &  & \text{(by \cref{ex:5.2.10})}
  \end{align*}
\end{proof}

\begin{ex}\label{ex:5.2.12}
  Let \(\T\) be an invertible linear operator on a finite-dimensional vector space \(\V\) over \(\F\).
  \begin{enumerate}
    \item Recall that for any eigenvalue \(\lambda\) of \(\T\), \(\lambda^{-1}\) is an eigenvalue of \(\T^{-1}\) (\cref{ex:5.1.8}).
          Prove that the eigenspace of \(\T\) corresponding to \(\lambda\) is the same as the eigenspace of \(\T^{-1}\) corresponding to \(\lambda^{-1}\).
    \item Prove that if \(\T\) is diagonalizable, then \(\T^{-1}\) is diagonalizable.
  \end{enumerate}
\end{ex}

\begin{proof}[\pf{ex:5.2.12}(a)]
  We have
  \begin{align*}
         & v \in \ns{\T - \lambda \IT[\V]}            &  & \text{(by \cref{5.2.4})}       \\
    \iff & \T(v) - \lambda \IT[\V](v) = \zv           &  & \text{(by \cref{2.1.10})}      \\
    \iff & \T(v) = \lambda v                          &  & \text{(by \cref{2.1.9})}       \\
    \iff & \T^{-1}(v) = \lambda^{-1} v                &  & \text{(by \cref{ex:5.1.8}(b))} \\
    \iff & \T^{-1}(v) - \lambda^{-1} \IT[\V](v) = \zv &  & \text{(by \cref{2.1.9})}       \\
    \iff & v \in \ns{\T^{-1} - \lambda^{-1} \IT[\V]}  &  & \text{(by \cref{2.1.10})}
  \end{align*}
  and thus \(\ns{\T - \lambda \IT[\V]} = \ns{\T^{-1} - \lambda^{-1} \IT[\V]}\).
\end{proof}

\begin{proof}[\pf{ex:5.2.12}(b)]
  Let \(\seq{\lambda}{1,,k} \in \F\) be distinct eigenvalues of \(\T\) with corresponding multiplicities \(\seq{m}{1,,k}\).
  Then we have
  \begin{align*}
         & \T \text{ is diagonalizable}                                                                                      \\
    \iff & \forall i \in \set{1, \dots, k}, m_i = \nt{\T - \lambda_i \IT[\V]}           &  & \text{(by \cref{5.9})}          \\
    \iff & \forall i \in \set{1, \dots, k}, m_i = \nt{\T^{-1} - \lambda_i^{-1} \IT[\V]} &  & \text{(by \cref{ex:5.2.12}(a))} \\
    \iff & \T^{-1} \text{ is diagonalizable}.                                           &  & \text{(by \cref{5.9})}
  \end{align*}
\end{proof}

\begin{ex}\label{ex:5.2.13}
  Let \(A \in \ms{n}{n}{\F}\).
  Recall from \cref{ex:5.1.14} that \(A\) and \(\tp{A}\) have the same characteristic polynomial and hence share the same eigenvalues with the same multiplicities.
  For any eigenvalue \(\lambda\) of \(A\) and \(\tp{A}\), let \(E_{\lambda}\) and \(E_{\lambda}'\) denote the corresponding eigenspaces for \(A\) and \(\tp{A}\), respectively.
  \begin{enumerate}
    \item Show by way of example that for a given common eigenvalue, these two eigenspaces need not be the same.
    \item Prove that for any eigenvalue \(\lambda\), \(\dim(E_{\lambda}) = \dim(E_{\lambda}')\).
    \item Prove that if \(A\) is diagonalizable, then \(\tp{A}\) is also diagonalizable.
  \end{enumerate}
\end{ex}

\begin{proof}[\pf{ex:5.2.13}(a)]
  Let \(A = \begin{pmatrix}
    1 & -2 \\
    1 & 0
  \end{pmatrix} \in \ms{2}{2}{\R}\).
  Since
  \begin{align*}
             & \det(A - t I_2) = \det\begin{pmatrix}
                                       1 - t & -2 \\
                                       1     & t
                                     \end{pmatrix} = -(t - 2)(t + 1)            &  & \text{(by \cref{4.1.1})} \\
    \implies & 2 \text{ and } -1 \text{ are eigenvalues of } A, &  & \text{(by \cref{5.2})}
  \end{align*}
  we see that
  \begin{align*}
             & A - 2I_2 = \begin{pmatrix}
                            1 - 2 & -2 \\
                            1     & 2
                          \end{pmatrix} \begin{pmatrix}
                                          a \\
                                          b
                                        \end{pmatrix} = \begin{pmatrix}
                                                          0 \\
                                                          0
                                                        \end{pmatrix}                   \\
    \implies & (a, b) = t(2, -1) \text{ for all } t \in \R                               \\
    \implies & E_2 = \spn{\set{(2, -1)}}                   &  & \text{(by \cref{5.2.4})}
  \end{align*}
  and
  \begin{align*}
             & \tp{A} - 2I_2 = \begin{pmatrix}
                                 1 - 2 & 1 \\
                                 -2    & 2
                               \end{pmatrix} \begin{pmatrix}
                                               a \\
                                               b
                                             \end{pmatrix} = \begin{pmatrix}
                                                               0 \\
                                                               0
                                                             \end{pmatrix}             \\
    \implies & (a, b) = t(1, 1) \text{ for all } t \in \R                               \\
    \implies & E_2' = \spn{\set{(1, 1)}}.                 &  & \text{(by \cref{5.2.4})}
  \end{align*}
  Clearly \(E_2 \neq E_2'\).
\end{proof}

\begin{proof}[\pf{ex:5.2.13}(b)]
  We have
  \begin{align*}
    \dim(E_{\lambda}) & = \nt{\L_A - \lambda \IT[\vs{F}^n]}            &  & \text{(by \cref{5.2.4})}    \\
                      & = n - \rk{\L_A - \lambda \IT[\vs{F}^n]}        &  & \text{(by \cref{2.3})}      \\
                      & = n - \rk{A - \lambda I_n}                     &  & \text{(by \cref{3.2.1})}    \\
                      & = n - \rk{\tp{(A - \lambda I_n)}}              &  & \text{(by \cref{3.2.5}(a))} \\
                      & = n - \rk{\tp{A} - \lambda I_n}                &  & \text{(by \cref{ex:1.3.3})} \\
                      & = n - \rk{\L_{\tp{A}} - \lambda \IT[\vs{F}^n]} &  & \text{(by \cref{3.2.1})}    \\
                      & = \nt{\L_{\tp{A}} - \lambda \IT[\vs{F}^n]}     &  & \text{(by \cref{2.3})}      \\
                      & = \dim(E_{\lambda}').                          &  & \text{(by \cref{5.2.4})}
  \end{align*}
\end{proof}

\begin{proof}[\pf{ex:5.2.13}(c)]
  Let \(\seq{\lambda}{1,,k} \in \F\) be distinct eigenvalues of \(A\) with corresponding multiplicities \(\seq{m}{1,,k}\).
  Then we have
  \begin{align*}
         & A \text{ is diagonalizable}                                                                                       \\
    \iff & \forall i \in \set{1, \dots, k}, m_i = \dim(\ns{A - \lambda_i \IT[\V]})      &  & \text{(by \cref{5.9})}          \\
    \iff & \forall i \in \set{1, \dots, k}, m_i = \dim(\ns{\tp{A} - \lambda_i \IT[\V]}) &  & \text{(by \cref{ex:5.2.13}(b))} \\
    \iff & \tp{A} \text{ is diagonalizable}.                                            &  & \text{(by \cref{5.9})}
  \end{align*}
\end{proof}

\setcounter{ex}{14}
\begin{ex}\label{ex:5.2.15}
  Let
  \[
    A = \begin{pmatrix}
      a_{1 1} & a_{1 2} & \cdots & a_{1 n} \\
      a_{2 1} & a_{2 2} & \cdots & a_{2 n} \\
      \vdots  & \vdots  &        & \vdots  \\
      a_{n 1} & a_{n 2} & \cdots & a_{n n}
    \end{pmatrix} \in \ms{n}{n}{\R}
  \]
  be the coefficient matrix of the system of differential equations
  \begin{align*}
    x_1' & = a_{1 1} x_1 + a_{1 2} x_2 + \cdots + a_{1 n} x_n  \\
    x_2' & = a_{2 1} x_1 + a_{2 2} x_2 + \cdots + a_{2 n} x_n  \\
    x_n' & = a_{n 1} x_1 + a_{n 2} x_2 + \cdots + a_{n n} x_n.
  \end{align*}
  Suppose that \(A\) is diagonalizable and that the distinct eigenvalues of \(A\) are \(\seq{\lambda}{1,,k}\).
  Prove that a differentiable function \(x : \R \to \R^n\) is a solution to the system iff \(x\) is of the form
  \[
    x(t) = e^{\lambda_1 t} z_1 + e^{\lambda_2 t} z_2 + \cdots + e^{\lambda_k t} z_k,
  \]
  where \(z_i \in E_{\lambda_i}\) for \(i \in \set{1, \dots, k}\).
  Use this result to prove that the set of solutions to the system is an \(n\)-dimensional real vector space.
\end{ex}

\begin{proof}[\pf{ex:5.2.15}]
  Let \(m_i\) be the multiplicity of \(\lambda_i\) for each \(i \in \set{1, \dots, k}\) and let \(\beta\) be the standard ordered basis for \(\R^n\) over \(\R\).
  Since \(A\) is diagonalizable, by \cref{5.1.1} there exists an ordered basis \(\gamma = \set{\seq{q}{1,,n}}\) for \(\R^n\) over \(\R\) such that \([\L_A]_{\gamma}\) is a diagonal matrix.
  Let \(D = [\L_A]_{\gamma}\) and let \(Q = [\IT[\R^n]]_{\gamma}^{\beta}\).
  Note that we can order \(\gamma\) such that \(D_{i i} = \lambda_1\) for all \(i \in \set{1, \dots, m_i}\), \(D_{i i} = \lambda_2\) for all \(i \in \set{m_1 + 1, \dots, m_1 + m_2}\), and in general
  \[
    \forall j \in \set{1, \dots, k}, D_{i i} = \lambda_j \text{ if } i \in \set{\pa{\sum_{p = 1}^{j - 1} m_p} - 1, \dots, \sum_{p = 1}^j m_p}.
  \]
  By \cref{2.5.3} and \cref{ex:2.5.11}(b) we have \(A = Q D Q^{-1}\).
  Observe that
  \begin{align*}
         & x = z \text{ is a solution of } Ax = x'                                                                       \\
    \iff & Q D Q^{-1} z = Az = z'                                                                                        \\
    \iff & D Q^{-1} z = Q^{-1} z'                                                   &  & \text{(by \cref{ex:2.5.11}(b))} \\
    \iff & D Q^{-1} z = (Q^{-1} z)'                                                 &  & \text{(by \cref{ex:5.2.16})}    \\
    \iff & y = Q^{-1} z \text{ is a solution of } D y = y'                                                               \\
    \iff & y = Q^{-1} z \text{ is a solution of } D_{i i} y_i = y_i'                                                     \\
         & \text{ for all } i \in \set{1, \dots, n}                                 &  & \text{(by \cref{1.3.8})}        \\
    \iff & y = Q^{-1} z \text{ is a solution of } y_i(t) = c_i e^{D_{i i} t}                                             \\
         & \text{for all } i \in \set{1, \dots, n} \text{ and for all } c_i \in \R. &  & \text{(by \cref{2.30})}
  \end{align*}
  Thus we see that \(x\) is a solution of \(Ax = x'\) iff \(x = Qy\), where \(y_i(t) = c_i e^{D_{i i} t}\) for all \(i \in \set{1, \dots, n}\) and \(c_i \in \R\).

  By \cref{2.5.1} we see that \(q_i \in \gamma\) is the \(i\)th column of \(Q\) for all \(i \in \set{1, \dots, n}\).
  By \cref{5.1.1} we see that \(\seq{q}{1,,m_1}\) are eigenvectors of \(A\) corresponding to \(\lambda_1\), and in general \(\seq{q}{\pa{\sum_{p = 1}^{j - 1} m_p} + 1,,\sum_{p = 1}^j m_p}\) are eigenvectors of \(A\) corresponding to \(\lambda_j\) for all \(j \in \set{1, \dots, k}\).
  Thus a solution of \(Ax = x'\) is in the form
  \begin{align*}
    x(t) & = Qy(t)                                                                                                                                            \\
         & = y_1(t) q_1 + \cdots + y_n(t) q_n                                           &  & \text{(by \cref{ex:2.3.14}(a))}                                  \\
         & = c_1 e^{D_{1 1} t} q_1 + \cdots + c_n e^{D_{n n} t} q_n                                                                                           \\
         & = e^{\lambda_1 t} (c_1 q_1 + \cdots + c_{m_1} q_{m_1}) + \cdots                                                                                    \\
         & \quad + e^{\lambda_k t} (c_{n - m_k + 1} q_{n - m_k + 1} + \cdots + c_n q_n)                                                                       \\
         & = e^{\lambda_1 t} z_1 + \cdots + e^{\lambda_k t} z_k.                        &  & (z_i \in E_{\lambda_i} \text{ for all } i \in \set{1, \dots, k})
  \end{align*}
  By \cref{2.33} we know that the set \(\set{e^{\lambda_1 t}, \dots, e^{\lambda_k t}}\) is linearly independent.
  If \(\seq{a}{1,,n} \in \F\) such that
  \[
    a_1 e^{\lambda_1 t} q_1 + \cdots + a_{m_1} e^{\lambda_1 t} q_{m_1} + \cdots + a_{n - m_k + 1} e^{\lambda_k t} q_{n - m_k + 1} + \cdots + a_n e^{\lambda_k t} q_n = \zv,
  \]
  then for each \(i \in \set{1, \dots, n}\), we have
  \begin{align*}
    a_1 Q_{i 1} + \cdots + a_{m_1} Q_{i m_1}                   & = 0  \\
    \vdots                                                            \\
    a_{n - m_k + 1} Q_{i (n - m_k + 1)} + \cdots + a_n Q_{i n} & = 0.
  \end{align*}
  In particular, we have
  \begin{align*}
    a_1 q_1 + \cdots + a_{m_1} q_{m_1}                 & = \zv  \\
    \vdots                                                      \\
    a_{n - m_k + 1} q_{n - m_k + 1} + \cdots + a_n q_n & = \zv.
  \end{align*}
  Summing all the equations together we get
  \[
    \seq[+]{a,q}{1,,n} = \zv.
  \]
  But \(\gamma = \set{\seq{q}{1,,n}}\) implies \(\seq[=]{a}{1,,n} = 0\).
  Thus the set
  \[
    \set{e^{\lambda_1 t} q_1, \dots, e^{\lambda_1 t} q_{m_1}, \dots, e^{\lambda_k t} q_{n - m_k + 1}, \dots, e^{\lambda_k t} q_n}
  \]
  is linearly independent.
  We conclude that the set of solution to \(Ax = x'\) is an \(n\)-dimensional real vector space.
\end{proof}

\begin{ex}\label{ex:5.2.16}
  Let \(C \in \ms{m}{n}{\R}\), and let \(Y\) be an \(n \times p\) matrix of differentiable functions.
  Prove \((CY)' = C Y'\), where \((Y')_{i j} = (Y_{i j})'\) for all \(i \in \set{1, \dots, n}\) and \(j \in \set{1, \dots, p}\).
\end{ex}

\begin{proof}[\pf{ex:5.2.16}]
  Let \(k \in \set{1, \dots, n}\) and let \(j \in \set{1, \dots, p}\).
  Then we have
  \begin{align*}
    ((CY)')_{k j} & = ((CY)_{k j})'                        &  & \text{(by \cref{ex:5.2.16})} \\
                  & = \pa{\sum_{i = 1}^n C_{k i} Y_{i j}}' &  & \text{(by \cref{2.3.1})}     \\
                  & = \sum_{i = 1}^n C_{k i} (Y_{i j})'                                      \\
                  & = \sum_{i = 1}^n C_{k i} (Y')_{i j}                                      \\
                  & = (C Y')_{k j}                         &  & \text{(by \cref{2.3.1})}
  \end{align*}
  and thus by \cref{1.2.8} \((CY)' = C Y'\).
\end{proof}

\section{Matrix Limits and Markov Chains}\label{sec:5.3}

\begin{defn}\label{5.3.1}
  Let \(L, \seq{A}{1,2,}\) be \(n \times p\) matrices having complex entries.
  The sequence \(\seq{A}{1,2,}\) is said to \textbf{converge} to the \(n \times p\) matrix \(L\), called the \textbf{limit} of the sequence, if
  \[
    \lim_{m \to \infty} (A_m)_{i j} = L_{i j}
  \]
  for all \(i \in \set{1, \dots, n}\) and \(j \in \set{1, \dots, p}\).
  To designate that \(L\) is the limit of the sequence, we write
  \[
    \lim_{m \to \infty} A_m = L.
  \]
\end{defn}

\begin{thm}\label{5.12}
  Let \(\seq{A}{1,2,}\) be a sequence of \(n \times p\) matrices with complex entries that converges to the matrix \(L\).
  Then for any \(P \in \ms{r}{n}{\C}\) and \(Q \in \ms{p}{s}{\C}\),
  \[
    \lim_{m \to \infty} P A_m = PL \quad \text{and} \quad \lim_{m \to \infty} A_m Q = LQ.
  \]
\end{thm}

\begin{proof}[\pf{5.12}]
  For any \(i \in \set{1, \dots, r}\) and \(j \in \set{1, \dots, p}\),
  \begin{align*}
    \lim_{m \to \infty} (P A_m)_{i j} & = \lim_{m \to \infty} \sum_{k = 1}^n P_{i k} (A_m)_{k j}      &  & \text{(by \cref{2.3.1})} \\
                                      & = \sum_{k = 1}^n P_{i k} \pa{\lim_{m \to \infty} (A_m)_{k j}}                               \\
                                      & = \sum_{k = 1}^n P_{i k} L_{k j}                              &  & \text{(by \cref{5.3.1})} \\
                                      & = (PL)_{i j}.                                                 &  & \text{(by \cref{2.3.1})}
  \end{align*}
  Hence \(\lim_{m \to \infty} P A_m = PL\).
  The proof that \(\lim_{m \to \infty} A_m Q = LQ\) is similar.
\end{proof}

\begin{cor}\label{5.3.2}
  Let \(A \in \ms{n}{n}{\C}\) be such that \(\lim_{m \to \infty} A^m = L\).
  Then for any invertible matrix \(Q \in \ms{n}{n}{\C}\),
  \[
    \lim_{m \to \infty} (Q A Q^{-1})^m = Q L Q^{-1}.
  \]
\end{cor}

\begin{proof}[\pf{5.3.2}]
  Since
  \[
    (Q A Q^{-1})^m = (Q A Q^{-1}) (Q A Q^{-1}) \cdots (Q A Q^{-1}) = Q A^m Q^{-1},
  \]
  we have
  \[
    \lim_{m \to \infty} (Q A Q^{-1})^m = \lim_{m \to \infty} Q A^m Q^{-1} = Q \pa{\lim_{m \to \infty} A^m} Q^{-1} = Q L Q^{-1}
  \]
  by applying \cref{5.12} twice.
\end{proof}

\begin{defn}\label{5.3.3}
  In the discussion that follows, we frequently encounter the set
  \[
    S = \set{\lambda \in \C : \abs{\lambda} < 1 \text{ or } \lambda = 1}.
  \]
  Geometrically, this set consists of the complex number \(1\) and the interior of the unit disk (the disk of radius \(1\) centered at the origin).
  This set is of interest because if \(\lambda\) is a complex number, then \(\lim_{m \to \infty} \lambda^m\) exists iff \(\lambda \in S\).
  This fact, which is obviously true if \(\lambda\) is real, can be shown to be true for complex numbers also.
\end{defn}

\begin{thm}\label{5.13}
  Let \(A\) be a square matrix with complex entries.
  Then \(\lim_{m \to \infty} A^m\) exists iff both of the following conditions hold.
  \begin{enumerate}
    \item Every eigenvalue of \(A\) is contained in \(S\).
    \item If \(1\) is an eigenvalue of \(A\), then the dimension of the eigenspace corresponding to \(1\) equals the multiplicity of \(1\) as an eigenvalue of \(A\).
  \end{enumerate}
\end{thm}

\begin{proof}[\pf{5.13}]
  The necessity of condition (a) is easily justified.
  For suppose that \(\lambda\) is an eigenvalue of \(A\) such that \(\lambda \notin S\).
  Let \(v\) be an eigenvector of \(A\) corresponding to \(\lambda\).
  Regarding \(v\) as an \(n \times 1\) matrix, we see that
  \[
    \lim_{m \to \infty} (A^m v) = \pa{\lim_{m \to \infty} A^m} v = Lv
  \]
  by \cref{5.12}, where \(L = \lim_{m \to \infty} A^m\).
  But \(\lim_{m \to \infty} (A^m v) = \lim_{m \to \infty} (\lambda^m v)\) diverges because \(\lim_{m \to \infty} \lambda^m\) does not exist.
  Hence if \(\lim_{m \to \infty} A^m\) exists, then condition (a) of \cref{5.13} must hold.

  We see in \cref{ch:7} that if \(A\) is a matrix for which condition (b) fails, then \(A\) is similar to a matrix whose upper left \(2 \times 2\) submatrix is precisely this matrix \(B = \begin{pmatrix}
    1 & 1 \\
    0 & 1
  \end{pmatrix}\).
\end{proof}

\begin{thm}\label{5.14}
  Let \(A \in \ms{n}{n}{\C}\) satisfy the following two conditions.
  \begin{enumerate}
    \item Every eigenvalue of \(A\) is contained in \(S\).
    \item \(A\) is diagonalizable.
  \end{enumerate}
  Then \(\lim_{m \to \infty} A^m\) exists.
\end{thm}

\begin{proof}[\pf{5.14}]
  Since \(A\) is diagonalizable, there exists an invertible matrix \(Q\) such that \(Q^{-1} A Q = D\) is a diagonal matrix.
  Suppose that
  \[
    D = \begin{pmatrix}
      \lambda_1 & 0         & \cdots & 0         \\
      0         & \lambda_2 & \cdots & 0         \\
      \vdots    & \vdots    &        & \vdots    \\
      0         & 0         & \cdots & \lambda_n
    \end{pmatrix}.
  \]
  Because \(\seq{\lambda}{1,,n}\) are the eigenvalues of \(A\), condition (i) requires that for each \(i \in \set{1, \dots, n}\), either \(\lambda_i = 1\) or \(\abs{\lambda_i} < 1\).
  Thus
  \[
    \lim_{m \to \infty} \lambda_i^m = \begin{dcases}
      1 & \text{if } \lambda_i = 1 \\
      0 & \text{otherwise}
    \end{dcases}.
  \]
  But since
  \[
    D^m = \begin{pmatrix}
      \lambda_1^m & 0           & \cdots & 0           \\
      0           & \lambda_2^m & \cdots & 0           \\
      \vdots      & \vdots      &        & \vdots      \\
      0           & 0           & \cdots & \lambda_n^m
    \end{pmatrix},
  \]
  the sequence \(D, D^2, \cdots\) converges to a limit \(L\).
  Hence
  \[
    \lim_{m \to \infty} A^m = \lim_{m \to \infty} (Q D Q^{-1})^m = Q L Q^{-1}
  \]
  by \cref{5.3.2}.
\end{proof}

\begin{defn}\label{5.3.4}
  Any square matrix having these two properties (nonnegative entries and columns that sum to \(1\)) is called a \textbf{transition matrix} or a \textbf{stochastic matrix}.
  For an arbitrary \(n \times n\) transition matrix \(M\), the rows and columns correspond to \(n\) \textbf{states}, and the entry \(M_{i j}\) represents the probability of moving from state \(j\) to state \(i\) in one \textbf{stage}.
  In general, for any transition matrix \(M\), the entry \((M^m)_{i j}\) represents the probability of moving from state \(j\) to state \(i\) in \(m\) stages.

  A column vector \(P\) contains nonnegative entries that sum to \(1\) is called a \textbf{probability vector}.
  In this terminology, each column of a transition matrix is a probability vector.
  It is often convenient to regard the entries of a transition matrix or a probability vector as proportions or percentages instead of probabilities.
\end{defn}

\begin{thm}\label{5.15}
  Let \(M \in \ms{n}{n}{\R}\) having real nonnegative entries, let \(v \in \R^n\) having nonnegative coordinates, and let \(u \in \R^n\) be the column vector in which each coordinate equals \(1\).
  Then
  \begin{enumerate}
    \item \(M\) is a transition matrix iff \(\tp{M} u = u\);
    \item \(v\) is a probability vector iff \(\tp{u} v = (1)\).
  \end{enumerate}
\end{thm}

\begin{proof}[\pf{5.15}]
  We have
  \begin{align*}
         & M \text{ is a transition matrix}                                                                                                 \\
    \iff & \forall j \in \set{1, \dots, n}, \sum_{i = 1}^n M_{i j} = 1              &  & \text{(by \cref{5.3.4})}                           \\
    \iff & \forall j \in \set{1, \dots, n}, \sum_{i = 1}^n (\tp{M})_{j i} = 1       &  & \text{(by \cref{1.3.3})}                           \\
    \iff & \forall j \in \set{1, \dots, n}, \sum_{i = 1}^n (\tp{M})_{j i} u_i = u_i &  & (u_i = 1 \text{ for all } i \in \set{1, \dots, n}) \\
    \iff & \tp{M} u = u                                                             &  & \text{(by \cref{2.3.1})}
  \end{align*}
  and
  \begin{align*}
         & v \text{ is a probability vector}                                                         \\
    \iff & \sum_{i = 1}^n v_i = 1            &  & \text{(by \cref{5.3.4})}                           \\
    \iff & \sum_{i = 1}^n (\tp{v})_i = 1     &  & \text{(by \cref{1.3.3})}                           \\
    \iff & \sum_{i = 1}^n (\tp{v})_i u_i = 1 &  & (u_i = 1 \text{ for all } i \in \set{1, \dots, n}) \\
    \iff & \tp{v} u = (1).                   &  & \text{(by \cref{2.3.1})}
  \end{align*}
\end{proof}

\begin{cor}\label{5.3.5}
  \begin{enumerate}
    \item The product of two \(n \times n\) transition matrices is an \(n \times n\) transition matrix.
          In particular, any power of a transition matrix is a transition matrix.
    \item The product of a transition matrix and a probability vector is a probability vector.
  \end{enumerate}
\end{cor}

\begin{proof}[\pf{5.3.5}]
  Let \(A, B \in \ms{n}{n}{\R}\) be transition matrices, let \(v \in \R^n\) be a probability vector and let \(u \in \R^n\) such that \(u_i = 1\) for all \(i \in \set{1, \dots, n}\).
  Since
  \begin{align*}
    \tp{(AB)} u & = (\tp{B} \tp{A}) u &  & \text{(by \cref{2.3.2})}   \\
                & = \tp{B} (\tp{A} u) &  & \text{(by \cref{2.16})}    \\
                & = \tp{B} u          &  & \text{(by \cref{5.15}(a))} \\
                & = u,                &  & \text{(by \cref{5.15}(a))}
  \end{align*}
  by \cref{5.15} we see that \(AB\) is a transition matrix.
  Since
  \begin{align*}
    \tp{(Av)} u & = (\tp{v} \tp{A}) u &  & \text{(by \cref{2.3.2})}   \\
                & = \tp{v} (\tp{A} u) &  & \text{(by \cref{2.16})}    \\
                & = \tp{v} u          &  & \text{(by \cref{5.15}(a))} \\
                & = 1,                &  & \text{(by \cref{5.15}(b))}
  \end{align*}
  by \cref{5.15} we see that \(Av\) is a probability vector.
\end{proof}

\begin{defn}\label{5.3.6}
  The city--suburb problem is an example of a process in which elements of a set are each classified as being in one of several fixed states that can switch over time.
  In general, such a process is called a \textbf{stochastic process}.
  The switching to a particular state is described by a probability, and in general this probability depends on such factors as the state in question, the time in question, some or all of the previous states in which the object has been (including the current state), and the states that other objects are in or have been in.

  If, however, the probability that an object in one state changes to a different state in a fixed interval of time depends only on the two states (and not on the time, earlier states, or other factors), then the stochastic process is called a \textbf{Markov process}.
  If, in addition, the number of possible states is finite, then the Markov process is called a \textbf{Markov chain}.
  Of course, a Markov process is usually only an idealization of reality because the probabilities involved are almost never constant over time.

  The vector that describes the initial probability of being in each state is called the \textbf{initial probability vector} for the Markov chain.
  We saw that \(\lim_{m \to \infty} A^m P\), where \(A\) is the transition matrix and \(P\) is the initial probability vector of the Markov chain, gives the eventual proportions in each state.
  In general, however, the limit of powers of a transition matrix need not exist.
  In fact, it can be shown (\cref{ex:7.2.20}) that the only transition matrices \(A\) such that \(\lim_{m \to \infty} A^m\) does not exist are precisely those matrices for which condition (a) of \cref{5.13} fails to hold.
\end{defn}

\begin{note}
  Even if the limit of powers of the transition matrix exists, the computation of the limit may be quite difficult.
  Fortunately, there is a large and important class of transition matrices for which this limit exists and is easily computed
  --- this is the class of regular transition matrices.
\end{note}

\begin{defn}\label{5.3.7}
  A transition matrix is called \textbf{regular} if some power of the matrix contains only positive entries.
\end{defn}

\begin{defn}\label{5.3.8}
  Let \(A \in \ms{n}{n}{\C}\).
  For \(i, j \in \set{1, \dots, n}\), define \(\rho_i(A)\) to be the sum of the absolute values of the entries of row \(i\) of \(A\), and define \(\nu_j(A)\) to be equal to the sum of the absolute values of the entries of column \(j\) of \(A\).
  Thus
  \[
    \rho_i(A) = \sum_{j = 1}^n \abs{A_{i j}} \quad \text{for } i \in \set{1, \dots, n}
  \]
  and
  \[
    \nu_j(A) = \sum_{i = 1}^n \abs{A_{i j}} \quad \text{for } j \in \set{1, \dots, n}.
  \]
  The \textbf{row sum} of \(A\), denoted \(\rho(A)\), and the \textbf{column sum} of \(A\), denoted \(\nu(A)\), are defined as
  \[
    \rho(A) = \max\set{\rho_i(A) : i \in \set{1, \dots, n}} \quad \text{and} \quad \nu(A) = \max\set{\nu_j(A) : j \in \set{1, \dots, n}}.
  \]
\end{defn}

\begin{defn}\label{5.3.9}
  For \(A \in \ms{n}{n}{\C}\), we define the \(i\)th \textbf{Gerschgorin disk} \(C_i\) to be the disk in the complex plane with center \(A_{i i}\) and radius \(r_i = \rho_i(A) - \abs{A_{i i}}\);
  that is,
  \[
    C_i = \set{z \in \C : \abs{z - A_{i i}} \leq r_i}
  \]
\end{defn}

\begin{thm}[Gerschgorin's Disk Theorem]\label{5.16}
  Let \(A \in \ms{n}{n}{\C}\).
  Then every eigenvalue of \(A\) is contained in a Gerschgorin disk.
\end{thm}

\begin{proof}[\pf{5.16}]
  Let \(\lambda\) be an eigenvalue of \(A\) with the corresponding eigenvector
  \[
    v = \begin{pmatrix}
      v_1    \\
      \vdots \\
      v_n
    \end{pmatrix}.
  \]
  Then \(v\) satisfies the matrix equation \(Av = \lambda v\), which can be written
  \[
    \sum_{j = 1}^n A_{i j} v_j = \lambda v_i \quad \text{for } i \in \set{1, \dots, n}.
  \]
  Suppose that \(v_k\) is the coordinate of \(v\) having the largest absolute value;
  note that \(v_k \neq 0\) because \(v\) is an eigenvector of \(A\).

  We show that \(\lambda\) lies in \(C_k\), that is, \(\abs{\lambda - A_{k k}} \leq r_k\).
  For \(i = k\), it follows from above equation that
  \begin{align*}
    \abs{\lambda v_k - A_{k k} v_k} & = \abs{\sum_{j = 1}^n A_{k j} v_j - A_{k k} v_k}                               \\
                                    & = \abs{\sum_{\substack{j = 1                                                   \\ j \neq k}}^n A_{k j} v_j}              \\
                                    & \leq \sum_{\substack{j = 1                                                     \\ j \neq k}}^n \abs{A_{k j}} \abs{v_j}    && \text{(by \cref{d.3}(a)(c))} \\
                                    & \leq \sum_{\substack{j = 1                                                     \\ j \neq k}}^n \abs{A_{k j}} \abs{v_k}     \\
                                    & = \abs{v_k} \sum_{\substack{j = 1                                              \\ j \neq k}}^n \abs{A_{k j}}        \\
                                    & = \abs{v_k} r_k.                                 &  & \text{(by \cref{5.3.9})}
  \end{align*}
  Thus
  \[
    \abs{v_k} \abs{\lambda - A_{k k}} \leq \abs{v_k} r_k;
  \]
  so
  \[
    \abs{\lambda - A_{k k}} \leq r_k
  \]
  because \(\abs{v_k} > 0\).
\end{proof}

\begin{cor}\label{5.3.10}
  Let \(\lambda\) be any eigenvalue of \(A \in \ms{n}{n}{\C}\).
  Then \(\abs{\lambda} \leq \rho(A)\).
\end{cor}

\begin{proof}[\pf{5.3.10}]
  By Gerschgorin's disk theorem (\cref{5.16}), \(\abs{\lambda - A_{k k}} \leq r_k\) for some \(k\).
  Hence
  \begin{align*}
    \abs{\lambda} & = \abs{(\lambda - A_{k k}) + A_{k k}}                                       \\
                  & \leq \abs{\lambda - A_{k k}} + \abs{A_{k k}} &  & \text{(by \cref{d.3}(c))} \\
                  & \leq r_k + \abs{A_{k k}}                     &  & \text{(by \cref{5.16})}   \\
                  & = \rho_k(A)                                  &  & \text{(by \cref{5.3.9})}  \\
                  & \leq \rho(A).                                &  & \text{(by \cref{5.3.8})}
  \end{align*}
\end{proof}

\begin{cor}\label{5.3.11}
  Let \(\lambda\) be any eigenvalue of \(A \in \ms{n}{n}{\C}\).
  Then
  \[
    \abs{\lambda} \leq \min\set{\rho(A), \nu(A)}.
  \]
\end{cor}

\begin{proof}[\pf{5.3.11}]
  Since \(\abs{\lambda} \leq \rho(A)\) by \cref{5.3.10}, it suffices to show that \(\abs{\lambda} \leq \nu(A)\).
  By \cref{ex:5.1.14}, \(\lambda\) is an eigenvalue of \(\tp{A}\), and so \(\abs{\lambda} \leq \rho(\tp{A})\) by \cref{5.3.10}.
  But the rows of \(\tp{A}\) are the columns of \(A\);
  consequently \(\rho(\tp{A}) = \nu(A)\).
  Therefore \(\abs{\lambda} \leq \nu(A)\).
\end{proof}

\begin{cor}\label{5.3.12}
  If \(\lambda\) is an eigenvalue of a transition matrix, then \(\abs{\lambda} \leq 1\).
\end{cor}

\begin{proof}[\pf{5.3.12}]
  This is the immediate consequence of \cref{5.3.4} and \cref{5.3.11}.
\end{proof}

\begin{thm}\label{5.17}
  Every transition matrix has \(1\) as an eigenvalue.
\end{thm}

\begin{proof}[\pf{5.17}]
  Let \(A \in \ms{n}{n}{\R}\) be a transition matrix, and let \(u \in \R^n\) be the column vector in which each coordinate is \(1\).
  Then \(\tp{A} u = u\) by \cref{5.15}, and hence \(u\) is an eigenvector of \(\tp{A}\) corresponding to the eigenvalue \(1\).
  But since \(A\) and \(\tp{A}\) have the same eigenvalues (\cref{ex:5.1.14}), it follows that \(1\) is also an eigenvalue of \(A\).
\end{proof}

\begin{note}
  Suppose that \(A\) is a transition matrix for which some eigenvector corresponding to the eigenvalue \(1\) has only nonnegative coordinates.
  Then some multiple of this vector is a probability vector \(P\) as well as an eigenvector of \(A\) corresponding to eigenvalue \(1\).
  It is interesting to observe that if \(P\) is the initial probability vector of a Markov chain having \(A\) as its transition matrix, then the Markov chain is completely static.
  For in this situation, \(A^m P = P\) for every positive integer \(m\);
  hence the probability of being in each state never changes.
\end{note}

\begin{thm}\label{5.18}
  Let \(A \in \ms{n}{n}{\C}\) be a matrix in which each entry is positive, and let \(\lambda\) be an eigenvalue of \(A\) such that \(\abs{\lambda} = \rho(A)\).
  Then \(\lambda = \rho(A)\) and \(\set{u}\) is a basis for \(E_{\lambda}\) over \(\C\), where \(u \in \C^n\) is the column vector in which each coordinate equals \(1\).
\end{thm}

\begin{proof}[\pf{5.18}]
  Let \(v\) be an eigenvector of \(A\) corresponding to \(\lambda\), with coordinates \(\seq{v}{1,,n}\).
  Suppose that \(v_k\) is the coordinate of \(v\) having the largest absolute value, and let \(b = \abs{v_k}\).
  Then
  \begin{align*}
    \abs{\lambda} b & = \abs{\lambda} \abs{v_k}                                               \\
                    & = \abs{\lambda v_k}                      &  & \text{(by \cref{d.3}(a))} \\
                    & = \abs{\sum_{j = 1}^n A_{k j} v_j}       &  & \text{(by \cref{5.1.2})}  \\
                    & \leq \sum_{j = 1}^n \abs{A_{k j} v_j}    &  & \text{(by \cref{d.3}(c))} \\
                    & = \sum_{j = 1}^n \abs{A_{k j}} \abs{v_j} &  & \text{(by \cref{d.3}(a))} \\
                    & \leq \sum_{j = 1}^n \abs{A_{k j}} b                                     \\
                    & = \rho_k(A) b                            &  & \text{(by \cref{5.3.8})}  \\
                    & \leq \rho(A) b.                          &  & \text{(by \cref{5.3.8})}
  \end{align*}
  Since \(\abs{\lambda} = \rho(A)\), the three inequalities above are actually equalities;
  that is,
  \begin{enumerate}
    \item \(\abs{\sum_{j = 1}^n A_{k j} v_j} = \sum_{j = 1}^n \abs{A_{k j} v_j}\),
    \item \(\sum_{j = 1}^n \abs{A_{k j}} \abs{v_j} = \sum_{j = 1}^n \abs{A_{k j}} b\), and
    \item \(\rho_k(A) = \rho(A)\).
  \end{enumerate}

  We see in \cref{ex:6.1.15}(b) that (a) holds iff all the terms \(A_{k j} v_j\) (\(j \in \set{1, \dots, n}\)) are nonnegative multiples of some nonzero complex number \(z\).
  Without loss of generality, we assume that \(\abs{z} = 1\).
  Thus there exist nonnegative real numbers \(\seq{c}{1,,n}\) such that
  \[
    A_{k j} v_j = c_j z
  \]

  By (b) and the assumption that \(A_{k j} \neq 0\) for all \(k, j \in \set{1, \dots, n}\), we have
  \[
    \abs{v_j} = b \quad \text{for } j \in \set{1, \dots, n}.
  \]
  Combining equations above, we obtain
  \[
    b = \abs{v_j} = \abs{\frac{c_j}{A_{k j}} z} = \frac{c_j}{A_{k j}} \quad \text{for } j \in \set{1, \dots, n},
  \]
  and therefore we have \(v_j = bz\) for all \(j\).
  So
  \[
    v = \begin{pmatrix}
      v_1    \\
      \vdots \\
      v_n
    \end{pmatrix} = \begin{pmatrix}
      bz     \\
      \vdots \\
      bz
    \end{pmatrix} = bzu,
  \]
  and hence \(\set{u}\) is a basis for \(E_{\lambda}\) over \(\C\).

  Finally, observe that all of the entries of \(Au\) are positive because the same is true for the entries of both \(A\) and \(u\).
  But \(Au = \lambda u\), and hence \(\lambda > 0\).
  Therefore, \(\lambda = \abs{\lambda} = \rho(A)\).
\end{proof}

\begin{cor}\label{5.3.13}
  Let \(A \in \ms{n}{n}{\C}\) be a matrix in which each entry is positive, and let \(\lambda\) be an eigenvalue of \(A\) such that \(\abs{\lambda} = \nu(A)\).
  Then \(\lambda = \nu(A)\), and the dimension of \(E_{\lambda} = 1\).
\end{cor}

\begin{proof}[\pf{5.3.13}]
  We have
  \begin{align*}
             & \abs{\lambda} = \nu(A) = \rho(\tp{A})                                                       &  & \text{(by \cref{5.3.8})} \\
    \implies & \begin{dcases}
                 \lambda = \rho(\tp{A}) = \nu(A) \\
                 \set{u} \text{ is a basis for } \ns{\tp{A} - \lambda I_n} \text{ over } \C
               \end{dcases} &  & \text{(by \cref{5.18})}                                \\
    \implies & \begin{dcases}
                 \lambda = \nu(A) \\
                 \dim(\ns{\tp{A} - \lambda I_n}) = 1
               \end{dcases}                          &  & \text{(by \cref{1.6.8})}                                                       \\
    \implies & \begin{dcases}
                 \lambda = \nu(A) \\
                 \dim(\ns{A - \lambda I_n}) = 1
               \end{dcases}.                                                              &  & \text{(by \cref{ex:5.2.13}(b))}
  \end{align*}
\end{proof}

\begin{cor}\label{5.3.14}
  Let \(A \in \ms{n}{n}{\C}\) be a transition matrix in which each entry is positive, and let \(\lambda\) be any eigenvalue of \(A\) other than \(1\).
  Then \(\abs{\lambda} < 1\).
  Moreover, the eigenspace corresponding to the eigenvalue \(1\) has dimension \(1\).
\end{cor}

\begin{proof}[\pf{5.3.14}]
  By \cref{5.3.12} we see that \(\abs{\lambda} \leq 1\).
  We claim that \(\abs{\lambda} < 1\).
  Suppose for sake of contradiction that \(\abs{\lambda} = 1\).
  Then by \cref{5.3.4,5.3.13} we know that \(\lambda = 1\).
  But by hypothesis we know that \(\lambda \neq 1\), a contradiction.
  Thus we must have \(\abs{\lambda} < 1\).
  The other statement follows from \cref{5.18}.
\end{proof}

\begin{thm}\label{5.19}
  Let \(A\) be a regular transition matrix, and let \(\lambda\) be an eigenvalue of \(A\).
  Then
  \begin{enumerate}
    \item \(\abs{\lambda} \leq 1\).
    \item If \(\abs{\lambda} = 1\), then \(\lambda = 1\), and \(\dim(E_{\lambda}) = 1\).
  \end{enumerate}
\end{thm}

\begin{proof}[\pf{5.19}]
  Statement (a) was proved as \cref{5.3.12}.

  Since \(A\) is regular, by \cref{5.3.7} there exists a positive integer \(s\) such that \(A^s\) has only positive entries.
  Because \(A\) is a transition matrix and the entries of \(A^s\) are positive, the entries of \(A^{s + 1} = A^s (A)\) are positive.
  Suppose that \(\abs{\lambda} = 1\).
  Then \(\lambda^s\) and \(\lambda^{s + 1}\) are eigenvalues of \(A^s\) and \(A^{s + 1}\), respectively, having absolute value \(1\).
  So by \cref{5.3.14}, \(\lambda^s = \lambda^{s + 1} = 1\).
  Thus \(\lambda = 1\).
  Let \(E_{\lambda}\) and \(E_{\lambda}'\) denote the eigenspaces of \(A\) and \(A^s\), respectively, corresponding to \(\lambda = 1\).
  Then \(E_{\lambda} \subseteq E_{\lambda}'\) and, by \cref{5.3.14}, \(\dim(E_{\lambda}') = 1\).
  Hence \(E_{\lambda} = E_{\lambda}'\), and \(\dim(E_{\lambda}) = 1\).
\end{proof}

\begin{cor}\label{5.3.15}
  Let \(A\) be a regular transition matrix that is diagonalizable.
  Then \(\lim_{m \to \infty} A^m\) exists.
\end{cor}

\begin{proof}[\pf{5.3.15}]
  By \cref{5.19} and \cref{5.14}(b) we see that this is true.
\end{proof}

\begin{note}
  In fact, it can be shown that if \(A\) is a regular transition matrix, then the multiplicity of \(1\) as an eigenvalue of \(A\) is \(1\).
  Thus, by \cref{5.7}, condition (b) of \cref{5.13} is satisfied.
  So if \(A\) is a regular transition matrix, \(\lim_{m \to \infty} A^m\) exists regardless of whether \(A\) is or is not diagonalizable.
  As with \cref{5.13}, however, the fact that the multiplicity of \(1\) as an eigenvalue of \(A\) is \(1\) cannot be proved at this time.
  Nevertheless, we state this result here (leaving the proof until \cref{ex:7.2.21}) and deduce further facts about \(\lim_{m \to \infty} A^m\) when \(A\) is a regular transition matrix.
\end{note}

\begin{thm}\label{5.20}
  Let \(A \in \ms{n}{n}{\C}\) be a regular transition matrix.
  Then
  \begin{enumerate}
    \item The multiplicity of \(1\) as an eigenvalue of \(A\) is \(1\).
    \item \(\lim_{m \to \infty} A^m\) exists.
    \item \(L = \lim_{m \to \infty} A^m\) is a transition matrix.
    \item \(AL = LA = L\).
    \item The columns of \(L\) are identical.
          In fact, each column of \(L\) is equal to the unique probability vector \(v\) that is also an eigenvector of \(A\) corresponding to the eigenvalue \(1\).
    \item For any probability vector \(w\), \(\lim_{m \to \infty} (A^m w) = v\).
  \end{enumerate}
\end{thm}

\begin{proof}[\pf{5.20}(a)]
  See \cref{ex:7.2.21}.
\end{proof}

\begin{proof}[\pf{5.20}(b)]
  This follows from \cref{5.20}(a), \cref{5.19,5.13}.
\end{proof}

\begin{proof}[\pf{5.20}(c)]
  By \cref{5.15}(a), we must show that \(\tp{u} L = \tp{u}\).
  Now \(A^m\) is a transition matrix by \cref{5.3.5}(a), so
  \[
    \tp{u} L = \tp{u} \lim_{m \to \infty} A^m = \lim_{m \to \infty} \tp{u} A^m = \lim_{m \to \infty} \tp{u} = \tp{u},
  \]
  and it follows that \(L\) is a transition matrix.
\end{proof}

\begin{proof}[\pf{5.20}(d)]
  By \cref{5.12},
  \[
    AL = A \lim_{m \to \infty} A^m = \lim_{m \to \infty} AA^m = \lim_{m \to \infty} A^{m + 1} = L.
  \]
  Similarly, \(LA = L\).
\end{proof}

\begin{proof}[\pf{5.20}(e)]
  Since \(AL = L\) by \cref{5.20}(d), each column of \(L\) is an eigenvector of \(A\) corresponding to the eigenvalue \(1\).
  Moreover, by \cref{5.20}(c), each column of \(L\) is a probability vector.
  Thus, by \cref{5.20}(a), each column of \(L\) is equal to the unique probability vector \(v\) corresponding to the eigenvalue \(1\) of \(A\).
\end{proof}

\begin{proof}[\pf{5.20}(f)]
  Let \(w\) be any probability vector, and set \(y = \lim_{m \to \infty} A^m w = Lw\).
  Then \(y\) is a probability vector by \cref{5.3.5}(b), and also \(Ay = ALw = Lw = y\) by \cref{5.20}(d).
  Hence \(y\) is also an eigenvector corresponding to the eigenvalue \(1\) of \(A\).
  So \(y = v\) by \cref{5.20}(e).
\end{proof}

\begin{defn}\label{5.3.16}
  The vector \(v\) in \cref{5.20}(e) is called the \textbf{fixed probability vector} or \textbf{stationary vector} of the regular transition matrix \(A\).
\end{defn}

\begin{note}
  \cref{5.20} can be used to deduce information about the eventual distribution in each state of a Markov chain having a regular transition matrix.
\end{note}

\begin{defn}\label{5.3.17}
  There is another interesting class of transition matrices that can be represented in the form
  \[
    \begin{pmatrix}
      I   & B \\
      \zm & C
    \end{pmatrix},
  \]
  where \(I\) is an identity matrix and \(\zm\) is a zero matrix.
  (Such transition matrices are not regular since the lower left block remains \(\zm\) in any power of the matrix.)
  The states corresponding to the identity submatrix are called \textbf{absorbing states} because such a state is never left once it is entered.
  A Markov chain is called an \textbf{absorbing Markov chain} if it is possible to go from an arbitrary state into an absorbing state in a finite number of stages.
\end{defn}

\exercisesection

\setcounter{ex}{2}
\begin{ex}\label{ex:5.3.3}
  Prove that if \(\seq{A}{1,2,}\) is a sequence of \(n \times p\) matrices with complex entries such that \(\lim_{m \to \infty} A_m = L\), then \(\lim_{m \to \infty} \tp{(A_m)} = \tp{L}\).
\end{ex}

\begin{proof}[\pf{ex:5.3.3}]
  We have
  \begin{align*}
         & \lim_{m \to \infty} A_m = L                                                                                                                                 \\
    \iff & \forall i \in \set{1, \dots, n} \text{ and } j \in \set{1, \dots, p}, \lim_{m \to \infty} (A_m)_{i j} = L_{i j}               &  & \text{(by \cref{5.3.1})} \\
    \iff & \forall i \in \set{1, \dots, n} \text{ and } j \in \set{1, \dots, p}, \lim_{m \to \infty} (\tp{(A_m)})_{j i} = (\tp{L})_{j i} &  & \text{(by \cref{1.3.3})} \\
    \iff & \lim_{m \to \infty} \tp{(A_m)} = \tp{L}.                                                                                      &  & \text{(by \cref{5.3.1})}
  \end{align*}
\end{proof}

\begin{ex}\label{ex:5.3.4}
  Prove that if \(A \in \ms{n}{n}{\C}\) is diagonalizable and \(L = \lim_{m \to \infty} A^m\) exists, then either \(L = I_n\) or \(\rk{L} < n\).
\end{ex}

\begin{proof}[\pf{ex:5.3.4}]
  Since \(A\) is diagonalizable, there exists some \(Q \in \ms{n}{n}{\C}\) such that \(D = Q^{-1} A Q\) is a diagonal matrix.
  Then we have
  \begin{align*}
    L & = \lim_{m \to \infty} A^m                                                                                   \\
      & = \lim_{m \to \infty} (Q D Q^{-1})^m                                                                        \\
      & = Q \pa{\lim_{m \to \infty} D^m} Q^{-1}                                       &  & \text{(by \cref{5.3.2})} \\
      & = Q \begin{pmatrix}
              \lim_{m \to \infty} (D_{1 1})^m & \cdots & 0                               \\
              \vdots                          & \ddots & \vdots                          \\
              0                               & \cdots & \lim_{m \to \infty} (D_{n n})^m
            \end{pmatrix} Q^{-1}. &  & \text{(by \cref{5.3.1})}                              \\
  \end{align*}
  Since \(D_{i i} \in \C\) for all \(i \in \set{1, \dots, n}\), by \cref{5.3.3} we see that \(\lim_{m \to \infty} D^m\) exists iff \(\abs{D_{i i}} < 1\) or \(D_{i i} = 1\).
  Now we split into two cases:
  \begin{itemize}
    \item If \(D_{i i} = 1\) for all \(i \in \set{1, \dots, n}\), then we have
          \begin{align*}
                     & D = I_n                                                                               \\
            \implies & \lim_{m \to \infty} D^m = \lim_{m \to \infty} I_n = I_n &  & \text{(by \cref{5.3.1})} \\
            \implies & L = Q I_n Q^{-1} = I_n.
          \end{align*}
    \item If \(\abs{D_{j j}} < 1\) for some \(j \in \set{1, \dots, n}\), then we have
          \begin{align*}
                     & \lim_{m \to \infty} D_{j j}^m = 0                                   \\
            \implies & \rk{\lim_{m \to \infty} D} < n    &  & \text{(by \cref{3.5})}       \\
            \implies & \rk{L} < n.                       &  & \text{(by \cref{3.7}(c)(d))}
          \end{align*}
  \end{itemize}
  From all cases above we conclude that \cref{ex:5.3.4} is true.
\end{proof}

\begin{ex}\label{ex:5.3.5}
  Find \(A, B \in \ms{2}{2}{\R}\) such that
  \begin{align*}
     & \lim_{m \to \infty} A^m \in \ms{2}{2}{\R}    \\
     & \lim_{m \to \infty} B^m \in \ms{2}{2}{\R}    \\
     & \lim_{m \to \infty} (AB)^m \in \ms{2}{2}{\R}
  \end{align*}
  but
  \[
    \lim_{m \to \infty} (AB)^m \neq (\lim_{m \to \infty} A^m) (\lim_{m \to \infty} B^m).
  \]
\end{ex}

\begin{proof}[\pf{ex:5.3.5}]
  Let
  \[
    A = \begin{pmatrix}
      \frac{1}{2} & \frac{1}{2} \\
      0           & 0
    \end{pmatrix} \quad \text{and} \quad B = \begin{pmatrix}
      1 & 0 \\
      1 & 0
    \end{pmatrix}.
  \]
  Then we have
  \begin{align*}
    \lim_{m \to \infty} A^m    & = \lim_{m \to \infty} \begin{pmatrix}
                                                         \frac{1}{2^m} & \frac{1}{2^m} \\
                                                         0             & 0
                                                       \end{pmatrix} = \begin{pmatrix}
                                                                         0 & 0 \\
                                                                         0 & 0
                                                                       \end{pmatrix} \\
    \lim_{m \to \infty} B^m    & = \lim_{m \to \infty} \begin{pmatrix}
                                                         1 & 0 \\
                                                         1 & 0
                                                       \end{pmatrix} = \begin{pmatrix}
                                                                         1 & 0 \\
                                                                         1 & 0
                                                                       \end{pmatrix} \\
    \lim_{m \to \infty} (AB)^m & = \lim_{m \to \infty} \begin{pmatrix}
                                                         1 & 0 \\
                                                         0 & 0
                                                       \end{pmatrix} = \begin{pmatrix}
                                                                         1 & 0 \\
                                                                         0 & 0
                                                                       \end{pmatrix}
  \end{align*}
  and thus
  \[
    (\lim_{m \to \infty} A^m) (\lim_{m \to \infty} B^m) = \begin{pmatrix}
      0 & 0 \\
      0 & 0
    \end{pmatrix} \neq \begin{pmatrix}
      1 & 0 \\
      0 & 0
    \end{pmatrix} = \lim_{m \to \infty} (AB)^m.
  \]
\end{proof}

\setcounter{ex}{14}
\begin{ex}\label{ex:5.3.15}
  Prove that if a \(1\)-dimensional subspace \(\W\) of \(\R^n\) over \(\R\) contains a nonzero vector with all nonnegative entries, then \(\W\) contains a unique probability vector.
\end{ex}

\begin{proof}[\pf{ex:5.3.15}]
  Let \(v \in \R^n\) such that \(v \neq \zv\) and all entries in \(v\) are nonzero.
  If we denote the \(i\)th entry of \(v\) as \(v_i\), then we have \(k = \sum_{i = 1}^n v_i > 0\).
  By \cref{5.3.4} we see that \(\frac{1}{k} v\) is a probability vector.
  Since \(\W\) is \(1\)-dimensional, we see that \(\set{v}\) is a basis for \(\W\) over \(\R\) and by \cref{1.8} we see that \(\frac{1}{k} v\) is the unique probability vector.
\end{proof}

\setcounter{ex}{18}
\begin{ex}\label{ex:5.3.19}
  Suppose that \(M, M' \in \ms{n}{n}{\R}\) are transition matrices.
  \begin{enumerate}
    \item Prove that if \(M\) is regular, \(N\) is any \(n \times n\) transition matrix, and \(c\) is a real number such that \(0 < c \leq 1\), then \(cM + (1 - c)N\) is a regular transition matrix.
    \item Suppose that for all \(i, j \in \set{1, \dots, n}\), we have that \(M_{i j}' > 0\) whenever \(M_{i j} > 0\).
          Prove that there exists a transition matrix \(N\) and a real number \(c\) with \(0 < c \leq 1\) such that \(M' = cM + (1 - c)N\).
    \item Deduce that if the nonzero entries of \(M\) and \(M'\) occur in the same positions, then \(M\) is regular iff \(M'\) is regular.
  \end{enumerate}
\end{ex}

\begin{proof}[\pf{ex:5.3.19}(a)]
  First we show that \(cM + (1 - c)N\) is a transition matrix.
  Let \(u \in \R^n\) such that all entries in \(u\) are \(1\).
  Since
  \begin{align*}
    \tp{(cM + (1 - c)N)} u & = (c \tp{M} + (1 - c) \tp{N}) u &  & \text{(by \cref{ex:1.3.3})} \\
                           & = c \tp{M} u + (1 - c) \tp{N} u &  & \text{(by \cref{2.3.5})}    \\
                           & = c u + (1 - c) u               &  & \text{(by \cref{5.15}(a))}  \\
                           & = u
  \end{align*}
  and each entries in \(cM + (1 - c)N\) are nonnegative, by \cref{5.15}(a) we see that \(cM + (1 - c)N\) is a transition matrix.

  Now we show that \(cM + (1 - c)N\) is regular.
  Since \(M\) is regular, by \cref{5.3.7} there exists a \(m \in \Z^+\) such that \((M^m)_{i j} > 0\) for all \(i, j \in \set{1, \dots, n}\).
  Then we have
  \[
    (cM + (1 - c)N)^m = (cM)^m + P
  \]
  where \(P \in \ms{n}{n}{\R}\) and \(P\) has nonnegative entries.
  Since
  \begin{align*}
             & \begin{dcases}
                 c > 0 \\
                 \forall i, j \in \set{1, \dots, n}, (M^m)_{i j} > 0
               \end{dcases}          \\
    \implies & \forall i, j \in \set{1, \dots, n}, c^m (M^m)_{i j} > 0     \\
    \implies & \forall i, j \in \set{1, \dots, n}, ((cM)^m)_{i j} > 0      \\
    \implies & \forall i, j \in \set{1, \dots, n}, ((cM)^m + P)_{i j} > 0,
  \end{align*}
  by \cref{5.3.7} we see that \(cM + (1 - c)N\) is regular.
\end{proof}

\begin{proof}[\pf{ex:5.3.19}(b)]
  Let \(k \in \R^+\) such that \(k M_{i j}' > M_{i j}\) for all \(i, j \in \set{1, \dots, n}\).
  Such \(k\) must exist since \(M_{i j} > 0 \implies M_{i j}' > 0\).
  In particular, we pick \(k\) such that \(k > 1\).
  Then we have
  \begin{align*}
             & \forall i, j \in \set{1, \dots, n}, k M_{i j}' > M_{i j}                                                                   \\
    \implies & \forall i, j \in \set{1, \dots, n}, M_{i j}' > \frac{1}{k} M_{i j}                                                         \\
    \implies & \forall i, j \in \set{1, \dots, n}, M_{i j}' - \frac{1}{k} M_{i j} > 0                                                     \\
    \implies & \forall i, j \in \set{1, \dots, n}, \frac{1}{1 - \frac{1}{k}} (M_{i j}' - \frac{1}{k} M_{i j}) > 0. &  & (\frac{1}{k} < 1)
  \end{align*}
  Now let \(c = \frac{1}{k}\) and let \(N = \frac{1}{1 - c} (M' - cM)\).
  Clearly \(c \in (0, 1]\).
  So we only need to show that \(N\) is a transition matrix.
  Let \(u \in \R^n\) such that all entries of \(u\) are \(1\).
  Then we have
  \begin{align*}
    \tp{N} u & = \tp{\pa{\frac{1}{1 - c} (M' - cM)}} u                                                      \\
             & = \pa{\frac{1}{1 - c} \tp{(M')} - \frac{c}{1 - c} \tp{M}} u &  & \text{(by \cref{ex:1.3.3})} \\
             & = \frac{1}{1 - c} \tp{(M')} u - \frac{c}{1 - c} \tp{M} u    &  & \text{(by \cref{2.3.5})}    \\
             & = \frac{1}{1 - c} u - \frac{c}{1 - c} u                     &  & \text{(by \cref{5.15}(a))}  \\
             & = u
  \end{align*}
  and thus by \cref{5.15}(a) \(N\) is a transition matrix.
\end{proof}

\begin{proof}[\pf{ex:5.3.19}(c)]
  This is the immediate consequence of \cref{ex:5.3.19}(a)(b).
\end{proof}

\section{Invariant Subspaces and the Cayley-Hamilton Theorem}\label{sec:5.4}

\begin{defn}\label{5.4.1}
  Let \(\T\) be a linear operator on a vector space \(\V\) over \(\F\).
  A subspace \(\W\) of \(\V\) over \(\F\) is called a \textbf{\(\T\)-invariant subspace} of \(\V\) over \(\F\) if \(\T(\W) \subseteq \W\), that is, if \(\T(v) \in \W\) for all \(v \in \W\).
\end{defn}

\begin{eg}\label{5.4.2}
  Suppose that \(\T\) is a linear operator on a vector space \(\V\) over \(\F\).
  Then the following subspaces of \(\V\) are \(\T\)-invariant:
  \begin{enumerate}
    \item \(\set{\zv}\)
    \item \(\V\)
    \item \(\rg{\T}\)
    \item \(\ns{\T}\)
    \item \(E_{\lambda}\), for any eigenvalue \(\lambda\) of \(\T\).
  \end{enumerate}
\end{eg}

\begin{proof}[\pf{5.4.2}]
  We have
  \begin{align*}
    \T(\set{\zv}) & = \set{\zv} \subseteq \set{\zv} &  & \text{(by \cref{2.1.2}(a))} \\
    \T(\V)        & = \rg{\T} \subseteq \V          &  & \text{(by \cref{2.5.2})}    \\
    \T(\rg{\T})   & \subseteq \T(\V) = \rg{\T}      &  & \text{(by \cref{2.1.10})}   \\
    \T(\ns{\T})   & = \set{\zv} \subseteq \ns{\T}   &  & \text{(by \cref{2.1.10})}
  \end{align*}
  and
  \begin{align*}
             & \forall v \in E_{\lambda}, \T(v) = \lambda v \in E_{\lambda} &  & \text{(by \cref{5.2.4})} \\
    \implies & \T(E_{\lambda}) \subseteq E_{\lambda}.
  \end{align*}
  Thus by \cref{5.4.1} \(\set{\zv}, \V, \rg{\T}, \ns{\T}, E_{\lambda}\) are \(\T\)-invariant subspace of \(\V\) over \(\F\).
\end{proof}

\begin{defn}\label{5.4.3}
  Let \(\T\) be a linear operator on a vector space \(\V\) over \(\F\), and let \(x\) be a nonzero vector in \(\V\).
  The subspace
  \[
    \W = \spn{\set{x, \T(x), \T^2(x), \dots}}
  \]
  is called the \textbf{\(\T\)-cyclic subspace of \(\V\) generated by \(x\)}.
  It is a simple matter to show that \(\W\) is \(\T\)-invariant.
  In fact, \(\W\) is the ``smallest'' \(\T\)-invariant subspace of \(\V\) containing \(x\).
  That is, any \(\T\)-invariant subspace of \(\V\) containing \(x\) must also contain \(\W\)
  (see \cref{ex:5.4.11}).
\end{defn}

\begin{thm}\label{5.21}
  Let \(\T\) be a linear operator on a finite-dimensional vector space \(\V\) over \(\F\), and let \(\W\) be a \(\T\)-invariant subspace of \(\V\) over \(\F\).
  Then the characteristic polynomial of \(\T_{\W}\) divides the characteristic polynomial of \(\T\).
\end{thm}

\begin{proof}[\pf{5.21}]
  Choose an ordered basis \(\gamma = \set{\seq{v}{1,,k}}\) for \(\W\) over \(\F\), and extend it to an ordered basis \(\beta = \set{\seq{v}{1,,k,k+1,n}}\) for \(\V\) over \(\F\).
  Let \(A = [\T]_{\beta}\) and \(B_1 = [\T_{\W}]_{\gamma}\).
  Then, by \cref{ex:5.4.12}, \(A\) can be written in the form
  \[
    A = \begin{pmatrix}
      B_1 & B_2 \\
      \zm & B_3
    \end{pmatrix}.
  \]
  Let \(f\) be the characteristic polynomial of \(\T\) and \(g\) the characteristic polynomial of \(\T_{\W}\).
  Then
  \[
    f(t) = \det(A - t I_n) = \det\begin{pmatrix}
      B_1 - t I_k & B_2               \\
      \zm         & B_3 - t I_{n - k}
    \end{pmatrix} = g(t) \cdot \det(B_3 - t I_{n - k})
  \]
  by \cref{ex:4.3.21}.
  Thus \(g\) divides \(f\).
\end{proof}

\begin{thm}\label{5.22}
  Let \(\T\) be a linear operator on a finite-dimensional vector space \(\V\) over \(\F\), and let \(\W\) denote the \(\T\)-cyclic subspace of \(\V\) generated by a nonzero vector \(v \in \V\).
  Let \(k = \dim(\W)\).
  Then
  \begin{enumerate}
    \item \(\set{v, \T(v), \T^2(v), \dots, \T^{k - 1}(v)}\) is a basis for \(\W\) over \(\F\).
    \item If \(a_0 v + a_1 \T(v) + \cdots + a_{k - 1} \T^{k - 1}(v) + \T^k(v) = \zv\), then the characteristic polynomial of \(\T_{\W}\) is \(f(t) = (-1)^k (a_0 + a_1 t + \cdots + a_{k - 1} t^{k - 1} + t^k)\).
  \end{enumerate}
\end{thm}

\begin{proof}[\pf{5.22}(a)]
  Since \(v \neq \zv\), the set \(\set{v}\) is linearly independent.
  Let \(j\) be the largest positive integer for which
  \[
    \beta = \set{v, \T(v), \dots, \T^{j - 1}(v)}
  \]
  is linearly independent.
  Such a \(j\) must exist because \(\V\) is finite-dimensional.
  Let \(\vs{Z} = \spn{\beta}\).
  Then \(\beta\) is a basis for \(\vs{Z}\) over \(\F\).
  Furthermore, \(\T^j(v) \in \vs{Z}\) by \cref{1.7}.
  We use this information to show that \(\vs{Z}\) is a \(\T\)-invariant subspace of \(\V\) over \(\F\).
  Let \(w \in \vs{Z}\).
  Since \(w\) is a linear combination of the vectors of \(\beta\), there exist scalars \(\seq{b}{0,,j-1} \in \F\) such that
  \[
    w = b_0 v + b_1 \T(v) + \cdots + b_{j - 1} \T^{j - 1}(v),
  \]
  and hence
  \[
    \T(w) = b_0 \T(v) + b_1 \T^2(v) + \cdots + b_{j - 1} \T^j(v).
  \]
  Thus \(\T(w)\) is a linear combination of vectors in \(\vs{Z}\), and hence belongs to \(\vs{Z}\).
  So \(\vs{Z}\) is \(\T\)-invariant.
  Furthermore, \(v \in \vs{Z}\).
  By \cref{ex:5.4.11}, \(\W\) is the smallest \(\T\)-invariant subspace of \(\V\) that contains \(v\), so that \(\W \subseteq \vs{Z}\).
  Clearly, \(\vs{Z} \subseteq \W\), and so we conclude that \(\vs{Z} = \W\).
  It follows that \(\beta\) is a basis for \(\W\) over \(\F\), and therefore \(\dim(\W) = j\).
  Thus \(j = k\).
  This proves (a).
\end{proof}

\begin{proof}[\pf{5.22}(b)]
  Now view \(\beta\) (from (a)) as an ordered basis for \(\W\).
  Let \(\seq{a}{0,,k-1} \in \F\) such that
  \[
    a_0 v + a_1 \T(v) + \cdots + a_{k - 1} \T^{k - 1}(v) + \T^k(v) = \zv.
  \]
  Observe that
  \[
    [\T_{\W}]_{\beta} = \begin{pmatrix}
      0      & 0      & \cdots & 0      & -a_0       \\
      1      & 0      & \cdots & 0      & -a_1       \\
      \vdots & \vdots &        & \vdots & \vdots     \\
      0      & 0      & \cdots & 1      & -a_{k - 1}
    \end{pmatrix}
  \]
  which has the characteristic polynomial
  \[
    f(t) = (-1)^k (a_0 + a_1 t + \cdots + a_{k - 1} t^{k - 1} + t^k)
  \]
  by \cref{ex:5.4.19}.
  Thus \(f\) is the characteristic polynomial of \(\T_{\W}\), proving (b).
\end{proof}

\begin{thm}[Cayley--Hamilton theorem]\label{5.23}
  Let \(\T\) be a linear operator on a finite-dimensional vector space \(\V\) over \(\F\), and let \(f\) be the characteristic polynomial of \(\T\).
  Then \(f(\T) = \zT\), the zero transformation.
  That is, \(\T\) ``satisfies'' its characteristic equation.
\end{thm}

\begin{proof}[\pf{5.23}]
  We show that \(f(\T)(v) = \zv\) for all \(v \in \V\).
  This is obvious if \(v = \zv\) because \(f(\T)\) is linear;
  so suppose that \(v \neq \zv\).
  Let \(\W\) be the \(\T\)-cyclic subspace generated by \(v\), and suppose that \(\dim(\W) = k\).
  By \cref{5.22}(a), there exist scalars \(\seq{a}{0,,k-1} \in \F\) such that
  \[
    a_0 v + a_1 \T(v) + \cdots + a_{k - 1} \T^{k - 1}(v) + \T^k(v) = \zv.
  \]
  Hence \cref{5.22}(b) implies that
  \[
    g(t) = (-1)^k (a_0 + a_1 t + \cdots + a_{k - 1} t^{k - 1} + t^k)
  \]
  is the characteristic polynomial of \(\T_{\W}\).
  Combining these two equations yields
  \[
    g(\T)(v) = (-1)^k (a_0 \IT[\V] + a_1 \T + \cdots + a_{k - 1} \T^{k - 1} + \T^k)(v) = \zv.
  \]
  By \cref{5.21}, \(g\) divides \(f\);
  hence there exists a polynomial \(q\) such that \(f\) = \(qg\).
  So
  \[
    f(\T)(v) = (q(\T) g(\T))(v) = q(\T)(g(\T)(v)) = q(\T)(\zv) = \zv.
  \]
\end{proof}

\begin{cor}[Cayley--Hamilton Theorem for Matrices]\label{5.4.4}
  Let \(A \in \ms{n}{n}{\F}\), and let \(f\) be the characteristic polynomial of \(A\).
  Then \(f(A) = \zm\), the \(n \times n\) zero matrix.
\end{cor}

\begin{proof}[\pf{5.4.4}]
  Let \(\beta\) be the standard ordered basis for \(\vs{F}^n\) over \(\F\).
  Then we have
  \begin{align*}
             & A = [\L_A]_{\beta}                                  &  & \text{(by \cref{2.15}(a))} \\
    \implies & f \text{ is the characteristic polynomial of } \L_A &  & \text{(by \cref{5.1.6})}   \\
    \implies & f(\L_A) = \det([\L_A]_{\beta} - t I_n)(\L_A) = \zT  &  & \text{(by \cref{5.23})}    \\
    \implies & f(A) = \det(A - t I_n)(A) = \zm.                    &  & \text{(by \cref{2.3.8})}
  \end{align*}
\end{proof}

\begin{note}
  It is useful to decompose a finite-dimensional vector space \(\V\) over \(\F\) into a direct sum of as many \(\T\)-invariant subspaces as possible because the behavior of \(\T\) on \(\V\) can be inferred from its behavior on the direct summands.
  For example, \(\T\) is diagonalizable iff \(\V\) can be decomposed into a direct sum of one-dimensional \(\T\)-invariant subspaces (see \cref{ex:5.4.36}).
  In \cref{ch:7}, we consider alternate ways of decomposing \(\V\) into direct sums of \(\T\)-invariant subspaces if \(\T\) is not diagonalizable.
\end{note}

\begin{thm}\label{5.24}
  Let \(\T\) be a linear operator on a finite-dimensional vector space \(\V\) over \(\F\), and suppose that \(\V = \seq[\oplus]{\W}{1,,k}\), where \(\W_i\) is a \(\T\)-invariant subspace of \(\V\) over \(\F\) for each \(i \in \set{1, \dots, k}\).
  Suppose that \(f_i\) is the characteristic polynomial of \(\T_{\W}\) for each \(i \in \set{1, \dots, k}\).
  Then \(\prod_{i = 1}^k f_i\) is the characteristic polynomial of \(\T\).
\end{thm}

\begin{proof}[\pf{5.24}]
  The proof is by mathematical induction on \(k\).
  In what follows, \(f\) denotes the characteristic polynomial of \(\T\).
  Suppose first that \(k = 2\).
  Let \(\beta_1\) be an ordered basis for \(\W_1\) over \(\F\), \(\beta_2\) an ordered basis for \(\W_2\) over \(\F\), and \(\beta = \beta_1 \cup \beta_2\).
  Then \(\beta\) is an ordered basis for \(\V\) over \(\F\) by \cref{5.10}(d).
  Let \(A = [\T]_{\beta}\), \(B_1 = [\T_{\W_1}]_{\beta_1}\), and \(B_2 = [\T_{\W_2}]_{\beta_2}\).
  By \cref{ex:5.4.34}, it follows that
  \[
    A = \begin{pmatrix}
      B_1  & \zm \\
      \zm' & B_2
    \end{pmatrix},
  \]
  where \(\zm\) and \(\zm'\) are zero matrices of the appropriate sizes.
  Then
  \[
    f(t) = \det(A - tI) = \det(B_1 - tI) \cdot \det(B_2 - tI) = f_1(t) \cdot f_2(t)
  \]
  as in the proof of \cref{5.21}, proving the result for \(k = 2\).
  Now assume that the theorem is valid for \(k\) summands, where \(k \geq 2\), and suppose that \(\V\) is a direct sum of \(k + 1\) subspaces, say,
  \[
    \V = \seq[\oplus]{\W}{1,,k+1}.
  \]
  Let \(\W = \seq[+]{\W}{1,,k}\).
  It is easily verified that \(\W\) is \(\T\)-invariant and that \(\V = \W \oplus \W_{k + 1}\).
  So by the case for \(k = 2\), \(f = g \cdot f_{k + 1}\), where \(g\) is the characteristic polynomial of \(\T_{\W}\).
  Clearly \(\W = \seq[\oplus]{\W}{1,,k}\), and therefore \(g = \prod_{i = 1}^k f_i\) by the induction hypothesis.
  We conclude that \(f = g \cdot f_{k + 1} = \prod_{i = 1}^{k + 1} f_i\).
\end{proof}

\begin{note}
  As an illustration of \cref{5.24}, suppose that \(\T\) is a diagonalizable linear operator on a finite-dimensional vector space \(\V\) over \(\F\) with distinct eigenvalues \(\seq{\lambda}{1,,k}\).
  By \cref{5.11}, \(\V\) is a direct sum of the eigenspaces of \(\T\).
  Since each eigenspace is \(\T\)-invariant, we may view this situation in the context of \cref{5.24}.
  For each eigenvalue \(\lambda_i\), the restriction of \(\T\) to \(E_{\lambda_i}\) has characteristic polynomial \((\lambda_i - t)^{m_i}\), where \(m_i\) is the dimension of \(E_{\lambda_i}\).
  By \cref{5.24}, the characteristic polynomial \(f\) of \(\T\) is the product
  \[
    f(t) = (\lambda_1 - t)^{m_1} (\lambda_2 - t)^{m_2} \cdots (\lambda_k - t)^{m_k}.
  \]
  It follows that the multiplicity of each eigenvalue is equal to the dimension of the corresponding eigenspace, as expected.
\end{note}

\begin{defn}\label{5.4.5}
  Let \(B_1 \in \ms{m}{m}{\F}\), and let \(B_2 \in \ms{n}{n}{\F}\).
  We define the \textbf{direct sum} of \(B_1\) and \(B_2\), denoted \(B_1 \oplus B_2\), as the \((m + n) \times (m + n)\) matrix \(A\) such that
  \[
    A_{i j} = \begin{dcases}
      (B_1)_{i j}             & \text{for } i, j \in \set{1, \dots, m}         \\
      (B_2)_{(i - m) (j - m)} & \text{for } i, j \in \set{m + 1, \dots, m + n} \\
      0                       & \text{otherwise}
    \end{dcases}.
  \]
  If \(\seq{B}{1,,k}\) are square matrices with entries from \(\F\), then we define the \textbf{direct sum} of \(\seq{B}{1,,k}\) recursively by
  \[
    \seq[\oplus]{B}{1,,k} = (\seq[\oplus]{B}{1,,k-1}) \oplus B_k.
  \]
  If \(A = \seq[\oplus]{B}{1,,k}\), then we often write
  \[
    A = \begin{pmatrix}
      B_1    & \zm    & \cdots & \zm    \\
      \zm    & B_2    & \cdots & \zm    \\
      \vdots & \vdots &        & \vdots \\
      \zm    & \zm    & \cdots & B_k
    \end{pmatrix}.
  \]
\end{defn}

\begin{thm}\label{5.25}
  Let \(\T\) be a linear operator on a finite-dimensional vector space \(\V\) over \(\F\), and let \(\seq{\W}{1,,k}\) be \(\T\)-invariant subspaces of \(\V\) over \(\F\) such that \(\V = \seq[\oplus]{\W}{1,,k}\).
  For each \(i \in \set{1, \dots, k}\), let \(\beta_i\) be an ordered basis for \(\W_i\), and let \(\beta = \bigcup_{i = 1}^k \beta_i\).
  Let \(A = [\T]_{\beta}\) and \(B_i = [\T_{\W_i}]_{\beta_i}\) for \(i \in \set{1, \dots, k}\).
  Then \(A = \seq[\oplus]{B}{1,,k}\).
\end{thm}

\begin{proof}[\pf{5.25}]
  We use induction on \(k\).
  The case for \(k = 1\) is trivial.
  Suppose inductively that the statement is true for some \(k \geq 1\).
  We need to show that it is also true for \(k + 1\).
  So let \(\V\) be a vector space over \(\F\), let \(\T \in \ls(\V)\) and let \(\seq{\W}{1,,k+1}\) be \(\T\)-invariant subspaces of \(\V\) over \(\F\) such that \(\V = \seq[\oplus]{\W}{1,,k+1}\).
  For each \(i \in \set{1, \dots, k + 1}\), let \(\beta_i\) be an ordered basis for \(\W_i\) over \(\F\).
  By \cref{5.10}(a)(d) we know that \(\beta = \bigcup_{i = 1}^{k + 1} \beta_i\) is an ordered basis for \(\V\) over \(\F\).
  If we let \(\W = \seq[\oplus]{\W}{2,,k+1}\), then we have \(\V = \W_1 \oplus \W\) and by \cref{ex:5.4.34} we have
  \[
    [\T]_{\beta} = \begin{pmatrix}
      [\T_{\W_1}]_{\beta_1} & \zm_1                                       \\
      \zm_2                 & [\T_{\W}]_{\bigcup_{i = 2}^{k + 1} \beta_i}
    \end{pmatrix}
  \]
  By induction hypothesis we have
  \[
    [\T_{\W}]_{\bigcup_{i = 2}^{k + 1} \beta_i} = \begin{pmatrix}
      [\T_{\W_2}]_{\beta_2} & \zm                   & \cdots & \zm                               \\
      \zm                   & [\T_{\W_3}]_{\beta_3} & \cdots & \zm                               \\
      \vdots                & \vdots                &        & \vdots                            \\
      \zm                   & \zm                   & \cdots & [\T_{\W_{k + 1}}]_{\beta_{k + 1}}
    \end{pmatrix}.
  \]
  Thus by \cref{5.4.5} we see that
  \[
    [\T]_{\beta} = \begin{pmatrix}
      [\T_{\W_1}]_{\beta_1} & \zm                   & \cdots & \zm                               \\
      \zm                   & [\T_{\W_2}]_{\beta_2} & \cdots & \zm                               \\
      \vdots                & \vdots                &        & \vdots                            \\
      \zm                   & \zm                   & \cdots & [\T_{\W_{k + 1}}]_{\beta_{k + 1}}
    \end{pmatrix} = [\T_{\W_1}]_{\beta_1} \oplus \cdots \oplus [\T_{\W_{k + 1}}]_{\beta_{k + 1}}
  \]
  and this closes the induction.
\end{proof}

\exercisesection

\setcounter{ex}{3}
\begin{ex}\label{ex:5.4.4}
  Let \(\T\) be a linear operator on a vector space \(\V\) over \(\F\), and let \(\W\) be a \(\T\)-invariant subspace of \(\V\) over \(\F\).
  Prove that \(\W\) is \(g(\T)\)-invariant for any polynomial \(g\).
\end{ex}

\begin{proof}[\pf{ex:5.4.4}]
  Let \(g(x) = a_0 + a_1 x + \cdots + a_n x^n\) for some \(\seq{a}{0,,n} \in \F\).
  Then we have
  \begin{align*}
    \forall w \in \W, g(\T)(w) & = (a_0 \IT[\V] + a_1 \T + \cdots + a_n \T^n)(w) &  & \text{(by \cref{e.0.7})} \\
                               & = a_0 w + a_1 \T(w) + \cdots + a_n \T^n(w)                                    \\
                               & \subseteq \W                                    &  & \text{(by \cref{5.4.1})}
  \end{align*}
  and thus by \cref{5.4.1} \(\W\) is \(g(\T)\)-invariant.
\end{proof}

\begin{ex}\label{ex:5.4.5}
  Let \(\T\) be a linear operator on a vector space \(\V\) over \(\F\).
  Prove that the intersection of any collection of \(\T\)-invariant subspaces of \(\V\) over \(\F\) is a \(\T\)-invariant subspace of \(\V\) over \(\F\).
\end{ex}

\begin{proof}[\pf{ex:5.4.5}]
  Let \(K\) be an index set and let \(\set{U_{\alpha} : \alpha \in K}\) be a set of \(\T\)-invariant subspaces of \(\V\) over \(\F\).
  Then we have
  \begin{align*}
             & \forall \alpha \in K, \begin{dcases}
                                       U_{\alpha} \text{ is a subspace of } \V \text{ over } \F \\
                                       \T(U_{\alpha}) \subseteq U_{\alpha}
                                     \end{dcases}                                               &  & \text{(by \cref{5.4.1})}                          \\
    \implies & \begin{dcases}
                 \bigcap_{\alpha \in K} U_{\alpha} \text{ is a subspace of } \V \text{ over } \F \\
                 \T\pa{\bigcap_{\alpha \in K} U_{\alpha}} \subseteq \bigcap_{\alpha \in K} U_{\alpha}
               \end{dcases} &  & \text{(by \cref{1.4})}                                \\
    \implies & \bigcap_{\alpha \in K} U_{\alpha} \text{ is } \T\text{-invariant}.                                        &  & \text{(by \cref{5.4.1})}
  \end{align*}
\end{proof}

\setcounter{ex}{6}
\begin{ex}\label{ex:5.4.7}
  Prove that the restriction of a linear operator \(\T\) to a \(\T\)-invariant subspace is a linear operator on that subspace.
\end{ex}

\begin{proof}[\pf{ex:5.4.7}]
  Let \(\V\) be a vector space over \(\F\), let \(\T \in \ls(\V)\) and let \(\W\) be a \(\T\)-invariant subspace of \(\V\) over \(\F\).
  Let \(x, y \in \W\) and let \(c \in \F\).
  Since
  \begin{align*}
    \T_{\W}(cx + y) & = \T(cx + y)                &  & \text{(by \cref{b.0.4})}    \\
                    & = c \T(x) + \T(y)           &  & \text{(by \cref{2.1.2}(b))} \\
                    & = c \T_{\W}(x) + \T_{\W}(y) &  & \text{(by \cref{b.0.4})}    \\
                    & \in \W,                     &  & \text{(by \cref{1.3})}
  \end{align*}
  by \cref{2.1.2}(b) we know that \(\T_{\W} \in \ls(\W)\).
\end{proof}

\begin{ex}\label{ex:5.4.8}
  Let \(\T\) be a linear operator on a vector space \(\V\) over \(\F\) with a \(\T\)-invariant subspace \(\W\) over \(\F\).
  Prove that if \(v\) is an eigenvector of \(\T_{\W}\) with corresponding eigenvalue \(\lambda\), then the same is true for \(\T\).
\end{ex}

\begin{proof}[\pf{ex:5.4.8}]
  We have
  \begin{align*}
             & \T_{\W}(v) = \lambda v &  & \text{(by \cref{5.1.2})} \\
    \implies & \T(v) = \lambda v.     &  & \text{(by \cref{b.0.4})}
  \end{align*}
\end{proof}

\setcounter{ex}{10}
\begin{ex}\label{ex:5.4.11}
  Let \(\T\) be a linear operator on a vector space \(\V\) over \(\F\), let \(v\) be a nonzero vector in \(\V\), and let \(\W\) be the \(\T\)-cyclic subspace of \(\V\) generated by \(v\).
  Prove that
  \begin{enumerate}
    \item \(\W\) is \(\T\)-invariant.
    \item Any \(\T\)-invariant subspace of \(\V\) over \(\F\) containing \(v\) also contains \(\W\).
  \end{enumerate}
\end{ex}

\begin{proof}[\pf{ex:5.4.11}(a)]
  Let \(x \in \W\).
  Then we have
  \begin{align*}
             & \begin{dcases}
                 \exists k \in \N \\
                 \exists \seq{a}{0,,k} \in \F
               \end{dcases} : x = a_0 v + a_1 \T(v) + \cdots + a_k \T^k(v)  &  & \text{(by \cref{5.4.3})} \\
    \implies & \begin{dcases}
                 \exists k \in \N \\
                 \exists \seq{a}{0,,k} \in \F
               \end{dcases} :                                                                \\
             & \T(x) = a_0 \T(x) + a_1 \T^2(v) + \cdots + a_k \T^{k + 1}(v) &  & \text{(by \cref{2.10})}  \\
    \implies & \T(x) \in \W.                                                &  & \text{(by \cref{5.4.3})}
  \end{align*}
  Since \(x\) is arbitrary, we see that \(\T(\W) \subseteq \W\) and by \cref{5.4.1} \(\W\) is \(\T\)-invariant.
\end{proof}

\begin{proof}[\pf{ex:5.4.11}(b)]
  Let \(\vs{X}\) be a \(\T\)-invariant subspace of \(\V\) over \(\F\) containing \(v\).
  Then we have
  \begin{align*}
             & v \in \vs{X}                                                         \\
    \implies & \T(v) \in \vs{X}                       &  & \text{(by \cref{5.4.1})} \\
    \implies & \forall k \in \Z^+, \T^k(v) \in \vs{X} &  & \text{(by \cref{5.4.1})} \\
    \implies & \W \subseteq \vs{X}.                   &  & \text{(by \cref{1.3})}
  \end{align*}
\end{proof}

\begin{ex}\label{ex:5.4.12}
  Prove that \(A = \begin{pmatrix}
    B_1 & B_2 \\
    \zm & B_3
  \end{pmatrix}\) in the proof of \cref{5.21}.
\end{ex}

\begin{proof}[\pf{ex:5.4.12}]
  Since \(\W\) is a \(\T\)-invariant subspace of \(\V\) over \(\F\), we know that \(\T(\gamma) \subseteq \W = \spn{\gamma}\).
  Thus if \(A = [\T]_{\beta}\) and \(B_1 = [\T_{\W}]_{\gamma}\), then the first \(\#(\gamma)\) columns of \(A\) can be uniquely express as linear combinations of \(\gamma\) (\cref{1.8}), with coefficients precisely those in the corresponding column of \(B_1\) (\cref{2.2.4}).
\end{proof}

\begin{ex}\label{ex:5.4.13}
  Let \(\T\) be a linear operator on a vector space \(\V\) over \(\F\), let \(v\) be a nonzero vector in \(\V\), and let \(\W\) be the \(\T\)-cyclic subspace of \(\V\) generated by \(v\).
  For any \(w \in \V\), prove that \(w \in \W\) iff there exists a polynomial \(g\) such that \(w = g(\T)(v)\).
\end{ex}

\begin{proof}[\pf{ex:5.4.13}]
  We have
  \begin{align*}
         & w \in \W                                                                                  \\
    \iff & \begin{dcases}
             \exists k \in \N \\
             \exists \seq{a}{0,,k} \in \F
           \end{dcases} : w = a_0 v + a_1 \T(v) + \cdots + a_k \T^k(v) &  & \text{(by \cref{5.4.3})} \\
    \iff & \begin{dcases}
             \exists k \in \N \\
             \exists \seq{a}{0,,k} \in \F
           \end{dcases} : \begin{dcases}
                            g(t) = a_0 + a_1 t + \cdots + a_k t^k \\
                            w = g(\T)(v)
                          \end{dcases}.                    &  & \text{(by \cref{e.0.7})}
  \end{align*}
\end{proof}

\begin{ex}\label{ex:5.4.14}
  Prove that the polynomial \(g\) of \cref{ex:5.4.13} can always be chosen so that its degree is less than \(\dim(\W)\).
\end{ex}

\begin{proof}[\pf{ex:5.4.14}]
  By \cref{5.22}(a) we see that if \(k = \dim(\W)\) then
  \[
    \set{v, \T(v), \dots, \T^{k - 1}(v)}
  \]
  is a basis for \(\W\) over \(\F\).
  Thus
  \[
    w \in \W \iff \exists \seq{a}{0,,k-1} : \begin{dcases}
      g(t) = a_0 + a_1 t + \cdots + a_{k - 1} t^{k - 1} \\
      w = g(\T)(v)
    \end{dcases}.
  \]
\end{proof}

\setcounter{ex}{15}
\begin{ex}\label{ex:5.4.16}
  Let \(\T\) be a linear operator on a finite-dimensional vector space \(\V\) over \(\F\).
  \begin{enumerate}
    \item Prove that if the characteristic polynomial of \(\T\) splits, then so does the characteristic polynomial of the restriction of \(\T\) to any \(\T\)-invariant subspace of \(\V\).
    \item Deduce that if the characteristic polynomial of \(\T\) splits, then any nontrivial \(\T\)-invariant subspace of \(\V\) contains an eigenvector of \(\T\).
  \end{enumerate}
\end{ex}

\begin{proof}[\pf{ex:5.4.16}(a)]
  Let \(\W\) be a \(\T\)-invariant subspace of \(\V\) over \(\F\), and let \(f, g\) be the characteristic polynomials of \(\T, \T_{\W}\), respectively.
  By \cref{5.21} we see that \(g\) divides \(f\), thus there exists some polynomial \(q\) such that \(f = g \cdot q\).
  Since \(f\) splits, we see that \(g\) and \(q\) must split.
\end{proof}

\begin{proof}[\pf{ex:5.4.16}(b)]
  Let \(\W\) be a nontrivial \(\T\)-invariant subspace of \(\V\) over \(\F\), and let \(f, g\) be the characteristic polynomials of \(\T, \T_{\W}\), respectively.
  By \cref{ex:5.4.16}(a) we know that \(f, g\) split and \(f = g \cdot q\) for some polynomial \(q\), thus there exists some \(\lambda \in \F\) such that \(g(\lambda) = f(\lambda) = 0\).
  By \cref{5.2} we see that \(\lambda\) is an eigenvalue of \(\T\) and \(\T_{\W}\).
  Since \(\W \neq \set{\zv}\), by \cref{5.7} we can find some \(v \in \W \setminus \set{\zv}\) such that \(\T_{\W}(v) = \lambda v\).
  Thus \(\T(v) = \lambda v\) and by \cref{5.1.2} \(v\) is an eigenvector of \(\T\).
\end{proof}

\begin{ex}\label{ex:5.4.17}
  Let \(A \in \ms{n}{n}{\F}\).
  Prove that
  \[
    \dim\pa{\spn{\set{I_n, A, A^2, \dots}}} \leq n.
  \]
\end{ex}

\begin{proof}[\pf{ex:5.4.17}]
  Let \(f\) be the characteristic polynomial of \(A\).
  By \cref{5.3} there exist some \(\seq{a}{0,,n-1} \in \F\) such that
  \[
    f(t) = a_0 + a_1 t + \cdots + a_{n - 1} t^{n - 1} + (-1)^n t^n.
  \]
  By Cayley--Hamilton theorem (\cref{5.4.4}) we see that
  \[
    f(A) = a_0 I_n + a_1 A + \cdots + a_{n - 1} A^{n - 1} + (-1)^n A^n = \zm.
  \]
  Thus \(A^n \in \spn{\set{I_n, A, A^2, \dots, A^{n - 1}}}\).
  Since
  \[
    a_0 A + a_1 A^2 + \cdots + a_{n - 1} A^n + (-1)^n A^{n + 1} = \zm,
  \]
  we see that \(A^{n + 1} \in \spn{\set{A, A^2, \dots, A^{n - 1}, A^n}} = \spn{\set{I_n, A, A^2, \dots, A^{n - 1}}}\).
  Thus for all \(m \geq n\), we have \(A^m \in \spn{\set{I_n, A, A^2, \dots, A^{n - 1}}}\).
  By \cref{1.6.8} this means \(\dim\pa{\spn{\set{I_n, A, A^2, \dots}}} = \dim\pa{\spn{\set{I_n, A, A^2, \dots, A^{n - 1}}}} \leq n\).
\end{proof}

\begin{ex}\label{ex:5.4.18}
  Let \(A \in \ms{n}{n}{\F}\) with characteristic polynomial
  \[
    f(t) = (-1)^n t^n + a_{n - 1} t^{n - 1} + \cdots + a_1 t + a_0.
  \]
  \begin{enumerate}
    \item Prove that \(A\) is invertible iff \(a_0 \neq 0\).
    \item Prove that if \(A\) is invertible, then
          \[
            A^{-1} = \frac{-1}{a_0} \pa{(-1)^n A^{n - 1} + a_{n - 1} A^{n - 2} + \cdots + a_1 I_n}.
          \]
    \item Use (b) to compute \(A^{-1}\) for
          \[
            A = \begin{pmatrix}
              1 & 2 & 1  \\
              0 & 2 & 3  \\
              0 & 0 & -1
            \end{pmatrix}.
          \]
  \end{enumerate}
\end{ex}

\begin{proof}[\pf{ex:5.4.18}(a)]
  See \cref{ex:5.1.20}
\end{proof}

\begin{proof}[\pf{ex:5.4.18}(b)]
  We have
  \begin{align*}
             & f(A) = (-1)^n A^n + a_{n - 1} A^{n - 1} + \cdots + a_1 A + a_0 I_n = \zm                   &  & \text{(by \cref{5.4.4})} \\
    \implies & (-1)^n A^n + a_{n - 1} A^{n - 1} + \cdots + a_1 A = -a_0 I_n                                                             \\
    \implies & A \pa{(-1)^n A^{n - 1} + a_{n - 1} A^{n - 2} + \cdots + a_1 I_n} = -a_0 I_n                &  & \text{(by \cref{2.3.5})} \\
    \implies & A \pa{\frac{-1}{a_0} \pa{(-1)^n A^{n - 1} + a_{n - 1} A^{n - 2} + \cdots + a_1 I_n}} = I_n &  & (a_0 \neq 0)             \\
    \implies & \frac{-1}{a_0} \pa{(-1)^n A^{n - 1} + a_{n - 1} A^{n - 2} + \cdots + a_1 I_n} = A^{-1}.    &  & \text{(by \cref{2.4.3})}
  \end{align*}
\end{proof}

\begin{proof}[\pf{ex:5.4.18}(c)]
  We have
  \begin{align*}
    \det(A - t I_3) & = \begin{pmatrix}
                          1 - t & 2     & 1      \\
                          0     & 2 - t & 3      \\
                          0     & 0     & -1 - t
                        \end{pmatrix} &  & \text{(by \cref{4.2.2})}               \\
                    & = (1 - t)(2 - t)(-1 - t)  &  & \text{(by \cref{ex:4.2.23})} \\
                    & = -t^3 + 2t^2 + t - 2
  \end{align*}
  and
  \begin{align*}
    A^{-1} & = \frac{-1}{-2} \pa{-A^2 + 2A + I_3} &  & \text{(by \cref{ex:5.4.18})} \\
           & = \frac{1}{2} \pa{-\begin{pmatrix}
                                    1 & 6 & 6 \\
                                    0 & 4 & 3 \\
                                    0 & 0 & 1
                                  \end{pmatrix} + 2\begin{pmatrix}
                                                     1 & 2 & 1  \\
                                                     0 & 2 & 3  \\
                                                     0 & 0 & -1
                                                   \end{pmatrix} + I_3}               \\
           & = \begin{pmatrix}
                 1 & -1          & -2          \\
                 0 & \frac{1}{2} & \frac{3}{2} \\
                 0 & 0           & -1
               \end{pmatrix}.
  \end{align*}
\end{proof}

\begin{ex}\label{ex:5.4.19}
  Let \(A \in \ms{k}{k}{\F}\)
  \[
    A = \begin{pmatrix}
      0      & 0      & \cdots & 0      & -a_0       \\
      1      & 0      & \cdots & 0      & -a_1       \\
      0      & 1      & \cdots & 0      & -a_2       \\
      \vdots & \vdots &        & \vdots & \vdots     \\
      0      & 0      & \cdots & 0      & -a_{k - 2} \\
      0      & 0      & \cdots & 1      & -a_{k - 1}
    \end{pmatrix}
  \]
  where \(\seq{a}{0,,k-1} \in \F\).
  Prove that the characteristic polynomial of \(A\) is
  \[
    (-1)^k (a_0 + a_1 t + \cdots + a_{k - 1} t^{k - 1} + t^k).
  \]
\end{ex}

\begin{proof}[\pf{ex:5.4.19}]
  We use induction on \(k\).
  For \(k = 1\), we have
  \begin{align*}
    \det(-a_0 - t) & = -a_0 - t         &  & \text{(by \cref{4.2.2})} \\
                   & = (-1)^1 (a_0 + t)
  \end{align*}
  and thus the base case holds.
  Suppose inductively that for some \(k \geq 1\) the statement is true.
  We need to show that the statement is also true for \(k + 1\).
  Let \(A \in \ms{(k + 1)}{(k + 1)}{\F}\) and let \(\seq{a}{0,,k} \in \F\) such that
  \[
    A = \begin{pmatrix}
      0      & 0      & \cdots & 0      & -a_0       \\
      1      & 0      & \cdots & 0      & -a_1       \\
      \vdots & \vdots &        & \vdots & \vdots     \\
      0      & 0      & \cdots & 0      & -a_{k - 1} \\
      0      & 0      & \cdots & 1      & -a_k
    \end{pmatrix}.
  \]
  Let \(B = A - t I_{k + 1}\).
  Then we have
  \begin{align*}
    \det(B) & = \sum_{j = 1}^{k + 1} (-1)^{1 + j} B_{1 j} \det(\tilde{B}_{1 j})                        &  & \text{(by \cref{4.2.2})}         \\
            & = B_{1 1} \det(\tilde{B}_{1 1}) + (-1)^{k + 2} B_{1 (k + 1)} \det(\tilde{B}_{1 (k + 1)})                                       \\
            & = (-t) \begin{pmatrix}
                       -t     & 0      & \cdots & 0  & -a_1     \\
                       1      & -t     & \cdots & 0  & -a_2     \\
                       \vdots & \vdots &        & -t & \vdots   \\
                       0      & 0      & \cdots & 1  & -a_k - t
                     \end{pmatrix}                                                                                \\
            & \quad + (-1)^{k + 2} (-a_0) \begin{pmatrix}
                                            1      & -t     & 0      & \cdots & 0      & 0      \\
                                            0      & 1      & -t     & \cdots & 0      & 0      \\
                                            \vdots & \vdots & \vdots &        & \vdots & \vdots \\
                                            0      & 0      & 0      & \cdots & 1      & -t     \\
                                            0      & 0      & 0      & \cdots & 0      & 1
                                          \end{pmatrix}                                                \\
            & = (-t) (-1)^k (a_1 + a_2 t + \cdots + a_k t^{k - 1} + t^k)                               &  & \text{(by induction hypothesis)} \\
            & \quad + (-1)^{k + 3} a_0                                                                 &  & \text{(by \cref{ex:4.2.23})}     \\
            & = (-1)^{k + 1} (a_0 + a_1 t + \cdots + a_k t^k + t^{k + 1}).
  \end{align*}
  This closes the induction.
\end{proof}

\begin{ex}\label{ex:5.4.20}
  Let \(\T\) be a linear operator on a vector space \(\V\) over \(\F\), and suppose that \(\V\) is a \(\T\)-cyclic subspace of itself.
  Prove that if \(\U\) is a linear operator on \(\V\) over \(\F\), then \(\U \T = \T \U\) iff \(\U = g(\T)\) for some polynomial \(g\).
\end{ex}

\begin{proof}[\pf{ex:5.4.20}]
  Since \(\V\) is a \(\T\)-cyclic subspace of itself, by \cref{5.4.3} there exists some \(v \in \V\) such that \(\V = \spn{\set{v, \T(v), \T^2(v), \dots,}}\).

  First suppose that \(\U \T = \T \U\).
  Since \(\U(v) \in \V\), by \cref{ex:5.4.13} there exists some \(g \in \ps{\F}\) such that \(\U(v) = g(\T)(v)\).
  Thus
  \[
    \exists \seq{a}{0,,n} \in \F : \begin{dcases}
      g(t) = a_0 + a_1 t + \cdots + a_n t^n            \\
      g(\T) = a_0 \IT[\V] + a_1 \T + \cdots + a_n \T^n \\
      g(\T)(v) = \U(v)
    \end{dcases}.
  \]
  Then we have
  \begin{align*}
             & \forall w \in \V, \begin{dcases}
                                   \exists k \in \N \\
                                   \exists \seq{b}{0,,k} \in \F
                                 \end{dcases} :                                                               \\
             & w = b_0 v + b_1 \T(v) + \cdots + b_k \T^k(v)                               &  & \text{(by \cref{5.4.3})}    \\
    \implies & \forall w \in \V, \begin{dcases}
                                   \exists k \in \N \\
                                   \exists \seq{b}{0,,k} \in \F
                                 \end{dcases} :                                                               \\
             & \U(w) = \U\pa{b_0 v + b_1 \T(v) + \cdots + b_k \T^k(v)}                                                     \\
             & = b_0 \U(v) + b_1 \U(\T(v)) \cdots + b_k \U(\T^k(v))                       &  & \text{(by \cref{2.1.2}(b))} \\
             & = b_0 \U(v) + b_1 \T(\U(v)) \cdots + b_k \T^k(\U(v))                       &  & (\U \T = \T \U)             \\
             & = b_0 (a_0 v + a_1 \T(v) + \cdots + a_n \T^n(v))                           &  & (g(\T)(v) = \U(v))          \\
             & \quad + b_1 (a_0 \T(v) + a_1 \T^2(v) + \cdots + a_n \T^{n + 1}(v))         &  & \text{(by \cref{2.1.2}(b))} \\
             & \quad + \cdots                                                                                              \\
             & \quad + b_k (a_0 \T^k(v) + a_1 \T^{k + 1}(v) + \cdots + a_n \T^{n + k}(v)) &  & \text{(by \cref{2.1.2}(b))} \\
             & = a_0 (b_0 v + b_1 \T(v) + \cdots + b_k \T^k(v))                           &  & \text{(by \cref{1.2.1})}    \\
             & \quad + a_1 \T\pa{b_0 v + b_1 \T(v) + \cdots + b_k \T^k(v)}                &  & \text{(by \cref{2.1.2}(b))} \\
             & \quad + \cdots                                                                                              \\
             & \quad + a_n \T^n\pa{b_0 v + b_1 \T(v) + \cdots + b_k \T^k(v)}              &  & \text{(by \cref{2.1.2}(b))} \\
             & = (a_0 + a_1 \T + \cdots + a_n \T^n)(w)                                                                     \\
             & = g(\T)(w)                                                                                                  \\
    \implies & \U = g(\T).
  \end{align*}

  Now suppose that \(\U = g(\T)\) for some \(g \in \ps{\F}\).
  Let \(g(t) = a_0 + a_1 t + \cdots + a_n t^n\) for some \(\seq{a}{0,,n} \in \F\).
  Then we have
  \begin{align*}
    \forall w \in \V, (\U \T)(w) & = (g(\T)(\T))(w)                                                                    \\
                                 & = (a_0 \T + a_1 \T^2 + \cdots + a_n \T^{n + 1})(w)  &  & \text{(by \cref{2.10}(a))} \\
                                 & = (\T(a_0 \IT[\V] + a_1 \T + \cdots + a_n \T^n))(w) &  & \text{(by \cref{2.10}(a))} \\
                                 & = ((\T) g(\T))(w)                                                                   \\
                                 & = (\T \U)(w)
  \end{align*}
  and thus \(\U \T = \T \U\).
\end{proof}

\begin{ex}\label{ex:5.4.21}
  Let \(\T\) be a linear operator on a two-dimensional vector space \(\V\) over \(\F\).
  Prove that either \(\V\) is a \(\T\)-cyclic subspace of itself or \(\T = c \IT[\V]\) for some \(c \in \F\).
\end{ex}

\begin{proof}[\pf{ex:5.4.21}]
  If \(\T = \zT\), then we see that \(\T = 0 \IT[\V]\).
  So suppose that \(\T \neq \zT\).
  Now we split into two cases:
  \begin{itemize}
    \item If there exists some \(v \in \V\) such that \(\dim(\spn{\set{v, \T(v)}}) = 2\), then by \cref{1.6.15}(b) we have \(\V = \spn{\set{v, \T(v)}}\).
          By \cref{5.4.3} we see that \(\V\) is a \(\T\)-cyclic subspace of itself generated by \(v\).
    \item If there does not exist such \(v \in \V\), then we have \(\dim(\spn{\set{v, \T(v)}}) < 2\) for all \(v \in \V\).
          Since \(\T \neq \zT\), there exist some \(w_1 \in \V\) such that \(\T(w_1) \neq \zv\).
          Fix such \(w_1\).
          Since \(\T(w_1) \in \spn{\set{w_1}} \setminus \set{\zv}\), there exists some \(c_1 \in \F \setminus \set{0}\) such that \(\T(w_1) = c_1 w_1\).
          Now we extend \(\set{w_1}\) to a basis \(\set{\seq{w}{1,2}}\) for \(\V\) over \(\F\).
          By our hypothesis we have
          \[
            \T(w_2) \in \spn{\set{w_2}} \quad \text{and} \quad \T(w_1 + w_2) \in \spn{\set{w_1 + w_2}}.
          \]
          This means \(\T(w_2) = c_2 w_2\) and \(\T(w_1 + w_2) = c_3 (w_1 + w_2)\) for some \(c_2, c_3 \in \F\).
          Now observe that
          \begin{align*}
            \T(w_1 + w_2) & = c_3 (w_1 + w_2)                                  \\
                          & = c_3 w_1 + c_3 w_2  &  & \text{(by \cref{1.2.1})} \\
                          & = \T(w_1) + \T(w_2)  &  & \text{(by \cref{2.1.1})} \\
                          & = c_1 w_1 + c_2 w_2.
          \end{align*}
          Since \(\set{w_1, w_2}\) is linearly independent, we know that
          \[
            (c_3 - c_1) w_1 + (c_3 - c_2) w_2 = \zv \implies c_1 = c_3 = c_2.
          \]
          Thus by setting \(c = c_1\) we see that \(\T = c \IT[\V]\).
  \end{itemize}
  From all cases above we see that \cref{ex:5.4.21} is true.
\end{proof}

\begin{ex}\label{ex:5.4.22}
  Let \(\T\) be a linear operator on a two-dimensional vector space \(\V\) over \(\F\) and suppose that \(\T \neq c \IT[\V]\) for any scalar \(c \in \F\).
  Show that if \(\U \in \ls(\V)\) such that \(\U \T = \T \U\), then \(\U = g(\T)\) for some polynomial \(g\).
\end{ex}

\begin{proof}[\pf{ex:5.4.22}]
  By \cref{ex:5.4.21} we see that \(\V\) is a \(\T\)-cyclic subspace of itself.
  By \cref{ex:5.4.20} we conclude that \(\U = g(\T)\) for some \(g \in \ps{\F}\).
\end{proof}

\begin{ex}\label{ex:5.4.23}
  Let \(\T\) be a linear operator on a finite-dimensional vector space \(\V\) over \(\F\), and let \(\W\) be a \(\T\)-invariant subspace of \(\V\) over \(\F\).
  Suppose that \(\seq{v}{1,,k}\) are eigenvectors of \(\T\) corresponding to distinct eigenvalues.
  Prove that if \(\seq[+]{v}{1,,k} \in \W\), then \(v_i \in \W\) for all \(i \in \set{1, \dots, k}\).
\end{ex}

\begin{proof}[\pf{ex:5.4.23}]
  We use induction on \(k\).
  The case for \(k = 1\) is trivial.
  So suppose inductively that for some \(k \geq 1\) the statement is true.
  We need to show that for \(k + 1\) the statement is true.
  For each \(i \in \set{1, \dots, k + 1}\), let \(v_i\) be an eigenvector of \(\T\) with corresponding eigenvalue \(\lambda_i\) such that \(\seq{\lambda}{1,,k+1}\) are distinct and \(\seq[+]{v}{1,,k+1} \in \W\).
  Since \(\W\) is \(\T\)-invariant, by \cref{1.3} we know that \(\W\) is also \((\T - \lambda_{k + 1} \IT[\V])\)-invariant.
  Thus we see that
  \begin{align*}
             & \seq[+]{v}{1,,k+1} \in \W                                                                                            \\
    \implies & (\T - \lambda_{k + 1} \IT[\V])(\seq[+]{v}{1,,k+1}) \in \W                              &  & \text{(by \cref{5.4.1})} \\
    \implies & (\lambda_1 - \lambda_{k + 1}) v_1 + \cdots + (\lambda_k - \lambda_{k + 1}) v_k \in \W. &  & \text{(by \cref{5.1.2})}
  \end{align*}
  Since \(\lambda_i - \lambda_{k + 1} \neq 0\) for all \(i \in \set{1, \dots, k}\), by \cref{5.2.4} we see that \((\lambda_i - \lambda_{k + 1}) v_i\) is an eigenvector of \(\T\) and \((\lambda_i - \lambda_{k + 1}) v_i \in E_{\lambda_i}\).
  Thus by induction hypothesis we see that \((\lambda_i - \lambda_{k + 1}) v_i \in \W\) for all \(i \in \set{1, \dots, k}\).
  Then we have
  \begin{align*}
             & \forall i \in \set{1, \dots, k}, \lambda_i - \lambda_{k + 1} \neq 0                                                                           \\
    \implies & \forall i \in \set{1, \dots, k}, \frac{\lambda_i - \lambda_{k + 1}}{\lambda_i - \lambda_{k + 1}} v_i = v_i \in \W &  & \text{(by \cref{1.3})} \\
    \implies & \seq[+]{v}{1,,k} \in \W                                                                                           &  & \text{(by \cref{1.3})} \\
    \implies & \seq[+]{v}{1,,k+1} + -(\seq[+]{v}{1,,k}) = v_{k + 1} \in \W                                                       &  & \text{(by \cref{1.3})}
  \end{align*}
  and this closes the induction.
\end{proof}

\begin{ex}\label{ex:5.4.24}
  Prove that the restriction of a diagonalizable linear operator \(\T\) to any nontrivial \(\T\)-invariant subspace is also diagonalizable.
\end{ex}

\begin{proof}[\pf{ex:5.4.24}]
  Let \(\V\) be a finite-dimensional vector space over \(\F\), let \(\T \in \ls(\V)\) such that \(\T\) is diagonalizable and let \(\W\) be a nontrivial \(\T\)-invariant subspace of \(\V\) over \(\F\).
  By \cref{5.1.1} there exist an ordered basis \(\beta\) such that \([\T]_{\beta}\) is diagonal matrix.
  We claim that \(\W \cap \beta \neq \varnothing\) and \(\W \cap \beta\) is an ordered basis for \(\W\).
  If not, then we must have \(\W = \set{\zv}\), which contradict to our hypothesis that \(\W\) is nontrivial.
  Let \(\beta_{\W} = \W \cap \beta\).
  Since \(\beta\) is consist of eigenvectors of \(\T\) and \(\W\) is \(\T\)-invariant, by \cref{5.1.2,5.4.1} we know that \(\T(\beta_{\W}) = \T_{\W}(\beta_{\W}) \subseteq \W\) and \(\beta_{\W}\) is consist of eigenvectors of \(\T_{\W}\).
  Thus \([\T_{\W}]_{\beta_{\W}}\) is diagonal matrix and by \cref{5.1.1} \(\T_{\W}\) is diagonalizable.
\end{proof}

\begin{ex}\label{ex:5.4.25}
  \begin{enumerate}
    \item Prove the converse to \cref{ex:5.2.18}(a):
          If \(\T\) and \(\U\) are diagonalizable linear operators on a finite-dimensional vector space \(\V\) over \(\F\) such that \(\U \T = \T \U\), then \(\T\) and \(\U\) are simultaneously diagonalizable.
          (See the definitions in the \cref{5.2.8}.)
    \item State and prove a matrix version of (a).
  \end{enumerate}
\end{ex}

\begin{proof}[\pf{ex:5.4.25}(a)]
  Let \(\lambda\) be an eigenvalue of \(\T\) and let \(E_{\lambda} = \ns{\T - \lambda \IT[\V]}\).
  We claim that \(E_{\lambda}\) is \(\U\)-invariant.
  This is true since
  \begin{align*}
             & \forall v \in E_{\lambda}, \T(v) = \lambda v                                     &  & \text{(by \cref{5.1.2})} \\
    \implies & \forall v \in E_{\lambda}, \U(\T(v)) = \U(\lambda v) = \lambda \U(v) = \T(\U(v)) &  & (\U \T = \T \U)          \\
    \implies & \forall v \in E_{\lambda}, \U(v) \in E_{\lambda}                                 &  & \text{(by \cref{5.2.4})} \\
    \implies & \U(E_{\lambda}) \subseteq E_{\lambda}.
  \end{align*}
  Since \(\U\) is diagonalizable and \(E_{\lambda}\) is \(\U\)-invariant, by \cref{ex:5.4.24} we see that \(\U_{E_{\lambda}}\) is diagonalizable.
  Thus we can find an ordered basis \(\beta_{\lambda}\) for \(E_{\lambda}\) over \(\F\) such that \([\U]_{\beta_{\lambda}}\) is a diagonal matrix.
  Note that since \(\beta_{\lambda}\) is an ordered basis for \(E_{\lambda}\) over \(\F\), \([\T]_{\beta_{\lambda}}\) is also a diagonal matrix.
  Applying the above argument to every eigenvalue of \(\T\) we can find an ordered basis \(\beta\) over \(\F\) such that \(\beta\) is an union of bases of eigenspaces (by \cref{5.9}(b)) and both \([\U]_{\beta}\) and \([\T]_{\beta}\) are diagonal matrix.
  Thus by \cref{5.2.8} \(\T, \U\) are simultaneously diagonalizable.
\end{proof}

\begin{proof}[\pf{ex:5.4.25}(b)]
  Let \(A, B \in \ms{n}{n}{\F}\) be diagonalizable matrix.
  We claim that if \(AB = BA\), then \(A, B\) are simultaneously diagonalizable.
  This is true since
  \begin{align*}
             & AB = BA                                                                                   \\
    \implies & \L_A \L_B = \L_{AB} = \L_{BA} = \L_B \L_A            &  & \text{(by \cref{2.15}(e))}      \\
    \implies & \L_A, \L_B \text{ are simultaneously diagonalizable} &  & \text{(by \cref{ex:5.4.25}(a))} \\
    \implies & A, B \text{ are simultaneously diagonalizable}.      &  & \text{(by \cref{ex:5.2.17}(a))}
  \end{align*}
\end{proof}

\begin{ex}\label{ex:5.4.26}
  Let \(\T\) be a linear operator on an \(n\)-dimensional vector space \(\V\) over \(\F\) such that \(\T\) has \(n\) distinct eigenvalues.
  Prove that \(\V\) is a \(\T\)-cyclic subspace of itself.
\end{ex}

\begin{proof}[\pf{ex:5.4.26}]
  For each \(i \in \set{1, \dots, n}\), let \(v_i\) be an eigenvector of \(\T\) corresponding to eigenvalue \(\lambda_i \in \F\).
  Let \(v = \seq[+]{v}{1,,n}\) and let \(\W\) be the \(\T\)-cyclic subspace generated by \(v\).
  By \cref{ex:5.4.11}(a) we know that \(\W\) is \(\T\)-invariant, thus by \cref{ex:5.4.23} we see that \(v_i \in \W\) for all \(i \in \set{1, \dots, n}\).
  But by \cref{5.5} we know that \(\set{\seq{v}{1,,n}}\) is a basis for \(\V\) over \(\F\), thus \(\W = \V\) and by \cref{5.4.3} \(\V\) is a \(\T\)-cyclic subspace of itself.
\end{proof}

\begin{defn}\label{5.4.6}
  Let \(\T\) be a linear operator on a vector space \(\V\) over \(\F\), and let \(\W\) be a \(\T\)-invariant subspace of \(\V\) over \(\F\).
  Define \(\overline{\T} : \V / \W \to \V / \W\) by
  \[
    \overline{\T}(v + \W) = \T(v) + \W \quad \text{for any } v + \W \in \V / \W.
  \]
\end{defn}

\begin{ex}\label{ex:5.4.27}
  Using the symbols in \cref{5.4.6}.
  \begin{enumerate}
    \item Prove that \(\overline{\T}\) is well defined.
          That is, show that \(\overline{\T}(v + \W) = \overline{\T}(v' + \W)\) whenever \(v + \W = v' + \W\).
    \item Prove that \(\overline{\T}\) is a linear operator on \(\V / \W\).
    \item Let \(\eta : \V \to \V / \W\) be the linear transformation defined in \cref{ex:2.1.40} by \(\eta(v) = v + \W\).
          prove that \(\eta \T = \overline{\T} \eta\).
  \end{enumerate}
\end{ex}

\begin{proof}[\pf{ex:5.4.27}(a)]
  We have
  \begin{align*}
             & v + \W = v' + \W                                                                     \\
    \implies & v - v' \in \W                                   &  & \text{(by \cref{ex:1.3.31}(b))} \\
    \implies & \T(v - v') \in \W                               &  & \text{(by \cref{5.4.1})}        \\
    \implies & \T(v) - \T(v') \in \W                           &  & \text{(by \cref{2.1.2}(c))}     \\
    \implies & \T(v) + \W = \T(v') + \W                        &  & \text{(by \cref{ex:1.3.31}(b))} \\
    \implies & \overline{\T}(v + \W) = \overline{\T}(v' + \W). &  & \text{(by \cref{5.4.6})}
  \end{align*}
\end{proof}

\begin{proof}[\pf{ex:5.4.27}(b)]
  Let \(x + \W, y + \W \in \V / \W\) and let \(c \in \F\).
  Then we have
  \begin{align*}
    \overline{\T}(c(x + \W) + (y + \W)) & = \overline{\T}((cx + y) + \W)                    &  & \text{(by \cref{ex:1.3.31}(c))} \\
                                        & = \T(cx + y) + \W                                 &  & \text{(by \cref{5.4.6})}        \\
                                        & = (c \T(x) + \T(y)) + \W                          &  & \text{(by \cref{2.1.2}(b))}     \\
                                        & = c (\T(x) + \W) + (\T(y) + \W)                   &  & \text{(by \cref{ex:1.3.31}(c))} \\
                                        & = c \overline{\T}(x + \W) + \overline{\T}(y + \W) &  & \text{(by \cref{5.4.6})}
  \end{align*}
  and thus by \cref{2.1.2}(b) \(\overline{\T} \in \ls(\V / \W)\).
\end{proof}

\begin{proof}[\pf{ex:5.4.27}(c)]
  We have
  \begin{align*}
    \forall v \in \V, (\eta \T)(v) & = \eta(\T(v))                                               \\
                                   & = \T(v) + \W              &  & \text{(by \cref{ex:2.1.40})} \\
                                   & = \overline{\T}(v + \W)   &  & \text{(by \cref{5.4.6})}     \\
                                   & = \overline{\T}(\eta(v))  &  & \text{(by \cref{ex:2.1.40})} \\
                                   & = (\overline{\T} \eta)(v)
  \end{align*}
  and thus \(\eta \T = \overline{\T} \eta\).
\end{proof}

\begin{ex}\label{ex:5.4.28}
  Let \(\V\) be an \(n\)-dimensional vector space over \(\F\), let \(\T \in \ls(\V)\) and let \(\W\) be a nonzero \(\T\)-invariant subspace of \(\V\) over \(\F\).
  Let \(f, g, h\) be the characteristic polynomials of \(\T, \T_{\W}, \overline{\T}\), respectively.
  Prove that \(f = gh\).
\end{ex}

\begin{proof}[\pf{ex:5.4.28}]
  Let \(\gamma = \set{\seq{v}{1,,k}}\) be an ordered basis for \(\W\) over \(\F\).
  By \cref{1.6.19} we can extend \(\gamma\) to an ordered basis \(\beta = \set{\seq{v}{1,,k,k+1,,n}}\) for \(\V\) over \(\F\).
  Let \(\alpha = \set{v_{k + 1} + \W, \dots, v_n + \W}\).
  By \cref{ex:1.6.35}(a) we know that \(\alpha\) is an ordered basis for \(\V / \W\) over \(\F\).
  Since \(\W\) is \(\T\)-invariant, by \cref{ex:5.4.12} we see that
  \[
    [\T]_{\beta} = \begin{pmatrix}
      [\T_{\W}]_{\gamma} & B_1 \\
      \zm                & B_2
    \end{pmatrix}
  \]
  where \(B_1 \in \ms{k}{(n - k)}{\F}\) and \(B_2 \in \ms{(n - k)}{(n - k)}{\F}\).
  If we can show that \(B_2 = [\overline{\T}]_{\alpha}\), then by \cref{ex:4.3.21} we see that \(f = gh\).

  Let \(i \in \set{1, \dots, k}\).
  Since
  \begin{align*}
    \overline{\T}(v_i + \W) & = \T(v_i) + \W                                                                  &  & \text{(by \cref{5.4.6})}                \\
                            & = \pa{\sum_{j = 1}^n ([\T]_{\beta})_{j i} v_j} + \W                             &  & \text{(by \cref{2.2.4})}                \\
                            & = \pa{\sum_{j = 1}^k (B_1)_{j i} v_j + \sum_{j = k + 1}^n (B_2)_{j i} v_j} + \W                                              \\
                            & = \pa{\sum_{j = k + 1}^n (B_2)_{j i} v_j} + \W                                  &  & (\sum_{j = 1}^k (B_1)_{j i} v_j \in \W) \\
                            & = \sum_{j = k + 1}^n (B_2)_{j i} (v_j + \W),                                    &  & \text{(by \cref{ex:1.3.31}(c))}
  \end{align*}
  by \cref{2.2.4} we see that \([\overline{\T}]_{\alpha} = B_2\).
  Thus we have \(f = gh\).
\end{proof}

\begin{ex}\label{ex:5.4.29}
  Let \(\V\) be a finite-dimensional vector space over \(\F\), let \(\T \in \ls(\V)\) and let \(\W\) be a nonzero \(\T\)-invariant subspace of \(\V\) over \(\F\).
  Prove that if \(\T\) is diagonalizable, then so is \(\overline{\T}\).
\end{ex}

\begin{proof}[\pf{ex:5.4.29}]
  By \cref{ex:5.4.24} we know that \(\T_{\W}\) is diagonalizable, thus there exists an ordered basis \(\gamma\) for \(\W\) over \(\F\) such that \([\T_{\W}]_{\gamma}\) is a diagonal matrix.
  By \cref{1.6.19} we can extend \(\gamma\) to an ordered basis \(\beta\) for \(\V\) over \(\F\) such that \([\T]_{\beta}\) is a diagonal matrix.
  If we let \(\alpha = \eta(\beta \setminus \gamma)\) (\cref{ex:5.4.27}(c)), then by \cref{ex:1.6.35}(a) we know that \(\alpha\) is an ordered basis for \(\V / \W\) over \(\F\).
  By \cref{ex:5.4.28} we see that
  \[
    [\T]_{\beta} = \begin{pmatrix}
      [\T_{\W}]_{\gamma} & \zm_1                    \\
      \zm_2              & [\overline{\T}]_{\alpha}
    \end{pmatrix}
  \]
  where \(\seq{\zm}{1,2}\) are zero matrix.
  Thus \([\overline{\T}]_{\alpha}\) is a diagonal matrix and by \cref{5.1.1} \(\overline{\T}\) is diagonalizable.
\end{proof}

\begin{ex}\label{ex:5.4.30}
  Let \(\V\) be a finite-dimensional vector space over \(\F\), let \(\T \in \ls(\V)\) and let \(\W\) be a nonzero \(\T\)-invariant subspace of \(\V\) over \(\F\).
  Prove that if both \(\T_{\W}\) and \(\overline{\T}\) are diagonalizable and have no common eigenvalues, then \(\T\) is diagonalizable.
\end{ex}

\begin{proof}[\pf{ex:5.4.30}]
  Since \(\T_{\W}\) is diagonalizable, by \cref{5.1.1} there exists an ordered basis \(\gamma = \set{\seq{v}{1,,k}}\) for \(\W\) over \(\F\) such that \([\T_{\W}]_{\gamma}\) is a diagonal matrix.
  Similarly, there exists an ordered basis \(\alpha' = \set{v_{k + 1} + \W, \dots, v_n + \W}\) for \(\V / \W\) over \(\F\) such that \([\overline{\T}]_{\alpha'}\) is a diagonal matrix.
  Let \(\alpha = \set{\seq{v}{k+1,,n}}\).
  We fix one such \(\gamma\) and \(\alpha\).

  Now we find an ordered basis for \(\V\) over \(\F\) consist of eigenvectors of \(\T\).
  For each \(i \in \set{1, \dots, k}\), let \(\lambda_i\) be the eigenvalue of \(v_i\).
  For each \(j \in \set{k + 1, \dots, n}\), let \(\lambda_j\) be the eigenvalue of \(v_j + \W\).
  Since
  \begin{align*}
             & \forall v_j \in \alpha, \overline{\T}(v_j + \W) = \lambda_j (v_j + \W)                                                 &  & \text{(by \cref{5.1.2})}        \\
    \implies & \forall v_j \in \alpha, \T(v_j) + \W = (\lambda_j v_j) + \W                                                            &  & \text{(by \cref{5.4.6})}        \\
    \implies & \forall v_j \in \alpha, \T(v_j) - \lambda_j v_j \in \W = \spn{\gamma}                                                  &  & \text{(by \cref{ex:1.3.31}(b))} \\
    \implies & \forall v_j \in \alpha, \exists a_{j 1}, \dots, a_{j k} \in \F : \T(v_j) - \lambda_j v_j = \sum_{i = 1}^k a_{j i} v_i, &  & \text{(by \cref{1.4.3})}
  \end{align*}
  if we define
  \[
    \forall j \in \set{k + 1, \dots, n}, v_j' = v_j + \sum_{i = 1}^k \frac{a_{j i}}{\lambda_j - \lambda_i} v_i,
  \]
  (note that \(\lambda_j - \lambda_i \neq 0\) by hypothesis)
  then we have
  \begin{align*}
    \T(v_j') & = \T\pa{v_j + \sum_{i = 1}^k \frac{a_{j i}}{\lambda_j - \lambda_i} v_i}                                                                            \\
             & = \T(v_j) + \sum_{i = 1}^k \frac{a_{j i}}{\lambda_j - \lambda_i} \T(v_i)                                          &  & \text{(by \cref{2.1.2}(d))} \\
             & = \lambda_j v_j + \sum_{i = 1}^k a_{j i} v_i + \sum_{i = 1}^k \frac{a_{j i} \lambda_i}{\lambda_j - \lambda_i} v_i &  & \text{(by \cref{5.1.2})}    \\
             & = \lambda_j v_j + \sum_{i = 1}^k \frac{a_{j i} \lambda_j}{\lambda_j - \lambda_i} v_i                                                               \\
             & = \lambda_j \pa{v_j + \sum_{i = 1}^k \frac{a_{j i}}{\lambda_j - \lambda_i} v_i}                                                                    \\
             & = \lambda_j v_j'.
  \end{align*}
  Thus the set \(\omega = \set{v_{k + 1}', \dots, v_n'}\) is consist of eigenvectors of \(\T\).
  Now we show that \(\beta = \gamma \cup \omega\) is an ordered basis for \(\V\) over \(\F\) which is consist of eigenvectors of \(\T\).
  Since \(\gamma\) is consist of eigenvectors of \(\T_{\W}\), by \cref{b.0.4} we know that \(\gamma\) is consist of eigenvectors of \(\T\).
  From the proof above we see that \(\beta\) is consist of eigenvectors of \(\T\).
  So we only need to show that \(\beta\) is an ordered basis for \(\V\) over \(\F\).

  We start by showing that \(\spn{\beta} = \V\).
  Clearly we have \(\spn{\beta} \subseteq \V\), so we show that \(\V \subseteq \spn{\beta}\).
  Let \(x \in \V\).
  Then \(x + \W \in \V / \W = \spn{\alpha'}\) and there exist some \(\seq{b}{k+1,,n} \in \F\) such that
  \begin{align*}
    x + \W & = \sum_{j = k + 1}^n b_j (v_j + \W)                                                                                                                        \\
           & = \pa{\sum_{j = k + 1}^n b_j v_j} + \W                                                                  &  & \text{(by \cref{ex:1.3.31}(c))}               \\
           & = \pa{\sum_{j = k + 1}^n b_j \pa{v_j' - \sum_{i = 1}^k \frac{a_{j i}}{\lambda_j - \lambda_i} v_i}} + \W                                                    \\
           & = \pa{\sum_{j = k + 1}^n b_j v_j'} + \W.                                                                &  & (\forall i \in \set{1, \dots, k}, v_i \in \W)
  \end{align*}
  By \cref{ex:1.3.31}(b) we know that \(x - \sum_{j = k + 1}^n b_j v_j' \in \W\).
  Thus there exist some \(\seq{b}{1,,k} \in \F\) such that
  \[
    \sum_{i = 1}^k b_i v_i = x - \sum_{j = k + 1}^n b_j v_j.
  \]
  Therefore \(x \in \spn{\beta}\).
  Since \(x\) is arbitrary, we have \(\V \subseteq \spn{\beta}\) and thus \(\V = \spn{\beta}\).

  Now we show that \(\beta\) is linearly independent.
  Clearly \(\gamma\) is linearly independent.
  If we can show that \(\omega\) is linearly independent, then by \cref{5.8} (and the fact that \(\set{\lambda_i : i \in \set{1, \dots k}} \cap \set{\lambda_j : j \in \set{k + 1, \dots, n}} = \varnothing\)) we see that \(\beta\) is linearly independent.
  Let \(\seq{b}{k+1,,n} \in \F\) such that \(\sum_{j = k + 1}^n b_j v_j' = \zv\).
  Since
  \begin{align*}
    \zv + \W & = \pa{\sum_{j = k + 1}^n b_j v_j'} + \W                                                                                                                   \\
             & = \pa{\sum_{j = k + 1}^n b_j \pa{v_j + \sum_{i = 1}^n \frac{a_{j i}}{\lambda_j - \lambda_i} v_i}} + \W                                                    \\
             & = \sum_{j = k + 1}^n b_j v_j + \W                                                                      &  & (\forall i \in \set{1, \dots, k}, v_i \in \W) \\
             & = \sum_{j = k + 1}^n b_j (v_j + \W)                                                                    &  & \text{(by \cref{ex:1.3.31}(c))}
  \end{align*}
  implies \(\seq[=]{b}{k+1,,n} = 0\), by \cref{1.5.3} we know that \(\omega\) is linearly independent.
  Thus we conclude that \(\beta\) is an ordered basis for \(\V\) over \(\F\) consist of eigenvectors of \(\T\), therefore by \cref{5.1.1} \(\T\) is diagonalizable.
\end{proof}

\setcounter{ex}{31}
\begin{ex}\label{ex:5.4.32}
  Prove the converse to \cref{ex:5.2.9}(a):
  If the characteristic polynomial of \(\T\) splits, then there is an ordered basis \(\beta\) for \(\V\) over \(\F\) such that \([\T]_{\beta}\) is an upper triangular matrix.
\end{ex}

\begin{proof}[\pf{ex:5.4.32}]
  We use induction on \(n\).
  For \(n = 1\), since \(\dim(\V) = 1\), there exists some \(v \in \V \setminus \set{\zv}\) such that \(\beta = \set{v}\) is an ordered basis for \(\V\) over \(\F\).
  Clearly \([\T]_{\beta}\) is an upper triangular matrix, thus the base case holds.
  Suppose inductively that the statement is true for some \(n \geq 1\).
  We need to show that the statement is also true for \(n + 1\).
  So let \(\dim(\V) = n + 1\) and let \(\T \in \ls(\V)\) such that the characteristic polynomial of \(\T\) splits.
  Let \(f\) be the characteristic polynomial of \(\T\).
  Since \(f\) splits, by \cref{5.2.2} there exists some \(\lambda \in \F\) such that \(f(\lambda) = 0\).
  By \cref{5.7} we have \(\dim(\ns{\T - \lambda \IT[\V]}) \geq 1\), thus there exists a \(v_1 \in \V \setminus \set{\zv}\) such that \(\T(v_1) = \lambda v_1\).
  Let \(\W = \spn{\set{v_1}}\).
  Clearly \(\W\) is a \(\T\)-invariant subspace of \(\V\) over \(\F\).
  If we define \(\overline{\T}\) as in \cref{5.4.6} and let \(g\) and \(h\) be the characteristic polynomial of \(\T_{\W}\) and \(\overline{\T}\), then by \cref{ex:5.4.28} we know that \(f = g h\).
  Since \(f\) splits, we know that \(g, h\) splits.
  By \cref{ex:1.6.35}(b) we know that \(\dim(\V / \W) = \dim(\V) - \dim(\W) = n + 1 - 1 = n\), thus by induction hypothesis there exists an ordered basis \(\alpha = \set{v_2 + \W, \dots, v_{n + 1} + \W}\) for \(\V / \W\) over \(\F\) such that \([\overline{\T}]_{\alpha}\) is an upper triangular matrix.
  Now we claim that \(\beta = \set{\seq{v}{1,,n+1}}\) is an ordered basis for \(\V\) over \(\F\).
  Since
  \begin{align*}
             & \forall \seq{a}{2,,n+1} \in \F, \sum_{i = 2}^{n + 1} a_i v_i = \zv                                                        \\
    \implies & \zv + \W = \sum_{i = 2}^{n + 1} (a_i v_i) + \W = \sum_{i = 2}^{n + 1} a_i (v_i + \W) &  & \text{(by \cref{ex:1.3.31}(c))} \\
    \implies & \seq[=]{a}{2,,n+1} = 0,                                                              &  & \text{(by \cref{1.5.3})}
  \end{align*}
  we know that \(\set{\seq{v}{2,,n+1}}\) is linearly independent.
  Since
  \begin{align*}
             & \zv + \W \notin \alpha                                         &  & \text{(by \cref{1.5.2})}        \\
    \implies & \forall i \in \set{2, \dots, n + 1}, v_i + \W \neq \zv + \W    &  & \text{(by \cref{ex:1.3.31}(d))} \\
    \implies & \forall i \in \set{2, \dots, n + 1}, v_i - \zv = v_i \notin \W &  & \text{(by \cref{ex:1.3.31}(b))} \\
    \implies & \set{\seq{v}{2,,n+1}} \cap \W = \varnothing,
  \end{align*}
  by \cref{1.7} we know that \(\beta\) is linearly independent.
  Thus by \cref{1.6.15}(b) \(\beta\) is an ordered basis for \(\V\) over \(\F\).
  By \cref{ex:5.4.28} we see that
  \[
    [\T]_{\beta} = \begin{pmatrix}
      [\T_{\W}]_{\set{v_1}} & B                        \\
      \zm                   & [\overline{\T}]_{\alpha}
    \end{pmatrix}
  \]
  where \(B \in \ms{1}{n}{\F}\) and \(\zm \in \ms{n}{1}{\F}\) is a zero matrix.
  Since \([\overline{\T}]_{\alpha}\) is an upper triangular matrix, we see that \([\T]_{\beta}\) is also an upper triangular matrix.
  This closes the induction.
\end{proof}

\begin{ex}\label{ex:5.4.33}
  Let \(\T\) be a linear operator on a vector space \(\V\) over \(\F\), and let \(\seq{\W}{1,,k}\) be \(\T\)-invariant subspaces of \(\V\) over \(\F\).
  Prove that \(\seq[+]{\W}{1,,k}\) is also a \(\T\)-invariant subspace of \(\V\) over \(\F\).
\end{ex}

\begin{proof}[\pf{ex:5.4.33}]
  Let \(\sum_{i = 1}^k v_i \in \sum_{i = 1}^k \W_i\).
  Since \(v_i \in \W_i\) for each \(i \in \set{1, \dots, k}\), by \cref{5.4.1} we know that \(\T(v_i) \in \W_i\).
  Thus by \cref{2.1.2}(d) we have \(\T\pa{\sum_{i = 1} v_i} = \sum_{i = 1}^k \T(v_i) \in \sum_{i = 1}^k \W_i\).
  By \cref{5.4.1} this means \(\sum_{i = 1}^k \W_i\) is \(\T\)-invariant.
\end{proof}

\begin{ex}\label{ex:5.4.34}
  Give a direct proof of \cref{5.25} for the case \(k = 2\).
  (This result is used in the proof of \cref{5.24}.)
\end{ex}

\begin{proof}[\pf{ex:5.4.34}]
  Let \(\V\) be an \(n\)-dimensional vector space over \(\F\) and let \(\W_1, \W_2\) be nontrivial \(\T\)-invariant subspaces of \(\V\) over \(\F\) such that \(\V = \W_1 \oplus \W_2\).
  Let \(\beta_1 = \set{\seq{v}{1,,k}}, \beta_2 = \set{\seq{v}{k+1,,n}}\) be ordered basis for \(\W_1, \W_2\) over \(\F\), respectively.
  By \cref{5.10}(a)(d) we know that \(\beta = \beta_1 \cup \beta_2\) is an ordered basis for \(\V\) over \(\F\).
  Then we have
  \begin{align*}
             & \forall i \in \set{1, \dots, k}, \T(v_i) = \T_{\W_1}(v_i)                               &  & \text{(by \cref{5.4.1})} \\
             & = \sum_{j = 1}^k ([\T_{\W_1}]_{\beta_1})_{j i} v_j                                      &  & \text{(by \cref{2.2.4})} \\
             & = \sum_{j = 1}^k ([\T_{\W_1}]_{\beta_1})_{j i} v_j + \sum_{j = k + 1}^n 0 v_j           &  & \text{(by \cref{1.2.1})} \\
    \implies & \forall (i, j) \in \set{1, \dots, k} \times \set{1, \dots, n},                                                        \\
             & ([\T]_{\beta})_{j i} = \begin{dcases}
                                        ([\T_{\W_1}]_{\beta_1})_{j i} & \text{if } j \in \set{1, \dots, k}     \\
                                        0                             & \text{if } j \in \set{k + 1, \dots, n}
                                      \end{dcases}
  \end{align*}
  and
  \begin{align*}
             & \forall i \in \set{k + 1, \dots, n}, \T(v_i) = \T_{\W_2}(v_i)                         &  & \text{(by \cref{5.4.1})} \\
             & = \sum_{j = k + 1}^n ([\T_{\W_2}]_{\beta_2})_{j i} v_j                                &  & \text{(by \cref{2.2.4})} \\
             & = \sum_{j = 1}^k 0 v_j + \sum_{j = k + 1}^n ([\T_{\W_2}]_{\beta_2})_{j i} v_j         &  & \text{(by \cref{1.2.1})} \\
    \implies & \forall (i, j) \in \set{k + 1, \dots, n} \times \set{1, \dots, n},                                                  \\
             & ([\T]_{\beta})_{j i} = \begin{dcases}
                                        0                             & \text{if } j \in \set{1, \dots, k}     \\
                                        ([\T_{\W_1}]_{\beta_1})_{j i} & \text{if } j \in \set{k + 1, \dots, n}
                                      \end{dcases}.
  \end{align*}
  Thus by \cref{5.4.5} we see that
  \[
    [\T]_{\beta} = \begin{pmatrix}
      [\T_{\W_1}]_{\beta_1} & \zm_1                 \\
      \zm_2                 & [\T_{\W_2}]_{\beta_2}
    \end{pmatrix}
  \]
  where \(\zm_1 \in \ms{k}{(n - k)}{\F}\) and \(\zm_2 \in \ms{(n - k)}{k}{\F}\) are zero matrices.
\end{proof}

\setcounter{ex}{35}
\begin{ex}\label{ex:5.4.36}
  Let \(\T\) be a linear operator on a finite-dimensional vector space \(\V\) over \(\F\).
  Prove that \(\T\) is diagonalizable iff \(\V\) is the direct sum of one-dimensional \(\T\)-invariant subspaces.
\end{ex}

\begin{proof}[\pf{ex:5.4.36}]
  First suppose that \(\T\) is diagonalizable.
  By \cref{5.1.1} there exists an ordered basis \(\beta\) for \(\V\) over \(\F\) such that \([\T]_{\beta}\) is a diagonal matrix.
  For each \(v \in \beta\), we know that \(\spn{\set{v}}\) is a one-dimensional \(\T\)-invariant subspace of \(\V\) over \(\F\).
  Thus by \cref{5.10}(a)(d) we know that \(\V\) is a direct sum of one-dimensional \(\T\)-invariant subspaces.

  Now suppose that \(\V\) is the direct sum of one-dimensional \(\T\)-invariant subspaces.
  Let \(\seq{\W}{1,,n}\) be \(\T\)-invariant subspaces of \(\V\) over \(\F\) such that \(\V = \seq[\oplus]{\W}{1,,n}\).
  If we let \(\beta_i\) be an ordered basis for \(\W_i\) over \(\F\) for each \(i \in \set{1, \dots, n}\), then by \cref{5.10}(a)(d) we know that \(\beta = \bigcup_{i = 1}^n \beta_i\) is an ordered basis for \(\V\) over \(\F\).
  By \cref{5.25} we have \([\T]_{\beta} = [\T_{\W_1}]_{\beta_1} \oplus \cdots \oplus [\T_{\W_n}]_{\beta_n}\).
  Therefore by \cref{5.4.5} \([\T]_{\beta}\) is a diagonal matrix and by \cref{5.1.1} \(\T\) is diagonalizable.
\end{proof}

\begin{ex}\label{ex:5.4.37}
  Let \(\T\) be a linear operator on a finite-dimensional vector space \(\V\) over \(\F\), and let \(\seq{\W}{1,,k}\) be \(\T\)-invariant subspaces of \(\V\) such that \(\V = \seq[\oplus]{\W}{1,,k}\).
  Prove that
  \[
    \det(\T) = \det(\T_{\W_1}) \cdots \det(\T_{\W_k}).
  \]
\end{ex}

\begin{proof}[\pf{ex:5.4.37}]
  For each \(i \in \set{1, \dots, k}\), let \(\beta_i\) be an ordered basis for \(\W_i\) over \(\F\).
  Let \(\beta = \bigcup_{i = 1}^k \beta_i\).
  Then we have
  \begin{align*}
    \det(\T) & = \det([\T]_{\beta})                                                                            &  & \text{(by \cref{ex:5.1.7})}  \\
             & = \det\begin{pmatrix}
                       [\T_{\W_1}]_{\beta_1} & \zm                   & \cdots & \zm                   \\
                       \zm                   & [\T_{\W_2}]_{\beta_2} & \cdots & \zm                   \\
                       \vdots                & \vdots                &        & \vdots                \\
                       \zm                   & \zm                   & \cdots & [\T_{\W_k}]_{\beta_k}
                     \end{pmatrix} &  & \text{(by \cref{5.25})}                                \\
             & = \det([\T_{\W_1}]_{\beta_1}) \cdots \det([\T_{\W_k}]_{\beta_k})                                &  & \text{(by \cref{ex:4.3.21})} \\
             & = \det(\T_{\W_1}) \cdot \det(\T_{\W_k}).                                                        &  & \text{(by \cref{ex:5.1.7})}
  \end{align*}
\end{proof}

\begin{ex}\label{ex:5.4.38}
  Let \(\T\) be a linear operator on a finite-dimensional vector space \(\V\) over \(\F\), and let \(\seq{\W}{1,,k}\) be \(\T\)-invariant subspaces of \(\V\) over \(\F\) such that \(\V = \seq[\oplus]{\W}{1,,k}\).
  Prove that \(\T\) is diagonalizable iff \(\T_{\W_i}\) is diagonalizable for all \(i \in \set{1, \dots, k}\).
\end{ex}

\begin{proof}[\pf{ex:5.4.38}]
  If \(\T\) is diagonalizable, then by \cref{ex:5.4.24} \(\T_{\W_i}\) is diagonalizable for all \(i \in \set{1, \dots, k}\).
  So suppose that \(\T_{\W_i}\) is diagonalizable for all \(i \in \set{1, \dots, k}\).
  By \cref{5.1.1}, for each \(i \in \set{1, \dots, k}\), there exists an ordered basis \(\beta_i\) for \(\W_i\) over \(\F\) such that \([\T_{\W_i}]_{\beta_i}\) is an diagonal matrix.
  By \cref{5.10}(a)(d) we know that \(\beta = \bigcup_{i = 1}^k \beta_i\) is an ordered basis for \(\V\) over \(\F\).
  Thus by \cref{5.25} we see that \([\T]_{\beta}\) is a diagonal matrix and by \cref{5.1.1} \(\T\) is diagonalizable.
\end{proof}

\begin{ex}\label{ex:5.4.39}
  Let \(\mathcal{C}\) be a collection of diagonalizable linear operators on a finite-dimensional vector space \(\V\).
  Prove that there is an ordered basis \(\beta\) such that \([\T]_{\beta}\) is a diagonal matrix for all \(\T \in \mathcal{C}\) iff the operators of \(\mathcal{C}\) commute under composition.
  (This is an extension of \cref{ex:5.4.25}.)
\end{ex}

\begin{proof}[\pf{ex:5.4.39}]
  First suppose that every \(\T \in \mathcal{C}\) are simultaneously diagonalizable.
  Then by \cref{ex:5.2.18}(a) we see that operators of \(\mathcal{C}\) commute under composition.

  Now suppose that operators of \(\mathcal{C}\) commute under composition.
  Let \(k = \dim(\V)\).
  We use induction on \(k\) to show that every \(\T \in \mathcal{C}\) are simultaneously diagonalizable.
  For \(k = 1\), for any ordered basis \(\beta\) for \(\V\) over \(\F\), \([\T]_{\beta}\) is a diagonal matrix for every \(\T \in \mathcal{C}\).
  Thus the base case holds.
  Suppose inductively that the statement is true for some \(k \geq 1\).
  We need to show that it is true for \(k + 1\).
  Let \(\V\) be a vector space over \(\F\) such that \(\dim(\V) = k + 1\).
  Let \(\mathcal{C}\) be a collection of diagonalizable linear operators on \(\V\).
  By \cref{5.11} we know that \(\V\) is a direct sum of eigenspaces of any \(\T \in \mathcal{C}\).
  If every \(\T \in \mathcal{C}\) has only one eigenvalue, then every \(\T \in \mathcal{C}\) are simultaneously diagonalizable with any ordered basis for \(\V\) over \(\F\) and we are done.
  So suppose that there exists a \(\T \in \mathcal{C}\) which has at least two eigenvalues.
  Fix one such \(\T\) and let \(\seq{\W}{1,,m}\) be eigenspaces of \(\T\) such that \(m \geq 2\) and \(\V = \seq[\oplus]{\W}{1,,m}\).
  We define
  \[
    \forall i \in \set{1, \dots, m}, \mathcal{C}_i = \set{\U_{\W_i} : \U \in \mathcal{C}}.
  \]
  Clearly \(\T_{\W_i} \in \mathcal{C}_i\) and \(\U_{\W_i} \T_{\W_i} = \T_{\W_i} \U_{\W_i}\) for all \(i \in \set{1, \dots, m}\) and \(\U_{\W_i} \in \mathcal{C}_i\).
  By \cref{5.4.2}(e) we know that \(\W_i\) is a \(\T\)-invariant subspace of \(\V\) over \(\F\) for all \(i \in \set{1, \dots, m}\), thus by \cref{ex:5.4.25}(a) we know that every \(\U_{\W_i} \in \mathcal{C}_i\) is diagonalizable for all \(i \in \set{1, \dots, m}\).
  Since \(m \geq 2\), by \cref{5.7} we know that \(\dim(\W_i) \leq k\) for all \(i \in \set{1, \dots, m}\).
  Thus by induction hypothesis we know that for each \(i \in \set{1, \dots, m}\), there exists an ordered basis for \(\W_i\) over \(\F\) such that \([\U_{\W_i}]_{\beta_i}\) is a diagonal matrix for all \(\U_{\W_i} \in \mathcal{C}_i\).
  Since \(\V = \seq[\oplus]{\W}{1,,m}\), by \cref{5.10}(a)(d) we know that \(\beta = \bigcup_{i = 1}^m \beta_i\) is an ordered basis for \(\V\).
  By \cref{5.25} this means \([\U]_{\beta}\) is a diagonal matrix for all \(\U \in \mathcal{C}\).
  Thus every \(\T \in \mathcal{C}\) are simultaneously diagonalizable, and this closes the induction.
\end{proof}

\begin{ex}\label{ex:5.4.40}
  Let \(\seq{B}{1,,k}\) be square matrices with entries in the same field, and let \(A = \seq[\oplus]{B}{1,,k}\).
  Prove that the characteristic polynomial of \(A\) is the product of the characteristic polynomials of the \(B_i\)'s.
\end{ex}

\begin{proof}[\pf{ex:5.4.40}]
  We have
  \begin{align*}
    \det(A - t I) & = \det\begin{pmatrix}
                            B_1 - t I & \zm       & \cdots & \zm       \\
                            \zm       & B_2 - t I & \cdots & \zm       \\
                            \vdots    & \vdots    &        & \vdots    \\
                            \zm       & \zm       & \cdots & B_k - t I
                          \end{pmatrix} &  & \text{(by \cref{5.4.5})}                               \\
                  & = \prod_{i = 1}^k \det(B_i - t I).            &  & \text{(by \cref{ex:4.3.21})}
  \end{align*}
\end{proof}

\begin{ex}\label{ex:5.4.41}
  Let
  \[
    A = \begin{pmatrix}
      1           & 2           & \cdots & n      \\
      n + 1       & n + 2       & \cdots & 2n     \\
      \vdots      & \vdots      &        & \vdots \\
      n^2 - n + 1 & n^2 - n + 2 & \cdots & n^2
    \end{pmatrix}.
  \]
  Find the characteristic polynomial of \(A\).
\end{ex}

\begin{proof}[\pf{ex:5.4.41}]
  Let \(B\) be the matrix obtained from \(A\) by subtracting the first row of \(A\) from each non-first row of \(A\), i.e.,
  \begin{align*}
    B & = \begin{pmatrix}
            1               & 2               & \cdots & n       \\
            n + 1 - 1       & n + 2 - 2       & \cdots & 2n - n  \\
            \vdots          & \vdots          &        & \vdots  \\
            n^2 - n + 1 - 1 & n^2 - n + 2 - 2 & \cdots & n^2 - n
          \end{pmatrix} \\
      & = \begin{pmatrix}
            1       & 2       & \cdots & n       \\
            n       & n       & \cdots & n       \\
            \vdots  & \vdots  &        & \vdots  \\
            n^2 - n & n^2 - n & \cdots & n^2 - n
          \end{pmatrix}.
  \end{align*}
  Let \(C\) be the matrix obtained from \(B\) by dividing each row of \(B\) by the first entry of each row, i.e.,
  \[
    C = \begin{pmatrix}
      \frac{1}{1}             & \frac{2}{1}             & \cdots & \frac{n}{1}             \\
      \frac{n}{n}             & \frac{n}{n}             & \cdots & \frac{n}{n}             \\
      \vdots                  & \vdots                  &        & \vdots                  \\
      \frac{n^2 - n}{n^2 - n} & \frac{n^2 - n}{n^2 - n} & \cdots & \frac{n^2 - n}{n^2 - n}
    \end{pmatrix} = \begin{pmatrix}
      1      & 2      & \cdots & n      \\
      1      & 1      & \cdots & 1      \\
      \vdots & \vdots &        & \vdots \\
      1      & 1      & \cdots & 1
    \end{pmatrix}.
  \]
  By \cref{3.2.5}(b) we see that \(\rk{C} = 2\).
  Since \(C\) is obtained from \(A\) be elementary row operations, by \cref{3.2.3} we see that \(\rk{A} = \rk{C} = 2\).
  By \cref{2.3} we see that
  \[
    \dim(\ns{\L_A - 0 \IT[\V]}) = \nt{\L_A} = \dim(\vs{F}^n) - \rk{\L_A} = n - \rk{A} = n - 2.
  \]
  If we can show that \(\rg{\L_A} \cap \ns{\L_A} = \set{\zv}\), then by \cref{ex:2.1.35}(a) we have \(\V = \rg{\L_A} \oplus \ns{\L_A}\).
  Observe that
  \begin{align*}
     & A \begin{pmatrix}
           1      \\
           1      \\
           \vdots \\
           1
         \end{pmatrix} = \begin{pmatrix}
                           \sum_{i = 1}^n i       \\
                           n^2 + \sum_{i = 1}^n i \\
                           \vdots                 \\
                           (n^2 - n) n + \sum_{i = 1}^n i
                         \end{pmatrix} = \begin{pmatrix}
                                           \frac{n(n + 1)}{2}       \\
                                           n^2 + \frac{n(n + 1)}{2} \\
                                           \vdots                   \\
                                           (n^2 - n) n + \frac{n(n + 1)}{2}
                                         \end{pmatrix}                  \\
     & = \begin{pmatrix}
           n^2 + \frac{n(n + 1)}{2} - n^2  \\
           2n^2 + \frac{n(n + 1)}{2} - n^2 \\
           \vdots                          \\
           n^3 + \frac{n(n + 1)}{2} - n^2
         \end{pmatrix} = n^2 \begin{pmatrix}
                               1      \\
                               2      \\
                               \vdots \\
                               n
                             \end{pmatrix} + \pa{\frac{n(n + 1)}{2} - n^2} \begin{pmatrix}
                                                                             1      \\
                                                                             1      \\
                                                                             \vdots \\
                                                                             1
                                                                           \end{pmatrix}
  \end{align*}
  and
  \begin{align*}
     & A \begin{pmatrix}
           1      \\
           2      \\
           \vdots \\
           n
         \end{pmatrix} = \begin{pmatrix}
                           \sum_{i = 1}^n i^2                     \\
                           \sum_{i = 1}^n in + \sum_{i = 1}^n i^2 \\
                           \vdots                                 \\
                           \sum_{i = 1}^n i(n^2 - n) + \sum_{i = 1}^n i^2
                         \end{pmatrix} = \begin{pmatrix}
                                           \frac{n(n + 1)(2n + 1)}{6}                             \\
                                           n \pa{\frac{n(n + 1)}{2}} + \frac{n(n + 1)(2n + 1)}{6} \\
                                           \vdots                                                 \\
                                           (n^2 - n) \pa{\frac{n(n + 1)}{2}} + \frac{n(n + 1)(2n + 1)}{6}
                                         \end{pmatrix}         \\
     & = \begin{pmatrix}
           \frac{n^2(n + 1)}{2} + \frac{n(n + 1)(2n + 1)}{6} - \frac{n^2(n + 1)}{2}        \\
           2 \pa{\frac{n^2(n + 1)}{2}} + \frac{n(n + 1)(2n + 1)}{6} - \frac{n^2(n + 1)}{2} \\
           \vdots                                                                          \\
           n \pa{\frac{n^2(n + 1)}{2}} + \frac{n(n + 1)(2n + 1)}{6} - \frac{n^2(n + 1)}{2}
         \end{pmatrix}                         \\
     & = \frac{n^2(n + 1)}{2} \begin{pmatrix}
                                1      \\
                                2      \\
                                \vdots \\
                                n
                              \end{pmatrix} + \pa{\frac{n(n + 1)(2n + 1)}{6} - \frac{n^2(n + 1)}{2}} \begin{pmatrix}
                                                                                                       1      \\
                                                                                                       1      \\
                                                                                                       \vdots \\
                                                                                                       1
                                                                                                     \end{pmatrix}.
  \end{align*}
  If we let
  \[
    \W = \spn{\begin{pmatrix}
        1      \\
        1      \\
        \vdots \\
        1
      \end{pmatrix}, \begin{pmatrix}
        1      \\
        2      \\
        \vdots \\
        n
      \end{pmatrix}},
  \]
  then from the observation above we know that \(\W\) is \(\L_A\)-invariant and \(\dim(\L_A(\W)) = 2\).
  Since \(\L_A(\W) \subseteq \W \subseteq \rg{\L_A}\) and \(\dim(\L_A(\W)) = 2 = \rk{\L_A}\), by \cref{1.11} we know that \(\W = \rg{\L_A}\).
  Clearly we have \(\W \cap \ns{\L_A} = \set{\zv}\), thus by \cref{ex:2.1.35} we have \(\V = \rg{\L_A} \oplus \ns{\L_A}\).
  Since \(\rg{\L_A}\) and \(\ns{\L_A}\) are \(\L_A\)-invariant (from the observation above and \cref{5.4.2}(d)), we have
  \begin{align*}
     & \det(A - t I_n)                                                                                                                                      \\
     & = \det(\L_A - t \IT[\V])                                                                                         &  & \text{(by \cref{ex:5.1.7}(a))} \\
     & = \det((\L_A)_{\rg{\L_A}} - t \IT[\rg{\L_A}]) \cdot \det((\L_A)_{\ns{\L_A}} - t \IT[\ns{\L_A}])                  &  & \text{(by \cref{ex:5.4.37})}   \\
     & = \det\begin{pmatrix}
               \frac{n(n + 1)}{2} - n^2 - t & \frac{n(n + 1)(2n + 1)}{6} - \frac{n^2 (n + 1)}{2} \\
               n^2                          & \frac{n^2 (n + 1)}{2} - t
             \end{pmatrix}                             &  & \text{(by the observation)}                                                              \\
     & \quad \cdot (-1)^{n - 2} t^{n - 2}                                                                               &  & \text{(by \cref{5.3}(a))}      \\
     & = \bigg(\frac{n^3 (n + 1)^2}{4} - \frac{n(n + 1)}{2} t - \frac{n^4 (n + 1)}{2} + n^2 t - \frac{n^2 (n + 1)}{2} t                                     \\
     & \quad + t^2 - \frac{n^3 (n + 1)(2n + 1)}{6} + \frac{n^4 (n + 1)}{2}\bigg) \cdot (-1)^{n - 2} t^{n - 2}                                               \\
     & = (-1)^{n - 2} t^{n - 2} \pa{\frac{-n^5 + n^3}{12} + \frac{-n^3 - n}{2} t + t^2}.
  \end{align*}
\end{proof}

\begin{ex}\label{ex:5.4.42}
  Let \(A \in \ms{n}{n}{\R}\) be the matrix defined by \(A_{i j} = 1\) for all \(i, j \in \set{1, \dots, n}\).
  Find the characteristic polynomial of \(A\).
\end{ex}

\begin{proof}[\pf{ex:5.4.42}]
  Since \(\rk{A} = 1\), by \cref{2.3} we know that
  \[
    \nt{\L_A} = \dim(\vs{F}^n) - \rk{\L_A} = n - \rk{A} = n - 1.
  \]
  Since the sum of columns of \(A - n I_n\) is \(\zv\), we know that \(\rk{A - n I_n} < n\) and by \cref{4.3.1} \(\det(A - n I_n) = 0\).
  Thus by \cref{5.7} we have \(\det(A - t I_n) = (-1)^n t^{n - 1} (t - n)\).
\end{proof}


\chapter{Inner Product Spaces}\label{ch:6}

\begin{note}
  Except for \cref{sec:6.8}, we assume that all vector spaces in \cref{ch:6} are over the field \(\F\), where \(\F\) denotes either \(\R\) or \(\C\).
\end{note}

% All sections are in separated files.  We include them here.
\section{Inner Products and Norms}\label{sec:6.1}

\begin{defn}\label{6.1.1}
	Let \(\V\) be a vector space over \(\F\).
	An \textbf{inner product} on \(\V\) over \(\F\) is a function that assigns, to every ordered pair of vectors \(x\) and \(y\) in \(\V\), a scalar in \(\F\), denoted \(\inn{x, y}\), such that for all \(x\), \(y\), and \(z\) in \(\V\) and all \(c\) in \(\F\), the following hold:
	\begin{enumerate}
		\item \(\inn{x + z, y} = \inn{x, y} + \inn{z, y}\).
		\item \(\inn{cx, y} = c \inn{x, y}\).
		\item \(\conj{\inn{x, y}} = \inn{y, x}\), where the bar denotes complex conjugation.
		\item \(\inn{x, x} > 0\) if \(x \neq \zv\).
	\end{enumerate}
\end{defn}

\begin{note}
	Note that \cref{6.1.1}(c) reduces to \(\inn{x, y} = \inn{y, x}\) if \(\F = \R\).
	\cref{6.1.1}(a)(b) simply require that the inner product be linear in the first component.
	It is easily shown that if \(\seq{a}{1,,n} \in \F\) and \(y, \seq{v}{1,,n} \in \V\), then
	\[
		\inn{\sum_{i = 1}^n a_i v_i, y} = \sum_{i = 1}^n a_i \inn{v_i, y}.
	\]
\end{note}

\begin{eg}\label{6.1.2}
	For \(x = \tuple{a}{1,,n}\) and \(y = \tuple{b}{1,,n}\) in \(\vs{F}^n\), define
	\[
		\inn{x, y} = \sum_{i = 1}^n a_i \conj{b_i}.
	\]
	The inner product defined above is called the \textbf{standard inner product} on \(\vs{F}^n\).
	When \(\F = \R\) the conjugations are not needed, and in early courses this standard inner product is usually called the \emph{dot product} and is denoted by \(x \cdot y\) instead of \(\inn{x, y}\).
\end{eg}

\begin{proof}[\pf{6.1.2}]
	Let \(x, y, z \in \vs{F}^n\) and let \(c \in \F\).
	Since
	\begin{align*}
		\inn{x + y, z}    & = \sum_{i = 1}^n (x + y)_i \conj{z_i}                           &  & \by{6.1.2}    \\
		                  & = \sum_{i = 1}^n x_i \conj{z_i} + \sum_{i = 1}^n y_i \conj{z_i} &  & \by{c.0.1}    \\
		                  & = \inn{x, z} + \inn{y, z}                                       &  & \by{6.1.2}    \\
		\inn{cx, y}       & = \sum_{i = 1}^n (cx)_i \conj{y_i}                              &  & \by{6.1.2}    \\
		                  & = c \sum_{i = 1}^n x_i \conj{y_i}                               &  & \by{c.0.1}    \\
		                  & = c \inn{x, y}                                                  &  & \by{6.1.2}    \\
		\conj{\inn{x, y}} & = \conj{\sum_{i = 1}^n x_i \conj{y_i}}                          &  & \by{6.1.2}    \\
		                  & = \sum_{i = 1}^n \conj{x_i} y_i                                 &  & \by{d.2}[abc] \\
		                  & = \sum_{i = 1}^n y_i \conj{x_i}                                 &  & \by{c.0.1}    \\
		                  & = \inn{y, x}                                                    &  & \by{6.1.2}
	\end{align*}
	and
	\begin{align*}
		         & x \neq \zv                                                                         \\
		\implies & \exists i \in \set{1, \dots, n} : x_i \neq 0                                       \\
		\implies & \exists i \in \set{1, \dots, n} : \abs{x_i}^2 = x_i \conj{x_i} > 0 &  & \by{d.0.5} \\
		\implies & \inn{x, x} = \sum_{i = 1}^n x_i \conj{x_i} > 0,                    &  & \by{6.1.2}
	\end{align*}
	by \cref{6.1.1} we know that \(\inn{\cdot, \cdot}\) is an inner product on \(\vs{F}^n\) over \(\F\).
\end{proof}

\begin{eg}\label{6.1.3}
	If \(\inn{x, y}\) is any inner product on a vector space \(\V\) over \(\F\) and \(r > 0\), we may define another inner product by the rule \(\inn{x, y}' = r \inn{x, y}\).
	If \(r \leq 0\), then \cref{6.1.1}(d) would not hold.
\end{eg}

\begin{proof}[\pf{6.1.3}]
	Let \(x, y, z \in \V\), let \(c \in \F\) and let \(r \in \R^+\).
	Since
	\begin{align*}
		\inn{x + y, z}'    & = r\inn{x + y, z}             &  & \by{6.1.3}    \\
		                   & = r (\inn{x, z} + \inn{y, z}) &  & \by{6.1.1}[a] \\
		                   & = r \inn{x, z} + r \inn{y, z} &  & \by{c.0.1}    \\
		                   & = \inn{x, z}' + \inn{y, z}'   &  & \by{6.1.3}    \\
		\inn{cx, y}'       & = r \inn{cx, y}               &  & \by{6.1.3}    \\
		                   & = rc \inn{x, y}               &  & \by{6.1.1}[b] \\
		                   & = cr \inn{x, y}               &  & \by{c.0.1}    \\
		                   & = c \inn{x, y}'               &  & \by{6.1.3}    \\
		\conj{\inn{x, y}'} & = \conj{r \inn{x, y}}         &  & \by{6.1.3}    \\
		                   & = \conj{r} \conj{\inn{x, y}}  &  & \by{d.2}[c]   \\
		                   & = \conj{r} \inn{y, x}         &  & \by{6.1.1}[c] \\
		                   & = r \inn{y, x}                &  & (r \in \R^+)  \\
		                   & = \inn{y, x}'                 &  & \by{6.1.3}
	\end{align*}
	and
	\begin{align*}
		         & \begin{dcases}
			           x \neq \zv \\
			           r > 0
		           \end{dcases}                                      \\
		\implies & \inn{x, x}' = r \inn{x, x} > 0, &  & \by{6.1.1}[d]
	\end{align*}
	by \cref{6.1.1} we see that \(\inn{\cdot, \cdot}'\) is an inner product on \(\V\) over \(\F\).
\end{proof}

\begin{eg}\label{6.1.4}
	Let \(\V = \cfs([0, 1], \R)\), the vector space of real-valued continuous functions on \([0, 1]\).
	For \(f, g \in \V\), define \(\inn{f, g} = \int_0^1 f(t) g(t) \; dt\).
	Since the preceding integral is linear in \(f\), \cref{6.1.1}(a)(b) are immediate, and \cref{6.1.1}(c) is trivial.
	If \(f \neq \zv\), then \(f^2\) is bounded away from zero on some subinterval of \([0, 1]\) (continuity is used here), and hence \(\inn{f, f} = \int_0^1 f^2(t) \; dt > 0\).
\end{eg}

\begin{defn}\label{6.1.5}
	Let \(A \in \MS\).
	We define the \textbf{conjugate transpose} or \textbf{adjoint} of \(A\) to be the \(n \times m\) matrix \(A^*\) such that \((A^*)_{i j} = \conj{A_{j i}}\) for all \(i \in \set{1, \dots, m}\) and \(j \in \set{1, \dots, n}\).
\end{defn}

\begin{note}
	If \(x\) and \(y\) are viewed as column vectors in \(\vs{F}^n\), then \(\inn{x, y} = y^* x\) where \(\inn{\cdot, \cdot}\) is the standard inner product.
	The conjugate transpose of a matrix plays a very important role in the remainder of \cref{ch:6}.
	In the case that \(A\) has real entries, \(A^*\) is simply the transpose of \(A\).
\end{note}

\begin{eg}\label{6.1.6}
	Let \(\V = \ms{n}{n}{\F}\), and define \(\inn{A, B} = \tr(B^* A)\) for \(A, B \in \V\).
	Then \(\inn{\cdot, \cdot}\) is called the \textbf{Frobenius inner product} and is an inner product on \(\V\).
\end{eg}

\begin{proof}[\pf{6.1.6}]
	Let \(A, B, C \in \V\) and let \(k \in \F\).
	Then
	\begin{align*}
		\inn{A + B, C}    & = \tr(C^* (A + B))                                                      &  & \by{6.1.6}    \\
		                  & = \tr(C^* A + C^* B)                                                    &  & \by{2.3.5}    \\
		                  & = \tr(C^* A) + \tr(C^* B)                                               &  & \by{ex:1.3.6} \\
		                  & = \inn{A, C} + \inn{B, C}                                               &  & \by{6.1.6}    \\
		\inn{kA, B}       & = \tr(B^* (kA))                                                         &  & \by{6.1.6}    \\
		                  & = k \tr(B^* A)                                                          &  & \by{ex:1.3.6} \\
		                  & = k \inn{A, B}                                                          &  & \by{6.1.6}    \\
		\conj{\inn{A, B}} & = \conj{\tr(B^* A)}                                                     &  & \by{6.1.6}    \\
		                  & = \conj{\sum_{i = 1}^n (B^* A)_{i i}}                                   &  & \by{1.3.9}    \\
		                  & = \conj{\sum_{i = 1}^n \sum_{j = 1}^n (B^*)_{i j} \cdot A_{j i}}        &  & \by{2.3.1}    \\
		                  & = \sum_{i = 1}^n \sum_{j = 1}^n \conj{(B^*)_{i j}} \cdot \conj{A_{j i}} &  & \by{d.2}[bc]  \\
		                  & = \sum_{i = 1}^n \sum_{j = 1}^n B_{j i} \cdot (A^*)_{i j}               &  & \by{6.1.5}    \\
		                  & = \sum_{i = 1}^n (A^* B)_{i i}                                          &  & \by{2.3.1}    \\
		                  & = \tr(A^* B)                                                            &  & \by{1.3.9}    \\
		                  & = \inn{B, A}.                                                           &  & \by{6.1.6}
	\end{align*}
	Also
	\begin{align*}
		\inn{A, A} & = \tr(A^* A)                                           &  & \by{6.1.6} \\
		           & = \sum_{i = 1}^n (A^* A)_{i i}                         &  & \by{1.3.9} \\
		           & = \sum_{i = 1}^n \sum_{k = 1}^n (A^*)_{i k} A_{k i}    &  & \by{2.3.1} \\
		           & = \sum_{i = 1}^n \sum_{k = 1}^n \conj{A_{k i}} A_{k i} &  & \by{6.1.5} \\
		           & = \sum_{i = 1}^n \sum_{k = 1}^n \abs{A_{k i}}^2.       &  & \by{d.0.5}
	\end{align*}
	Now if \(A \neq \zm\), then \(A_{k i} \neq 0\) for some \(i, k \in \set{1, \dots, n}\).
	So \(\inn{A, A} > 0\).
	Thus by \cref{6.1.1} we see that \(\inn{\cdot, \cdot}\) is an inner product on \(\V\).
\end{proof}

\begin{defn}\label{6.1.7}
	A vector space \(\V\) over \(\F\) endowed with a specific inner product is called an \textbf{inner product space}.
	If \(\F = \C\), we call \(\V\) a \textbf{complex inner product space}, whereas if \(\F = \R\), we call \(\V\) a \textbf{real inner product space}.

	It is clear that if \(\V\) has an inner product \(\inn{x, y}\) and \(\W\) is a subspace of \(\V\) over \(\F\), then \(\W\) is also an inner product space when the same function \(\inn{x, y}\) is restricted to the vectors \(x, y \in \W\).
\end{defn}

\begin{note}
	For the remainder of \cref{ch:6}, \(\vs{F}^n\) denotes the inner product space with the standard inner product as defined in \cref{6.1.2}.
	Likewise, \(\ms{n}{n}{\F}\) denotes the inner product space with the Frobenius inner product as defined in \cref{6.1.6}.
\end{note}

\begin{eg}\label{6.1.8}
	Let \(\vs{H} = \cfs([0, 2 \pi], \C)\).
	For \(f, g \in \vs{H}\), define
	\[
		\inn{f, g} = \frac{1}{2 \pi} \int_0^{2 \pi} f(t) \conj{g(t)} \; dt.
	\]
	This inner product space, which arises often in the context of physical situations, is examined more closely in later sections.
\end{eg}

\begin{proof}
	Let \(f, g, h \in \vs{H}\) and let \(c \in \C\).
	Then
	\begin{align*}
		\inn{f + g, h}    & = \frac{1}{2 \pi} \int_0^{2 \pi} (f + g)(t) \conj{h(t)} \; dt                                                   &  & \by{6.1.8}  \\
		                  & = \frac{1}{2 \pi} \int_0^{2 \pi} f(t) \conj{h(t)} \; dt + \frac{1}{2 \pi} \int_0^{2 \pi} g(t) \conj{h(t)} \; dt                  \\
		                  & = \inn{f, h} + \inn{g, h}                                                                                       &  & \by{6.1.8}  \\
		\inn{cf, g}       & = \frac{1}{2 \pi} \int_0^{2 \pi} (cf)(t) \conj{g(t)} \; dt                                                      &  & \by{6.1.8}  \\
		                  & = \frac{c}{2 \pi} \int_0^{2 \pi} f(t) \conj{g(t)} \; dt                                                                          \\
		                  & = c \inn{f, g}                                                                                                                   \\
		\conj{\inn{f, g}} & = \conj{\frac{1}{2 \pi} \int_0^{2 \pi} f(t) \conj{g(t)} \; dt}                                                  &  & \by{6.1.8}  \\
		                  & = \frac{1}{2 \pi} \int_0^{2 \pi} \conj{f(t)} g(t) \; dt                                                         &  & \by{d.2}[c] \\
		                  & = \inn{g, f}.                                                                                                   &  & \by{6.1.8}
	\end{align*}
	Also, if \(f \neq \zv\), then we have
	\begin{align*}
		\inn{f, f} & = \frac{1}{2 \pi} \int_0^{2 \pi} f(t) \conj{f(t)} \; dt &  & \by{6.1.8}   \\
		           & = \frac{1}{2 \pi} \int_0^{2 \pi} \abs{f(t)}^2 \; dt     &  & \by{d.0.5}   \\
		           & > 0.                                                    &  & (f \neq \zv)
	\end{align*}
	Thus by \cref{6.1.1} \(\inn{\cdot, \cdot}\) is an inner product on \(\vs{H}\).
\end{proof}

\begin{thm}\label{6.1}
	Let \(\V\) be an inner product space over \(\F\).
	Then for \(x, y, z \in \V\) and \(c \in \F\), the following statements are true.
	\begin{enumerate}
		\item \(\inn{x, y + z} = \inn{x, y} + \inn{x, z}\).
		\item \(\inn{x, cy} = \conj{c} \inn{x, y}\).
		\item \(\inn{x, \zv} = \inn{\zv, x} = 0\).
		\item \(\inn{x, x} = 0\) iff \(x = \zv\).
		\item If \(\inn{x, y} = \inn{x, z}\) for all \(x \in \V\), then \(y = z\).
	\end{enumerate}
\end{thm}

\begin{proof}[\pf{6.1}(a)]
	We have
	\begin{align*}
		\inn{x, y + z} & = \conj{\inn{y + z, x}}                 &  & \by{6.1.1}[c] \\
		               & = \conj{\inn{y, x} + \inn{z, x}}        &  & \by{6.1.1}[a] \\
		               & = \conj{\inn{y, x}} + \conj{\inn{z, x}} &  & \by{d.2}[b]   \\
		               & = \inn{x, y} + \inn{x, z}.              &  & \by{6.1.1}[c]
	\end{align*}
\end{proof}

\begin{proof}[\pf{6.1}(b)]
	We have
	\begin{align*}
		\inn{x, cy} & = \conj{\inn{cy, x}}         &  & \by{6.1.1}[c] \\
		            & = \conj{c \inn{y, x}}        &  & \by{6.1.1}[b] \\
		            & = \conj{c} \conj{\inn{y, x}} &  & \by{d.2}[c]   \\
		            & = \conj{c} \inn{x, y}        &  & \by{6.1.1}[c]
	\end{align*}
\end{proof}

\begin{proof}[\pf{6.1}(c)]
	We have
	\begin{align*}
		\inn{x, \zv} & = \inn{x, \zv + \zv}          &  & \by{1.2.1}    \\
		             & = \inn{x, \zv} + \inn{x, \zv} &  & \by{6.1}[a]   \\
		\inn{\zv, x} & = \inn{\zv + \zv, x}          &  & \by{1.2.1}    \\
		             & = \inn{\zv, x} + \inn{\zv, x} &  & \by{6.1.1}[a]
	\end{align*}
	and thus \(\inn{x, \zv} = \inn{\zv, x} = 0\).
\end{proof}

\begin{proof}[\pf{6.1}(d)]
	By \cref{6.1.1}(d) and \cref{6.1}(c) we have \(\inn{x, x} = 0 \iff x = \zv\).
\end{proof}

\begin{proof}[\pf{6.1}(e)]
	Since
	\begin{align*}
		\inn{y - z, y - z} & = \inn{y, y - z} - \inn{z, y - z}                   &  & \by{6.1.1}[ab]                              \\
		                   & = \inn{y, y} - \inn{y, z} - \inn{z, y} + \inn{z, z} &  & \by{6.1}[ab]                                \\
		                   & = \inn{z, y} - \inn{y, z} - \inn{z, y} + \inn{y, z} &  & (\forall x \in \V, \inn{x, y} = \inn{x, z}) \\
		                   & = 0,
	\end{align*}
	by \cref{6.1}(d) we know that \(y - z = \zv\), thus \(y = z\).
\end{proof}

\begin{note}
	The reader should observe that \cref{6.1}(a)(b) show that the inner product is \textbf{conjugate linear} in the second component.
\end{note}

\begin{defn}\label{6.1.9}
	Let \(\V\) be an inner product space over \(\F\).
	For \(x \in \V\), we define the \textbf{norm} or \textbf{length} of \(x\) by \(\norm{x} = \sqrt{\inn{x, x}}\).
\end{defn}

\begin{eg}\label{6.1.10}
	Let \(\V = \vs{F}^n\).
	If \(x = \tuple{a}{1,,n}\), then
	\[
		\norm{x} = \norm{\tuple{a}{1,,n}} = \pa{\sum_{i = 1}^n \abs{a_i}^2}^{\frac{1}{2}}
	\]
	is the Euclidean definition of length.
	Note that if \(n = 1\), we have \(\norm{a} = \abs{a}\).
\end{eg}

\begin{thm}\label{6.2}
	Let \(\V\) be an inner product space over \(\F\).
	Then for all \(x, y \in \V\) and \(c \in \F\), the following statements are true.
	\begin{enumerate}
		\item \(\norm{cx} = \abs{c} \cdot \norm{x}\).
		\item \(\norm{x} = 0\) iff \(x = \zv\).
		      In any case \(\norm{x} \geq 0\).
		\item (Cauchy--Schwarz Inequality)
		      \(\abs{\inn{x, y}} \leq \norm{x} \cdot \norm{y}\).
		\item (Triangle Inequality)
		      \(\norm{x + y} \leq \norm{x} + \norm{y}\).
	\end{enumerate}
\end{thm}

\begin{proof}[\pf{6.2}(a)]
	We have
	\begin{align*}
		\norm{cx} & = \sqrt{\inn{cx, cx}}          &  & \by{6.1.9}    \\
		          & = \sqrt{c \inn{x, cx}}         &  & \by{6.1.1}[b] \\
		          & = \sqrt{c \conj{c} \inn{x, x}} &  & \by{6.1}[b]   \\
		          & = \sqrt{\abs{c}^2 \inn{x, x}}  &  & \by{d.0.5}    \\
		          & = \abs{c} \sqrt{\inn{x, x}}                       \\
		          & = \abs{c} \norm{x}.            &  & \by{6.1.9}
	\end{align*}
\end{proof}

\begin{proof}[\pf{6.2}(b)]
	We have
	\begin{align*}
		     & x = \zv                                \\
		\iff & \inn{x, x} = 0        &  & \by{6.1}[d] \\
		\iff & \sqrt{\inn{x, x}} = 0                  \\
		\iff & \norm{x} = 0          &  & \by{6.1.9}
	\end{align*}
	and
	\begin{align*}
		         & \forall x \in \V, \begin{dcases}
			                             \inn{x, x} = 0 & \text{if } x = \zv    \\
			                             \inn{x, x} > 0 & \text{if } x \neq \zv
		                             \end{dcases}  &  & \text{(by \cref{6.1.1}(d) and \cref{6.1}(d))} \\
		\implies & \forall x \in \V, \inn{x, x} \geq 0                                                \\
		\implies & \forall x \in \V, \sqrt{\inn{x, x}} \geq 0                                         \\
		\implies & \forall x \in \V, \norm{x} \geq 0.         &  & \by{6.1.9}
	\end{align*}
\end{proof}

\begin{proof}[\pf{6.2}(c)]
	If \(y = \zv\), then the result is immediate.
	So assume that \(y \neq \zv\).
	For any \(c \in \F\), we have
	\begin{align*}
		0 & \leq \norm{x - cy}^2                                                       &  & \by{6.2}[b]    \\
		  & = \inn{x - cy, x - cy}                                                     &  & \by{6.1.9}     \\
		  & = \inn{x, x - cy} - c \inn{y, x - cy}                                      &  & \by{6.1.1}[ab] \\
		  & = \inn{x, x} - \conj{c} \inn{x, y} - c \inn{y, x} + c \conj{c} \inn{y, y}. &  & \by{6.1}[ab]
	\end{align*}
	In particular, if we set
	\[
		c = \frac{\inn{x, y}}{\inn{y, y}},
	\]
	the inequality becomes
	\[
		0 \leq \inn{x, x} - \frac{\abs{\inn{x, y}}^2}{\inn{y, y}} = \norm{x}^2 - \frac{\abs{\inn{x, y}}^2}{\norm{y}^2},
	\]
	from which (c) follows.
\end{proof}

\begin{proof}[\pf{6.2}(d)]
	We have
	\begin{align*}
		 & \norm{x + y}^2                                                                                         \\
		 & = \inn{x + y, x + y}                                &  & \by{6.1.9}                                    \\
		 & = \inn{x, x} + \inn{y, x} + \inn{x, y} + \inn{y, y} &  & \text{(by \cref{6.1.1}(a) and \cref{6.1}(a))} \\
		 & = \norm{x}^2 + 2 \Re(\inn{x, y}) + \norm{y}^2       &  & \by{6.1.1}(d) and \cref{6.1.9}                \\
		 & \leq \norm{x}^2 + 2 \abs{\inn{x, y}} + \norm{y}^2   &  & \by{d.0.5}                                    \\
		 & \leq \norm{x}^2 + 2 \norm{x} \norm{y} + \norm{y}^2  &  & \by{6.2}[c]                                   \\
		 & = \pa{\norm{x} + \norm{y}}^2,
	\end{align*}
	where \(\Re(\inn{x, y})\) denotes the real part of the complex number \(\inn{x, y}\).
\end{proof}

\begin{eg}\label{6.1.11}
	For \(\vs{F}^n\), we may apply \cref{6.2}(c)(d) to the standard inner product to obtain the following well-known inequalities:
	\[
		\abs{\sum_{i = 1}^n a_i \conj{b_i}} \leq \pa{\sum_{i = 1}^n \abs{a_i}^2}^{\frac{1}{2}} \pa{\sum_{i = 1}^n \abs{b_i}^2}^{\frac{1}{2}}
	\]
	and
	\[
		\pa{\sum_{i = 1}^n \abs{a_i + b_i}^2}^{\frac{1}{2}} \leq \pa{\sum_{i = 1}^n \abs{a_i}^2}^{\frac{1}{2}} + \pa{\sum_{i = 1}^n \abs{b_i}^2}^{\frac{1}{2}}.
	\]
\end{eg}

\begin{note}
	The reader may recall from earlier courses that, for \(x\) and \(y\) in \(\R^3\) or \(\R^2\), we have that \(\inn{x, y} = \norm{x} \cdot \norm{y} \cos \theta\), where \(\theta \in [0, \pi]\) denotes the angle between \(x\) and \(y\).
	This equation implies \cref{6.2}(c) immediately since \(\abs{\cos \theta} \leq 1\).
	Notice also that nonzero vectors \(x\) and \(y\) are perpendicular iff \(\cos \theta = 0\), that is, iff \(\inn{x, y} = 0\).
\end{note}

\begin{defn}\label{6.1.12}
	Let \(\V\) be an inner product space over \(\F\).
	Vectors \(x\) and \(y\) in \(\V\) are \textbf{orthogonal} (\textbf{perpendicular}) if \(\inn{x, y} = 0\).
	A subset \(S\) of \(\V\) is \textbf{orthogonal} if any two distinct vectors in \(S\) are orthogonal.
	A vector \(x\) in \(\V\) is a \textbf{unit vector} if \(\norm{x} = 1\).
	Finally, a subset \(S\) of \(\V\) is \textbf{orthonormal} if \(S\) is orthogonal and consists entirely of unit vectors.

	Note that if \(S = \set{\seq{v}{1,2,}}\), then \(S\) is orthonormal iff \(\inn{v_i, v_j} = \delta_{i j}\), where \(\delta_{i j}\) denotes the Kronecker delta.
	Also, observe that multiplying vectors by nonzero scalars does not affect their orthogonality and that if \(x\) is any nonzero vector, then \((1 / \norm{x}) x\) is a unit vector.
	The process of multiplying a nonzero vector by the reciprocal of its length is called \textbf{normalizing}.
\end{defn}

\begin{eg}\label{6.1.13}
	Recall the inner product space \(\vs{H}\) (defined in \cref{6.1.8}).
	We introduce an important orthonormal subset \(S\) of \(\vs{H}\).
	For what follows, \(i\) is the imaginary number such that \(i^2 = -1\).
	For any integer \(n\), let \(f_n(t) = e^{int}\), where \(0 \leq t \leq 2 \pi\).
	(Recall that \(e^{int} = \cos(nt) + i \sin(nt)\).)
	Now define \(S = \set{f_n : n \in \Z}\).
	Clearly \(S\) is a subset of \(\vs{H}\).
	Using the property that \(\conj{e^{it}} = e^{-it}\) for every real number \(t\), we have, for \(m \neq n\),
	\begin{align*}
		\inn{f_m, f_n} & = \frac{1}{2 \pi} \int_0^{2 \pi} e^{imt} \conj{e^{int}} \; dt &  & \by{6.1.8} \\
		               & = \frac{1}{2 \pi} \int_0^{2 \pi} e^{i(m - n)t} \; dt                          \\
		               & = \eval{\frac{1}{2 \pi i(m - n)} e^{i(m - n)t}}_0^{2 \pi}                     \\
		               & = 0.
	\end{align*}
	Also,
	\begin{align*}
		\inn{f_n, f_n} & = \frac{1}{2 \pi} \int_0^{2 \pi} e^{i(n - n)t} \; dt &  & \by{6.1.8} \\
		               & = \frac{1}{2 \pi} \int_0^{2 \pi} 1 \; dt                             \\
		               & = 1.
	\end{align*}
	In other words, \(\inn{f_m, f_n} = \delta_{m n}\).
\end{eg}

\exercisesection

\setcounter{ex}{8}
\begin{ex}\label{ex:6.1.9}
	Let \(\beta\) be a basis for a \(n\)-dimensional inner product space \(\V\) over \(\F\).
	\begin{enumerate}
		\item Prove that if \(x \in \V\) and \(\inn{x, z} = 0\) for all \(z \in \beta\), then \(x = \zv\).
		\item Prove that if \(x \in \V\) and \(\inn{x, z} = \inn{y, z}\) for all \(z \in \beta\), then \(x = y\).
	\end{enumerate}
\end{ex}

\begin{proof}[\pf{ex:6.1.9}(a)]
	Let \(\beta = \set{\seq{v}{1,,n}}\).
	Then we have
	\begin{align*}
		         & x \in \V = \spn{\beta}                                                                                            \\
		\implies & \exists \seq{a}{1,,n} \in \F : x = \sum_{i = 1}^n a_i v_i                             &  & \by{1.6.1}             \\
		\implies & \inn{x, x} = \inn{x, \sum_{i = 1}^n a_i v_i} = \sum_{i = 1}^n \conj{a_i} \inn{x, v_i} &  & \by{6.1}[ab]           \\
		         & = \sum_{i = 1}^n \conj{a_i} 0 = 0                                                     &  & \text{(by hypothesis)} \\
		\implies & x = \zv.                                                                              &  & \by{6.1}[d]
	\end{align*}
\end{proof}

\begin{proof}[\pf{ex:6.1.9}(b)]
	We have
	\begin{align*}
		         & \forall z \in \beta, \inn{x, z} = \inn{y, z}                               \\
		\implies & \forall v \in \V = \spn{\beta}, \inn{x, v} = \inn{y, v} &  & \by{6.1}[ab]  \\
		\implies & \forall v \in \V = \spn{\beta}, \inn{v, x} = \inn{v, y} &  & \by{6.1.1}[c] \\
		\implies & x = y.                                                  &  & \by{6.1}[e]
	\end{align*}
\end{proof}

\begin{ex}\label{ex:6.1.10}
	Let \(\V\) be an inner product space over \(\F\), and suppose that \(x\) and \(y\) are orthogonal vectors in \(\V\).
	Prove that \(\norm{x + y}^2 = \norm{x}^2 + \norm{y}^2\).
	Deduce the Pythagorean theorem in \(\R^2\).
\end{ex}

\begin{proof}[\pf{ex:6.1.10}]
	We have
	\begin{align*}
		 & \norm{x + y}^2                                                                                         \\
		 & = \inn{x + y, x + y}                                &  & \by{6.1.9}                                    \\
		 & = \inn{x, x} + \inn{y, x} + \inn{x, y} + \inn{y, y} &  & \text{(by \cref{6.1.1}(a) and \cref{6.1}(a))} \\
		 & = \inn{x, x} + \inn{y, y}                           &  & \by{6.1.12}                                   \\
		 & = \norm{x}^2 + \norm{y}^2.                          &  & \by{6.1.9}
	\end{align*}
	By setting \(\V = \R^2\) and define \(\inn{\cdot, \cdot}\) as in \cref{6.1.2} we see that the Pythagorean theorem is true.
\end{proof}

\begin{ex}\label{ex:6.1.11}
	Prove the \emph{parallelogram law} on an inner product space \(\V\) over \(\F\);
	that is, show that
	\[
		\norm{x + y}^2 + \norm{x - y}^2 = 2 \norm{x}^2 + 2 \norm{y}^2 \quad \text{for all } x, y \in \V.
	\]
	What does this equation state about parallelograms in \(\R^2\)?
\end{ex}

\begin{proof}[\pf{ex:6.1.11}]
	We have
	\begin{align*}
		 & \norm{x + y}^2 + \norm{x - y}^2                                                         \\
		 & = \inn{x + y, x + y} + \inn{x - y, x - y}                           &  & \by{6.1.9}     \\
		 & = \inn{x, x + y} + \inn{y, x + y} + \inn{x, x - y} - \inn{y, x - y} &  & \by{6.1.1}[ab] \\
		 & = \inn{x, 2x} + \inn{y, 2y}                                         &  & \by{6.1}[ab]   \\
		 & = 2 \inn{x, x} + 2 \inn{y, y}                                       &  & \by{6.1}[b]    \\
		 & = 2 \norm{x}^2 + 2 \norm{y}^2.                                      &  & \by{6.1.9}
	\end{align*}
	On \(\R^2\) we see that if \(x, y\) are the two vectors forming a parallelogram, then \(x + y\) and \(x - y\) are the two diagonal vectors of the parallelogram.
	Thus we see that the sum of the square of the length of the \(4\) sides of a parallelogram equal to the sum of the square of the length of diagonal lines of the same parallelogram.
\end{proof}

\begin{ex}\label{ex:6.1.12}
	Let \(\set{\seq{v}{1,,k}}\) be an orthogonal set in \(\V\) over \(\F\), and let \(\seq{a}{1,,k} \in \F\).
	Prove that
	\[
		\norm{\sum_{i = 1}^k a_i v_i}^2 = \sum_{i = 1}^k \abs{a_i}^2 \norm{v_i}^2.
	\]
\end{ex}

\begin{proof}[\pf{ex:6.1.12}]
	We have
	\begin{align*}
		\norm{\sum_{i = 1}^k a_i v_i}^2 & = \sum_{i = 1}^k \norm{a_i v_i}^2          &  & \by{ex:6.1.10} \\
		                                & = \sum_{i = 1}^k \abs{a_i}^2 \norm{v_i}^2. &  & \by{6.2}[a]
	\end{align*}
\end{proof}

\begin{ex}\label{ex:6.1.13}
	Suppose that \(\inn{\cdot, \cdot}_1\) and \(\inn{\cdot, \cdot}_2\) are two inner products on a vector space \(\V\) over \(\F\).
	Prove that \(\inn{\cdot, \cdot} = \inn{\cdot, \cdot}_1 + \inn{\cdot, \cdot}_2\) is another inner product on \(\V\) over \(\F\).
\end{ex}

\begin{proof}[\pf{ex:6.1.13}]
	Let \(x, y, z \in \V\) and let \(c \in \F\).
	Then we have
	\begin{align*}
		\inn{x + y, z}    & = \inn{x + y, z}_1 + \inn{x + y, z}_2                       &  & \by{ex:6.1.13} \\
		                  & = \inn{x, z}_1 + \inn{y, z}_1 + \inn{x, z}_2 + \inn{y, z}_2 &  & \by{6.1.1}[a]  \\
		                  & = \inn{x, z} + \inn{y, z}                                   &  & \by{ex:6.1.13} \\
		\inn{cx, y}       & = \inn{cx, y}_1 + \inn{cx, y}_2                             &  & \by{ex:6.1.13} \\
		                  & = c \inn{x, y}_1 + c \inn{x, y}_2                           &  & \by{6.1.1}[b]  \\
		                  & = c \inn{x, y}                                              &  & \by{ex:6.1.13} \\
		\conj{\inn{x, y}} & = \conj{\inn{x, y}_1 + \inn{x, y}_2}                        &  & \by{ex:6.1.13} \\
		                  & = \conj{\inn{x, y}_1} + \conj{\inn{x, y}_2}                 &  & \by{d.2}[b]    \\
		                  & = \inn{y, x}_1 + \inn{y, x}_2                               &  & \by{6.1.1}[c]  \\
		                  & = \inn{y, x}.                                               &  & \by{ex:6.1.13}
	\end{align*}
	If \(x \neq \zv\), then we have
	\begin{align*}
		\inn{x, x} & = \inn{x, x}_1 + \inn{x, x}_2 &  & \by{ex:6.1.13} \\
		           & > 0 + 0                       &  & \by{6.1.1}[d]  \\
		           & = 0.
	\end{align*}
	Thus by \cref{6.1.1} \(\inn{\cdot, \cdot}\) is an inner product on \(\V\) over \(\F\).
\end{proof}

\begin{ex}\label{ex:6.1.14}
	Let \(A, B \in \ms{n}{n}{\F}\) and let \(c \in \F\).
	Prove that \((A + cB)^* = A^* + \conj{c} B^*\).
\end{ex}

\begin{proof}[\pf{ex:6.1.14}]
	We have
	\begin{align*}
		\forall i, j \in \set{1, \dots, n}, ((A + cB)^*)_{i j} & = \conj{(A + cB)_{j i}}                    &  & \by{6.1.5}   \\
		                                                       & = \conj{A_{j i} + c B_{j i}}               &  & \by{1.2.9}   \\
		                                                       & = \conj{A_{j i}} + \conj{c} \conj{B_{j i}} &  & \by{d.2}[bc] \\
		                                                       & = (A^*)_{i j} + \conj{c} (B^*)_{i j}       &  & \by{6.1.5}   \\
		                                                       & = (A^* + \conj{c} B^*)_{i j}               &  & \by{1.2.9}
	\end{align*}
	and thus by \cref{1.2.8} \((A + cB)^* = A^* + \conj{c} B^*\).
\end{proof}

\begin{ex}\label{ex:6.1.15}
	\begin{enumerate}
		\item Prove that if \(\V\) is an inner product space over \(\F\), then \(\abs{\inn{x, y}} = \norm{x} \cdot \norm{y}\) iff one of the vectors \(x\) or \(y\) is a multiple of the other.
		\item Derive a similar result for the equality \(\norm{x + y} = \norm{x} + \norm{y}\), and generalize it to the case of \(n\) vectors.
	\end{enumerate}
\end{ex}

\begin{proof}[\pf{ex:6.1.15}(a)]
	If \(y = \zv\) then
	\begin{align*}
		     & \abs{\inn{x, \zv}} = \abs{0} = 0 = \norm{x} \norm{\zv} &  & \by{6.1}[c] \\
		\iff & y = 0x = \zv.                                          &  & \by{1.2}[a]
	\end{align*}
	So suppose that \(y \neq \zv\).

	First suppose that \(\abs{\inn{x, y}} = \norm{x} \norm{y}\).
	Define
	\[
		a = \frac{\inn{x, y}}{\norm{y}^2} \quad \text{and} \quad z = x - ay.
	\]
	Then we have
	\begin{align*}
		\inn{z, y} & = \inn{x - \frac{\inn{x, y}}{\norm{y}^2} y, y}                                \\
		           & = \inn{x, y} - \frac{\inn{x, y}}{\norm{y}^2} \inn{y, y} &  & \by{6.1.1}[ab]   \\
		           & = \inn{x, y} - \frac{\inn{x, y}}{\inn{y, y}} \inn{y, y} &  & \by{6.1.9}       \\
		           & = 0                                                     &  & (\inn{y, y} > 0)
	\end{align*}
	and thus by \cref{6.1.12} we see that \(y, z\) are orthogonal.
	Since
	\begin{align*}
		         & \abs{\inn{x, y}} = \norm{x} \norm{y}                                                       \\
		\implies & \abs{a} = \abs{\frac{\inn{x, y}}{\norm{y}^2}} = \frac{\norm{x}}{\norm{y}}                  \\
		\implies & \abs{a} \norm{y} = \norm{ay} = \norm{x}                                   &  & \by{6.2}[a] \\
	\end{align*}
	we have
	\begin{align*}
		         & \begin{dcases}
			           z \perp y \\
			           x = z + ay
		           \end{dcases}                                          &  & \text{(from the proof above)}   \\
		\implies & \norm{x}^2 = \norm{z + ay}^2 = \norm{z}^2 + \norm{ay}^2 &  & \by{ex:6.1.10}                \\
		\implies & \norm{x}^2 = \norm{z}^2 + \norm{x}^2                    &  & \text{(from the proof above)} \\
		\implies & \norm{z}^2 = 0                                                                             \\
		\implies & \inn{z, z} = 0                                          &  & \by{6.1.9}                    \\
		\implies & x - ay = z = \zv                                        &  & \by{6.1}[d]                   \\
		\implies & x = ay.
	\end{align*}

	Now suppose that there exists an \(a \in \F\) such that \(x = ay\).
	Then we have
	\begin{align*}
		\norm{x} \norm{y} & = \norm{ay} \norm{y}                          \\
		                  & = \abs{a} \norm{y}^2       &  & \by{6.2}[a]   \\
		                  & = \abs{a} \inn{y, y}       &  & \by{6.1.9}    \\
		                  & = \abs{a} \abs{\inn{y, y}} &  & \by{6.2}[b]   \\
		                  & = \abs{a \inn{y, y}}       &  & \by{d.3}[a]   \\
		                  & = \abs{\inn{ay, y}}        &  & \by{6.1.1}[b] \\
		                  & = \abs{\inn{x, y}}.
	\end{align*}
\end{proof}

\begin{proof}[\pf{ex:6.1.15}(b)]
	We claim that \(\norm{x + y} = \norm{x} + \norm{y}\) iff there exists a nonnegative scalar \(c \in \F\) such that \(x = cy\) or \(y = cx\).
	If \(y = \zv\), then we have
	\begin{align*}
		     & \norm{x + \zv} = \norm{x} + \norm{\zv} = \norm{x} &  & \by{6.1}[c] \\
		\iff & \zv = 0x.                                         &  & \by{1.2}[a]
	\end{align*}
	So suppose that \(y \neq \zv\).
	As in the proof of \cref{6.2}(d), we see that
	\[
		\norm{x + y}^2 = \norm{x}^2 + 2 \Re(\inn{x, y}) + \norm{y}^2.
	\]
	Thus we have
	\begin{align*}
		         & \norm{x + y} = \norm{x} + \norm{y}                                                              \\
		\implies & \norm{x + y}^2 = \norm{x}^2 + 2 \norm{x} \norm{y} + \norm{y}^2                                  \\
		\implies & \norm{x} \norm{y} = \Re(\inn{x, y})                                                             \\
		\implies & \abs{\inn{x, y}} \leq \norm{x} \norm{y} = \Re(\inn{x, y})                &  & \by{6.2}[d]       \\
		         & \leq \sqrt{(\Re(\inn{x, y}))^2 + (\Im(\inn{x, y}))^2} = \abs{\inn{x, y}} &  & \by{d.0.5}        \\
		\implies & \abs{\inn{x, y}} = \norm{x} \norm{y}                                                            \\
		\implies & \exists c \in \F : (x = cy) \lor (y = cx)                                &  & \by{ex:6.1.15}[a]
	\end{align*}
	and
	\begin{align*}
		         & \exists c \in \F : \begin{dcases}
			                              c > 0 \\
			                              x = cy
		                              \end{dcases}                                               \\
		\implies & \norm{x} + \norm{y} = \norm{cy} + \norm{y} = (c + 1) \norm{y} &  & \by{6.2}[a] \\
		         & = \norm{(c + 1) y} = \norm{x + y}.                            &  & \by{6.2}[a]
	\end{align*}
	In general we see that
	\[
		\norm{\sum_{i = 1}^n x_i} = \sum_{i = 1}^n \norm{x_i} \iff \begin{dcases}
			\exists \seq{c}{1,,n} \in \F \\
			\exists j \in \set{1, \dots, n}
		\end{dcases} : \forall i \in \set{1, \dots, n}, \begin{dcases}
			c_i > 0 \\
			x_i = c_i x_j
		\end{dcases}.
	\]
\end{proof}

\setcounter{ex}{16}
\begin{ex}\label{ex:6.1.17}
	Let \(\T\) be a linear operator on an inner product space \(\V\) over \(\F\), and suppose that \(\norm{\T(x)} = \norm{x}\) for all \(x \in \V\).
	Prove that \(\T\) is one-to-one.
\end{ex}

\begin{proof}[\pf{ex:6.1.17}]
	We have
	\begin{align*}
		         & \norm{x} = \norm{\T(x)} = 0                  \\
		\implies & x = \T(x) = \zv             &  & \by{6.2}[b] \\
		\implies & \T \text{ is one-to-one}.   &  & \by{2.4}
	\end{align*}
\end{proof}

\begin{ex}\label{ex:6.1.18}
	Let \(\V\) be a vector space over \(\F\), where \(\F = \R\) or \(\F = \C\), and let \(\W\) be an inner product space over \(\F\) with inner product \(\inn{\cdot, \cdot}\).
	If \(\T \in \ls(\V, \W)\), prove that \(\inn{x, y}' = \inn{\T(x), \T(y)}\) defines an inner product on \(\V\) over \(\F\) iff \(\T\) is one-to-one.
\end{ex}

\begin{proof}[\pf{ex:6.1.18}]
	First suppose that \(\inn{\cdot, \cdot}'\) is an inner product on \(\V\) over \(\F\).
	Then we have
	\begin{align*}
		         & \forall x \in \V, \inn{x, x}' = \inn{\T(x), \T(x)} = 0               \\
		\implies & \begin{dcases}
			           x = \zv_{\V} \\
			           \T(x) = \zv_{\W}
		           \end{dcases}                                       &  & \by{6.1}[d]  \\
		\implies & \T \text{ is one-to-one}.                              &  & \by{2.4}
	\end{align*}

	Now suppose that \(\T\) is one-to-one.
	Let \(x, y, z \in \V\) and let \(c \in \F\).
	Then we have
	\begin{align*}
		\inn{x + y, z}'    & = \inn{\T(x + y), \T(z)}                                     \\
		                   & = \inn{\T(x) + \T(y), \T(z)}              &  & \by{2.1.1}[a] \\
		                   & = \inn{\T(x), \T(z)} + \inn{\T(y), \T(z)} &  & \by{6.1.1}[a] \\
		                   & = \inn{x, z}' + \inn{y, z}'                                  \\
		\inn{cx, y}'       & = \inn{\T(cx), \T(y)}                                        \\
		                   & = \inn{c \T(x), \T(y)}                    &  & \by{2.1.1}[b] \\
		                   & = c \inn{\T(x), \T(y)}                    &  & \by{6.1.1}[b] \\
		                   & = c \inn{x, y}'                                              \\
		\conj{\inn{x, y}'} & = \conj{\inn{\T(x), \T(y)}}                                  \\
		                   & = \inn{\T(y), \T(x)}                      &  & \by{6.1.1}[c] \\
		                   & = \inn{y, x}'.
	\end{align*}
	If \(x \neq \zv_{\V}\), then we have
	\begin{align*}
		         & \T(x) \neq \zv_{\W}       &  & \by{2.4}      \\
		\implies & \inn{\T(x), \T(x)} \neq 0 &  & \by{6.1.1}[d] \\
		\implies & \inn{x, x}' \neq 0.
	\end{align*}
	Thus by \cref{6.1.1} \(\inn{\cdot, \cdot}'\) is an inner product on \(\V\) over \(\F\).
\end{proof}

\begin{ex}\label{ex:6.1.19}
	Let \(\V\) be an inner product space over \(\F\).
	Prove that
	\begin{enumerate}
		\item \(\norm{x \pm y}^2 = \norm{x}^2 \pm 2 \Re(\inn{x, y}) + \norm{y}^2\) for all \(x, y \in \V\).
		\item \(\abs{\norm{x} - \norm{y}} \leq \norm{x - y}\) for all \(x, y \in \V\).
	\end{enumerate}
\end{ex}

\begin{proof}[\pf{ex:6.1.19}(a)]
	We have
	\begin{align*}
		 & \norm{x \pm y}^2                                                          \\
		 & = \inn{x \pm y, x \pm y}                                &  & \by{6.1.9}   \\
		 & = \begin{dcases}
			     \inn{x, x + y} + \inn{y, x + y} \\
			     \inn{x, x - y} - \inn{y, x - y}
		     \end{dcases}                      &  & \by{6.1.1}[ab]                   \\
		 & = \inn{x, x} \pm (\inn{x, y} + \inn{y, x}) + \inn{y, y} &  & \by{6.1}[ab] \\
		 & = \norm{x}^2 \pm 2 \Re(\inn{x, y}) + \norm{y}^2.        &  & \by{6.1.9}
	\end{align*}
\end{proof}

\begin{proof}[\pf{ex:6.1.19}(b)]
	We have
	\begin{align*}
		         & \norm{x} = \norm{x - y + y} \leq \norm{x - y} + \norm{y} &  & \by{6.2}[d] \\
		\implies & \norm{x} - \norm{y} \leq \norm{x - y}
	\end{align*}
	and
	\begin{align*}
		         & \norm{y} = \norm{y - x + x} \leq \norm{y - x} + \norm{x}                   &  & \by{6.2}[d] \\
		\implies & \norm{y} - \norm{x} \leq \norm{y - x} = \norm{(-1)(x - y)} = \norm{x - y}. &  & \by{6.2}[a]
	\end{align*}
	Thus \(\abs{\norm{x} - \norm{y}} \leq \norm{x - y}\).
\end{proof}

\begin{ex}\label{ex:6.1.20}
	Let \(\V\) be an inner product space over \(\F\).
	Prove the \emph{polar identities}:
	For all \(x, y \in \V\),
	\begin{enumerate}
		\item \(\inn{x, y} = \frac{1}{4} \norm{x + y}^2 - \frac{1}{4} \norm{x - y}^2\) if \(\F = \R\);
		\item \(\inn{x, y} = \frac{1}{4} \sum_{k = 1}^4 i^k \norm{x + i^k y}^2\) if \(\F = \C\), where \(i^2 = -1\).
	\end{enumerate}
\end{ex}

\begin{proof}[\pf{ex:6.1.20}(a)]
	We have
	\begin{align*}
		\frac{1}{4} \norm{x + y}^2 - \frac{1}{4} \norm{x - y}^2 & = \Re(\inn{x, y}) &  & \by{ex:6.1.19}[a] \\
		                                                        & = \inn{x, y}.     &  & (\F = \R)
	\end{align*}
\end{proof}

\begin{proof}[\pf{ex:6.1.20}(b)]
	We have
	\begin{align*}
		 & \frac{1}{4} \sum_{k = 1}^4 i^k \norm{x + i^k y}^2                                                                                   \\
		 & = \frac{1}{4} \sum_{k = 1}^4 i^k \pa{\norm{x}^2 + 2 \Re(\inn{x, i^k y}) + \norm{i^k y}^2}            &  & \by{ex:6.1.19}[a]         \\
		 & = \frac{1}{4} \sum_{k = 1}^4 i^k \pa{\norm{x}^2 + 2 \Re((-i)^k \inn{x, y}) + \norm{i^k y}^2}         &  & \by{6.1}[b]               \\
		 & = \frac{1}{4} \sum_{k = 1}^4 i^k \pa{\norm{x}^2 + 2 \Re((-i)^k \inn{x, y}) + \abs{i^k}^2 \norm{y}^2} &  & \by{6.2}[a]               \\
		 & = \frac{1}{4} \sum_{k = 1}^4 i^k \pa{\norm{x}^2 + 2 \Re((-i)^k \inn{x, y}) + \norm{y}^2}             &  & (\abs{i} = 1)             \\
		 & = \frac{1}{4} \sum_{k = 1}^4 i^k 2 \Re((-i)^k \inn{x, y})                                            &  & (i + (-1) + (-i) + 1 = 0) \\
		 & = \frac{1}{2} \pa{i \Re(-i \inn{x, y}) - \Re(-\inn{x, y}) - i \Re(i \inn{x, y}) + \Re(\inn{x, y})}                                  \\
		 & = \Re(\inn{x, y}) - i \Re(i \inn{x, y})                                                                                             \\
		 & = \Re(\inn{x, y}) - i \Im(-\inn{x, y})                                                                                              \\
		 & = \Re(\inn{x, y}) + i \Im(\inn{x, y})                                                                                               \\
		 & = \inn{x, y}.
	\end{align*}
\end{proof}

\begin{ex}\label{ex:6.1.21}
	Let \(A \in \ms{n}{n}{\F}\).
	Define
	\[
		A_1 = \frac{1}{2} (A + A^*) \quad \text{and} \quad A_2 = \frac{1}{2i} (A - A^*).
	\]
	\begin{enumerate}
		\item Prove that \(A_1^* = A_1\), \(A_2^* = A_2\), and \(A = A_1 + i A_2\).
		      Would it be reasonable to define \(A_1\) and \(A_2\) to be the real and imaginary parts, respectively, of the matrix \(A\)?
		\item Prove that the representation in (a) is unique.
		      That is, prove that if \(A = B_1 + i B_2\), where \(B_1^* = B_1\) and \(B_2^* = B_2\), then \(B_1 = A_1\) and \(B_2 = A_2\).
	\end{enumerate}
\end{ex}

\begin{proof}[\pf{ex:6.1.21}(a)]
	We have
	\begin{align*}
		\forall i, j \in \set{1, \dots, n}, (A_1^*)_{i j} & = \conj{(A_1)_{j i}}                                  &  & \by{6.1.5}     \\
		                                                  & = \conj{\pa{\frac{1}{2} (A + A^*)}_{j i}}             &  & \by{ex:6.1.21} \\
		                                                  & = \conj{\frac{1}{2} (A_{j i} + (A^*)_{j i})}          &  & \by{1.2.9}     \\
		                                                  & = \frac{1}{2} (\conj{A_{j i}} + \conj{(A^*)_{j i}})   &  & \by{d.2}[bc]   \\
		                                                  & = \frac{1}{2} ((A^*)_{i j} + \conj{\conj{A_{i j}}})   &  & \by{6.1.5}     \\
		                                                  & = \frac{1}{2} ((A^*)_{i j} + A_{i j})                 &  & \by{d.2}[a]    \\
		                                                  & = \pa{\frac{1}{2} (A^* + A)}_{i j}                    &  & \by{1.2.9}     \\
		                                                  & = \pa{\frac{1}{2} (A + A^*)}_{i j}                    &  & \by{1.2.9}     \\
		                                                  & = (A_1)_{i j}                                         &  & \by{ex:6.1.21} \\
		\forall i, j \in \set{1, \dots, n}, (A_2^*)_{i j} & = \conj{(A_2)_{j i}}                                  &  & \by{6.1.5}     \\
		                                                  & = \conj{\pa{\frac{1}{2i} (A - A^*)}_{j i}}            &  & \by{ex:6.1.21} \\
		                                                  & = \conj{\frac{1}{2i} (A_{j i} - (A^*)_{j i})}         &  & \by{1.2.9}     \\
		                                                  & = \frac{-1}{2i} (\conj{A_{j i}} - \conj{(A^*)_{j i}}) &  & \by{d.2}[bc]   \\
		                                                  & = \frac{-1}{2i} ((A^*)_{i j} - \conj{\conj{A_{i j}}}) &  & \by{6.1.5}     \\
		                                                  & = \frac{-1}{2i} ((A^*)_{i j} - A_{i j})               &  & \by{d.2}[a]    \\
		                                                  & = \pa{\frac{-1}{2i} (A^* - A)}_{i j}                  &  & \by{1.2.9}     \\
		                                                  & = \pa{\frac{1}{2i} (A - A^*)}_{i j}                   &  & \by{1.2.9}     \\
		                                                  & = (A_2)_{i j}                                         &  & \by{ex:6.1.21}
	\end{align*}
	and thus by \cref{1.2.8} \(A_1^* = A_1\) and \(A_2^* = A_2\).
	Observe that
	\begin{align*}
		A_1 + i A_2 & = \frac{1}{2} (A + A^*) + \frac{i}{2i} (A - A^*) &  & \by{ex:6.1.21} \\
		            & = A.                                             &  & \by{1.2.9}
	\end{align*}
	If we let \(A \in \ms{2}{2}{\C}\) where \(A = \begin{pmatrix}
		1 & -i \\
		i & 1
	\end{pmatrix}\), then we have
	\begin{align*}
		A_1          & = \frac{1}{2} (A + A^*)           &  & \by{ex:6.1.21}            \\
		             & = \frac{1}{2} \pa{\begin{pmatrix}
				                                 1 & -i \\
				                                 i & 1
			                                 \end{pmatrix} + \begin{pmatrix}
				                                                 1 & -i \\
				                                                 i & 1
			                                                 \end{pmatrix}} &  & \by{6.1.5} \\
		             & = A                                                              \\
		A_2          & = \frac{1}{2} (A - A^*)           &  & \by{ex:6.1.21}            \\
		             & = \frac{1}{2} \pa{\begin{pmatrix}
				                                 1 & -i \\
				                                 i & 1
			                                 \end{pmatrix} - \begin{pmatrix}
				                                                 1 & -i \\
				                                                 i & 1
			                                                 \end{pmatrix}} &  & \by{6.1.5} \\
		             & = \zm                                                            \\
		A + \conj{A} & = \begin{pmatrix}
			                 1 & -i \\
			                 i & 1
		                 \end{pmatrix} + \begin{pmatrix}
			                                 1  & i \\
			                                 -i & 1
		                                 \end{pmatrix}   &  & \by{ex:4.3.13}            \\
		             & = \begin{pmatrix}
			                 2 & 0 \\
			                 0 & 2
		                 \end{pmatrix}                                                 \\
		             & \neq 2 A_1.
	\end{align*}
	Thus it does not make sense to define \(A_1\) as the real part of \(A\) and \(A_2\) as the imaginary part of \(A\).
\end{proof}

\begin{proof}[\pf{ex:6.1.21}(b)]
	We have
	\begin{align*}
		         & A = B_1 + i B_2 = A_1 + i A_2                                    \\
		\implies & (A_1 - B_1) + i (A_2 - B_2) = \zm         &  & \by{1.2.9}        \\
		\implies & ((A_1 - B_1) + i (A_2 - B_2))^* = \zm     &  & \by{6.1.5}        \\
		\implies & (A_1 - B_1)^* - i (A_2 - B_2)^* = \zm     &  & \by{ex:6.1.14}    \\
		\implies & (A_1^* - B_1^*) - i (A_2^* - B_2^*) = \zm &  & \by{ex:6.1.14}    \\
		\implies & (A_1 - B_1) - i (A_2 - B_2) = \zm         &  & \by{ex:6.1.21}[a] \\
		\implies & \begin{dcases}
			           (A_1 - B_1) + i (A_2 - B_2) = \zm \\
			           (A_1 - B_1) - i (A_2 - B_2) = \zm
		           \end{dcases}                                \\
		\implies & \begin{dcases}
			           A_1 = B_1 \\
			           A_2 = B_2
		           \end{dcases}.                            &  & \by{1.2.9}
	\end{align*}
\end{proof}

\begin{ex}\label{ex:6.1.22}
	Let \(\V\) be a vector space over \(\F\) (possibly infinite-dimensional) where \(\F = \R\) or \(\F = \C\), and let \(\beta\) be a basis for \(\V\) over \(\F\).
	For \(x, y \in \V\) there exist \(\seq{v}{1,,n} \in \beta\) such that
	\[
		\sum_{i = 1}^n a_i v_i \quad \text{and} \quad y = \sum_{i = 1}^n b_i v_i.
	\]
	Define
	\[
		\inn{x, y} = \sum_{i = 1}^n a_i \conj{b_i}.
	\]
	\begin{enumerate}
		\item Prove that \(\inn{\cdot, \cdot}\) is an inner product on \(\V\) over \(\F\) and that \(\beta\) is an orthonormal basis for \(\V\) over \(\F\).
		      Thus every real or complex vector space may be regarded as an inner product space.
		\item Prove that if \(\V = \R^n\) or \(\V = \C^n\) and \(\beta\) is the standard ordered basis for \(\V\) over \(\F\), then the inner product defined above is the standard inner product.
	\end{enumerate}
\end{ex}

\begin{proof}[\pf{ex:6.1.22}(a)]
	Let \(x, y, z \in \V\) and let \(k \in \F\).
	Then
	\[
		\begin{dcases}
			\exists \seq{a}{1,,n} \in \F \\
			\exists \seq{b}{1,,n} \in \F \\
			\exists \seq{c}{1,,n} \in \F \\
			\exists \seq{v}{1,,n} \in \F
		\end{dcases} : \begin{dcases}
			x = \sum_{i = 1}^n a_i v_i \\
			y = \sum_{i = 1}^n b_i v_i \\
			z = \sum_{i = 1}^n c_i v_i
		\end{dcases}.
	\]
	Thus we have
	\begin{align*}
		\inn{x + y, z}    & = \inn{\sum_{i = 1}^n (a_i + b_i) v_i, \sum_{i = 1}^n c_i v_i}  &  & \by{1.2.1}     \\
		                  & = \sum_{i = 1}^n (a_i + b_i) \conj{c_i}                         &  & \by{ex:6.1.22} \\
		                  & = \sum_{i = 1}^n a_i \conj{c_i} + \sum_{i = 1}^n b_i \conj{c_i}                     \\
		                  & = \inn{x, z} + \inn{y, z}                                       &  & \by{ex:6.1.22} \\
		\inn{kx, y}       & = \inn{\sum_{i = 1}^n k a_i v_i, \sum_{i = 1}^n b_i v_i}        &  & \by{1.2.1}     \\
		                  & = \sum_{i = 1}^n (k a_i) \conj{b_j}                             &  & \by{ex:6.1.22} \\
		                  & = k \sum_{i = 1}^n a_i \conj{b_i}                                                   \\
		                  & = k \inn{x, y}                                                  &  & \by{ex:6.1.22} \\
		\conj{\inn{x, y}} & = \conj{\sum_{i = 1}^n a_i \conj{b_i}}                          &  & \by{ex:6.1.22} \\
		                  & = \sum_{i = 1}^n \conj{a_i} b_i                                 &  & \by{d.2}[abc]  \\
		                  & = \inn{y, x}.                                                   &  & \by{ex:6.1.22}
	\end{align*}
	If \(x \neq \zv\), then we have
	\begin{align*}
		         & \exists i \in \set{1, \dots, n} : a_i \neq 0                       &  & \by{1.5.3}     \\
		\implies & \exists i \in \set{1, \dots, n} : a_i \conj{a_i} = \abs{a_i}^2 > 0 &  & \by{d.0.5}     \\
		\implies & \sum_{i = 1}^n a_i \conj{a_i} = \sum_{i = 1}^n \abs{a_i}^2 > 0                         \\
		\implies & \inn{x, x} > 0.                                                    &  & \by{ex:6.1.22}
	\end{align*}
	Thus by \cref{6.1.1} \(\inn{\cdot, \cdot}\) is an inner product on \(\V\) over \(\F\).

	Now we show that \(\beta\) is orthonormal.
	Since
	\begin{align*}
		         & \forall v_i, v_j \in \beta, \exists \seq{v}{1,,n} \in \beta : \begin{dcases}
			                                                                         n = \max(i, j)                        \\
			                                                                         v_i = \sum_{k = 1}^n \delta_{k i} v_k \\
			                                                                         v_j = \sum_{k = 1}^n \delta_{k j} v_k
		                                                                         \end{dcases}                 &  & \by{1.8}         \\
		\implies & \inn{v_i, v_j} = \inn{\sum_{k = 1}^n \delta_{k i} v_k, \sum_{k = 1}^n \delta_{k j} v_k}                          \\
		         & = \sum_{k = 1}^n \delta_{k i} \conj{\delta_{k j}} = \sum_{k = 1}^n \delta_{k i} \delta_{k j} &  & \by{ex:6.1.22} \\
		         & = \begin{dcases}
			             \delta_{i i} \delta_{i j} + \delta_{j i} \delta_{j j} = 0 & \text{if } i \neq j \\
			             \delta_{i i} = 1                                          & \text{if } i = j
		             \end{dcases}           &  & \by{2.3.4}                                \\
		         & = \delta_{i j},                                                                              &  & \by{2.3.4}
	\end{align*}
	by \cref{6.1.12} we know that \(\beta\) is orthonormal.
\end{proof}

\begin{proof}[\pf{ex:6.1.22}(b)]
	Since
	\[
		\forall x, y \in \V, \inn{x, y} = \inn{\sum_{i = 1}^n x_i e_i, \sum_{i = 1}^n y_i e_i} = \sum_{i = 1}^n x_i \conj{y_i},
	\]
	by \cref{6.1.2} we see that \(\inn{\cdot, \cdot}\) is the standard inner product on \(\V\) over \(\F\).
\end{proof}

\begin{ex}\label{ex:6.1.23}
	Let \(\inn{\cdot, \cdot}\) be the standard inner product on \(\vs{F}^n\) over \(\F\), and let \(A \in \ms{n}{n}{\F}\).
	\begin{enumerate}
		\item Prove that \(\inn{x, Ay} = \inn{A^* x, y}\) for all \(x, y \in \vs{F}^n\).
		\item Suppose that for some \(B \in \ms{n}{n}{\F}\), we have \(\inn{x, Ay} = \inn{Bx, y}\) for all \(x, y \in \vs{F}^n\).
		      Prove that \(B = A^*\).
		\item Let \(\alpha\) be the standard ordered basis for \(\vs{F}^n\) over \(\F\).
		      For any orthonormal basis \(\beta\) for \(\vs{F}^n\) over \(\F\), let \(Q \in \ms{n}{n}{\F}\) whose columns are the vectors in \(\beta\).
		      Prove that \(Q^* = Q^{-1}\).
		\item Define \(\T, \U \in \ls(\vs{F}^n)\) by \(\T(x) = Ax\) and \(\U(x) = A^* x\).
		      Show that \([\U]_{\beta} = [\T]_{\beta}^*\) for any orthonormal basis \(\beta\) for \(\vs{F}^n\) over \(\F\).
	\end{enumerate}
\end{ex}

\begin{proof}[\pf{ex:6.1.23}(a)]
	We have
	\begin{align*}
		\inn{x, Ay} & = \sum_{i = 1}^n x_i \conj{(Ay)_i}                                 &  & \by{6.1.2}   \\
		            & = \sum_{i = 1}^n x_i \conj{\pa{\sum_{j = 1}^n A_{i j} y_j}}        &  & \by{2.3.1}   \\
		            & = \sum_{i = 1}^n x_i \pa{\sum_{j = 1}^n \conj{A_{i j}} \conj{y_j}} &  & \by{d.2}[bc] \\
		            & = \sum_{j = 1}^n \conj{y_j} \sum_{i = 1}^n \conj{A_{i j}} x_i                        \\
		            & = \sum_{j = 1}^n \conj{y_j} \sum_{i = 1}^n (A^*)_{j i} x_i         &  & \by{6.1.5}   \\
		            & = \sum_{j = 1}^n \conj{y_j} (A^* x)_j                              &  & \by{2.3.1}   \\
		            & = \inn{A^* x, y}.                                                  &  & \by{6.1.2}
	\end{align*}
\end{proof}

\begin{proof}[\pf{ex:6.1.23}(b)]
	We have
	\begin{align*}
		         & \forall x, y \in \vs{F}^n, \inn{x, Ay} = \inn{A^* x, y} = \inn{Bx, y} &  & \by{ex:6.1.23}[a] \\
		\implies & \forall x, y \in \vs{F}^n, \inn{y, A^* x} = \inn{y, Bx}               &  & \by{6.1.1}[c]     \\
		\implies & \forall x \in \vs{F}^n, A^* x = Bx                                    &  & \by{6.1}[e]       \\
		\implies & A^* = B.                                                              &  & \by{2.1.13}
	\end{align*}
\end{proof}

\begin{proof}[\pf{ex:6.1.23}(c)]
	For each \(i \in \set{1, \dots, n}\), define \(v_i\) be the \(i\)th column of \(Q\).
	Then we have
	\begin{align*}
		         & Q = [\IT[\vs{F}^n]]_{\beta}^{\alpha}                                                   &  & \by{2.5.1}        \\
		\implies & \forall i, j \in \set{1, \dots, n}, (Q^* Q)_{i j} = \sum_{k = 1}^n (Q^*)_{i k} Q_{k j} &  & \by{2.3.1}        \\
		         & = \sum_{k = 1}^n \conj{Q_{k i}} Q_{k j}                                                &  & \by{6.1.5}        \\
		         & = \inn{v_j, v_i}                                                                       &  & \by{2.5.1}        \\
		         & = \delta_{j i}                                                                         &  & \by{6.1.12}       \\
		\implies & Q^* Q = I_n                                                                            &  & \by{2.3.4}        \\
		\implies & Q^{-1} = Q^*.                                                                          &  & \by{ex:2.4.10}[b]
	\end{align*}
\end{proof}

\begin{proof}[\pf{ex:6.1.23}(d)]
	Let \(\alpha\) be the standard ordered basis for \(\vs{F}^n\) over \(\F\).
	Then we have
	\begin{align*}
		[\T]_{\beta}^* & = ([\IT[\vs{F}^n]]_{\alpha}^{\beta} \cdot [\T]_{\alpha} \cdot [\IT[\vs{F}^n]]_{\beta}^{\alpha})^*                         &  & \by{2.23}         \\
		               & = ([\IT[\vs{F}^n]]_{\alpha}^{\beta} \cdot A \cdot [\IT[\vs{F}^n]]_{\beta}^{\alpha})^*                                     &  & \by{2.15}[a]      \\
		               & = \conj{\tp{([\IT[\vs{F}^n]]_{\alpha}^{\beta} \cdot A \cdot [\IT[\vs{F}^n]]_{\beta}^{\alpha})}}                           &  & \by{6.1.5}        \\
		               & = \conj{\tp{([\IT[\vs{F}^n]]_{\beta}^{\alpha})} \cdot \tp{A} \cdot \tp{([\IT[\vs{F}^n]]_{\alpha}^{\beta})}}               &  & \by{2.3.2}        \\
		               & = \conj{\tp{([\IT[\vs{F}^n]]_{\beta}^{\alpha})}} \cdot \conj{\tp{A}} \cdot \conj{\tp{([\IT[\vs{F}^n]]_{\alpha}^{\beta})}} &  & \by{ex:4.3.13}    \\
		               & = ([\IT[\vs{F}^n]]_{\beta}^{\alpha})^* \cdot A^* \cdot ([\IT[\vs{F}^n]]_{\alpha}^{\beta})^*                               &  & \by{6.1.5}        \\
		               & = ([\IT[\vs{F}^n]]_{\beta}^{\alpha})^{-1} \cdot A^* \cdot ([\IT[\vs{F}^n]]_{\alpha}^{\beta})^*                            &  & \by{ex:6.1.23}[c] \\
		               & = [\IT[\vs{F}^n]]_{\alpha}^{\beta} \cdot A^* \cdot (([\IT[\vs{F}^n]]_{\beta}^{\alpha})^{-1})^*                            &  & \by{2.23}         \\
		               & = [\IT[\vs{F}^n]]_{\alpha}^{\beta} \cdot A^* \cdot (([\IT[\vs{F}^n]]_{\beta}^{\alpha})^*)^*                               &  & \by{ex:6.1.23}    \\
		               & = [\IT[\vs{F}^n]]_{\alpha}^{\beta} \cdot A^* \cdot [\IT[\vs{F}^n]]_{\beta}^{\alpha}                                       &  & \by{6.1.5}        \\
		               & = [\IT[\vs{F}^n]]_{\alpha}^{\beta} \cdot [\U]_{\alpha} \cdot [\IT[\vs{F}^n]]_{\beta}^{\alpha}                             &  & \by{2.15}[a]      \\
		               & = [\U]_{\beta}.                                                                                                           &  & \by{2.23}
	\end{align*}
\end{proof}

\begin{defn}\label{6.1.14}
	Let \(\V\) be a vector space over \(\F\), where \(\F\) is either \(\R\) or \(\C\).
	Regardless of whether \(\V\) is or is not an inner product space, we may still define a norm \(\norm{\cdot}\) as a real-valued function on \(\V\) satisfying the following three conditions for all \(x, y \in \V\) and \(a \in \F\):
	\begin{enumerate}
		\item \(\norm{x} \geq 0\), and \(\norm{x} = 0\) iff \(x = \zv\).
		\item \(\norm{ax} = \abs{a} \norm{x}\).
		\item \(\norm{x + y} \leq \norm{x} + \norm{y}\).
	\end{enumerate}
\end{defn}

\begin{ex}\label{ex:6.1.24}
	Let \(\F = \R\) or \(\F = \C\).
	Prove that the following are norms on the given vector spaces.
	\begin{enumerate}
		\item \(\norm{A} = \max_{1 \leq i \leq m, 1 \leq j \leq n} \abs{A_{i j}}\) for all \(A \in \MS\).
		\item \(\norm{f} = \max_{t \in [0, 1]} \abs{f(t)}\) for all \(f \in \cfs([0, 1], \F)\).
		\item \(\norm{f} = \int_0^1 \abs{f(t)} \; dt\) for all \(f \in \cfs([0, 1], \F)\).
		\item \(\norm{(a, b)} = \max\set{\abs{a}, \abs{b}}\) for all \((a, b) \in \F^2\).
	\end{enumerate}
\end{ex}

\begin{proof}[\pf{ex:6.1.24}(a)]
	\begin{description}
		\item[For \cref{6.1.14}(a):]
			Let \(A \in \MS\).
			Then we have
			\begin{align*}
				         & \forall (i, j) \in \set{1, \dots, m} \times \set{1, \dots, n}, \abs{A_{i j}} \geq 0 &  & \by{d.0.5} \\
				\implies & \norm{A} = \max_{1 \leq i \leq m, 1 \leq j \leq n} \abs{A_{i j}} \geq 0
			\end{align*}
			and
			\begin{align*}
				     & \norm{A} = \max_{1 \leq i \leq m, 1 \leq n \leq n} \abs{A_{i j}} = 0                             \\
				\iff & \forall (i, j) \in \set{1, \dots, m} \times \set{1, \dots, n}, \abs{A_{i j}} = 0                 \\
				\iff & \forall (i, j) \in \set{1, \dots, m} \times \set{1, \dots, n}, A_{i j} = 0       &  & \by{d.0.5} \\
				\iff & A = \zm.
			\end{align*}
		\item[For \cref{6.1.14}(b):]
			Let \(A \in \MS\) and let \(a \in \F\).
			Then we have
			\begin{align*}
				\norm{a A} & = \max_{1 \leq i \leq m, 1 \leq j \leq n} \abs{(a A)_{i j}}                     \\
				           & = \max_{1 \leq i \leq m, 1 \leq j \leq n} \abs{a A_{i j}}       &  & \by{1.2.9} \\
				           & = \max_{1 \leq i \leq m, 1 \leq j \leq n} \abs{a} \abs{A_{i j}}                 \\
				           & = \abs{a} \max_{1 \leq i \leq m, 1 \leq j \leq n} \abs{A_{i j}}                 \\
				           & = \abs{a} \norm{A}.
			\end{align*}
		\item[For \cref{6.1.14}(c):]
			Let \(A, B \in \MS\).
			Then we have
			\begin{align*}
				\norm{A + B} & = \max_{1 \leq i \leq m, 1 \leq j \leq n} \abs{(A + B)_{i j}}                                                                    \\
				             & = \max_{1 \leq i \leq m, 1 \leq j \leq n} \abs{A_{i j} + B_{i j}}                                               &  & \by{1.2.9}  \\
				             & \leq \max_{1 \leq i \leq m, 1 \leq j \leq n} \abs{A_{i j}} + \abs{B_{i j}}                                      &  & \by{d.3}[c] \\
				             & = \max_{1 \leq i \leq m, 1 \leq j \leq n} \abs{A_{i j}} + \max_{1 \leq i \leq m, 1 \leq j \leq n} \abs{B_{i j}}                  \\
				             & = \norm{A} + \norm{B}.
			\end{align*}
	\end{description}
	From all proofs above we conclude by \cref{6.1.14} that \(\norm{\cdot}\) is a norm on \(\MS\) over \(\F\).
\end{proof}

\begin{proof}[\pf{ex:6.1.24}(b)]
	\begin{description}
		\item[For \cref{6.1.14}(a):]
			Let \(f \in \cfs([0, 1], \F)\).
			Then we have
			\begin{align*}
				         & \forall t \in [0, 1], \abs{f(t)} \geq 0          &  & \by{d.0.5} \\
				\implies & \norm{f} = \max_{t \in [0, 1]} \abs{f(t)} \geq 0
			\end{align*}
			and
			\begin{align*}
				     & \norm{f} = \max_{t \in [0, 1]} \abs{f(t)} = 0                 \\
				\iff & \forall t \in [0, 1], \abs{f(t)} = 0                          \\
				\iff & \forall t \in [0, 1], f(t) = 0                &  & \by{d.0.5} \\
				\iff & f = \zv.
			\end{align*}
		\item[For \cref{6.1.14}(b):]
			Let \(f \in \cfs([0, 1], \F)\) and let \(c \in \F\).
			Then we have
			\begin{align*}
				\norm{cf} & = \max_{t \in [0, 1]} \abs{(cf)(t)}      \\
				          & = \max_{t \in [0, 1]} \abs{cf(t)}        \\
				          & = \max_{t \in [0, 1]} \abs{c} \abs{f(t)} \\
				          & = \abs{c} \max_{t \in [0, 1]} \abs{f(t)} \\
				          & = \abs{c} \norm{f}.
			\end{align*}
		\item[For \cref{6.1.14}(c):]
			Let \(f, g \in \cfs([0, 1], \F)\).
			Then we have
			\begin{align*}
				\norm{f + g} & = \max_{t \in [0, 1]} \abs{(f + g)(t)}                            \\
				             & = \max_{t \in [0, 1]} \abs{f(t) + g(t)}                           \\
				             & \leq \max_{t \in [0, 1]} \abs{f(t)} + \abs{g(t)}                  \\
				             & = \max_{t \in [0, 1]} \abs{f(t)} + \max_{t \in [0, 1]} \abs{g(t)} \\
				             & = \norm{f} + \norm{g}.
			\end{align*}
	\end{description}
	From all proofs above we conclude by \cref{6.1.14} that \(\norm{\cdot}\) is a norm on \(\cfs([0, 1], \F)\) over \(\F\).
\end{proof}

\begin{proof}[\pf{ex:6.1.24}(c)]
	\begin{description}
		\item[For \cref{6.1.14}(a):]
			Let \(f \in \cfs([0, 1], \F)\).
			Then we have
			\begin{align*}
				         & \forall t \in [0, 1], \abs{f(t)} \geq 0     &  & \by{d.0.5} \\
				\implies & \norm{f} = \int_0^1 \abs{f(t)} \; dt \geq 0
			\end{align*}
			and
			\begin{align*}
				     & \norm{f} = \int_0^1 \abs{f(t)} \; dt = 0                 \\
				\iff & \forall t \in [0, 1], \abs{f(t)} = 0                     \\
				\iff & \forall t \in [0, 1], f(t) = 0           &  & \by{d.0.5} \\
				\iff & f = \zv.
			\end{align*}
		\item[For \cref{6.1.14}(b):]
			Let \(f \in \cfs([0, 1], \F)\) and let \(c \in \F\).
			Then we have
			\begin{align*}
				\norm{cf} & = \int_0^1 \abs{(cf)(t)} \; dt      \\
				          & = \int_0^1 \abs{cf(t)} \; dt        \\
				          & = \int_0^1 \abs{c} \abs{f(t)} \; dt \\
				          & = \abs{c} \int_0^1 \abs{f(t)} \; dt \\
				          & = \abs{c} \norm{f}.
			\end{align*}
		\item[For \cref{6.1.14}(c):]
			Let \(f, g \in \cfs([0, 1], \F)\).
			Then we have
			\begin{align*}
				\norm{f + g} & = \int_0^1 \abs{(f + g)(t)} \; dt                       \\
				             & = \int_0^1 \abs{f(t) + g(t)} \; dt                      \\
				             & \leq \int_0^1 \abs{f(t)} + \abs{g(t)} \; dt             \\
				             & = \int_0^1 \abs{f(t)} \; dt + \int_0^1 \abs{g(t)} \; dt \\
				             & = \norm{f} + \norm{g}.
			\end{align*}
	\end{description}
	From all proofs above we conclude by \cref{6.1.14} that \(\norm{\cdot}\) is a norm on \(\cfs([0, 1], \F)\) over \(\F\).
\end{proof}

\begin{proof}[\pf{ex:6.1.24}(d)]
	This is the special case of \cref{ex:6.1.24}(a) where \(m = 2\) and \(n = 1\).
\end{proof}

\begin{ex}\label{ex:6.1.25}
	Use \cref{ex:6.1.20} to show that there is no inner product \(\inn{\cdot, \cdot}\) on \(\R^2\) such that \(\norm{x}^2 = \inn{x, x}\) for all \(x \in \R^2\) if the norm is defined as in \cref{ex:6.1.24}(d).
\end{ex}

\begin{proof}[\pf{ex:6.1.25}]
	Suppose for sake of contradiction that there exist an inner product \(\inn{\cdot, \cdot}\) on \(\R^2\) over \(\R\) such that \(\inn{x, x} = \norm{x}^2 = (\max\set{\abs{x_1}, \abs{x_2}})^2\) for all \(x \in \R^2\).
	But then we have
	\begin{align*}
		\inn{(2, 0), (1, 1)}   & = \frac{1}{4} \norm{(3, 1)}^2 - \frac{1}{4} \norm{(1, -1)}^2        &  & \by{ex:6.1.20}[a] \\
		                       & = \frac{9}{4} - \frac{1}{4}                                         &  & \by{ex:6.1.24}[d] \\
		                       & = 2                                                                                        \\
		2 \inn{(1, 0), (1, 1)} & = 2 \pa{\frac{1}{4} \norm{(2, 1)}^2 - \frac{1}{4} \norm{(0, -1)}^2} &  & \by{ex:6.1.20}[a] \\
		                       & = 2 \pa{\frac{4}{4} - \frac{1}{4}}                                  &  & \by{ex:6.1.24}[d] \\
		                       & = \frac{3}{2}                                                                              \\
		                       & \neq \inn{(2, 0), (1, 1)},
	\end{align*}
	a contradiction.
	Thus there does not exist an inner product on \(\R^2\) such that \(\inn{x, x} = \norm{x}^2 = \max\set{\abs{x_1}, \abs{x_2}}\) for all \(x \in \R^2\).
\end{proof}

\begin{ex}\label{ex:6.1.26}
	Let \(\norm{\cdot}\) be a norm on a vector space \(\V\) over \(\F\), and define, for each ordered pair of vectors, the scalar \(d(x, y) = \norm{x - y}\), called the \textbf{distance} between \(x\) and \(y\).
	Prove the following results for all \(x, y, z \in \V\).
	\begin{enumerate}
		\item \(d(x, y) \geq 0\).
		\item \(d(x, y) = d(y, x)\).
		\item \(d(x, y) \leq d(x, z) + d(z, y)\).
		\item \(d(x, x) = 0\).
		\item \(d(x, y) \neq 0\) if \(x \neq y\).
	\end{enumerate}
\end{ex}

\begin{proof}[\pf{ex:6.1.26}]
	We have
	\begin{align*}
		d(x, y) & = \norm{x - y}                   &  & \by{ex:6.1.26} \\
		        & \geq 0                           &  & \by{6.1.14}[a] \\
		d(x, y) & = \norm{x - y}                   &  & \by{ex:6.1.26} \\
		        & = \norm{(-1)(y - x)}             &  & \by{1.2.1}     \\
		        & = \abs{-1} \norm{y - x}          &  & \by{6.1.14}[b] \\
		        & = \norm{y - x}                                       \\
		        & = d(y, x)                        &  & \by{ex:6.1.26} \\
		d(x, y) & = \norm{x - y}                   &  & \by{ex:6.1.26} \\
		        & = \norm{x - z + z - y}                               \\
		        & \leq \norm{x - z} + \norm{z - y} &  & \by{6.1.14}[c] \\
		d(x, x) & = \norm{x - x}                   &  & \by{ex:6.1.26} \\
		        & = \norm{\zv}                                         \\
		        & = 0                              &  & \by{6.1.14}[a]
	\end{align*}
	and
	\begin{align*}
		         & x \neq y                                \\
		\implies & x - y \neq \zv                          \\
		\implies & \norm{x - y} \neq 0 &  & \by{6.1.14}[a] \\
		\implies & d(x, y) \neq 0.     &  & \by{ex:6.1.26}
	\end{align*}
\end{proof}

\begin{ex}\label{ex:6.1.27}
	Let \(\norm{\cdot}\) be a norm on \(\V\) over \(\R\) satisfying the parallelogram law given in \cref{ex:6.1.11}.
	Define
	\[
		\inn{x, y} = \frac{1}{4} \pa{\norm{x + y}^2 - \norm{x - y}^2}.
	\]
	Prove that \(\inn{\cdot, \cdot}\) defines an inner product on \(\V\) over \(\R\) such that \(\norm{x}^2 = \inn{x, x}\) for all \(x \in \V\).
\end{ex}

\begin{proof}[\pf{ex:6.1.27}]
	Let \(x, y, z \in \V\).
	First we show that \(\inn{2x, y} = 2 \inn{x, y}\).
	This is true since
	\begin{align*}
		\inn{x, 2y} & = \frac{1}{4} \pa{\norm{x + 2y}^2 - \norm{x - 2y}^2}                                                     \\
		            & = \frac{1}{4} \pa{\norm{x + 2y}^2 + \norm{x}^2 - \norm{x}^2 - \norm{x - 2y}^2}                           \\
		            & = \frac{1}{4} \pa{2 \norm{x + y}^2 + 2 \norm{y}^2 - 2 \norm{x - y}^2 - 2 \norm{y}^2} &  & \by{ex:6.1.11} \\
		            & = 2 \inn{x, y}.
	\end{align*}

	Next we show that \(\inn{x + y, z} = \inn{x, z} + \inn{y, z}\).
	This is true since
	\begin{align*}
		 & \inn{x, z} + \inn{y, z}                                                                                                                             \\
		 & = \frac{1}{4} \pa{\norm{x + z}^2 - \norm{x - z}^2 + \norm{y + z}^2 - \norm{y - z}^2}                                                                \\
		 & = \frac{1}{4} \pa{2 \norm{\frac{1}{2} x + \frac{1}{2} y + z}^2 + 2 \norm{\frac{1}{2} x - \frac{1}{2} y}^2}       &  & \by{ex:6.1.11}                \\
		 & \quad - \frac{1}{4} \pa{2 \norm{\frac{1}{2} x + \frac{1}{2} y - z}^2 + 2 \norm{\frac{1}{2} x - \frac{1}{2} y}^2} &  & \by{ex:6.1.11}                \\
		 & = \frac{1}{8} \pa{\norm{x + y + 2z}^2 - \norm{x + y - 2z}^2}                                                     &  & \by{6.1.14}[b]                \\
		 & = \frac{1}{2} \inn{x + y, 2z}                                                                                                                       \\
		 & = \inn{x + y, z}.                                                                                                &  & \text{(from the proof above)}
	\end{align*}
	Thus \cref{6.1.1}(a) is satisfied.

	Next we show that \(\inn{nx, y} = n \inn{x, y}\) for all \(n \in \Z^+\).
	This is true since
	\begin{align*}
		\forall n \in \Z^+, \inn{nx, y} & = \inn{\sum_{i = 1}^n x, y}                                    \\
		                                & = \sum_{i = 1}^n \inn{x, y} &  & \text{(from the proof above)} \\
		                                & = n \inn{x, y}.
	\end{align*}

	Next we show that \(\inn{nx, y} = n \inn{x, y}\) for all \(n \in \Z^-\).
	This is true since
	\begin{align*}
		         & \frac{1}{4} \pa{\norm{\zv + y}^2 - \norm{\zv - y}^2} = \frac{1}{4} \pa{\norm{y}^2 - \norm{y}^2} = 0 &  & \by{6.1.14}[b]                \\
		\implies & 0 = \inn{\zv, y} = \inn{0x, y}                                                                      &  & \by{1.2}[a]                   \\
		\implies & \forall n \in \Z^-, 0 = \inn{(n + (-n)) x, y}                                                                                          \\
		         & = \inn{nx, y} + \inn{(-n)x, y}                                                                      &  & \text{(from the proof above)} \\
		         & = \inn{nx, y} + (-n) \inn{x, y}                                                                     &  & (-n \in \Z^+)                 \\
		\implies & \forall n \in \Z^-, n \inn{x, y} = \inn{nx, y}.
	\end{align*}
	This implies \(\inn{nx, y} = n \inn{x, y}\) for all \(n \in \Z\).

	Next we show that \(m \inn{\frac{1}{m} x, y} = \inn{x, y}\) for all \(m \in \Z^+\).
	This is true since
	\begin{align*}
		\forall m \in \Z^+, m \inn{\frac{1}{m} x, y} & = \inn{m \frac{1}{m} x, y} &  & \text{(from the proof above)} \\
		                                             & = \inn{x, y}.
	\end{align*}
	This implies \(\inn{\frac{1}{m} x, y} = \frac{1}{m} \inn{x, y}\) for all \(m \in \Z^+\).

	Next we show that \(\inn{rx, y} = r \inn{x, y}\) for all \(r \in \Q\).
	This is true since
	\begin{align*}
		         & \forall r \in \Q, \exists a, b \in \Z : \begin{dcases}
			                                                   b > 0 \\
			                                                   r = a / b
		                                                   \end{dcases}                                                               \\
		\implies & \forall r \in \Q, \inn{rx, y} = \inn{\frac{a}{b} x, y} = \frac{1}{b} \inn{ax, y} &  & \text{(from the proof above)} \\
		         & = \frac{a}{b} \inn{x, y} = r \inn{x, y}.                                         &  & \text{(from the proof above)}
	\end{align*}

	Next we show that \(\abs{\inn{x, y}} \leq \norm{x} \norm{y}\).
	Observe that
	\begin{align*}
		         & \norm{x + y}^2 = 2 \norm{x}^2 - \norm{x - y}^2 + 2 \norm{y}^2       &  & \by{ex:6.1.11} \\
		         & = 2 \norm{x}^2 + 4 \inn{x, y} + 2 \norm{y}^2 - \norm{x + y}^2                           \\
		\implies & \norm{x + y}^2 = \norm{x}^2 + 2 \inn{x, y} + \norm{y}^2                                 \\
		\implies & \norm{x}^2 + 2 \inn{x, y} + \norm{y}^2 \leq (\norm{x} + \norm{y})^2 &  & \by{6.1.14}[c] \\
		         & = \norm{x}^2 + 2 \norm{x} \norm{y} + \norm{y}^2                                         \\
		\implies & \inn{x, y} \leq \norm{x} \norm{y}
	\end{align*}
	and
	\begin{align*}
		         & \norm{x - y}^2 = 2 \norm{x}^2 - \norm{x + y}^2 + 2 \norm{y}^2        &  & \by{ex:6.1.11} \\
		         & = 2 \norm{x}^2 - 4 \inn{x, y} + 2 \norm{y}^2 - \norm{x - y}^2                            \\
		\implies & \norm{x - y}^2 = \norm{x}^2 - 2 \inn{x, y} + \norm{y}^2                                  \\
		\implies & \norm{x}^2 - 2 \inn{x, y} + \norm{y}^2 \leq (\norm{x} + \norm{-y})^2 &  & \by{6.1.14}[c] \\
		         & = (\norm{x} + \norm{y})^2                                            &  & \by{6.1.14}[b] \\
		         & = \norm{x}^2 + 2 \norm{x} \norm{y} + \norm{y}^2                                          \\
		\implies & - \inn{x, y} \leq \norm{x} \norm{y}.
	\end{align*}
	Thus
	\[
		\begin{dcases}
			\inn{x, y} \leq \norm{x} \norm{y} \\
			-\inn{x, y} \leq \norm{x} \norm{y}
		\end{dcases} \implies \abs{\inn{x, y}} \leq \norm{x} \norm{y}.
	\]

	Next we show that
	\[
		\abs{c \inn{x, y} - \inn{cx, y}} = \abs{(c - r) \inn{x, y} - \inn{(c - r) x, y}} \leq 2 \abs{c - r} \norm{x} \norm{y}
	\]
	for all \((c, r) \in \R \times \Q\).
	This is true since
	\begin{align*}
		 & \forall (c, r) \in \R \times \Q, \abs{c \inn{x, y} - \inn{cx, y}}                                               \\
		 & = \abs{(c - r + r) \inn{x, y} - \inn{(c - r + r) x, y}}                                                         \\
		 & = \abs{(c - r) \inn{x, y} + r \inn{x, y} - \inn{(c - r)x + rx, y}}                                              \\
		 & = \abs{(c - r) \inn{x, y} + r \inn{x, y} - \inn{(c - r)x, y} - \inn{rx, y}}  &  & \text{(from the proof above)} \\
		 & = \abs{(c - r) \inn{x, y} + r \inn{x, y} - \inn{(c - r)x, y} - r \inn{x, y}} &  & \text{(from the proof above)} \\
		 & = \abs{(c - r) \inn{x, y} - \inn{(c - r) x, y}}                                                                 \\
		 & \leq \abs{(c - r) \inn{x, y}} + \abs{\inn{(c - r) x, y}}                     &  & \by{d.3}[c]                   \\
		 & = \abs{c - r} \abs{\inn{x, y}} + \abs{\inn{(c - r) x, y}}                                                       \\
		 & \leq \abs{c - r} \abs{\inn{x, y}} + \norm{(c - r) x} \norm{y}                &  & \text{(from the proof above)} \\
		 & = \abs{c - r} \abs{\inn{x, y}} + \abs{c - r} \norm{x} \norm{y}               &  & \by{6.1.14}[b]                \\
		 & \leq 2 \abs{c - r} \norm{x} \norm{y}.                                        &  & \text{(from the proof above)}
	\end{align*}

	Next we show that \(c \inn{x, y} = \inn{cx, y}\) for all \(c \in \R\).
	This is true since
	\begin{align*}
		         & \forall c \in \R, \exists (r_n)_{n = 0}^\infty \subseteq \Q : \lim_{n \to \infty} r_n = c                                                 \\
		\implies & \forall c \in \R, \exists (r_n)_{n = 0}^\infty \subseteq \Q :                                                                             \\
		         & 0 \leq \abs{c \inn{x, y} - \inn{cx, y}} \leq \lim_{n \to \infty} 2 \abs{c - r_n} \norm{x} \norm{y} = 0 &  & \text{(from the proof above)} \\
		\implies & \forall c \in \R, \abs{c \inn{x, y} - \inn{cx, y}} = 0                                                                                    \\
		\implies & \forall c \in \R, c \inn{x, y} - \inn{cx, y} = 0                                                                                          \\
		\implies & \forall c \in \R, c \inn{x, y} = \inn{cx, y}.
	\end{align*}
	Thus \cref{6.1.1}(b) is satisfied.

	Next we show that \cref{6.1.1}(c) is satisfied.
	This is true since
	\begin{align*}
		\conj{\inn{x, y}} & = \inn{x, y}                                              &  & (\inn{x, y} \in \R) \\
		                  & = \frac{1}{4} \pa{\norm{x + y}^2 - \norm{x - y}^2}                                 \\
		                  & = \frac{1}{4} \pa{\norm{y + x}^2 - \norm{(-1) (y - x)}^2} &  & \by{1.2.1}          \\
		                  & = \frac{1}{4} \pa{\norm{y + x}^2 - \norm{y - x}^2}        &  & \by{6.1.14}[b]      \\
		                  & = \inn{y, x}.
	\end{align*}

	Finally we show that \cref{6.1.1}(d) is satisfied.
	Suppose that \(x \neq \zv\).
	Then we have
	\begin{align*}
		\inn{x, x} & = \frac{1}{4} \pa{\norm{2x}^2 - \norm{\zv}^2}                      \\
		           & = \norm{x}^2                                  &  & \by{6.1.14}[ab] \\
		           & > 0.                                          &  & \by{6.1.14}[a]
	\end{align*}
	We conclude by \cref{6.1.1} that \(\inn{\cdot, \cdot}\) is an inner product on \(\V\) over \(\R\).
\end{proof}

\begin{ex}\label{ex:6.1.28}
	Let \(\V\) be an inner product space over \(\C\) with an inner product \(\inn{\cdot, \cdot}\).
	Let \([\cdot, \cdot]\) be the real-valued function such that \([x, y]\) is the real part of the complex number \(\inn{x, y}\) for all \(x, y \in \V\).
	Prove that \([\cdot, \cdot]\) is an inner product for \(\V\) over \(\R\) (instead of \(\C\)).
	Prove, furthermore, that \([x, ix] = 0\) for all \(x \in \V\).
\end{ex}

\begin{proof}[\pf{ex:6.1.28}]
	Let \(x, y, z \in \V\) and let \(c \in \R\).
	Then we have
	\begin{align*}
		[x + y, z]    & = \Re(\inn{x + y, z})               &  & \by{ex:6.1.28} \\
		              & = \Re(\inn{x, z} + \inn{y, z})      &  & \by{6.1.1}[a]  \\
		              & = \Re(\inn{x, z}) + \Re(\inn{y, z}) &  & \by{d.0.1}     \\
		              & = [x, z] + [y, z]                   &  & \by{ex:6.1.28} \\
		[cx, y]       & = \Re(\inn{cx, y})                  &  & \by{ex:6.1.28} \\
		              & = \Re(c \inn{x, y})                 &  & \by{6.1.1}[b]  \\
		              & = c \Re(\inn{x, y})                 &  & \by{d.0.1}     \\
		              & = c [x, y]                          &  & \by{ex:6.1.28} \\
		\conj{[x, y]} & = \conj{\Re(\inn{x, y})}            &  & \by{ex:6.1.28} \\
		              & = \Re(\conj{\inn{x, y}})            &  & \by{d.0.4}     \\
		              & = \Re(\inn{y, x})                   &  & \by{6.1.1}[c]  \\
		              & = [y, x].                           &  & \by{ex:6.1.28}
	\end{align*}
	If \(x \neq \zv\), then we have
	\begin{align*}
		         & \inn{x, x} > 0                             &  & \by{6.1.1}[d]                           \\
		\implies & \inn{x, x} \in \R                          &  & \text{(\(\C\) is not an ordered field)} \\
		\implies & [x, x] = \Re(\inn{x, x}) = \inn{x, x} > 0. &  & \by{ex:6.1.28}
	\end{align*}
	Thus by \cref{6.1.1} \([\cdot, \cdot]\) is an inner product on \(\V\) over \(\R\).

	Now we show that \([x, ix] = 0\) for all \(x \in \V\).
	This is true since
	\begin{align*}
		\forall x \in \V, [x, ix] & = \Re(\inn{x, ix})   &  & \by{ex:6.1.28}      \\
		                          & = \Re(-i \inn{x, x}) &  & \by{6.1}[b]         \\
		                          & = 0.                 &  & (\inn{x, x} \in \R)
	\end{align*}
\end{proof}

\begin{ex}\label{ex:6.1.29}
	Let \(\V\) be a vector space over \(\C\), and suppose that \([\cdot, \cdot]\) is an inner product on \(\V\) over \(\R\) (instead of \(\C\)), such that \([x, ix] = 0\) for all \(x \in \V\).
	Let \(\inn{\cdot, \cdot}\) be the complex-valued function defined by
	\[
		\inn{x, y} = [x, y] + i[x, iy] \quad \text{for } x, y \in \V.
	\]
	Prove that \(\inn{\cdot, \cdot}\) is an inner product on \(\V\) over \(\C\).
\end{ex}

\begin{proof}[\pf{ex:6.1.29}]
	Let \(x, y, z \in \V\) and let \(c \in \C\).
	First observe that
	\begin{align*}
		0 & = [x + iy, i(x + iy)]                   &  & \text{(by hypothesis)} \\
		  & = [x + iy, ix - y]                                                  \\
		  & = [x, ix - y] + [iy, ix - y]            &  & \by{6.1.1}[a]          \\
		  & = [x, ix] - [x, y] + [iy, ix] - [iy, y] &  & \by{6.1}[ab]           \\
		  & = [x, ix] - [x, y] + [ix, iy] - [y, iy] &  & \by{6.1.1}[c]          \\
		  & = [ix, iy] - [x, y].                    &  & \text{(by hypothesis)}
	\end{align*}
	Thus \([x, y] = [ix, iy]\) and
	\begin{align*}
		[ix, y] & = [i(ix), iy]                    \\
		        & = [-x, iy]                       \\
		        & = -[x, iy].   &  & \by{6.1.1}[a]
	\end{align*}
	Then we have
	\begin{align*}
		\inn{x + y, z}    & = [x + y, z] + i[x + y, iz]                                 &  & \by{ex:6.1.29}                \\
		                  & = [x, z] + [y, z] + i[x, iz] + i[y, iz]                     &  & \by{6.1.1}[a]                 \\
		                  & = \inn{x, z} + \inn{y, z}                                   &  & \by{ex:6.1.29}                \\
		\inn{cx, y}       & = [cx, y] + i[cx, iy]                                       &  & \by{ex:6.1.29}                \\
		                  & = [(\Re(c) + i \Im(c)) x, y] + i[(\Re(c) + i \Im(c)) x, iy]                                    \\
		                  & = \Re(c) [x, y] + \Im(c) [ix, y]                            &  & \by{6.1.1}[ab]                \\
		                  & \quad + i \Re(c) [x, iy] + i \Im(c) [ix, iy]                &  & \by{6.1.1}[ab]                \\
		                  & = \Re(c) ([x, y] + i[x, iy])                                                                   \\
		                  & \quad + i \Im(c) ([ix, iy] - i[ix, y])                                                         \\
		                  & = \Re(c) ([x, y] + i[x, iy])                                                                   \\
		                  & \quad + i \Im(c) ([x, y] + i[x, iy])                        &  & \text{(from the proof above)} \\
		                  & = c \inn{x, y}                                              &  & \by{ex:6.1.29}                \\
		\conj{\inn{x, y}} & = \conj{[x, y] + i[x, iy]}                                  &  & \by{ex:6.1.29}                \\
		                  & = [x, y] - i[x, iy]                                         &  & \by{d.0.4}                    \\
		                  & = [y, x] - i[iy, x]                                         &  & \by{6.1.1}[c]                 \\
		                  & = [y, x] + i[y, ix]                                         &  & \text{(from the proof above)} \\
		                  & = \inn{y, x}.                                               &  & \by{ex:6.1.29}
	\end{align*}
	If \(x \neq \zv\), then we have
	\begin{align*}
		         & [x, x] > 0                                   &  & \by{6.1.1}[d]  \\
		\implies & \inn{x, x} = [x, x] + i[x, ix] = [x, x] > 0. &  & \by{ex:6.1.29}
	\end{align*}
	Thus by \cref{6.1.1} \(\inn{\cdot, \cdot}\) is an inner product on \(V\) over \(\C\).
\end{proof}

\begin{ex}\label{ex:6.1.30}
	Let \(\norm{\cdot}\) be a norm (as in \cref{6.1.14}) on a vector space \(\V\) over \(\C\) satisfying the parallelogram law given in \cref{ex:6.1.11}.
	Prove that there is an inner product \(\inn{\cdot, \cdot}\) on \(\V\) over \(\C\) such that \(\norm{x}^2 = \inn{x, x}\) for all \(x \in \V\).
\end{ex}

\begin{proof}[\pf{ex:6.1.30}]
	First we define a real-valued function \([\cdot, \cdot]\) as follow:
	\[
		\forall x, y \in \V, [x, y] = \frac{1}{4} \pa{\norm{x + y}^2 - \norm{x - y}^2}.
	\]
	By \cref{ex:6.1.27} we see that \([x, y]\) is an inner product on \(\V\) over \(\R\).
	Observe that
	\begin{align*}
		[x, ix] & = \frac{1}{4} \pa{\norm{x + ix}^2 - \norm{x - ix}^2}                         \\
		        & = \frac{1}{4} \pa{\norm{x + ix}^2 - \norm{(-i) (x + ix)}^2}                  \\
		        & = \frac{1}{4} \pa{\norm{x + ix}^2 - \norm{x + ix}^2}        &  & \by{6.2}[a] \\
		        & = 0.
	\end{align*}
	Now we define a complex-valued function \(\inn{\cdot, \cdot}\) as follow:
	\[
		\forall x, y \in \V, \inn{x, y} = [x, y] + i[x, iy].
	\]
	By \cref{ex:6.1.29} we see that \(\inn{\cdot, \cdot}\) is an inner product on \(\V\) over \(\C\).
	Then we have
	\begin{align*}
		\forall x \in \V, \norm{x}^2 & = [x, x]            &  & \by{ex:6.1.27}                \\
		                             & = [x, x] + 0                                           \\
		                             & = [x, x] + i[x, ix] &  & \text{(from the proof above)} \\
		                             & = \inn{x, x}.       &  & \by{ex:6.1.29}
	\end{align*}
\end{proof}

\section{The Gram-Schmidt Orthogonalization Process and Orthogonal Complements}\label{sec:6.2}

\begin{defn}\label{6.2.1}
  Let \(\V\) be an inner product space over \(\F\).
  A subset of \(\V\) is an \textbf{orthonormal basis} for \(\V\) over \(\F\) if it is an ordered basis that is orthonormal.
\end{defn}

\begin{eg}\label{6.2.2}
  The standard ordered basis for \(\vs{F}^n\) over \(\F\) is an orthonormal basis for \(\vs{F}^n\) over \(\F\).
\end{eg}

\begin{thm}\label{6.3}
  Let \(\V\) be an inner product space over \(\F\) and \(S = \set{\seq{v}{1,,k}}\) be an orthogonal subset of \(\V\) consisting of nonzero vectors.
  If \(y \in \spn{S}\), then
  \[
    y = \sum_{i = 1}^k \frac{\inn{y, v_i}}{\norm{v_i}^2} v_i.
  \]
\end{thm}

\begin{proof}[\pf{6.3}]
  Write \(y = \sum_{i = 1}^k a_i v_i\), where \(\seq{a}{1,,k} \in \F\).
  Then, for \(j \in \set{1, \dots, k}\), we have
  \begin{align*}
    \inn{y, v_j} & = \inn{\sum_{i = 1}^k a_i v_i, v_j}                                     \\
                 & = \sum_{i = 1}^k a_i \inn{v_i, v_j} &  & \text{(by \cref{6.1.1}(a)(b))} \\
                 & = a_j \inn{v_j, v_j}                &  & \text{(by \cref{6.1.12})}      \\
                 & = a_j \norm{v_j}^2.                 &  & \text{(by \cref{6.1.9})}
  \end{align*}
  So \(a_j = \frac{\inn{y, v_j}}{\norm{v_j}^2}\), and the result follows.
\end{proof}

\begin{cor}\label{6.2.3}
  If, in addition to the hypotheses of \cref{6.3}, \(S\) is orthonormal and \(y \in \spn{S}\), then
  \[
    y = \sum_{i = 1}^k \inn{y, v_i} v_i.
  \]
\end{cor}

\begin{proof}[\pf{6.2.3}]
  We have
  \begin{align*}
    y & = \sum_{i = 1}^k \frac{\inn{y, v_i}}{\norm{v_i}^2} v_i &  & \text{(by \cref{6.3})}    \\
      & = \sum_{i = 1}^k \inn{y, v_i} v_i.                     &  & \text{(by \cref{6.1.12})}
  \end{align*}
\end{proof}

\begin{note}
  If \(\V\) possesses a finite orthonormal basis, then \cref{6.2.3} allows us to compute the coefficients in a linear combination very easily.
\end{note}

\begin{cor}\label{6.2.4}
  Let \(\V\) be an inner product space over \(\F\), and let \(S\) be an orthogonal subset of \(\V\) consisting of nonzero vectors.
  Then \(S\) is linearly independent.
\end{cor}

\begin{proof}[\pf{6.2.4}]
  Suppose that \(\seq{v}{1,,k} \in \S\) and
  \[
    \sum_{i = 1}^k a_i v_i = \zv.
  \]
  As in the proof of \cref{6.3} with \(y = \zv\), we have \(a_j = \inn{\zv, v_j} / \norm{v_j}^2 = 0\) for all \(j \in \set{1, \dots, k}\).
  So \(S\) is linearly independent.
\end{proof}

\begin{note}
  \cref{6.2.4} tells us that the vector space \(\vs{H}\) in \cref{6.1.13} contains an infinite linearly independent set, and hence \(\vs{H}\) is not a finite-dimensional vector space.
\end{note}

\begin{thm}\label{6.4}
  Let \(\V\) be an inner product space over \(\F\) and \(S = \set{\seq{w}{1,,n}}\) be a linearly independent subset of \(\V\).
  Define \(S' = \set{\seq{v}{1,,n}}\), where \(v_1 = w_1\) and
  \begin{equation}\label{eq:6.2.1}
    v_k = w_k - \sum_{j = 1}^{k - 1} \frac{\inn{w_k, v_j}}{\norm{v_j}^2} v_j \quad \text{for } k \in \set{2, \dots, n}.
  \end{equation}
  Then \(S'\) is an orthogonal set of nonzero vectors such that \(\spn{S'} = \spn{S}\).
  The construction of \(\set{\seq{v}{1,,n}}\) is called the \textbf{Gram--Schmidt process}.
\end{thm}

\begin{proof}[\pf{6.4}]
  The proof is by mathematical induction on \(n\), the number of vectors in \(S\).
  For \(k \in \set{1, \dots, n}\), let \(S_k = \set{\seq{w}{1,,k}}\).
  If \(n = 1\), then the theorem is proved by taking \(S_1' = S_1\);
  i.e., \(v_1 = w_1 \neq \zv\).
  Assume then that the set \(S_n' = \set{\seq{v}{1,,n}}\) with the desired properties has been constructed by the repeated use of \cref{eq:6.2.1}.
  We show that the set \(S_{n + 1}' = \set{\seq{v}{1,,n+1}}\) also has the desired properties, where \(v_{n + 1}\) is obtained from \(S_n'\) by \cref{eq:6.2.1}.
  If \(v_{n + 1} = \zv\), then \cref{eq:6.2.1} implies that \(w_{n + 1} \in \spn{S_n'} = \spn{S_n}\), which contradicts the assumption that \(S_{n + 1}\) is linearly independent.
  For \(i \in \set{1, \dots, n}\), it follows from \cref{eq:6.2.1} that
  \begin{align*}
    \inn{v_{n + 1}, v_i} & = \inn{w_{n + 1}, v_i} - \sum_{j = 1}^n \frac{\inn{w_{n + 1}, v_j}}{\norm{v_j}^2} \inn{v_j, v_i} &  & \text{(by \cref{6.1.1}(a)(b))}   \\
                         & = \inn{w_{n + 1}, v_i} - \frac{\inn{w_{n + 1}, v_i}}{\norm{v_i}^2} \norm{v_i}^2                  &  & \text{(by induction hypotheses)} \\
                         & = 0,
  \end{align*}
  since \(\inn{v_j, v_i} = 0\) if \(i \neq j\) by the induction assumption that \(S_n'\) is orthogonal.
  Hence \(S_{n + 1}'\) is an orthogonal set of nonzero vectors.
  Now, by \cref{eq:6.2.1}, we have that \(\spn{S_{n + 1}'} \subseteq \spn{S_{n + 1}}\).
  But by \cref{6.2.4}, \(S_{n + 1}'\) is linearly independent;
  so \(\dim(\spn{S_{n + 1}'}) = \dim(\spn{S_{n + 1}}) = n + 1\).
  Therefore by \cref{1.11} we have \(\spn{S_{n + 1}'} = \spn{S_{n + 1}}\) and this closes the induction.
\end{proof}

\begin{defn}\label{6.2.5}
  If we continue applying the Gram--Schmidt orthogonalization process to the basis \(\set{1, x, x^2, \dots}\) for \(\ps{\R}\), we obtain an orthogonal basis whose elements are called the \emph{Legendre polynomials}.
\end{defn}

\begin{thm}\label{6.5}
  Let \(\V\) be a nonzero finite-dimensional inner product space over \(\F\).
  Then \(\V\) has an orthonormal basis \(\beta\).
  Furthermore, if \(\beta = \set{\seq{v}{1,,n}}\) and \(x \in \V\), then
  \[
    x = \sum_{i = 1}^n \inn{x, v_i} v_i.
  \]
\end{thm}

\begin{proof}[\pf{6.5}]
  Let \(\beta_0\) be an ordered basis for \(\V\) over \(\F\).
  Apply \cref{6.4} to obtain an orthogonal set \(\beta'\) of nonzero vectors with \(\spn{\beta'} = \spn{\beta_0} = \V\).
  By normalizing each vector in \(\beta'\), we obtain an orthonormal set \(\beta\) that generates \(\V\).
  By \cref{6.2.4}, \(\beta\) is linearly independent;
  therefore \(\beta\) is an orthonormal basis for \(\V\) over \(\F\).
  The remainder of the theorem follows from \cref{6.2.3}.
\end{proof}

\begin{cor}\label{6.2.6}
  Let \(\V\) be a finite-dimensional inner product space over \(\F\) with an orthonormal basis \(\beta = \set{\seq{v}{1,,n}}\).
  Let \(\T \in \ls(\V)\), and let \(A = [\T]_{\beta}\).
  Then for any \(i, j \in \set{1, \dots, n}\), \(A_{i j} = \inn{\T(v_j), v_i}\).
\end{cor}

\begin{proof}[\pf{6.2.6}]
  From \cref{6.5}, we have
  \[
    \T(v_j) = \sum_{i = 1}^n \inn{\T(v_j), v_i} v_i.
  \]
  Hence by \cref{2.2.4} \(A_{i j} = \inn{\T(v_j), v_i}\).
\end{proof}

\begin{defn}\label{6.2.7}
  Let \(\beta\) be an orthonormal subset (possibly infinite) of an inner product space \(\V\) over \(\F\), and let \(x \in \V\).
  We define the \textbf{Fourier coefficients} of \(x\) relative to \(\beta\) to be the scalars \(\inn{x, y}\), where \(y \in \beta\).

  In the first half of the 19th century, the French mathematician Jean Baptiste Fourier was associated with the study of the scalars
  \[
    \int_0^{2\pi} f(t) \sin(nt) \; dt \quad \text{and} \quad \int_0^{2\pi} f(t) \cos(nt) \; dt,
  \]
  or more generally,
  \[
    c_n = \frac{1}{2\pi} \int_0^{2\pi} f(t) e^{-int} \; dt,
  \]
  for a function \(f\).
  In the context of \cref{6.1.8}, we see that \(c_n = \inn{f, f_n}\), where \(f_n(t) = e^{int}\);
  that is, \(c_n\) is the \(n\)th Fourier coefficient for a continuous function \(f \in \V\) relative to \(S\).
  These coefficients are the ``classical'' Fourier coefficients of a function, and the literature concerning the behavior of these coefficients is extensive.
\end{defn}

\begin{eg}\label{6.2.8}
  Let \(S = \set{e^{int} : n \text{ is an integer}}\).
  In \cref{6.1.13}, \(S\) was shown to be an orthonormal set in \(\vs{H}\).
  We compute the Fourier coefficients of \(f(t) = t\) relative to \(S\).
  Using integration by parts, we have, for \(n \neq 0\),
  \begin{align*}
    \inn{f, f_n} & = \frac{1}{2\pi} \int_0^{2\pi} t \conj{e^{int}} \; dt                                         \\
                 & = \frac{1}{2\pi} \int_0^{2\pi} t e^{-int} \; dt                                               \\
                 & = \frac{1}{2\pi} \pa{\eval{\frac{-1}{in} t e^{-int}}_0^{2\pi} - \int_0^{2\pi} e^{-int} \; dt} \\
                 & = \frac{1}{2\pi} \pa{\frac{-2\pi}{in} - \frac{-1}{in} \eval{e^{-int}}_0^{2\pi}}               \\
                 & = \frac{-1}{in},
  \end{align*}
  and, for \(n = 0\),
  \[
    \inn{f, 1} = \frac{1}{2\pi} \int_0^{2\pi} t(1) \; dt = \pi.
  \]
  As a result of these computations, and using \cref{ex:6.2.16}, we obtain an upper bound for the sum of a special infinite series as follows:
  \begin{align*}
    \norm{f}^2 & \geq \sum_{n = -k}^{-1} \abs{\inn{f, f_n}}^2 + \abs{\inn{f, 1}}^2 + \sum_{i = 1}^k \abs{\inn{f, f_n}}^2 \\
               & = \sum_{n = -k}^1 \frac{1}{n^2} + \pi^2 + \sum_{n = 1}^k \frac{1}{n^2}                                  \\
               & = 2 \sum_{n = 1}^k \frac{1}{n^2} + \pi^2
  \end{align*}
  for every \(k \in \Z^+\).
  Now, using the fact that \(\norm{f}^2 = \frac{4}{3} \pi^2\), we obtain
  \[
    \frac{4}{3} \pi^2 \geq 2 \sum_{n = 1}^k \frac{1}{n^2} + \pi^2,
  \]
  or
  \[
    \frac{\pi^2}{6} \geq \sum_{n = 1}^k \frac{1}{n^2}.
  \]
  Because this inequality holds for all \(k \in \Z^+\), we may let \(k \to \infty\) to obtain
  \[
    \frac{\pi^2}{6} \geq \sum_{n = 1}^\infty \frac{1}{n^2}.
  \]
  Additional results may be produced by replacing \(f\) by other functions.
\end{eg}

\exercisesection

\begin{ex}\label{ex:6.2.16}

\end{ex}

\section{The Adjoint of a Linear Operator}\label{sec:6.3}

\begin{thm}\label{6.8}
	Let \(\V\) be a finite-dimensional inner product space over \(\F\), and let \(g \in \ls(\V, \F)\).
	Then there exists a unique vector \(y \in \V\) such that \(g(x) = \inn{x, y}\) for all \(x \in \V\).
\end{thm}

\begin{proof}[\pf{6.8}]
	Let \(\beta = \set{\seq{v}{1,,n}}\) be an orthonormal basis for \(\V\) over \(\F\), and let
	\[
		y = \sum_{i = 1}^n \conj{g(v_i)} v_i.
	\]
	Define \(h : \V \to \F\) by \(h(x) = \inn{x, y}\), which is clearly linear (by \cref{6.1.1}(a)(b)).
	Furthermore, for \(j \in \set{1, \dots, n}\) we have
	\begin{align*}
		h(v_j) & = \inn{v_j, y}                                                  \\
		       & = \inn{v_j, \sum_{i = 1}^n \conj{g(v_i)} v_i}                   \\
		       & = \sum_{i = 1}^n g(v_i) \inn{v_j, v_i}        &  & \by{6.1}[ab] \\
		       & = \sum_{i = 1}^n g(v_i) \delta_{j i}          &  & \by{6.1.12}  \\
		       & = g(v_j).
	\end{align*}
	Since \(g\) and \(h\) both agree on \(\beta\), we have that \(g = h\) by \cref{2.1.13}.
	To show that \(y\) is unique, suppose that \(g(x) = \inn{x, y'}\) for all \(x \in \V\).
	Then \(\inn{x, y} = \inn{x, y'}\) for all \(x \in \V\);
	so by \cref{6.1}(e), we have \(y = y'\).
\end{proof}

\begin{thm}\label{6.9}
	Let \(\V\) be a finite-dimensional inner product space over \(\F\), and let \(\T \in \ls(\V)\).
	Then there exists a unique \(\T^* \in \ls(\V)\) such that \(\inn{\T(x), y} = \inn{x, \T^*(y)}\) for all \(x, y \in \V\).
	\(\T^*\) is called the \textbf{adjoint} of the operator \(\T\).
	The symbol \(\T^*\) is read ``\(\T\) star.''
\end{thm}

\begin{proof}[\pf{6.9}]
	Let \(y \in \V\).
	Define \(g : \V \to \F\) by \(g(x) = \inn{\T(x), y}\) for all \(x \in \V\).
	We first show that \(g\) is linear.
	Let \(x_1, x_2 \in \V\) and \(c \in \F\).
	Then
	\begin{align*}
		g(c x_1 + x_2) & = \inn{\T(c x_1 + x_2), y}                                  \\
		               & = \inn{c \T(x_1) + \T(x_2), y}          &  & \by{2.1.2}[b]  \\
		               & = c \inn{\T(x_1), y} + \inn{\T(x_2), y} &  & \by{6.1.1}[ab] \\
		               & = c g(x_1) + g(x_2).
	\end{align*}
	Hence by \cref{2.1.2}(b) \(g \in \ls(\V, \F)\).

	We now apply \cref{6.8} to obtain a unique vector \(y' \in \V\) such that \(g(x) = \inn{x, y'}\);
	that is, \(\inn{\T(x), y} = \inn{x, y'}\) for all \(x \in \V\).
	Defining \(\T^* : \V \to \V\) by \(\T^*(y) = y'\), we have \(\inn{\T(x), y} = \inn{x, \T^*(y)}\).

	To show that \(\T^*\) is linear, let \(y_1, y_2 \in \V\) and \(c \in \F\).
	Then for any \(x \in \V\),
	we have
	\begin{align*}
		\inn{x, \T^*(c y_1 + y_2)} & = \inn{\T(x), c y_1 + y_2}                                           \\
		                           & = \conj{c} \inn{\T(x), y_1} + \inn{\T(x), y_2}     &  & \by{6.1}[ab] \\
		                           & = \conj{c} \inn{x, \T^*(y_1)} + \inn{x, \T^*(y_2)}                   \\
		                           & = \inn{x, c \T^*(y_1) + \T^*(y_2)}.                &  & \by{6.1}[ab]
	\end{align*}
	Since \(x\) is arbitrary, \(\T^*(c y_1 + y_2) = c \T^*(y_1) + \T^*(y_2)\) by \cref{6.1}(e).

	Finally, we need to show that \(\T^*\) is unique. Suppose that \(\U \in \ls(\V)\) and that it satisfies \(\inn{\T(x), y} = \inn{x, \U(y)}\) for all \(x, y \in \V\).
	Then \(\inn{x, \T^*(y)} = \inn{x, \U(y)}\) for all \(x, y \in \V\), so \(\T^* = \U\).
\end{proof}

\begin{note}
	Thus by \cref{6.9} \(\T^*\) is the unique operator on \(\V\) satisfying \(\inn{\T(x), y} = \inn{x, \T^*(y)}\) for all \(x, y \in \V\).
	Note that we also have
	\begin{align*}
		\inn{x, \T(y)} & = \conj{\inn{\T(y), x}}   &  & \by{6.1.1}[c] \\
		               & = \conj{\inn{y, \T^*(x)}} &  & \by{6.9}      \\
		               & = \inn{\T^*(x), y};       &  & \by{6.1.1}[c]
	\end{align*}
	so \(\inn{x, \T(y)} = \inn{\T^*(x), y}\) for all \(x, y \in \V\).
	We may view these equations symbolically as adding a \(*\) to \(\T\) when shifting its position inside the inner product symbol.
\end{note}

\begin{note}
	For an infinite-dimensional inner product space, the adjoint of a linear operator \(\T\) may be defined to be the function \(\T^*\) such that \(\inn{\T(x), y} = \inn{x, \T^*(y)}\) for all \(x, y \in \V\), provided it exists.
	Although the uniqueness and linearity of \(\T^*\) follow as before, the existence of the adjoint is not guaranteed (see \cref{ex:6.3.24}).
	The reader should observe the necessity of the hypothesis of finite-dimensionality in the proof of \cref{6.8}.
	Many of the theorems we prove about adjoints, nevertheless, do not depend on \(\V\) being finite-dimensional.
	\emph{Thus, unless stated otherwise, for the remainder of this chapter we adopt the convention that a reference to the adjoint of a linear operator on an infinite-dimensional inner product space assumes its existence}.
\end{note}

\begin{thm}\label{6.10}
	Let \(\V\) be a finite-dimensional inner product space over \(\F\), and let \(\beta\) be an orthonormal basis for \(\V\) over \(\F\).
	If \(\T \in \ls(\V)\), then
	\[
		[\T^*]_{\beta} = [\T]_{\beta}^*.
	\]
\end{thm}

\begin{proof}[\pf{6.10}]
	Let \(A = [\T]_{\beta}\), \(B = [\T^*]_{\beta}\), and \(\beta = \set{\seq{v}{1,,n}}\).
	Then we have
	\begin{align*}
		\forall i, j \in \set{1, \dots, n}, B_{i j} & = \inn{\T^*(v_j), v_i}        &  & \by{6.2.6}    \\
		                                            & = \conj{\inn{v_i, \T^*(v_j)}} &  & \by{6.1.1}[c] \\
		                                            & = \conj{\inn{\T(v_i), v_j}}   &  & \by{6.9}      \\
		                                            & = \conj{A_{j i}}              &  & \by{6.2.6}    \\
		                                            & = (A^*)_{i j}.                &  & \by{6.1.5}
	\end{align*}
	Hence \(B = A^*\).
\end{proof}

\begin{cor}\label{6.3.1}
	Let \(A \in \ms{n}{n}{\F}\).
	Then \(\L_{A^*} = (\L_A)^*\).
\end{cor}

\begin{proof}[\pf{6.3.1}]
	If \(\beta\) is the standard ordered basis for \(\vs{F}^n\), then, by \cref{2.15}(a), we have \([\L_A]_{\beta} = A\).
	Hence \([(\L_A)^*]_{\beta} = [\L_A]_{\beta}^* = A^* = [\L_{A^*}]_{\beta}\), and so \((\L_A)^* = \L_{A^*}\).
\end{proof}

\begin{thm}\label{6.11}
	Let \(\V\) be an inner product space over \(\F\) and let \(\T, \U \in \ls(\V)\).
	Then
	\begin{enumerate}
		\item \((\T + \U)^* = \T^* + \U^*\);
		\item \((c \T)^* = \conj{c} \T^*\) for any \(c \in \F\);
		\item \((\T \U)^* = \U^* \T^*\);
		\item \(\T^{**} = \T\);
		\item \(\IT[\V]^* = \IT[\V]\).
	\end{enumerate}
\end{thm}

\begin{proof}[\pf{6.11}(a)]
	Since
	\begin{align*}
		\forall x, y \in \V, \inn{x, (\T + \U)^*(y)} & = \inn{(\T + \U)(x), y}               &  & \by{6.9}      \\
		                                             & = \inn{\T(x) + \U(x), y}              &  & \by{2.2.5}    \\
		                                             & = \inn{\T(x), y} + \inn{\U(x), y}     &  & \by{6.1.1}[a] \\
		                                             & = \inn{x, \T^*(y)} + \inn{x, \U^*(y)} &  & \by{6.9}      \\
		                                             & = \inn{x, \T^*(y) + \U^*(y)}          &  & \by{6.1}[a]   \\
		                                             & = \inn{x, (\T^* + \U^*)(y)},          &  & \by{2.2.5}
	\end{align*}
	by \cref{6.1}(e) we know that \(\T^* + \U^* = (\T + \U)^*\).
\end{proof}

\begin{proof}[\pf{6.11}(b)]
	Since
	\begin{align*}
		\forall x, y \in \V, \inn{x, (c \T)^*(y)} & = \inn{(c \T)(x), y}           &  & \by{6.9}      \\
		                                          & = \inn{c \T(x), y}             &  & \by{2.2.5}    \\
		                                          & = c \inn{\T(x), y}             &  & \by{6.1.1}[b] \\
		                                          & = c \inn{x, \T^*(y)}           &  & \by{6.9}      \\
		                                          & = \inn{x, \conj{c} \T^*(y)}    &  & \by{6.1}[b]   \\
		                                          & = \inn{x, (\conj{c} \T^*)(y)}, &  & \by{2.2.5}
	\end{align*}
	by \cref{6.1}(e) we know that \((c \T)^* = \conj{c} \T^*\).
\end{proof}

\begin{proof}[\pf{6.11}(c)]
	Since
	\begin{align*}
		\forall x, y \in \V, \inn{x, (\T \U)^*(y)} & = \inn{(\T \U)(x), y}      &  & \by{6.9} \\
		                                           & = \inn{\T(\U(x)), y}                     \\
		                                           & = \inn{\U(x), \T^*(y)}     &  & \by{6.9} \\
		                                           & = \inn{x, \U^*(\T^*(y))}   &  & \by{6.9} \\
		                                           & = \inn{x, (\U^* \T^*)(y)},
	\end{align*}
	by \cref{6.1}(e) we know that \((\T \U)^* = \U^* \T^*\).
\end{proof}

\begin{proof}[\pf{6.11}(d)]
	Since
	\begin{align*}
		\forall x, y \in \V, \inn{x, \T(y)} & = \inn{\T^*(x), y}     &  & \by{6.9} \\
		                                    & = \inn{x, \T^{**}(y)}, &  & \by{6.9}
	\end{align*}
	by \cref{6.1}(e) we know that \(\T^{**} = \T\).
\end{proof}

\begin{proof}[\pf{6.11}(e)]
	Since
	\begin{align*}
		\forall x, y \in \V, \inn{x, \IT[\V](y)} & = \inn{x, y}             &  & \by{2.1.9} \\
		                                         & = \inn{\IT[\V](x), y}    &  & \by{2.1.9} \\
		                                         & = \inn{x, \IT[\V]^*(y)}, &  & \by{6.9}
	\end{align*}
	by \cref{6.1}(e) we know that \(\IT[\V]^* = \IT[\V]\).
\end{proof}

\begin{note}
	The same proof works for \cref{6.11} in the infinite-dimensional case, provided that the existence of \(\T^*\) and \(\U^*\) is assumed.
\end{note}

\begin{cor}\label{6.3.2}
	Let \(A, B \in \ms{n}{n}{\F}\).
	Then
	\begin{enumerate}
		\item \((A + B)^* = A^* + B^*\);
		\item \((cA)^* = \conj{c} A^*\) for all \(c \in \F\).
		\item \((AB)^* = B^* A^*\);
		\item \(A^{**} = A\);
		\item \(I_n^* = I_n\).
	\end{enumerate}
\end{cor}

\begin{proof}[\pf{6.3.2}(a)]
	Since
	\begin{align*}
		\L_{(A + B)^*} & = (\L_{A + B})^*      &  & \by{6.3.1}   \\
		               & = (\L_A + \L_B)^*     &  & \by{2.15}[c] \\
		               & = (\L_A)^* + (\L_B)^* &  & \by{6.11}[a] \\
		               & = \L_{A^*} + \L_{B^*} &  & \by{6.3.1}   \\
		               & = \L_{A^* + B^*},     &  & \by{2.15}[c]
	\end{align*}
	by \cref{2.15}(b) we have \((A + B)^* = A^* + B^*\).
\end{proof}

\begin{proof}[\pf{6.3.2}(b)]
	Since
	\begin{align*}
		\L_{(cA)^*} & = (\L_{cA})^*        &  & \by{6.3.1}   \\
		            & = (c \L_A)^*         &  & \by{2.15}[c] \\
		            & = \conj{c} (\L_A)^*  &  & \by{6.11}[b] \\
		            & = \conj{c} \L_{A^*}  &  & \by{6.3.1}   \\
		            & = \L_{\conj{c} A^*}, &  & \by{2.15}[c]
	\end{align*}
	by \cref{2.15}(b) we have \((cA)^* = \conj{c} A^*\).
\end{proof}

\begin{proof}[\pf{6.3.2}(c)]
	Since
	\begin{align*}
		\L_{(AB)^*} & = (\L_{AB})^*       &  & \by{6.3.1}   \\
		            & = (\L_A \L_B)^*     &  & \by{2.15}[e] \\
		            & = (\L_B)^* (\L_A)^* &  & \by{6.11}[c] \\
		            & = \L_{B^*} \L_{A^*} &  & \by{6.3.1}   \\
		            & = \L_{B^* A^*},     &  & \by{2.15}[e]
	\end{align*}
	by \cref{2.15}(b) we have \((AB)^* = B^* A^*\).
\end{proof}

\begin{proof}[\pf{6.3.2}(d)]
	Since
	\begin{align*}
		\L_A & = (\L_A)^{**}  &  & \by{6.11}[d] \\
		     & = (\L_{A^*})^* &  & \by{6.3.1}   \\
		     & = \L_{A^{**}}, &  & \by{6.3.1}
	\end{align*}
	by \cref{2.15}(b) we have \(A = A^{**}\).
\end{proof}

\begin{proof}[\pf{6.3.2}(e)]
	Since
	\begin{align*}
		\L_{I_n} & = \IT[\vs{F}^n]   &  & \by{2.15}[f] \\
		         & = \IT[\vs{F}^n]^* &  & \by{6.11}[e] \\
		         & = (\L_{I_n})^*    &  & \by{2.15}[f] \\
		         & = \L_{I_n^*},     &  & \by{6.3.1}
	\end{align*}
	by \cref{2.15}(b) we have \(I_n = I_n^*\).
\end{proof}

\begin{defn}\label{6.3.3}
	For \(x, y \in \vs{F}^n\), let \(\inn{x, y}_n\) denote the standard inner product of \(x\) and \(y\) in \(\vs{F}^n\).
	Recall that if \(x\) and \(y\) are regarded as column vectors, then \(\inn{x, y}_n = y^* x\).
\end{defn}

\begin{lem}\label{6.3.4}
	Let \(A \in \MS\), \(x \in \vs{F}^n\), and \(y \in \vs{F}^m\).
	Then
	\[
		\inn{Ax, y}_m = \inn{x, A^* y}_n.
	\]
\end{lem}

\begin{proof}[\pf{6.3.4}]
	We have
	\begin{align*}
		\inn{Ax, y}_m & = y^* (Ax)          &  & \by{6.3.3}        \\
		              & = (y^* A) x         &  & \by{2.16}         \\
		              & = (A^* y)^* x       &  & \by{ex:6.3.5}[cd] \\
		              & = \inn{x, A^* y}_n. &  & \by{6.3.3}
	\end{align*}
\end{proof}

\begin{lem}\label{6.3.5}
	Let \(A \in \MS\).
	Then \(\rk{A^* A} = \rk{A}\).
\end{lem}

\begin{proof}[\pf{6.3.5}]
	By the dimension theorem (\cref{2.3}), we need only show that, for \(x \in \vs{F}^n\), we have \(A^* Ax = \zv\) iff \(Ax = \zv\).
	Clearly, \(Ax = \zv\) implies that \(A^* Ax = \zv\).
	So assume that \(A^* Ax = \zv\).
	Then
	\begin{align*}
		0 & = \inn{A^* Ax, x}_n    &  & \by{6.1}[c]      \\
		  & = \inn{Ax, A^{**} x}_m &  & \by{6.3.4}       \\
		  & = \inn{Ax, Ax}_m,      &  & \by{ex:6.3.5}[d]
	\end{align*}
	so that \(Ax = \zv\) by \cref{6.1}(d).
\end{proof}

\begin{cor}\label{6.3.6}
	If \(A \in \MS\) such that \(\rk{A} = n\), then \(A^* A\) is invertible.
\end{cor}

\begin{proof}[\pf{6.3.6}]
	By \cref{6.3.5} we have \(n = \rk{A} = \rk{A^* A}\).
	Since \(A^* A \in \ms{n}{n}{\F}\), by \cref{3.2.2} we know that \(A^* A\) is invertible.
\end{proof}

\begin{thm}\label{6.12}
	Let \(A \in \MS\) and \(y \in \vs{F}^m\).
	Then there exists \(x_0 \in \vs{F}^n\) such that \((A^* A) x_0 = A^* y\) and \(\norm{A x_0 - y} \leq \norm{Ax - y}\) for all \(x \in \vs{F}^n\).
	Furthermore, if \(\rk{A} = n\), then \(x_0 = (A^* A)^{-1} A^* y\).
\end{thm}

\begin{proof}[\pf{6.12}]
	Define \(\W = \set{Ax : x \in \vs{F}^n}\);
	that is, \(\W = \rg{\L_A}\).
	By \cref{6.2.12}, there exists a unique vector in \(\W\) that is closest to \(y\).
	Call this vector \(A x_0\), where \(x_0 \in \vs{F}^n\).
	Then \(\norm{A x_0 - y} \leq \norm{Ax - y}\) for all \(x \in \vs{F}^n\).

	To develop a practical method for finding such an \(x_0\), we note from \cref{6.6,6.2.12} that \(A x_0 - y \in \W^{\perp}\);
	so \(\inn{Ax, A x_0 - y}_m = 0\) for all \(x \in \vs{F}^n\).
	Thus, by \cref{6.3.4}, we have that \(\inn{x, A^* (A x_0 - y)}_n = 0\) for all \(x \in \vs{F}^n\).
	By \cref{6.1}(e) we have \(A^* (A x_0 - y) = 0\).
	So we need only find a solution \(x_0\) to \(A^* Ax = A^* y\).
	If, in addition, we assume that \(\rk{A} = n\), then by \cref{6.3.5} we have \(x_0 = (A^* A)^{-1} A^* y\).
\end{proof}

\begin{note}
	Consider the following problem:
	An experimenter collects data by taking measurements \(\seq{y}{1,,m}\) at times \(\seq{t}{1,,m}\), respectively.
	Suppose that the data \((t_1, y_1), \dots, (t_m, y_m)\) are plotted as points in the plane.
	From this plot, the experimenter feels that there exists an essentially linear relationship between \(y\) and \(t\), say \(y = ct + d\), and would like to find the constants \(c\) and \(d\) so that the line \(y = ct + d\) represents the best possible fit to the data collected.
	One such estimate of fit is to calculate the error \(E\) that represents the sum of the squares of the vertical distances from the points to the line;
	that is,
	\[
		E = \sum_{i = 1}^m (y_i - (c t_i + d))^2.
	\]
	Thus the problem is reduced to finding the constants \(c\) and \(d\) that minimize \(E\).
	(For this reason the line \(y = ct + d\) is called the \textbf{least squares line}.)
	If we let
	\[
		A = \begin{pmatrix}
			t_1    & 1      \\
			\vdots & \vdots \\
			t_m    & 1
		\end{pmatrix}, x = \begin{pmatrix}
			c \\
			d
		\end{pmatrix}, y = \begin{pmatrix}
			y_1    \\
			\vdots \\
			y_m
		\end{pmatrix},
	\]
	then it follows that \(E = \norm{y - Ax}^2\).

	\cref{6.12} develop a general method for finding an explicit vector \(x_0 \in \vs{F}^n\) that minimizes \(E\);
	that is, given \(A \in \MS\), we find \(x_0 \in \vs{F}^n\) such that \(\norm{y - A x_0} \leq \norm{y - Ax}\) for all vectors \(x \in \vs{F}^n\).
	This method not only allows us to find the linear function that best fits the data, but also, for any positive integer \(n\), the best fit using a polynomial of degree at most \(n\).

	Suppose that the experimenter chose the times \(t_i\) for \(i \in \set{1, \dots, m}\) to satisfy
	\[
		\sum_{i = 1}^m t_i = 0.
	\]
	Then the two columns of \(A\) would be orthogonal, so \(A^* A\) would be a diagonal matrix (see \cref{ex:6.3.19}).
	In this case, the computations are greatly simplified.

	In practice, the \(m \times 2\) matrix \(A\) in our least squares application has rank equal to two, and hence \(A^* A\) is invertible by \cref{6.3.6}.
	For, otherwise, the first column of \(A\) is a multiple of the second column, which consists only of ones.
	But this would occur only if the experimenter collects all the data at exactly one time.

	Finally, the method in \cref{6.12} may also be applied if, for some \(k\), the experimenter wants to fit a polynomial of degree at most \(k\) to the data.
	For instance, if a polynomial \(y = a_0 + a_1 t + \cdots + a_k t^k\) of degree at most \(k\) is desired, the appropriate model is
	\[
		A = \begin{pmatrix}
			t_1^k  & \cdots & t_1    & 1      \\
			\vdots &        & \vdots & \vdots \\
			t_m^k  & \cdots & t_m    & 1
		\end{pmatrix}, x = \begin{pmatrix}
			a_k    \\
			\vdots \\
			a_0
		\end{pmatrix}, y = \begin{pmatrix}
			y_1    \\
			\vdots \\
			y_m
		\end{pmatrix}.
	\]
\end{note}

\begin{defn}\label{6.3.7}
	A solution \(s\) to \(Ax = b\) is called a \textbf{minimal solution} if \(\norm{s} \leq \norm{u}\) for all other solutions \(u\).
\end{defn}

\begin{thm}\label{6.13}
	Let \(A \in \MS\) and \(b \in \vs{F}^m\).
	Suppose that \(Ax = b\) is consistent.
	Then the following statements are true.
	\begin{enumerate}
		\item There exists exactly one minimal solution \(s\) of \(Ax = b\), and \(s \in \rg{\L_{A^*}}\).
		\item The vector \(s\) is the only solution to \(Ax = b\) that lies in \(\rg{\L_{A^*}}\);
		      that is, if \(u\) satisfies \((A A^*) u = b\), then \(s = A^* u\).
	\end{enumerate}
\end{thm}

\begin{proof}[\pf{6.13}(a)]
	For simplicity of notation, we let \(\W = \rg{\L_{A^*}}\) and \(\W' = \ns{\L_A}\).
	Let \(x\) be any solution to \(Ax = b\).
	By \cref{6.6}, \(x = s + y\) for some \(s \in \W\) and \(y \in \W^{\perp}\).
	But \(\W^{\perp} = \W'\) by \cref{ex:6.3.12}, and therefore \(b = Ax = As + Ay = As\).
	So \(s\) is a solution to \(Ax = b\) that lies in \(\W\).
	To prove (a), we need only show that \(s\) is the unique minimal solution.
	Let \(v\) be any solution to \(Ax = b\).
	By \cref{3.9}, we have that \(v = s + u\), where \(u \in \W'\).
	Since \(s \in \W\), which equals \((\W')^{\perp}\) by \cref{ex:6.3.12}, we have
	\[
		\norm{v}^2 = \norm{s + u}^2 = \norm{s}^2 + \norm{u}^2 \geq \norm{s}^2
	\]
	by \cref{ex:6.1.10}.
	Thus \(s\) is a minimal solution.
	We can also see from the preceding calculation that if \(\norm{v} = \norm{s}\), then \(u = \zv\);
	hence \(v = s\).
	Therefore \(s\) is the unique minimal solution to \(Ax = b\), proving (a).
\end{proof}

\begin{proof}[\pf{6.13}(b)]
	Assume that \(v\) is also a solution to \(Ax = b\) that lies in \(\W\).
	Then
	\[
		v - s \in \W \cap \W' = \W \cap \W^{\perp} = \set{\zv};
	\]
	so \(v = s\).

	Finally, suppose that \((A A^*) u = b\), and let \(v = A^* u\).
	Then \(v \in \W\) and \(Av = b\).
	Therefore \(s = v = A^* u\) by the discussion above.
\end{proof}

\begin{note}
	\cref{6.13} assures that every consistent system of linear equations has a unique minimal solution and provides a method for computing it.
\end{note}

\exercisesection

\setcounter{ex}{4}
\begin{ex}\label{ex:6.3.5}
	Let \(A, B \in \MS\) and let \(C \in \ms{n}{p}{\F}\).
	Then
	\begin{enumerate}
		\item \((A + B)^* = A^* + B^*\);
		\item \((cA)^* = \conj{c} A^*\) for all \(c \in \F\).
		\item \((AC)^* = C^* A^*\);
		\item \(A^{**} = A\);
	\end{enumerate}
\end{ex}

\begin{proof}[\pf{ex:6.3.5}(a)]
	We have
	\begin{align*}
		((A + B)^*)_{i j} & = \conj{(A + B)_{j i}}            &  & \by{6.1.5}  \\
		                  & = \conj{A_{j i} + B_{j i}}        &  & \by{1.2.9}  \\
		                  & = \conj{A_{j i}} + \conj{B_{j i}} &  & \by{d.2}[b] \\
		                  & = (A^*)_{i j} + (B^*)_{i j}       &  & \by{6.1.5}  \\
		                  & = (A^* + B^*)_{i j}               &  & \by{1.2.9}
	\end{align*}
	where \(i \in \set{1, \dots, n}\) and \(j \in \set{1, \dots, m}\).
	Thus by \cref{1.2.8} \((A + B)^* = A^* + B^*\).
\end{proof}

\begin{proof}[\pf{ex:6.3.5}(b)]
	We have
	\begin{align*}
		((cA)^*)_{i j} & = \conj{(cA)_{j i}}       &  & \by{6.1.5}  \\
		               & = \conj{c A_{j i}}        &  & \by{1.2.9}  \\
		               & = \conj{c} \conj{A_{j i}} &  & \by{d.2}[c] \\
		               & = \conj{c} (A^*)_{i j}    &  & \by{6.1.5}  \\
		               & = (\conj{c} A^*)_{i j}    &  & \by{1.2.9}
	\end{align*}
	where \(i \in \set{1, \dots, n}\) and \(j \in \set{1, \dots, m}\).
	Thus by \cref{1.2.8} \((cA)^* = \conj{c} A^*\).
\end{proof}

\begin{proof}[\pf{ex:6.3.5}(c)]
	We have
	\begin{align*}
		((AC)^*)_{i j} & = \conj{(AC)_{j i}}                            &  & \by{6.1.5}   \\
		               & = \conj{\sum_{k = 1}^n A_{j k} C_{k i}}        &  & \by{2.3.1}   \\
		               & = \sum_{k = 1}^n \conj{A_{j k}} \conj{C_{k i}} &  & \by{d.2}[bc] \\
		               & = \sum_{k = 1}^n (A^*)_{k j} (C^*)_{i k}       &  & \by{6.1.5}   \\
		               & = (C^* A^*)_{i j}                              &  & \by{2.3.1}
	\end{align*}
	where \(i \in \set{1, \dots, n}\) and \(j \in \set{1, \dots, m}\).
	Thus by \cref{1.2.8} \((AC)^* = C^* A^*\).
\end{proof}

\begin{proof}[\pf{ex:6.3.5}(d)]
	We have
	\begin{align*}
		(A^{**})_{i j} & = \conj{(A^*)_{j i}}    &  & \by{6.1.5}  \\
		               & = \conj{\conj{A_{i j}}} &  & \by{6.1.5}  \\
		               & = A_{i j}               &  & \by{d.2}[a]
	\end{align*}
	where \(i \in \set{1, \dots, m}\) and \(j \in \set{1, \dots, n}\).
	Thus by \cref{1.2.8} \(A = A^{**}\).
\end{proof}

\begin{ex}\label{ex:6.3.6}
	Let \(\T\) be a linear operator on an inner product space \(\V\) over \(\F\).
	Let \(\U_1 = \T + \T^*\) and \(\U_2 = \T \T^*\).
	Prove that \(\U_1 = \U_1^*\) and \(\U_2 = \U_2^*\).
\end{ex}

\begin{proof}[\pf{ex:6.3.6}]
	We have
	\begin{align*}
		\U_1^* & = (\T + \T^*)^*                    \\
		       & = \T^* + \T^{**} &  & \by{6.11}[a] \\
		       & = \T^* + \T      &  & \by{6.11}[d] \\
		       & = \U_1
	\end{align*}
	and
	\begin{align*}
		\U_2^* & = (\T \T^*)^*                    \\
		       & = \T^{**} \T^* &  & \by{6.11}[c] \\
		       & = \T \T^*      &  & \by{6.11}[d] \\
		       & = \U_2.
	\end{align*}
\end{proof}

\begin{ex}\label{ex:6.3.7}
	Give an example of a linear operator \(\T\) on an inner product space \(\V\) over \(\F\) such that \(\ns{\T} \neq \ns{\T^*}\).
\end{ex}

\begin{proof}[\pf{ex:6.3.7}]
	Let \(A = \begin{pmatrix}
		0 & 0 \\
		1 & 0
	\end{pmatrix}\).
	Then we have
	\[
		A^* = \begin{pmatrix}
			0 & 1 \\
			0 & 0
		\end{pmatrix}
	\]
	and by \cref{6.3.1}
	\[
		\ns{\L_A} = \spn{e_2} \neq \spn{e_1} = \ns{\L_{A^*}} = \ns{(\L_A)^*}.
	\]
\end{proof}

\begin{ex}\label{ex:6.3.8}
	Let \(\V\) be a finite-dimensional inner product space over \(\F\), and let \(\T \in \ls(\V)\).
	Prove that if \(\T\) is invertible, then \(\T^*\) is invertible and \((\T^*)^{-1} = (\T^{-1})^*\).
\end{ex}

\begin{proof}[\pf{ex:6.3.8}]
	We have
	\begin{align*}
		\IT[\V] & = \IT[\V]^*        &  & \by{6.11}[e] \\
		        & = (\T \T^{-1})^*                     \\
		        & = (\T^{-1})^* \T^* &  & \by{6.11}[c] \\
		        & = (\T^{-1} \T)^*                     \\
		        & = \T^* (\T^{-1})^*
	\end{align*}
	and thus \((\T^*)^{-1} = (\T^{-1})^*\).
\end{proof}

\begin{ex}\label{ex:6.3.9}
	Prove that if \(\V = \W \oplus \W^{\perp}\) and \(\T\) is the projection on \(\W\) along \(\W^{\perp}\), then \(\T = \T^*\).
\end{ex}

\begin{proof}[\pf{ex:6.3.9}]
	We have
	\begin{align*}
		         & \V = \W \oplus \W^{\perp}                                                                        \\
		\implies & \forall x \in \V, \exists (x_1, x_2) \in \W \times \W^{\perp} : x = x_1 + x_2 &  & \by{5.2.7}    \\
		\implies & \forall x, y \in \V, \inn{x, \T^*(y)} = \inn{\T(x), y}                        &  & \by{6.9}      \\
		         & = \inn{x_1, y}                                                                &  & \by{2.1.14}   \\
		         & = \inn{x_1, y_1 + y_2} = \inn{x_1, y_1} + \inn{x_1, y_2}                      &  & \by{6.1}[a]   \\
		         & = \inn{x_1, y_1}                                                              &  & \by{6.2.9}    \\
		         & = \inn{x_1, y_1} + \inn{x_2, y_1}                                             &  & \by{6.2.9}    \\
		         & = \inn{x_1 + x_2, y_1}                                                        &  & \by{6.1.1}[a] \\
		         & = \inn{x, \T(y)}                                                              &  & \by{2.1.14}   \\
		\implies & \T = \T^*.                                                                    &  & \by{6.1}[e]
	\end{align*}
\end{proof}

\begin{ex}\label{ex:6.3.10}
	Let \(\T\) be a linear operator on an inner product space \(\V\) over \(\F\).
	Prove that \(\norm{\T(x)} = \norm{x}\) for all \(x \in \V\) iff \(\inn{\T(x), \T(y)} = \inn{x,y}\) for all \(x, y \in \V\).
\end{ex}

\begin{proof}[\pf{ex:6.3.10}]
	First suppose that \(\norm{\T(x)} = \norm{x}\) for all \(x \in \V\).
	If \(\F = \C\), then we have
	\begin{align*}
		\forall x, y \in \V, \inn{\T(x), \T(y)} & = \frac{1}{4} \sum_{k = 1}^4 i^k \norm{\T(x) + i^k \T(y)}^2 &  & \by{ex:6.1.20}[b]      \\
		                                        & = \frac{1}{4} \sum_{k = 1}^4 i^k \norm{\T(x + i^k y)}^2     &  & \by{2.1.2}[b]          \\
		                                        & = \frac{1}{4} \sum_{k = 1}^4 i^k \norm{x + i^k y}^2         &  & \text{(by hypothesis)} \\
		                                        & = \inn{x, y}.                                               &  & \by{ex:6.1.20}[b]
	\end{align*}
	If \(\F = \R\), then we have
	\begin{align*}
		 & \forall x, y \in \V, \inn{\T(x), \T(y)}                                                        \\
		 & = \frac{1}{4} \pa{\norm{\T(x) + \T(y)}^2 - \norm{\T(x) - \T(y)}^2} &  & \by{ex:6.1.20}[a]      \\
		 & = \frac{1}{4} \pa{\norm{\T(x + y)}^2 - \norm{\T(x - y)}^2}         &  & \by{2.1.2}[b]          \\
		 & = \frac{1}{4} \pa{\norm{x + y}^2 - \norm{x - y}^2}                 &  & \text{(by hypothesis)} \\
		 & = \inn{x, y}.                                                      &  & \by{ex:6.1.20}[a]
	\end{align*}
	In either cases we have \(\inn{\T(x), \T(y)} = \inn{x, y}\) for all \(x, y \in \V\).

	Now suppose that \(\inn{\T(x), \T(y)} = \inn{x, y}\) for all \(x, y \in \V\).
	Then we have
	\begin{align*}
		         & \forall x \in \V, \norm{\T(x)}^2 = \inn{\T(x), \T(x)} = \inn{x, x} = \norm{x}^2 &  & \by{6.1.9}  \\
		\implies & \forall x \in \V, \norm{\T(x)} = \norm{x}.                                      &  & \by{6.2}[b]
	\end{align*}
\end{proof}

\begin{ex}\label{ex:6.3.11}
	For a linear operator \(\T\) on an inner product space \(\V\) over \(\F\), prove that \(\T^* \T = \zT\) implies \(\T = \zT\).
	Is the same result true if we assume that \(\T \T^* = \zT\)?
\end{ex}

\begin{proof}[\pf{ex:6.3.11}]
	First we show that \(\T^* \T = \zT \implies \T = \zT\).
	This is true since
	\begin{align*}
		         & \T^* \T = \zT                                                                  \\
		\implies & \forall x, y \in \V, 0 = \inn{\zv, y} = \inn{(\T^* \T)(x), y} &  & \by{6.1}[c] \\
		         & = \inn{\T^*(\T(x)), \T(y)} = \inn{\T(x), \T(y)}               &  & \by{6.9}    \\
		\implies & \forall x \in \V, \inn{\T(x), \T(x)} = 0                                       \\
		\implies & \forall x \in \V, \T(x) = \zv                                 &  & \by{6.1}[d] \\
		\implies & \T = \zT.
	\end{align*}

	Now we show that \(\T \T^* = \zT \implies \T = \zT\).
	Since
	\begin{align*}
		         & \forall x, y \in \V, 0 = \inn{\zv, y} = \inn{\zT(x), y} &  & \by{6.1}[c]       \\
		         & = \inn{x, \zT^*(y)}                                     &  & \by{6.9}          \\
		\implies & \forall x \in \V, \inn{\zT^*(x), \zT^*(x)} = 0          &  & (\zT^*(x) \in \V) \\
		\implies & \forall x \in \V, \zT^*(x) = \zv                        &  & \by{6.1}[d]       \\
		\implies & \zT^* = \zT,                                            &  & \by{2.1.9}
	\end{align*}
	we have
	\begin{align*}
		         & \T^{**} \T^* = \T \T^* = \zT &  & \by{6.11}[d]                  \\
		\implies & \T^* = \zT                   &  & \text{(from the proof above)} \\
		\implies & \T = \T^{**} = \zT^* = \zT.  &  & \text{(from the proof above)}
	\end{align*}
\end{proof}

\begin{ex}\label{ex:6.3.12}
	Let \(\V\) be an inner product space over \(\F\), and let \(\T \in \ls(\V)\).
	Prove the following results.
	\begin{enumerate}
		\item \(\rg{\T^*}^{\perp} = \ns{\T}\).
		\item If \(\V\) is finite-dimensional, then \(\rg{\T^*} = \ns{\T}^{\perp}\).
	\end{enumerate}
\end{ex}

\begin{proof}[\pf{ex:6.3.12}(a)]
	We have
	\begin{align*}
		     & v \in \rg{\T^*}^{\perp}                                    \\
		\iff & \forall y \in \rg{\T^*}, \inn{v, y} = 0 &  & \by{6.2.9}    \\
		\iff & \forall x \in \V, \inn{v, \T^*(x)} = 0  &  & \by{2.1.10}   \\
		\iff & \forall x \in \V, \inn{\T(v), x} = 0    &  & \by{6.9}      \\
		\iff & \forall x \in \V, \inn{x, \T(v)} = 0    &  & \by{6.1.1}[c] \\
		\iff & \T(v) = \zv                             &  & \by{6.1}[ce]  \\
		\iff & v \in \ns{\T}                           &  & \by{2.1.10}
	\end{align*}
	and thus \(\rg{\T^*}^{\perp} = \ns{\T}\).
\end{proof}

\begin{proof}[\pf{ex:6.3.12}(b)]
	We have
	\begin{align*}
		\rg{\T^*} & = (\rg{\T^*}^{\perp})^{\perp} &  & \by{ex:6.2.13}[c] \\
		          & = \ns{\T}^{\perp}.            &  & \by{ex:6.3.12}[a]
	\end{align*}
\end{proof}

\begin{ex}\label{ex:6.3.13}
	Let \(\T\) be a linear operator on a finite-dimensional inner product space \(\V\) over \(\F\).
	Prove the following results.
	\begin{enumerate}
		\item \(\ns{\T^* \T} = \ns{\T}\).
		      Deduce that \(\rk{\T^* \T} = \rk{\T}\).
		\item \(\rk{\T} = \rk{\T^*}\).
		      Deduce from (a) that \(\rk{\T \T^*} = \rk{\T}\).
		\item For any \(A \in \ms{n}{n}{\F}\), \(\rk{A^* A} = \rk{A A^*} = \rk{A}\).
	\end{enumerate}
\end{ex}

\begin{proof}[\pf{ex:6.3.13}(a)]
	We have
	\begin{align*}
		     & v \in \ns{\T^* \T}                                           \\
		\iff & (\T^* \T)(v) = \T^*(\T(v)) = \zv           &  & \by{2.1.10}  \\
		\iff & \forall x \in \V, \inn{x, \T^*(\T(v))} = 0 &  & \by{6.1}[cd] \\
		\iff & \forall x \in \V, \inn{\T(x), \T(v)} = 0   &  & \by{6.9}     \\
		\iff & \T(v) = \zv                                &  & \by{6.1}[cd] \\
		\iff & v \in \ns{\T}                              &  & \by{2.1.10}
	\end{align*}
	and thus \(\ns{\T^* \T} = \ns{\T}\).
	Since \(\V\) is finite-dimensional, we have
	\begin{align*}
		         & \dim(\V) = \begin{dcases}
			                      \rk{\T^* \T} + \nt{\T^* \T} \\
			                      \rk{\T} + \nt{\T}
		                      \end{dcases}                  &  & \by{2.3}                             \\
		\implies & \rk{\T^* \T} = \rk{\T} + \nt{\T} - \nt{\T^* \T}                                    \\
		         & = \rk{\T}.                                      &  & \text{(from the proof above)}
	\end{align*}
\end{proof}

\begin{proof}[\pf{ex:6.3.13}(b)]
	Since
	\begin{align*}
		\rg{\T^* \T} & = \rg{(\T^* \T)^*}     &  & \by{6.11}[cd]     \\
		             & = \ns{\T^* \T}^{\perp} &  & \by{ex:6.3.12}[b] \\
		             & = \ns{\T}^{\perp}      &  & \by{ex:6.3.13}[a] \\
		             & = \rg{\T^*},           &  & \by{ex:6.3.12}[b]
	\end{align*}
	by \cref{ex:6.3.13}(a) we have \(\rk{\T^* \T} = \rk{\T^*} = \rk{\T}\).
	Thus
	\begin{align*}
		         & \rk{(\T^*)^* \T^*} = \rk{(\T^*)^*} &  & \text{(from the proof above)} \\
		\implies & \rk{\T \T^*} = \rk{\T}.            &  & \by{6.11}[d]
	\end{align*}
\end{proof}

\begin{proof}[\pf{ex:6.3.13}(c)]
	We have
	\begin{align*}
		\rk{A^* A} & = \rk{\L_{A^* A}}    &  & \by{3.2.1}        \\
		           & = \rk{\L_{A^*} \L_A} &  & \by{2.15}[e]      \\
		           & = \rk{(\L_A)^* \L_A} &  & \by{6.3.1}        \\
		           & = \rk{\L_A}          &  & \by{ex:6.3.13}[a] \\
		           & = \rk{A}             &  & \by{3.2.1}        \\
		           & = \rk{\L_A (\L_A)^*} &  & \by{ex:6.3.13}[b] \\
		           & = \rk{\L_A \L_{A^*}} &  & \by{6.3.1}        \\
		           & = \rk{\L_{A A^*}}    &  & \by{2.15}[e]      \\
		           & = \rk{A A^*}.        &  & \by{3.2.1}
	\end{align*}
\end{proof}

\begin{ex}\label{ex:6.3.14}
	Let \(\V\) be an inner product space over \(\F\), and let \(y, z \in \V\).
	Define \(\T : \V \to \V\) by \(\T(x) = \inn{x, y} z\) for all \(x \in \V\).
	First prove that \(\T \in \ls(\V)\).
	Then show that \(\T^*\) exists, and find an explicit expression for it.
\end{ex}

\begin{proof}[\pf{ex:6.3.14}]
	Let \(x_1, x_2 \in \V\) and let \(c \in \F\).
	Since
	\begin{align*}
		\T(c x_1 + x_2) & = \inn{c x_1 + x_2, y} z            &  & \by{ex:6.3.14} \\
		                & = (c \inn{x_1, y} + \inn{x_2, y}) z &  & \by{6.1.1}[ab] \\
		                & = c \inn{x_1, y} z + \inn{x_2, y} z &  & \by{1.2.1}     \\
		                & = c \T(x_1) + \T(x_2),              &  & \by{ex:6.3.14}
	\end{align*}
	by \cref{2.1.2}(b) we see that \(\T \in \ls(\V)\).
	Now we define \(\U : \V \to \V\) as follow:
	\[
		\forall x \in \V, \U(x) = \inn{x, z} y.
	\]
	Since
	\begin{align*}
		\inn{\T(x_1), x_2} & = \inn{\inn{x_1, y} z, x_2}               &  & \by{ex:6.3.14} \\
		                   & = \conj{\inn{x_2, \inn{x_1, y} z}}        &  & \by{6.1.1}[c]  \\
		                   & = \conj{\conj{\inn{x_1, y}} \inn{x_2, z}} &  & \by{6.1}[b]    \\
		                   & = \inn{x_1, y} \conj{\inn{x_2, z}}        &  & \by{d.2}[ac]   \\
		                   & = \conj{\inn{x_2, z}} \inn{x_1, y}                            \\
		                   & = \inn{x_1, \inn{x_2, z} y}               &  & \by{6.1}[b]    \\
		                   & = \inn{x_1, \U(x_2)},
	\end{align*}
	by \cref{6.9} we see that \(\T^*\) exists and \(\T^* = \U\).
\end{proof}

\begin{defn}\label{6.3.8}
	Let \(\T \in \ls(\V, \W)\), where \(\V\) and \(\W\) are finite-dimensional inner product spaces over \(\F\) with inner products \(\inn{\cdot, \cdot}_1\) and \(\inn{\cdot, \cdot}_2\), respectively.
	A function \(\T^* : \W \to \V\) is called an \textbf{adjoint} of \(\T\) if \(\inn{\T(x), y}_2 = \inn{x, \T^*(y)}_1\) for all \(x \in \V\) and \(y \in \W\).
\end{defn}

\begin{ex}\label{ex:6.3.15}
	Let \(\T \in \ls(\V, \W)\), where \(\V\) and \(\W\) are finite-dimensional inner product spaces over \(\F\) with inner products \(\inn{\cdot, \cdot}_1\) and \(\inn{\cdot, \cdot}_2\), respectively.
	Prove the following results.
	\begin{enumerate}
		\item There is a unique adjoint \(\T^*\) of \(\T\), and \(\T^* \in \ls(\W, \V)\).
		\item If \(\beta\) and \(\gamma\) are orthonormal bases for \(\V\) and \(\W\) over \(\F\), respectively, then \([\T^*]_{\gamma}^{\beta} = ([\T]_{\beta}^{\gamma})^*\).
		\item \(\rk{\T^*} = \rk{\T}\).
		\item \(\inn{\T^*(x), y}_1 = \inn{x, \T(y)}_2\) for all \(x \in \W\) and \(y \in \V\).
		\item For all \(x \in \V\), \(\T^* \T(x) = \zv\) iff \(\T(x) = \zv\).
	\end{enumerate}
\end{ex}

\begin{proof}[\pf{ex:6.3.15}(a)]
	Let \(y \in \W\).
	Define \(g : \V \to \F\) by \(g(x) = \inn{\T(x), y}_2\) for all \(x \in \V\).
	We first show that \(g \in \ls(\V, \F)\).
	Let \(x_1, x_2 \in \V\) and let \(c \in \F\).
	Then
	\begin{align*}
		g(c x_1 + x_2) & = \inn{\T(c x_1 + x_2), y}_2                                    \\
		               & = \inn{c \T(x_1) + \T(x_2), y}_2            &  & \by{2.1.2}[b]  \\
		               & = c \inn{\T(x_1), y}_2 + \inn{\T(x_2), y}_2 &  & \by{6.1.1}[ab] \\
		               & = c g(x_1) + g(x_2).
	\end{align*}
	Hence by \cref{2.1.2}(b) we see that \(g \in \ls(\V, \F)\).

	We now apply \cref{6.8} to obtain a unique vector \(y' \in \V\) such that \(g(x) = \inn{x, y'}_1\);
	that is, \(\inn{\T(x), y}_2 = \inn{x, y'}_1\) for all \(x \in \V\).
	Defining \(\T^* : \W \to \V\) by \(\T^*(y) = y'\), we have \(\inn{\T(x), y}_2 = \inn{x, \T^*(y)}_1\).

	To show that \(\T^*\) is linear, let \(y_1, y_2 \in \W\) and \(c \in \F\).
	Then for any \(x \in \V\),
	we have
	\begin{align*}
		\inn{x, \T^*(c y_1 + y_2)}_1 & = \inn{\T(x), c y_1 + y_2}_2                                             \\
		                             & = \conj{c} \inn{\T(x), y_1}_2 + \inn{\T(x), y_2}_2     &  & \by{6.1}[ab] \\
		                             & = \conj{c} \inn{x, \T^*(y_1)}_1 + \inn{x, \T^*(y_2)}_1                   \\
		                             & = \inn{x, c \T^*(y_1) + \T^*(y_2)}_1.                  &  & \by{6.1}[ab]
	\end{align*}
	Since \(x\) is arbitrary, \(\T^*(c y_1 + y_2) = c \T^*(y_1) + \T^*(y_2)\) by \cref{6.1}(e).

	Finally, we need to show that \(\T^*\) is unique. Suppose that \(\U \in \ls(\W, \V)\) and that it satisfies \(\inn{\T(x), y}_2 = \inn{x, \U(y)}_1\) for all \((x, y) \in \V \times \W\).
	Then \(\inn{x, \T^*(y)}_1 = \inn{x, \U(y)}_1\) for all \((x, y) \in \V \times \W\), so \(\T^* = \U\).
\end{proof}

\begin{proof}[\pf{ex:6.3.15}(b)]
	Let \(i \in \set{1, \dots, m}\) and let \(j \in \set{1, \dots, n}\).
	Then we have
	\begin{align*}
		([\T^*]_{\gamma}^{\beta})_{i j} & = \inn{\T^*(w_j), v_i}_1               &  & \by{6.5}      \\
		                                & = \conj{\inn{v_i, \T^*(w_j)}_1}        &  & \by{6.1.1}[c] \\
		                                & = \conj{\inn{\T(v_i), w_j}_2}          &  & \by{6.3.8}    \\
		                                & = \conj{([\T]_{\beta}^{\gamma})_{j i}} &  & \by{6.5}      \\
		                                & = (([\T]_{\beta}^{\gamma})^*)_{i j}.   &  & \by{6.1.5}
	\end{align*}
	Hence \([\T^*]_{\gamma}^{\beta} = ([\T]_{\beta}^{\gamma})^*\).
\end{proof}

\begin{proof}[\pf{ex:6.3.15}(c)]
	First we show that \(\ns{\T} = \rg{\T^*}^{\perp}\).
	This is true since
	\begin{align*}
		\iff & v \in \ns{\T}                                                \\
		\iff & \T(v) = \zv_{\W}                          &  & \by{2.1.10}   \\
		\iff & \forall y \in \W, \inn{y, \T(v)}_2 = 0    &  & \by{6.1}[cd]  \\
		\iff & \forall y \in \W, \inn{\T(v), y}_2 = 0    &  & \by{6.1.1}[c] \\
		\iff & \forall y \in \W, \inn{v, \T^*(y)}_1 = 0  &  & \by{6.3.8}    \\
		\iff & \forall x \in \rg{\T^*}, \inn{v, x}_1 = 0                    \\
		\iff & v \in \rg{\T^*}^{\perp}.                  &  & \by{6.2.9}
	\end{align*}

	Now we show that \(\rk{\T^*} = \rk{\T}\).
	This is true since
	\begin{align*}
		\rk{\T} & = \dim(\V) - \nt{\T}                 &  & \by{2.3}                      \\
		        & = \dim(\V) - \dim(\ns{\T})           &  & \by{2.1.12}                   \\
		        & = \dim(\V) - \dim(\rg{\T^*}^{\perp}) &  & \text{(from the proof above)} \\
		        & = \dim(\rg{\T^*})                    &  & \by{ex:6.2.13}[d]             \\
		        & = \rk{\T^*}.                         &  & \by{2.1.12}
	\end{align*}
\end{proof}

\begin{proof}[\pf{ex:6.3.15}(d)]
	We have
	\begin{align*}
		\inn{\T^*(x), y}_1 & = \conj{\inn{y, \T^*(x)}_1} &  & \by{6.1.1}[c] \\
		                   & = \conj{\inn{\T(y), x}_2}   &  & \by{6.3.8}    \\
		                   & = \inn{x, \T(y)}_2.         &  & \by{6.1.1}[c]
	\end{align*}
\end{proof}

\begin{proof}[\pf{ex:6.3.15}(e)]
	We have
	\begin{align*}
		     & v \in \ns{\T^* \T}                                             \\
		\iff & (\T^* \T)(v) = \T^*(\T(v)) = \zv_{\V}        &  & \by{2.1.10}  \\
		\iff & \forall x \in \V, \inn{x, \T^*(\T(v))}_1 = 0 &  & \by{6.1}[cd] \\
		\iff & \forall x \in \V, \inn{\T(x), \T(v)}_2 = 0   &  & \by{6.3.8}   \\
		\iff & \T(v) = \zv_{\W}                             &  & \by{6.1}[cd] \\
		\iff & v \in \ns{\T}                                &  & \by{2.1.10}
	\end{align*}
	and thus \(\ns{\T^* \T} = \ns{\T}\).
\end{proof}

\begin{ex}\label{ex:6.3.16}
	Let \(\V, \W, \vs{X}\) be finite-dimensional inner product spaces over \(\F\).
	Let \(\seq{\inn{\cdot, \cdot}}{1,2,3}\) be inner products on \(\V, \W, \vs{X}\) over \(\F\), respectively.
	Let \(\T, \U \in \ls(\V, \W)\) and let \(\lt{S} \in \ls(\W, \vs{X})\).
	Then
	\begin{enumerate}
		\item \((\T + \U)^* = \T^* + \U^*\);
		\item \((c \T)^* = \conj{c} \T^*\) for any \(c \in \F\);
		\item \((\lt{S} \T)^* = \T^* \lt{S}^*\);
		\item \(\T^{**} = \T\);
	\end{enumerate}
\end{ex}

\begin{proof}[\pf{ex:6.3.16}(a)]
	Since
	\begin{align*}
		 & \forall (x, y) \in \V \times \W, \inn{x, (\T + \U)^*(y)}_1                    \\
		 & = \inn{(\T + \U)(x), y}_2                                  &  & \by{6.3.8}    \\
		 & = \inn{\T(x) + \U(x), y}_2                                 &  & \by{2.2.5}    \\
		 & = \inn{\T(x), y}_2 + \inn{\U(x), y}_2                      &  & \by{6.1.1}[a] \\
		 & = \inn{x, \T^*(y)}_1 + \inn{x, \U^*(y)}_1                  &  & \by{6.3.8}    \\
		 & = \inn{x, \T^*(y) + \U^*(y)}_1                             &  & \by{6.1}[a]   \\
		 & = \inn{x, (\T^* + \U^*)(y)}_1,                             &  & \by{2.2.5}
	\end{align*}
	by \cref{6.1}(e) we know that \(\T^* + \U^* = (\T + \U)^*\).
\end{proof}

\begin{proof}[\pf{ex:6.3.16}(b)]
	Since
	\begin{align*}
		\forall (x, y) \in \V \times \W, \inn{x, (c \T)^*(y)}_1 & = \inn{(c \T)(x), y}_2           &  & \by{6.3.8}    \\
		                                                        & = \inn{c \T(x), y}_2             &  & \by{2.2.5}    \\
		                                                        & = c \inn{\T(x), y}_2             &  & \by{6.1.1}[b] \\
		                                                        & = c \inn{x, \T^*(y)}_1           &  & \by{6.3.8}    \\
		                                                        & = \inn{x, \conj{c} \T^*(y)}_1    &  & \by{6.1}[b]   \\
		                                                        & = \inn{x, (\conj{c} \T^*)(y)}_1, &  & \by{2.2.5}
	\end{align*}
	by \cref{6.1}(e) we know that \((c \T)^* = \conj{c} \T^*\).
\end{proof}

\begin{proof}[\pf{ex:6.3.16}(c)]
	Since
	\begin{align*}
		\forall (x, y) \in \V \times \vs{X}, \inn{x, (\lt{S} \T)^*(y)}_1 & = \inn{(\lt{S} \T)(x), y}_3      &  & \by{6.3.8} \\
		                                                                 & = \inn{\lt{S}(\T(x)), y}_3                       \\
		                                                                 & = \inn{\T(x), \lt{S}^*(y)}_2     &  & \by{6.3.8} \\
		                                                                 & = \inn{x, \T^*(\lt{S}^*(y))}_1   &  & \by{6.3.8} \\
		                                                                 & = \inn{x, (\T^* \lt{S}^*)(y)}_1,
	\end{align*}
	by \cref{6.1}(e) we know that \((\lt{S} \T)^* = \T^* \lt{S}^*\).
\end{proof}

\begin{proof}[\pf{ex:6.3.16}(d)]
	Since
	\begin{align*}
		\forall (x, y) \in \V \times \W, \inn{x, \T(y)}_1 & = \inn{\T^*(x), y}_2     &  & \by{ex:6.3.15}[d] \\
		                                                  & = \inn{x, \T^{**}(y)}_1, &  & \by{6.3.8}
	\end{align*}
	by \cref{6.1}(e) we know that \(\T^{**} = \T\).
\end{proof}

\begin{ex}\label{ex:6.3.17}
	Let \(T \in \ls(\V, \W\), where \(\V\) and \(\W\) are finite-dimensional inner product spaces over \(\F\).
	Prove that \(\rg{\T^*}^{\perp} = \ns{\T}\), using \cref{6.3.8}.
\end{ex}

\begin{proof}[\pf{ex:6.3.17}]
	See proof of \cref{ex:6.3.15}(c).
\end{proof}

\begin{ex}\label{ex:6.3.18}
	Let \(A \in \ms{n}{n}{\F}\).
	Prove that \(\det(A^*) = \conj{\det(A)}\).
\end{ex}

\begin{proof}[\pf{ex:6.3.18}]
	We have
	\begin{align*}
		\det(A^*) & = \det(\conj{\tp{A}}) &  & \by{6.1.5}        \\
		          & = \conj{\det(\tp{A})} &  & \by{ex:4.3.13}[a] \\
		          & = \conj{\det(A)}.     &  & \by{4.8}
	\end{align*}
\end{proof}

\begin{ex}\label{ex:6.3.19}
	Suppose that \(A \in \MS\) and no two columns of \(A\) are identical.
	Prove that \(A^* A\) is a diagonal matrix iff every pair of columns of \(A\) is orthogonal.
\end{ex}

\begin{proof}[\pf{ex:6.3.19}]
	For each \(i \in \set{1, \dots, n}\), let \(a_i\) be the \(i\)th column of \(A\).
	Then we have
	\begin{align*}
		     & A^* A \text{ is diagonal}                                                                                                  \\
		\iff & \forall i, j \in \set{1, \dots, n}, (A^* A)_{i j} = (A^* A)_{i j} \delta_{i j}                         &  & \by{ex:2.3.10} \\
		\iff & \forall i, j \in \set{1, \dots, n}, \sum_{k = 1}^m (A^*)_{i k} A_{k j} = (A^* A)_{i j} \delta_{i j}    &  & \by{2.3.1}     \\
		\iff & \forall i, j \in \set{1, \dots, n}, \sum_{k = 1}^m \conj{A_{k i}} A_{k j} = (A^* A)_{i j} \delta_{i j} &  & \by{6.1.5}     \\
		\iff & \forall i, j \in \set{1, \dots, n}, \inn{a_j, a_i} = (A^* A)_{i j} \delta_{i j}                        &  & \by{6.1.2}     \\
		\iff & \text{the set of columns of } A \text{ is orthogonal}.                                                 &  & \by{6.1.12}
	\end{align*}
\end{proof}

\setcounter{ex}{22}
\begin{ex}\label{ex:6.3.23}
	Consider the problem of finding the least squares line \(y = ct + d\) corresponding to the \(m\) observations \((t_1, y_1), \dots, (t_m, y_m)\).
	\begin{enumerate}
		\item Show that the equation \((A^* A) x_0 = A^* y\) of \cref{6.12} takes the form of the \emph{normal equations}:
		      \[
			      \pa{\sum_{i = 1}^m t_i^2} c + \pa{\sum_{i = 1}^m t_i} d = \sum_{i = 1}^m t_i y_i
		      \]
		      and
		      \[
			      \pa{\sum_{i = 1}^m t_i} c + md = \sum_{i = 1}^m y_i.
		      \]
		      These equations may also be obtained from the error \(E\) by setting the partial derivatives of \(E\) with respect to both \(c\) and \(d\) equal to zero.
		\item Use the second normal equation of (a) to show that the least squares line must pass through the \emph{center of mass}, \((\overline{t}, \overline{y})\), where
		      \[
			      \overline{t} = \frac{1}{m} \sum_{i = 1}^m t_i \quad \text{and} \quad \overline{y} = \frac{1}{m} \sum_{i = 1}^m y_i.
		      \]
	\end{enumerate}
\end{ex}

\begin{proof}[\pf{ex:6.3.23}(a)]
	Since
	\begin{align*}
		A         & = \begin{pmatrix}
			              t_1    & 1      \\
			              \vdots & \vdots \\
			              t_m    & 1
		              \end{pmatrix}                                                               &  & \by{6.12}    \\
		A^*       & = \begin{pmatrix}
			              t_1 & \cdots & t_m \\
			              1   & \cdots & 1
		              \end{pmatrix}                                                            &  & \by{6.1.5}      \\
		A^* A     & = \begin{pmatrix}
			              \sum_{i = 1}^m t_i^2 & \sum_{i = 1}^m t_i \\
			              \sum_{i = 1}^m t_i   & m
		              \end{pmatrix}                                     &  & \by{2.3.1}                             \\
		A^* A x_0 & = \begin{pmatrix}
			              \sum_{i = 1}^m t_i^2 & \sum_{i = 1}^m t_i \\
			              \sum_{i = 1}^m t_i   & m
		              \end{pmatrix} \begin{pmatrix}
			                            c \\
			                            d
		                            \end{pmatrix}                                     &  & \by{6.12}                \\
		          & = \begin{pmatrix}
			              \pa{\sum_{i = 1}^m t_i^2} c + \pa{\sum_{i = 1}^m t_i} d \\
			              \pa{\sum_{i = 1}^m t_i} c + md
		              \end{pmatrix} &  & \by{2.3.1}                 \\
		A^* y     & = \begin{pmatrix}
			              t_1 & \cdots & t_m \\
			              1   & \cdots & 1
		              \end{pmatrix} \begin{pmatrix}
			                            y_1    \\
			                            \vdots \\
			                            y_m
		                            \end{pmatrix}                                                    &  & \by{6.12} \\
		          & = \begin{pmatrix}
			              \sum_{i = 1}^m t_i y_i \\
			              \sum_{i = 1}^m y_i
		              \end{pmatrix},                                                        &  & \by{2.3.1}
	\end{align*}
	we have
	\begin{align*}
		         & A^* A x_0 = A^* y                                                                                         &  & \by{6.12} \\
		\implies & \begin{dcases}
			           \pa{\sum_{i = 1}^m t_i^2} c + \pa{\sum_{i = 1}^m t_i} d = \sum_{i = 1}^m t_i y_i \\
			           \pa{\sum_{i = 1}^m t_i} c + md = \sum_{i = 1}^m y_i
		           \end{dcases}.
	\end{align*}
\end{proof}

\begin{proof}[\pf{ex:6.3.23}(b)]
	We have
	\begin{align*}
		         & \frac{1}{m} \pa{\pa{\sum_{i = 1}^m t_i} c + md} = \frac{1}{m} \pa{\sum_{i = 1}^m y_i} &  & \by{ex:6.3.23}[a] \\
		\implies & \overline{t} c + d = \overline{y}                                                                            \\
		\implies & (\overline{t}, \overline{y}) \text{ pass through } y = ct + d.                        &  & \by{6.12}
	\end{align*}
\end{proof}

\begin{ex}\label{ex:6.3.24}
	Let \(\V\) and \(\set{\seq{e}{1,2,}}\) be defined as in \cref{ex:6.2.23}.
	Define \(\T : \V \to \V\) by
	\[
		\T(\sigma)(k) = \sum_{i = k}^\infty \sigma(i) \quad \text{for } k \in \Z^+.
	\]
	Notice that the infinite series in the definition of \(\T\) converges because \(\sigma(i) \neq 0\) for only finitely many \(i\).
	\begin{enumerate}
		\item Prove that \(\T \in \ls(\V)\).
		\item Prove that for any positive integer \(n\), \(\T(e_n) = \sum_{i = 1}^n e_i\).
		\item Prove that \(\T\) has no adjoint
	\end{enumerate}
\end{ex}

\begin{proof}[\pf{ex:6.3.24}(a)]
	Let \(a, b \in \V\) and let \(c \in \F\).
	Since
	\begin{align*}
		\forall k \in \Z^+, \T(ca + b)(k) & = \sum_{i = k}^\infty (ca + b)(i)                       &  & \by{ex:6.3.24} \\
		                                  & = c \sum_{i = k}^\infty a(i) + \sum_{i = k}^\infty b(i)                     \\
		                                  & = c \T(a)(k) + \T(b)(k)                                 &  & \by{ex:6.3.24} \\
		                                  & = (c \T(a) + \T(b))(k),
	\end{align*}
	by \cref{1.2.13} and \cref{2.1.2}(b) we see that \(\T \in \ls(\V)\).
\end{proof}

\begin{proof}[\pf{ex:6.3.24}(b)]
	Let \(n \in \Z^+\).
	Then we have
	\begin{align*}
		\forall k \in \Z^+, \T(e_n) & = \sum_{i = k}^\infty e_n(i)       &  & \by{ex:6.3.24}    \\
		                            & = \sum_{i = k}^\infty \delta_{n i} &  & \by{ex:6.2.23}[b] \\
		                            & = \begin{dcases}
			                                1 & \text{if } k \leq n \\
			                                0 & \text{if } k > n
		                                \end{dcases}         &  & \by{2.3.4}                    \\
		                            & = \sum_{i = 1}^n e_i.              &  & \by{ex:6.2.23}[b]
	\end{align*}
\end{proof}

\begin{proof}[\pf{ex:6.3.24}(c)]
	Suppose for sake of contradiction that \(\T^* \in \ls(\V)\) exists.
	Then we have
	\begin{align*}
		\forall k \in \Z^+, \T^*(e_1)(k) & = \conj{\conj{\T^*(e_1)(k)}}                                  &  & \by{d.2}[a]       \\
		                                 & = \conj{\delta_{k k} \cdot \conj{\T^*(e_1)(k)}}               &  & \by{2.3.4}        \\
		                                 & = \conj{\sum_{i = 1}^\infty e_k(i) \cdot \conj{\T^*(e_1)(i)}} &  & \by{ex:6.2.23}[b] \\
		                                 & = \conj{\inn{e_k, \T^*(e_1)}}                                 &  & \by{ex:6.2.23}    \\
		                                 & = \conj{\inn{\T(e_k), e_1}}                                   &  & \by{6.9}          \\
		                                 & = \conj{\inn{\sum_{i = 1}^k e_k, e_1}}                        &  & \by{ex:6.3.24}[b] \\
		                                 & = \conj{\sum_{i = 1}^k \inn{e_k, e_1}}                        &  & \by{6.1.1}[a]     \\
		                                 & = \conj{1} = 1.                                               &  & \by{ex:6.2.23}[b]
	\end{align*}
	But \(\T^*(e_1) \in \V\) implies there are only finite number of nonzeros in \(\T^*(e_1)\), a contradiction.
	Thus \(\T^*\) does not exist.
\end{proof}

\section{Normal and Self-Adjoint Operators}\label{sec:6.4}

\begin{lem}\label{6.4.1}
  Let \(\T\) be a linear operator on a finite-dimensional inner product space \(\V\) over \(\F\).
  If \(\T\) has an eigenvector, then so does \(\T^*\).
\end{lem}

\begin{proof}[\pf{6.4.1}]
  Suppose that \(v\) is an eigenvector of \(\T\) with corresponding eigenvalue \(\lambda\).
  Then for any \(x \in \V\),
  \begin{align*}
    0 & = \inn{\zv, x}                                 &  & \text{(by \cref{6.1}(c))}        \\
      & = \inn{(\T - \lambda \IT[\V])(v), x}           &  & \text{(by \cref{5.2.4})}         \\
      & = \inn{v, (\T - \lambda \IT[\V])^*(x)}         &  & \text{(by \cref{6.9})}           \\
      & = \inn{v, (\T^* - \conj{\lambda} \IT[\V])(x)}, &  & \text{(by \cref{6.11}(a)(b)(e))}
  \end{align*}
  and hence \(v\) is orthogonal to the range of \(\T^* - \conj{\lambda} \IT[\V]\).
  So \(\T^* - \conj{\lambda} \IT[\V]\) is not onto (\(v \notin \rg{\T^* - \conj{\lambda} \IT[\V]}\)) and hence is not one-to-one (by \cref{2.5}).
  Thus \(\T^* - \conj{\lambda} \IT[\V]\) has a nonzero null space, and any nonzero vector in this null space is an eigenvector of \(\T^*\) with corresponding eigenvalue \(\conj{\lambda}\).
\end{proof}

\begin{thm}[Schur's theorem]\label{6.14}
  Let \(\T\) be a linear operator on a finite-dimensional inner product space \(\V\) over \(\F\).
  Suppose that the characteristic polynomial of \(\T\) splits.
  Then there exists an orthonormal basis \(\beta\) for \(\V\) over \(\F\) such that the matrix \([\T]_{\beta}\) is upper triangular.
\end{thm}

\begin{proof}[\pf{6.14}]
  The proof is by mathematical induction on the dimension \(n\) of \(\V\).
  The result is immediate if \(n = 1\).
  So suppose that the result is true for linear operators on \(n\)-dimensional inner product spaces whose characteristic polynomials split.
  By \cref{6.4.1}, we can assume that \(\T^*\) has a unit eigenvector \(z\).
  Suppose that \(\T^*(z) = \lambda z\) and that \(\W = \spn{\set{z}}\).
  We show that \(\W^{\perp}\) is \(\T\)-invariant.
  If \(y \in \W^{\perp}\) and \(x = cz \in \W\), then
  \begin{align*}
    \inn{\T(y), x} & = \inn{\T(y), cz}                                              \\
                   & = \inn{y, \T^*(cz)}           &  & \text{(by \cref{6.9})}      \\
                   & = \inn{y, c \T^*(z)}          &  & \text{(by \cref{2.1.1}(b))} \\
                   & = \inn{y, c \lambda z}        &  & \text{(by \cref{5.1.2})}    \\
                   & = \conj{c \lambda} \inn{y, z} &  & \text{(by \cref{6.1}(b))}   \\
                   & = \conj{c \lambda} (0)        &  & \text{(by \cref{6.2.9})}    \\
                   & = 0.
  \end{align*}
  So \(\T(y) \in \W^{\perp}\).
  It is easy to show (see \cref{5.21}, or as a consequence of \cref{ex:4.4.6}) that the characteristic polynomial of \(\T_{\W^{\perp}}\) divides the characteristic polynomial of \(\T\) and hence splits.
  By \cref{6.7}(c), \(\dim(\W^{\perp}) = n\), so we may apply the induction hypothesis to \(\T_{\W^{\perp}}\) and obtain an orthonormal basis \(\gamma\) for \(\W^{\perp}\) over \(\F\) such that \([\T_{\W^{\perp}}]_{\gamma}\) is upper triangular.
  Clearly, \(\beta = \gamma \cup \set{z}\) is an orthonormal basis for \(\V\) over \(\F\) (by \cref{1.6.15,6.2.4}) such that \([\T]_{\beta}\) is upper triangular.
\end{proof}

\begin{cor}\label{6.4.2}
  Let \(\V\) be a finite-dimensional inner product space over \(\F\) and let \(\T \in \ls(\V)\).
  Suppose that there exists an orthonormal basis of eigenvectors of \(\T\).
  Then \(\T \T^* = \T^* \T\).
\end{cor}

\begin{proof}[\pf{6.4.2}]
  If such an orthonormal basis \(\beta\) exists, then \([\T]_{\beta}\) is a diagonal matrix, and hence \([\T^*]_{\beta} = [\T]_{\beta}^*\) is also a diagonal matrix.
  Because diagonal matrices commute, we conclude that \(\T\) and \(\T^*\) commute.
\end{proof}

\begin{defn}\label{6.4.3}
  Let \(\V\) be an inner product space over \(\F\), and let \(\T \in \ls(\V)\).
  We say that \(\T\) is \textbf{normal} if \(\T \T^* = \T^* \T\).
  An \(n \times n\) real or complex matrix \(A\) is \textbf{normal} if \(A A^* = A^* A\).
\end{defn}

\begin{cor}\label{6.4.4}
  Let \(\V\) be an inner product space over \(\F\), and let \(\T \in \ls(\V)\).
  \(\T\) is normal iff \([\T]_{\beta}\) is normal, where \(\beta\) is an orthonormal basis.
\end{cor}

\begin{proof}[\pf{6.4.4}]
  We have
  \begin{align*}
         & \T \text{ is normal}                                                                    \\
    \iff & \T \T^* = \T^* \T                                         &  & \text{(by \cref{6.4.3})} \\
    \iff & [\T]_{\beta} [\T]_{\beta}^* = [\T]_{\beta}^* [\T]_{\beta} &  & \text{(by \cref{6.10})}  \\
    \iff & [\T]_{\beta} \text{ is normal}.                           &  & \text{(by \cref{6.4.3})}
  \end{align*}
\end{proof}

\begin{eg}\label{6.4.5}
  Let \(\T : \R^2 \to \R^2\) be rotation by \(\theta\), where \(0 < \theta < \pi\).
  The matrix representation of \(\T\) in the standard ordered basis is given by
  \[
    A = \begin{pmatrix}
      \cos\theta & -\sin\theta \\
      \sin\theta & \cos\theta
    \end{pmatrix}.
  \]
  Note that \(A A^* = I_2 = A^* A\);
  so \(A\), and hence \(\T\), is normal.
\end{eg}

\begin{eg}\label{6.4.6}
  Suppose that \(A\) is a real skew-symmetric matrix;
  that is, \(\tp{A} = -A\).
  Then \(A\) is normal because both \(A \tp{A}\) and \(\tp{A} A\) are equal to \(-A^2\).
\end{eg}

\begin{note}
  Clearly, the operator \(\T\) in \cref{6.4.5} does not even possess one eigenvector.
  So in the case of a real inner product space, we see that normality is not sufficient to guarantee an orthonormal basis of eigenvectors.
  All is not lost, however.
  We show that normality suffices if \(\V\) is a complex inner product space.
\end{note}

\begin{thm}\label{6.15}
  Let \(\V\) be an inner product space over \(\F\), and let \(\T\) be a normal operator on \(\V\).
  Then the following statements are true.
  \begin{enumerate}
    \item \(\norm{\T(x)} = \norm{\T^*(x)}\) for all \(x \in \V\).
    \item \(\T - c \IT[\V]\) is normal for every \(c \in \F\).
    \item If \(x\) is an eigenvector of \(\T\), then \(x\) is also an eigenvector of \(\T^*\).
          In fact, if \(\T(x) = \lambda x\), then \(\T^*(x) = \conj{\lambda} x\).
    \item If \(\lambda_1\) and \(\lambda_2\) are distinct eigenvalues of \(\T\) with corresponding eigenvectors \(x_1\) and \(x_2\), then \(x_1\) and \(x_2\) are orthogonal.
  \end{enumerate}
\end{thm}

\begin{proof}[\pf{6.15}(a)]
  For any \(x \in \V\), we have
  \begin{align*}
    \norm{\T(x)}^2 & = \inn{\T(x), \T(x)}     &  & \text{(by \cref{6.1.9})} \\
                   & = \inn{\T^* \T(x), x}    &  & \text{(by \cref{6.9})}   \\
                   & = \inn{\T \T^*(x), x}    &  & \text{(by \cref{6.4.3})} \\
                   & = \inn{\T^*(x), \T^*(x)} &  & \text{(by \cref{6.9})}   \\
                   & = \norm{\T^*(x)}^2.      &  & \text{(by \cref{6.1.9})}
  \end{align*}
\end{proof}

\begin{proof}[\pf{6.15}(b)]
  We have
  \begin{align*}
    (\T - c \IT[\V]) (\T - c \IT[\V])^* & = (\T - c \IT[\V]) (\T^* - \conj{c} \IT[\V])                                  &  & \text{(by \cref{6.11}(a)(b)(e))} \\
                                        & = \T \T^* - \conj{c} \T \IT[\V] - c \IT[\V] \T^* + c \conj{c} \IT[\V] \IT[\V] &  & \text{(by \cref{2.10})}          \\
                                        & = \T^* \T - \conj{c} \IT[\V] \T - c \T^* \IT[\V] + c \conj{c} \IT[\V] \IT[\V] &  & \text{(by \cref{6.4.3})}         \\
                                        & = (\T^* - \conj{c} \IT[\V]) (\T - c \IT[\V])                                  &  & \text{(by \cref{2.10})}          \\
                                        & = (\T - c \IT[\V])^* (\T - c \IT[\V])                                         &  & \text{(by \cref{6.11}(a)(b)(e))}
  \end{align*}
  and thus by \cref{6.4.3} \(\T - c \IT[\V]\) is normal for all \(c \in \F\).
\end{proof}

\begin{proof}[\pf{6.15}(c)]
  Suppose that \(\T(x) = \lambda x\) for some \(x \in \V\).
  Let \(\U = \T - \lambda \IT[\V]\).
  Then \(\U(x) = \zv\), and \(\U\) is normal by (b).
  Thus (a) implies that
  \begin{align*}
    0 & = \norm{\U(x)}                              &  & \text{(by \cref{6.2}(b))}        \\
      & = \norm{\U^*(x)}                            &  & \text{(by \cref{6.15}(a))}       \\
      & = \norm{(\T^* - \conj{\lambda} \IT[\V])(x)} &  & \text{(by \cref{6.11}(a)(b)(e))} \\
      & = \norm{\T^*(x) - \conj{\lambda} x}.        &  & \text{(by \cref{2.2.5})}
  \end{align*}
  Hence by \cref{6.2}(b) \(\T^*(x) = \conj{\lambda} x\).
  So \(x\) is an eigenvector of \(\T^*\).
\end{proof}

\begin{proof}[\pf{6.15}(d)]
  Let \(\lambda_1\) and \(\lambda_2\) be distinct eigenvalues of \(\T\) with corresponding eigenvectors \(x_1\) and \(x_2\).
  Then, using (c), we have
  \begin{align*}
    \lambda_1 \inn{x_1, x_2} & = \inn{\lambda_1 x_1, x_2}        &  & \text{(by \cref{6.1.1}(b))} \\
                             & = \inn{\T(x_1), x_2}              &  & \text{(by \cref{5.1.2})}    \\
                             & = \inn{x_1, \T^*(x_2)}            &  & \text{(by \cref{6.9})}      \\
                             & = \inn{x_1, \conj{\lambda_2} x_2} &  & \text{(by \cref{6.15}(c))}  \\
                             & = \lambda_2 \inn{x_1, x_2}.       &  & \text{(by \cref{6.1}(b))}
  \end{align*}
  Since \(\lambda_1 \neq \lambda_2\), we conclude that \(\inn{x_1, x_2} = 0\).
\end{proof}

\begin{thm}\label{6.16}
  Let \(\T\) be a linear operator on a finite-dimensional complex inner product space \(\V\).
  Then \(\T\) is normal iff there exists an orthonormal basis for \(\V\) consisting of eigenvectors of \(\T\).
\end{thm}

\begin{proof}[\pf{6.16}]
  Suppose that \(\T\) is normal.
  By the fundamental theorem of algebra (\cref{d.4}), the characteristic polynomial of \(\T\) splits.
  So we may apply Schur's theorem (\cref{6.14}) to obtain an orthonormal basis \(\beta = \set{\seq{v}{1,,n}}\) for \(\V\) over \(\F\) such that \([\T]_{\beta} = A\) is upper triangular.
  We know that \(v_1\) is an eigenvector of \(\T\) because \(A\) is upper triangular.
  Assume that \(\seq{v}{1,,k-1}\) are eigenvectors of \(\T\).
  We claim that \(v_k\) is also an eigenvector of \(\T\).
  It then follows by mathematical induction on \(k\) that all of the \(v_i\)'s are eigenvectors of \(\T\).
  Consider any \(j < k\), and let \(\lambda_j\) denote the eigenvalue of \(\T\) corresponding to \(v_j\).
  By \cref{6.15}(c), \(\T^*(v_j) = \conj{\lambda_j} v_j\).
  Since \(A\) is upper triangular, by \cref{2.2.4} we have
  \[
    \T(v_k) = A_{1 k} v_1 + \cdots + A_{j k} v_j + \cdots + A_{k k} v_k.
  \]
  Furthermore,
  \begin{align*}
    A_{j k} & = \inn{\T(v_k), v_j}              &  & \text{(by \cref{6.2.6})}   \\
            & = \inn{v_k, \T^*(v_j)}            &  & \text{(by \cref{6.9})}     \\
            & = \inn{v_k, \conj{\lambda_j} v_j} &  & \text{(by \cref{6.15}(c))} \\
            & = \lambda_j \inn{v_k, v_j}        &  & \text{(by \cref{6.1}(b))}  \\
            & = 0.                              &  & \text{(by \cref{6.1.12})}
  \end{align*}
  It follows that \(\T(v_k) = A_{k k} v_k\), and hence \(v_k\) is an eigenvector of \(\T\).
  So by induction, all the vectors in \(\beta\) are eigenvectors of \(\T\).

  The converse was already proved by \cref{6.4.2}.
\end{proof}

\begin{note}
  Interestingly, as \cref{6.4.7} shows, \cref{6.16} does not extend to infinite-dimensional complex inner product spaces.
\end{note}

\begin{eg}\label{6.4.7}
  Consider the inner product space \(\vs{H}\) with the orthonormal set \(S\) from \cref{6.1.13}.
  Let \(\V = \spn{S}\), and let \(\T, \U \in \ls(\V)\) defined by \(\T(f) = f_1 f\) and \(\U(f) = f_{-1} f\).
  Then
  \[
    \T(f_n) = f_{n + 1} \quad \text{and} \quad \U(f_n) = f_{n - 1}
  \]
  for all integers \(n\).
  Thus
  \begin{align*}
    \inn{\T(f_m), f_n} & = \inn{f_{m + 1}, f_n}                                \\
                       & = \delta_{(m + 1), n}  &  & \text{(by \cref{6.1.13})} \\
                       & = \delta_{m, (n - 1)}                                 \\
                       & = \inn{f_m, f_{n - 1}} &  & \text{(by \cref{6.1.13})} \\
                       & = \inn{f_m, \U(f_n)}.
  \end{align*}
  It follows that \(\U = \T^*\).
  Furthermore, \(\T \T^* = \IT[\V] = \T^* \T\);
  so \(\T\) is normal.

  We show that \(\T\) has no eigenvectors.
  Suppose that \(f\) is an eigenvector of \(\T\), say, \(\T(f) = \lambda f\) for some \(\lambda\).
  Since \(\V\) equals the span of \(S\), we may write
  \[
    f = \sum_{i = n}^m a_i f_i, \quad \text{where } a_m \neq 0.
  \]
  Hence
  \[
    \sum_{i = n}^m a_i f_{i + 1} = \T(f) = \lambda f = \sum_{i = n}^m \lambda a_i f_i.
  \]
  Since \(a_m \neq 0\), we can write \(f_{m + 1}\) as a linear combination of \(\seq{f}{n,n+1,,m}\).
  But this is a contradiction because \(S\) is linearly independent.
\end{eg}

\begin{note}
  \cref{6.4.5} illustrates that normality is not sufficient to guarantee the existence of an orthonormal basis of eigenvectors for real inner product spaces.
  For real inner product spaces, we must replace normality by the stronger condition that \(\T = \T^*\) in order to guarantee such a basis.
\end{note}

\begin{defn}\label{6.4.8}
  Let \(\T\) be a linear operator on an inner product space \(\V\) over \(\F\).
  We say that \(\T\) is \textbf{self-adjoint} (\textbf{Hermitian}) if \(\T = \T^*\).
  An \(n \times n\) real or complex matrix \(A\) is \textbf{self-adjoint} (\textbf{Hermitian}) if \(A = A^*\).
\end{defn}

\begin{cor}\label{6.4.9}
  If \(\beta\) is an orthonormal basis, then \(\T\) is self-adjoint iff \([\T]_{\beta}\) is self-adjoint.
  For real matrices, this condition reduces to the requirement that \(A\) be symmetric.
\end{cor}

\begin{proof}[\pf{6.4.9}]
  We have
  \begin{align*}
         & \T \text{ is self-adjoint}                                                   \\
    \iff & \T = \T^*                                      &  & \text{(by \cref{6.4.8})} \\
    \iff & [\T]_{\beta} = [\T^*]_{\beta} = [\T]_{\beta}^* &  & \text{(by \cref{6.10})}  \\
    \iff & [\T]_{\beta} \text{ is self-adjoint}.          &  & \text{(by \cref{6.4.8})}
  \end{align*}
\end{proof}

\begin{lem}\label{6.4.10}
  Let \(\T\) be a self-adjoint operator on a finite-dimensional inner product space \(\V\) over \(\F\).
  Then
  \begin{enumerate}
    \item Every eigenvalue of \(\T\) is real.
    \item Suppose that \(\V\) is a real inner product space.
          Then the characteristic polynomial of \(\T\) splits.
  \end{enumerate}
\end{lem}

\begin{proof}[\pf{6.4.10}(a)]
  Suppose that \(\T(x) = \lambda x\) for \(x \neq \zv\).
  Because a self-adjoint operator is also normal, we have
  \begin{align*}
    \lambda x & = \T(x)             &  & \text{(by \cref{5.1.2})}   \\
              & = \T^*(x)           &  & \text{(by \cref{6.4.8})}   \\
              & = \conj{\lambda} x. &  & \text{(by \cref{6.15}(c))}
  \end{align*}
  So \(\lambda = \conj{\lambda}\);
  that is, \(\lambda\) is real.
\end{proof}

\begin{proof}[\pf{6.4.10}(b)]
  Let \(n = \dim(\V)\), \(\beta\) be an orthonormal basis for \(\V\) over \(\R\), and \(A = [\T]_{\beta}\).
  Then \(A\) is self-adjoint by \cref{6.4.9}.
  Let \(\T_A\) be the linear operator on \(\C^n\) defined by \(\T_A(x) = Ax\) for all \(x \in \C^n\).
  Note that \(\T_A\) is self-adjoint because \([\T_A]_{\gamma} = A\), where \(\gamma\) is the standard ordered (orthonormal) basis for \(\C^n\) over \(\C\) (see \cref{2.15}(a)).
  So, by (a), the eigenvalues of \(\T_A\) are real.
  By the fundamental theorem of algebra (\cref{d.4}), the characteristic polynomial of \(\T_A\) splits into factors of the form \(t - \lambda\).
  Since each \(\lambda\) is real, the characteristic polynomial splits over \(\R\).
  But \(\T_A\) has the same characteristic polynomial as \(A\), which has the same characteristic polynomial as \(\T\) (see \cref{5.1.6}).
  Therefore the characteristic polynomial of \(\T\) splits.
\end{proof}

\begin{thm}\label{6.17}
  Let \(\T\) be a linear operator on a finite-dimensional real inner product space \(\V\).
  Then \(\T\) is self-adjoint iff there exists an orthonormal basis \(\beta\) for \(\V\) over \(\R\) consisting of eigenvectors of \(\T\).
\end{thm}

\begin{proof}[\pf{6.17}]
  Suppose that \(\T\) is self-adjoint.
  By \cref{6.4.10}(b), we may apply Schur's theorem (\cref{6.14}) to obtain an orthonormal basis \(\beta\) for \(\V\) over \(\R\) such that the matrix \(A = [\T]_{\beta}\) is upper triangular.
  But
  \begin{align*}
    A^* & = [\T]_{\beta}^*                               \\
        & = [\T^*]_{\beta} &  & \text{(by \cref{6.10})}  \\
        & = [\T]_{\beta}   &  & \text{(by \cref{6.4.8})} \\
        & = A.
  \end{align*}
  So \(A\) and \(A^*\) are both upper triangular, and therefore \(A\) is a diagonal matrix.
  Thus \(\beta\) must consist of eigenvectors of \(\T\).

  Now suppose that there exists an orthonormal basis \(\beta\) for \(\V\) over \(\R\) consisting of eigenvectors of \(\T\).
  By \cref{5.1.1} we know that \([\T]_{\beta}\) is a diagonal matrix.
  Since \(\V\) is over \(\R\), we have
  \begin{align*}
    [\T^*]_{\beta} & = [\T]_{\beta}^*      &  & \text{(by \cref{6.10})}                        \\
                   & = \tp{([\T]_{\beta})} &  & \text{(\(\V\) is over \(\R\))}                 \\
                   & = [\T]_{\beta}.       &  & \text{(\([\T]_{\beta}\) is a diagonal matrix)}
  \end{align*}
  Thus by \cref{6.4.8} \([\T]_{\beta}\) is self-adjoint and by \cref{6.4.9} \(\T\) is self-adjoint.
\end{proof}

\begin{note}
  \cref{6.17} is used extensively in many areas of mathematics and statistics.
  We restate this theorem in matrix form in \cref{sec:6.5}.
\end{note}

\exercisesection

\setcounter{ex}{3}
\begin{ex}\label{ex:6.4.4}
  Let \(\T\) and \(\U\) be self-adjoint operators on an inner product space \(\V\) over \(\F\).
  Prove that \(\T \U\) is self-adjoint iff \(\T \U = \U \T\).
\end{ex}

\begin{proof}[\pf{ex:6.4.4}]
  We have
  \begin{align*}
         & \T \U \text{ is self-adjoint}                                 \\
    \iff & \T \U = (\T \U)^*             &  & \text{(by \cref{6.4.8})}   \\
         & = \U^* \T^*                   &  & \text{(by \cref{6.11}(c))} \\
         & = \U \T.                      &  & \text{(by \cref{6.4.8})}
  \end{align*}
\end{proof}

\setcounter{ex}{5}
\begin{ex}\label{ex:6.4.6}
  Let \(\V\) be a complex inner product space, and let \(\T \in \ls(\V)\).
  Define
  \[
    \T_1 = \frac{1}{2} (\T + \T^*) \quad \text{and} \quad \T_2 = \frac{1}{2i} (\T - \T^*).
  \]
  \begin{enumerate}
    \item Prove that \(\T_1\) and \(\T_2\) are self-adjoint and that \(\T = \T_1 + i \T_2\).
    \item Suppose also that \(\T = \U_1 + i \U_2\), where \(\U_1\) and \(\U_2\) are self-adjoint.
          Prove that \(\U_1 = \T_1\) and \(\U_2 = \T_2\).
    \item Prove that \(\T\) is normal iff \(\T_1 \T_2 =  \T_2 \T_1\).
  \end{enumerate}
\end{ex}

\begin{proof}[\pf{ex:6.4.6}(a)]
  We have
  \begin{align*}
    \T_1^* & = \pa{\frac{1}{2} (\T + \T^*)}^*                                        \\
           & = \frac{1}{2} (\T^* + \T)         &  & \text{(by \cref{6.11}(a)(b)(d))} \\
           & = \T_1                                                                  \\
    \T_2^* & = \pa{\frac{1}{2i} (\T - \T^*)}^*                                       \\
           & = \frac{-1}{2i} (\T^* - \T)       &  & \text{(by \cref{6.11}(a)(b)(d))} \\
           & = \frac{1}{2i} (\T - \T^*)                                              \\
           & = \T_2.
  \end{align*}
  Thus by \cref{6.4.8} \(\T_1, \T_2\) are self-adjoint.
  By definition we have
  \begin{align*}
    \T_1 + i \T_2 & = \frac{1}{2} (\T + \T^*) + \frac{i}{2i} (\T - \T^*) \\
                  & = \T.
  \end{align*}
\end{proof}

\begin{proof}[\pf{ex:6.4.6}(b)]
  We have
  \begin{align*}
             & \T = \begin{dcases}
                      \T_1 + i \T_2 \\
                      \U_1 + i \U_2
                    \end{dcases}                      &  & \text{(by \cref{ex:6.4.6}(a))} \\
    \implies & \T^* = \begin{dcases}
                        \T_1^* - i \T_2^* \\
                        \U_1^* - i \U_2^*
                      \end{dcases}                    &  & \text{(by \cref{6.11}(a)(b))}  \\
             & = \begin{dcases}
                   \T_1 - i \T_2 \\
                   \U_1 - i \U_2
                 \end{dcases}                         &  & \text{(by \cref{6.4.8})}       \\
    \implies & \begin{dcases}
                 \T_1 = \frac{1}{2} (\T + \T^*) = \U_1 \\
                 \T_2 = \frac{1}{2i} (\T - \T^*) = \U_2
               \end{dcases}.
  \end{align*}
\end{proof}

\begin{proof}[\pf{ex:6.4.6}(c)]
  We have
  \begin{align*}
         & \T \text{ is normal}                                                                          \\
    \iff & \T \T^* = \T^* \T                                               &  & \text{(by \cref{6.4.3})} \\
    \iff & (\T + \T^*) (\T - \T^*) = \T^2 - \T \T^* + \T^* \T - (\T^*)^2                                 \\
         & = \T^2 + \T \T^* - \T^* \T - (\T^*)^2 = (\T - \T^*) (\T + \T^*) &  & \text{(by \cref{2.10})}  \\
    \iff & \T_1 \T_2 = \frac{1}{4i} (\T + \T^*) (\T - \T^*)                                              \\
         & = \frac{1}{4i} (\T - \T^*) (\T + \T^*) = \T_2 \T_1.
  \end{align*}
\end{proof}

\begin{ex}\label{ex:6.4.7}
  Let \(\T\) be a linear operator on an inner product space \(\V\) over \(\F\), and let \(\W\) be a \(\T\)-invariant subspace of \(\V\) over \(\F\).
  Prove the following results.
  \begin{enumerate}
    \item If \(\T\) is self-adjoint, then \(\T_{\W}\) is self-adjoint.
    \item \(\W^{\perp}\) is \(\T^*\)-invariant.
    \item If \(\W\) is both \(\T\)- and \(\T^*\)-invariant, then \((\T_{\W})^* = (\T^*)_{\W}\).
    \item If \(\W\) is both \(\T\)- and \(\T^*\)-invariant and \(\T\) is normal, then \(\T_{\W}\) is normal.
  \end{enumerate}
\end{ex}

\begin{proof}[\pf{ex:6.4.7}(a)]
  We have
  \begin{align*}
    \forall x, y \in \W, \inn{x, (\T_{\W})^*(y)} & = \inn{\T_{\W}(x), y} &  & \text{(by \cref{6.9})}   \\
                                                 & = \inn{\T(x), y}      &  & \text{(by \cref{5.4.1})} \\
                                                 & = \inn{x, \T^*(y)}    &  & \text{(by \cref{6.9})}   \\
                                                 & = \inn{x, \T(y)}      &  & \text{(by \cref{6.4.8})} \\
                                                 & = \inn{x, \T_{\W}(y)} &  & \text{(by \cref{5.4.1})}
  \end{align*}
  and thus by \cref{6.1}(e) \(\T_{\W} = (\T_{\W})^*\).
  By \cref{6.4.8} this means \(\T_{\W}\) is self-adjoint.
\end{proof}

\begin{proof}[\pf{ex:6.4.7}(b)]
  Since
  \begin{align*}
             & v \in \W^{\perp}                                                     \\
    \implies & \forall x \in \W, \inn{v, x} = 0       &  & \text{(by \cref{6.2.9})} \\
    \implies & \forall x \in \W, \inn{v, \T(x)} = 0   &  & \text{(by \cref{5.4.1})} \\
    \implies & \forall x \in \W, \inn{\T^*(v), x} = 0 &  & \text{(by \cref{6.9})}   \\
    \implies & \T^*(v) \in \W^{\perp},                &  & \text{(by \cref{6.2.9})}
  \end{align*}
  by \cref{5.4.1} we see that \(\W^{\perp}\) is \(\T^*\)-invariant.
\end{proof}

\begin{proof}[\pf{ex:6.4.7}(c)]
  Since
  \begin{align*}
    \forall x, y \in \W, \inn{x, (\T^*)_{\W}(y)} & = \inn{x, \T^*(y)}         &  & \text{(by \cref{5.4.1})} \\
                                                 & = \inn{\T(x), y}           &  & \text{(by \cref{6.9})}   \\
                                                 & = \inn{\T_{\W}(x), y}      &  & \text{(by \cref{5.4.1})} \\
                                                 & = \inn{x, (\T_{\W})^*(y)}, &  & \text{(by \cref{6.9})}
  \end{align*}
  by \cref{6.1}(e) we see that \((\T^*)_{\W} = (\T_{\W})^*\).
\end{proof}

\begin{proof}[\pf{ex:6.4.7}(d)]
  We have
  \begin{align*}
    \forall x \in \W, (\T_{\W} (\T_{\W})^*)(x) & = (\T_{\W} (\T^*)_{\W})(x) &  & \text{(by \cref{ex:6.4.7}(c))} \\
                                               & = (\T \T^*)(x)             &  & \text{(by \cref{5.4.1})}       \\
                                               & = (\T^* \T)(x)             &  & \text{(by \cref{6.4.5})}       \\
                                               & = ((\T^*)_{\W} \T_{\W})(x) &  & \text{(by \cref{5.4.1})}       \\
                                               & = ((\T_{\W})^* \T_{\W})(x) &  & \text{(by \cref{ex:6.4.7}(c))}
  \end{align*}
  and thus \(\T_{\W} (\T_{\W})^* = (\T_{\W})^* \T_{\W}\).
  By \cref{6.4.5} this means \(\T_{\W}\) is normal.
\end{proof}

\begin{ex}\label{ex:6.4.8}
  Let \(\T\) be a normal operator on a finite-dimensional complex inner product space \(\V\), and let \(\W\) be a subspace of \(\V\) over \(\C\).
  Prove that if \(\W\) is \(\T\)-invariant, then \(\W\) is also \(\T^*\)-invariant.
\end{ex}

\begin{proof}[\pf{ex:6.4.8}]
  If \(\W = \set{\zv}\), then by \cref{2.1.2}(a) we have \(\T^*(\set{\zv}) = \set{\zv}\) (this is true since \(\T^* \in \ls(\V)\) by \cref{6.9}).
  Thus \(\W\) is \(\T^*\)-invariant in this case.

  Now suppose that \(\W \neq \set{\zv}\).
  Since \(\T\) is normal and \(\V\) is over \(\C\), by \cref{6.16} there exists an orthonormal basis \(\beta\) for \(\V\) over \(\C\) consisting of eigenvectors of \(\T\).
  By \cref{ex:5.4.24} this means \(\T_{\W}\) is diagonalizable.
  If we let \(\beta_{\W} = \beta \cap \W\), then by \cref{ex:5.4.24} we know that \(\beta_{\W}\) is an orthonormal basis for \(\W\) over \(\C\) consisting of eigenvectors of \(\T_{\W}\).
  By \cref{6.15}(c) we know that \(\beta_{\W}\) is a set of eigenvectors of \(\T^*\).
  Thus by \cref{5.4.2}(e) we see that \(\W\) is \(\T^*\)-invariant.
\end{proof}

\begin{ex}\label{ex:6.4.9}
  Let \(\T\) be a normal operator on a finite-dimensional inner product space \(\V\) over \(\F\).
  Prove that \(\ns{\T} = \ns{\T^*}\) and \(\rg{\T} = \rg{\T^*}\).
\end{ex}

\begin{proof}[\pf{ex:6.4.9}]
  Since
  \begin{align*}
         & x \in \ns{\T}                                      \\
    \iff & \T(x) = \zv        &  & \text{(by \cref{2.1.10})}  \\
    \iff & \norm{\T(x)} = 0   &  & \text{(by \cref{6.2}(b))}  \\
    \iff & \norm{\T^*(x)} = 0 &  & \text{(by \cref{6.15}(a))} \\
    \iff & \T^*(x) = \zv      &  & \text{(by \cref{6.2}(b))}  \\
    \iff & x \in \ns{\T^*},   &  & \text{(by \cref{2.1.10})}
  \end{align*}
  we have \(\ns{\T} = \ns{\T^*}\).
  Thus
  \begin{align*}
    \rg{\T^*} & = \ns{\T}^{\perp}   &  & \text{(by \cref{ex:6.3.12}(b))} \\
              & = \ns{\T^*}^{\perp} &  & \text{(from the proof above)}   \\
              & = \rg{\T^{**}}      &  & \text{(by \cref{ex:6.3.12}(b))} \\
              & = \rg{\T}.          &  & \text{(by \cref{6.11}(d))}
  \end{align*}
\end{proof}

\begin{ex}\label{ex:6.4.10}
  Let \(\T\) be a self-adjoint operator on a finite-dimensional inner product space \(\V\) over \(\F\).
  Prove that for all \(x \in \V\)
  \[
    \norm{\T(x) \pm ix}^2 = \norm{\T(x)}^2 + \norm{x}^2.
  \]
  Deduce that \(\T - i \IT[\V]\) is invertible and that \(\pa{(\T - i \IT[\V])^{-1}}^* = (\T + i \IT[\V])^{-1}\).
\end{ex}

\begin{proof}[\pf{ex:6.4.10}]
  Since
  \begin{align*}
    \inn{\T(x), x} & = \inn{\T^*(x), x}       &  & \text{(by \cref{6.4.8})}    \\
                   & = \inn{x, \T(x)}         &  & \text{(by \cref{6.9})}      \\
                   & = \conj{\inn{\T(x), x}}, &  & \text{(by \cref{6.1.1}(c))}
  \end{align*}
  we know that \(\inn{\T(x), x} \in \R\) for all \(x \in \V\).
  Thus
  \begin{align*}
    \norm{\T(x) \pm ix}^2 & = \norm{\T(x)}^2 \pm 2 \Re(\inn{\T(x), ix}) + \norm{ix}^2   &  & \text{(by \cref{ex:6.1.19}(a))} \\
                          & = \norm{\T(x)}^2 \pm 2 \Re(-i \inn{\T(x), x}) + \norm{ix}^2 &  & \text{(by \cref{6.1}(b))}       \\
                          & = \norm{\T(x)}^2 + \norm{ix}^2                              &  & \text{(from the proof above)}   \\
                          & = \norm{\T(x)}^2 + \norm{x}^2.                              &  & \text{(by \cref{6.2}(a))}
  \end{align*}

  Next we show that \(\T - i \IT[\V]\) is invertible.
  Observe that
  \begin{align*}
             & x \in \ns{\T - i \IT[\V]}                                          \\
    \implies & \T(x) - ix = \zv                &  & \text{(by \cref{2.1.10})}     \\
    \implies & \norm{\T(x) - ix} = 0           &  & \text{(by \cref{6.2}(b))}     \\
    \implies & \norm{\T(x)}^2 + \norm{x}^2 = 0 &  & \text{(from the proof above)} \\
    \implies & \norm{\T(x)}^2 = \norm{x}^2 = 0 &  & \text{(by \cref{6.2}(b))}     \\
    \implies & x = 0.                          &  & \text{(by \cref{6.2}(b))}
  \end{align*}
  Thus by \cref{2.5} \(\T - i \IT[\V]\) is invertible.

  Finally we show that \(\pa{(\T - i \IT[\V])^{-1}}^* = (\T + i \IT[\V])^{-1}\).
  Since
  \begin{align*}
    \forall x, y \in \V, & \inn{x, \pa{\pa{(\T - i \IT[\V])^{-1}}^* (\T + i \IT[\V])}(y)}                                       \\
                         & = \inn{(\T - i \IT[\V])^{-1}(x), (\T + i \IT[\V])(y)}          &  & \text{(by \cref{6.9})}           \\
                         & = \inn{(\T - i \IT[\V])^{-1}(x), (\T^* + i \IT[\V])(y)}        &  & \text{(by \cref{6.4.8})}         \\
                         & = \inn{(\T - i \IT[\V])^{-1}(x), (\T - i \IT[\V])^*(y)}        &  & \text{(by \cref{6.11}(a)(b)(e))} \\
                         & = \inn{\pa{(\T - i \IT[\V]) (\T - i \IT[\V])^{-1}}(x), y}      &  & \text{(by \cref{6.9})}           \\
                         & = \inn{x, y}                                                                                         \\
                         & = \inn{x, \IT[\V](y)},
  \end{align*}
  by \cref{6.1}(e) we have \(\pa{(\T - i \IT[\V])^{-1}}^* (\T + i \IT[\V]) = \IT[\V]\).
  Thus \(\pa{(\T - i \IT[\V])^{-1}}^* = (\T + i \IT[\V])^{-1}\).
\end{proof}

\begin{ex}\label{ex:6.4.11}
  Assume that \(\T\) is a linear operator on a complex (not necessarily finite-dimensional) inner product space \(\V\) with an adjoint \(\T^*\).
  Prove the following results.
  \begin{enumerate}
    \item If \(\T\) is self-adjoint, then \(\inn{\T(x), x}\) is real for all \(x \in \V\).
    \item If \(\T\) satisfies \(\inn{\T(x), x} = 0\) for all \(x \in \V\), then \(\T = \zT\).
    \item If \(\inn{\T(x), x}\) is real for all \(x \in \V\), then \(\T = \T^*\).
  \end{enumerate}
\end{ex}

\begin{proof}[\pf{ex:6.4.11}(a)]
  Since
  \begin{align*}
    \forall x \in \V, \inn{\T(x), x} & = \inn{\T^*(x), x}       &  & \text{(by \cref{6.4.8})}    \\
                                     & = \inn{x, \T(x)}         &  & \text{(by \cref{6.9})}      \\
                                     & = \conj{\inn{\T(x), x}}, &  & \text{(by \cref{6.1.1}(c))}
  \end{align*}
  we see that \(\inn{\T(x), x} \in \R\) for all \(x \in \V\).
\end{proof}

\begin{proof}[\pf{ex:6.4.11}(b)]
  Let \(x, y \in \V\).
  Since
  \begin{align*}
    0 & = \inn{\T(x + y), x + y}                                                                             \\
      & = \inn{\T(x) + \T(y), x + y}                                        &  & \text{(by \cref{2.1.1}(a))} \\
      & = \inn{\T(x), x + y} + \inn{\T(y), x + y}                           &  & \text{(by \cref{6.1.1}(a))} \\
      & = \inn{\T(x), x} + \inn{\T(x), y} + \inn{\T(y), x} + \inn{\T(y), y} &  & \text{(by \cref{6.1}(a))}   \\
      & = \inn{\T(x), y} + \inn{\T(y), x},
  \end{align*}
  we have \(\inn{\T(x), y} = - \inn{\T(y), x}\).
  Since
  \begin{align*}
    0 & = \inn{\T(x + iy), x + iy}                                                                                  \\
      & = \inn{\T(x) + i \T(y), x + iy}                                         &  & \text{(by \cref{2.1.2}(b))}    \\
      & = \inn{\T(x), x + iy} + i \inn{\T(y), x + iy}                           &  & \text{(by \cref{6.1.1}(a)(b))} \\
      & = \inn{\T(x), x} - i \inn{\T(x), y} + i \inn{\T(y), x} + \inn{\T(y), y} &  & \text{(by \cref{6.1}(a)(b))}   \\
      & = -i \inn{\T(x), y} + i \inn{\T(y), x},
  \end{align*}
  we have \(\inn{\T(x), y} = \inn{\T(y), x}\).
  Thus
  \begin{align*}
             & \forall x, y \in \V, \begin{dcases}
                                      \inn{\T(x), y} = - \inn{\T(y), x} \\
                                      \inn{\T(x), y} = \inn{\T(y), x}
                                    \end{dcases}                                  \\
    \implies & \forall x, y \in \V, \inn{\T(x), y} = \inn{\T(y), x} = 0                                \\
    \implies & \T = \zT.                                                &  & \text{(by \cref{6.1}(e))}
  \end{align*}
\end{proof}

\begin{proof}[\pf{ex:6.4.11}(c)]
  We have
  \begin{align*}
    \forall x \in \V, \inn{x, \T(x)} & = \conj{\inn{\T(x), x}} &  & \text{(by \cref{6.1.1}(c))} \\
                                     & = \inn{\T(x), x}        &  & (\inn{\T(x), x} \in \R)     \\
                                     & = \inn{x, \T^*(x)}      &  & \text{(by \cref{6.9})}
  \end{align*}
  and
  \begin{align*}
    \forall x \in \V, 0 & = \inn{x, \T(x)} - \inn{x, \T^*(x)} &  & \text{(from the proof above)} \\
                        & = \inn{x, \T(x) - \T^*(x)}          &  & \text{(by \cref{6.1}(a)(b))}  \\
                        & = \inn{x, (\T - \T^*)(x)}           &  & \text{(by \cref{2.10})}       \\
                        & = \inn{(\T - \T^*)(x), x}.          &  & \text{(by \cref{6.1.1}(c))}
  \end{align*}
  Thus
  \begin{align*}
             & \T - \T^* = \zT &  & \text{(by \cref{ex:6.4.11}(b))} \\
    \implies & \T = \T^*.
  \end{align*}
\end{proof}

\begin{ex}\label{ex:6.4.12}
  Let \(\T\) be a normal operator on a finite-dimensional real inner product space \(\V\) whose characteristic polynomial splits.
  Prove that \(\V\) has an orthonormal basis of eigenvectors of \(\T\).
  Hence prove that \(\T\) is self-adjoint.
\end{ex}

\begin{proof}[\pf{ex:6.4.12}]
  By \cref{6.14} we know that there exists an orthonormal basis \(\beta = \set{\seq{v}{1,,n}}\) for \(\V\) over \(\F\) such that \([\T]_{\beta}\) is upper triangular.
  We claim that for all \(k \in \set{1, \dots, n}\), the set \(\set{\seq{v}{1,,k}} \subseteq \beta\) is consist of eigenvectors of \(\T\).
  We use induction on \(k\) to proof the claim.
  For \(k = 1\), since \([\T]_{\beta}\) is upper triangular, by \cref{2.2.4} this means \(v_1\) is an eigenvector of \(\T\).
  Thus the base case holds.
  Suppose inductively that \(\set{\seq{v}{1,,k}}\) is consist of eigenvectors of \(\T\) for some \(k \geq 1\).
  We show that the claim is true for \(k + 1\).
  By induction hypothesis we know that \(\seq{v}{1,,k}\) are eigenvectors of \(\T\) corresponding to eigenvalues \(\seq{\lambda}{1,,k} \in \R\).
  Since \(\T\) is normal, by \cref{6.15}(c) we know that \(\T^*(v_i) = \conj{\lambda_i} v_i\) for all \(i \in \set{1, \dots, k}\).
  Then we have
  \begin{align*}
    \T(v_{k + 1}) & = \sum_{j = 1}^n \inn{\T(v_{k + 1}), v_j} v_j                                                           &  & \text{(by \cref{6.2.6})}         \\
                  & = \sum_{j = 1}^{k + 1} \inn{\T(v_{k + 1}), v_j} v_j                                                     &  & \text{(by \cref{6.14})}          \\
                  & = \sum_{j = 1}^{k + 1} \inn{v_{k + 1}, \T^*(v_j)} v_j                                                   &  & \text{(by \cref{6.9})}           \\
                  & = \inn{v_{k + 1}, \T^*(v_{k + 1})} v_{k + 1} + \sum_{j = 1}^k \inn{v_{k + 1}, \T^*(v_j)} v_j                                                  \\
                  & = \inn{v_{k + 1}, \T^*(v_{k + 1})} v_{k + 1} + \sum_{j = 1}^k \inn{v_{k + 1}, \conj{\lambda_j} v_j} v_j &  & \text{(by induction hypothesis)} \\
                  & = \inn{v_{k + 1}, \T^*(v_{k + 1})} v_{k + 1} + \sum_{j = 1}^k \lambda_j \inn{v_{k + 1}, v_j} v_j        &  & \text{(by \cref{6.1}(b))}        \\
                  & = \inn{v_{k + 1}, \T^*(v_{k + 1})} v_{k + 1}                                                            &  & \text{(by \cref{6.1.12})}
  \end{align*}
  and thus \(v_{k + 1}\) is an eigenvector of \(\T\).
  This closes the induction.
  Hence \(\beta\) is consist of eigenvectors of \(\T\).
  By \cref{5.1.1} this means \([\T]_{\beta}\) is a diagonal matrix.
  Thus \([\T]_{\beta}^* = [\T]_{\beta}\) and by \cref{6.4.9} \(\T\) is self-adjoint.
\end{proof}

\begin{ex}\label{ex:6.4.13}
  An \(A \in \ms{n}{n}{\R}\) is said to be a \textbf{Gramian} matrix if there exists a \(B \in \ms{n}{n}{\R}\) such that \(A = \tp{B} B\).
  Prove that \(A\) is a Gramian matrix iff \(A\) is symmetric and all of its eigenvalues are nonnegative.
\end{ex}

\begin{proof}[\pf{ex:6.4.13}]
  First suppose that \(A\) is Gramian.
  By \cref{ex:6.4.13} there exists a \(B \in \ms{n}{n}{\R}\) such that \(A = \tp{B} B\).
  Then we have
  \begin{align*}
    \tp{A} & = \tp{(\tp{B} B)}                                       \\
           & = \tp{B} \tp{(\tp{B})} &  & \text{(by \cref{2.3.2})}    \\
           & = \tp{B} B             &  & \text{(by \cref{ex:1.3.4})} \\
           & = A.
  \end{align*}
  Thus by \cref{1.3.4} \(A\) is symmetric.
  If \(\lambda\) is an eigenvalue of \(A\) and \(x\) is an eigenvector corresponding to \(\lambda\) with unit length, then we have
  \begin{align*}
    \lambda & = \lambda \norm{x}^2  &  & \text{(by \cref{6.1.12})}   \\
            & = \lambda \inn{x, x}  &  & \text{(by \cref{6.1.9})}    \\
            & = \inn{\lambda x, x}  &  & \text{(by \cref{6.1.1}(b))} \\
            & = \inn{Ax, x}         &  & \text{(by \cref{5.1.2})}    \\
            & = \inn{\tp{B} B x, x}                                  \\
            & = \inn{B^* B x, x}    &  & \text{(by \cref{6.1.5})}    \\
            & = \inn{Bx, Bx}        &  & \text{(by \cref{6.9})}      \\
            & = \norm{Bx}^2         &  & \text{(by \cref{6.1.9})}    \\
            & \geq 0.               &  & \text{(by \cref{6.2}(b))}
  \end{align*}
  Thus all eigenvalues of \(A\) are nonnegative.

  Now suppose that \(A\) is symmetric and all eigenvalues of \(A\) are nonnegative.
  Since
  \begin{align*}
    A & = \tp{A} &  & \text{(by \cref{1.3.4})} \\
      & = A^*,   &  & \text{(by \cref{6.1.5})}
  \end{align*}
  by \cref{6.4.8} we know that \(A\) is self-adjoint.
  By \cref{6.17} there exists an orthonormal basis \(\beta = \set{\seq{v}{1,,n}}\) for \(\vs{\R}^n\) over \(\R\) consisting of eigenvectors of \(A\).
  For each \(i \in \set{1, \dots, n}\), let \(\lambda_i\) be an eigenvalue of \(A\) such that \(A v_i = \lambda v_i\).
  By \cref{2.6} we can define an \(\U \in \ls(\vs{\R}^n)\) such that \(\U(v_i) = \sqrt{v_i} v_i\) for all \(i \in \set{1, \dots, n}\).
  Then we have
  \begin{align*}
             & [\U]_{\beta} = \begin{pmatrix}
                                \sqrt{v_1} & 0          & \cdots & 0          \\
                                0          & \sqrt{v_2} & \cdots & 0          \\
                                \vdots     & \vdots     &        & \vdots     \\
                                0          & 0          & \cdots & \sqrt{v_n}
                              \end{pmatrix}                      &  & \text{(by \cref{2.2.4})}                     \\
    \implies & [\U]_{\beta}^* = \tp{([\U]_{\beta})} = [\U]_{\beta}                   &  & \text{(by \cref{6.1.5})} \\
    \implies & \tp{([\U]_{\beta})} [\U]_{\beta} = ([\U]_{\beta})^2 = [\L_A]_{\beta}. &  & \text{(by \cref{2.2.4})}
  \end{align*}
  If \(\gamma\) is the standard ordered basis for \(\vs{\R}^n\) over \(\R\) (which is orthonormal), then we have
  \begin{align*}
    A & = [\L_A]_{\gamma}                                                                                              &  & \text{(by \cref{2.15}(a))}      \\
      & = \pa{[\IT[\vs{F}^n]]_{\gamma}^{\beta}}^{-1} [\L_A]_{\beta} [\IT[\vs{F}^n]]_{\gamma}^{\beta}                   &  & \text{(by \cref{2.23})}         \\
      & = \pa{[\IT[\vs{F}^n]]_{\gamma}^{\beta}}^{-1} \tp{([\U]_{\beta})} [\U]_{\beta} [\IT[\vs{F}^n]]_{\gamma}^{\beta} &  & \text{(from the proof above)}   \\
      & = \pa{[\IT[\vs{F}^n]]_{\gamma}^{\beta}}^* \tp{([\U]_{\beta})} [\U]_{\beta} [\IT[\vs{F}^n]]_{\gamma}^{\beta}    &  & \text{(by \cref{ex:6.1.23}(c))} \\
      & = \tp{\pa{[\IT[\vs{F}^n]]_{\gamma}^{\beta}}} \tp{([\U]_{\beta})} [\U]_{\beta} [\IT[\vs{F}^n]]_{\gamma}^{\beta} &  & \text{(by \cref{6.1.5})}        \\
      & = \tp{\pa{[\U]_{\beta} [\IT[\vs{F}^n]]_{\gamma}^{\beta}}} [\U]_{\beta} [\IT[\vs{F}^n]]_{\gamma}^{\beta}.       &  & \text{(by \cref{2.3.2})}
  \end{align*}
  Thus \(A\) is Gramian.
\end{proof}

\begin{ex}[Simultaneous Diagonalization]\label{ex:6.4.14}
  Let \(\V\) be a finite-dimensional real inner product space, and let \(\U\) and \(\T\) be self-adjoint linear operators on \(\V\) such that \(\U \T = \T \U\).
  Prove that there exists an orthonormal basis for \(\V\) over \(\R\) consisting of vectors that are eigenvectors of both \(\U\) and \(\T\).
  (The complex version of this result appears as \cref{ex:6.6.10}.)
\end{ex}

\begin{proof}[\pf{ex:6.4.14}]
  Let \(n = \dim(\V)\).
  We use induction on \(n\).
  For \(n = 1\), any orthonormal basis for \(\V\) over \(\R\) makes \([\T]_{\beta}\) and \([\U]_{\beta}\) diagonal matrices.
  Thus by \cref{5.2.8} \(\T, \U\) are simultaneously diagonalizable by some orthonormal bases and the base case holds.
  Suppose inductively that for some \(n \geq 1\), self-adjoint linear operators on \(\V\) which commute are simultaneously diagonalizable by some orthonormal bases.
  We need to show that this is true for \(n + 1\).
  Let \(\dim(\V) = n + 1\), let \(\T, \U \in \ls(\V)\) such that \(\T^* = \T\), \(\U^* = \U\) and \(\U \T = \T \U\).
  Since \(\T\) is self-adjoint, by \cref{6.17} there exists an orthonormal basis \(\beta\) for \(\V\) over \(\R\) consisting of eigenvectors of \(\T\).
  Let \(\lambda \in \R\) be an eigenvalue of \(\T\) and let \(E_{\lambda}\) be the eigenspace of \(\lambda\).
  Now we split into two cases:
  \begin{itemize}
    \item If \(E_{\lambda} = \V\), then any basis for \(\V\) over \(\R\) is consist of eigenvectors of \(\T\).
          Since \(\U\) is self-adjoint, by \cref{6.17} we know that there exists an orthonormal basis for \(\V\) over \(\R\) consist of eigenvectors of \(\U\).
          Thus by \cref{5.2.8} \(\T, \U\) are simultaneously diagonalizable by some orthonormal bases for \(\V\) over \(\R\).
    \item If \(E_{\lambda} \neq \V\), then by \cref{5.7} we have \(1 \leq \dim(E_{\lambda}) < n + 1\).
          By \cref{5.4.2}(e) we know that \(E_{\lambda}\) is \(\T\)-invariant.
          we claim that \(E_{\lambda}\) is \(\U\)-invariant.
          Since
          \begin{align*}
                     & \forall v \in E_{\lambda}, \T(v) = \lambda v             &  & \text{(by \cref{5.2.4})}    \\
            \implies & \forall v \in E_{\lambda}, \lambda \U(v) = \U(\lambda v) &  & \text{(by \cref{2.1.1}(b))} \\
                     & = \U(\T(v)) = \T(\U(v))                                  &  & \text{(by \cref{5.1.2})}    \\
            \implies & \forall v \in E_{\lambda}, \U(v) \in E_{\lambda}         &  & \text{(by \cref{5.1.2})}    \\
            \implies & \U(E_{\lambda}) \subseteq E_{\lambda},
          \end{align*}
          by \cref{5.4.1} we know that \(E_{\lambda}\) is \(\U\)-invariant.
          Since \(\T, \U\) are self-adjoint, by \cref{ex:6.4.7}(b) we know that \(E_{\lambda}^{\perp}\) is both \(\T\)- and \(\U\)-invariant.
          By \cref{6.7}(c) we know that \(\dim(E_{\lambda}^{\perp}) < n + 1\).
          Thus by induction hypothesis we can find some orthonormal bases \(\beta_1\) and \(\beta_2\) for \(E_{\lambda}\) and \(E_{\lambda}^{\perp}\) over \(\R\), respectively, consist of eigenvectors of both \(\T\) and \(\U\).
          By \cref{5.10}(d) and \cref{6.7}(c) we know that \(\beta = \beta_1 \cup \beta_2\) is an orthonormal basis for \(\V\) over \(\R\) consist of eigenvectors of both \(\T\) and \(\U\).
          Thus by \cref{5.2.8} \(\T, \U\) are simultaneously diagonalizable by some orthonormal bases for \(\V\) over \(\R\).
  \end{itemize}
  From all cases above the induction is closed.
\end{proof}

\begin{ex}\label{ex:6.4.15}
  Let \(A, B \in \ms{n}{n}{\R}\) be symmetric such that \(AB = BA\).
  Use \cref{ex:6.4.14} to prove that there exists an orthogonal matrix \(P\) such that \(\tp{P} A P\) and \(\tp{P} B P\) are both diagonal matrices.
\end{ex}

\begin{proof}[\pf{ex:6.4.15}]
  Since
  \begin{align*}
    A & = \tp{A} &  & \text{(by \cref{1.3.4})} \\
      & = A^*    &  & \text{(by \cref{6.1.5})} \\
    B & = \tp{B} &  & \text{(by \cref{1.3.4})} \\
      & = B^*    &  & \text{(by \cref{6.1.5})}
  \end{align*}
  and
  \begin{align*}
    \L_A \L_B & = \L_{AB}    &  & \text{(by \cref{2.15}(e))} \\
              & = \L_{BA}                                    \\
              & = \L_B \L_A, &  & \text{(by \cref{2.15}(e))}
  \end{align*}
  by \cref{ex:6.4.14} there exists an orthonormal basis \(\gamma\) for \(\vs{R}^n\) over \(\R\) such that \([\L_A]_{\gamma}, [\L_B]_{\gamma}\) are diagonal matrices.
  If we let \(\beta\) be the standard ordered basis for \(\vs{R}^n\) over \(\R\) (which is orthonormal), then we have
  \begin{align*}
    [\L_A]_{\gamma} & = \pa{[\IT[\vs{F}^n]]_{\gamma}^{\beta}}^{-1} [\L_A]_{\beta} [\IT[\vs{F}^n]]_{\gamma}^{\beta} &  & \text{(by \cref{2.23})}         \\
                    & = \pa{[\IT[\vs{F}^n]]_{\gamma}^{\beta}}^{-1} A [\IT[\vs{F}^n]]_{\gamma}^{\beta}              &  & \text{(by \cref{2.15}(a))}      \\
                    & = \pa{[\IT[\vs{F}^n]]_{\gamma}^{\beta}}^* A [\IT[\vs{F}^n]]_{\gamma}^{\beta}                 &  & \text{(by \cref{ex:6.1.23}(c))} \\
                    & = \tp{\pa{[\IT[\vs{F}^n]]_{\gamma}^{\beta}}} A [\IT[\vs{F}^n]]_{\gamma}^{\beta}.             &  & \text{(by \cref{6.1.5})}
  \end{align*}
  Similarly we have \([\L_A]_{\gamma} = \tp{\pa{[\IT[\vs{F}^n]]_{\gamma}^{\beta}}} A [\IT[\vs{F}^n]]_{\gamma}^{\beta}\).
  By setting \(P = [\IT[\vs{F}^n]]_{\gamma}^{\beta}\) we are done.
\end{proof}

\begin{ex}\label{ex:6.4.16}
  Prove the \emph{Cayley--Hamilton theorem} for a complex \(n \times n\) matrix \(A\).
  That is, if \(f\) is the characteristic polynomial of \(A\), prove that \(f(A) = \zm\).
  (The general case is proved in \cref{5.23}.)
\end{ex}

\begin{proof}[\pf{ex:6.4.16}]
  Since \(A \in \ms{n}{n}{\C}\), by \cref{d.4} we know that \(\det(A - t I_n)\) splits.
  By \cref{6.14} there exists an orthonormal basis \(\gamma\) for \(\vs{C}^n\) over \(\C\) such that \([\L_A]_{\gamma}\) is upper triangular.
  By \cref{ex:4.2.23} we have
  \[
    \forall t \in \C, f(t) = \prod_{i = 1}^n (([\L_A]_{\gamma})_{i i} - t)
  \]
  and thus by \cref{e.0.7}
  \[
    f([\L_A]_{\gamma}) = \prod_{i = 1}^n (([\L_A]_{\gamma})_{i i} I_n - [\L_A]_{\gamma}).
  \]
  Observe that for each \(j \in \set{1, \dots, n}\), we have
  \begin{align*}
     & f([\L_A]_{\gamma})(e_j)                                       \\
     & = \pa{\prod_{\substack{i = 1                                  \\ i \neq j}}^n (([\L_A]_{\gamma})_{i i} I_n - [\L_A]_{\gamma})} \cdot (([\L_A]_{\gamma})_{j j} I_n - [\L_A]_{\gamma} e_j))         \\
     & = \pa{\prod_{\substack{i = 1                                  \\ i \neq j}}^n (([\L_A]_{\gamma})_{i i} I_n - [\L_A]_{\gamma})} \cdot (([\L_A]_{\gamma})_{j j} e_j - ([\L_A]_{\gamma})_{j j} e_j)) && \text{(\([\L_A]_{\gamma}\) is diagonal)} \\
     & = \pa{\prod_{\substack{i = 1                                  \\ i \neq j}}^n (([\L_A]_{\gamma})_{i i} I_n - [\L_A]_{\gamma})} \cdot \zv \\
     & = \zv.                       &  & \text{(by \cref{2.1.2}(a))}
  \end{align*}
  Thus
  \begin{align*}
             & \forall j \in \set{1, \dots, n}, f([\L_A]_{\gamma})(e_j) = \zv                                \\
    \implies & f([\L_A]_{\gamma}) = \zm.                                      &  & \text{(by \cref{2.1.13})}
  \end{align*}
  If we let \(\beta\) be the standard ordered basis for \(\C^n\) over \(\C\), then we have
  \begin{align*}
    f(A) & = f([\L_A]_{\beta})                                                                                                                                                               &  & \text{(by \cref{2.15}(a))}    \\
         & = f([\IT[\C^n]]_{\gamma}^{\beta} [\L_A]_{\gamma} [\IT[\C^n]]_{\beta}^{\gamma})                                                                                                    &  & \text{(by \cref{2.23})}       \\
         & = \prod_{i = 1}^n (([\L_A]_{\gamma})_{i i} I_n - [\IT[\C^n]]_{\gamma}^{\beta} [\L_A]_{\gamma} [\IT[\C^n]]_{\beta}^{\gamma})                                                                                          \\
         & = \prod_{i = 1}^n (([\L_A]_{\gamma})_{i i} [\IT[\C^n]]_{\gamma}^{\beta} [\IT[\C^n]]_{\beta}^{\gamma} - [\IT[\C^n]]_{\gamma}^{\beta} [\L_A]_{\gamma} [\IT[\C^n]]_{\beta}^{\gamma}) &  & \text{(by \cref{2.23})}       \\
         & = \prod_{i = 1}^n [\IT[\C^n]]_{\gamma}^{\beta} (([\L_A]_{\gamma})_{i i} I_n - [\L_A]_{\gamma}) [\IT[\C^n]]_{\beta}^{\gamma}                                                                                          \\
         & = [\IT[\C^n]]_{\gamma}^{\beta} \pa{\prod_{i = 1}^n (([\L_A]_{\gamma})_{i i} I_n - [\L_A]_{\gamma})} [\IT[\C^n]]_{\beta}^{\gamma}                                                                                     \\
         & = [\IT[\C^n]]_{\gamma}^{\beta} f([\L_A]_{\gamma}) [\IT[\C^n]]_{\beta}^{\gamma}                                                                                                                                       \\
         & = \zm.                                                                                                                                                                            &  & \text{(from the proof above)}
  \end{align*}
\end{proof}

\begin{defn}\label{6.4.11}
  A linear operator \(\T\) on a finite-dimensional inner product space is called \textbf{positive definite} (\textbf{positive semidefinite}) if \(\T\) is self-adjoint and \(\inn{\T(x), x} > 0\) (\(\inn{\T(x), x} \geq 0\)) for all \(x \neq \zv\).

  An \(n \times n\) matrix \(A\) with entries from \(\R\) or \(\C\) is called \textbf{positive definite} (\textbf{positive semidefinite}) if \(\L_A\) is positive definite (positive semidefinite).
\end{defn}

\begin{ex}\label{ex:6.4.17}
  Let \(\T\) and \(\U\) be a self-adjoint linear operators on an \(n\)-dimensional inner product space \(\V\) over \(\F\), and let \(A = [\T]_{\beta}\), where \(\beta\) is an orthonormal basis for \(\V\) over \(\F\).
  Prove the following results.
  \begin{enumerate}
    \item \(\T\) is positive definite (semidefinite) iff all of its eigenvalues are positive (nonnegative).
    \item \(\T\) is positive definite (semidefinite) iff
          \[
            \sum_{i = 1}^n \sum_{j = 1}^n A_{i j} a_j \conj{a_i} > 0 \; (\geq 0) \text{ for all nonzero } n\text{-tuples } \tuple{a}{1,,n}.
          \]
    \item \(\T\) is positive semidefinite iff \(A = B^* B\) for some square matrix \(B\).
    \item If \(\T\) and \(\U\) are positive semidefinite operators such that \(\T^2 = \U^2\), then \(\T = \U\).
    \item If \(\T\) and \(\U\) are positive definite operators such that \(\T \U = \U \T\), then \(\T \U\) is positive definite.
    \item \(\T\) is positive definite (semidefinite) iff \(A\) is positive definite (semidefinite).
  \end{enumerate}
  Because of (f), results analogous to items (a) through (d) hold for matrices as well as operators.
\end{ex}

\begin{proof}[\pf{ex:6.4.17}(a)]
  First suppose that \(\T\) is positive definite (semidefinite).
  Suppose that \(\lambda \in \F\) is an eigenvalue of \(\T\) and \(v \in \V\) is an eigenvector of \(\T\) corresponding to \(\lambda\).
  Then we have
  \begin{align*}
    \lambda & = \lambda \frac{\norm{v}^2}{\norm{v}^2}                &  & (v \neq \zv)                \\
            & = \frac{\lambda \inn{v, v}}{\norm{v}^2}                &  & \text{(by \cref{6.1.9})}    \\
            & = \frac{\inn{\lambda v, v}}{\norm{v}^2}                &  & \text{(by \cref{6.1.1}(b))} \\
            & = \frac{\inn{\T(v), v}}{\norm{v}^2}                    &  & \text{(by \cref{5.1.2})}    \\
            & \begin{dcases}
                > 0    & \text{if } \T \text{ is positive definite} \\
                \geq 0 & \text{if } \T \text{ is semidefinite}
              \end{dcases}. &  & \text{(by \cref{6.2}(b) and \cref{6.4.11})}
  \end{align*}

  Now suppose that eigenvalues of \(\T\) are positive (nonnegative).
  Since \(\T\) is self-adjoint (by hypothesis), by \cref{6.4.8} we have \(\T = \T^*\).
  Thus we have \(\T^* \T = \T^2 = \T \T^*\) and by \cref{6.4.3} \(\T\) is normal.
  By \cref{6.16} we know that there exists an orthonormal basis \(\gamma = \set{\seq{v}{1,,n}}\) for \(\V\) over \(\F\) consist of eigenvectors of \(\T\).
  For each \(i \in \set{1, \dots, n}\), let \(\lambda_i\) be the eigenvalue of \(\T\) correspond to \(v_i\).
  Then we have
  \begin{align*}
     & \forall x \in \V \setminus \set{\zv}, \inn{\T(x), x}                                                                   \\
     & = \inn{\T\pa{\sum_{i = 1}^n \inn{x, v_i} v_i}, \sum_{j = 1}^n \inn{x, v_j} v_j}    &  & \text{(by \cref{6.5})}         \\
     & = \inn{\sum_{i = 1}^n \inn{x, v_i} \T(v_i), \sum_{j = 1}^n \inn{x, v_j} v_j}       &  & \text{(by \cref{2.1.2}(d))}    \\
     & = \inn{\sum_{i = 1}^n \inn{x, v_i} \lambda_i v_i, \sum_{j = 1}^n \inn{x, v_j} v_j} &  & \text{(by \cref{5.1.2})}       \\
     & = \sum_{i = 1}^n \inn{x, v_i} \lambda_i \inn{v_i, \sum_{j = 1}^n \inn{x, v_j} v_j} &  & \text{(by \cref{6.1.1}(a)(b))} \\
     & = \sum_{i = 1}^n \inn{x, v_i} \lambda_i \inn{v_i, \inn{x, v_i} v_i}                &  & \text{(by \cref{6.1.12})}      \\
     & = \sum_{i = 1}^n \abs{\inn{x, v_i}}^2 \lambda_i \inn{v_i, v_i}                     &  & \text{(by \cref{6.1}(b))}      \\
     & = \sum_{i = 1}^n \abs{\inn{x, v_i}}^2 \lambda_i \norm{v_i}^2                       &  & \text{(by \cref{6.1.9})}       \\
     & = \sum_{i = 1}^n \abs{\inn{x, v_i}}^2 \lambda_i                                    &  & \text{(by \cref{6.1.12})}      \\
     & \begin{dcases}
         > 0    & \text{if } \lambda_i > 0 \text{ for all } i \in \set{1, \dots, n}    \\
         \geq 0 & \text{if } \lambda_i \geq 0 \text{ for all } i \in \set{1, \dots, n}
       \end{dcases}.   &  & (x \neq \zv)
  \end{align*}
  Thus by \cref{6.4.11} \(\T\) is positive definite (semidefinite).
  We conclude that \(\T\) is positive definite (semidefinite) iff all of its eigenvalues are positive (nonnegative).
\end{proof}

\begin{proof}[\pf{ex:6.4.17}(b)]
  Let \(\beta = \set{\seq{v}{1,,n}}\).
  Observe that
  \begin{align*}
     & \inn{\T\pa{\sum_{j = 1}^n a_j v_j}, \sum_{k = 1}^n a_k v_k}                                                            \\
     & = \inn{\sum_{j = 1}^n a_j \T(v_j), \sum_{k = 1}^n a_k v_k}                         &  & \text{(by \cref{2.1.2}(d))}    \\
     & = \inn{\sum_{j = 1}^n a_j \pa{\sum_{i = 1}^n A_{i j} v_i}, \sum_{k = 1}^n a_k v_k} &  & \text{(by \cref{2.2.4})}       \\
     & = \sum_{j = 1}^n \sum_{i = 1}^n a_j A_{i j} \inn{v_i, \sum_{k = 1}^n a_k v_k}      &  & \text{(by \cref{6.1.1}(a)(b))} \\
     & = \sum_{j = 1}^n \sum_{i = 1}^n a_j A_{i j} \inn{v_i, a_i v_i}                     &  & \text{(by \cref{6.1.12})}      \\
     & = \sum_{j = 1}^n \sum_{i = 1}^n a_j A_{i j} \conj{a_i} \inn{v_i, v_i}              &  & \text{(by \cref{6.1}(b))}      \\
     & = \sum_{j = 1}^n \sum_{i = 1}^n a_j A_{i j} \conj{a_i}.                            &  & \text{(by \cref{6.1.12})}
  \end{align*}
  Thus we have
  \begin{align*}
         & \T \text{ is positive definite (semidefinite)}                                                                                                  \\
    \iff & \forall x \in \V \setminus \set{\zv}, 0 < (\leq) \inn{\T(x), x}                                              &  & \text{(by \cref{6.4.11})}     \\
         & = \inn{\T\pa{\sum_{j = 1}^n \inn{x, v_j} v_j}, \sum_{k = 1}^n \inn{x, v_k} v_k}                              &  & \text{(by \cref{6.5})}        \\
         & = \sum_{j = 1}^n \sum_{i = 1}^n \inn{x, v_j} A_{i j} \conj{\inn{x, v_i}}                                     &  & \text{(from the proof above)} \\
    \iff & \forall a \in \vs{F}^n \setminus \set{\zv}, \sum_{i = 1}^n \sum_{j = 1}^n A_{i j} a_j \conj{a_i} > (\geq) 0.
  \end{align*}
\end{proof}

\begin{proof}[\pf{ex:6.4.17}(c)]
  First suppose that \(\T\) is positive semidefinite.
  As in the proof of \cref{ex:6.4.17}(a), \(\T\) is self-adjoint implies there exists an orthonormal basis \(\gamma = \set{\seq{v}{1,,n}}\) for \(\V\) over \(\F\) consist of eigenvectors of \(\T\).
  For each \(i \in \set{1, \dots, n}\), let \(\lambda_i\) be the eigenvalue of \(\T\) corresponding to \(v_i\).
  By \cref{ex:6.4.17}(a) we know that \(\lambda_i \geq 0\) for all \(i \in \set{1, \dots, n}\).
  If we let \(\alpha\) be the standard ordered basis for \(\vs{F}^n\) over \(\F\) and
  \[
    C = \begin{pmatrix}
      \sqrt{\lambda_1} & 0                & \cdots & 0                \\
      0                & \sqrt{\lambda_2} & \cdots & 0                \\
      \vdots           & \vdots           &        & \vdots           \\
      0                & 0                & \cdots & \sqrt{\lambda_n}
    \end{pmatrix},
  \]
  then we have
  \begin{align*}
             & C^* = \tp{C} = C                                                                                                                          &  & \text{(by \cref{6.1.5})}      \\
    \implies & [\T]_{\gamma} = C^* C                                                                                                                     &  & \text{(by \cref{5.1.1})}      \\
    \implies & A = [\T]_{\beta} = \pa{[\IT[\V]]_{\beta}^{\gamma}}^{-1} [\T]_{\gamma} [\IT[\V]]_{\beta}^{\gamma}                                          &  & \text{(by \cref{2.23})}       \\
             & = \pa{[\IT[\V]]_{\beta}^{\gamma}}^{-1} C^* C [\IT[\V]]_{\beta}^{\gamma}                                                                   &  & \text{(from the proof above)} \\
             & = \pa{[\IT[\V]]_{\alpha}^{\gamma} [\IT[\V]]_{\beta}^{\alpha}}^{-1} C^* C [\IT[\V]]_{\alpha}^{\gamma} [\IT[\V]]_{\beta}^{\alpha}           &  & \text{(by \cref{2.11})}       \\
             & = \pa{[\IT[\V]]_{\beta}^{\alpha}}^{-1} \pa{[\IT[\V]]_{\alpha}^{\gamma}}^{-1} C^* C [\IT[\V]]_{\alpha}^{\gamma} [\IT[\V]]_{\beta}^{\alpha} &  & \text{(by \cref{ex:2.4.4})}   \\
             & = \pa{[\IT[\V]]_{\beta}^{\alpha}}^* \pa{[\IT[\V]]_{\alpha}^{\gamma}}^* C^* C [\IT[\V]]_{\alpha}^{\gamma} [\IT[\V]]_{\beta}^{\alpha}       &  & \text{(by \cref{ex:6.1.23})}  \\
             & = (C [\IT[\V]]_{\alpha}^{\gamma} [\IT[\V]]_{\beta}^{\alpha})^* C [\IT[\V]]_{\alpha}^{\gamma} [\IT[\V]]_{\beta}^{\alpha}.                  &  & \text{(by \cref{6.3.2}(c))}
  \end{align*}
  By setting \(B = C [\IT[\V]]_{\alpha}^{\gamma} [\IT[\V]]_{\beta}^{\alpha}\) we see that \(A = B^* B\).

  Now suppose that there exists a \(B \in \ms{n}{n}{\F}\) such that \(A = B^* B\).
  Let \(\beta = \set{\seq{v}{1,,n}}\) and let \([\cdot, \cdot]\) be the standard inner product on \(\vs{F}^n\) over \(\F\) as defined in \cref{6.1.2}.
  For each \(i \in \set{1, \dots, n}\), let \(b_i\) denote the \(i\)th column of \(B\).
  Then we have
  \begin{align*}
     & \forall x \in \V \setminus \set{\zv}, \inn{\T(x), x}                                                                                       \\
     & = \inn{\T\pa{\sum_{j = 1}^n \inn{x, v_j} v_j}, \sum_{k = 1}^n \inn{x, v_k} v_k}                        &  & \text{(by \cref{6.5})}         \\
     & = \inn{\sum_{j = 1}^n \inn{x, v_j} \T(v_j), \sum_{k = 1}^n \inn{x, v_k} v_k}                           &  & \text{(by \cref{2.1.2}(d))}    \\
     & = \inn{\sum_{j = 1}^n \inn{x, v_j} \pa{\sum_{i = 1}^n A_{i j} v_i}, \sum_{k = 1}^n \inn{x, v_k} v_k}   &  & \text{(by \cref{2.2.4})}       \\
     & = \sum_{i = 1}^n \sum_{j = 1}^n \inn{x, v_j} A_{i j} \inn{v_i, \sum_{k = 1}^n \inn{x, v_k} v_k}        &  & \text{(by \cref{6.1.1}(a)(b))} \\
     & = \sum_{i = 1}^n \sum_{j = 1}^n \inn{x, v_j} A_{i j} \inn{v_i, \inn{x, v_i} v_i}                       &  & \text{(by \cref{6.1.12})}      \\
     & = \sum_{i = 1}^n \sum_{j = 1}^n \inn{x, v_j} A_{i j} \conj{\inn{x, v_i}} \inn{v_i, v_i}                &  & \text{(by \cref{6.1}(b))}      \\
     & = \sum_{i = 1}^n \sum_{j = 1}^n \inn{x, v_j} A_{i j} \conj{\inn{x, v_i}}                               &  & \text{(by \cref{6.1.12})}      \\
     & = \sum_{i = 1}^n \sum_{j = 1}^n \inn{x, v_j} (B^* B)_{i j} \conj{\inn{x, v_i}}                         &  & \text{(by hypothesis)}         \\
     & = \sum_{i = 1}^n \sum_{j = 1}^n \sum_{k = 1}^n \inn{x, v_j} (B^*)_{i k} B_{k j} \conj{\inn{x, v_i}}    &  & \text{(by \cref{2.3.1})}       \\
     & = \sum_{i = 1}^n \sum_{j = 1}^n \sum_{k = 1}^n \inn{x, v_j} \conj{B_{k i}} B_{k j} \conj{\inn{x, v_i}} &  & \text{(by \cref{6.1.5})}       \\
     & = \sum_{i = 1}^n \sum_{j = 1}^n \sum_{k = 1}^n \inn{x, v_j} B_{k j} \conj{\inn{x, v_i} B_{k i}}        &  & \text{(by \cref{d.2}(c))}      \\
     & = \sum_{i = 1}^n \sum_{j = 1}^n [\inn{x, v_j} b_j, \inn{x, v_i} b_i]                                   &  & \text{(by \cref{6.1.2})}       \\
     & = \sum_{i = 1}^n \br{\sum_{j = 1}^n \inn{x, v_j} b_j, \inn{x, v_i} b_i}                                &  & \text{(by \cref{6.1.1}(a))}    \\
     & = \br{\sum_{j = 1}^n \inn{x, v_j} b_j, \sum_{i = 1}^n \inn{x, v_i} b_i}                                &  & \text{(by \cref{6.1}(a))}      \\
     & = \norm{\sum_{i = 1}^n \inn{x, v_i} b_i}^2                                                             &  & \text{(by \cref{6.1.9})}       \\
     & \geq 0.                                                                                                &  & \text{(by \cref{6.2}(b))}
  \end{align*}
  Thus by \cref{6.4.11} \(\T\) is positive semidefinite.
\end{proof}

\begin{proof}[\pf{ex:6.4.17}(d)]
  As in the proof of \cref{ex:6.4.17}(a), \(\T\) is self-adjoint implies there exists an orthonormal basis \(\gamma\) for \(\V\) over \(\F\) consist of eigenvectors of \(\T\).
  Let \(v \in \gamma\) and let \(\lambda \in \F\) be an eigenvalue of \(\T\) corresponding to \(v\).
  By \cref{ex:6.4.17}(a) we know that \(\lambda \geq 0\).
  Now we split into two cases:
  \begin{itemize}
    \item If \(\lambda = 0\), then we have
          \begin{align*}
                     & \T(v) = 0v = \zv                  &  & \text{(by \cref{5.1.2})}    \\
            \implies & \U^2(v) = \T^2(v) = \T(\zv) = \zv &  & \text{(by \cref{2.1.2}(a))} \\
            \implies & 0 = \inn{\U^2(v), v}              &  & \text{(by \cref{6.1}(c))}   \\
                     & = \inn{(\U^* \U)(v), v}           &  & \text{(by \cref{6.4.8})}    \\
                     & = \inn{\U(v), \U(v)}              &  & \text{(by \cref{6.9})}      \\
            \implies & \U(v) = \zv = \T(v).              &  & \text{(by \cref{6.1}(d))}
          \end{align*}
    \item If \(\lambda > 0\), then we have
          \begin{align*}
                     & \U^2(v) = \T^2(v) = \lambda^2 v                                                                           &  & \text{(by \cref{5.1.2})} \\
            \implies & \zv = \U^2(v) - \lambda^2 v = ((\U + \lambda \IT[\V]) (\U - \lambda \IT[\V]))(v)                          &  & \text{(by \cref{2.10})}  \\
            \implies & \begin{dcases}
                         \ns{\U - (-\lambda) \IT[\V]} = \ns{\U + \lambda \IT[\V]} = \set{\zv} \\
                         (\U - \lambda \IT[\V])(v) \in \ns{\U + \lambda \IT[\V]}
                       \end{dcases} &  & \text{(by \cref{ex:6.4.17}(a))}                                  \\
            \implies & (\U - \lambda \IT[\V])(v) = \zv                                                                                                         \\
            \implies & \U(v) = \lambda v = \T(v).
          \end{align*}
  \end{itemize}
  From all cases above we see that \(\U(v) = \T(v)\).
  Since \(v\) is arbitrary, we see that \(\T, \U\) argee on \(\gamma\).
  Thus by \cref{2.1.13} we have \(\T = \U\).
\end{proof}

\begin{proof}[\pf{ex:6.4.17}(e)]
  Since \(\T, \U\) are self-adjoint and \(\T \U = \U \T\), by \cref{ex:6.4.14} there exists an orthonormal basis \(\gamma\) for \(\V\) over \(\F\) consist of eigenvectors of \(\T\) and \(\U\).
  Let \(v \in \gamma\) and let \(\lambda_{\T}, \lambda_{\U} \in \F\) be eigenvalues of \(\T\) and \(\U\) corresponding to \(v\), respectively.
  Since
  \begin{align*}
    \U(\T(v)) & = \U(\lambda_{\T} v)           &  & \text{(by \cref{5.1.2})}    \\
              & = \lambda_{\T} \U(v)           &  & \text{(by \cref{2.1.1}(b))} \\
              & = \lambda_{\T} \lambda_{\U} v, &  & \text{(by \cref{5.1.2})}
  \end{align*}
  we know that \(v\) is an eigenvector of \(\U \T\) corresponding to eigenvalue \(\lambda_{\T} \lambda_{\U}\).
  By \cref{ex:6.4.17}(a) we know that \(\lambda_{\T} > 0\) and \(\lambda_{\U} > 0\), thus \(\lambda_{\T} \lambda_{\U} > 0\).
  Since \(v\) is arbitrary, we know that \(\gamma\) is consist of eigenvectors of \(\U \T\) and all eigenvalues of \(\U \T\) are positive.
  Since
  \begin{align*}
    (\U \T)^* & = \T^* \U^* &  & \text{(by \cref{6.11}(c))} \\
              & = \T \U     &  & \text{(by \cref{6.4.8})}   \\
              & = \U \T,
  \end{align*}
  by \cref{6.4.8} we know that \(\U \T\) is self-adjoint.
  Thus by \cref{ex:6.4.17}(a) \(\U \T = \T \U\) is positive definite.
\end{proof}

\begin{proof}[\pf{ex:6.4.17}(f)]
  Let \([\cdot, \cdot]\) be the standard inner product on \(\vs{F}^n\) over \(\F\).
  Then we have
  \begin{align*}
         & \T \text{ is positive definite (semidefinite)}                                                                               \\
    \iff & \forall x \in \V, \inn{\T(x), x} > (\geq) 0                                             &  & \text{(by \cref{6.4.11})}       \\
    \iff & \forall x \in \vs{F}^n, \sum_{i = 1}^n \sum_{j = 1}^n A_{i j} x_j \conj{x_i} > (\geq) 0 &  & \text{(by \cref{ex:6.4.17}(b))} \\
    \iff & \forall x \in \vs{F}^n, [Ax, x] > (\geq) 0                                              &  & \text{(by \cref{6.1.2})}        \\
    \iff & \forall x \in \vs{F}^n, [\L_A(x), x] > (\geq) 0                                         &  & \text{(by \cref{2.3.8})}        \\
    \iff & A \text{ is positive definite (semidefinite)}.                                          &  & \text{(by \cref{6.4.11})}
  \end{align*}
\end{proof}

\begin{ex}\label{ex:6.4.18}
  Let \(\T \in \ls(\V, \W)\) where \(\V\) and \(\W\) are finite-dimensional inner product spaces.
  Prove the following results.
  \begin{enumerate}
    \item \(\T^* \T\) and \(\T \T^*\) are positive semidefinite.
    \item \(\rk{\T^* \T} = \rk{\T \T^*} = \rk{\T}\).
  \end{enumerate}
\end{ex}

\begin{proof}[\pf{ex:6.4.18}(a)]
  Let \(\inn{\cdot, \cdot}_1, \inn{\cdot, \cdot}_2\) be inner products on \(\V\) and \(\W\) over \(\F\), respectively.
  Since
  \begin{align*}
    (\T^* \T)^* & = \T^* \T^{**} &  & \text{(by \cref{ex:6.3.16}(c))} \\
                & = \T^* \T      &  & \text{(by \cref{ex:6.3.16}(d))} \\
    (\T \T^*)^* & = \T^{**} \T^* &  & \text{(by \cref{ex:6.3.16}(c))} \\
                & = \T \T^*      &  & \text{(by \cref{ex:6.3.16}(d))}
  \end{align*}
  and
  \begin{align*}
    \forall x \in \V \setminus \set{\zv}, \inn{(\T^* \T)(x), x}_1 & = \inn{\T^*(\T(x)), x}_1                                        \\
                                                                  & = \inn{\T(x), \T(x)}_2     &  & \text{(by \cref{ex:6.3.15}(a))} \\
                                                                  & \geq 0                     &  & \text{(by \cref{6.1}(d))}       \\
    \forall y \in \W \setminus \set{\zv}, \inn{(\T \T^*)(y), y}_2 & = \inn{\T(\T^*(y)), y}_2                                        \\
                                                                  & = \inn{\T^*(y), \T^*(y)}_1 &  & \text{(by \cref{ex:6.3.15}(a))} \\
                                                                  & \geq 0,                    &  & \text{(by \cref{6.1}(d))}
  \end{align*}
  by \cref{6.4.11} we know that \(\T^* \T\) and \(\T \T^*\) are positive semidefinite.
\end{proof}

\begin{proof}[\pf{ex:6.4.18}(b)]
  By \cref{ex:6.3.15}(e) we have \(\ns{\T^* \T} = \ns{\T}\).
  Since
  \begin{align*}
             & \dim(\V) = \begin{dcases}
                            \rk{\T} + \nt{\T} \\
                            \rk{\T^* \T} + \nt{\T^* \T}
                          \end{dcases} &  & \text{(by \cref{2.3})}            \\
    \implies & \rk{\T} = \rk{\T^* \T},     &  & \text{(from the proof above)}
  \end{align*}
  we have
  \begin{align*}
    \rk{\T \T^*} & = \rk{\T^{**} \T^*} &  & \text{(by \cref{ex:6.3.16}(d))} \\
                 & = \rk{\T^*}         &  & \text{(from the proof above)}   \\
                 & = \rk{\T}.          &  & \text{(by \cref{ex:6.3.15}(c))}
  \end{align*}
\end{proof}

\begin{ex}\label{ex:6.4.19}
  Let \(\T\) and \(\U\) be positive definite operators on an inner product space \(\V\) over \(\F\).
  Prove the following results.
  \begin{enumerate}
    \item \(\T + \U\) is positive definite.
    \item If \(c > 0\), then \(c \T\) is positive definite.
    \item \(\T^{-1}\) is positive definite.
  \end{enumerate}
\end{ex}

\begin{proof}[\pf{ex:6.4.19}(a)]
  We have
  \begin{align*}
    (\T + \U)^* & = \T^* + \U^* &  & \text{(by \cref{6.11}(a))} \\
                & = \T + \U     &  & \text{(by \cref{6.4.11})}
  \end{align*}
  and
  \begin{align*}
    \forall x \in \V \setminus \set{\zv}, \inn{(\T + \U)(x), x} & = \inn{\T(x) + \U(x), x}                                           \\
                                                                & = \inn{\T(x), x} + \inn{\U(x), x} &  & \text{(by \cref{6.1.1}(a))} \\
                                                                & > 0.                              &  & \text{(by \cref{6.4.11})}
  \end{align*}
  Thus by \cref{6.4.11} \(\T + \U\) is positive definite.
\end{proof}

\begin{proof}[\pf{ex:6.4.19}(b)]
  We have
  \begin{align*}
    (c\T)^* & = \conj{c} \T^* &  & \text{(by \cref{6.11}(b))} \\
            & = \conj{c} \T   &  & \text{(by \cref{6.4.11})}  \\
            & = c \T          &  & (c > 0)
  \end{align*}
  and
  \begin{align*}
    \forall x \in \V \setminus \set{\zv}, \inn{(c \T)(x), x} & = \inn{c \T(x), x}                                  \\
                                                             & = c \inn{\T(x), x} &  & \text{(by \cref{6.1.1}(b))} \\
                                                             & > 0.               &  & \text{(by \cref{6.4.11})}
  \end{align*}
  Thus by \cref{6.4.11} \(c \T\) is positive definite.
\end{proof}

\begin{proof}[\pf{ex:6.4.19}(c)]
  We have
  \begin{align*}
    (\T^{-1})^* & = (\T^*)^{-1} &  & \text{(by \cref{ex:6.3.8})} \\
                & = \T^{-1}     &  & \text{(by \cref{6.4.11})}
  \end{align*}
  and
  \begin{align*}
             & \lambda \in \F \text{ is an eigenvalue of } \T^{-1}                                      \\
    \implies & \lambda^{-1} \in \F \text{ is an eigenvalue of } \T &  & \text{(by \cref{ex:5.1.8}(b))}  \\
    \implies & \lambda^{-1} > 0                                    &  & \text{(by \cref{ex:6.4.17}(a))} \\
    \implies & \lambda > 0.
  \end{align*}
  Thus by \cref{ex:6.4.17}(a) \(\T^{-1}\) is positive definite.
\end{proof}

\begin{ex}\label{ex:6.4.20}
  Let \(\V\) be an inner product space over \(\F\) with inner product \(\inn{\cdot, \cdot}\), and let \(\T\) be a positive definite linear operator on \(\V\).
  Prove that \(\inn{x, y}' = \inn{\T(x), y}\) defines another inner product on \(\V\) over \(\F\).
\end{ex}

\begin{proof}[\pf{ex:6.4.20}]
  Let \(x, y, z \in \V\) and let \(c \in \F\).
  Then we have
  \begin{align*}
    \inn{x + y, z}'    & = \inn{\T(x + y), z}              &  & \text{(by \cref{ex:6.4.20})} \\
                       & = \inn{\T(x) + \T(y), z}          &  & \text{(by \cref{2.1.1}(a))}  \\
                       & = \inn{\T(x), z} + \inn{\T(y), z} &  & \text{(by \cref{6.1.1}(a))}  \\
                       & = \inn{x, z}' + \inn{y, z}'       &  & \text{(by \cref{ex:6.4.20})} \\
    \inn{cx, y}'       & = \inn{\T(cx), y}                 &  & \text{(by \cref{ex:6.4.20})} \\
                       & = \inn{c \T(x), y}                &  & \text{(by \cref{2.1.1}(b))}  \\
                       & = c \inn{\T(x), y}                &  & \text{(by \cref{6.1.1}(b))}  \\
                       & = c \inn{x, y}'                   &  & \text{(by \cref{ex:6.4.20})} \\
    \conj{\inn{x, y}'} & = \conj{\inn{\T(x), y}}           &  & \text{(by \cref{ex:6.4.20})} \\
                       & = \inn{y, \T(x)}                  &  & \text{(by \cref{6.1.1}(c))}  \\
                       & = \inn{\T^*(y), x}                &  & \text{(by \cref{6.9})}       \\
                       & = \inn{\T(y), x}                  &  & \text{(by \cref{6.4.11})}    \\
                       & = \inn{y, x}'.                    &  & \text{(by \cref{ex:6.4.20})}
  \end{align*}
  If \(x \neq \zv\), then we have
  \begin{align*}
    \inn{x, x}' & = \inn{\T(x), x} &  & \text{(by \cref{ex:6.4.20})} \\
                & > 0.             &  & \text{(by \cref{6.4.11})}
  \end{align*}
  Thus by \cref{6.1.1} \(\inn{\cdot, \cdot}'\) is an inner product on \(\V\) over \(\F\).
\end{proof}

\begin{ex}\label{ex:6.4.21}
  Let \(\V\) be a finite-dimensional inner product space over \(\F\), and let \(\T\) and \(\U\) be self-adjoint operators on \(\V\) such that \(\T\) is positive definite.
  Prove that both \(\T \U\) and \(\U \T\) are diagonalizable linear operators that have only real eigenvalues.
\end{ex}

\begin{proof}[\pf{ex:6.4.21}]
  Let \(\inn{\cdot, \cdot}\) be an inner product on \(\V\) over \(\F\).
  For each \(\lt{S} \in \ls(\V)\), we denote the adjoint of \(\lt{S}\) with respect to \(\inn{\cdot, \cdot}\) as \(\vs{S}^*\).
  Suppose that \(\T, \U\) are self-adjoint with respect to \(\inn{\cdot, \cdot}\) and \(\T\) is positive definite with respect to \(\inn{\cdot, \cdot}\).

  Define inner product \(\inn{\cdot, \cdot}'\) on \(\V\) over \(\F\) by setting \(\inn{x, y}' = \inn{\T(x), y}\) for all \(x, y \in \V\).
  By \cref{ex:6.4.20} we see that \(\inn{\cdot, \cdot}'\) is indeed an inner product on \(\V\) over \(\F\).
  For each \(\lt{S} \in \ls(\V)\), we denote the adjoint of \(\lt{S}\) with respect to \(\inn{\cdot, \cdot}'\) as \(\vs{S}^{\dag}\).
  These adjoint operators are well-defined thanks to \cref{6.9}.
  Now we show that \(\U \T\) is self-adjoint with respect to \(\inn{\cdot, \cdot}'\).
  Since
  \begin{align*}
    \forall x, y \in \V, \inn{x, (\U \T)^{\dag}(y)}' & = \inn{(\U \T)(x), y}'       &  & \text{(by \cref{6.9})}   \\
                                                     & = \inn{\T(\U(\T(x))), y}                                   \\
                                                     & = \inn{\T(x), \U^*(\T^*(y))} &  & \text{(by \cref{6.9})}   \\
                                                     & = \inn{\T(x), \U(\T(y))}     &  & \text{(by \cref{6.4.8})} \\
                                                     & = \inn{x, (\U \T)(y)}',
  \end{align*}
  by \cref{6.1}(e) we have \((\U \T)^{\dag} = \U \T\).
  Thus by \cref{6.4.8} \(\U \T\) is self-adjoint with respect to \(\inn{\cdot, \cdot}'\), and by \cref{6.16,6.17} we know that \(\U \T\) is diagonalizable.
  By \cref{6.4.10}(a) this means \(\U \T\) has only real eigenvalues.

  Since \(\T\) is self-adjoint with respect to \(\inn{\cdot, \cdot}\), by \cref{6.16,6.17} we know that there exists an orthonormal basis for \(\V\) over \(\F\) consist of eigenvectors of \(\T\).
  Since \(\T\) is positive definite with respect to \(\inn{\cdot, \cdot}\), by \cref{ex:6.4.17}(a) we know that every eigenvalue of \(\T\) is positive.
  Thus \(\T\) is invertible, and we can define inner product \(\inn{x, y}'' = \inn{\T^{-1}(x), y}\) for all \(x, y\).
  By \cref{ex:6.4.19}(c) we know that \(\T^{-1}\) is positive definite with respect to \(\inn{\cdot, \cdot}\), thus by \cref{ex:6.4.20} \(\inn{\cdot, \cdot}''\) is indeed an inner product on \(\V\) over \(\F\).
  For each \(\lt{S} \in \ls(\V)\), we denote the adjoint of \(\lt{S}\) with respect to \(\inn{\cdot, \cdot}''\) as \(\vs{S}^{\ddag}\).
  These adjoint operators are well-defined thanks to \cref{6.9}.
  Now we show that \(\T \U\) is self-adjoint with respect to \(\inn{\cdot, \cdot}''\).
  Since
  \begin{align*}
    \forall x, y \in \V, \inn{x, (\T \U)^{\ddag}(y)}'' & = \inn{(\T \U)(x), y}''           &  & \text{(by \cref{6.9})}      \\
                                                       & = \inn{\T^{-1}(\T(\U(x))), y}                                      \\
                                                       & = \inn{\U(x), y}                                                   \\
                                                       & = \inn{x, \U^*(y)}                &  & \text{(by \cref{6.9})}      \\
                                                       & = \inn{x, \U(y)}                  &  & \text{(by \cref{6.4.8})}    \\
                                                       & = \inn{x, \T^{-1}(\T(\U(y)))}                                      \\
                                                       & = \inn{(\T^{-1})^*(x), \T(\U(y))} &  & \text{(by \cref{6.9})}      \\
                                                       & = \inn{(\T^*)^{-1}(x), \T(\U(y))} &  & \text{(by \cref{ex:6.3.8})} \\
                                                       & = \inn{\T^{-1}(x), \T(\U(y))}     &  & \text{(by \cref{6.4.8})}    \\
                                                       & = \inn{x, (\T \U)(y)}'',
  \end{align*}
  by \cref{6.1}(e) we have \((\T \U)^{\ddag} = \T \U\).
  Thus by \cref{6.4.8} \(\T \U\) is self-adjoint with respect to \(\inn{\cdot, \cdot}''\), and by \cref{6.16,6.17} we know that \(\T \U\) is diagonalizable.
  By \cref{6.4.10}(a) this means \(\T \U\) has only real eigenvalues.
\end{proof}

\section{Unitary and Orthogonal Operators and Their Matrices}\label{sec:6.5}

\begin{note}
	In \cref{sec:6.5}, we study those linear operators \(\T\) on an inner product space \(\V\) over \(\F\) such that \(\T \T^* = \T^* \T = \IT[\V]\).
	We will see that these are precisely the linear operators that ``preserve length'' in the sense that \(\norm{\T(x)} = \norm{x}\) for all \(x \in \V\).
	As another characterization, we prove that, on a finite-dimensional complex inner product space, these are the normal operators whose eigenvalues all have absolute value \(1\).

	In past chapters, we were interested in studying those functions that preserve the structure of the underlying space.
	In particular, linear operators preserve the operations of vector addition and scalar multiplication, and isomorphisms preserve all the vector space structure.
	It is now natural to consider those linear operators \(\T\) on an inner product space that preserve length.
	We will see that this condition guarantees, in fact, that \(\T\) preserves the inner product.
\end{note}

\begin{defn}\label{6.5.1}
	Let \(\T\) be a linear operator on a finite-dimensional inner product space \(\V\) over \(\F\).
	If \(\norm{\T(x)} = \norm{x}\) for all \(x \in \V\), we call \(\T\) a \textbf{unitary operator} if \(\F = \C\) and an \textbf{orthogonal operator} if \(\F = \R\).

	in the infinite-dimensional case, an operator satisfying the preceding norm requirement is generally called an \textbf{isometry}.
	If, in addition, the operator is onto (the condition guarantees one-to-one, see \cref{ex:6.1.17}), then the operator is called a \textbf{unitary} or \textbf{orthogonal operator}.
\end{defn}

\begin{note}
	Clearly, any rotation or reflection in \(\R^2\) preserves length and hence is an orthogonal operator.
	We study these operators in much more detail in \cref{sec:6.11}.
\end{note}

\begin{eg}\label{6.5.2}
	Let \(h \in \vs{H}\) (see \cref{6.1.8}) satisfy \(\abs{h(x)} = 1\) for all \(x \in [0, 2 \pi]\).
	Define \(\T \in \ls(\vs{H})\) by \(\T(f) = hf\).
	Then
	\begin{align*}
		\norm{\T(f)}^2 & = \norm{hf}^2                                                                     \\
		               & = \inn{hf, hf}                                                    &  & \by{6.1.9} \\
		               & = \frac{1}{2 \pi} \int_0^{2 \pi} h(t) f(t) \conj{h(t) f(t)} \; dt &  & \by{6.1.8} \\
		               & = \inn{f, f}                                                      &  & \by{6.1.8} \\
		               & = \norm{f}^2                                                      &  & \by{6.1.9}
	\end{align*}
	since \(\abs{h(t)}^2 = 1\) for all \(t \in [0, 2 \pi]\).
	So \(\T\) is a unitary operator.
\end{eg}

\section{Orthogonal Projections and the Spectral Theorem}\label{sec:6.6}

\begin{note}
  In this section, we rely heavily on \cref{6.16,6.17} to develop an elegant representation of a normal (if \(\F = \C\)) or a self-adjoint (if \(\F = \R\)) operator \(\T\) on a finite-dimensional inner product space.
  We prove that \(\T\) can be written in the form \(\seq[+]{\lambda,\T}{1,,k}\), where \(\seq{\lambda}{1,,k} \in \F\) are the distinct eigenvalues of \(\T\) and \(\seq{\T}{1,,k}\) are \emph{orthogonal projections}.

  Recall from \cref{2.1.14} that if \(\V = \W_1 \oplus \W_2\), then a linear operator \(\T\) on \(\V\) is the \textbf{projection on \(\W_1\) along \(\W_2\)} if, whenever \(x = x_1 + x_2\), with \(x_1 \in \W_1\) and \(x_2 \in \W_2\), we have \(\T(x) = x_1\).
  By \cref{ex:2.1.26}(a)(b), we have
  \[
    \rg{\T} = \W_1 = \set{x \in \V : \T(x) = x} \quad \text{and} \quad \ns{\T} = \W_2.
  \]
  So \(\V = \rg{\T} \oplus \ns{\T}\).
  Thus there is no ambiguity if we refer to \(\T\) as a ``projection on \(\W_1\)'' or simply as a ``projection.''
  In fact, it can be shown (see \cref{ex:2.3.17}) that \(\T\) is a projection iff \(\T = \T^2\).
  Because \(\V = \W_1 \oplus \W_2 = \W_1 \oplus \W_3\) does \emph{not} imply that \(\W_2 = \W_3\), we see that \(\W_1\) does not uniquely determine \(\T\).
  For an \emph{orthogonal} projection \(\T\), however, \(\T\) is uniquely determined by its range (see \cref{6.6.2}).
\end{note}

\begin{defn}\label{6.6.1}
  Let \(\V\) be an inner product space over \(\F\), and let \(\T \in \ls(\V)\) be a projection.
  We say that \(\T\) is an \textbf{orthogonal projection} if \(\rg{\T}^{\perp} = \ns{\T}\) and \(\ns{\T}^{\perp} = \rg{\T}\).
\end{defn}

\begin{note}
  By \cref{ex:6.2.13}(c), if \(\V\) is finite-dimensional, we need only assume that one of the preceding conditions holds.
  For example, if \(\rg{\T}^{\perp} = \ns{\T}\), then \(\rg{\T} = \rg{\T}^{\perp \perp} = \ns{\T}^{\perp}\).
\end{note}

\begin{prop}\label{6.6.2}
  Assume that \(\W\) is a finite-dimensional subspace of an inner product space \(\V\) over \(\F\).
  In the notation of \cref{6.6}, we can define a function \(\T : \V \to \V\) by \(\T(y) = u\).
  Then \(\T\) is the unique orthogonal projection on \(\W\).
  We call \(\T\) the \textbf{orthogonal projection} of \(\V\) on \(\W\).
\end{prop}

\begin{proof}[\pf{6.6.2}]
  For convenience, for each \(v \in \V\), we define the unique tuple \((v_1, v_2) \in \W \times \W^{\perp}\) such that \(v = v_1 + v_2\) (such definition is well-defined thanks to \cref{6.6}).

  First we show that \(\T\) is an orthogonal projection of \(\V\) on \(\W\).
  By \cref{2.1.14}, \cref{6.6} and \cref{ex:6.2.13}(d) we see that \(\T\) is a projection on \(\W\) along \(\W^{\perp}\).
  We claim that \(\inn{\T(x), y} = \inn{x, \T(y)}\) for all \(x, y \in \V\).
  This is true since
  \begin{align*}
    \forall x, y \in \V, \inn{x, \T(y)} & = \inn{x, y_1}                    &  & \by{2.1.14}   \\
                                        & = \inn{x_1 + x_2, y_1}            &  & \by{6.6}      \\
                                        & = \inn{x_1, y_1} + \inn{x_2, y_1} &  & \by{6.1.1}[a] \\
                                        & = \inn{x_1, y_1}                  &  & \by{6.2.9}    \\
                                        & = \inn{x_1, y_1} + \inn{x_1, y_2} &  & \by{6.2.9}    \\
                                        & = \inn{x_1, y_1 + y_2}            &  & \by{6.1}[a]   \\
                                        & = \inn{x_1, y}                    &  & \by{6.6}      \\
                                        & = \inn{\T(x), y}.                 &  & \by{2.1.14}
  \end{align*}
  Now we use the claim to show that \(\T\) is an orthogonal projection of \(\V\) on \(\W\).
  By \cref{ex:2.1.26}(b) we have \(\W = \rg{\T}\) and \(\W^{\perp} = \ns{\T}\).
  Since
  \begin{align*}
         & v \in \rg{\T}^{\perp}                                                    \\
    \iff & \forall y \in \rg{\T}, \inn{v, y} = 0 &  & \by{6.2.9}                    \\
    \iff & \forall x \in \V, \inn{v, \T(x)} = 0  &  & \by{2.1.10}                   \\
    \iff & \forall x \in \V, \inn{\T(v), x} = 0  &  & \text{(from the claim above)} \\
    \iff & \inn{\T(v), \T(v)} = 0                &  & (\T(v) \in \V)                \\
    \iff & \T(v) = \zv                           &  & \by{6.1}[d]                   \\
    \iff & v \in \ns{\T},                        &  & \by{2.1.10}
  \end{align*}
  we have \(\rg{\T}^{\perp} = \ns{\T}\).
  Since
  \begin{align*}
             & v \in \ns{\T}^{\perp}                                                                       \\
    \implies & \forall x \in \ns{\T}, \inn{v, x} = 0                    &  & \by{6.2.9}                    \\
    \implies & \forall x \in \ns{\T}, \inn{v, \T(x)} = \inn{v, \zv} = 0 &  & \by{2.1.10}                   \\
    \implies & \forall x \in \ns{\T}, \inn{\T(v), x} = 0                &  & \text{(from the claim above)} \\
    \implies & \forall x \in \ns{\T}, \inn{v - \T(v), x} = 0            &  & \by{6.1}[a,b]                 \\
    \implies & \inn{v - \T(v), v - \T(v)} = 0                           &  & (v - \T(v) \in \ns{\T})       \\
    \implies & v - \T(v) = \zv                                          &  & \by{6.1}[d]                   \\
    \implies & \T(v) = v                                                                                   \\
    \implies & v \in \rg{\T}                                            &  & \by{ex:2.1.26}[b]
  \end{align*}
  and
  \begin{align*}
             & v \in \rg{\T}                                                                         \\
    \implies & \forall x \in \ns{\T}, \inn{v, x} = \inn{\T(v), x} &  & \by{ex:2.1.26}[b]             \\
             & = \inn{v, \T(x)}                                   &  & \text{(from the claim above)} \\
             & = \inn{v, \zv}                                     &  & \by{2.1.10}                   \\
             & = 0                                                &  & \by{6.1}[c]                   \\
    \implies & v \in \ns{\T}^{\perp},                             &  & \by{6.2.9}
  \end{align*}
  we have \(\ns{\T}^{\perp} \subseteq \rg{\T}\) and \(\rg{\T} \subseteq \ns{\T}^{\perp}\).
  Thus \(\ns{\T}^{\perp} = \rg{\T}\).
  By \cref{6.6.1} \(\T\) is an orthogonal projection of \(\V\) on \(\W\).

  Now we show that \(\T\) is unique.
  For if \(\T\) and \(\U\) are orthogonal projections on \(\W\), then \(\rg{\T} = \W = \rg{\U}\).
  Hence \(\ns{\T} = \rg{\T}^{\perp} = \rg{\U}^{\perp} = \ns{\U}\), and since every projection is uniquely determined by its range and null space, we have \(\T = \U\).
\end{proof}

\begin{note}
  If \(\T\) is the orthogonal projection of \(\V\) on \(\W\), then \(\T(v)\) is the ``best approximation in \(\W\) to \(v\)'';
  that is, if \(w \in \W\), then \(\norm{w - v} \geq \norm{\T(v) - v}\).
  In fact, this approximation property characterizes \(\T\).
  These results follow immediately from \cref{6.2.12}.
\end{note}

\begin{defn}\label{6.6.3}
  As an application to Fourier analysis, recall the inner product space \(\vs{H}\) and the orthonormal set \(S\) in \cref{6.1.13}.
  Define a \textbf{trigonometric polynomial of degree \(n\)} to be a function \(g \in \vs{H}\) of the form
  \[
    g(t) = \sum_{j = -n}^n a_j f_j(t) = \sum_{j = -n}^n a_j e^{i j t},
  \]
  where \(a_n\) or \(a_{-n}\) is nonzero.
\end{defn}

\begin{prop}\label{6.6.4}
  Let \(f \in \vs{H}\).
  The best approximation to \(f\) by a trigonometric polynomial of degree less than or equal to \(n\) is the trigonometric polynomial whose coefficients are the Fourier coefficients of \(f\) relative to the orthonormal set \(S\) (see \cref{6.1.13}).
\end{prop}

\begin{proof}[\pf{6.6.4}]
  Let \(\W = \spn{\set{f_j : \abs{j} \leq n}}\), and let \(\T\) be the orthogonal projection of \(\vs{H}\) on \(\W\).
  The \cref{6.2.12} tells us that the best approximation to \(f\) by a function in \(\W\) is
  \[
    \T(f) = \sum_{j = -n}^n \inn{f, f_i} f_j.
  \]
\end{proof}

\begin{thm}\label{6.24}
  Let \(\V\) be an inner product space over \(\F\), and let \(\T\) be a linear operator on \(\V\).
  Then \(\T\) is an orthogonal projection iff \(\T\) has an adjoint \(\T^*\) and \(\T^2 = \T = \T^*\).
\end{thm}

\begin{proof}[\pf{6.24}]
  Suppose that \(\T\) is an orthogonal projection.
  Since \(\T^2 = \T\) because \(\T\) is a projection (see \cref{ex:2.3.17}), we only need to show that \(\T^*\) exists and \(\T = \T^*\).
  Now \(\V = \rg{\T} \oplus \ns{\T}\) and \(\rg{\T}^{\perp} = \ns{\T}\) (see \cref{6.6.2}).
  Let \(x, y \in \V\).
  Then we can write \(x = x_1 + x_2\) and \(y = y_1 + y_2\), where \(x_1, y_1 \in \rg{\T}\) and \(x_2, y_2 \in \ns{\T}\).
  Hence
  \begin{align*}
    \inn{x, \T(y)} & = \inn{x, y_1}                    &  & \by{2.1.14}   \\
                   & = \inn{x_1 + x_2, y_1}            &  & \by{6.6}      \\
                   & = \inn{x_1, y_1} + \inn{x_2, y_1} &  & \by{6.1.1}[a] \\
                   & = \inn{x_1, y_1}                  &  & \by{6.2.9}
  \end{align*}
  and
  \begin{align*}
    \inn{\T(x), y} & = \inn{x_1, y}                    &  & \by{2.1.14} \\
                   & = \inn{x_1, y_1 + y_2}            &  & \by{6.6}    \\
                   & = \inn{x_1, y_1} + \inn{x_1, y_2} &  & \by{6.1}[a] \\
                   & = \inn{x_1, y_1}.                 &  & \by{6.2.9}
  \end{align*}
  So \(\inn{x, \T(y)} = \inn{\T(x), y}\) for all \(x, y \in \V\);
  thus \(\T^*\) exists and \(\T = \T^*\).

  Now suppose that \(\T^2 = \T = \T^*\).
  Then \(\T\) is a projection by \cref{ex:2.3.17}, and hence we must show that \(\rg{\T} = \ns{\T}^{\perp}\) and \(\rg{\T}^{\perp} = \ns{\T}\).
  Let \(x \in \rg{\T}\) and \(y \in \ns{\T}\).
  Then \(x = \T(x) = \T^*(x)\), and so
  \begin{align*}
    \inn{x, y} & = \inn{\T(x), y}   &  & \by{2.1.14} \\
               & = \inn{\T^*(x), y}                  \\
               & = \inn{x, \T(y)}   &  & \by{6.9}    \\
               & = \inn{x, \zv}     &  & \by{2.1.14} \\
               & = 0.
  \end{align*}
  Therefore \(x \in \ns{\T}^{\perp}\), from which it follows that \(\rg{\T} \subseteq \ns{\T}^{\perp}\).

  Let \(y \in \ns{\T}^{\perp}\).
  We must show that \(y \in \rg{\T}\), that is, \(\T(y) = y\).
  Now
  \begin{align*}
    \norm{y - \T(y)}^2 & = \inn{y - \T(y), y - \T(y)}                   &  & \by{6.1.9}      \\
                       & = \inn{y, y - \T(y)} - \inn{\T(y), y - \T(y)}. &  & \by{6.1.1}[a,b]
  \end{align*}
  Since \(y - \T(y) \in \ns{\T}\) (see \cref{ex:2.3.17}), the first term must equal zero.
  But also
  \begin{align*}
    \inn{\T(y), y - \T(y)} & = \inn{y, \T^*(y - \T(y))} &  & \by{6.9}      \\
                           & = \inn{y, \T(y - \T(y))}   &  & (\T = \T^*)   \\
                           & = \inn{y, \T(y) - \T^2(y)} &  & \by{2.1.2}[c] \\
                           & = \inn{y, \zv}             &  & (\T^2 = \T)   \\
                           & = 0.                       &  & \by{6.1}[c]
  \end{align*}
  Thus by \cref{6.2}(b) \(y - \T(y) = \zv\);
  that is, \(y = \T(y) \in \rg{\T}\).
  Hence \(\rg{\T} = \ns{\T}^{\perp}\).

  Using the preceding results, we have \(\rg{\T}^{\perp} = \ns{\T}^{\perp \perp} \supseteq \ns{\T}\) by \cref{ex:6.2.13}(b).
  Now suppose that \(x \in \rg{\T}^{\perp}\).
  For any \(y \in \V\), we have \(\inn{\T(x), y} = \inn{x, \T^*(y)} = \inn{x, \T(y)} = 0\).
  So \(\T(x) = \zv\), and thus \(x \in \ns{\T}\).
  Hence \(\rg{\T}^{\perp} = \ns{\T}\).
\end{proof}

\begin{prop}\label{6.6.5}
  Let \(\V\) be a finite-dimensional inner product space over \(\F\), \(\W\) be a subspace of \(\V\) over \(\F\), and \(\T\) be the orthogonal projection of \(\V\) on \(\W\).
  We may choose an orthonormal basis \(\beta = \set{\seq{v}{1,,n}}\) for \(\V\) over \(\F\) such that \(\set{\seq{v}{1,,k}}\) is a basis for \(\W\) over \(\F\) (see \cref{6.5}).
  Then \([\T]_{\beta}\) is a diagonal matrix with ones as the first \(k\) diagonal entries and zeros elsewhere.
  In fact, \([\T]_{\beta}\) has the form
  \[
    \begin{pmatrix}
      I_k   & \zm_1 \\
      \zm_2 & \zm_3
    \end{pmatrix}.
  \]
  If \(\U\) is any projection on \(\W\), we may choose a basis \(\gamma\) for \(\V\) such that \([\U]_{\gamma}\) has the form above;
  however \(\gamma\) is not necessarily orthonormal.
\end{prop}

\begin{proof}[\pf{6.6.5}]
  Since
  \[
    \forall i \in \set{1, \dots, n}, \T(v_i) = \begin{dcases}
      v_i & \text{if } i \in \set{1, \dots, k}     \\
      \zv & \text{if } i \in \set{k + 1, \dots, n}
    \end{dcases},
  \]
  by \cref{2.2.4} we know that
  \[
    [\T]_{\beta} = \begin{pmatrix}
      I_k   & \zm_1 \\
      \zm_2 & \zm_3
    \end{pmatrix}.
  \]
  Thus by \cref{1.3.8} \([\T]_{\beta}\) is a diagonal matrix.
\end{proof}

\begin{thm}[The Spectral Theorem]\label{6.25}
  Suppose that \(\T\) is a linear operator on a finite-dimensional inner product space \(\V\) over \(\F\) with the distinct eigenvalues \(\seq{\lambda}{1,,k} \in \F\).
  Assume that \(\T\) is normal if \(\F = \C\) and that \(\T\) is self-adjoint if \(\F = \R\).
  For each \(i \in \set{1, \dots, k}\), let \(\W_i\) be the eigenspace of \(\T\) corresponding to the eigenvalue \(\lambda_i\), and let \(\T_i\) be the orthogonal projection of \(\V\) on \(\W_i\).
  Then the following statements are true.
  \begin{enumerate}
    \item \(\V = \seq[\oplus]{\W}{1,,k}\).
    \item If \(\W_i'\) denotes the direct sum of the subspaces \(\W_j\) for \(j \neq i\), then \(\W_i^{\perp} = \W_i'\).
    \item \(\T_i \T_j = \delta_{i j} \T_i\) for \(i, j \in \set{1, \dots, k}\).
    \item \(\IT[\V] = \seq[+]{\T}{1,,k}\).
    \item \(\T = \seq[+]{\lambda,\T}{1,,k}\).
  \end{enumerate}
  The set \(\set{\seq{\lambda}{1,,k}}\) of eigenvalues of \(\T\) is called the \textbf{spectrum} of \(\T\), the sum \(\IT[\V] = \seq[+]{\T}{1,,k}\) in (d) is called the \textbf{resolution of the identity operator} induced by \(\T\), and the sum \(\T = \seq[+]{\lambda,\T}{1,,k}\) in (e) is called the \textbf{spectral decomposition} of \(\T\).
  The spectral decomposition of \(\T\) is unique up to the order of its eigenvalues.
\end{thm}

\begin{proof}[\pf{6.25}(a)]
  By \cref{6.16,6.17}, \(\T\) is diagonalizable;
  so
  \[
    \V = \seq[\oplus]{\W}{1,,k}
  \]
  by \cref{5.11}.
\end{proof}

\begin{proof}[\pf{6.25}(b)]
  If \(x \in \W_i\) and \(y \in \W_j\) for some \(i \neq j\), then \(\inn{x, y} = 0\) by \cref{6.15}(d).
  It follows easily from this result that \(\W_i' \subseteq \W_i^{\perp}\).
  From \cref{6.25}(a), we have
  \[
    \dim(\W_i') = \sum_{\substack{j = 1 \\ j \neq i}}^k \dim(\W_j) = \dim(\V) - \dim(\W_i).
  \]
  On the other hand, we have \(\dim(\W_i^{\perp}) = \dim(\V) - \dim(\W_i)\) by \cref{6.7}(c).
  Hence \(\W_i' = \W_i^{\perp}\) (see \cref{1.11}).
\end{proof}

\begin{proof}[\pf{6.25}(c)]
  For each \(x \in \V\), we can define unique \(k\)-tuple \(\tuple{x}{1,,k} \in \seq[\times]{\W}{1,,k}\) such that \(x = \sum_{i = 1}^k x_i\) thanks to \cref{6.25}(a).
  Then we have
  \begin{align*}
    \forall x \in \V, \T_i(\T_j(x)) & = \T_i(x_j)                  &  & \by{6.6.1} \\
                                    & = \begin{dcases}
                                          x_i & \text{if } i = j    \\
                                          \zv & \text{if } i \neq j
                                        \end{dcases} &  & \by{6.6.1}               \\
                                    & = \delta_{i j} x_i           &  & \by{2.3.4} \\
                                    & = \delta_{i j} \T_i(x)       &  & \by{6.6.1}
  \end{align*}
  and thus \(\T_i \T_j = \delta_{i j} \T_i\).
\end{proof}

\begin{proof}[\pf{6.25}(d)]
  Since \(\T_i\) is the orthogonal projection of \(\V\) on \(\W_i\), it follows from \cref{6.25}(b) that \(\ns{\T_i} = \rg{\T_i}^{\perp} = \W_i^{\perp} = \W_i'\).
  Hence, for \(x \in \V\), we have \(x = \seq[+]{x}{1,,k}\), where \(\T_i(x) = x_i \in \W_i\) for all \(i \in \set{1, \dots, k}\).
\end{proof}

\begin{proof}[\pf{6.25}(e)]
  For \(x \in \V\), write \(x = \seq[+]{x}{1,,k}\), where \(x_i \in \W_i\) for all \(i \in \set{1, \dots, k}\).
  Then
  \begin{align*}
    \T(x) & = \T(\seq[+]{x}{1,,k})                                              \\
          & = \T(x_1) + \cdots + \T(x_k)                     &  & \by{2.1.1}[a] \\
          & = \seq[+]{\lambda,x}{1,,k}                       &  & \by{5.1.2}    \\
          & = \lambda_1 \T_1(x) + \cdots + \lambda_k \T_k(x) &  & \by{6.6.1}    \\
          & = (\lambda_1 \T_1 + \cdots + \lambda_k \T_k)(x). &  & \by{2.2.5}
  \end{align*}
\end{proof}

\begin{prop}\label{6.6.6}
  With the notation of \cref{6.25}, let \(\beta\) be the union of orthonormal bases of the \(\W_i\)'s and let \(m_i = \dim(\W_i)\).
  (Thus \(m_i\) is the multiplicity of \(\lambda_i\), see \cref{5.2.3,5.7}.)
  Then \([\T]_{\beta}\) has the form
  \[
    \begin{pmatrix}
      \lambda_1 I_{m_1} & \zm               & \cdots & \zm               \\
      \zm               & \lambda_2 I_{m_2} & \cdots & \zm               \\
      \vdots            & \vdots            &        & \vdots            \\
      \zm               & \zm               & \cdots & \lambda_k I_{m_k}
    \end{pmatrix};
  \]
  that is, \([\T]_{\beta}\) is a diagonal matrix in which the diagonal entries are the eigenvalues \(\lambda_i\) of \(\T\), and each \(\lambda_i\) is repeated \(m_i\) times.
  If \(\seq[+]{\lambda,\T}{1,,k}\) is the spectral decomposition of \(\T\), then it follows (from \cref{ex:6.6.7}(a)) that \(g(\T) = g(\lambda_1) \T_1 + \cdots + g(\lambda_k) \T_k\) for any polynomial \(g\).
\end{prop}

\begin{proof}[\pf{6.6.6}]
  By \cref{6.25}(a) and \cref{5.7} we see that this is true.
\end{proof}

\begin{cor}\label{6.6.7}
  Let \(\V\) be a finite-dimensional complex inner product space and let \(\T \in \ls(\V)\).
  Then \(\T\) is normal iff \(\T^* = g(\T)\) for some polynomial \(g\).
\end{cor}

\begin{proof}[\pf{6.6.7}]
  Suppose first that \(\T\) is normal.
  Let \(\T = \seq[+]{\lambda,\T}{1,,k}\) be the spectral decomposition of \(\T\).
  Taking the adjoint of both sides of the preceding equation, we have \(\T^* = \seq[+]{\conj{\lambda},\T}{1,,k}\) (\cref{6.11}(a)(b)) since each \(\T_i\) is self-adjoint (\cref{6.24}).
  Using the Lagrange interpolation formula (see \cref{1.6.20}), we may choose a polynomial \(g\) such that \(g(\lambda_i) = \conj{\lambda_i}\) for \(i \in \set{1, \dots, k}\).
  Then
  \begin{align*}
    g(\T) & = g(\lambda_1) \T_1 + \cdots + g(\lambda_k) \T_k             &  & \by{ex:6.6.7}[a] \\
          & = \seq[+]{\conj{\lambda},\T}{1,,k}                                                 \\
          & = \conj{\lambda}_1 \T_1^* + \cdots + \conj{\lambda}_k \T_k^* &  & \by{6.24}        \\
          & = (\seq[+]{\lambda,\T}{1,,k})^*                              &  & \by{6.11}[a,b]   \\
          & = \T^*.                                                      &  & \by{6.25}[e]
  \end{align*}

  Conversely, if \(\T^* = g(\T)\) for some polynomial \(g\), then \(\T\) commutes with \(\T^*\) since \(\T\) commutes with every polynomial in \(\T\).
  So \(\T\) is normal (\cref{6.4.3}).
\end{proof}

\begin{cor}\label{6.6.8}
  Let \(\V\) be a finite-dimensional complex inner product space and let \(\T \in \ls(\V)\).
  Then \(\T\) is unitary iff \(\T\) is normal and \(\abs{\lambda} = 1\) for every eigenvalue \(\lambda\) of \(\T\).
\end{cor}

\begin{proof}[\pf{6.6.8}]
  If \(\T\) is unitary, then \(\T\) is normal and every eigenvalue of \(\T\) has absolute value \(1\) by \cref{6.5.5}.

  Let \(\T = \seq[+]{\lambda,\T}{1,,k}\) be the spectral decomposition of \(\T\).
  If \(\abs{\lambda} = 1\) for every eigenvalue \(\lambda\) of \(\T\), then
  \begin{align*}
    \T \T^* & = (\seq[+]{\lambda,\T}{1,,k}) (\lambda_1 \T_1 + \cdots + \lambda_k \T_k)^*                 &  & \by{6.25}[e]                                           \\
            & = (\seq[+]{\lambda,\T}{1,,k}) (\conj{\lambda}_1 \T_1^* + \cdots + \conj{\lambda}_k \T_k^*) &  & \by{6.11}[a,b]                                         \\
            & = (\seq[+]{\lambda,\T}{1,,k}) (\seq[+]{\conj{\lambda},\T}{1,,k})                           &  & \by{6.24}                                              \\
            & = \abs{\lambda_1}^2 \T_1 + \cdots + \abs{\lambda_k}^2 \T_k                                 &  & \by{6.25}[c]                                           \\
            & = \seq[+]{\T}{1,,k}                                                                        &  & (\forall i \in \set{1, \dots, k}, \abs{\lambda_i} = 1) \\
            & = \IT[\V].                                                                                 &  & \by{6.25}[d]
  \end{align*}
  Hence \(\T\) is unitary by \cref{6.18}(a)(e).
\end{proof}

\begin{cor}\label{6.6.9}
  Let \(\V\) be a finite-dimensional complex inner product space and let \(\T \in \ls(\V)\).
  If \(\T\) is normal, then \(\T\) is self-adjoint iff every eigenvalue of \(\T\) is real.
\end{cor}

\begin{proof}[\pf{6.6.9}]
  Let \(\T = \seq[+]{\lambda,\T}{1,,k}\) be the spectral decomposition of \(\T\).
  Suppose that every eigenvalue of \(\T\) is real.
  Then
  \begin{align*}
    \T^* & = (\seq[+]{\lambda,\T}{1,,k})^*                                &  & \by{6.25}[e]                                        \\
         & = (\conj{\lambda}_1 \T_1^* + \cdots + \conj{\lambda}_k \T_k^*) &  & \by{6.11}[a,b]                                      \\
         & = \seq[+]{\conj{\lambda},\T}{1,,k}                             &  & \by{6.24}                                           \\
         & = \seq[+]{\lambda,\T}{1,,k}                                    &  & (\forall i \in \set{1, \dots, k}, \lambda_i \in \R) \\
         & = \T.                                                          &  & \by{6.25}[e]
  \end{align*}
  The converse has been proved by \cref{6.4.10}(a).
\end{proof}

\begin{cor}\label{6.6.10}
  Let \(\T\) be as in the spectral theorem with spectral decomposition \(\T = \seq[+]{\lambda,\T}{1,,k}\).
  Then each \(\T_j\) is a polynomial in \(\T\).
\end{cor}

\begin{proof}[\pf{6.6.10}]
  Using the Lagrange interpolation formula (see \cref{1.6.20}), we may choose a polynomial \(g_j\) (\(j \in \set{1, \dots, k}\)) such that \(g_j(\lambda_i) = \delta_{i j}\).
  Then
  \begin{align*}
    g_j(\T) & = g_j(\lambda_1) \T_1 + \cdots + g_j(\lambda_k) \T_k &  & \by{ex:6.6.7}[a] \\
            & = \delta_{1 j} \T_1 + \cdots + \delta_{k j} \T_k                           \\
            & = \T_j.                                              &  & \by{2.3.4}
  \end{align*}
\end{proof}

\exercisesection

\setcounter{ex}{3}
\begin{ex}\label{ex:6.6.4}
  Let \(\W\) be a finite-dimensional subspace of an inner product space \(\V\) over \(\F\).
  Show that if \(\T\) is the orthogonal projection of \(\V\) on \(\W\), then \(\IT[\V] - \T\) is the orthogonal projection of \(\V\) on \(\W^{\perp}\).
\end{ex}

\begin{proof}[\pf{ex:6.6.4}]
  For each \(x \in \V\), we can define a unique tuple \(\tuple{x}{1,2} \in \W \times \W^{\perp}\) such that \(x = x_1 + x_2\) thanks to \cref{6.6}.
  First we show that \(\IT[\V] - \T\) is a projection of \(\V\) on \(\W^{\perp}\).
  Since
  \begin{align*}
    \forall x \in \V, (\IT[\V] - \T)(x) & = \IT[\V](x) - \T(x)  &  & \by{2.2.5} \\
                                        & = x - \T(x)           &  & \by{2.1.9} \\
                                        & = x_1 + x_2 - x_1     &  & \by{6.6.1} \\
                                        & = x_2 \in \W^{\perp},
  \end{align*}
  by \cref{2.1.14} we see that \(\IT[\V] - \T\) is a projection of \(\V\) on \(\W^{\perp}\).

  Next we show that \(\inn{x, (\IT[\V] - \T)(y)} = \inn{(\IT[\V] - \T)(x), y}\) for all \(x, y \in \V\).
  This is true since
  \begin{align*}
    \forall x, y \in \V, \inn{x, (\IT[\V] - \T)(y)} & = \inn{x, y_2}                    &  & \text{(from the proof above)} \\
                                                    & = \inn{x_1 + x_2, y_2}            &  & \by{6.6}                      \\
                                                    & = \inn{x_1, y_2} + \inn{x_2, y_2} &  & \by{6.1.1}[a]                 \\
                                                    & = \inn{x_2, y_2}                  &  & \by{6.2.9}                    \\
                                                    & = \inn{x_2, y_1} + \inn{x_2, y_2} &  & \by{6.2.9}                    \\
                                                    & = \inn{x_2, y_1 + y_2}            &  & \by{6.1}[a]                   \\
                                                    & = \inn{x_2, y}                    &  & \by{6.6}                      \\
                                                    & = \inn{(\IT[\V] - \T)(x), y}.     &  & \text{(from the proof above)}
  \end{align*}
  By \cref{6.9} this means \((\IT[\V] - \T)^* = (\IT[\V] - \T)\).

  Now we show that \(\IT[\V] - \T\) is the orthogonal projection of \(\V\) on \(\W^{\perp}\).
  Since
  \begin{align*}
    (\IT[\V] - \T)^2 & = (\IT[\V] - \T)(\IT[\V] - \T)                                \\
                     & = \IT[\V]^2 - \IT[\V] \T - \T \IT[\V] + \T^2 &  & \by{2.2.5}  \\
                     & = \IT[\V] - 2 \T + \T^2                      &  & \by{2.1.10} \\
                     & = \IT[\V] - 2 \T + \T                        &  & \by{6.24}   \\
                     & = \IT[\V] - \T,
  \end{align*}
  by \cref{6.24} we know that \(\IT[\V] - \T\) is the orthogonal projection of \(\V\) on \(\W^{\perp}\).
\end{proof}

\begin{ex}\label{ex:6.6.5}
  Let \(\T\) be a linear operator on a finite-dimensional inner product space \(\V\) over \(\F\).
  \begin{enumerate}
    \item If \(\T\) is an orthogonal projection, prove that \(\norm{\T(x)} \leq \norm{x}\) for all \(x \in \V\).
          Give an example of a projection for which this inequality does not hold.
          What can be concluded about a projection for which the inequality is actually an equality for all \(x \in \V\)?
    \item Suppose that \(\T\) is a projection such that \(\norm{\T(x)} \leq \norm{x}\) for \(x \in \V\).
          Prove that \(\T\) is an orthogonal projection.
  \end{enumerate}
\end{ex}

\begin{proof}[\pf{ex:6.6.5}(a)]
  Since \(\T\) is an orthogonal projection, by \cref{6.6.1} there exists a finite-dimensional subspace \(\W\) of \(\V\) over \(\F\) such that \(\V = \W \oplus \W^{\perp}\) and \(\W = \rg{\T}\).
  For each \(x \in \V\), we can define a unique tuple \(\tuple{x}{1,2} \in \W \times \W^{\perp}\) such that \(x = x_1 + x_2\) thanks to \cref{6.6}.
  Then we have
  \begin{align*}
    \forall x \in \V, \norm{\T(x)}^2 & = \norm{x_1}^2                   &  & \by{6.6}       \\
                                     & \leq \norm{x_1}^2 + \norm{x_2}^2 &  & \by{6.2}[b]    \\
                                     & = \norm{x_1 + x_2}^2             &  & \by{ex:6.1.10} \\
                                     & = \norm{x}^2.                    &  & \by{6.6}
  \end{align*}
  Thus by \cref{6.2}(b) we have \(\norm{\T(x)} \leq \norm{x}\) for all \(x \in \V\).

  Next we show that there exist some projection \(\U\) such that \(\norm{\U} > \norm{x}\) for some \(x \in \V\).
  Let \(\U \in \ls(\R^2)\) such that \(\U(x, y) = (x, x)\) for all \((x, y) \in \R^2\).
  Then we have
  \begin{align*}
             & \forall (x, y) \in \R^2, \U^2(x, y) = \U(\U(x, y)) = \U(x, x) = (x, x) = \U(x, y)                     \\
    \implies & \U \text{ is a projection of } \V \text{ on } \spn{\set{(1, 1)}}                  &  & \by{ex:2.3.17}
  \end{align*}
  and
  \begin{align*}
    \norm{\U(1, 0)} & = \norm{(1, 1)}                  \\
                    & = \sqrt{2}       &  & \by{6.1.2} \\
                    & > 1                              \\
                    & = \norm{(1, 0)}. &  & \by{6.1.2}
  \end{align*}

  Now suppose that \(\T\) is a projection such that \(\norm{\T(x)} = \norm{x}\) for all \(x \in \V\).
  Then by \cref{ex:6.1.17,2.4} we have \(\ns{\T} = \set{\zv}\).
  By \cref{ex:2.1.26}(b) we see that \(\rg{\T} = \V\), therefore \(\T = \IT[\V]\).
\end{proof}

\begin{proof}[\pf{ex:6.6.5}(b)]
  Since \(\T\) is a projection, by \cref{2.1.14} there exist two subspaces \(\W_1\) and \(\W_2\) of \(\V\) over \(\F\) such that \(\V = \W_1 \oplus \W_2\), \(\T_{\W_1} = \IT[\W_1]\) and \(\ns{\T} = \W_2\).
  For each \(x \in \V\), we define \((x_1, x_2) \in \W_1 \times \W_2\) such that \(x = x_1 + x_2\).
  We claim that \(\W_2 = \W_1^{\perp}\).
  Suppose for sake of contradiction that \(\W_2 \neq \W_1^{\perp}\).
  Then there exists a tuple \((y_1, y_2) \in \W_1 \times \W_2\) such that \(\inn{y_1, y_2} \neq 0\).
  Now fix one such tuple.
  For any \(c \in \F\), we have
  \begin{align*}
    \norm{c y_1 + y_2}^2     & = \norm{c y_1}^2 + 2 \Re(\inn{c y_1, y_2}) + \norm{y_2}^2 &  & \by{ex:6.1.19}[a] \\
                             & = \norm{c y_1}^2 + 2 \Re(c \inn{y_1, y_2}) + \norm{y_2}^2 &  & \by{6.1.1}[b]     \\
    \norm{\T(c y_1 + y_2)}^2 & = \norm{c \T(y_1) + \T(y_2)}                              &  & \by{2.1.2}[b]     \\
                             & = \norm{c y_1}^2.                                         &  & \by{2.1.14}
  \end{align*}
  By hypothesis we have
  \begin{align*}
             & \forall x \in \V, \norm{\T(x)} \leq \norm{x}                                \\
    \implies & \norm{\T(c y_1 + y_2)}^2 \leq \norm{c y_1 + y_2}^2                          \\
    \implies & \norm{c y_1}^2 \leq \norm{c y_1}^2 + 2 \Re(c \inn{y_1, y_2}) + \norm{y_2}^2 \\
    \implies & 0 \leq 2 \Re(c \inn{y_1, y_2}) + \norm{y_2}^2.
  \end{align*}
  Since \(\inn{y_1, y_2} \neq 0\), we can define \(c\) to be
  \[
    c = \frac{-\norm{y_2}^2}{\Re(\inn{y_1, y_2})}.
  \]
  Note that \(c \in \R\).
  Then we have
  \begin{align*}
    0 & \leq 2 \Re\pa{\frac{-\norm{y_2}^2}{\Re(\inn{y_1, y_2})} \inn{y_1, y_2}} + \norm{y_2}^2 \\
      & = 2 \frac{-\norm{y_2}^2}{\Re(\inn{y_1, y_2})} \Re\pa{\inn{y_1, y_2}} + \norm{y_2}^2    \\
      & = - 2 \norm{y_2}^2 + \norm{y_2}^2                                                      \\
      & = - \norm{y_2}^2
  \end{align*}
  and thus \(\norm{y_2}^2 \leq 0\).
  But by \cref{6.2}(b) this implies \(y_2 = \zv\), which means \(\inn{y_1, y_2} = 0\), a contradiction.
  Thus we must have \(\W_2 = \W_1^{\perp}\), and by \cref{6.6.2} \(\T\) is the orthogonal projection of \(\V\) on \(\W_1\).
\end{proof}

\begin{ex}\label{ex:6.6.6}
  Let \(\T\) be a normal operator on a finite-dimensional inner product space \(\V\) over \(\F\).
  Prove that if \(\T\) is a projection, then \(\T\) is also an orthogonal projection.
\end{ex}

\begin{proof}[\pf{ex:6.6.6}]
  By \cref{ex:2.1.26}(b) we have \(\V = \rg{\T} \oplus \ns{\T}\).
  We claim that \(\rg{\T}^{\perp} = \ns{\T}\).
  Since
  \begin{align*}
             & v \in \rg{\T}^{\perp}                                      \\
    \implies & \inn{\T(v), \T(v)} = \inn{v, \T^*(\T(v))} &  & \by{6.9}    \\
             & = \inn{v, \T(\T^*(v))}                    &  & \by{6.4.3}  \\
             & = 0                                       &  & \by{6.2.9}  \\
    \implies & \T(v) = \zv                               &  & \by{6.1}[d] \\
    \implies & v \in \ns{\T}                             &  & \by{2.1.10}
  \end{align*}
  and
  \begin{align*}
             & v \in \ns{\T}                                                            \\
    \implies & 0 = \inn{v, \zv}                                      &  & \by{6.1}[c]   \\
             & = \inn{v, \T^*(\zv)}                                  &  & \by{2.1.2}[a] \\
             & = \inn{v, \T^*(\T(v))}                                &  & \by{2.1.10}   \\
             & = \inn{v, \T(\T^*(v))}                                &  & \by{6.4.3}    \\
             & = \inn{\T^*(v), \T^*(v)}                              &  & \by{6.9}      \\
    \implies & \T^*(v) = \zv                                         &  & \by{6.1}[d]   \\
    \implies & \forall x \in \V, 0 = \inn{\zv, x} = \inn{\T^*(v), x} &  & \by{6.1}[d]   \\
             & = \inn{v, \T(x)}                                      &  & \by{6.9}      \\
    \implies & \forall y \in \rg{\T}, \inn{v, y} = 0                 &  & \by{2.1.10}   \\
    \implies & v \in \rg{\T}^{\perp},                                &  & \by{6.2.9}
  \end{align*}
  we have \(\rg{\T}^{\perp} \subseteq \ns{\T}\) and \(\ns{\T} \subseteq \rg{\T}^{\perp}\).
  Thus \(\rg{\T}^{\perp} = \ns{\T}\).
  Since \(\V\) is finite-dimensional, by \cref{ex:6.2.13}(c) we have \(\rg{\T} = (\rg{\T}^{\perp})^{\perp} = \ns{\T}^{\perp}\).
  Thus by \cref{6.6.1} \(\T\) is an orthogonal projection.
\end{proof}

\begin{ex}\label{ex:6.6.7}
  Let \(\T\) be a normal operator on a finite-dimensional complex inner product space \(\V\).
  Use the spectral decomposition \(\seq[+]{\lambda,\T}{1,,k}\) of \(\T\) to prove the following results.
  \begin{enumerate}
    \item If \(g\) is a polynomial, then
          \[
            g(\T) = \sum_{i = 1}^k g(\lambda_i) \T_i.
          \]
    \item If \(\T^n = \zT\) for some \(n \in \Z^+\), then \(\T = \zT\).
    \item Let \(\U\) be a linear operator on \(\V\).
          Then \(\U\) commutes with \(\T\) iff \(\U\) commutes with each \(\T_i\).
    \item There exists a normal operator \(\U\) on \(\V\) such that \(\U^2 = \T\).
    \item \(\T\) is invertible iff \(\lambda_i \neq 0\) for \(i \in \set{1, \dots, k}\).
    \item \(\T\) is a projection iff every eigenvalue of \(\T\) is \(1\) or \(0\).
    \item \(\T = -\T^*\) iff every \(\lambda_i\) is an imaginary number.
  \end{enumerate}
\end{ex}

\begin{proof}[\pf{ex:6.6.7}(a)]
  Let \(\seq{a}{0,,n} \in \C\) such that \(g(x) = a_0 + a_1 x + \cdots + a_n x^n\) for all \(x \in \C\).
  Then we have
  \begin{align*}
    g(\T) & = a_0 \IT[\V] + a_1 \T + \cdots + a_n \T^n                    &  & \by{e.0.7}   \\
          & = a_0 (\seq[+]{\T}{1,,k})                                     &  & \by{6.25}[d] \\
          & \quad + a_1 (\seq[+]{\lambda,\T}{1,,k})                       &  & \by{6.25}[e] \\
          & \quad + \cdots                                                                  \\
          & \quad + a_n (\seq[+]{\lambda,\T}{1,,k})^n                     &  & \by{6.25}[e] \\
          & = a_0 (\seq[+]{\T}{1,,k})                                                       \\
          & \quad + a_1 (\seq[+]{\lambda,\T}{1,,k})                                         \\
          & \quad + \cdots                                                                  \\
          & \quad + a_n (\lambda_1^n \T_1 + \cdots + \lambda_k^n \T_k)    &  & \by{6.25}[c] \\
          & = (a_0 + a_1 \lambda_1 + \cdots + a_n \lambda_1^n) \T_1       &  & \by{2.2.5}   \\
          & \quad + \cdots                                                                  \\
          & \quad + (a_0 + a_1 \lambda_1 + \cdots + a_n \lambda_k^n) \T_k                   \\
          & = g(\lambda_1) \T_1 + \cdots + g(\lambda_k) \T_k                                \\
          & = \sum_{i = 1}^k g(\lambda_i) \T_i.
  \end{align*}
\end{proof}

\begin{proof}[\pf{ex:6.6.7}(b)]
  For each \(i \in \set{1, \dots, k}\), let \(\W_i\) be the eigenspace of \(\T\) corresponding to \(\lambda_i\).
  Then we have
  \begin{align*}
    \zT & = \T^n                                                           \\
        & = (\seq[+]{\lambda,\T}{1,,k})^n                &  & \by{6.25}[e] \\
        & = \lambda_1^n \T_1 + \cdots + \lambda_k^n \T_k &  & \by{6.25}[c] \\
        & = \sum_{i = 1}^k \lambda_i^n \T_i
  \end{align*}
  and
  \begin{align*}
             & \forall j \in \set{1, \dots, k}, \forall x \in \W_j, \zv = \zT(x) = \pa{\sum_{i = 1}^k \lambda_i^n \T_i}(x) &  & \by{2.1.9}   \\
    \implies & \forall j \in \set{1, \dots, k}, \forall x \in \W_j, \zv = \lambda_j^n x                                    &  & \by{6.6.1}   \\
    \implies & \forall j \in \set{1, \dots, k}, \lambda_j^n = 0                                                            &  & \by{5.7}     \\
    \implies & \forall j \in \set{1, \dots, k}, \lambda_j = 0                                                                                \\
    \implies & \T = \sum_{i = 1}^k 0 \T_i = \zT.                                                                           &  & \by{6.25}[e]
  \end{align*}
\end{proof}

\begin{proof}[\pf{ex:6.6.7}(c)]
  First suppose that \(\U \T = \T \U\).
  By \cref{6.6.10} we can define polynomial \(g_i\) such that \(g_i(\T) = \T_i\) for all \(i \in \set{1, \dots, k}\).
  Then we have
  \begin{align*}
    \forall i \in \set{1, \dots, k}, \U \T_i & = \U g_i(\T) &  & \by{6.6.10}     \\
                                             & = g_i(\T) \U &  & (\U \T = \T \U) \\
                                             & = \T_i \U.   &  & \by{6.6.10}
  \end{align*}

  Now suppose that \(\U \T_i = \T_i \U\) for all \(i \in \set{1, \dots, k}\).
  Then we have
  \begin{align*}
    \U \T & = \U (\seq[+]{\lambda,\T}{1,,k})                 &  & \by{6.25}[e] \\
          & = \lambda_1 \U \T_1 + \cdots + \lambda_k \U \T_k &  & \by{2.2.5}   \\
          & = \lambda_1 \T_1 \U + \cdots + \lambda_k \T_k \U                   \\
          & = (\seq[+]{\lambda,\T}{1,,k}) \U                 &  & \by{2.2.5}   \\
          & = \T \U.                                         &  & \by{6.25}[e]
  \end{align*}
\end{proof}

\begin{proof}[\pf{ex:6.6.7}(d)]
  For each \(i \in \set{1, \dots, k}\), define \(\mu_i \in \C\) such that \(\mu_i^2 = \lambda_i\).
  Let \(\U = \sum_{i = 1}^k \mu_i \T_i\).
  Then we have
  \begin{align*}
    \U^2 & = \pa{\sum_{i = 1}^k \mu_i \T_i}^2                   \\
         & = \sum_{i = 1}^k \mu_i^2 \T_i      &  & \by{6.25}[c] \\
         & = \sum_{i = 1}^k \lambda_i \T_i                      \\
         & = \T.                              &  & \by{6.25}[e]
  \end{align*}
\end{proof}

\begin{proof}[\pf{ex:6.6.7}(e)]
  We have
  \begin{align*}
         & \T \text{ is invertible}                                                                         \\
    \iff & \ns{\T} = \set{\zv}                                                            &  & \by{2.4,2.5} \\
    \iff & \forall x \in \V \setminus \set{\zv}, \T(x) \neq \zv                                             \\
    \iff & \forall x \in \V \setminus \set{\zv}, (\seq[+]{\lambda,\T}{1,,k})(x) \neq \zv  &  & \by{6.25}[e] \\
    \iff & \forall i \in \set{1, \dots, k}, \exists x \in \V : \lambda_i \T_i(x) \neq \zv &  & \by{6.25}[a] \\
    \iff & \forall i \in \set{1, \dots, k}, \lambda_i \neq 0.                             &  & \by{5.1.2}
  \end{align*}
\end{proof}

\begin{proof}[\pf{ex:6.6.7}(f)]
  First suppose that \(\T\) is a projection.
  By \cref{ex:6.6.6} we see that \(\T\) is an orthogonal projection.
  By \cref{6.6.1,2.1.14} we see that \(\V = \rg{\T} \oplus \ns{\T}\) and \(\T_{\rg{\T}} = \IT[\rg{\T}]\).
  Then by \cref{2.1.10,5.1.2} we see that eigenvalues of \(\T\) are \(0\) or \(1\).

  Now suppose that eigenvalues of \(\T\) are \(0\) or \(1\).
  Let \(\lambda_1 = 1\) and \(\lambda_2 = 0\).
  Then by \cref{6.25}(e) we have \(\T = \seq[+]{\lambda,\T}{1,2} = \T_1\), where \(\T_1\) is a projection.
\end{proof}

\begin{proof}[\pf{ex:6.6.7}(g)]
  First suppose that \(\T = -\T^*\).
  Then we have
  \begin{align*}
             & \T = -\T^*                                                                                                           \\
    \implies & \seq[+]{\lambda,\T}{1,,k} = -(\seq[+]{\lambda,\T}{1,,k})^*                                      &  & \by{6.25}[e]    \\
             & = - (\conj{\lambda}_1 \T_1^* + \cdots + \conj{\lambda}_k \T_k^*)                                &  & \by{6.3.2}[a,b] \\
             & = - (\seq[+]{\conj{\lambda},\T}{1,,k})                                                          &  & \by{6.24}       \\
    \implies & (\lambda_1 + \conj{\lambda}_1) \T_1 + \cdots + (\lambda_k + \conj{\lambda}_k) \T_k = \zT                             \\
    \implies & \forall i \in \set{1, \dots, k}, \forall x \in \V, (\lambda_i + \conj{\lambda}_i) \T_i(x) = \zv &  & \by{6.25}[a]    \\
    \implies & \forall i \in \set{1, \dots, k}, \lambda_i + \conj{\lambda}_i = 0                                                    \\
    \implies & \forall i \in \set{1, \dots, k}, \Re(\lambda_i) = 0.                                            &  & \by{d.0.4}
  \end{align*}

  Now suppose that \(\Re(\lambda_i) = 0\) for all \(i \in \set{1, \dots, k}\).
  Then we have
  \begin{align*}
    -\T^* & = -(\seq[+]{\lambda,\T}{1,,k})^*                                 &  & \by{6.25}[e]    \\
          & = - (\conj{\lambda}_1 \T_1^* + \cdots + \conj{\lambda}_k \T_k^*) &  & \by{6.3.2}[a,b] \\
          & = - (\seq[+]{\conj{\lambda},\T}{1,,k})                           &  & \by{6.24}       \\
          & = - ((-\lambda_1) \T_1 + \cdots + (-\lambda_k) \T_k)                                  \\
          & = \seq[+]{\lambda,\T}{1,,k}                                      &  & \by{2.2.5}      \\
          & = \T.                                                            &  & \by{6.25}[e]
  \end{align*}
\end{proof}

\begin{ex}\label{ex:6.6.8}
  Use \cref{6.6.7} to show that if \(\T\) is a normal operator on a complex finite-dimensional inner product space and \(\U\) is a linear operator that commutes with \(\T\), then \(\U\) commutes with \(\T^*\).
\end{ex}

\begin{proof}[\pf{ex:6.6.8}]
  By \cref{6.6.7} there exists a polynomial \(g\) such that \(\T^* = g(\T)\).
  Then we have
  \begin{align*}
    \U \T^* & = \U g(\T) &  & \by{6.6.7}      \\
            & = g(\T) \U &  & (\U \T = \T \U) \\
            & = \T^* \U. &  & \by{6.6.7}
  \end{align*}
\end{proof}

\begin{ex}\label{ex:6.6.9}
  Referring to \cref{ex:6.5.20}, prove the following facts about a partial isometry \(\U\).
  \begin{enumerate}
    \item \(\U^* \U\) is an orthogonal projection on \(\W\).
    \item \(\U \U^* \U = \U\).
  \end{enumerate}
\end{ex}

\begin{proof}[\pf{ex:6.6.9}(a)]
  Let \(\W\) be a subspace of \(\V\) over \(\F\) such that \(\norm{\U(x)} = \norm{x}\) for all \(x \in \W\) and \(\U(x) = \zv\) for all \(x \in \W^{\perp}\).
  By \cref{2.1.2}(a) this means \((\U^* \U)(\W^{\perp}) = \set{\zv}\).
  By \cref{ex:6.5.20}(e) we see that \((\U^* \U)_{\W} = \IT[\W]\).
  Thus by \cref{6.6.1,6.6.2} \(\U^* \U\) is the orthogonal projection of \(\V\) on \(\W\).
\end{proof}

\begin{proof}[\pf{ex:6.6.9}(b)]
  Continue from the proof of \cref{ex:6.6.9}(a), we see that
  \begin{align*}
    (\U \U^* \U)_{\W} & = \U((\U^* \U)_{\W})                       \\
                      & = \U(\IT[\W])        &  & \by{ex:6.6.9}[a] \\
                      & = \U_{\W}
  \end{align*}
  and
  \begin{align*}
    (\U \U^* \U)_{\W^{\perp}} & = \U((\U^* \U)_{\W^{\perp}})                       \\
                              & = \U(\zT)                    &  & \by{ex:6.6.9}[a] \\
                              & = \zT                        &  & \by{2.1.2}[a]    \\
                              & = \U_{\W^{\perp}}.           &  & \by{ex:6.5.20}
  \end{align*}
  Thus \(\U \U^* \U = \U\).
\end{proof}

\begin{ex}[Simultaneous diagonalization]\label{ex:6.6.10}
  Let \(\U\) and \(\T\) be normal operators on a finite-dimensional complex inner product space \(\V\) such that \(\T \U = \U \T\).
  Prove that there exists an orthonormal basis for \(\V\) over \(\C\) consisting of vectors that are eigenvectors of both \(\T\) and \(\U\).
\end{ex}

\begin{proof}[\pf{ex:6.6.10}]
  Let \(n = \dim(\V)\).
  We use induction on \(n\).
  For \(n = 1\), any orthonormal basis for \(\V\) over \(\C\) makes \([\T]_{\beta}\) and \([\U]_{\beta}\) diagonal matrices.
  Thus by \cref{5.2.8} \(\T, \U\) are simultaneously diagonalizable by some orthonormal bases and the base case holds.
  Suppose inductively that for some \(n \geq 1\), normal linear operators on \(\V\) which commute are simultaneously diagonalizable by some orthonormal bases.
  We need to show that this is true for \(n + 1\).
  Let \(\dim(\V) = n + 1\), let \(\T, \U \in \ls(\V)\) such that \(\T \T^* = \T^* \T\), \(\U \U^* = \U^* \U\) and \(\U \T = \T \U\).
  Since \(\T\) is normal, by \cref{6.16} there exists an orthonormal basis \(\beta\) for \(\V\) over \(\C\) consisting of eigenvectors of \(\T\).
  Let \(\lambda \in \C\) be an eigenvalue of \(\T\) and let \(E_{\lambda}\) be the eigenspace of \(\lambda\).
  Now we split into two cases:
  \begin{itemize}
    \item If \(E_{\lambda} = \V\), then any basis for \(\V\) over \(\C\) is consist of eigenvectors of \(\T\).
          Since \(\U\) is normal, by \cref{6.16} we know that there exists an orthonormal basis for \(\V\) over \(\C\) consist of eigenvectors of \(\U\).
          Thus by \cref{5.2.8} \(\T, \U\) are simultaneously diagonalizable by some orthonormal bases for \(\V\) over \(\C\).
    \item If \(E_{\lambda} \neq \V\), then by \cref{5.7} we have \(1 \leq \dim(E_{\lambda}) < n + 1\).
          By \cref{5.4.2}(e) we know that \(E_{\lambda}\) is \(\T\)-invariant.
          We claim that \(E_{\lambda}\) is \(\U\)-invariant.
          Since
          \begin{align*}
                     & \forall v \in E_{\lambda}, \T(v) = \lambda v             &  & \by{5.2.4}    \\
            \implies & \forall v \in E_{\lambda}, \lambda \U(v) = \U(\lambda v) &  & \by{2.1.1}[b] \\
                     & = \U(\T(v)) = \T(\U(v))                                  &  & \by{5.1.2}    \\
            \implies & \forall v \in E_{\lambda}, \U(v) \in E_{\lambda}         &  & \by{5.1.2}    \\
            \implies & \U(E_{\lambda}) \subseteq E_{\lambda},
          \end{align*}
          by \cref{5.4.1} we know that \(E_{\lambda}\) is \(\U\)-invariant.
          Since \(\T, \U\) are normal, by \cref{ex:6.4.7}(b) and \cref{ex:6.4.8} we know that \(E_{\lambda}^{\perp}\) is both \(\T\)- and \(\U\)-invariant.
          By \cref{6.7}(c) we know that \(\dim(E_{\lambda}^{\perp}) < n + 1\).
          Thus by induction hypothesis we can find some orthonormal bases \(\beta_1\) and \(\beta_2\) for \(E_{\lambda}\) and \(E_{\lambda}^{\perp}\) over \(\C\), respectively, consist of eigenvectors of both \(\T\) and \(\U\).
          By \cref{5.10}(d) and \cref{6.7}(c) we know that \(\beta = \beta_1 \cup \beta_2\) is an orthonormal basis for \(\V\) over \(\C\) consist of eigenvectors of both \(\T\) and \(\U\).
          Thus by \cref{5.2.8} \(\T, \U\) are simultaneously diagonalizable by some orthonormal bases for \(\V\) over \(\C\).
  \end{itemize}
  From all cases above the induction is closed.
\end{proof}

\section{The Singular Value Decomposition and the Pseudoinverse}\label{sec:6.7}

\begin{thm}[Singular Value Theorem for Linear Transformations]\label{6.26}
  Let \(\V\) and \(\W\) be finite-dimensional inner product spaces over \(\F\), and let \(\T \in \ls(\V, \W)\) be of rank \(r\).
  Then there exist orthonormal bases \(\set{\seq{v}{1,,n}}\) for \(\V\) over \(\F\) and \(\set{\seq{u}{1,,m}}\) for \(\W\) over \(\F\) and positive scalars \(\seq[\geq]{\sigma}{1,,r}\) such that
  \begin{equation}\label{eq:6.7.1}
    \T(v_i) = \begin{dcases}
      \sigma_i u_i & \text{if } i \in \set{1, \dots, r}     \\
      \zv_{\W}     & \text{if } i \in \set{r + 1, \dots, n}
    \end{dcases}.
  \end{equation}
  Conversely, suppose that the preceding conditions are satisfied.
  Then for \(i \in \set{1, \dots, n}\), \(v_i\) is an eigenvector of \(\T^* \T\) with corresponding eigenvalue \(\sigma_i^2\) if \(i \in \set{1, \dots, r}\) and \(0\) if \(i \in \set{r + 1, \dots, n}\).
  Therefore the scalars \(\seq{\sigma}{1,,r}\) are uniquely determined by \(\T\).
\end{thm}

\begin{proof}[\pf{6.26}]
  Let \(\inn{\cdot, \cdot}_{\V}\) and \(\inn{\cdot, \cdot}_{\W}\) be inner products on \(\V\) and \(\W\), respectively.
  For each \(\T \in \ls(\V, \W)\), we denote the adjoint of \(\T\) with respect to \(\inn{\cdot, \cdot}_{\V}\) and \(\inn{\cdot, \cdot}_{\W}\) as \(\T^*\), i.e.,
  \[
    \forall (x, y) \in \V \times \W, \inn{\T(x), y}_{\W} = \inn{x, \T^*(y)}_{\V}.
  \]
  The existence and uniqueness of adjoint operator has been proved by \cref{ex:6.3.15}.

  We first establish the existence of the bases and scalars.
  By \cref{ex:6.4.18} and \cref{ex:6.3.15}(d), \(\T^* \T\) is a positive semidefinite linear operator of rank \(r\) on \(\V\);
  hence there is an orthonormal basis \(\set{\seq{v}{1,,n}}\) for \(\V\) over \(\F\) with respect to \(\inn{\cdot, \cdot}_{\V}\) consisting of eigenvectors of \(\T^* \T\) with corresponding eigenvalues \(\lambda_i\), where \(\seq[\geq]{\lambda}{1,,r} > 0\), and \(\lambda_i = 0\) for \(i \in \set{r + 1, \dots, n}\) (see \cref{ex:6.4.17}(a)).
  For \(i \in \set{1, \dots, r}\), define \(\sigma_i = \sqrt{\lambda_i}\) and \(u_i = \frac{1}{\sigma_i} \T(v_i)\).
  We show that \(\set{\seq{u}{1,,r}}\) is an orthonormal subset of \(\W\) with respect to \(\inn{\cdot, \cdot}_{\W}\).
  Suppose \(i, j \in \set{1, \dots, r}\).
  Then
  \begin{align*}
    \inn{u_i, u_j}_{\W} & = \inn{\frac{1}{\sigma_i} \T(v_i), \frac{1}{\sigma_j} \T(v_j)}_{\W}                        \\
                        & = \frac{1}{\sigma_i} \inn{\T(v_i), \frac{1}{\sigma_j} \T(v_j)}_{\W} &  & \by{6.1.1}[b]     \\
                        & = \frac{1}{\sigma_i \sigma_j} \inn{\T(v_i), \T(v_j)}_{\W}           &  & \by{6.1}[b]       \\
                        & = \frac{1}{\sigma_i \sigma_j} \inn{\T^*(\T(v_i)), v_j}_{\V}         &  & \by{ex:6.3.15}[a] \\
                        & = \frac{1}{\sigma_i \sigma_j} \inn{\lambda_i v_i, v_j}_{\V}         &  & \by{5.1.2}        \\
                        & = \frac{\sigma_i^2}{\sigma_i \sigma_j} \inn{v_i, v_j}_{\V}          &  & \by{6.1.1}[b]     \\
                        & = \frac{\sigma_i^2}{\sigma_i \sigma_j} \delta_{i j}                 &  & \by{6.1.12}       \\
                        & = \delta_{i j},                                                     &  & \by{2.3.4}
  \end{align*}
  and hence \(\set{\seq{u}{1,,r}}\) is orthonormal.
  By \cref{6.7}(a), this set extends to an orthonormal basis \(\set{\seq{u}{1,,r,,m}}\) for \(\W\) over \(\F\).
  Clearly \(\T(v_i) = \sigma_i u_i\) if \(i \in \set{1, \dots, r}\).
  If \(i \in \set{r + 1, \dots, n}\), then \(\T^* \T(v_i) = \zv_{\V}\), and so \(\T(v_i) = \zv_{\W}\) by \cref{ex:6.3.15}(d).

  To establish uniqueness, suppose that \(\set{\seq{v}{1,,n}}\), \(\set{\seq{u}{1,,m}}\), and \(\seq[\geq]{\sigma}{1,,r} > 0\) satisfy the properties stated in the first part of the theorem.
  Then for \(i \in \set{1, \dots, m}\) and \(j \in \set{1, \dots, n}\),
  \begin{align*}
    \inn{\T^*(u_i), v_j}_{\V} & = \inn{u_i, \T(v_j)}_{\W}                                                            &  & \by{ex:6.3.15}[a] \\
                              & = \begin{dcases}
                                    \inn{u_i, \sigma_j u_j}_{\W} & \text{if } j \in \set{1, \dots, r}     \\
                                    \inn{u_i, \zv_{\W}}_{\W}     & \text{if } j \in \set{r + 1, \dots, n}
                                  \end{dcases}                       \\
                              & = \begin{dcases}
                                    \sigma_i & \text{if } i = j \leq r \\
                                    0        & \text{otherwise}
                                  \end{dcases},                                                &  & \by{6.1}[b,c]
  \end{align*}
  and hence by \cref{6.5}, for any \(i \in \set{1, \dots, m}\),
  \begin{equation}\label{eq:6.7.2}
    \T^*(u_i) = \sum_{j = 1}^n \inn{\T^*(u_i), v_j}_{\V} v_j = \begin{dcases}
      \sigma_i v_i & \text{if } i = j \leq r \\
      \zv_{\V}     & \text{otherwise}
    \end{dcases}.
  \end{equation}
  So for \(i \in \set{1, \dots, r}\),
  \[
    \T^* \T(v_i) = \T^*(\sigma_i u_i) = \sigma_i \T^*(u_i) = \sigma_i^2 v_i
  \]
  and \(\T^* \T(v_i) = \T^*(\zv_{\W}) = \zv_{\V}\) for \(i \in \set{r + 1, \dots, n}\).
  Therefore each \(v_i\) is an eigenvector of \(\T^* \T\) with corresponding eigenvalue \(\sigma_i^2\) if \(i \in \set{1, \dots, r}\) and \(0\) if \(i \in \set{r + 1, \dots, n}\).
\end{proof}

\begin{defn}\label{6.7.1}
  The unique scalars \(\seq{\sigma}{1,,r}\) in \cref{6.26} are called the \textbf{singular values} of \(\T\).
  If \(r\) is less than both \(m\) and \(n\), then the term singular value is extended to include \(\seq[=]{\sigma}{r+1,,k} = 0\), where \(k\) is the minimum of \(m\) and \(n\).
\end{defn}

\begin{note}
  Although the singular values of a linear transformation \(\T\) are uniquely determined by \(\T\), the orthonormal bases given in the statement of \cref{6.26} are not uniquely determined because there is more than one orthonormal basis of eigenvectors of \(\T^* \T\).

  In view of \cref{eq:6.7.2}, the singular values of a linear transformation \(\T \in \ls(\V, \W)\) and its adjoint \(\T^* \in \ls(\W, \V)\) are identical.
  Furthermore, the orthonormal bases for \(\V\) and \(\W\) given in \cref{6.26} are simply reversed for \(\T^*\).
\end{note}

\begin{defn}\label{6.7.2}
  Let \(A \in \ms\).
  We define the \textbf{singular values} of \(A\) to be the singular values of the linear transformation \(\L_A\).
\end{defn}

\begin{thm}[Singular Value Decomposition Theorem for Matrices]\label{6.27}
  Let \(A \in \ms\) be of rank \(r\) with the positive singular values \(\seq[\geq]{\sigma}{1,,r}\), and let \(\Sigma \in \ms\) defined by
  \[
    \Sigma_{i j} = \begin{dcases}
      \sigma_i & \text{if } i = j \leq r \\
      0        & \text{otherwise}
    \end{dcases}.
  \]
  Then there exists unitary matrices \(U \in \ms[m][m][\F]\) and \(\V \in \ms[n][n][\F]\) such that
  \[
    A = U \Sigma V^*.
  \]
\end{thm}

\begin{proof}[\pf{6.27}]
  Let \(\T = \L_A \in \ls(\vs{F}^n, \vs{F}^m)\).
  By \cref{6.26}, there exist orthonormal bases \(\beta = \set{\seq{v}{1,,n}}\) for \(\vs{F}^n\) over \(\F\) and \(\gamma = \set{\seq{u}{1,,m}}\) for \(\vs{F}^m\) over \(\F\) such that \(\T(v_i) = \sigma_i u_i\) for \(i \in \set{1, \dots, r}\) and \(\T(v_i) = \zv_{\W}\) for \(i \in \set{r + 1, \dots, n}\).
  Let \(U \in \ms[m][m][\F]\) whose \(j\)th column is \(u_j\) for all \(j \in \set{1, \dots, m}\), and let \(V \in \ms[n][n][\F]\) whose \(j\)th column is \(v_j\) for all \(j \in \set{1, \dots, n}\).
  Note that both \(U\) and \(V\) are unitary matrices.

  By \cref{2.13}(a), the \(j\)th column of \(AV\) is \(A v_j = \sigma_j u_j\).
  Observe that the \(j\)th column of \(\Sigma\) is \(\sigma_j e_j\), where \(e_j\) is the \(j\)th standard vector of \(\vs{F}^m\).
  So by \cref{2.13}(a)(b), the \(j\)th column of \(U \Sigma\) is given by
  \[
    U(\sigma_j e_j) = \sigma_j (U e_j) = \sigma_j u_j.
  \]
  It follows that \(AV\) and \(U \Sigma\) are \(m \times n\) matrices whose corresponding columns are equal, and hence \(AV = U \Sigma\).
  Therefore \(A = A V V^* = U \Sigma V^*\).
\end{proof}

\begin{defn}\label{6.7.3}
  Let \(A \in \ms\) be of rank \(r\) with positive singular values \(\seq[\geq]{\sigma}{1,,r}\).
  A factorization \(A = U \Sigma V^*\) where \(U\) and \(V\) are unitary matrices and \(\Sigma \in \ms\) is defined as in \cref{6.27} is called a \textbf{singular value decomposition} of \(A\).
\end{defn}

\begin{note}
  In the proof of \cref{6.27}, the columns of \(V\) are the vectors in \(\beta\), and the columns of \(U\) are the vectors in \(\gamma\).
  Furthermore, the nonzero singular values of \(A\) are the same as those of \(\L_A\);
  hence they are the square roots of the nonzero eigenvalues of \(A^* A\) or of \(A A^*\).
  (See \cref{ex:6.7.9}.)
\end{note}

\begin{note}
  A singular value decomposition of a matrix can be used to factor a square matrix in a manner analogous to the factoring of a complex number as the product of a complex number of length \(1\) and a nonnegative number.
  In the case of matrices, the complex number of length \(1\) is replaced by a unitary matrix, and the nonnegative number is replaced by a positive semidefinite matrix.
\end{note}

\begin{thm}[Polar Decomposition]\label{6.28}
  For any square matrix \(A\), there exists a unitary matrix \(W\) and a positive semidefinite matrix \(P\) such that
  \[
    A = WP.
  \]
  Furthermore, if \(A\) is invertible, then the representation is unique.
\end{thm}

\begin{proof}[\pf{6.28}]
  By \cref{6.27}, there exist unitary matrices \(U\) and \(V\) and a diagonal matrix \(\Sigma\) with nonnegative diagonal entries such that \(A = U \Sigma V^*\).
  So
  \begin{align*}
    A & = U \Sigma \V^* \by{6.27}                 \\
      & = U I \Sigma \V^*         &  & \by{2.3.4} \\
      & = U V^* V \Sigma V^*      &  & \by{6.5.9} \\
      & = WP,
  \end{align*}
  where \(W = U V^*\) and \(P = V \Sigma V^*\).
  Since \(W\) is the product of unitary matrices, \(W\) is unitary (\cref{ex:6.5.3}), and since \(\Sigma\) is positive semidefinite (\cref{ex:6.4.17}(a)) and \(P\) is unitarily equivalent to \(\Sigma\), \(P\) is positive semidefinite by \cref{ex:6.5.14}.

  Now suppose that \(A\) is invertible and factors as the products
  \[
    A = WP = ZQ,
  \]
  where \(W\) and \(Z\) are unitary and \(P\) and \(Q\) are positive semidefinite.
  Since \(A\) is invertible, it follows that \(P\) and \(Q\) are positive definite and invertible \cref{ex:6.4.19}(c), and therefore \(Z^* W = Q P^{-1}\).
  Thus \(Q P^{-1}\) is unitary (\cref{ex:6.5.3}), and so
  \begin{align*}
    I & = (Q P^{-1})^* (Q P^{-1}) &  & \by{6.5.9}        \\
      & = (P^{-1})^* Q^* Q P^{-1} &  & \by{6.3.2}[c]     \\
      & = (P^{-1})^* Q^2 P^{-1}   &  & \by{6.4.11}       \\
      & = P^{-1} Q^2 P^{-1}.      &  & \by{ex:6.4.19}[c]
  \end{align*}
  Hence \(P^2 = Q^2\).
  Since both \(P\) and \(Q\) are positive definite, it follows that \(P = Q\) by \cref{ex:6.4.17}(d).
  Therefore \(W = Z\), and consequently the factorization is unique.
\end{proof}

\begin{defn}\label{6.7.4}
  The factorization of a square matrix \(A\) as \(WP\) where \(W\) is unitary and \(P\) is positive semidefinite, is called a \textbf{polar decomposition} of \(A\).
\end{defn}

\begin{prop}\label{6.7.5}
  Let \(\V\) and \(\W\) be finite-dimensional inner product spaces over the same field \(\F\), and let \(\T \in \ls(\V, \W)\).
  It is desirable to have a linear transformation from \(\W\) to \(\V\) that captures some of the essence of an inverse of \(\T\) even if \(\T\) is not invertible.
  A simple approach to this problem is to focus on the ``part'' of \(\T\) that is invertible, namely, the restriction of \(\T\) to \(\ns{\T}^{\perp}\).
  Let \(\lt{L} : \ns{\T}^{\perp} \to \rg{\T}\) be the linear transformation defined by \(\lt{L}(x) = \T(x)\) for all \(x \in \ns{\T}^{\perp}\).
  Then \(\lt{L}\) is invertible, and we can use the inverse of \(\lt{L}\) to construct a linear transformation from \(\W\) to \(\V\) that salvages some of the benefits of an inverse of \(\T\).
\end{prop}

\begin{proof}[\pf{6.7.5}]
  Let \(y \in \rg{\T}\).
  Then there exists an \(x \in \V\) such that \(\T(x) = y\).
  By \cref{ex:6.2.13}(d) we have \(\V = \ns{\T} \oplus \ns{\T}^{\perp}\), thus by \cref{6.6} there exists an unique tuple \(\tuple{x}{1,2} \in \ns{\T} \times \ns{\T}^{\perp}\) such that \(x = x_1 + x_2\).
  Then we have
  \begin{align*}
    y & = \T(x)                                 \\
      & = \T(x_1 + x_2)      &  & \by{6.6}      \\
      & = \T(x_1) + \T(x_2)  &  & \by{2.1.2}[a] \\
      & = \zv_{\W} + \T(x_2) &  & \by{2.1.10}   \\
      & = \T(x_2)                               \\
      & = \lt{L}(x_2).
  \end{align*}
  Thus \(\lt{L}\) is onto.
  Since
  \begin{align*}
    \dim(\ns{\T}^{\perp}) & = \dim(\V) - \dim(\ns{\T}) &  & \by{6.7}[c] \\
                          & = \dim(\V) - \nt{\T}       &  & \by{2.1.12} \\
                          & = \rk{\T},                 &  & \by{2.3}
  \end{align*}
  by \cref{2.5} we know that \(\lt{L}\) is one-to-one, and thus invertible.
\end{proof}

\begin{defn}\label{6.7.6}
  Let \(\V\) and \(\W\) be finite-dimensional inner product spaces over the same field \(\F\), and let \(\T \in \ls(\V, \W)\).
  Let \(\lt{L} \in \ls(\ns{\T}^{\perp}, \rg{\T})\) defined by \(\lt{L}(x) = \T(x)\) for all \(x \in \ns{\T}^{\perp}\).
  The \textbf{pseudoinverse} (or Moore-Penrose generalized inverse) of \(\T\), denoted by \(\T^{\dag}\), is defined as the unique linear transformation from \(\W\) to \(\V\) such that
  \[
    \T^{\dag}(y) = \begin{dcases}
      \lt{L}^{-1}(y) & \text{for } y \in \rg{\T}         \\
      \zv_{\V}       & \text{for } y \in \rg{\T}^{\perp}
    \end{dcases}.
  \]
  The uniqueness of pseudoinverse is guaranteed by \cref{2.1.13}.
  The pseudoinverse of a linear transformation \(\T\) on a finite-dimensional inner product space exists even if \(\T\) is not invertible (see \cref{6.7.5}).
  Furthermore, if \(\T\) is invertible, then \(\T^{\dag} = \T^{-1}\) because \(\ns{\T}^{\perp} = \V\) (see \cref{2.4}), and \(\lt{L}\) (as just defined) coincides with \(\T\).
\end{defn}

\begin{eg}\label{6.7.7}
  Consider the zero transformation \(\zT \in \ls(\V, \W)\) between two finite-dimensional inner product spaces \(\V\) and \(\W\) over \(\F\).
  Then \(\rg{\zT} = \set{\zv_{\W}}\), and therefore \(\zT^{\dag}\) is the zero transformation from \(\W\) to \(\V\).
\end{eg}

\begin{prop}\label{6.7.8}
  We can use the singular value theorem to describe the pseudoinverse of a linear transformation.
  Suppose that \(\V\) and \(\W\) are finite-dimensional vector spaces and \(\T \in \ls(\V, \W)\) is of rank \(r\).
  Let \(\set{\seq{v}{1,,n}}\) and \(\set{\seq{u}{1,,m}}\) be orthonormal bases for \(\V\) and \(\W\) over \(\F\), respectively, and let \(\seq[\geq]{\sigma}{1,,r}\) be the nonzero singular values of \(\T\) satisfying \cref{eq:6.7.1} in \cref{6.26}.
  Then \(\set{\seq{v}{1,,r}}\) is a basis for \(\ns{\T}^{\perp}\) over \(\F\), \(\set{\seq{v}{r+1,,n}}\) is a basis for \(\ns{\T}\) over \(\F\), \(\set{\seq{u}{1,,r}}\) is a basis for \(\rg{\T}\) over \(\F\), and \(\set{\seq{u}{r+1,,m}}\) is a basis for \(\rg{\T}^{\perp}\) over \(\F\).
  Let \(\lt{L}\) be the restriction of \(\T\) to \(\ns{\T}^{\perp}\), as in the definition of pseudoinverse (\cref{6.7.6}).
  Then \(\lt{L}^{-1}(u_i) = \frac{1}{\sigma_i} v_i\) for \(i \in \set{1, \dots, r}\).
  Therefore
  \begin{equation}\label{eq:6.7.3}
    \T^{\dag}(u_i) = \begin{dcases}
      \frac{1}{\sigma_i} v_i & \text{if } i \in \set{1, \dots, r}     \\
      \zv_{\V}               & \text{if } i \in \set{r + 1, \dots, m}
    \end{dcases}.
  \end{equation}
\end{prop}

\begin{proof}[\pf{6.7.8}]
  By \cref{6.7}(c) and \cref{ex:6.2.13} we see that \(\set{\seq{v}{1,,r}}\) is a basis for \(\ns{\T}^{\perp}\) over \(\F\), \(\set{\seq{v}{r+1,,n}}\) is a basis for \(\ns{\T}\) over \(\F\), \(\set{\seq{u}{1,,r}}\) is a basis for \(\rg{\T}\) over \(\F\), and \(\set{\seq{u}{r+1,,m}}\) is a basis for \(\rg{\T}^{\perp}\).
  Then we have
  \begin{align*}
             & \forall x \in \ns{\T}^{\perp}, \lt{L}(x) = \T(x)                           &  & \by{6.7.6}                        \\
    \implies & \forall i \in \set{1, \dots, r}, \lt{L}(v_i) = \T(v_i) = \sigma_i u_i      &  & \by{6.26}                         \\
    \implies & \forall i \in \set{1, \dots, r}, \lt{L}^{-1}(u_i) = \frac{1}{\sigma_i} v_i &  & \by{2.17}                         \\
    \implies & \forall i \in \set{1, \dots, m}, \T^{\dag}(u_i) = \begin{dcases}
                                                                   \frac{1}{\sigma_i} v_i & \text{if } i \in \set{1, \dots, r}     \\
                                                                   \zv_{\V}               & \text{if } i \in \set{r + 1, \dots, m}
                                                                 \end{dcases}.         &  & \by{6.7.6}
  \end{align*}
\end{proof}

\begin{defn}\label{6.7.9}
  Let \(A \in \ms\).
  Then there exists a unique \(B \in \ms[n][m][\F]\) such that \((\L_A)^{\dag} : \vs{F}^m \to \vs{F}^n\) is equal to the left-multiplication transformation \(\L_B\) (see \cref{2.20}).
  We call \(B\) the \textbf{pseudoinverse} of \(A\) and denote it by \(B = A^{\dag}\).
  Thus
  \[
    (\L_A)^{\dag} = \L_{A^{\dag}}.
  \]
\end{defn}

\begin{thm}\label{6.29}
  Let \(A \in \ms\) be of rank \(r\) with a singular value decomposition \(A = U \Sigma V^*\) and nonzero singular values \(\seq[\geq]{\sigma}{1,,r}\).
  Let \(\Sigma^{\dag} \in \ms[n][m][\F]\) defined by
  \[
    \Sigma_{i j}^{\dag} = \begin{dcases}
      \frac{1}{\sigma_i} & \text{if } i = j \leq r \\
      0                  & \text{otherwise}
    \end{dcases}.
  \]
  Then \(A^{\dag} = V \Sigma^{\dag} U^*\), and this is a singular value decomposition of \(A^{\dag}\).
\end{thm}

\begin{proof}[\pf{6.29}]
  Let \(\beta\) and \(\gamma\) be the ordered bases whose vectors are the columns of \(V\) and \(U\), respectively.
  Then \(\beta\) and \(\gamma\) are orthonormal bases for \(\vs{F}^n\) and \(\vs{F}^m\) over \(\F\), respectively, and \cref{eq:6.7.1,eq:6.7.3} are satisfied for \(\T = \L_A\).
  Reversing the roles of \(\beta\) and \(\gamma\) in the proof of \cref{6.27}, we obtain the result.
  Note that \(\Sigma^{\dag}\) is actually the pseudoinverse of \(\Sigma\).
\end{proof}

\begin{note}
  Let \(A \in \ms\).
  Then for any \(b \in \vs{F}^m\), the matrix equation \(Ax = b\) is a system of linear equations, and so it either has no solutions, a unique solution, or infinitely many solutions.
  We know that the system has a unique solution for every \(b \in \vs{F}^m\) iff \(A\) is invertible, in which case the solution is given by \(A^{-1} b\).
  Furthermore, if \(A\) is invertible, then \(A^{-1} = A^{\dag}\), and so the solution can be written as \(x = A^{\dag} b\).
  If, on the other hand, \(A\) is not invertible or the system \(Ax = b\) is inconsistent, then \(A^{\dag} b\) still exists.
  We therefore pose the following question:
  In general, how is the vector \(A^{\dag} b\) related to the system of linear equations \(Ax = b\)?
  In order to answer this question, we need the following lemma.
\end{note}

\begin{lem}\label{6.7.10}
  Let \(\V\) and \(\W\) be finite-dimensional inner product spaces over \(\F\), and let \(\T \in \ls(\V, \W)\).
  Then
  \begin{enumerate}
    \item \(\T^{\dag} \T\) is the orthogonal projection of \(\V\) on \(\ns{\T}^{\perp}\).
    \item \(\T \T^{\dag}\) is the orthogonal projection of \(\W\) on \(\rg{\T}\).
  \end{enumerate}
\end{lem}

\begin{proof}[\pf{6.7.10}(a)]
  As in \cref{6.7.6}, we define \(\lt{L} \in \ls(\ns{\T}^{\perp}, \rg{\T})\) by \(\lt{L}(x) = \T(x)\) for all \(x \in \ns{\T}^{\perp}\).
  Observe that
  \begin{align*}
    \forall x \in \ns{\T}^{\perp}, (\T^{\dag} \T)(x) & = \T^{\dag}(\T(x))                       \\
                                                     & = \T^{\dag}(\lt{L}(x))   &  & \by{6.7.6} \\
                                                     & = \lt{L}^{-1}(\lt{L}(x)) &  & \by{6.7.6} \\
                                                     & = x,
  \end{align*}
  and
  \begin{align*}
    \forall x \in \ns{\T}, (\T^{\dag} \T)(x) & = \T^{\dag}(\T(x))                       \\
                                             & = \T^{\dag}(\zv_{\W}) &  & \by{2.1.10}   \\
                                             & = \zv_{\V}.           &  & \by{2.1.2}[a]
  \end{align*}
  Consequently \(\T^{\dag} \T\) is the orthogonal projection of \(\V\) on \(\ns{\T}^{\perp}\) (see \cref{6.6.1}).
\end{proof}

\begin{proof}[\pf{6.7.10}(b)]
  Continue the proof of \cref{6.7.10}(a), we have
  \begin{align*}
    \forall x \in \rg{\T}, (\T \T^{\dag})(x) & = \T(\T^{\dag}(x))                                                 \\
                                             & = \T(\lt{L}^{-1}(x))     &  & \by{6.7.6}                           \\
                                             & = \lt{L}(\lt{L}^{-1}(x)) &  & (\lt{L}^{-1}(x) \in \ns{\T}^{\perp}) \\
                                             & = x
  \end{align*}
  and
  \begin{align*}
    \forall x \in \rg{\T}^{\perp}, (\T \T^{\dag})(x) & = \T(\T^{\dag}(x))                    \\
                                                     & = \T(\zv_{\V})     &  & \by{6.7.6}    \\
                                                     & = \zv_{\W}.        &  & \by{2.1.2}[a]
  \end{align*}
  Thus by \cref{6.6.1} \(\T \T^{\dag}\) is the orthogonal projection of \(\W\) on \(\rg{\T}\).
\end{proof}

\begin{thm}\label{6.30}
  Consider the system of linear equations \(Ax = b\), where \(A \in \ms\) and \(b \in \vs{F}^m\).
  If \(z = A^{\dag} b\), then \(z\) has the following properties.
  \begin{enumerate}
    \item If \(Ax = b\) is consistent, then \(z\) is the unique solution to the system having minimum norm.
          That is, \(z\) is a solution to the system, and if \(y\) is any solution to the system, then \(\norm{z} \leq \norm{y}\) with equality iff \(z = y\).
    \item If \(Ax = b\) is inconsistent, then \(z\) is the unique best approximation to a solution having minimum norm.
          That is, \(\norm{Az - b} \leq \norm{Ay - b}\) for any \(y \in \vs{F}^n\), with equality iff \(Az = Ay\).
          Furthermore, if \(Az = Ay\), then \(\norm{z} \leq \norm{y}\) with equality iff \(z = y\).
  \end{enumerate}
\end{thm}

\begin{proof}[\pf{6.30}(a)]
  Suppose that \(Ax = b\) is consistent, and let \(z = A^{\dag} b\).
  Observe that \(b \in \rg{\L_A}\), and therefore
  \begin{align*}
    Az & = A A^{\dag} b                             \\
       & = \L_A(\L_A^{\dag}(b)) &  & \by{2.3.8}     \\
       & = b.                   &  & \by{6.7.10}[b]
  \end{align*}
  Thus \(z\) is a solution to the system.
  Now suppose that \(y\) is any solution to the system.
  Then
  \begin{align*}
    \L_A^{\dag} \L_A(y) & = A^{\dag} A y &  & \by{2.3.8} \\
                        & = A^{\dag} b   &  & (Ay = b)   \\
                        & = z,
  \end{align*}
  and hence \(z\) is the orthogonal projection of \(y\) on \(\ns{\L_A}^{\perp}\) by \cref{6.7.10}(a).
  Therefore, by \cref{6.2.12}, we have that \(\norm{z} \leq \norm{y}\) with equality iff \(z = y\).
\end{proof}

\begin{proof}[\pf{6.30}(b)]
  Suppose that \(Ax = b\) is inconsistent.
  By \cref{6.7.10}, \(Az = A A^{\dag} b = \L_A \L_A^{\dag}(b) = b\) is the orthogonal projection of \(b\) on \(\rg{\L_A}\);
  therefore, by \cref{6.2.12}, \(Az\) is the vector in \(\rg{\L_A}\) nearest \(b\).
  That is, if \(Ay\) is any other vector in \(\rg{\L_A}\), then \(\norm{Az - b} \leq \norm{Ay - b}\) with equality iff \(Az = Ay\).

  Finally, suppose that \(y\) is any vector in \(\vs{F}^n\) such that \(Az = Ay = c\).
  Then
  \begin{align*}
    A^{\dag} c & = A^{\dag} A z                              \\
               & = A^{\dag} A A^{\dag} b                     \\
               & = A^{\dag} b            &  & \by{ex:6.7.23} \\
               & = z;
  \end{align*}
  hence we may apply \cref{6.30}(a) to the system \(Ax = c\) to conclude that \(\norm{z} \leq \norm{y}\) with equality iff \(z = y\).
\end{proof}

\begin{note}
  The vector \(z = A^{\dag} b\) in \cref{6.30} is the vector \(x_0\) described in \cref{6.12} that arises in the least squares application.
\end{note}

\exercisesection

\setcounter{ex}{8}
\begin{ex}\label{ex:6.7.9}
  Let \(\V\) and \(\W\) be finite-dimensional inner product spaces over \(\F\), and suppose that \(\set{\seq{v}{1,,n}}\) and \(\set{\seq{u}{1,,m}}\) are orthonormal bases for \(\V\) and \(\W\) over \(\F\), respectively.
  Let \(\T \in \ls(\V, \W)\) be of rank \(r\), and suppose that \(\seq[\geq]{\sigma}{1,,r} > 0\) are such that
  \[
    \T(v_i) = \begin{dcases}
      \sigma_i u_i & \text{if } i \in \set{1, \dots, r}     \\
      \zv_{\W}     & \text{if } i \in \set{r + 1, \dots, n}
    \end{dcases}.
  \]
  \begin{enumerate}
    \item Prove that \(\set{\seq{u}{1,,m}}\) is a set of eigenvectors of \(\T \T^*\) with corresponding eigenvalues \(\seq{\lambda}{1,,m} \in \F\), where
          \[
            \lambda_i = \begin{dcases}
              \sigma_i^2 & \text{if } i \in \set{1, \dots, r}     \\
              0          & \text{if } i \in \set{r + 1, \dots, m}
            \end{dcases}.
          \]
    \item Let \(A \in \ms\) with real or complex entries.
          Prove that the nonzero singular values of \(A\) are the positive square roots of the nonzero eigenvalues of \(A A^*\), including repetitions.
    \item Prove that \(\T \T^*\) and \(\T^* \T\) have the same nonzero eigenvalues, including repetitions.
    \item State and prove a result for matrices analogous to (c).
  \end{enumerate}
\end{ex}

\begin{proof}[\pf{ex:6.7.9}(a)]
  By \cref{6.26} we see that \(\set{\seq{v}{1,,n}}\) is consist of eigenvectors of \(\T^* \T\), and \(\seq{\sigma}{1,,r}\) are singular values of \(\T\).
  Since
  \begin{align*}
    \forall i \in \set{1, \dots, m}, (\T \T^*)(u_i) & = \T(\T^*(u_i))                                             \\
                                                    & = \begin{dcases}
                                                          \T(\sigma_i v_i) & \text{if } i \in \set{1, \dots, r}     \\
                                                          \T(\zv_{\V})     & \text{if } i \in \set{r + 1, \dots, m}
                                                        \end{dcases} &  & \by{eq:6.7.2} \\
                                                    & = \begin{dcases}
                                                          \sigma_i \T(v_i) & \text{if } i \in \set{1, \dots, r}     \\
                                                          \zv_{\W}         & \text{if } i \in \set{r + 1, \dots, m}
                                                        \end{dcases} &  & \by{2.1.2}[a,b] \\
                                                    & = \begin{dcases}
                                                          \sigma_i^2 u_i & \text{if } i \in \set{1, \dots, r}     \\
                                                          \zv_{\V}       & \text{if } i \in \set{r + 1, \dots, m}
                                                        \end{dcases}   &  & \by{eq:6.7.1}   \\
                                                    & = \begin{dcases}
                                                          \sigma_i^2 u_i & \text{if } i \in \set{1, \dots, r}     \\
                                                          0 u_i          & \text{if } i \in \set{r + 1, \dots, m}
                                                        \end{dcases}   &  & \by{1.2}[a]   \\
                                                    & = \begin{dcases}
                                                          \lambda_i u_i & \text{if } i \in \set{1, \dots, r}     \\
                                                          0 u_i         & \text{if } i \in \set{r + 1, \dots, m}
                                                        \end{dcases},
  \end{align*}
  by \cref{5.1.2} we know that \(\set{\seq{u}{1,,m}}\) is consist of eigenvectors of \(\T \T^*\).
\end{proof}

\begin{proof}[\pf{ex:6.7.9}(b)]
  Since
  \begin{align*}
    \L_{A A^*} & = \L_A \L_{A^*}  &  & \by{2.15}[e] \\
               & = \L_A (\L_A)^*, &  & \by{6.3.1}
  \end{align*}
  by \cref{5.1.2,6.7.2} and \cref{ex:6.7.9}(a) we see that the nonzero singular values of \(A\) are the positive square roots of the nonzero eigenvalues of \(A A^*\), including repetitions.
\end{proof}

\begin{proof}[\pf{ex:6.7.9}(c)]
  By \cref{6.26} and \cref{ex:6.7.9}(a) we see that this is true.
\end{proof}

\begin{proof}[\pf{ex:6.7.9}(d)]
  We claim that \(A A^*\) and \(A^* A\) have the same nonzero eigenvalues, including repetitions.
  By \cref{6.7.2} and \cref{ex:6.7.9}(c) we see that the claim is true.
\end{proof}

\begin{ex}\label{ex:6.7.10}
  Use \cref{ex:2.5.8} to obtain another proof of \cref{6.27}, the singular value decomposition theorem for matrices.
\end{ex}

\begin{proof}[\pf{ex:6.7.10}]
  Let \(A \in \ms\) be of rank \(r\).
  By \cref{6.26} there exist orthonormal basis (with respect to standard inner product, \cref{6.1.2}) \(\beta = \set{\seq{v}{1,,n}}\) and \(\gamma = \set{\seq{u}{1,,m}}\) for \(\vs{F}^n\) and \(\vs{F}^m\) over \(\F\), respectively, such that
  \[
    \forall i \in \set{1, \dots, n}, \L_A(v_i) = \begin{dcases}
      \sigma_i u_i   & \text{if } i \in \set{1, \dots, r}     \\
      \zv_{\vs{F}^m} & \text{if } i \in \set{r + 1, \dots, n}
    \end{dcases},
  \]
  where \(\seq[\geq]{\sigma}{1,,r}\) are the singular values of \(A\).
  Let \(\beta'\) and \(\gamma'\) be the standard ordered basis for \(\vs{F}^n\) and \(\vs{F}^m\) over \(\F\), respectively.
  Note that \(\beta'\) and \(\gamma'\) are orthonormal with respect to standard inner product (\cref{6.1.2}).
  Then we have
  \begin{align*}
    A & = [\L_A]_{\beta'}^{\gamma'}                                                                          &  & \by{2.15}[a]        \\
      & = [\IT[\vs{F}^m]]_{\gamma}^{\gamma'} [\L_A]_{\beta}^{\gamma} [\IT[\vs{F}^n]]_{\beta'}^{\beta}        &  & \by{ex:2.5.8}       \\
      & = [\IT[\vs{F}^m]]_{\gamma}^{\gamma'} [\L_A]_{\beta}^{\gamma} \pa{[\IT[\vs{F}^n]]_{\beta}^{\beta'}}^* &  & \by{ex:6.1.23}[c]   \\
      & = \begin{pmatrix}
            u_1 & \cdots & u_m
          \end{pmatrix} \begin{pmatrix}
                          \sigma_1 & \cdots & 0        &     \\
                          \vdots   & \ddots & \vdots   & \zm \\
                          0        & \cdots & \sigma_r &     \\
                                   & \zm    &          & \zm
                        \end{pmatrix} \begin{pmatrix}
                                        v_1 & \cdots & v_n
                                      \end{pmatrix}^*.                                                                &  & \by{2.2.4}
  \end{align*}
\end{proof}

\begin{ex}\label{ex:6.7.11}
  This exercise relates the singular values of a well-behaved linear operator or matrix to its eigenvalues.
  \begin{enumerate}
    \item Let \(\T\) be a normal linear operator on an \(n\)-dimensional inner product space \(\V\) over \(\F\) with eigenvalues \(\seq{\lambda}{1,,n} \in \F\).
          Prove that the singular values of \(\T\) are \(\abs{\lambda_1}, \dots, \abs{\lambda_n}\).
    \item State and prove a result for matrices analogous to (a).
  \end{enumerate}
\end{ex}

\begin{proof}[\pf{ex:6.7.11}(a)]
  By \cref{6.26} we see that if \(\lambda_i\) is an eigenvalue of \(\T^* \T\), then \(\sqrt{\lambda_i}\) is a singular value of \(\T\).
  Then we have
  \begin{align*}
             & \forall i \in \set{1, \dots, n}, \lambda_i \text{ is an eigenvalue of } \T                                                             \\
    \implies & \forall i \in \set{1, \dots, n}, \conj{\lambda_i} \text{ is an eigenvalue of } \T^*                                  &  & \by{6.15}[c] \\
    \implies & \forall i \in \set{1, \dots, n}, \abs{\lambda_i}^2 = \conj{\lambda_i} \lambda_i \text{ is an eigenvalue of } \T^* \T &  & \by{6.15}[c] \\
    \implies & \forall i \in \set{1, \dots, n}, \abs{\lambda_i} \text{ is a singular value of } \T.                                 &  & \by{6.26}
  \end{align*}
\end{proof}

\begin{proof}[\pf{ex:6.7.11}(b)]
  Let \(A \in \ms[n][n][\F]\) be normal with eigenvalues \(\seq{\lambda}{1,,n} \in \F\).
  We claim that the singular values of \(A\) are \(\abs{\lambda_1}, \dots, \abs{\lambda_n}\).
  Since \(A\) is normal, by \cref{6.4.4} we see that \(\L_A\) is normal.
  Thus by \cref{5.1.2,6.7.2,ex:6.7.11}(a) we see that the claim is true.
\end{proof}

\begin{ex}\label{ex:6.7.12}
  Let \(A\) be a normal matrix with an orthonormal basis of eigenvectors \(\beta = \set{\seq{v}{1,,n}}\) and corresponding eigenvalues \(\seq{\lambda}{1,,n}\).
  Let \(V \in \ms[n][n][\F]\) whose columns are the vectors in \(\beta\).
  Prove that for each \(i \in \set{1, \dots, n}\) there is a scalar \(\theta_i\) of absolute value \(1\) such that if \(U\) is the \(n \times n\) matrix with \(\theta_i v_i\) as column \(i\) and \(\Sigma\) is the diagonal matrix such that \(\Sigma_{i i} = \abs{\lambda_i}\) for each \(i \in \set{1, \dots, n}\), then \(U \Sigma V^*\) is a singular value decomposition of \(A\).
\end{ex}

\begin{proof}[\pf{ex:6.7.12}]
  Since \(A\) is normal, by \cref{ex:6.7.11}(b) we know that the singular values of \(A\) are \(\abs{\lambda_1}, \dots, \abs{\lambda_n}\).
  Observe that
  \begin{align*}
    \forall i \in \set{1, \dots, n}, \L_A(v_i) & = A v_i                                                                                   &  & \by{2.3.8} \\
                                               & = \lambda_i v_i                                                                           &  & \by{5.1.2} \\
                                               & = \begin{dcases}
                                                     \abs{\lambda_i} \pa{\frac{\lambda_i}{\abs{\lambda_i}} v_i} & \text{if } \lambda_i \neq 0 \\
                                                     \zv                                                        & \text{if } \lambda_i = 0
                                                   \end{dcases} &  & \by{1.2}[a]                \\
                                               & = \begin{dcases}
                                                     \abs{\lambda_i} \pa{\frac{\lambda_i}{\abs{\lambda_i}} v_i} & \text{if } \lambda_i \neq 0 \\
                                                     1 \zv                                                      & \text{if } \lambda_i = 0
                                                   \end{dcases}. &  & \by{1.2}[c]
  \end{align*}
  Now we define \(\theta_i\) as follow:
  \[
    \forall i \in \set{1, \dots, n}, \theta_i = \begin{dcases}
      \frac{\lambda_i}{\abs{\lambda_i}} & \text{if } \lambda_i \neq 0 \\
      1                                 & \text{if } \lambda_i = 0
    \end{dcases}.
  \]
  Clearly \(\abs{\theta_i} = 1\) for all \(i \in \set{1, \dots, n}\).
  If we can show that \(\set{\seq{\theta,v}{1,,n}}\) is orthonormal, then by \cref{6.26} we know that \(U \Sigma V^*\) is a singular value decomposition of \(A\).
  This is true since
  \begin{align*}
    \forall i, j \in \set{1, \dots, n}, \inn{\theta_i v_i, \theta_j v_j} & = \theta_i \inn{v_i, \theta_j v_j}        &  & \by{6.1.1}[b]                                         \\
                                                                         & = \theta_i \conj{\theta_j} \inn{v_i, v_j} &  & \by{6.1}[b]                                           \\
                                                                         & = \theta_i \conj{\theta_j} \delta_{i j}   &  & \by{6.1.12}                                           \\
                                                                         & = \delta_{i j}.                           &  & (\forall i \in \set{1, \dots, n}, \abs{\theta_i} = i)
  \end{align*}
\end{proof}

\begin{ex}\label{ex:6.7.13}
  Prove that if \(A \in \ms[n][n][\F]\) is a positive semidefinite matrix, then the singular values of \(A\) are the same as the eigenvalues of \(A\).
\end{ex}

\begin{proof}[\pf{ex:6.7.13}]
  Since \(A\) is positive semidefinite, by \cref{6.4.11} we know that \(A\) is self-adjoint and by \cref{ex:6.4.17}(a) we know that the eigenvalues of \(A\) are nonnegative real numbers.
  If \(v \in \vs{F}^n\) is an eigenvector corresponding to eigenvalue \(\lambda \in \F\) of \(A\), then
  \begin{align*}
    A^* A v & = A A v       &  & \by{6.4.8} \\
            & = \lambda^2 v &  & \by{5.1.2}
  \end{align*}
  and thus by \cref{6.26} \(\sqrt{\lambda^2} = \lambda\) is a singular value of \(A\).
  We conclude that eigenvalues of \(A\) are the same as the singular values of \(A\).
\end{proof}

\begin{ex}\label{ex:6.7.14}
  Prove that if \(A \in \ms[n][n][\F]\) is a positive definite matrix and \(A = U \Sigma V^*\) is a singular value decomposition of \(A\), then \(U = V\).
\end{ex}

\begin{proof}[\pf{ex:6.7.14}]
  For each \(i \in \set{1, \dots, n}\), let \(u_i\) and \(v_i\) be the \(i\)th column of \(U\) and \(V\), respectively.
  Since \(A\) is positive definite, by \cref{6.4.11} we know that \(A\) is self-adjoint.
  If \(\seq{\lambda}{1,,n} \in \F\) are eigenvalues of \(A\) corresponding to \(\seq{v}{1,,n}\), then by \cref{ex:6.4.17}(a) we know that \(\lambda_i > 0\) for all \(i \in \set{1, \dots, n}\).
  Then we have
  \begin{align*}
    \forall i \in \set{1, \dots, n}, u_i & = \frac{1}{\lambda_i} \L_A(v_i)   &  & \by{ex:6.7.13} \\
                                         & = \frac{\lambda_i}{\lambda_i} v_i &  & \by{5.1.2}     \\
                                         & = v_i.
  \end{align*}
  Thus \(U = V\).
\end{proof}

\begin{ex}\label{ex:6.7.15}
  Let \(A\) be a square matrix with a polar decomposition \(A = WP\).
  \begin{enumerate}
    \item Prove that \(A\) is normal iff \(W P^2 = P^2 W\).
    \item Use (a) to prove that \(A\) is normal iff \(WP = PW\).
  \end{enumerate}
\end{ex}

\begin{proof}[\pf{ex:6.7.15}(a)]
  By \cref{6.7.4} we know that \(W\) is unitary and \(P\) is positive semidefinite.
  Then we have
  \begin{align*}
         & A \text{ is normal}                              \\
    \iff & A A^* = A^* A             &  & \by{6.4.3}        \\
    \iff & WP (WP)^* = (WP)^* WP     &  & \by{6.28}         \\
    \iff & W P P^* W^* = P^* W^* W P &  & \by{6.3.2}[c]     \\
    \iff & W P P^* W^* = P^* P       &  & \by{ex:6.1.23}[c] \\
    \iff & W P^2 W^* = P^2           &  & \by{6.4.11}       \\
    \iff & W P^2 = P^2 W.            &  & \by{ex:6.1.23}[c]
  \end{align*}
\end{proof}

\begin{proof}[\pf{ex:6.7.15}(b)]
  By \cref{6.7.4} we know that \(W\) is unitary and \(P\) is positive semidefinite.
  By \cref{ex:6.5.14} we know that \(W^* P W\) is positive semidefinite.
  Then we have
  \begin{align*}
         & A \text{ is normal}                                        \\
    \iff & W P^2 = P^2 W                       &  & \by{ex:6.7.15}[a] \\
    \iff & P^2 = W^* P^2 W                     &  & \by{ex:6.1.23}[c] \\
    \iff & P^2 = W^* P W W^* P W = (W^* P W)^2 &  & \by{ex:6.1.23}[c] \\
    \iff & P = W^* P W                         &  & \by{ex:6.4.17}[d] \\
    \iff & WP = PW.                            &  & \by{ex:6.1.23}[c]
  \end{align*}
\end{proof}

\begin{ex}\label{ex:6.7.16}
  Let \(A\) be a square matrix.
  Prove an alternate form of the polar decomposition for \(A\):
  There exists a unitary matrix \(W\) and a positive semidefinite matrix \(P\) such that \(A = PW\).
\end{ex}

\begin{proof}[\pf{ex:6.7.16}]
  Let \(U \Sigma V^*\) be a singular value decomposition of \(A\).
  Let \(P = U \Sigma U^*\) and let \(W = W V^*\).
  By \cref{6.27} we know that \(U, V\) are unitary, thus we have
  \begin{align*}
    A & = U \Sigma V^*       &  & \by{6.27}  \\
      & = U \Sigma U^* U V^* &  & \by{6.5.6} \\
      & = PW.
  \end{align*}
  Since \(W\) is the product of unitary matrices, \(W\) is unitary (\cref{ex:6.5.3}), and since \(\Sigma\) is positive semidefinite (\cref{ex:6.4.17}(a)) and \(P\) is unitarily equivalent to \(\Sigma\), \(P\) is positive semidefinite by \cref{ex:6.5.14}.
\end{proof}

\setcounter{ex}{17}
\begin{ex}\label{ex:6.7.18}
  Let \(A \in \ms\).
  Prove the following results.
  \begin{enumerate}
    \item For any \(m \times m\) unitary matrix \(G\), \((GA)^{\dag} = A^{\dag} G^*\).
    \item For any \(n \times n\) unitary matrix \(H\), \((AH)^{\dag} = H^* A^{\dag}\).
  \end{enumerate}
\end{ex}

\begin{proof}[\pf{ex:6.7.18}(a)]
  Let \(U \Sigma V^*\) be a singular value decomposition of \(GA\).
  By \cref{ex:6.1.23}(c) we know that \(A = G^* U \Sigma V^*\).
  Since \(G\) is unitary, by \cref{6.5.9} we know that \(G^*\) is unitary.
  By \cref{6.7.3} we know that \(U\) is unitary, thus by \cref{ex:6.5.3} \(G^* U\) is unitary and by \cref{6.7.3} \((G^* U) \Sigma V^*\) is a singular value decomposition of \(A\).
  Then we have
  \begin{align*}
    (GA)^{\dag} & = (G G^* U \Sigma V^*)^{\dag}                        \\
                & = (U \Sigma V^*)^{\dag}         &  & \by{6.5.9}      \\
                & = V \Sigma^{\dag} U^*           &  & \by{6.29}       \\
                & = V \Sigma^{\dag} U^* G G^*     &  & \by{6.5.9}      \\
                & = V \Sigma^{\dag} (G^* U)^* G^* &  & \by{6.3.2}[c,d] \\
                & = (G^* U \Sigma V^*)^{\dag} G^* &  & \by{6.29}       \\
                & = A^{\dag} G^*.
  \end{align*}
\end{proof}

\begin{proof}[\pf{ex:6.7.18}(b)]
  Let \(U \Sigma V^*\) be a singular value decomposition of \(AH\).
  By \cref{ex:6.1.23}(c) we know that \(A = U \Sigma V^* H^*\).
  Since \(H, V\) are unitary (by \cref{6.7.3}), by \cref{ex:6.5.3} we know that \(HV\) is unitary.
  By \cref{6.3.2}(c) we know that \((HV)^* = V^* H^*\), thus by \cref{6.7.3} \(U \Sigma V^* H^* = U \Sigma (HV)^*\) is a singular value decomposition of \(A\).
  Then we have
  \begin{align*}
    (AH)^{\dag} & = (U \Sigma V^* H^* H)^{\dag}                  \\
                & = (U \Sigma V^*)^{\dag}        &  & \by{6.5.9} \\
                & = V \Sigma^{\dag} U^*          &  & \by{6.29}  \\
                & = H^* H V \Sigma^{\dag} U^*    &  & \by{6.5.9} \\
                & = H^* (U \Sigma (HV)^*)^{\dag} &  & \by{6.29}  \\
                & = H^* A^{\dag}.
  \end{align*}
\end{proof}

\begin{ex}\label{ex:6.7.19}
  Let \(A\) be a matrix with real or complex entries.
  Prove the following results.
  \begin{enumerate}
    \item The nonzero singular values of \(A\) are the same as the nonzero singular values of \(A^*\), which are the same as the nonzero singular values of \(\tp{A}\).
    \item \((A^{\dag})^* = (A^*)^{\dag}\).
    \item \(\tp{(A^{\dag})} = (\tp{A})^{\dag}\).
  \end{enumerate}
\end{ex}

\begin{proof}[\pf{ex:6.7.19}(a)]
  Let \(U \Sigma V^*\) be a singular value decomposition of \(A\).
  Then we have
  \begin{align*}
    A^* & = (U \Sigma V^*)^*  &  & \by{6.27}        \\
        & = V \Sigma^* U^*    &  & \by{ex:6.3.5}[c] \\
        & = V \tp{\Sigma} U^* &  & \by{6.27}
  \end{align*}
  and
  \begin{align*}
    \tp{A} & = \tp{(U \Sigma V^*)}              &  & \by{6.27}  \\
           & = \tp{(V^*)} \tp{\Sigma} \tp{U}    &  & \by{2.3.2} \\
           & = \conj{V} \tp{\Sigma} \conj{U}^*. &  & \by{6.1.5}
  \end{align*}
  Since singular values are nonnegative, we know that the nonzero values of \(\Sigma\) and \(\tp{\Sigma}\) are the same.
  By \cref{6.7.3} we know that \(V \tp{\Sigma} U^*\) is a singular value decomposition of \(A^*\).
  If we can show that the complex conjugate of an unitary matrix is unitary, then by \cref{6.7.3} again we see that \(\conj{V} \tp{\Sigma} \conj{U}^*\) is a singular value decomposition of \(\tp{A}\).
  So let \(B \in \ms[n][n][\F]\) be unitary.
  Then we have
  \begin{align*}
    \conj{B} \, \conj{B}^* & = \conj{B B^*}        &  & \by{6.1.5} \\
                           & = \conj{I}            &  & \by{6.5.9} \\
                           & = I                   &  & \by{6.1.5} \\
                           & = \conj{B^* B}        &  & \by{6.5.9} \\
                           & = \conj{B}^* \conj{B} &  & \by{6.1.5}
  \end{align*}
  and thus by \cref{6.5.9} \(\conj{B}\) is unitary.
\end{proof}

\begin{proof}[\pf{ex:6.7.19}(b)]
  Let \(U \Sigma V^*\) be a singular value decomposition of \(A\).
  Then we have
  \begin{align*}
    (A^{\dag})^* & = (V \Sigma^{\dag} U^*)^*    &  & \by{6.29}        \\
                 & = U (\Sigma^{\dag})^* V^*    &  & \by{ex:6.3.5}[c] \\
                 & = U \tp{(\Sigma^{\dag})} V^* &  & \by{6.29}        \\
                 & = U (\tp{\Sigma})^{\dag} V^* &  & \by{6.29}        \\
                 & = (V \tp{\Sigma} U^*)^{\dag} &  & \by{6.29}        \\
                 & = (V \Sigma^* U^*)^{\dag}    &  & \by{6.27}        \\
                 & = (A^*)^{\dag}.              &  & \by{ex:6.3.5}[c]
  \end{align*}
\end{proof}

\begin{proof}[\pf{ex:6.7.19}(c)]
  Let \(U \Sigma V^*\) be a singular value decomposition of \(A\).
  Then we have
  \begin{align*}
    \tp{(A^{\dag})} & = \tp{(V \Sigma^{\dag} U^*)}                              &  & \by{6.29}     \\
                    & = \tp{(U^*)} \tp{(\Sigma^{\dag})} \tp{V}                  &  & \by{2.3.2}    \\
                    & = \tp{(U^*)} (\tp{\Sigma})^{\dag} \tp{V}                  &  & \by{6.29}     \\
                    & = \tp{(U^*)} (\tp{\Sigma})^{\dag} (\tp{V})^{**}           &  & \by{6.3.2}[d] \\
                    & = \pa{\pa{\tp{V}}^* \tp{\Sigma} \pa{\tp{(U^*)}}^*}^{\dag} &  & \by{6.29}     \\
                    & = \pa{\tp{\pa{V^*}} \tp{\Sigma} \tp{U}}^{\dag}            &  & \by{6.1.5}    \\
                    & = (\tp{A})^{\dag}.                                        &  & \by{2.3.2}
  \end{align*}
\end{proof}

\begin{ex}\label{ex:6.7.20}
  Let \(A \in \ms[n][n][\F]\) such that \(A^2 = \zm\).
  Prove that \((A^{\dag})^2 = \zm\).
\end{ex}

\begin{proof}[\pf{ex:6.7.20}]
  Suppose that \(\rk{A} = r\).
  Let \(U \Sigma V^*\) be a singular value decomposition of \(A\) such that the first \(r\) diagonal entries are positive.
  For each \(i \in \set{1, \dots, n}\), define \(u_i\) and \(v_i\) to be the \(i\)th column of \(U\) and \(V\), respectively.
  Let \(\inn{\cdot, \cdot}\) be the standard inner product as defined in \cref{6.1.2}.
  Since
  \begin{align*}
             & (U \Sigma V^*) (U \Sigma V^*) = A^2 = \zm                               &  & \by{6.27}         \\
    \implies & \zm = U^* \zm V = \Sigma V^* U \Sigma                                   &  & \by{ex:6.1.23}[c] \\
             & = \Sigma^* V^* U \Sigma                                                 &  & \by{ex:6.7.19}[a] \\
             & = (V \Sigma)^* U \Sigma                                                 &  & \by{6.3.2}[c]     \\
    \implies & \forall i, j \in \set{1, \dots, n}, 0 = ((V \Sigma)^* U \Sigma)_{i j}   &  & \by{1.2.8}        \\
             & = \sum_{k = 1}^n ((V \Sigma)^*)_{i k} (U \Sigma)_{k j}                  &  & \by{2.3.1}        \\
             & = \sum_{k = 1}^n (\conj{V \Sigma})_{k i} (U \Sigma)_{k j}               &  & \by{6.1.5}        \\
             & = \inn{\Sigma_{j j} u_j, \Sigma_{i i} v_i}                              &  & \by{6.1.2}        \\
             & = \Sigma_{j j} \inn{u_j, \Sigma_{i i} v_i}                              &  & \by{6.1.1}[b]     \\
             & = \Sigma_{i i} \Sigma_{j j} \inn{u_j, v_i}                              &  & \by{6.1}[b]       \\
    \implies & \forall i, j \in \set{1, \dots, r}, \inn{u_j, v_i} = 0                  &  & \by{6.27}         \\
             & = \inn{v_i, u_j}                                                        &  & \by{6.1.1}[c]     \\
             & = \Sigma_{i i}^{\dag} \Sigma_{j j}^{\dag} \inn{v_i, u_j}                &  & \by{6.29}         \\
             & = \Sigma_{j j}^{\dag} \inn{\Sigma_{i i}^{\dag} v_i, u_j}                &  & \by{6.1.1}[b]     \\
             & = \inn{\Sigma_{i i}^{\dag} v_i, \Sigma_{j j}^{\dag} u_j}                &  & \by{6.1}[b]       \\
             & = \sum_{k = 1}^n (\conj{U \Sigma^{\dag}})_{k j} (V \Sigma^{\dag})_{k i} &  & \by{6.1.2}        \\
             & = \sum_{k = 1}^n ((U \Sigma^{\dag})^*)_{j k} (V \Sigma^{\dag})_{k i}    &  & \by{6.1.5}        \\
             & = ((U \Sigma^{\dag})^* V \Sigma^{\dag})_{j i}                           &  & \by{2.3.1}        \\
             & = ((\Sigma^{\dag})^* U^* V \Sigma^{\dag})_{j i}                         &  & \by{6.3.2}[c]     \\
             & = (\Sigma^{\dag} U^* V \Sigma^{\dag})_{j i},                            &  & \by{6.29}
  \end{align*}
  we have
  \begin{align*}
     & \forall i, j \in \set{1, \dots, n}, (\Sigma^{\dag} U^* V \Sigma^{\dag})_{j i} = \inn{\Sigma_{i i}^{\dag} v_i, \Sigma_{j j}^{\dag} u_j}                  \\
     & = \begin{dcases}
           0                                    & \text{if } i, j \in \set{1, \dots, r}                                       \\
           \inn{0 v_i, \Sigma_{j j}^{\dag} u_j} & \text{if } i \in \set{r + 1, \dots, n} \text{ and } j \in \set{1, \dots, r} \\
           \inn{\Sigma_{i i}^{\dag} v_i, 0 u_j} & \text{if } i \in \set{1, \dots, r} \text{ and } j \in \set{r + 1, \dots, n} \\
           \inn{0 v_i, 0 u_j}                   & \text{if } i, j \in \set{r + 1, \dots, n}
         \end{dcases}              &  & \by{6.29}                                \\
     & = 0.                                                                                                                                   &  & \by{6.1}[c]
  \end{align*}
  Thus
  \begin{align*}
             & \zm = \Sigma^{\dag} U^* V \Sigma^{\dag}                   &  & \by{1.2.8} \\
    \implies & V \Sigma^{\dag} U^* V \Sigma^{\dag} U^* = V \zm U^* = \zm &  & \by{1.2.8} \\
    \implies & (A^{\dag})^2 = \zm.                                       &  & \by{6.29}
  \end{align*}
\end{proof}

\begin{ex}\label{ex:6.7.21}
  Let \(\V\) and \(\W\) be finite-dimensional inner product spaces over \(\F\), and let \(\T \in \ls(\V, \W)\).
  Prove the following results.
  \begin{enumerate}
    \item \(\T \T^{\dag} \T = \T\).
    \item \(\T^{\dag} \T \T^{\dag} = \T^{\dag}\).
    \item Both \(\T^{\dag} \T\) and \(\T \T^{\dag}\) are self-adjoin.
  \end{enumerate}
  The preceding three statements are called the \textbf{Penrose conditions}, and they characterize the pseudoinverse of a linear transformation as shown in \cref{ex:6.7.22}.
\end{ex}

\begin{proof}[\pf{ex:6.7.21}(a)]
  Let \(n = \dim(\V)\), \(m = \dim(\W)\) and \(r = \rk{\T}\).
  By \cref{6.26} there exist orthonormal bases \(\set{\seq{v}{1,,n}}\) and \(\set{\seq{u}{1,,m}}\) for \(\V\) and \(\W\) over \(\F\), respectively, such that
  \[
    \forall i \in \set{1, \dots, n}, \T(v_i) = \begin{dcases}
      \sigma_i u_i & \text{if } i \in \set{1, \dots, r}     \\
      \zv_{\W}     & \text{if } i \in \set{r + 1, \dots, n}
    \end{dcases},
  \]
  where \(\seq{\sigma}{1,,r}\) are positive.
  By \cref{6.7.8} we have
  \[
    \forall i \in \set{1, \dots, m}, \T^{\dag}(u_i) = \begin{dcases}
      \frac{1}{\sigma_i} v_i & \text{if } i \in \set{1, \dots, r}     \\
      \zv_{\V}               & \text{if } i \in \set{r + 1, \dots, m}
    \end{dcases}.
  \]
  Thus we have
  \begin{align*}
    \forall i \in \set{1, \dots, r}, (\T \T^{\dag} \T)(v_i) & = \T(\T^{\dag}(\T(v_i)))                                 \\
                                                            & = \T(\T^{\dag}(\sigma_i u_i))            &  & \by{6.26}  \\
                                                            & = \T(\sigma_i \T^{\dag}(u_i))            &  & \by{6.7.6} \\
                                                            & = \T\pa{\sigma_i \frac{1}{\sigma_i} v_i} &  & \by{6.7.8} \\
                                                            & = \T(v_i)
  \end{align*}
  and
  \begin{align*}
    \forall i \in \set{r + 1, \dots, n}, (\T \T^{\dag} \T)(v_i) & = \T(\T^{\dag}(\T(v_i)))                     \\
                                                                & = \T(\T^{\dag}(\zv_{\W})) &  & \by{6.26}     \\
                                                                & = \T(\zv_{\V})            &  & \by{2.1.2}[a] \\
                                                                & = \zv_{\W}                &  & \by{2.1.2}[a] \\
                                                                & = \T(v_i).                &  & \by{6.26}
  \end{align*}
  Since \(\set{\seq{v}{1,,n}}\) is a basis for \(\V\) over \(\F\), by \cref{2.1.13} this means \(\T \T^{\dag} \T = \T\).
\end{proof}

\begin{proof}[\pf{ex:6.7.21}(b)]
  Continue the proof of \cref{ex:6.7.21}(a), we have
  \begin{align*}
    \forall i \in \set{1, \dots, r}, (\T^{\dag} \T \T^{\dag})(u_i) & = \T^{\dag}(\T(\T^{\dag}(u_i)))                                    \\
                                                                   & = \T^{\dag}\pa{\T\pa{\frac{1}{\sigma_i} v_i}}   &  & \by{6.7.8}    \\
                                                                   & = \T^{\dag}\pa{\frac{1}{\sigma_i} \T(v_i)}      &  & \by{2.1.2}[b] \\
                                                                   & = \T^{\dag}\pa{\frac{1}{\sigma_i} \sigma_i u_i} &  & \by{6.26}     \\
                                                                   & = \T^{\dag}(u_i)
  \end{align*}
  and
  \begin{align*}
    \forall i \in \set{r + 1, \dots, m}, (\T^{\dag} \T \T^{\dag})(u_i) & = \T^{\dag}(\T(\T^{\dag}(u_i)))                    \\
                                                                       & = \T^{\dag}(\T(\zv_{\V}))       &  & \by{6.7.8}    \\
                                                                       & = \T^{\dag}(\zv_{\W})           &  & \by{2.1.2}[a] \\
                                                                       & = \zv_{\V}                      &  & \by{2.1.2}[a] \\
                                                                       & = \T^{\dag}(u_i).               &  & \by{6.7.8}
  \end{align*}
  Since \(\set{\seq{u}{1,,m}}\) is a basis for \(\W\) over \(\F\), by \cref{2.1.13} this means \(\T^{\dag} \T \T^{\dag} = \T^{\dag}\).
\end{proof}

\begin{proof}[\pf{ex:6.7.21}(c)]
  By \cref{6.7.10} we know that \(\T^{\dag} \T\) and \(\T \T^{\dag}\) are orthogonal projections.
  Thus by \cref{6.24} we know that \(\T^{\dag} \T\) and \(\T \T^{\dag}\) are self-adjoint.
\end{proof}

\begin{ex}\label{ex:6.7.22}
  Let \(\V\) and \(\W\) be finite-dimensional inner product spaces over \(\F\).
  Let \(\T \in \ls(\V, \W)\) and \(\U \in \ls(\W, \V)\) such that \(\T \U \T = \T\), \(\U \T \U = \U\), and both \(\U \T\) and \(\T \U\) are self-adjoint.
  Prove that \(\U = \T^{\dag}\).
\end{ex}

\begin{proof}[\pf{ex:6.7.22}]
  Let \(\inn{\cdot, \cdot}_{\V}\) and \(\inn{\cdot, \cdot}_{\W}\) be inner products on \(\V\) and \(\W\) over \(\F\), respectively.
  For each \(\lt{S} \in \ls(\V, \W)\), we denote the adjoin of \(\lt{S}\) with respect to \(\inn{\cdot, \cdot}_{\V}\) and \(\inn{\cdot, \cdot}_{\W}\) as \(\lt{S}^*\).
  Suppose that \(\U \T\) and \(\T \U\) are self-adjoint with respect to \(\inn{\cdot, \cdot}_{\V}\) and \(\inn{\cdot, \cdot}_{\W}\).

  First we claim that \(\T^{\dag} \T = \U \T\).
  Since
  \begin{align*}
    (\U \T)^2 & = \U \T \U \T                             \\
              & = \U \T       &  & \text{(by hypothesis)} \\
              & = (\U \T)^*,  &  & \by{6.4.8}
  \end{align*}
  by \cref{6.24} we know that \(\U \T\) is the orthogonal projection of \(\V\) on \(\rg{\U \T}\).
  Since
  \begin{align*}
    \U \T \T^* & = (\U \T)^* \T^* &  & \by{6.4.8}             \\
               & = \T^* \U^* \T^* &  & \by{ex:6.3.16}[c]      \\
               & = (\T \U \T)^*   &  & \by{ex:6.3.16}[c]      \\
               & = \T^*,          &  & \text{(by hypothesis)}
  \end{align*}
  we know that \(\rg{\T^*} = \rg{\U \T \T^*} \subseteq \rg{\U \T}\).
  Since
  \begin{align*}
    \U \T & = (\U \T)^*  &  & \by{6.4.8}        \\
          & = \T^* \U^*, &  & \by{ex:6.3.16}[c]
  \end{align*}
  we know that \(\rg{\U \T} = \rg{\T^* \U^*} \subseteq \rg{\T^*}\).
  Thus we have \(\rg{\U \T} = \rg{\T^*}\) and \(\U \T\) is the orthogonal projection of \(\V\) on \(\rg{\T^*}\).
  By \cref{6.7.10}(a) and \cref{ex:6.3.12}(b) we know that \(\T^{\dag} \T\) is the orthogonal projection of \(\V\) on \(\ns{\T}^{\perp} = \rg{\T^*}\), thus by \cref{6.6.2} we have \(\U \T = \T^{\dag} \T\).

  Next we claim that \(\T \T^{\dag} = \T \U\).
  Since
  \begin{align*}
    (\T \U)^2 & = \T \U \T \U                             \\
              & = \T \U       &  & \text{(by hypothesis)} \\
              & = (\T \U)^*,  &  & \by{6.4.8}
  \end{align*}
  by \cref{6.24} we know that \(\T \U\) is the orthogonal projection of \(\W\) on \(\rg{\T \U}\).
  Since \(\T \U \T = \T\), we know that \(\rg{\T} = \rg{\T \U \T} \subseteq \rg{\T \U} \subseteq \rg{\T}\).
  Thus we have \(\rg{\T \U} = \rg{\T}\) and \(\T \U\) is the orthogonal projection of \(\W\) on \(\rg{\T}\).
  By \cref{6.7.10}(b) we know that \(\T \T^{\dag}\) is the orthogonal projection of \(\W\) on \(\rg{\T}\), thus by \cref{6.6.2} we have \(\T \U = \T \T^{\dag}\).

  Finally we show that \(\U = \T^{\dag}\).
  This is true since
  \begin{align*}
    \U & = \U \T \U                            &  & \text{(by hypothesis)}        \\
       & = \U \T \T^{\dag} \T \U               &  & \by{ex:6.7.21}[a]             \\
       & = \T^{\dag} \T \T^{\dag} \T \T^{\dag} &  & \text{(from the claim above)} \\
       & = \T^{\dag} \T \T^{\dag}              &  & \by{ex:6.7.21}[b]             \\
       & = \T^{\dag}.                          &  & \by{ex:6.7.21}[b]
  \end{align*}
\end{proof}

\begin{ex}\label{ex:6.7.23}
  Let \(A \in \ms\).
  Prove the following results.
  \begin{enumerate}
    \item \(A A^{\dag} A = A\).
    \item \(A^{\dag} A A^{\dag} = A^{\dag}\).
    \item Both \(A^{\dag} A\) and \(A A^{\dag}\) are self-adjoin.
  \end{enumerate}
\end{ex}

\begin{proof}[\pf{ex:6.7.23}(a)]
  Let \(r = \rk{A}\), let \(U \Sigma V^*\) be a singular value decomposition of \(A\), and let \(\seq{\sigma}{1,,r} \in \F\) be nonzero singular values of \(A\).
  Let \(\zm_1 \in \ms[r][(n-r)][\F], \zm_2 \in \ms[(m-r)][r][\F], \zm_3 \in \ms[(m-r)][(n-r)][\F], \zm_4 \in \ms[(m-r)][(m-r)][\F]\) be zero matrices.
  Since
  \begin{align*}
     & \Sigma \Sigma^{\dag} \Sigma = \begin{pmatrix}
                                       \sigma_1 & \cdots & 0        &       \\
                                       \vdots   & \ddots & \vdots   & \zm_1 \\
                                       0        & \cdots & \sigma_r &       \\
                                                & \zm_2  &          & \zm_3
                                     \end{pmatrix} \begin{pmatrix}
                                                     \sigma_1^{-1} & \cdots     & 0             &            \\
                                                     \vdots        & \ddots     & \vdots        & \tp{\zm_2} \\
                                                     0             & \cdots     & \sigma_r^{-1} &            \\
                                                                   & \tp{\zm_1} &               & \tp{\zm_3}
                                                   \end{pmatrix} \Sigma &  & \by{6.29} \\
     & = \begin{pmatrix}
           I_r   & \tp{\zm_2} \\
           \zm_2 & \zm_4
         \end{pmatrix} \begin{pmatrix}
                         \sigma_1 & \cdots & 0        &       \\
                         \vdots   & \ddots & \vdots   & \zm_1 \\
                         0        & \cdots & \sigma_r &       \\
                                  & \zm_2  &          & \zm_3
                       \end{pmatrix} = \Sigma,                    &  & \by{2.3.1}
  \end{align*}
  we have
  \begin{align*}
    A A^{\dag} A & = U \Sigma V^* V \Sigma^{\dag} U^* U \Sigma V^* &  & \by{6.29}                     \\
                 & = U \Sigma \Sigma^{\dag} \Sigma V^*             &  & \by{ex:6.1.23}[c]             \\
                 & = U \Sigma V^*                                  &  & \text{(from the proof above)} \\
                 & = A.                                            &  & \by{6.27}
  \end{align*}
\end{proof}

\begin{proof}[\pf{ex:6.7.23}(b)]
  Continue the proof of \cref{ex:6.7.23}(a). Let \(\zm_5 \in \ms[(n-r)][(n-r)][\F]\).
  Since
  \begin{align*}
     & \Sigma^{\dag} \Sigma \Sigma^{\dag} = \begin{pmatrix}
                                              \sigma_1^{-1} & \cdots     & 0             &            \\
                                              \vdots        & \ddots     & \vdots        & \tp{\zm_2} \\
                                              0             & \cdots     & \sigma_r^{-1} &            \\
                                                            & \tp{\zm_1} &               & \tp{\zm_3}
                                            \end{pmatrix} \begin{pmatrix}
                                                            \sigma_1 & \cdots & 0        &       \\
                                                            \vdots   & \ddots & \vdots   & \zm_1 \\
                                                            0        & \cdots & \sigma_r &       \\
                                                                     & \zm_2  &          & \zm_3
                                                          \end{pmatrix} \Sigma^{\dag} &  & \by{6.29} \\
     & = \begin{pmatrix}
           I_r        & \zm_1 \\
           \tp{\zm_1} & \zm_5
         \end{pmatrix} \begin{pmatrix}
                         \sigma_1^{-1} & \cdots     & 0             &            \\
                         \vdots        & \ddots     & \vdots        & \tp{\zm_2} \\
                         0             & \cdots     & \sigma_r^{-1} &            \\
                                       & \tp{\zm_1} &               & \tp{\zm_3}
                       \end{pmatrix} = \Sigma^{\dag},
  \end{align*}
  we have
  \begin{align*}
    A^{\dag} A A^{\dag} & = V \Sigma^{\dag} U^* U \Sigma V^* V \Sigma^{\dag} U^* &  & \by{6.29}                     \\
                        & = V \Sigma^{\dag} \Sigma \Sigma^{\dag} U^*             &  & \by{ex:6.1.23}[c]             \\
                        & = V \Sigma^{\dag} U^*                                  &  & \text{(from the proof above)} \\
                        & = A^{\dag}.                                            &  & \by{6.29}
  \end{align*}
\end{proof}

\begin{proof}[\pf{ex:6.7.23}(c)]
  Let \(r = \rk{A}\) and let \(U \Sigma V^*\) be a singular value decomposition of \(A\).
  Then we have
  \begin{align*}
    (A^{\dag} A)^* & = (V \Sigma^{\dag} U^* U \Sigma V^*)^* &  & \by{6.29}         \\
                   & = (V \Sigma^{\dag} \Sigma V^*)^*       &  & \by{ex:6.1.23}[c] \\
                   & = V \Sigma^* (\Sigma^{\dag})^* V^*     &  & \by{ex:6.3.5}[c]  \\
                   & = V (\Sigma^{\dag} \Sigma)^* V^*       &  & \by{ex:6.3.5}[c]  \\
                   & = V \begin{pmatrix}
                           I_r & \zm \\
                           \zm & \zm
                         \end{pmatrix}^* V^*                    &  & \by{6.29}     \\
                   & = V \Sigma^{\dag} \Sigma V^*           &  & \by{6.3.2}[e]     \\
                   & = V \Sigma^{\dag} U^* U \Sigma V^*     &  & \by{ex:6.1.23}[c] \\
                   & = A^{\dag} A                           &  & \by{6.29}
  \end{align*}
  and
  \begin{align*}
    (A A^{\dag})^* & = (U \Sigma V^* V \Sigma^{\dag} U^*)^* &  & \by{6.29}         \\
                   & = (U \Sigma \Sigma^{\dag} U^*)^*       &  & \by{ex:6.1.23}[c] \\
                   & = U (\Sigma^{\dag})^* \Sigma^* U^*     &  & \by{ex:6.3.5}[c]  \\
                   & = U (\Sigma \Sigma^{\dag})^* U^*       &  & \by{ex:6.3.5}[c]  \\
                   & = U \begin{pmatrix}
                           I_r & \zm \\
                           \zm & \zm
                         \end{pmatrix}^* U^*                    &  & \by{6.29}     \\
                   & = U \Sigma \Sigma^{\dag} U^*           &  & \by{6.3.2}[e]     \\
                   & = U \Sigma V^* V \Sigma^{\dag} U^*     &  & \by{ex:6.1.23}[c] \\
                   & = A A^{\dag}.                          &  & \by{6.29}
  \end{align*}
  Thus by \cref{6.4.8} \(A^{\dag} A\) and \(A A^{\dag}\) are self-adjoint.
\end{proof}

\section{Bilinear and Quadratic Forms}\label{sec:6.8}

\begin{defn}\label{6.8.1}
  Let \(\V\) be a vector space over a field \(\F\).
  A function \(H\) from the set \(\V \times \V\) of ordered pairs of vectors to \(\F\) is called a \textbf{bilinear form} on \(\V\) if \(H\) is linear in each variable when the other variable is held fixed;
  that is, \(H\) is a bilinear form on \(\V\) if
  \begin{enumerate}
    \item \(H(a x_1 + x_2, y) = a H(x_1, y) + H(x_2, y)\) for all \(x_1, x_2, y \in \V\) and \(a \in \F\).
    \item \(H(x, a y_1 + y_2) = a H(x, y_1) + H(x, y_2)\) for all \(x, y_1, y_2 \in \V\) and \(a \in \F\).
  \end{enumerate}
  We denote the set of all bilinear forms on \(\V\) by \(\bi(\V)\).
  Observe that an inner product on a vector space is a bilinear form if the underlying field is real, but not if the underlying field is complex (\cref{6.1}(b)).
\end{defn}

\begin{eg}\label{6.8.2}
  Let \(\V = \vs{F}^n\), where the vectors are considered as column vectors.
  For any \(A \in \ms[n][n][\F]\), define \(H : \V \times \V \to \F\) by
  \[
    H(x, y) = \tp{x} A y \quad \text{for } x, y \in \V.
  \]
  Notice that since \(x\) and \(y\) are \(n \times 1\) matrices and \(A \in \ms[n][n][\F]\), \(H(x, y)\) is a \(1 \times 1\) matrix.
  Then \(H \in \bi(\V)\).
\end{eg}

\begin{proof}[\pf{6.8.2}]
  Let \(x, y, z \in \V\) and let \(c \in \F\).
  Since
  \begin{align*}
    H(ax + y, z) & = \tp{(ax + y)} A z         &  & \by{6.8.2}     \\
                 & = (a \tp{x} + \tp{y}) A z   &  & \by{ex:1.3.3}  \\
                 & = a \tp{x} A z + \tp{y} A z &  & \by{2.12}[a,b] \\
                 & = a H(x, z) + H(y, z);      &  & \by{6.8.2}     \\
    H(z, ax + y) & = \tp{z} A (ax + y)         &  & \by{6.8.2}     \\
                 & = a \tp{z} A x + \tp{z} A y &  & \by{2.12}[a,b] \\
                 & = a H(z, x) + H(z, y),      &  & \by{6.8.2}
  \end{align*}
  by \cref{6.8.1} we know that \(H \in \bi(\V)\).
\end{proof}

\begin{prop}\label{6.8.3}
  For any bilinear form \(H\) on a vector space \(\V\) over a field \(\F\), the following properties hold.
  \begin{enumerate}
    \item If, for any \(x \in \V\), the functions \(\lt{L}_x, \lt{R}_x : \V \to \F\) are defined by
          \[
            \lt{L}_x(y) = H(x, y) \quad \text{and} \quad \lt{R}_x(y) = H(y, x) \quad \text{for all } y \in \V,
          \]
          then \(\lt{L}_x, \lt{R}_x \in \ls(\V, \F)\).
    \item \(H(\zv, x) = H(x, \zv) = 0\) for all \(x \in \V\).
    \item For all \(x, y, z, w \in \V\),
          \[
            H(x + y, z + w) = H(x, z) + H(x, w) + H(y, z) + H(y, w).
          \]
    \item If \(J : \V \times \V \to \F\) is defined by \(J(x, y) = H(y, x)\), then \(J\) is a bilinear form.
  \end{enumerate}
\end{prop}

\begin{proof}[\pf{6.8.3}(a)]
  Let \(y, z \in \V\) and let \(c \in \F\).
  Since
  \begin{align*}
    \lt{L}_x(cy + z) & = H(x, cy + z)                 &  & \by{6.8.3}[a] \\
                     & = c H(x, y) + H(x, z)          &  & \by{6.8.1}[b] \\
                     & = c \lt{L}_x(y) + \lt{L}_x(z); &  & \by{6.8.3}[a] \\
    \lt{R}_x(cy + z) & = H(cy + z, x)                 &  & \by{6.8.3}[a] \\
                     & = c H(y, x) + H(z, x)          &  & \by{6.8.1}[a] \\
                     & = c \lt{R}_x(y) + \lt{R}_x(z), &  & \by{6.8.3}[a]
  \end{align*}
  by \cref{2.1.2}(b) we know that \(\lt{L}_x, \lt{R}_x \in \ls(\V, \F)\).
\end{proof}

\begin{proof}[\pf{6.8.3}(b)]
  Since
  \begin{align*}
    H(\zv, x) & = H(\zv + \zv, x)        &  & \by{1.2.1}    \\
              & = H(\zv, x) + H(\zv, x); &  & \by{6.8.1}[a] \\
    H(x, \zv) & = H(x, \zv + \zv)        &  & \by{1.2.1}    \\
              & = H(x, \zv) + H(x, \zv), &  & \by{6.8.1}[b]
  \end{align*}
  by \cref{c.1} we know that
  \[
    H(\zv, x) = 0 = H(x, \zv).
  \]
\end{proof}

\begin{proof}[\pf{6.8.3}(c)]
  We have
  \begin{align*}
    H(x + y, z + w) & = H(x, z + w) + H(y, z + w)              &  & \by{6.8.1}[a] \\
                    & = H(x, z) + H(x, w) + H(y, z) + H(y, w). &  & \by{6.8.1}[b]
  \end{align*}
\end{proof}

\begin{proof}[\pf{6.8.3}(d)]
  Let \(x, y, z \in \V\) and \(c \in \F\).
  Since
  \begin{align*}
    J(cx + y, z) & = H(z, cx + y)         &  & \by{6.8.3}[d] \\
                 & = c H(z, x) + H(z, y)  &  & \by{6.8.1}[b] \\
                 & = c J(x, z) + J(y, z); &  & \by{6.8.3}[d] \\
    J(z, cx + y) & = H(cx + y, z)         &  & \by{6.8.3}[d] \\
                 & = c H(x, z) + H(y, z)  &  & \by{6.8.1}[a] \\
                 & = c J(z, x) + J(z, y), &  & \by{6.8.3}[d]
  \end{align*}
  by \cref{6.8.1} we know that \(J \in \bi(\V)\).
\end{proof}

\begin{defn}\label{6.8.4}
  Let \(\V\) be a vector space over \(\F\), let \(H_1, H_2 \in \bi(\V)\), and let \(a \in \F\).
  We define the \textbf{sum} \(H_1 + H_2\) and the \textbf{scalar product} \(a H_1\) by the equations
  \[
    (H_1 + H_2)(x, y) = H_1(x, y) + H_2(x, y)
  \]
  and
  \[
    (a H_1)(x, y) = a (H_1(x, y)) \quad \text{for all } x, y \in \V.
  \]
\end{defn}

\begin{thm}\label{6.31}
  For any vector space \(\V\) over \(\F\), the sum of two bilinear forms and the product of a scalar and a bilinear form on \(\V\) are again bilinear forms on \(\V\).
  Furthermore, \(\bi(\V)\) is a vector space with respect to these operations.
\end{thm}

\begin{proof}[\pf{6.31}]
  By \cref{1.2.10} we know that the set \(\fs(\V \times \V, \F)\) is a vector space and \(\bi(\V) \subseteq \fs(\V \times \V, \F)\).
  Let \(\zT\) be the zero function of \(\fs(\V \times \V, \F)\).
  Let \(f, g \in \bi(\V)\), let \(x, y, z \in \V\) and let \(a, c \in \F\).
  Since
  \begin{align*}
    (cf + g)(ax + y, z) & = c f(ax + y, z) + g(ax + y, z)                 &  & \by{6.8.4}    \\
                        & = c (a f(x, z) + f(y, z)) + a g(x, z) + g(y, z) &  & \by{6.8.1}[a] \\
                        & = a (c f(x, z) + g(x, z)) + c f(y, z) + g(y, z)                    \\
                        & = a (cf + g)(x, z) + (cf + g)(y, z);            &  & \by{6.8.4}    \\
    (cf + g)(z, ax + y) & = c f(z, ax + y) + g(z, ax + y)                 &  & \by{6.8.4}    \\
                        & = c (a f(z, x) + f(z, y)) + a g(z, x) + g(z, y) &  & \by{6.8.1}[b] \\
                        & = a (c f(z, x) + g(z, x)) + c f(z, y) + g(z, y)                    \\
                        & = a (cf + g)(z, x) + (cf + g)(z, y),            &  & \by{6.8.4}
  \end{align*}
  by \cref{6.8.1} we have \(cf + g \in \bi(\V)\).
  Since
  \begin{align*}
    \zT(ax + y, z) & = 0                        \\
                   & = a 0 + 0                  \\
                   & = a \zT(x, z) + \zT(y, z); \\
    \zT(z, ax + y) & = 0                        \\
                   & = a 0 + 0                  \\
                   & = a \zT(z, x) + \zT(z, y),
  \end{align*}
  by \cref{6.8.1} we have \(\zT \in \bi(\V)\).
  Thus by \cref{ex:1.3.17} we know that \(\bi(\V)\) is a vector space over \(\F\).
\end{proof}

\begin{defn}\label{6.8.5}
  Let \(\beta = \set{\seq{v}{1,,n}}\) be an ordered basis for an \(n\)-dimensional vector space \(\V\) over \(\F\), and let \(H \in \bi(\V)\).
  We can associate \(H\) with an \(n \times n\) matrix \(A\) whose entry in row \(i\) and column \(j\) is defined by
  \[
    A_{i j} = H(v_i, v_j) \quad \text{for } i, j \in \set{1, \dots, n}.
  \]
  The matrix \(A\) above is called the \textbf{matrix representation} of \(H\) with respect to the ordered basis \(\beta\) and is denoted by \(\psi_{\beta}(H)\).
  We can therefore regard \(\psi_{\beta}\) as a mapping from \(\bi(\V)\) to \(\ms[n][n][\F]\), where \(\F\) is the field of scalars for \(\V\), that takes a bilinear form \(H\) into its matrix representation \(\psi_{\beta}(H)\).
\end{defn}

\begin{thm}\label{6.32}
  For any \(n\)-dimensional vector space \(\V\) over \(\F\) and any ordered basis \(\beta\) for \(\V\) over \(\F\), \(\psi_{\beta} : \bi(\V) \to \ms[n][n][\F]\) is an isomorphism.
\end{thm}

\begin{proof}[\pf{6.32}]
  First we show that \(\psi_{\beta} \in \ls(\bi(\V), \ms[n][n][\F])\).
  Let \(f, g \in \bi(\V)\) and let \(c \in \F\).
  Since
  \begin{align*}
    \forall i, j \in \set{1, \dots, n}, \pa{\psi_{\beta}(cf + g)}_{i j} & = (cf + g)(v_i, v_j)                                        &  & \by{6.8.5} \\
                                                                        & = c f(v_i, v_j) + g(v_i, v_j)                               &  & \by{6.8.4} \\
                                                                        & = c \pa{\psi_{\beta}(f)}_{i j} + \pa{\psi_{\beta}(g)}_{i j} &  & \by{6.8.5} \\
                                                                        & = \pa{c \psi_{\beta}(f) + \psi_{\beta}(g)}_{i j},           &  & \by{1.2.9}
  \end{align*}
  by \cref{1.2.8} we know that \(\psi_{\beta}(cf + g) = c \psi_{\beta}(f) + \psi_{\beta}(g)\) and thus by \cref{2.1.2}(b) \(\psi_{\beta} \in \ls(\bi(\V), \ms[n][n][\F])\).

  To show that \(\psi_{\beta}\) is one-to-one, suppose that \(\psi_{\beta}(H) = \zm\) for some \(H \in \bi(\V)\).
  Fix \(v_i \in \beta\), and recall the mapping \(\lt{L}_{v_i} : \V \to \F\), which is linear by \cref{6.8.3}(a).
  By hypothesis, \(\lt{L}_{v_i}(v_j) = H(v_i, v_j) = 0\) for all \(v_j \in \beta\).
  Hence \(\lt{L}_{v_i}\) is the zero transformation from \(\V\) to \(\F\) by \cref{2.1.13}.
  So
  \[
    H(v_i, x) = \lt{L}_{v_i}(x) = 0 \quad \text{for all } x \in \V \text{ and } v_i \in \beta.
  \]
  Next fix an arbitrary \(y \in \V\), and recall the linear mapping \(\lt{R}_y : \V \to \F\) defined in \cref{6.8.3}(a).
  From the equation above we have \(\lt{R}_y(v_i) = H(v_i, y) = 0\) for all \(v_i \in \beta\), and hence \(\lt{R}_y\) is the zero transformation by \cref{2.1.13}.
  So \(H(x, y) = \lt{R}_y(x) = 0\) for all \(x, y \in \V\).
  Thus \(H\) is the zero bilinear form, and therefore \(\psi_{\beta}\) is one-to-one by \cref{2.4}.

  To show that \(\psi_{\beta}\) is onto, consider any \(A \in \ms[n][n][\F]\).
  Recall the isomorphism \(\phi_{\beta} \in \ls(\V, \vs{\F}^n)\) defined in \cref{2.4.11}.
  For \(x \in \V\), we view \(\phi_{\beta}(x) \in \vs{F}^n\) as a column vector.
  Let \(H : \V \times \V \to \F\) be the mapping defined by
  \[
    H(x, y) = \tp{(\phi_{\beta}(x))} A (\phi_{\beta}(y)) \quad \text{for all } x, y \in \V.
  \]
  By \cref{6.8.2} we know that \(H \in \bi(\V)\).
  We show that \(\psi_{\beta}(H) = A\).
  Let \(v_i, v_j \in \beta\).
  Then \(\phi_{\beta}(v_i) = e_i\) and \(\phi_{\beta}(v_j) = e_j\);
  hence, for any \(i, j \in \set{1, \dots, n}\),
  \[
    H(v_i, v_j) = \tp{(\phi_{\beta}(v_i))} A (\phi_{\beta}(v_j)) = \tp{e_i} A e_j = A_{i j}.
  \]
  We conclude that \(\psi_{\beta}(H) = A\) and \(\psi_{\beta}\) is onto.
\end{proof}

\begin{cor}\label{6.8.6}
  For any \(n\)-dimensional vector space \(\V\), \(\bi(\V)\) has dimension \(n^2\).
\end{cor}

\begin{proof}[\pf{6.8.6}]
  By \cref{1.6.11,2.19,6.32} we see that this is true.
\end{proof}

\begin{cor}\label{6.8.7}
  Let \(\V\) be an \(n\)-dimensional vector space over \(\F\) with ordered basis \(\beta\).
  If \(H \in \bi(\V)\) and \(A \in \ms[n][n][\F]\), then \(\psi_{\beta}(H) = A\) iff \(H(x, y) = \tp{(\phi_{\beta}(x))} A (\phi_{\beta}(y))\) for all \(x, y \in \V\).
\end{cor}

\begin{proof}[\pf{6.8.7}]
  Let \(\beta = \set{\seq{v}{1,,n}}\) and let \(x, y \in \V\).
  Since \(\beta\) is a basis for \(\V\) over \(\F\), there exist \(\seq{a}{1,,n}, \seq{b}{1,,n} \in \F\) such that \(x = \sum_{i = 1}^n a_i v_i\) and \(y = \sum_{i = 1}^n b_i v_i\).
  Then we have
  \begin{align*}
         & \psi_{\beta}(H) = A                                                                                                                   \\
    \iff & H(x, y) = H\pa{\sum_{i = 1}^n a_i v_i, \sum_{j = 1}^n b_j v_j} = \sum_{i = 1}^n \sum_{j = 1}^n a_i b_j H(v_i, v_j) &  & \by{6.8.3}[c] \\
         & = \sum_{i = 1}^n \sum_{j = 1}^n a_i b_j A_{i j}                                                                    &  & \by{6.32}     \\
         & = \sum_{i = 1}^n \sum_{j = 1}^n a_i b_j \tp{e_i} A e_j                                                             &  & \by{2.3.1}    \\
         & = \pa{\sum_{i = 1}^n a_i \tp{e_i}} A \pa{\sum_{j = 1}^n b_j e_j}                                                   &  & \by{2.3.5}    \\
         & = \tp{\pa{\sum_{i = 1}^n a_i e_i}} A \pa{\sum_{j = 1}^n b_j e_j}                                                   &  & \by{ex:1.3.3} \\
         & = \tp{(\phi_{\beta}(x))} A (\phi_{\beta}(y)).                                                                      &  & \by{2.4.11}
  \end{align*}
\end{proof}

\begin{cor}\label{6.8.8}
  Let \(\F\) be a field, \(n\) a positive integer, and \(\beta\) be the standard ordered basis for \(\vs{F}^n\) over \(\F\).
  Then for any \(H \in \bi(\vs{F}^n)\), there exists an unique matrix \(A \in \ms[n][n][\F]\), namely, \(A = \psi_{\beta}(H)\), such that
  \[
    H(x, y) = \tp{x} A y \quad \text{for all } x, y \in \vs{F}^n.
  \]
\end{cor}

\begin{proof}[\pf{6.8.8}]
  This is an immediate consequence of \cref{6.8.7}.
\end{proof}

\begin{defn}\label{6.8.9}
  Let \(A, B \in \ms[n][n][\F]\).
  Then \(B\) is said to be \textbf{congruent} to \(A\) if there exists an invertible matrix \(Q \in \ms[n][n][\F]\) such that \(B = \tp{Q} A Q\).
  Observe that the relation of congruence is an equivalence relation (see \cref{ex:6.8.12}).
\end{defn}

\begin{thm}\label{6.33}
  Let \(\V\) be a finite-dimensional vector space over \(\F\) with ordered bases \(\beta = \set{\seq{v}{1,,n}}\) and \(\gamma = \set{\seq{w}{1,,n}}\), and let \(Q\) be the change of coordinate matrix changing \(\gamma\)-coordinates into \(\beta\)-coordinates.
  Then, for any \(H \in \bi(\V)\), we have \(\psi_{\gamma}(H) = \tp{Q} \psi_{\beta}(H) Q\).
  Therefore \(\psi_{\gamma}(H)\) is congruent to \(\psi_{\beta}(H)\).
\end{thm}

\begin{proof}[\pf{6.33}]
  There are essentially two proofs of this theorem.
  One involves a direct computation, while the other follows immediately from a clever observation.
  We give the more direct proof here, leaving the other proof for the exercises (see \cref{ex:6.8.13}).

  Suppose that \(A = \psi_{\beta}(H)\) and \(B = \psi_{\gamma}(H)\).
  Then for \(i, j \in \set{1, \dots, n}\),
  \[
    w_i = \sum_{k = 1}^n Q_{k i} v_k \quad \text{and} \quad w_j = \sum_{r = 1}^n Q_{r j} v_r.
  \]
  (See \cref{2.2.4}.)
  Thus
  \begin{align*}
    B_{i j} & = H(w_i, w_j)                                                  &  & \by{6.8.5}    \\
            & = H\pa{\sum_{k = 1}^n Q_{k i} v_k, w_j}                        &  & \by{2.2.4}    \\
            & = \sum_{k = 1}^n Q_{k i} H(v_k, w_j)                           &  & \by{6.8.1}[a] \\
            & = \sum_{k = 1}^n Q_{k i} H\pa{v_k, \sum_{r = 1}^n Q_{r j} v_r} &  & \by{2.2.4}    \\
            & = \sum_{k = 1}^n Q_{k i} \sum_{r = 1}^n Q_{r j} H(v_k, v_r)    &  & \by{6.8.1}[b] \\
            & = \sum_{k = 1}^n Q_{k i} \sum_{r = 1}^n Q_{r j} A_{k r}        &  & \by{6.8.5}    \\
            & = \sum_{k = 1}^n Q_{k i} \sum_{r=  1}^n A_{k r} Q_{r j}                           \\
            & = \sum_{k = 1}^n Q_{k i} (AQ)_{k j}                            &  & \by{2.3.1}    \\
            & = \sum_{k = 1}^n (\tp{Q})_{i k} (AQ)_{k j}                     &  & \by{1.3.3}    \\
            & = (\tp{Q} A Q)_{i j}.                                          &  & \by{2.3.1}
  \end{align*}
  Hence by \cref{1.2.8} \(B = \tp{Q} A Q\).
\end{proof}

\begin{cor}\label{6.8.10}
  Let \(\V\) be an \(n\)-dimensional vector space over \(\F\) with ordered basis \(\beta\), and let \(H \in \bi(\V)\).
  For any \(B \in \ms[n][n][\F]\), if \(B\) is congruent to \(\psi_{\beta}(H)\), then there exists an ordered basis \(\gamma\) for \(\V\) over \(\F\) such that \(\psi_{\gamma}(H) = B\).
  Furthermore, if \(B = \tp{Q} \psi_{\beta}(H) Q\) for some invertible matrix \(Q\), then \(Q\) changes \(\gamma\)-coordinates into \(\beta\)-coordinates.
\end{cor}

\begin{proof}[\pf{6.8.10}]
  Suppose that \(B = \tp{Q} \psi_{\beta}(H) Q\) for some invertible matrix \(Q\) and that \(\beta = \set{\seq{v}{1,,n}}\).
  Let \(\gamma = \set{\seq{w}{1,,n}}\), where
  \[
    w_j = \sum_{i = 1}^n Q_{i j} v_i \quad \text{for } j \in \set{1, \dots, n}.
  \]
  Since \(Q\) is invertible, \(\gamma\) is an ordered basis for \(\V\) over \(\F\) (\cref{2.2,2.3}), and \(Q\) is the change of coordinate matrix that changes \(\gamma\)-coordinates into \(\beta\)-coordinates (\cref{2.5.1}).
  Therefore, by \cref{6.33},
  \[
    B = \tp{Q} \psi_{\beta}(H) Q = \psi_{\gamma}(H).
  \]
\end{proof}

\begin{note}
  \cref{6.8.10} is the converse of \cref{6.33}.
\end{note}

\begin{note}
  Like the diagonalization problem for linear operators, there is an analogous \emph{diagonalization} problem for bilinear forms, namely, the problem of determining those bilinear forms for which there are diagonal matrix representations.
  As we will see, there is a close relationship between \emph{diagonalizable} bilinear forms and those that are called \emph{symmetric}.
\end{note}

\begin{defn}\label{6.8.11}
  A bilinear form \(H\) on a vector space \(\V\) over \(\F\) is \textbf{symmetric} if \(H(x, y) = H(y, x)\) for all \(x, y \in \V\).
\end{defn}

\begin{thm}\label{6.34}
  Let \(H\) be a bilinear form on a finite-dimensional vector space \(\V\) over \(\F\), and let \(\beta\) be an ordered basis for \(\V\) over \(\F\).
  Then \(H\) is symmetric iff \(\psi_{\beta}(H)\) is symmetric.
\end{thm}

\begin{proof}[\pf{6.34}]
  Let \(\beta = \set{\seq{v}{1,,n}}\) and \(B = \psi_{\beta}(H)\).

  First assume that \(H\) is symmetric.
  Then for \(i, j \in \set{1, \dots, n}\),
  \begin{align*}
    B_{i j} & = H(v_i, v_j) &  & \by{6.8.5}  \\
            & = H(v_j, v_i) &  & \by{6.8.11} \\
            & = B_{j i},    &  & \by{6.8.5}
  \end{align*}
  and it follows that \(B\) is symmetric (\cref{1.3.4}).

  Conversely, suppose that \(B\) is symmetric.
  Let \(J : \V \times \V \to \F\) be the mapping defined by \(J(x, y) = H(y, x)\) for all \(x, y \in \V\).
  By \cref{6.8.3}(d), \(J\) is a bilinear form.
  Let \(C = \psi_{\beta}(J)\).
  Then, for \(i, j \in \set{1, \dots, n}\),
  \begin{align*}
    C_{i j} & = J(v_i, v_j) &  & \by{6.8.5}    \\
            & = H(v_j, v_i) &  & \by{6.8.3}[d] \\
            & = B_{j i}     &  & \by{6.8.5}    \\
            & = B_{i j}.    &  & \by{1.3.4}
  \end{align*}
  Thus \(C = B\) (\cref{1.2.8}).
  Since \(\psi_{\beta}\) is one-to-one (\cref{6.32}), we have \(J = H\).
  Hence \(H(y, x) = J(x, y) = H(x, y)\) for all \(x, y \in \V\), and therefore \(H\) is symmetric (\cref{6.8.11}).
\end{proof}

\begin{defn}\label{6.8.12}
  A bilinear form \(H\) on a finite-dimensional vector space \(\V\) over \(\F\) is called \textbf{diagonalizable} if there is an ordered basis \(\beta\) for \(\V\) over \(\F\) such that \(\psi_{\beta}(H)\) is a diagonal matrix.
\end{defn}

\begin{cor}\label{6.8.13}
  Let \(H\) be a diagonalizable bilinear form on a finite-dimensional vector space \(\V\) over \(\F\).
  Then \(H\) is symmetric.
\end{cor}

\begin{proof}[\pf{6.8.13}]
  Suppose that \(H\) is diagonalizable.
  Then there is an ordered basis \(\beta\) for \(\V\) over \(\F\) such that \(\psi_{\beta}(H) = D\) is a diagonal matrix (\cref{6.8.12}).
  Trivially, \(D\) is a symmetric matrix, and hence, by \cref{6.34}, \(H\) is symmetric.
\end{proof}

\begin{note}
  Unfortunately, the converse of \cref{6.8.13} is not true, as is illustrated by \cref{6.8.14}.
\end{note}

\begin{eg}\label{6.8.14}
  Let \(\F = \Z_2\), \(\V = \vs{F}^2\), and \(H : \V \times \V \to \F\) be the bilinear form defined by
  \[
    H\pa{\begin{pmatrix}
        a_1 \\
        a_2
      \end{pmatrix}, \begin{pmatrix}
        b_1 \\
        b_2
      \end{pmatrix}} = a_1 b_2 + a_2 b_1.
  \]
  Clearly \(H\) is symmetric.
  In fact, if \(\beta\) is the standard ordered basis for \(\V\) over \(\F\), then
  \[
    A = \psi_{\beta}(H) = \begin{pmatrix}
      0 & 1 \\
      1 & 0
    \end{pmatrix},
  \]
  a symmetric matrix.
  We show that \(H\) is not diagonalizable.
\end{eg}

\begin{proof}[\pf{6.8.14}]
  By way of contradiction, suppose that \(H\) is diagonalizable.
  Then there is an ordered basis \(\gamma\) for \(\V\) over \(\F\) such that \(B = \psi_{\gamma}(H)\) is a diagonal matrix (\cref{6.8.12}).
  So by \cref{6.33}, there exists an invertible matrix \(Q\) such that \(B = \tp{Q} A Q\).
  Since \(Q\) is invertible, it follows that \(\rk{B} = \rk{A} = 2\) (\cref{3.4}(c)), and consequently the diagonal entries of \(B\) are nonzero.
  Since the only nonzero scalar of \(\F\) is \(1\),
  \[
    B = \begin{pmatrix}
      1 & 0 \\
      0 & 1
    \end{pmatrix}.
  \]
  Suppose that
  \[
    Q = \begin{pmatrix}
      a & b \\
      c & d
    \end{pmatrix}.
  \]
  Then
  \begin{align*}
    \begin{pmatrix}
      1 & 0 \\
      0 & 1
    \end{pmatrix} & = B                           &  & (\F = \Z_2) \\
                    & = \tp{Q} A Q                  &  & \by{6.33} \\
                    & = \begin{pmatrix}
                          a & c \\
                          b & d
                        \end{pmatrix} \begin{pmatrix}
                                        0 & 1 \\
                                        1 & 0
                                      \end{pmatrix} \begin{pmatrix}
                                                      a & b \\
                                                      c & d
                                                    \end{pmatrix} \\
                    & = \begin{pmatrix}
                          ac + ac & bc + ad \\
                          bc + ad & bd + bd
                        \end{pmatrix}.          &  & \by{2.3.1}
  \end{align*}
  But \(p + p = 0\) for all \(p \in \F\);
  hence \(ac + ac = 0\).
  Thus, comparing the row \(1\), column \(1\) entries of the matrices in the equation above, we conclude that \(1 = 0\), a contradiction.
  Therefore \(H\) is not diagonalizable.
\end{proof}

\begin{note}
  The bilinear form of \cref{6.8.14} is an anomaly.
  Its failure to be diagonalizable is due to the fact that the scalar field \(\Z_2\) is of characteristic two.
  Recall from \cref{c.0.4} that a field \(\F\) is of \textbf{characteristic two} if \(1 + 1 = 0\) in \(\F\).
  If \(\F\) is not of characteristic two, then \(1 + 1 = 2\) has a multiplicative inverse, which we denote by \(1 / 2\).
\end{note}

\begin{lem}\label{6.8.15}
  Let \(H\) be a nonzero symmetric bilinear form on a vector space \(\V\) over a field \(\F\) not of characteristic two.
  Then there is a vector \(x \in \V\) such that \(H(x, x) \neq 0\).
\end{lem}

\begin{proof}[\pf{6.8.15}]
  Since \(H\) is nonzero, we can choose vectors \(u, v \in \V\) such that \(H(u, v) \neq 0\).
  If \(H(u, u) \neq 0\) or \(H(v, v) \neq 0\), there is nothing to prove.
  Otherwise, set \(x = u + v\).
  Then
  \begin{align*}
    H(x, x) & = H(u, u) + H(u, v) + H(v, u) + H(v, v) &  & \by{6.8.3}[c]           \\
            & = 2 H(u, v)                             &  & (H(u, u) = 0 = H(v, v)) \\
            & \neq 0
  \end{align*}
  because \(2 \neq 0\) and \(H(u, v) \neq 0\).
\end{proof}

\begin{thm}\label{6.35}
  Let \(\V\) be a finite-dimensional vector space over a field \(\F\) not of characteristic two.
  Then every symmetric bilinear form on \(\V\) is diagonalizable.
\end{thm}

\begin{proof}[\pf{6.35}]
  We use mathematical induction on \(n = \dim(\V)\).
  If \(n = 1\), then every element of \(\bi(\V)\) is diagonalizable.
  Now suppose that the theorem is valid for vector spaces of dimension less than \(n + 1\) for some fixed integer \(n \geq 1\), and suppose that \(\dim(\V) = n + 1\).
  If \(H\) is the zero bilinear form on \(\V\), then trivially \(H\) is diagonalizable;
  so suppose that \(H\) is a nonzero symmetric bilinear form on \(\V\).
  By \cref{6.8.15}, there exists a nonzero vector \(x \in \V\) such that \(H(x, x) \neq 0\).
  Recall the function \(\lt{L}_x : \V \to \F\) defined by \(\lt{L}_x(y) = H(x, y)\) for all \(y \in \V\).
  By \cref{6.8.3}(a), \(\lt{L}_x \in \ls(\V, \F)\).
  Furthermore, since \(\lt{L}_x(x) = H(x, x) \neq 0\), \(\lt{L}_x\) is nonzero.
  Consequently, \(\rk{\lt{L}_x} = 1\), and hence \(\nt{\lt{L}_x} = n\) (\cref{2.3}).

  The restriction of \(H\) to \(\ns{\lt{L}_x}\) is obviously a symmetric bilinear form on a vector space of dimension \(n\), i.e.,
  \[
    \forall y_1, y_2 \in \ns{\lt{L}_x}, H(y_1, y_2) = H(y_2, y_1).
  \]
  Thus, by the induction hypothesis, there exists an ordered basis \(\set{\seq{v}{1,,n}}\) for \(\ns{\lt{L}_x}\) such that \(H(v_i, v_j) = 0\) for \(i \neq j\) (\(i, j \in \set{1, \dots, n}\)).
  Set \(v_{n + 1} = x\).
  Then \(v_{n + 1} \notin \ns{\lt{L}_x}\), and so \(\beta = \set{\seq{v}{1,,n+1}}\) is an ordered basis for \(\V\) over \(\F\).
  In addition, \(H(v_i, v_{n + 1}) = H(v_{n + 1}, v_i) = 0\) for \(i \in \set{1, \dots, n}\) (\cref{6.8.11}).
  We conclude that \(\psi_{\beta}(H)\) is a diagonal matrix, and therefore \(H\) is diagonalizable (\cref{6.8.12}).
\end{proof}

\begin{cor}\label{6.8.16}
  Let \(\F\) be a field that is not of characteristic two.
  If \(A \in \ms[n][n][\F]\) is a symmetric matrix, then \(A\) is congruent to a diagonal matrix.
\end{cor}

\begin{proof}[\pf{6.8.16}]
  Let \(\beta\) be an ordered basis for \(\vs{F}^n\) over \(\F\).
  By \cref{6.32} there exists an unique \(H \in \bi(\V)\) such that \(\psi_{\beta}(H) = A\).
  Since \(A\) is symmetric, by \cref{6.34} we know that \(H\) is symmetric.
  Since \(\F \neq \Z_2\), by \cref{6.35} we know that \(H\) is diagonalizable.
  By \cref{6.8.12} there exists an ordered basis \(\gamma\) for \(\vs{F}^n\) over \(\F\) such that \(\psi_{\gamma}(H)\) is a diagonal matrix.
  Thus by \cref{6.33} \(A\) is congruent to \(\psi_{\gamma}(H)\).
\end{proof}

\begin{prop}\label{6.8.17}
  Let \(\F\) be a field and \(\F \neq \Z_2\).
  By means of several elementary column operations and the corresponding row operations, \(A \in \ms[n][n][\F]\) can be transformed into a diagonal matrix \(D\).
  Furthermore, if \(\seq{E}{1,,k}\) are the elementary matrices corresponding to these elementary column operations indexed in the order performed, and if \(Q = \seq[]{E}{1,,k}\), then \(\tp{Q} A Q = D\).
\end{prop}

\begin{proof}[\pf{6.8.17}]
  If \(E\) is an elementary \(n \times n\) matrix, then \(AE\) can be obtained by performing an elementary column operation on \(A\) (\cref{3.1}).
  By \cref{ex:6.8.21}, \(\tp{E} A\) can be obtained by performing the same operation on the rows of \(A\) rather than on its columns.
  Thus \(\tp{E} A E\) can be obtained from \(A\) by performing an elementary operation on the columns of \(A\) and then performing the same operation on the rows of \(AE\).
  (Note that the order of the operations can be reversed because of the associative property of matrix multiplication, see \cref{2.16}.)
  Suppose that \(Q\) is an invertible matrix and \(D\) is a diagonal matrix such that \(\tp{Q} A Q = D\) (\cref{6.8.16}).
  By \cref{3.2.6}, \(Q\) is a product of elementary matrices, say \(Q = \seq[]{E}{1,,k}\).
  Thus
  \[
    D = \tp{Q} A Q = \tp{E_k} \cdots \tp{E_1} A \seq{E}{1,,k}.
  \]
\end{proof}

\begin{prop}\label{6.8.18}
  Let \(\F\) be a field and \(\F \neq \Z_2\).
  Let \(A \in \ms[n][n][\F]\) be symmetric.
  Then \((A | I_n)\) can be transformed into \((D | \tp{Q})\) where \(D = \tp{Q} A Q\).
\end{prop}

\begin{proof}[\pf{6.8.18}]
  Since \(A\) is symmetric, by \cref{6.8.16} there exists a \(Q \in \ms[n][n][\F]\) such that \(D = \tp{Q} A Q\) is a diagonal matrix.
  Let \(Q = \seq[]{E}{1,,k}\) where \(\seq{E}{1,,k}\) are elementary column operations.
  By performing \(\seq{E}{1,,k}\) on the first \(n\) column of \((A | I_n)\) we have \((AQ | I_n)\).
  Then by \cref{ex:3.2.15} we have
  \[
    \tp{Q} (AQ | I_n) = (\tp{Q} A Q | \tp{Q}) = (D | \tp{Q}).
  \]
\end{proof}

\begin{defn}\label{6.8.19}
  Let \(\V\) be a vector space over \(\F\).
  A function \(K : \V \to \F\) is called a \textbf{quadratic form} if there exists a symmetric bilinear form \(H \in \bi(\V)\) such that
  \begin{equation}\label{eq:6.8.1}
    K(x) = H(x, x) \quad \text{for all } x \in \V.
  \end{equation}
  If \(\F \neq \Z_2\), then by \cref{6.32} there is a one-to-one correspondence between symmetric bilinear forms and quadratic forms given by \cref{eq:6.8.1}.
  In fact, if \(K\) is a quadratic form on a vector space \(\V\) over a field \(\F\) not of characteristic two, and \(K(x) = H(x, x)\) for some symmetric bilinear form \(H\) on \(\V\), then we can recover \(H\) from \(K\) because
  \begin{equation}\label{eq:6.8.2}
    H(x, y) = \dfrac{1}{2} (K(x + y) - K(x) - K(y))
  \end{equation}
  (see \cref{ex:6.8.16}).
\end{defn}

\begin{eg}\label{6.8.20}
  The classic example of a quadratic form is the homogeneous second-degree polynomial of several variables.
  Given the variables \(\seq{x}{1,,n}\) that take values in a field \(\F\) not of characteristic two and given (not necessarily distinct) scalars \(a_{i j} \in \F\) (\(i, j \in \set{1, \dots, n}\) and \(i \leq j\)), define the polynomial
  \[
    f\tuple{x}{1,,n} = \sum_{i = 1}^n \sum_{j = i}^n a_{i j} x_i x_j.
  \]
  Any such polynomial is a quadratic form.
  In fact, if \(\beta\) is the standard ordered basis for \(\vs{F}^n\) over \(\F\), then the symmetric bilinear form \(H\) corresponding to the quadratic form \(f\) has the matrix representation \(\psi_{\beta}(H) = A\), where
  \[
    A_{i j} = A_{j i} = \begin{dcases}
      a_{i i}              & \text{if } i = j    \\
      \dfrac{1}{2} a_{i j} & \text{if } i \neq j
    \end{dcases}.
  \]
\end{eg}

\begin{proof}[\pf{6.8.20}]
  Let \(A \in \ms[n][n][\F]\) be defined as above.
  Clearly \(A\) is symmetric.
  By \cref{6.32} there exists an unique \(H \in \bi(\vs{F}^n)\) such that \(\psi_{\beta}(H) = A\).
  By \cref{6.8.19} we can define \(K : \V \to \F\) to be the quadratic form associated with \(H\).
  Now we show that \(K = f\).
  This is true since
  \begin{align*}
     & \forall x \in \vs{F}^n, K(x) = H(x, x)                                                                                                 &  & \by{6.8.19} \\
     & = \tp{x} A x                                                                                                                           &  & \by{6.8.8}  \\
     & = \sum_{i = 1}^n x_i (Ax)_{i}                                                                                                          &  & \by{2.3.1}  \\
     & = \sum_{i = 1}^n \sum_{j = 1}^n x_i A_{i j} x_j                                                                                        &  & \by{2.3.1}  \\
     & = \pa{\sum_{i = 1}^n x_i A_{i i} x_i} + \pa{\sum_{i = 1}^n \sum_{j = i + 1}^n x_i A_{i j} x_j + x_j A_{j i} x_i}                                        \\
     & = \pa{\sum_{i = 1}^n x_i a_{i i} x_i} + \pa{\sum_{i = 1}^n \sum_{j = i + 1}^n x_i \dfrac{a_{i j}}{2} x_j + x_j \dfrac{a_{i j}}{2} x_i}                  \\
     & = \pa{\sum_{i = 1}^n a_{i i} x_i x_i} + \pa{\sum_{i = 1}^n \sum_{j = i + 1}^n a_{i j} x_i x_j}                                                          \\
     & = \sum_{i = 1}^n \sum_{j = i}^n a_{i j} x_i x_j                                                                                                         \\
     & = f(x).
  \end{align*}
  Thus \(f\) is a quadratic form.
\end{proof}

\begin{note}
  Since symmetric matrices over \(\R\) are orthogonally diagonalizable (see \cref{6.20}), the theory of symmetric bilinear forms and quadratic forms on finite-dimensional vector spaces over \(\R\) is especially nice.
  The following theorem (\cref{6.36}) and its corollary (\cref{6.8.21}) are useful.
\end{note}

\begin{thm}\label{6.36}
  Let \(\V\) be a finite-dimensional real inner product space, and let \(H\) be a symmetric bilinear form on \(\V\).
  Then there exists an orthonormal basis \(\beta\) for \(\V\) over \(\R\) such that \(\psi_{\beta}(H)\) is a diagonal matrix.
\end{thm}

\begin{proof}[\pf{6.36}]
  Choose any orthonormal basis \(\gamma = \set{\seq{v}{1,,n}}\) for \(\V\) over \(\R\), and let \(A = \psi_{\gamma}(H)\).
  Since \(A\) is symmetric, there exists an orthogonal matrix \(Q\) and a diagonal matrix \(D\) such that \(D = \tp{Q} A Q\) by \cref{6.20}.
  Let \(\beta = \set{\seq{w}{1,,n}}\) be defined by
  \[
    w_j = \sum_{i = 1}^n Q_{i j} v_i \quad \text{for } j \in \set{1, \dots, n}.
  \]
  By \cref{6.33}, \(\psi_{\beta}(H) = D\).
  Furthermore, since \(Q\) is orthogonal and \(\gamma\) is orthonormal, \(\beta\) is orthonormal by \cref{ex:6.5.30}.
\end{proof}

\begin{cor}\label{6.8.21}
  Let \(K\) be a quadratic form on a finite-dimensional real inner product space \(\V\).
  There exists an orthonormal basis \(\beta = \set{\seq{v}{1,,n}}\) for \(\V\) over \(\R\) and scalars \(\seq{\lambda}{1,,n} \in \R\) (not necessarily distinct) such that if \(x \in \V\) and
  \[
    x = \sum_{i = 1}^n s_i v_i, \quad s_i \in \R,
  \]
  then
  \[
    K(x) = \sum_{i = 1}^n \lambda_i s_i^2.
  \]
  In fact, if \(H\) is the symmetric bilinear form determined by \(K\), then \(\beta\) can be chosen to be any orthonormal basis for \(\V\) over \(\R\) such that \(\psi_{\beta}(H)\) is a diagonal matrix.
\end{cor}

\begin{proof}[\pf{6.8.21}]
  Let \(H\) be the symmetric bilinear form for which \(K(x) = H(x, x)\) for all \(x \in \V\).
  By \cref{6.36}, there exists an orthonormal basis \(\beta = \set{\seq{v}{1,,n}}\) for \(\V\) over \(\R\) such that \(\psi_{\beta}(H)\) is the diagonal matrix
  \[
    D = \begin{pmatrix}
      \lambda_1 & 0         & \cdots & 0         \\
      0         & \lambda_2 & \cdots & 0         \\
      \vdots    & \vdots    &        & \vdots    \\
      0         & 0         & \cdots & \lambda_n
    \end{pmatrix}.
  \]
  Let \(x \in \V\), and suppose that \(x = \sum_{i = 1}^n s_i v_i\).
  Then
  \begin{align*}
    K(x) & = H(x, x)                                     &  & \by{6.8.19} \\
         & = \tp{\pa{\phi_{\beta}(x)}} D \phi_{\beta}(x) &  & \by{6.8.7}  \\
         & = \begin{pmatrix}
               s_1 & \cdots & s_n
             \end{pmatrix} D \begin{pmatrix}
                               s_1    \\
                               \vdots \\
                               s_n
                             \end{pmatrix}               &  & \by{2.4.11} \\
         & = \sum_{i = 1}^n \lambda_i s_i^2.             &  & \by{2.3.1}
  \end{align*}
\end{proof}

\begin{defn}\label{6.8.22}
  Let \(z = f\tuple{t}{1,,n}\) be a fixed real-valued function of \(n\) real variables for which all third-order partial derivatives exist and are continuous.
  The function \(f\) is said to have a \textbf{local maximum} at a point \(p \in \R^n\) if there exists a \(\delta > 0\) such that \(f(p) \geq f(x)\) whenever \(\norm{x - p} < \delta\).
  Likewise, \(f\) has a \textbf{local minimum} at \(p \in \R^n\) if there exists a \(\delta > 0\) such that \(f(p) \leq f(x)\) whenever \(\norm{x - p} < \delta\).
  If \(f\) has either a local minimum or a local maximum at \(p\), we say that \(f\) has a \textbf{local extremum} at \(p\).
  A point \(p \in \R^n\) is called a \textbf{critical point} of \(f\) if \(\dfrac{\partial f}{\partial t_i}(p) = 0\) for \(i \in \set{1, \dots, n}\).
  It is a well-known fact that if \(f\) has a local extremum at a point \(p \in \R^n\), then \(p\) is a critical point of \(f\).
  For, if \(f\) has a local extremum at \(p = \tuple{p}{1,,n}\), then for any \(i \in \set{1, \dots, n}\), the function \(\phi_i : \R \to \R\) defined by \(\phi_i(t) = f(\seq{p}{1,,i-1}, t, \seq{p}{i+1,,n})\) has a local extremum at \(t = p_i\).
  So, by an elementary single-variable argument,
  \[
    \dfrac{\partial f}{\partial t_i}(p) = \dfrac{d \phi_i}{dt}(p_i) = 0.
  \]
  Thus \(p\) is a critical point of \(f\).
  But critical points are not necessarily local extrema.

  The second-order partial derivatives of \(f\) at a critical point \(p\) can often be used to test for a local extremum at \(p\).
  These partials determine a matrix \(A(p)\) in which the row \(i\), column \(j\) entry is
  \[
    \dfrac{\partial^2 f}{\partial t_i \partial t_j}(p).
  \]
  This matrix is called the \textbf{Hessian matrix} of \(f\) at \(p\).
  Note that if the third-order partial derivatives of \(f\) are continuous, then the mixed second-order partials of \(f\) at \(p\) are independent of the order in which they are taken, and hence \(A(p)\) is a symmetric matrix.
  In this case, all of the eigenvalues of \(A(p)\) are real.
\end{defn}

\begin{thm}[The Second Derivative Test]\label{6.37}
  Let \(f\tuple{t}{1,,n}\) be a real-valued function in \(n\) real variables for which all third-order partial derivatives exist and are continuous.
  Let \(p = \tuple{p}{1,,n}\) be a critical point of \(f\), and let \(A(p)\) be the Hessian of \(f\) at \(p\).
  \begin{enumerate}
    \item If all eigenvalues of \(A(p)\) are positive, then \(f\) has a local minimum at \(p\).
    \item If all eigenvalues of \(A(p)\) are negative, then \(f\) has a local maximum at \(p\).
    \item If \(A(p)\) has at least one positive and at least one negative eigenvalue, then \(f\) has no local extremum at \(p\)
          (\(p\) is called a \textbf{saddle-point} of \(f\)).
    \item If \(\rk{A(p)} < n\) and \(A(p)\) does not have both positive and negative eigenvalues, then the second derivative test is inconclusive.
  \end{enumerate}
\end{thm}

\begin{proof}[\pf{6.37}]
  If \(p \neq \zv\), we may define a function \(g : \R^n \to \R\) by
  \[
    g\tuple{t}{1,,n} = f(t_1 + p_1, \dots, t_n + p_n) - f(p).
  \]
  The following facts are easily verified.
  \begin{enumerate}[label=(\arabic*)]
    \item The function \(f\) has a local maximum (minimum) at \(p\) iff \(g\) has a local maximum (minimum) at \(\zv = (0, \dots, 0)\).
    \item The partial derivatives of \(g\) at \(\zv\) are equal to the corresponding partial derivatives of \(f\) at \(p\).
    \item \(\zv\) is a critical point of \(g\).
    \item \(A_{i j}(p) = \dfrac{\partial^2 g}{\partial t_i \partial t_j}(\zv)\) for all \(i, j \in \set{1, \dots, n}\).
  \end{enumerate}

  In view of these facts, we may assume without loss of generality that \(p = \zv\) and \(f(p) = 0\).

  Now we apply Taylor's theorem to \(f\) to obtain the first-order approximation of \(f\) around \(\zv\).
  We have
  \begin{equation}\label{eq:6.8.3}
    \begin{aligned}
       & f\tuple{t}{1,,n}                                                                                                                                                   \\
       & = f(\zv) + \sum_{i = 1}^n \dfrac{\partial f}{\partial t_i}(\zv) t_i                                                                                                \\
       & \quad + \dfrac{1}{2} \sum_{i = 1}^n \sum_{j = 1}^n \dfrac{\partial^2 f}{\partial t_i \partial t_j}(\zv) t_i t_j + S\tuple{t}{1,,n}                                 \\
       & = \sum_{i = 1}^n \dfrac{\partial f}{\partial t_i}(\zv) t_i                                                                                                         \\
       & \quad + \dfrac{1}{2} \sum_{i = 1}^n \sum_{j = 1}^n \dfrac{\partial^2 f}{\partial t_i \partial t_j}(\zv) t_i t_j + S\tuple{t}{1,,n} &  & (f(p) = 0)                 \\
       & = \dfrac{1}{2} \sum_{i = 1}^n \sum_{j = 1}^n \dfrac{\partial^2 f}{\partial t_i \partial t_j}(\zv) t_i t_j + S\tuple{t}{1,,n},      &  & \text{(\(p\) is critical)}
    \end{aligned}
  \end{equation}
  where \(S\) is a real-valued function on \(\R^n\) such that
  \begin{equation}\label{eq:6.8.4}
    \Lim_{x \to \zv} \dfrac{S(x)}{\norm{x}^2} = \Lim_{\tuple{t}{1,,n} \to \zv} \dfrac{S\tuple{t}{1,,n}}{t_1^2 + \cdots + t_n^2} = 0.
  \end{equation}
  Let \(K : \R^n \to \R\) be the quadratic form defined by
  \begin{equation}\label{eq:6.8.5}
    K \begin{pmatrix}
      t_1    \\
      \vdots \\
      t_n
    \end{pmatrix} = \dfrac{1}{2} \sum_{i = 1}^n \sum_{j = 1}^n \dfrac{\partial^2 f}{\partial t_i \partial t_j}(\zv) t_i t_j,
  \end{equation}
  let \(H\) be the symmetric bilinear form corresponding to \(K\), and \(\beta\) be the standard ordered basis for \(\R^n\) over \(\R\).
  It is easy to verify that \(\psi_{\beta}(H) = \dfrac{1}{2} A(p)\).
  Since \(A(p)\) is symmetric, \cref{6.20} implies that there exists an orthogonal matrix \(Q\) such that
  \[
    \tp{Q} A(p) Q = \begin{pmatrix}
      \lambda_1 & 0         & \cdots & 0         \\
      0         & \lambda_2 & \cdots & 0         \\
      \vdots    & \vdots    &        & \vdots    \\
      0         & 0         & \cdots & \lambda_n
    \end{pmatrix}
  \]
  is a diagonal matrix whose diagonal entries are the eigenvalues of \(A(p)\).
  Let \(\gamma = \set{\seq{v}{1,,n}}\) be the orthogonal basis for \(\R^n\) whose \(i\)th vector is the \(i\)th column of \(Q\).
  Then \(Q\) is the change of coordinate matrix changing \(\gamma\)-coordinates into \(\beta\)-coordinates, and by \cref{6.33}
  \[
    \psi_{\gamma}(H) = \tp{Q} \psi_{\beta}(H) Q = \dfrac{1}{2} \tp{Q} A(p) Q = \begin{pmatrix}
      \dfrac{\lambda_1}{2} & 0                    & \cdots & 0                    \\
      0                    & \dfrac{\lambda_2}{2} & \cdots & 0                    \\
      \vdots               & \vdots               &        & \vdots               \\
      0                    & 0                    & \cdots & \dfrac{\lambda_n}{2}
    \end{pmatrix}.
  \]

  Suppose that \(A(p)\) is not the zero matrix.
  Then \(A(p)\) has nonzero eigenvalues.
  Choose \(\epsilon > 0\) such that \(\epsilon < \abs{\lambda_i} / 2\) for all \(\lambda_i \neq 0\).
  By \cref{eq:6.8.4}, there exists \(\delta > 0\) such that for any \(x \in \R^n\) satisfying \(0 < \norm{x} < \delta\), we have \(\abs{S(x)} < \epsilon \norm{x}^2\).
  Consider any \(x \in \R^n\) such that \(0 < \norm{x} < \delta\).
  Then, by \cref{eq:6.8.3,eq:6.8.5},
  \[
    \abs{f(x) - K(x)} = \abs{S(x)} < \epsilon \norm{x}^2,
  \]
  and hence
  \begin{equation}\label{eq:6.8.6}
    K(x) - \epsilon \norm{x}^2 < f(x) < K(x) + \epsilon \norm{x}^2.
  \end{equation}
  Suppose that \(x = \sum_{i = 1}^n s_i v_i\).
  Then by \cref{ex:6.1.10,6.8.2} we have
  \[
    \norm{x}^2 = \sum_{i = 1}^n s_i^2 \quad \text{and} \quad K(x) = \dfrac{1}{2} \sum_{i = 1}^n \lambda_i s_i^2.
  \]
  Combining these equations with \cref{eq:6.8.6}, we obtain
  \begin{equation}\label{eq:6.8.7}
    \sum_{i = 1}^n \pa{\dfrac{1}{2} \lambda_i - \epsilon} s_i^2 < f(x) < \sum_{i = 1}^n \pa{\dfrac{1}{2} \lambda_i + \epsilon} s_i^2.
  \end{equation}

  Now suppose that all eigenvalues of \(A(p)\) are positive.
  Then \(\dfrac{1}{2} \lambda_i - \epsilon > 0\) for all \(i \in \set{1, \dots, n}\), and hence, by the left inequality in \cref{eq:6.8.7},
  \[
    f(\zv) = 0 \leq \sum_{i = 1}^n \pa{\dfrac{1}{2} \lambda_i - \epsilon} s_i^2 < f(x).
  \]
  Thus \(f(\zv) \leq f(x)\) for \(\norm{x} < \delta\), and so \(f\) has a local minimum at \(\zv\).
  By a similar argument using the right inequality in \cref{eq:6.8.7}, we have that if all of the eigenvalues of \(A(p)\) are negative, then \(f\) has a local maximum at \(\zv\).
  This establishes (a) and (b) of the theorem.

  Next, suppose that \(A(p)\) has both a positive and a negative eigenvalue, say, \(\lambda_i > 0\) and \(\lambda_j < 0\) for some \(i, j \in \set{1, \dots, n}\).
  Then \(\dfrac{1}{2} \lambda_i - \epsilon > 0\) and \(\dfrac{1}{2} \lambda_j + \epsilon < 0\).
  Let \(s\) be any real number such that \(0 < \abs{s} < \delta\).
  Substituting \(x = s v_i\) and \(x = s v_j\) into the left inequality and the right inequality of \cref{eq:6.8.7}, respectively, we obtain
  \[
    f(\zv) = 0 < \pa{\dfrac{1}{2} \lambda_i - \epsilon} s^2 < f(s v_i) \quad \text{and} \quad f(s v_j) < \pa{\dfrac{1}{2} \lambda_j + \epsilon} s^2 < 0 = f(\zv).
  \]
  Thus \(f\) attains both positive and negative values arbitrarily close to \(\zv\);
  so \(f\) has neither a local maximum nor a local minimum at \(\zv\).
  This establishes (c).

  To show that the second-derivative test is inconclusive under the conditions stated in (d), consider the functions
  \[
    f\tuple{t}{1,2} = t_1^2 - t_2^4 \quad \text{and} \quad g\tuple{t}{1,2} = t_1^2 + t_2^4
  \]
  at \(p = \zv\).
  In both cases, the function has a critical point at \(p\), and
  \[
    A(p) = \begin{pmatrix}
      2 & 0 \\
      0 & 0
    \end{pmatrix}.
  \]
  However, \(f\) does not have a local extremum at \(\zv\), whereas \(g\) has a local minimum at \(\zv\).
\end{proof}

\begin{defn}\label{6.8.23}
  Any two matrix representations of a bilinear form have the same rank because rank is preserved under congruence (\cref{ex:6.8.14}).
  We can therefore define the \textbf{rank} of a bilinear form to be the rank of any of its matrix representations.
  If a matrix representation is a diagonal matrix, then the rank is equal to the number of nonzero diagonal entries of the matrix.
\end{defn}

\begin{thm}[Sylvester's Law of Inertia]\label{6.38}
  Let \(H\) be a symmetric bilinear form on a finite-dimensional real vector space \(\V\).
  Then the number of positive diagonal entries and the number of negative diagonal entries in any diagonal matrix representation of \(H\) are each independent of the diagonal representation.
\end{thm}

\begin{proof}[\pf{6.38}]
  Suppose that \(\beta\) and \(\gamma\) are ordered bases for \(\V\) over \(\R\) that determine diagonal representations of \(H\).
  Without loss of generality, we may assume that \(\beta\) and \(\gamma\) are ordered so that on each diagonal the entries are in the order of positive, negative, and zero.
  It suffices to show that both representations have the same number of positive entries because the number of negative entries is equal to the difference between the rank and the number of positive entries.
  Let \(p\) and \(q\) be the number of positive diagonal entries in the matrix representations of \(H\) with respect to \(\beta\) and \(\gamma\), respectively.
  We suppose that \(p \neq q\) and arrive at a contradiction.
  Without loss of generality, assume that \(p < q\).
  Let
  \[
    \beta = \set{\seq{v}{1,,p,,r,,n}} \quad \text{and} \quad \gamma = \set{\seq{w}{1,,q,,r,,n}},
  \]
  where \(r\) is the rank of \(H\) and \(n = \dim(\V)\).
  Let \(\lt{L} : \V \to \R^{p + r - q}\) be the mapping defined by
  \[
    \lt{L}(x) = (H(x, v_1), \dots, H(x, v_p), H(x, w_{q + 1}), \dots, H(x, w_r)).
  \]
  It is easily verified that \(\lt{L}\) is linear (\cref{6.8.3}(a)) and \(\rk{\lt{L}} \leq p + r - q\).
  Hence
  \[
    \nt{\lt{L}} \geq n - (p + r - q) > n - r.
  \]
  So there exists a nonzero vector \(v_0\) such that \(v_0 \notin \spn{\set{\seq{v}{r+1,,n}}}\), but \(v_0 \in \ns{\lt{L}}\).
  Since \(v_0 \in \ns{\lt{L}}\), it follows that \(H(v_0, v_i) = 0\) for \(i \in \set{1, \dots, p}\) and \(H(v_0, w_i) = 0\) for \(i \in \set{q + 1, \dots, r}\).
  Suppose that
  \[
    v_0 = \sum_{j = 1}^n a_j v_j = \sum_{j = 1}^n b_j w_j.
  \]
  For any \(i \in \set{1, \dots, p}\),
  \begin{align*}
    H(v_0, v_i) & = H\pa{\sum_{j = 1}^n a_j v_j, v_i} &  & \by{1.6.1}    \\
                & = \sum_{j = 1}^n a_j H(v_j, v_i)    &  & \by{6.8.1}[a] \\
                & = a_i H(v_i, v_i).                  &  & \by{6.8.5}
  \end{align*}
  But for \(i \in \set{1, \dots, p}\), we have \(H(v_i, v_i) > 0\) and \(H(v_0, v_i) = 0\), so that \(a_i = 0\).
  Similarly, \(b_i = 0\) for \(i \in \set{q + 1, \dots, r}\).
  Since \(v_0\) is not in the span of \(\set{\seq{v}{r+1,,n}}\), it follows that \(a_i \neq 0\) for some \(i \in \set{p + 1, \dots, r}\).
  Thus
  \begin{align*}
    H(v_0, v_0) & = H\pa{\sum_{j = 1}^n a_j v_j, \sum_{i = 1}^n a_i v_i}                                          \\
                & = \sum_{j = 1}^n \sum_{i = 1}^n a_j a_i H(v_j, v_i)    &  & \by{6.8.1}                          \\
                & = \sum_{j = 1}^n a_j^2 H(v_j, v_j)                     &  & \by{6.8.5}                          \\
                & = \sum_{j = p + 1}^r a_j^2 H(v_j, v_j)                                                          \\
                & < 0.                                                   &  & \text{(by the definition of \(p\))}
  \end{align*}
  Furthermore,
  \begin{align*}
    H(v_0, v_0) & = H\pa{\sum_{j = 1}^n b_j w_j, \sum_{i = 1}^n b_i w_i}                 \\
                & = \sum_{j = 1}^n \sum_{i = 1}^n b_j b_i H(w_j, w_i)    &  & \by{6.8.1} \\
                & = \sum_{j = 1}^n b_j^2 H(w_j, w_j)                     &  & \by{6.8.5} \\
                & = \sum_{j = p + 1}^r b_j^2 H(w_j, w_j)                                 \\
                & \geq 0.                                                &  & (p < q)
  \end{align*}
  So \(H(v_0, v_0) < 0\) and \(H(v_0, v_0) \geq 0\), which is a contradiction.
  We conclude that \(p = q\).
\end{proof}

\begin{defn}\label{6.8.24}
  The number of positive diagonal entries in a diagonal representation of a symmetric bilinear form on a real vector space is called the \textbf{index} of the form.
  The difference between the number of positive and the number of negative diagonal entries in a diagonal representation of a symmetric bilinear form is called the \textbf{signature} of the form.
  The three terms \emph{rank}, \emph{index}, and \emph{signature} are called the \textbf{invariants} of the bilinear form because they are invariant with respect to matrix representations.
  These same terms apply to the associated quadratic form.
  Notice that the values of any two of these invariants determine the value of the third.
\end{defn}

\begin{eg}\label{6.8.25}
  The matrix representation of the bilinear form corresponding to the quadratic form \(K(x, y) = x^2 - y^2\) on \(\R^2\) with respect to the standard ordered basis is the diagonal matrix with diagonal entries of \(1\) and \(-1\).
  Therefore the rank of \(K\) is \(2\), the index of \(K\) is \(1\), and the signature of \(K\) is \(0\).
\end{eg}

\begin{proof}[\pf{6.8.25}]
  Let \(\beta = \set{\seq{e}{1,2}}\) and let \(H \in \bi(\R^2)\) such that
  \[
    \forall u, v \in \R^2, H(u, v) = \dfrac{1}{2} (K(u + v) - K(u) - K(v)).
  \]
  By \cref{6.8.19}, \(H\) is the unique bilinear form associated with \(K\).
  Since
  \begin{align*}
    H(e_1, e_1) & = \dfrac{1}{2} (K(2 e_1) - K(e_1) - K(e_1))                         \\
                & = \dfrac{1}{2} (2^2 - 0^2 - 1^2 + 0^2 - 1^2 + 0^2)                  \\
                & = 1;                                                                \\
    H(e_1, e_2) & = H(e_2, e_1)                                      &  & \by{6.8.19} \\
                & = \dfrac{1}{2} (K(e_1 + e_2) - K(e_1) - K(e_2))                     \\
                & = \dfrac{1}{2} (1^2 - 1^2 - 1^2 + 0^2 - 0^2 + 1^2)                  \\
                & = 0;                                                                \\
    H(e_2, e_2) & = \dfrac{1}{2} (K(2 e_2) - K(e_2) - K(e_2))                         \\
                & = \dfrac{1}{2} (0^2 - 2^2 - 0^2 + 1^2 - 0^2 + 1^2)                  \\
                & = -1,
  \end{align*}
  by \cref{6.8.5} we have
  \[
    \psi_{\beta}(H) = \begin{pmatrix}
      1 & 0  \\
      0 & -1
    \end{pmatrix}.
  \]
\end{proof}

\begin{cor}[Sylvester's Law of Inertia for Matrices]\label{6.8.26}
  Let \(A\) be a real symmetric matrix.
  Then the number of positive diagonal entries and the number of negative diagonal entries in any diagonal matrix congruent to \(A\) is independent of the choice of the diagonal matrix.
\end{cor}

\begin{proof}[\pf{6.8.26}]
  Let \(A \in \ms[n][n][\R]\) be symmetric, and suppose that \(D\) and \(E\) are each diagonal matrices congruent to \(A\).
  By \cref{6.8.8}, \(A\) is the matrix representation of the bilinear form \(H\) on \(\R^n\) defined by \(H(x, y) = \tp{x} A y\) with respect to the standard ordered basis for \(\R^n\).
  Therefore Sylvester's law of inertia (\cref{6.38}) tells us that \(D\) and \(E\) have the same number of positive and negative diagonal entries.
\end{proof}

\begin{defn}\label{6.8.27}
  Let \(A\) be a real symmetric matrix, and let \(D\) be a diagonal matrix that is congruent to \(A\).
  The number of positive diagonal entries of \(D\) is called the \textbf{index} of \(A\).
  The difference between the number of positive diagonal entries and the number of negative diagonal entries of \(D\) is called the \textbf{signature} of \(A\).
  As before, the rank, index, and signature of a matrix are called the \textbf{invariants} of the matrix, and the values of any two of these invariants determine the value of the third.
\end{defn}

\begin{cor}\label{6.8.28}
  Two real symmetric \(n \times n\) matrices are congruent iff they have the same invariants.
\end{cor}

\begin{proof}[\pf{6.8.28}]
  If \(A\) and \(B\) are congruent \(n \times n\) real symmetric matrices, then by \cref{6.33,6.35} they are both congruent to the same diagonal matrix, and it follows from \cref{6.8.26} that they have the same invariants.

  Conversely, suppose that \(A\) and \(B\) are \(n \times n\) symmetric matrices with the same invariants.
  Let \(D\) and \(E\) be diagonal matrices congruent to \(A\) and \(B\), respectively, chosen so that the diagonal entries are in the order of positive, negative, and zero.
  (\cref{ex:6.8.23} allows us to do this.)
  Since \(A\) and \(B\) have the same invariants, so do \(D\) and \(E\).
  Let \(p\) and \(r\) denote the index and the rank, respectively, of both \(D\) and \(E\).
  Let \(d_i\) denote the \(i\)th diagonal entry of \(D\), and let \(Q\) be the \(n \times n\) diagonal matrix whose \(i\)th diagonal entry \(q_i\) is given by
  \[
    q_i = \begin{dcases}
      \dfrac{1}{\sqrt{d_i}}  & \text{if } i \in \set{1, \dots, p}     \\
      \dfrac{1}{\sqrt{-d_i}} & \text{if } i \in \set{p + 1, \dots, r} \\
      1                      & \text{if } i \in \set{r + 1, \dots, n}
    \end{dcases}.
  \]
  Then \(\tp{Q} D Q = J_{pr}\), where
  \[
    J_{pr} = \begin{pmatrix}
      I_p & \zm         & \zm \\
      \zm & - I_{r - p} & \zm \\
      \zm & \zm         & \zm
    \end{pmatrix}.
  \]
  It follows that \(A\) is congruent to \(J_{pr}\).
  Similarly, \(B\) is congruent to \(J_{pr}\), and hence \(A\) is congruent to \(B\).
\end{proof}

\begin{note}
  By \cref{6.8.27,6.8.28}, any two of the invariants can be used to determine an equivalence class of congruent real symmetric matrices.

  The matrix \(J_{pr}\) in \cref{6.28} acts as a \emph{canonical form} for the theory of real symmetric matrices.
  \cref{6.8.29} describes the role of \(J_{pr}\).
\end{note}

\begin{cor}\label{6.8.29}
  A real symmetric \(n \times n\) matrix \(A\) has index \(p\) and rank \(r\) iff \(A\) is congruent to \(J_{pr}\) (as defined in \cref{6.8.28}).
\end{cor}

\begin{proof}[\pf{6.8.29}]
  See the proof of \cref{6.8.28}.
\end{proof}

\exercisesection

\setcounter{ex}{5}
\begin{ex}\label{ex:6.8.6}
  Let \(H : \R^2 \to \R\) be the function defined by
  \[
    H\pa{\begin{pmatrix}
        a_1 \\
        a_2
      \end{pmatrix}, \begin{pmatrix}
        b_1 \\
        b_2
      \end{pmatrix}} = a_1 b_2 + a_2 b_1 \quad \text{for } \begin{pmatrix}
      a_1 \\
      a_2
    \end{pmatrix}, \begin{pmatrix}
      b_1 \\
      b_2
    \end{pmatrix} \in \R^2.
  \]
  \begin{enumerate}
    \item Prove that \(H\) is a bilinear form.
    \item Find the \(2 \times 2\) matrix \(A\) such that \(H(x, y) = \tp{x} A y\) for all \(x, y \in \R^2\).
  \end{enumerate}
  For a \(2 \times 2\) matrix \(M\) with columns \(x\) and \(y\), the bilinear form \(H(M) = H(x, y)\) is called the \textbf{permanent} of \(M\).
\end{ex}

\begin{proof}[\pf{ex:6.8.6}(a)]
  Let \(x, y, z \in \R^2\) and let \(c \in \R\).
  Since
  \begin{align*}
    H(cx + y, z) & = (cx + y)_1 z_2 + (cx + y)_2 z_1           &  & \by{ex:6.8.6} \\
                 & = c x_1 z_2 + y_1 z_2 + c x_2 z_1 + y_2 z_1 &  & \by{1.2.5}    \\
                 & = c (x_1 z_2 + x_2 z_1) + y_1 z_2 + y_2 z_1                    \\
                 & = c H(x, z) + H(y, z);                      &  & \by{ex:6.8.6} \\
    H(z, cx + y) & = z_1 (cx + y)_2 + z_2 (cx + y)_1           &  & \by{ex:6.8.6} \\
                 & = z_1 c x_2 + z_1 y_2 + z_2 c x_1 + z_2 y_1 &  & \by{1.2.5}    \\
                 & = c (z_1 x_2 + z_2 x_1) + z_1 y_2 + z_2 y_1                    \\
                 & = c H(z, x) + H(z, y),                      &  & \by{ex:6.8.6}
  \end{align*}
  by \cref{6.8.1} we know that \(H \in \bi(\R^2)\).
\end{proof}

\begin{proof}[\pf{ex:6.8.6}(b)]
  By \cref{6.8.8} we see that
  \[
    A = \begin{pmatrix}
      H(e_1, e_1) & H(e_1, e_2) \\
      H(e_2, e_1) & H(e_2, e_2)
    \end{pmatrix} = \begin{pmatrix}
      0 & 1 \\
      1 & 0
    \end{pmatrix}.
  \]
\end{proof}

\begin{ex}\label{ex:6.8.7}
  Let \(\V\) and \(\W\) be vector spaces over the same field \(\F\), and let \(\T \in \ls(\V, \W)\).
  For any \(H \in \bi(\W)\), define \(\widehat{\T}(H) : \V \times \V \to \F\) by \(\widehat{\T}(H)(x, y) = H(\T(x), \T(y))\) for all \(x, y \in \V\).
  Prove the following results.
  \begin{enumerate}
    \item If \(H \in \bi(\W)\), then \(\widehat{\T}(H) \in \bi(\V)\).
    \item \(\widehat{\T} \in \ls(\bi(\W), \bi(\V))\).
    \item If \(\T\) is an isomorphism, then so is \(\widehat{\T}\).
  \end{enumerate}
\end{ex}

\begin{proof}[\pf{ex:6.8.7}(a)]
  Let \(x, y, z \in \V\) and let \(c \in \F\).
  Since
  \begin{align*}
    \widehat{\T}(H)(cx + y, z) & = H(\T(cx + y), \T(z))                             &  & \by{ex:6.8.7} \\
                               & = H(c \T(x) + \T(y), \T(z))                        &  & \by{2.1.2}[b] \\
                               & = c H(\T(x), \T(z)) + H(\T(y), \T(z))              &  & \by{6.8.1}[a] \\
                               & = c \widehat{\T}(H)(x, z) + \widehat{\T}(H)(y, z); &  & \by{ex:6.8.7} \\
    \widehat{\T}(H)(z, cx + y) & = H(\T(z), \T(cx + y))                             &  & \by{ex:6.8.7} \\
                               & = H(\T(z), c \T(x) + \T(y))                        &  & \by{2.1.2}[b] \\
                               & = c H(\T(z), \T(x)) + H(\T(z), \T(y))              &  & \by{6.8.1}[a] \\
                               & = c \widehat{\T}(H)(z, x) + \widehat{\T}(H)(z, y), &  & \by{ex:6.8.7}
  \end{align*}
  by \cref{6.8.1} we see that \(\widehat{\T}(H) \in \bi(\V)\).
\end{proof}

\begin{proof}[\pf{ex:6.8.7}(b)]
  Let \(f, g \in \bi(\W)\) and let \(c \in \F\).
  Since
  \begin{align*}
    \forall x, y \in \V, \widehat{\T}(cf + g)(x, y) & = (cf + g)(\T(x), \T(y))                           &  & \by{ex:6.8.7} \\
                                                    & = c f(\T(x), \T(y)) + g(\T(x), \T(y))              &  & \by{6.31}     \\
                                                    & = c \widehat{\T}(f)(x, y) + \widehat{\T}(g)(x, y), &  & \by{ex:6.8.7}
  \end{align*}
  we have \(\widehat{\T}(cf + g) = c \widehat{\T}(f) + \widehat{\T}(g)\).
  Thus by \cref{2.1.2}(b) and \cref{ex:6.8.7}(a) we know that \(\widehat{\T} \in \ls(\bi(\W), \bi(\V))\).
\end{proof}

\begin{proof}[\pf{ex:6.8.7}(c)]
  By \cref{ex:6.8.7}(b) we know that \(\widehat{\T} \in \ls(\bi(\W), \bi(\V))\), thus by \cref{2.4.8} we only need to show that \(\widehat{\T}\) is one-to-one and onto.

  First we show that \(\widehat{\T}\) is one-to-one.
  Let \(H \in \ns{\widehat{\T}}\).
  Then \(\widehat{\T}(H)\) is the zero bilinear transformation in \(\bi(\V)\).
  Since
  \begin{align*}
             & \forall x, y \in \V, \widehat{\T}(H)(x, y) = H(\T(x), \T(y)) = 0 &  & \by{ex:6.8.7} \\
    \implies & \forall u, v \in \W, H(u, v) = 0,                                &  & \by{2.4.8}
  \end{align*}
  we know that \(H\) is the zero bilinear transformation in \(\bi(\W)\).
  Thus \(\ns{\widehat{\T}} = \set{\zv}\) and by \cref{2.4} \(\widehat{\T}\) is one-to-one.

  Now we show that \(\widehat{\T}\) is onto.
  Let \(g \in \bi(\V)\).
  Now we define \(f \in \W \times \W \to \F\) by \(f = \widehat{\T^{-1}}(g)\).
  By \cref{ex:6.8.7}(a) we know that \(f \in \bi(\W)\).
  Since
  \begin{align*}
    \forall x, y \in \V, \widehat{\T}(f)(x, y) & = f\pa{\T(x), \T(y)}                   &  & \by{ex:6.8.7} \\
                                               & = \widehat{\T^{-1}}(g)(\T(x), \T(y))                      \\
                                               & = g\pa{\T^{-1}(\T(x)), \T^{-1}(\T(y))} &  & \by{ex:6.8.7} \\
                                               & = g(x, y),
  \end{align*}
  we know that \(\widehat{\T}(f) = g\).
  Thus \(\widehat{\T}\) is onto, and we conclude that \(\widehat{\T}\) is an isomorphism from \(\bi(\W)\) to \(\bi(\V)\).
\end{proof}

\begin{ex}\label{ex:6.8.8}
  Assume the notation of \cref{6.32}.
  \begin{enumerate}
    \item Prove that for any ordered basis \(\beta\), \(\psi_{\beta}\) is linear.
    \item Let \(\beta\) be an ordered basis for an \(n\)-dimensional space \(\V\) over \(\F\), and let \(\phi_{\beta} : \V \to \vs{F}^n\) be the standard representation of \(\V\) with respect to \(\beta\).
          For \(A \in \ms[n][n][\F]\), define \(H : \V \times \V \to \F\) by \(H(x, y) = \tp{(\phi_{\beta}(x))} A \phi_{\beta}(y)\).
          Prove that \(H \in \bi(V)\).
          Can you establish this as a corollary to \cref{ex:6.8.7}?
    \item Prove the converse of (b):
          Let \(H\) be a bilinear form on \(\V\).
          If \(A = \psi_{\beta}(H)\), then \(H(x, y) = \tp{(\phi_{\beta}(x))} A \phi_{\beta}(y)\).
  \end{enumerate}
\end{ex}

\begin{proof}[\pf{ex:6.8.8}(a)]
  See \cref{6.32}.
\end{proof}

\begin{proof}[\pf{ex:6.8.8}(b)]
  Define \(G : \vs{F}^n \times \vs{F}^n \to \F\) by
  \[
    \forall x, y \in \vs{F}^n, G(x, y) = \tp{x} A y.
  \]
  By \cref{6.8.2} we know that \(G \in \bi(\vs{F}^n)\).
  By \cref{ex:6.8.7} we see that \(H = \widehat{\phi_{\beta}}(G)\) and by \cref{ex:6.8.7}(a) we have \(H \in \bi(\V)\).
\end{proof}

\begin{proof}[\pf{ex:6.8.8}(c)]
  See \cref{6.8.7}.
\end{proof}

\begin{ex}\label{ex:6.8.9}
  \begin{enumerate}
    \item Prove \cref{6.8.6}.
    \item For a finite-dimensional vector space \(\V\) over \(\F\), describe a method for finding an ordered basis for \(\bi(\V)\).
  \end{enumerate}
\end{ex}

\begin{proof}[\pf{ex:6.8.9}(a)]
  See \cref{6.8.6}.
\end{proof}

\begin{proof}[\pf{ex:6.8.9}(b)]
  Let \(n = \dim(\V)\) and let \(\beta\) be an ordered basis for \(\V\) over \(\F\).
  By \cref{6.32} we know that \(\psi_{\beta}\) is an isomorphism from \(\bi(\V)\) to \(\ms[n][n][\F]\).
  Now we define \(\gamma = \set{E^{i j} \in \ms[n][n][\F] : i, j \in \set{1, \dots, n}}\) as in \cref{1.6.4}.
  Then by \cref{2.2,2.3} we know that \(\psi_{\beta}^{-1}(\gamma)\) is an ordered basis for \(\bi(\V)\) over \(\F\).
\end{proof}

\begin{ex}\label{ex:6.8.10}
  Prove \cref{6.8.7}.
\end{ex}

\begin{proof}[\pf{ex:6.8.10}]
  See \cref{6.8.7}.
\end{proof}

\begin{ex}\label{ex:6.8.11}
  Prove \cref{6.8.8}.
\end{ex}

\begin{proof}[\pf{ex:6.8.11}]
  See \cref{6.8.8}.
\end{proof}

\begin{ex}\label{ex:6.8.12}
  Prove that the relation of congruence is an equivalence relation.
\end{ex}

\begin{proof}[\pf{ex:6.8.12}]
  Let \(A, B, C \in \ms[n][n][\F]\).
  Since \(A = \tp{I_n} A I_n\), by \cref{6.8.9} we know that the congruence relation is reflexive.
  Since
  \begin{align*}
             & B \text{ is congruent to } A                                           \\
    \implies & \exists Q \in \ms[n][n][\F] : \begin{dcases}
                                               Q \text{ is invertible} \\
                                               B = \tp{Q} A Q
                                             \end{dcases}        &  & \by{6.8.9}      \\
    \implies & A = (\tp{Q})^{-1} B Q^{-1} = \tp{(Q^{-1})} B Q^{-1} &  & \by{ex:2.4.5} \\
    \implies & A \text{ is congruent to } B,
  \end{align*}
  by \cref{6.8.9} we know that the congruence relation is symmetric.
  Since
  \begin{align*}
             & \begin{dcases}
                 B \text{ is congruent to } A \\
                 C \text{ is congruent to } B
               \end{dcases}                                    \\
    \implies & \exists P, Q \in \ms[n][n][\F] : \begin{dcases}
                                                  P, Q \text{ are invertible} \\
                                                  B = \tp{Q} A Q              \\
                                                  C = \tp{P} B P
                                                \end{dcases} &  & \by{6.8.9}   \\
    \implies & C = \tp{P} \tp{Q} A Q P = \tp{QP} A QP          &  & \by{2.3.2} \\
    \implies & C \text{ is congruent to } A,                   &  & \by{6.8.9}
  \end{align*}
  by \cref{6.8.9} we know that the congruence relation is transitive.
  Thus the congruence relation is an equivalence relation.
\end{proof}

\begin{ex}\label{ex:6.8.13}
  The following outline provides an alternative proof to \cref{6.33}.
  \begin{enumerate}
    \item Suppose that \(\beta\) and \(\gamma\) are ordered bases for a finite-dimensional vector space \(\V\) over \(\F\), and let \(Q\) be the change of coordinate matrix changing \(\gamma\)-coordinates into \(\beta\)-coordinates.
          Prove that \(\phi_{\beta} = \L_Q \phi_{\gamma}\), where \(\phi_{\beta}\) and \(\phi_{\gamma}\) are the standard representations of \(\V\) with respect to \(\beta\) and \(\gamma\), respectively.
    \item Apply \cref{6.8.7} to (a) to obtain an alternative proof of \cref{6.33}.
  \end{enumerate}
\end{ex}

\begin{proof}[\pf{ex:6.8.13}(a)]
  Suppose that \(n = \dim(\V)\).
  Then we have
  \begin{align*}
    \forall x \in \V, (\L_Q \phi_{\gamma})(x) & = \L_Q(\phi_{\gamma}(x))                                        \\
                                              & = \L_Q([x]_{\gamma})                            &  & \by{2.2.3} \\
                                              & = Q [x]_{\gamma}                                &  & \by{2.3.8} \\
                                              & = [\IT[\vs{F}^n]]_{\gamma}^{\beta} [x]_{\gamma} &  & \by{2.5.1} \\
                                              & = [x]_{\beta}                                   &  & \by{2.14}  \\
                                              & = \phi_{\beta}(x)                               &  & \by{2.2.3}
  \end{align*}
  and thus \(\L_Q \phi_{\gamma} = \phi_{\beta}\).
\end{proof}

\begin{proof}[\pf{ex:6.8.13}(b)]
  Continue the proof of \cref{ex:6.8.13}(a).
  Let \(H \in \bi(\V)\).
  Then we have
  \begin{align*}
    \forall x, y \in \V, H(x, y) & = \tp{(\phi_{\beta}(x))} \psi_{\beta}(H) \phi_{\beta}(y)    &  & \by{6.8.7}        \\
                                 & = \tp{([x]_{\beta})} \psi_{\beta}(H) [y]_{\beta}            &  & \by{2.2.3}        \\
                                 & = \tp{(Q [x]_{\gamma})} \psi_{\beta}(H) Q [y]_{\gamma}      &  & \by{ex:6.8.13}[a] \\
                                 & = \tp{([x]_{\gamma})} \tp{Q} \psi_{\beta}(H) Q [y]_{\gamma} &  & \by{2.3.2}        \\
                                 & = \tp{([x]_{\gamma})} \psi_{\gamma}(H) [y]_{\gamma}.        &  & \by{6.8.7}
  \end{align*}
  Thus by \cref{2.15}(b) we have \(\psi_{\gamma}(H) = \tp{Q} \psi_{\beta}(H) Q\).
\end{proof}

\begin{ex}\label{ex:6.8.14}
  Let \(\V\) be a finite-dimensional vector space over \(\F\) and \(H \in \bi(\V)\).
  Prove that, for any ordered bases \(\beta\) and \(\gamma\) of \(\V\) over \(\F\), \(\rk{\psi_{\beta}(H)} = \rk{\psi_{\gamma}(H)}\).
\end{ex}

\begin{proof}[\pf{ex:6.8.14}]
  Let \(n = \dim(\V)\).
  By \cref{6.33} there exists a \(Q \in \ms[n][n][\F]\) such that \(Q\) is invertible and \(\psi_{\gamma}(H) = \tp{Q} \psi_{\beta}(H) Q\).
  Then we have
  \begin{align*}
    \rk{\psi_{\gamma}(H)} & = \rk{\tp{Q} \psi_{\beta}(H) Q} &  & \by{6.33}   \\
                          & = \rk{\psi_{\beta}(H)}.         &  & \by{3.4}[c]
  \end{align*}
\end{proof}

\begin{ex}\label{ex:6.8.15}
  Prove the following results.
  \begin{enumerate}
    \item Any square diagonal matrix is symmetric.
    \item Any matrix congruent to a diagonal matrix is symmetric.
    \item \cref{6.8.16}.
  \end{enumerate}
\end{ex}

\begin{proof}[\pf{ex:6.8.15}(a)]
  See \cref{ex:1.3.7}.
\end{proof}

\begin{proof}[\pf{ex:6.8.15}(b)]
  Let \(A, D, Q \in \ms[n][n][\F]\) such that \(Q\) is invertible and \(D = \tp{Q} A Q\) is a diagonal matrix.
  Then we have
  \begin{align*}
    \tp{A} & = \tp{\pa{\pa{\tp{Q}}^{-1} D Q^{-1}}}                &  & \by{2.4.3}    \\
           & = \tp{\pa{Q^{-1}}} \tp{D} \tp{\pa{\pa{\tp{Q}}^{-1}}} &  & \by{2.3.2}    \\
           & = \pa{\tp{Q}}^{-1} \tp{D} \tp{\pa{\tp{\pa{Q^{-1}}}}} &  & \by{ex:2.4.5} \\
           & = \pa{\tp{Q}}^{-1} \tp{D} Q^{-1}                     &  & \by{ex:1.3.4} \\
           & = \pa{\tp{Q}}^{-1} D Q^{-1}                          &  & \by{ex:1.3.7} \\
           & = A.
  \end{align*}
  Thus by \cref{1.3.4} \(A\) is symmetric.
\end{proof}

\begin{proof}[\pf{ex:6.8.15}(c)]
  See \cref{6.8.16}.
\end{proof}

\begin{ex}\label{ex:6.8.16}
  Let \(\V\) be a vector space over a field \(\F\) not of characteristic two, and let \(H\) be a symmetric bilinear form on \(\V\).
  Prove that if \(K(x) = H(x, x)\) is the quadratic form associated with \(H\), then, for all \(x, y \in \V\),
  \[
    H(x, y) = \dfrac{1}{2} (K(x + y) - K(x) - K(y)).
  \]
\end{ex}

\begin{proof}[\pf{ex:6.8.16}]
  We have
  \begin{align*}
     & \forall x, y \in \V, \dfrac{1}{2} (K(x + y) - K(x) - K(y))                    \\
     & = \dfrac{1}{2} (H(x + y, x + y) - H(x, x) - H(y, y))       &  & \by{6.8.19}   \\
     & = \dfrac{1}{2} (H(x, y) + H(y, x))                         &  & \by{6.8.3}[c] \\
     & = \dfrac{1}{2} 2 H(x, y)                                   &  & \by{6.8.11}   \\
     & = H(x, y).
  \end{align*}
\end{proof}

\begin{ex}\label{ex:6.8.21}
\end{ex}

\begin{ex}\label{ex:6.8.23}
\end{ex}

\section{Einstein's Special Theory of Relativity}\label{sec:6.9}

\begin{note}
  As a consequence of physical experiments performed in the latter half of the nineteenth century (most notably the Michelson--Morley experiment of 1887), physicists concluded that \emph{the results obtained in measuring the speed of light are independent of the velocity of the instrument used to measure the speed of light}.

  This revelation led to a new way of relating coordinate systems used to locate events in space--time.
  The result was Albert Einstein's \emph{special theory of relativity}.
  In this section, we develop via a linear algebra viewpoint the essence of Einstein's theory.
\end{note}

\begin{defn}\label{6.9.1}
  The basic problem is to compare two different inertial (nonaccelerating) coordinate systems \(S\) and \(S'\) in three-space (\(\R^3\)) that are in motion relative to each other under the assumption that the speed of light is the same when measured in either system.
  We assume that \(S'\) moves at a constant velocity in relation to \(S\) as measured from \(S\).
  To simplify matters, let us suppose that the following conditions hold:
  \begin{enumerate}
    \item The corresponding axes of \(S\) and \(S'\) (\(x\) and \(x'\), \(y\) and \(y'\), \(z\) and \(z'\)) are parallel, and the origin of \(S'\) moves in the positive direction of the \(x\)-axis of \(S\) at a constant velocity \(v > 0\) relative to \(S\).
    \item Two clocks \(C\) and \(C'\) are placed in space---the first stationary relative to the coordinate system \(S\) and the second stationary relative to the coordinate system \(S'\).
          These clocks are designed to give real numbers in units of seconds as readings.
          The clocks are calibrated so that at the instant the origins of \(S\) and \(S'\) coincide, both clocks give the reading zero.
    \item The unit of length is the \textbf{light second} (the distance light travels in \(1\) second), and the unit of time is the second.
          Note that, with respect to these units, the speed of light is \(1\) light second per second.
  \end{enumerate}

  Given any event (something whose position and time of occurrence can be described), we may assign a set of \emph{space--time coordinates} to it.
  For example, if \(p\) is an event that occurs at position
  \[
    \begin{pmatrix}
      x \\
      y \\
      z
    \end{pmatrix}
  \]
  relative to \(S\) and at time \(t\) as read on clock \(C\), we can assign to \(p\) the set of coordinates
  \[
    \begin{pmatrix}
      x \\
      y \\
      z \\
      t
    \end{pmatrix}.
  \]
  This ordered \(4\)-tuple is called the \textbf{space--time coordinates} of \(p\) relative to \(S\) and \(C\).
  Likewise, \(p\) has a set of space--time coordinates
  \[
    \begin{pmatrix}
      x' \\
      y' \\
      z' \\
      t'
    \end{pmatrix}
  \]
  relative to \(S'\) and \(C'\).
\end{defn}

\begin{ax}[Axioms of the Special Theory of Relativity]\label{6.9.2}
  Einstein made certain assumptions about \(\T_v : \R^4 \to \R^4\) that led to his special theory of relativity.
  We formulate an equivalent set of assumptions.
  \begin{enumerate}[label=(R \arabic*)]
    \item The speed of any light beam, when measured in either coordinate system using a clock stationary relative to that coordinate system, is \(1\).
    \item The mapping \(\T_v : \R^4 \to \R^4\) is an isomorphism.
    \item If
          \[
            \T_v\begin{pmatrix}
              x \\
              y \\
              z \\
              t
            \end{pmatrix} = \begin{pmatrix}
              x' \\
              y' \\
              z' \\
              t'
            \end{pmatrix},
          \]
          then \(y' = y\) and \(z' = z\).
    \item If
          \[
            \T_v\begin{pmatrix}
              x   \\
              y_1 \\
              z_1 \\
              t
            \end{pmatrix} = \begin{pmatrix}
              x' \\
              y' \\
              z' \\
              t'
            \end{pmatrix} \quad \text{and} \quad \T_v\begin{pmatrix}
              x   \\
              y_2 \\
              z_2 \\
              t
            \end{pmatrix} = \begin{pmatrix}
              x'' \\
              y'' \\
              z'' \\
              t''
            \end{pmatrix},
          \]
          then \(x'' = x'\) and \(t'' = t'\).
    \item The origin of \(S\) moves in the negative direction of the \(x'\)-axis of \(S'\) at the constant velocity \(-v < 0\) as measured from \(S'\).
  \end{enumerate}
  Axioms (R 3) and (R 4) tell us that for \(p \in \R^4\), the second and third coordinates of \(\T_v(p)\) are unchanged and the first and fourth coordinates of \(\T_v(p)\) are independent of the second and third coordinates of \(p\).

  As we will see, these five axioms completely characterize \(\T_v\).
  The operator \(\T_v\) is called the \textbf{Lorentz transformation} in direction \(x\).
  We intend to compute \(\T_v\) and use it to study the curious phenomenon of time contraction.
\end{ax}

\begin{thm}\label{6.39}
  On \(\R^4\), the following statements are true.
  \begin{enumerate}
    \item \(\T_v(e_i) = e_i\) for \(i \in \set{2, 3}\).
    \item \(\spn{\set{\seq{e}{2,3}}}\) is \(\T_v\)-invariant.
    \item \(\spn{\set{\seq{e}{1,4}}}\) is \(\T_v\)-invariant.
    \item Both \(\spn{\set{\seq{e}{2,3}}}\) and \(\spn{\set{\seq{e}{1,4}}}\) are \(\T_v^*\)-invariant.
    \item \(\T_v^*(e_i) = e_i\) for \(i \in \set{2, 3}\).
  \end{enumerate}
\end{thm}

\begin{proof}[\pf{6.39}(a)]
  By \cref{6.9.2} (R 2),
  \[
    \T_v\begin{pmatrix}
      0 \\
      0 \\
      0 \\
      0
    \end{pmatrix} = \begin{pmatrix}
      0 \\
      0 \\
      0 \\
      0
    \end{pmatrix},
  \]
  and hence, by \cref{6.9.2} (R 4), the first and fourth coordinates of
  \[
    \T_v\begin{pmatrix}
      0 \\
      a \\
      b \\
      0
    \end{pmatrix}
  \]
  are both zero for any \(a, b \in \R\).
  Thus, by \cref{6.9.2} (R 3),
  \[
    \T_v\begin{pmatrix}
      0 \\
      1 \\
      0 \\
      0
    \end{pmatrix} = \begin{pmatrix}
      0 \\
      1 \\
      0 \\
      0
    \end{pmatrix} \quad \text{and} \quad \T_v\begin{pmatrix}
      0 \\
      0 \\
      1 \\
      0
    \end{pmatrix} = \begin{pmatrix}
      0 \\
      0 \\
      1 \\
      0
    \end{pmatrix}.
  \]
\end{proof}

\begin{proof}[\pf{6.39}(b)]
  Since
  \begin{align*}
    \T_v(\spn{\set{\seq{e}{2,3}}}) & = \spn{\set{\T_v(e_2), \T_v(e_3)}} &  & \by{2.2}     \\
                                   & = \spn{\set{\seq{e}{2,3}}},        &  & \by{6.39}[a]
  \end{align*}
  by \cref{5.4.1} we see that \(\spn{\set{\seq{e}{2,3}}}\) is \(\T\)-invariant.
\end{proof}

\begin{proof}[\pf{6.39}(c)]
  By \cref{6.9.2} (R 2) we have
  \[
    \T_v(e_1) = \begin{pmatrix}
      a \\
      0 \\
      0 \\
      b
    \end{pmatrix} = a e_1 + b e_4 \quad \text{and} \quad \T_v(e_4) = \begin{pmatrix}
      c \\
      0 \\
      0 \\
      d
    \end{pmatrix} = c e_1 + d e_4
  \]
  for some \(a, b, c, d \in \R\).
  Thus by \cref{5.4.1} we see that \(\spn{\set{\seq{e}{1,4}}}\) is \(\T\)-invariant.
\end{proof}

\begin{proof}[\pf{6.39}(d)]
  Since \(\set{\seq{e}{1,2,3,4}}\) is orthonormal with respect to the standard inner product on \(\R^4\) over \(\R\), by \cref{6.7}(b) we see that
  \[
    \spn{\set{\seq{e}{1,4}}}^{\perp} = \spn{\set{\seq{e}{2,3}}} \quad \text{and} \quad \spn{\set{\seq{e}{2,3}}}^{\perp} = \spn{\set{\seq{e}{1,4}}}.
  \]
  Thus by \cref{ex:6.4.7}(b) we know that \(\spn{\set{\seq{e}{1,4}}}\) and \(\spn{\set{\seq{e}{2,3}}}\) are \(\T_v^*\) invariant.
\end{proof}

\begin{proof}[\pf{6.39}(e)]
  For any \(j \in \set{1, 3, 4}\),
  \begin{align*}
    \inn{\T_v^*(e_2), e_j} & = \inn{e_2, \T_v(e_j)} &  & \by{6.9}       \\
                           & = 0;                   &  & \by{6.39}[a,c]
  \end{align*}
  for \(j = 2\),
  \begin{align*}
    \inn{\T_v^*(e_2), e_2} & = \inn{e_2, \T_v(e_2)} &  & \by{6.9}     \\
                           & = \inn{e_2, e_2}       &  & \by{6.39}[a] \\
                           & = 1.                   &  & \by{6.1.2}
  \end{align*}
  We conclude by \cref{6.2.6} that \(\T_v^*(e_2)\) is a multiple of \(e_2\) (i.e., that \(\T_v^*(e_2) = k e_2\) for some \(k \in \R\)).
  Thus,
  \begin{align*}
    1 & = \inn{e_2, e_2}         &  & \by{6.1.2}    \\
      & = \inn{e_2, \T_v(e_2)}   &  & \by{6.39}[a]  \\
      & = \inn{\T_v^*(e_2), e_2} &  & \by{6.9}      \\
      & = \inn{k e_2, e_2}                          \\
      & = k,                     &  & \by{6.1.1}[b]
  \end{align*}
  and hence \(\T_v^*(e_2) = e_2\).
  Similarly, \(\T_v^*(e_3) = e_3\).
\end{proof}

\begin{thm}\label{6.40}
  Suppose that, at the instant the origins of \(S\) and \(S'\) coincide, a light flash is emitted from their common origin.
  The event of the light flash when measured either relative to \(S\) and \(C\) or relative to \(S'\) and \(C'\) has space--time
  coordinates
  \[
    \begin{pmatrix}
      0 \\
      0 \\
      0 \\
      0
    \end{pmatrix}.
  \]
  Let \(P\) be the set of all events whose space--time coordinates
  \[
    \begin{pmatrix}
      x \\
      y \\
      z \\
      t
    \end{pmatrix}
  \]
  relative to \(S\) and \(C\) are such that the flash is observable from the point with coordinates
  \[
    \begin{pmatrix}
      x \\
      y \\
      z
    \end{pmatrix}
  \]
  (as measured relative to \(S\)) at the time \(t\) (as measured on \(C\)).
  Let us characterize \(P\) in terms of \(x, y, z\), and \(t\).
  Since the speed of light is \(1\), at any time \(t \geq 0\) the light flash is observable from any point whose distance to the origin of \(S\) (as measured on \(S\)) is \(t \cdot 1 = t\).
  These are precisely the points that lie on the sphere of radius \(t\) with center at the origin.
  The coordinates (relative to \(S\)) of such points satisfy the equation \(x^2 + y^2 + z^2 - t^2 = 0\).
  Hence an event lies in \(P\) iff its space--time coordinates
  \[
    \begin{pmatrix}
      x \\
      y \\
      z \\
      t
    \end{pmatrix} \quad (t \geq 0)
  \]
  relative to \(S\) and \(C\) satisfy the equation \(x^2 + y^2 + z^2 - t^2 = 0\).
  By virtue of \cref{6.9.2} (R 1), we can characterize \(P\) in terms of the space--time coordinates relative to \(S'\) and \(C'\) similarly:
  An event lies in \(P\) iff, relative to \(S'\) and \(C'\), its space--time coordinates
  \[
    \begin{pmatrix}
      x' \\
      y' \\
      z' \\
      t'
    \end{pmatrix} \quad (t' \geq 0)
  \]
  satisfy the equation \((x')^2 + (y')^2 + (z')^2 - (t')^2 = 0\).
  Let
  \[
    A = \begin{pmatrix}
      1 & 0 & 0 & 0  \\
      0 & 1 & 0 & 0  \\
      0 & 0 & 1 & 0  \\
      0 & 0 & 0 & -1
    \end{pmatrix}.
  \]
  If \(\inn{\L_A(w), w} = 0\) for some \(w \in \R^4\), then
  \[
    \inn{\T_v^* \L_A \T_v(w), w} = 0.
  \]
\end{thm}

\begin{proof}[\pf{6.40}]
  Let
  \[
    w = \begin{pmatrix}
      x \\
      y \\
      z \\
      t
    \end{pmatrix} \in \R^4,
  \]
  and suppose that \(\inn{\L_A(w), w} = 0\).
  \begin{description}
    \item[Case 1.]
      \(t \geq 0\).
      Since \(\inn{\L_A(w), w} = x^2 + y^2 + z^2 - t^2\), the vector \(w\) gives the coordinates of an event in \(P\) relative to \(S\) and \(C\).
      Because
      \[
        \begin{pmatrix}
          x \\
          y \\
          z \\
          t
        \end{pmatrix} \quad \text{and} \quad \begin{pmatrix}
          x' \\
          y' \\
          z' \\
          t'
        \end{pmatrix}
      \]
      are the space--time coordinates of the same event relative to \(S'\) and \(C'\), \cref{6.9.2} (R 1) yields
      \[
        (x')^2 + (y')^2 + (z')^2 - (t')^2 = 0.
      \]
      Thus \(\inn{\T_v^* \L_A \T_v(w), w} = \inn{\L_A \T_v(w), \T_v(w)} = (x')^2 + (y')^2 + (z')^2 - (t')^2 = 0\), and the conclusion follows.
    \item[Case 2.]
      \(t < 0\).
      The proof follows by applying Case 1 to \(-w\), i.e.,
      \begin{align*}
        0 & = \inn{\T_v^* \L_A \T_v(-w), -w} &  & \text{(by Case 1)} \\
          & = \inn{-\T_v^* \L_A \T_v(w), -w} &  & \by{2.1.1}[b]      \\
          & = -\inn{\T_v^* \L_A \T_v(w), -w} &  & \by{6.1.1}[b]      \\
          & = \inn{\T_v^* \L_A \T_v(w), w}.  &  & \by{6.1}[b]
      \end{align*}
  \end{description}
\end{proof}

\begin{thm}\label{6.41}
  We now proceed to deduce information about \(\T_v\).
  Let
  \[
    w_1 = \begin{pmatrix}
      1 \\
      0 \\
      0 \\
      1
    \end{pmatrix} \quad \text{and} \quad w_2 = \begin{pmatrix}
      1 \\
      0 \\
      0 \\
      -1
    \end{pmatrix}.
  \]
  By \cref{ex:6.9.3}, \(\set{\seq{w}{1,2}}\) is an orthogonal basis for \(\spn{\set{\seq{e}{1,4}}}\) over \(\R\), and \(\spn{\set{\seq{e}{1,4}}}\) is \(\T_v^* \L_A \T_v\)-invariant.
  Then there exist nonzero scalars \(a, b \in \R\) such that
  \begin{enumerate}
    \item \(\T_v^* \L_A \T_v(w_1) = a w_2\).
    \item \(\T_v^* \L_A \T_v(w_2) = b w_1\).
  \end{enumerate}
\end{thm}

\begin{proof}[\pf{6.41}]
  Because \(\inn{\L_A(w_1), w_1} = 0\), \(\inn{\T_v^* \L_A \T_v(w_1), w_1} = 0\) by \cref{6.40}.
  Thus \(\T_v^* \L_A \T_v(w_1)\) is orthogonal to \(w_1\).
  Since \(\spn{\set{\seq{e}{1,4}}} = \spn{\set{\seq{w}{1,2}}}\) is \(\T_v^* \L_A \T_v\)-invariant, \(\T_v^* \L_A \T_v(w_1)\) must lie in this set.
  But \(\set{\seq{w}{1,2}}\) is an orthogonal basis for this subspace, and so \(\T_v^* \L_A \T_v(w_1)\) must be a multiple of \(w_2\).
  Thus \(\T_v^* \L_A \T_v (w_1) = a w_2\) for some scalar \(a \in \R\).
  Since \(\T_v\) and \(A\) are invertible, so is \(\T_v^* \L_A \T_v\) (\cref{ex:6.3.8}).
  Thus \(a \neq 0\), proving (a).

  The proof of (b) is similar to (a).
\end{proof}

\begin{cor}\label{6.9.3}
  Let \(B_v = [\T_v]_{\beta}\), where \(\beta\) is the standard ordered basis for \(\R^4\) over \(\R\).
  Then
  \begin{enumerate}
    \item \(B_v^* A B_v = A\).
    \item \(\T_v^* \L_A \T_v = \L_A\).
  \end{enumerate}
\end{cor}

\begin{proof}[\pf{6.9.3}]
  By \cref{6.41} we know that there exist \(a, b \in \R \setminus \set{0}\) such that
  \begin{align*}
    \T_v^* \L_A \T_v(e_1 + e_4) & = \T_v^* \L_A \T_v(e_1) + \T_v^* \L_A \T_v(e_4) &  & \by{2.1.1}[a] \\
                                & = a e_1 - a e_4;                                &  & \by{6.41}     \\
    \T_v^* \L_A \T_v(e_1 - e_4) & = \T_v^* \L_A \T_v(e_1) - \T_v^* \L_A \T_v(e_4) &  & \by{2.1.2}[c] \\
                                & = b e_1 + b e_4.                                &  & \by{6.41}
  \end{align*}
  Thus we have
  \begin{align*}
    \T_v^* \L_A \T_v(e_1) & = \dfrac{a + b}{2} e_1 - \dfrac{a - b}{2} e_4; \\
    \T_v^* \L_A \T_v(e_4) & = \dfrac{a - b}{2} e_1 - \dfrac{a + b}{2} e_4.
  \end{align*}
  Since
  \begin{align*}
    \forall i \in \set{2, 3}, \T_v^* \L_A \T_v(e_i) & = \T_v^* \L_A(e_i) &  & \by{6.39}[a] \\
                                                    & = \T_v^*(e_i)      &  & \by{2.3.8}   \\
                                                    & = e_i,             &  & \by{6.39}[e]
  \end{align*}
  we have
  \begin{align*}
    [\T_v^* \L_A \T_v]_{\beta} & = [\T_v^*]_{\beta} [\L_A]_{\beta} [\T_v]_{\beta} &  & \by{2.3.3}   \\
                               & = [\T_v]_{\beta}^* [\L_A]_{\beta} [\T_v]_{\beta} &  & \by{6.10}    \\
                               & = [\T_v]_{\beta}^* A [\T_v]_{\beta}              &  & \by{2.15}[a] \\
                               & = B_v^* A B_v                                                      \\
                               & = \begin{pmatrix}
                                     \dfrac{a + b}{2}  & 0 & 0 & \dfrac{a - b}{2}  \\
                                     0                 & 1 & 0 & 0                 \\
                                     0                 & 0 & 1 & 0                 \\
                                     -\dfrac{a - b}{2} & 0 & 0 & -\dfrac{a + b}{2}
                                   \end{pmatrix}. &  & \by{2.2.4}
  \end{align*}
  Then we have
  \begin{align*}
             & (B_v^* A B_v)^* = B_v^* A^* B_v = B_v^* A B_v   &  & \by{6.3.2}[c,d] \\
    \implies & \dfrac{a - b}{2} = -\dfrac{a - b}{2}            &  & \by{6.1.5}      \\
    \implies & \dfrac{a - b}{2} = 0                                                 \\
    \implies & B_v^* A B_v = \begin{pmatrix}
                               \dfrac{a + b}{2} & 0 & 0 & 0                 \\
                               0                & 1 & 0 & 0                 \\
                               0                & 0 & 1 & 0                 \\
                               0                & 0 & 0 & -\dfrac{a + b}{2}
                             \end{pmatrix}.
  \end{align*}
  But
  \begin{align*}
    \inn{\L_A(e_2 + e_4), e_2 + e_4} & = \inn{\L_A(e_2) + \L_A(e_4), e_2 + e_4} &  & \by{2.1.1}[a] \\
                                     & = \inn{e_2 - e_4, e_2 + e_4}             &  & \by{6.39}[a]  \\
                                     & = 0                                      &  & \by{6.1.2}
  \end{align*}
  implies
  \begin{align*}
             & \inn{\T_v^* \L_A \T_v(e_2 + e_4), e_2 + e_4} = 0 &  & \by{6.40}       \\
    \implies & \inn{\begin{pmatrix}
                        0 \\
                        1 \\
                        0 \\
                        -\dfrac{a + b}{2}
                      \end{pmatrix}, e_2 + e_4} = 0                    &  & \by{2.3.1} \\
    \implies & 1 - \dfrac{a + b}{2} = 0                         &  & \by{6.1.2}      \\
    \implies & \dfrac{a + b}{2} = 1                                                  \\
    \implies & B_v^* A B_v = \begin{pmatrix}
                               1 & 0 & 0 & 0  \\
                               0 & 1 & 0 & 0  \\
                               0 & 0 & 1 & 0  \\
                               0 & 0 & 0 & -1
                             \end{pmatrix} = A.
  \end{align*}
  Thus by \cref{2.3.3} we have \(\T_v^* \L_A \T_v = \L_A\).
\end{proof}

\exercisesection

\setcounter{ex}{2}
\begin{ex}\label{ex:6.9.3}
  For
  \[
    w_1 = \begin{pmatrix}
      1 \\
      0 \\
      0 \\
      1
    \end{pmatrix} \quad \text{and} \quad w_2 = \begin{pmatrix}
      1 \\
      0 \\
      0 \\
      -1
    \end{pmatrix},
  \]
  show that
  \begin{enumerate}
    \item \(\set{\seq{w}{1,2}}\) is an orthogonal basis for \(\spn{\set{\seq{e}{1,4}}}\) over \(\R\);
    \item \(\spn{\set{\seq{e}{1,4}}}\) is \(\T_v^* \L_A \T_v\)-invariant, where \(\L_A\) is defined in \cref{6.40}.
  \end{enumerate}
\end{ex}

\begin{proof}[\pf{ex:6.9.3}(a)]
  By \cref{6.1.2} \(\set{\seq{w}{1,2}}\) is orthogonal with respect to the standard inner product on \(\R^4\) over \(\R\).
  Since
  \begin{align*}
    w_1                            & = e_1 + e_4; \\
    w_2                            & = e_1 - e_4; \\
    \dim(\spn{\set{\seq{e}{1,4}}}) & = 2,
  \end{align*}
  by \cref{1.6.15}(b) and \cref{6.2.4} we know that \(\set{\seq{w}{1,2}}\) is an orthogonal basis for \(\spn{\set{\seq{e}{1,4}}}\) over \(\R\).
\end{proof}

\begin{proof}[\pf{ex:6.9.3}(b)]
  Since
  \begin{align*}
    (\T_v^* \L_A \T_v)(\spn{\set{\seq{e}{1,4}}}) & = \T_v^*(\L_A(\T_v(\spn{\set{\seq{e}{1,4}}})))                         \\
                                                 & \subseteq \T_v^*(\L_A(\spn{\set{\seq{e}{1,4}}})) &  & \by{6.39}[c]     \\
                                                 & = \T_v^*(\L_A(\spn{\set{\seq{w}{1,2}}}))         &  & \by{ex:6.9.3}[a] \\
                                                 & = \T_v^*(\spn{\set{\seq{w}{1,2}}})               &  & \by{2.3.1}       \\
                                                 & = \T_v^*(\spn{\set{\seq{e}{1,4}}})               &  & \by{ex:6.9.3}[a] \\
                                                 & \subseteq \spn{\set{\seq{e}{1,4}}},              &  & \by{6.39}[d]
  \end{align*}
  by \cref{5.4.1} we know that \(\spn{\set{\seq{e}{1,4}}}\) is \(\T_v^* \L_A \T_v\)-invariant.
\end{proof}

\section{Conditioning and the Rayleigh Quotient}\label{sec:6.10}

\begin{defn}\label{6.10.1}
  In \cref{sec:3.4}, we studied specific techniques that allow us to solve systems of linear equations in the form \(Ax = b\), where \(A \in \ms\) and \(b \in \ms[m][1][\F]\).
  Such systems often arise in applications to the real world.
  The coefficients in the system are frequently obtained from experimental data, and, in many cases, both \(m\) and \(n\) are so large that a computer must be used in the calculation of the solution.
  Thus two types of errors must be considered.
  First, experimental errors arise in the collection of data since no instruments can provide completely accurate measurements.
  Second, computers introduce roundoff errors.
  One might intuitively feel that small relative changes in the coefficients of the system cause small relative errors in the solution.
  A system that has this property is called \textbf{well-conditioned};
  otherwise, the system is called \textbf{ill-conditioned}.
\end{defn}

\begin{note}
  We now consider several examples of these types of errors, concentrating primarily on changes in \(b\) rather than on changes in the entries of \(A\).
  In addition, we assume that \(A\) is a square, complex (or real), invertible matrix since this is the case most frequently encountered in applications.

  Of course, we are really interested in \emph{relative changes} since a change in the solution of, say, \(10\), is considered large if the original solution is of the order \(10^{-2}\), but small if the original solution is of the order \(10^6\).
\end{note}

\begin{defn}\label{6.10.2}
  We use the notation \(\delta b\) to denote the vector \(b' - b\), where \(b\) is the vector in the original system and \(b'\) is the vector in the modified system.
  We now define the \textbf{relative change} in \(b\) to be the scalar \(\dfrac{\norm{\delta b}}{\norm{b}}\), where \(\norm{\cdot}\) denotes the standard norm on \(\C^n\) (or \(\R^n\));
  that is, \(\norm{b} = \sqrt{\inn{b, b}}\).
  Most of what follows, however, is true for any norm.
  Similar definitions hold for the \textbf{relative change} in \(x\).
\end{defn}

\begin{note}
  If the lines defined by the two equations are nearly coincident, then a small change in either line could greatly alter the point of intersection, that is, the solution to the system.
\end{note}

\begin{defn}\label{6.10.3}
  Let \(A\) be a complex (or real) \(n \times n\) matrix.
  Define the \textbf{(Euclidean) norm} of \(A\) by
  \[
    \norm{A} = \max_{x \neq \zv} \dfrac{\norm{Ax}}{\norm{x}},
  \]
  where \(x \in \C^n\) or \(x \in \R^n\).
\end{defn}

\begin{note}
  Intuitively, \(\norm{A}\) represents the maximum \emph{magnification} of a vector by the matrix \(A\).
  The question of whether or not this maximum exists, as well as the problem of how to compute it, can be answered by the use of the so-called \emph{Rayleigh quotient}.
\end{note}

\begin{defn}\label{6.10.4}
  Let \(B\) be an \(n \times n\) self-adjoint matrix.
  The \textbf{Rayleigh quotient} for \(x \neq \zv\) is defined to be the scalar \(R(x) = \dfrac{\inn{Bx, x}}{\norm{x}^2}\).
\end{defn}

\begin{thm}\label{6.43}
  For a self-adjoint matrix \(B \in \ms[n][n][\F]\), we have that \(\max_{x \neq \zv} R(x)\) is the largest eigenvalue of \(B\) and \(\min_{x \neq \zv} R(x)\) is the smallest eigenvalue of \(B\).
\end{thm}

\begin{proof}[\pf{6.10.4}]
  By \cref{6.19,6.20}, we may choose an orthonormal basis \(\set{\seq{v}{1,,n}}\) of eigenvectors of \(B\) over \(\F\) such that \(B v_i = \lambda_i v_i\) (\(i \in \set{1, \dots, n}\)), where \(\seq[\geq]{\lambda}{1,,n}\).
  (Recall that by \cref{6.4.10}(a), the eigenvalues of \(B\) are real.)
  Now, for \(x \in \vs{F}^n\), there exist scalars \(\seq{a}{1,,n} \in \F\) such that
  \[
    x = \sum_{i = 1}^n a_i v_i.
  \]
  Hence
  \begin{align*}
    R(x) & = \dfrac{\inn{Bx, x}}{\norm{x}^2}                                                           &  & \by{6.10.4}     \\
         & = \dfrac{\inn{\sum_{i = 1}^n a_i \lambda_i v_i, \sum_{j = 1}^n a_j v_j}}{\norm{x}^2}        &  & \by{5.1.2}      \\
         & = \dfrac{\sum_{i = 1}^n a_i \lambda_i \inn{v_i, \sum_{j = 1}^n a_j v_j}}{\norm{x}^2}        &  & \by{6.1.1}[a,b] \\
         & = \dfrac{\sum_{i = 1}^n \sum_{j = 1}^n a_i \conj{a_j} \lambda_i \inn{v_i, v_j}}{\norm{x}^2} &  & \by{6.1}[a,b]   \\
         & = \dfrac{\sum_{i = 1}^n a_i \conj{a_i} \lambda_i}{\norm{x}^2}                               &  & \by{6.1.12}     \\
         & = \dfrac{\sum_{i = 1}^n \lambda_i \abs{a_i}^2}{\norm{x}^2}                                  &  & \by{d.0.5}      \\
         & \leq \dfrac{\lambda_1 \sum_{i = 1}^n \abs{a_i}^2}{\norm{x}^2}                               &  & \by{6.2}[b]     \\
         & = \dfrac{\lambda_1 \norm{x}^2}{\norm{x}^2}                                                  &  & \by{ex:6.1.12}  \\
         & = \lambda_1.
  \end{align*}
  It is easy to see that \(R(v_1) = \lambda_1\), so we have demonstrated the first half of the theorem.
  The second half is proved similarly, i.e.,
  \begin{align*}
    R(x) & = \dfrac{\sum_{i = 1}^n \lambda_i \abs{a_i}^2}{\norm{x}^2}    &  & \text{(from the proof above)} \\
         & \geq \dfrac{\lambda_n \sum_{i = 1}^n \abs{a_i}^2}{\norm{x}^2} &  & \by{6.2}[b]                   \\
         & = \dfrac{\lambda_n \norm{x}^2}{\norm{x}^2}                    &  & \by{ex:6.1.12}                \\
         & = \lambda_n.
  \end{align*}
\end{proof}

\begin{cor}\label{6.10.5}
  For any square matrix \(A\), \(\norm{A}\) is finite and, in fact, equals \(\sqrt{\lambda}\), where \(\lambda\) is the largest eigenvalue of \(A^* A\).
\end{cor}

\begin{proof}[\pf{6.10.5}]
  Let \(B\) be the self-adjoint matrix \(A^* A\) (\cref{ex:6.4.18}(a)), and let \(\lambda\) be the largest eigenvalue of \(B\) (\cref{ex:6.4.17}(a)).
  Since, for \(x \neq \zv\),
  \begin{align*}
    0 & \leq \dfrac{\norm{Ax}^2}{\norm{x}^2}   &  & \by{6.2}[b] \\
      & = \dfrac{\inn{Ax, Ax}}{\norm{x}^2}     &  & \by{6.1.9}  \\
      & = \dfrac{\inn{A^* A x, x}}{\norm{x}^2} &  & \by{6.3.4}  \\
      & = \dfrac{\inn{Bx, x}}{\norm{x}^2}                       \\
      & = R(x),                                &  & \by{6.10.4}
  \end{align*}
  it follows from \cref{6.43} that \(\norm{A}^2 = \lambda\).
\end{proof}

\begin{note}
  Observe that the proof of \cref{6.10.5} shows that all the eigenvalues of \(A^* A\) are nonnegative
  (positive semidefinite by \cref{ex:6.4.17}(a)).
\end{note}

\begin{lem}\label{6.10.6}
  For any square matrix \(A\), \(\lambda\) is an eigenvalue of \(A^* A\) iff \(\lambda\) is an eigenvalue of \(A A^*\).
\end{lem}

\begin{proof}[\pf{6.10.6}]
  Let \(\lambda\) be an eigenvalue of \(A^* A\).
  If \(\lambda = 0\), then \(A^* A\) is not invertible.
  Hence \(A\) and \(A^*\) are not invertible (\cref{6.3.6}), so that \(\lambda\) is also an eigenvalue of \(A A^*\).
  The proof of the converse is similar.

  Now suppose that \(\lambda \neq 0\) and there exists \(x \neq \zv\) such that \(A^* A x = \lambda x\).
  Apply \(A\) to both sides to obtain
  \begin{align*}
    A A^* A x & = A (\lambda x) &  & \by{5.1.2}   \\
              & = \lambda (Ax). &  & \by{2.12}[b]
  \end{align*}
  Note that \(Ax \neq \zv\), otherwise we would have \(A^* A x = A^* \zv = \zv \neq \lambda x\).
  So we have that \(\lambda\) is an eigenvalue of \(A A^*\) and \(Ax\) is an eigenvector of \(A A^*\).

  Finally suppose that \(\lambda \neq 0\) and \(A A^* x = \lambda x\) for some \(x \neq \zv\).
  Apply \(A^*\) to both sides to obtain
  \begin{align*}
    A^* A A^* x & = A^* (\lambda x) &  & \by{5.1.2}   \\
                & = \lambda A^* x.  &  & \by{2.12}[b]
  \end{align*}
  Note that \(A^* x \neq \zv\), otherwise we would have \(A A^* x = A \zv = \zv \neq \lambda x\).
  So we have that \(\lambda\) is an eigenvalue of \(A^* A\) and \(A^* x\) is an eigenvector of \(A^* A\).
\end{proof}

\begin{cor}\label{6.10.7}
  Let \(A\) be an invertible matrix.
  Then \(\norm{A^{-1}} = 1 / \sqrt{\lambda}\), where \(\lambda\) is the smallest eigenvalue of \(A^* A\).
\end{cor}

\begin{proof}[\pf{6.10.7}]
  Recall that \(\lambda\) is an eigenvalue of an invertible matrix iff \(\lambda^{-1}\) is an eigenvalue of its inverse (\cref{ex:5.1.8}(b)).

  Now let \(\seq[\geq]{\lambda}{1,,n}\) be the eigenvalues of \(A^* A\), which by \cref{6.10.6} are the eigenvalues of \(A A^*\).
  Then \(\norm{A^{-1}}^2\) equals the largest eigenvalue of \((A A^*)^{-1}\) which equals \(1 / \lambda_n\).
  This is true since
  \begin{align*}
    \pa{A^{-1}}^* A^{-1} & = \pa{A^*}^{-1} A^{-1} &  & \by{ex:6.3.8} \\
                         & = (A A^*)^{-1}.        &  & \by{ex:2.4.4}
  \end{align*}
\end{proof}

\begin{note}
  For many applications, it is only the largest and smallest eigenvalues that are of interest.
  For example, in the case of vibration problems, the smallest eigenvalue represents the lowest frequency at which vibrations can occur.
\end{note}

\begin{cor}\label{6.10.8}
  Now that we know \(\norm{A}\) exists for every square matrix \(A\) (\cref{6.10.5}), we can make use of the inequality \(\norm{Ax} \leq \norm{A} \cdot \norm{x}\), which holds for every \(x\).
\end{cor}

\begin{proof}[\pf{6.10.8}]
  Suppose that \(x = \zv\).
  Then we have
  \begin{align*}
             & \zv = A \zv                                                    &  & \by{2.1.2}[a] \\
    \implies & 0 = \norm{\zv} = \norm{A \zv} = \norm{A} \cdot \norm{\zv} = 0. &  & \by{6.2}[b]
  \end{align*}

  Now suppose that \(x \neq \zv\).
  Then we have
  \begin{align*}
             & \norm{A} = \max_{y \neq \zv} \dfrac{\norm{Ay}}{\norm{y}} \geq \dfrac{\norm{Ax}}{\norm{x}} &  & \by{6.10.3} \\
    \implies & \norm{Ax} \leq \norm{A} \cdot \norm{x}.
  \end{align*}
\end{proof}

\begin{cor}\label{6.10.9}
  Assume in what follows that \(A\) is invertible, \(b \neq \zv\), and \(Ax = b\).
  For a given \(\delta b\), let \(\delta x\) be the vector that satisfies \(A(x + \delta x) = b + \delta b\).
  Prove that \(A(\delta x) = \delta b\), \(\delta x = A^{-1} (\delta b)\), and
  \[
    \dfrac{1}{\norm{A} \cdot \norm{A^{-1}}} \cdot \dfrac{\norm{\delta b}}{\norm{b}} \leq \dfrac{\norm{\delta x}}{\norm{x}} \leq \norm{A} \cdot \norm{A^{-1}} \cdot \dfrac{\norm{\delta b}}{\norm{b}}.
  \]
\end{cor}

\begin{proof}[\pf{6.10.9}]
  Since
  \begin{align*}
             & A (x + \delta x) = b + \delta b                 \\
    \implies & b + A (\delta x) = b + \delta b &  & (Ax = b)   \\
    \implies & A (\delta x) = \delta b         &  & \by{1.1}   \\
    \implies & (\delta x) = A^{-1} (\delta b), &  & \by{2.4.4}
  \end{align*}
  by \cref{6.10.8} we have
  \begin{align*}
    \norm{b}        & = \norm{Ax} \leq \norm{A} \cdot \norm{x};                            \\
    \norm{\delta b} & = \norm{A (\delta x)} \leq \norm{A} \cdot \norm{\delta x};           \\
    \norm{x}        & = \norm{A^{-1} b} \leq \norm{A^{-1}} \cdot \norm{b};                 \\
    \norm{\delta x} & = \norm{A^{-1} (\delta b)} \leq \norm{A^{-1}} \cdot \norm{\delta b}.
  \end{align*}
  Since \(b \neq \zv\), by \cref{2.4} we have \(x \neq \zv\).
  Thus
  \begin{align*}
    \dfrac{\norm{\delta x}}{\norm{x}} & \leq \norm{A} \cdot \dfrac{\norm{\delta x}}{\norm{b}}                     &  & (\norm{b} \leq \norm{A} \cdot \norm{x})                    \\
                                      & \leq \norm{A} \cdot \norm{A^{-1}} \cdot \dfrac{\norm{\delta b}}{\norm{b}} &  & (\norm{\delta x} \leq \norm{A^{-1}} \cdot \norm{\delta b})
  \end{align*}
  and
  \begin{align*}
    \dfrac{\norm{\delta x}}{\norm{x}} & \geq \dfrac{1}{\norm{A^{-1}}} \cdot \dfrac{\norm{\delta x}}{\norm{b}}                 &  & (\norm{x} \leq \norm{A^{-1}} \cdot \norm{b})          \\
                                      & \geq \dfrac{1}{\norm{A} \cdot \norm{A^{-1}}} \cdot \dfrac{\norm{\delta b}}{\norm{b}}. &  & (\norm{\delta b} \leq \norm{A} \cdot \norm{\delta x})
  \end{align*}
\end{proof}

\begin{defn}\label{6.10.10}
  The number \(\norm{A} \cdot \norm{A^{-1}}\) is called the \textbf{condition number} of \(A\) and is denoted \(\cond{A}\).
  It should be noted that the definition of \(\cond{A}\) depends on how the norm of \(A\) is defined.
  There are many reasonable ways of defining the norm of a matrix.
  In fact, the only property needed to establish the inequalities above is that \(\norm{Ax} \leq \norm{A} \cdot \norm{x}\) for all \(x\).
  We summarize these results in \cref{6.44}.
\end{defn}

\begin{thm}\label{6.44}
  For the system \(Ax = b\), where \(A\) is invertible and \(b \neq \zv\), the following statements are true.
  \begin{enumerate}
    \item For any norm \(\norm{\cdot}\), we have
          \[
            \dfrac{1}{\cond{A}} \dfrac{\norm{\delta b}}{\norm{b}} \leq \dfrac{\norm{\delta x}}{\norm{x}} \leq \cond{A} \dfrac{\norm{\delta b}}{\norm{b}}.
          \]
    \item If \(\norm{\cdot}\) is the Euclidean norm, then \(\cond{A} = \sqrt{\dfrac{\lambda_1}{\lambda_n}}\), where \(\lambda_1\) and \(\lambda_n\) are the largest and smallest eigenvalues, respectively, of \(A^* A\).
    \item \(\cond{A} \geq 1\).
  \end{enumerate}
\end{thm}

\begin{proof}[\pf{6.44}(a)]
  See \cref{6.10.9,6.10.10}.
\end{proof}

\begin{proof}[\pf{6.44}(b)]
  See \cref{6.10.5,6.10.7,6.10.10}.
\end{proof}

\begin{proof}[\pf{6.44}(c)]
  This follows from \cref{6.44}(b).
\end{proof}

\begin{note}
  In \cref{ex:6.10.11} we show that \(\cond{A} = 1\) iff \(A\) is a scalar multiple of a unitary or orthogonal matrix.
  Moreover, it can be shown with some work that equality can be obtained in \cref{6.44}(a) by an appropriate choice of \(b\) and \(\delta b\).

  We can see immediately from \cref{6.44}(a) that if \(\cond{A}\) is close to \(1\), then a small relative error in \(b\) forces a small relative error in \(x\).
  If \(\cond{A}\) is large, however, then the relative error in \(x\) may be small even though the relative error in \(b\) is large, or the relative error in \(x\) may be large even though the relative error in \(b\) is small!
  In short, \(\cond{A}\) merely indicates the \emph{potential} for large relative errors.

  We have so far considered only errors in the vector \(b\).
  If there is an error \(\delta A\) in the coefficient matrix of the system \(Ax = b\), the situation is more complicated.
  For example, \(A + \delta A\) may fail to be invertible.
  But under the appropriate assumptions, it can be shown that a bound for the relative error in \(x\) can be given in terms of \(\cond{A}\).
  For example, Charles Cullen (Charles G. Cullen, An Introduction to Numerical Linear Algebra, PWS Publishing Co., Boston 1994, p. 60) shows that if \(A + \delta A\) is invertible, then
  \[
    \dfrac{\norm{\delta x}}{\norm{x + \delta x}} \leq \cond{A} \dfrac{\norm{\delta A}}{\norm{A}}.
  \]

  It should be mentioned that, in practice, one never computes \(\cond{A}\) from its definition, for it would be an unnecessary waste of time to compute \(A^{-1}\) merely to determine its norm.
  In fact, if a computer is used to find \(A^{-1}\), the computed inverse of \(A\) in all likelihood only approximates \(A^{-1}\), and the error in the computed inverse is affected by the size of \(\cond{A}\).
  So we are caught in a vicious circle!
  There are, however, some situations in which a usable approximation of \(\cond{A}\) can be found.
  Thus, in most cases, the estimate of the relative error in \(x\) is based on an estimate of \(\cond{A}\).
\end{note}

\exercisesection

\begin{ex}\label{ex:6.10.11}
\end{ex}

\section{The Geometry of Orthogonal Operators}\label{sec:6.11}

\begin{note}
  By \cref{6.22}, any rigid motion on a finite-dimensional real inner product space is the composite of an orthogonal operator and a translation.
  Thus, to understand the geometry of rigid motions thoroughly, we must analyze the structure of orthogonal operators.
  Such is the aim of this section.
  We show that any orthogonal operator on a finite-dimensional real inner product space is the composite of rotations and reflections.
\end{note}

\begin{defn}\label{6.11.1}
  Let \(\T\) be a linear operator on a finite-dimensional real inner product space \(\V\).
  The operator \(\T\) is called a \textbf{rotation} if \(\T\) is the identity on \(\V\) or if there exists a two-dimensional subspace \(\W\) of \(\V\) over \(\R\), an orthonormal basis \(\beta = \set{\seq{x}{1,2}}\) for \(\W\) over \(\R\), and a real number \(\theta\) such that
  \[
    \T(x_1) = \cos(\theta) x_1 + \sin(\theta) x_2, \quad \T(x_2) = -\sin(\theta) x_1 + \cos(\theta) x_2,
  \]
  and \(\T(y) = y\) for all \(y \in \W^{\perp}\).
  In this context, \(\T\) is called a \textbf{rotation of \(\W\) about \(\W^{\perp}\)}.
  The subspace \(\W^{\perp}\) is called the \textbf{axis of rotation}.
\end{defn}

\begin{defn}\label{6.11.2}
  Let \(\T\) be a linear operator on a finite-dimensional real inner product space \(\V\).
  The operator \(\T\) is called a \textbf{reflection} if there exists a one-dimensional subspace \(\W\) of \(\V\) over \(\R\) such that \(\T(x) = -x\) for all \(x \in \W\) and \(\T(y) = y\) for all \(y \in \W^{\perp}\).
  In this context, \(\T\) is called a \textbf{reflection of \(\V\) about \(\W^{\perp}\)}.
\end{defn}

\begin{note}
  It should be noted that rotations and reflections (or composites of these) are orthogonal operators (see \cref{ex:6.11.2}).
  The principal aim of this section is to establish that the converse is also true, that is, any orthogonal operator on a finite-dimensional real inner product space is the composite of rotations and reflections.
\end{note}

\begin{eg}\label{6.11.3}
  A Characterization of Orthogonal Operators on a One-Dimensional Real Inner Product Space.

  Let \(\T\) be an orthogonal operator on a one-dimensional real inner product space \(\V\).
  Choose any nonzero vector \(x\) in \(\V\).
  Then \(\V = \spn{\set{x}}\), and so \(\T(x) = \lambda x\) for some \(\lambda \in \R\).
  Since \(\T\) is orthogonal and \(\lambda\) is an eigenvalue of \(\T\), \(\lambda = \pm 1\) (see \cref{6.2}(a) and \cref{6.5.1}).
  If \(\lambda = 1\), then \(\T\) is the identity on \(\V\), and hence \(\T\) is a rotation (\cref{6.11.1}).
  If \(\lambda = -1\), then \(\T(x) = -x\) for all \(x \in \V\);
  so \(\T\) is a reflection of \(\V\) about \(\V^{\perp} = \set{\zv}\) (\cref{6.2.11,6.11.2}).
  Thus \(\T\) is either a rotation or a reflection.
  Note that in the first case, \(\det(\T) = 1\), and in the second case, \(\det(\T) = -1\) (\cref{ex:5.1.7}).
\end{eg}

\begin{cor}\label{6.11.4}
  Let \(\V\) be an \(n\)-dimensional vector space over \(\R\).
  Let \(\T \in \ls(\V)\) be a rotation.
  Then there exists an orthonormal basis \(\gamma\) for \(\V\) over \(\R\) and a \(\theta \in \R\) such that
  \[
    [\T]_{\gamma} = \begin{pmatrix}
      \cos(\theta) & -\sin(\theta) & \zm       \\
      \sin(\theta) & \cos(\theta)  & \zm       \\
      \zm          & \zm           & I_{n - 2}
    \end{pmatrix}.
  \]
\end{cor}

\begin{proof}[\pf{6.11.4}]
  Following the notation of \cref{6.11.1}, let \(\gamma = \set{\seq{x}{1,2}, \seq{v}{1,,n-2}}\) be an orthonormal basis for \(\V\) over \(\R\), where \(\set{\seq{v}{1,,n-2}}\) is an orthonormal basis for \(\W^{\perp}\).
  By \cref{6.5} such \(\gamma\) must exist.
  The proof follows from \cref{2.2.4,6.11.1}.
\end{proof}

\begin{thm}\label{6.45}
  Let \(\T\) be an orthogonal operator on a two-dimensional real inner product space \(\V\).
  Then \(\T\) is either a rotation or a reflection.
  Furthermore, \(\T\) is a rotation iff \(\det(\T) = 1\), and \(\T\) is a reflection iff \(\det(\T) = -1\).
\end{thm}

\begin{proof}[\pf{6.45}]
  Let \(\inn{\cdot, \cdot}\) be an inner product on \(\V\) over \(\R\) and let \(\inn{\cdot, \cdot}'\) be the standard inner product on \(\R^2\) over \(\R\).
  Let \(\norm{\cdot}\) be a norm on \(\V\) over \(\R\) such that \(\norm{\cdot}^2 = \inn{\cdot, \cdot}\).
  Let \(\norm{\cdot}'\) be a norm on \(\R^2\) over \(\R\) such that \((\norm{\cdot}')^2 = \inn{\cdot, \cdot}'\).
  Let \(\T \in \ls(\V)\) be orthogonal with respect to \(\norm{\cdot}\) and let \(\beta = \set{\seq{v}{1,2}}\) be an orthonormal basis with respect to \(\inn{\cdot, \cdot}\) for \(\V\) over \(\R\).
  Let \(\gamma\) be the standard ordered basis for \(\R^2\) over \(\R\).
  Define \(\phi_{\beta} \in \ls(\V, \R^2)\) as in \cref{2.4.11}.

  Because \(\T\) is an orthogonal operator, \(\T(\beta)\) is an orthonormal basis for \(\V\) over \(\R\) by \cref{6.18}(c).
  Since
  \begin{align*}
    1 & = \norm{v_1}^2                                      &  & \by{6.1.12}    \\
      & = \norm{\T(v_1)}^2                                  &  & \by{6.5.1}     \\
      & = \inn{\T(v_1), \T(v_1)}                            &  & \by{6.1.9}     \\
      & = \inn{\phi_{\beta} \T(v_1), \phi_{\beta} \T(v_1)}' &  & \by{ex:6.2.15} \\
      & = (\norm{\phi_{\beta} \T(v_1)}')^2,                 &  & \by{6.1.9}
  \end{align*}
  we know that \(\phi_{\beta} \T(v_1)\) is a unit vector and there is an unique angle \(\theta \in [0, 2 \pi)\) such that \(\phi_{\beta} \T(v_1) = (\cos(\theta), \sin(\theta))\).
  Similarly \(\phi_{\beta} \T(v_2)\) is a unit vector and is orthogonal to \(\phi_{\beta} \T(v_1)\) (\cref{ex:6.2.15}), there are only two possible choices for \(\phi_{\beta} \T(v_2)\).
  Either
  \[
    \phi_{\beta} \T(v_2) = (-\sin(\theta), \cos(\theta)) \quad \text{or} \quad \phi_{\beta} \T(v_2) = (\sin(\theta), -\cos(\theta)).
  \]
  First, suppose that \(\phi_{\beta} \T(v_2) = (-\sin(\theta), \cos(\theta))\).
  Then
  \begin{align*}
    \begin{pmatrix}
      \cos(\theta) & -\sin(\theta) \\
      \sin(\theta) & \cos(\theta)
    \end{pmatrix} & = [\phi_{\beta} \T]_{\beta}^{\gamma}           &  & \by{2.2.4}                   \\
                                    & = [\phi_{\beta}]_{\beta}^{\gamma} [\T]_{\beta} &  & \by{2.11}  \\
                                    & = I_2 [\T]_{\beta}                             &  & \by{2.2.4} \\
                                    & = [\T]_{\beta}.                                &  & \by{2.3.8}
  \end{align*}
  It follows from \cref{6.11.1} that \(\T\) is a rotation of \(\V\) about \(\V^{\perp} = \set{\zv}\) (\cref{6.2.11}).
  Also
  \begin{align*}
    \det(\T) & = \det([\T]_{\beta})                       &  & \by{ex:5.1.7} \\
             & = (\cos(\theta))^2 + (\sin(\theta))^2 = 1. &  & \by{4.1.1}
  \end{align*}
  Now suppose that \(\phi_{\beta} \T(v_2) = (\sin(\theta), -\cos(\theta))\).
  Then
  \[
    \begin{pmatrix}
      \cos(\theta) & \sin(\theta)  \\
      \sin(\theta) & -\cos(\theta)
    \end{pmatrix} = [\phi_{\beta} \T]_{\beta}^{\gamma} = [\T]_{\beta}.
  \]
  It follows from \cref{6.11.2} that \(\T\) is a reflection on \(\V\) about
  \[
    \W^{\perp} = \spn{\set{\sin(\theta) v_1 + (1 - \cos(\theta)) v_2}}
  \]
  where
  \[
    \W = \spn{\set{(\cos(\theta) - 1) v_1 + \sin(\theta) v_2}}.
  \]
  Furthermore,
  \begin{align*}
    \det(\T) & = \det([\T]_{\beta})                          &  & \by{ex:5.1.7} \\
             & = - (\cos(\theta))^2 - (\sin(\theta))^2 = -1. &  & \by{4.1.1}
  \end{align*}
\end{proof}

\begin{cor}\label{6.11.5}
  Let \(\V\) be a two-dimensional real inner product space.
  The composite of a reflection and a rotation on \(\V\) is a reflection on \(\V\).
\end{cor}

\begin{proof}[\pf{6.11.5}]
  If \(\T_1\) is a reflection on \(\V\) and \(\T_2\) is a rotation on \(\V\), then by \cref{6.45}, \(\det(\T_1) = 1\) and \(\det(\T_2) = -1\).
  Let \(\T = \T_2 \T_1\) be the composite.
  Since \(\T_2\) and \(\T_1\) are orthogonal, so is \(\T\) (\cref{ex:6.5.3}).
  Moreover,
  \begin{align*}
    \det(\T) & = \det(\T_2) \cdot \det(\T_1) &  & \by{ex:5.1.7}[d] \\
             & = -1.                         &  & \by{6.45}
  \end{align*}
  Thus, by \cref{6.45}, \(\T\) is a reflection.
  The proof for \(\T_1 \T_2\) is similar, i.e.,
  \begin{align*}
    \det(\T_1 \T_2) & = \det(\T_1) \cdot \det(\T_2) &  & \by{ex:5.1.7}[d] \\
                    & = -1.
  \end{align*}
\end{proof}

\begin{lem}\label{6.11.6}
  If \(\T\) is a linear operator on a nonzero finite-dimensional real vector space \(\V\), then there exists a \(\T\)-invariant subspace \(\W\) of \(\V\) over \(\R\) such that \(1 \leq \dim(\W) \leq 2\).
\end{lem}

\begin{proof}[\pf{6.11.6}]
  Fix an ordered basis \(\beta = \set{\seq{y}{1,,n}}\) for \(\V\) over \(\R\), and let \(A = [\T]_{\beta}\).
  Let \(\phi_{\beta} \in \ls(\V, \R^n)\) be the isomorphism defined by \cref{2.4.11}.
  By \cref{2.4.12} \(\L_A \phi_{\beta} = \phi_{\beta} \T\).
  As a consequence, it suffices to show that there exists an \(\L_A\)-invariant subspace \(\vs{Z}\) of \(\R^n\) such that \(1 \leq \dim(\vs{Z}) \leq 2\).
  If we then define \(\W = \phi_{\beta}^{-1}(\vs{Z})\), it follows that \(\W\) satisfies the conclusions of the lemma (see \cref{ex:6.11.13}).

  The matrix \(A\) can be considered as an \(n \times n\) matrix over \(\C\) and, as such, can be used to define a linear operator \(\U\) on \(\C^n\) by \(\U(v) = Av\).
  Since \(\U\) is a linear operator on a finite-dimensional vector space over \(\C\), it has an eigenvalue \(\lambda \in \C\) (\cref{5.2,d.4}).
  Let \(x \in \C^n\) be an eigenvector corresponding to \(\lambda\).
  We may write \(\lambda = \lambda_1 + i \lambda_2\), where \(\lambda_1\) and \(\lambda_2\) are real, and
  \[
    x = \begin{pmatrix}
      a_1 + i b_1 \\
      \vdots      \\
      a_n + i b_n
    \end{pmatrix},
  \]
  where the \(a_i\)'s and \(b_i\)'s are real.
  Thus, setting
  \[
    x_1 = \begin{pmatrix}
      a_1    \\
      \vdots \\
      a_n
    \end{pmatrix} \quad \text{and} \quad x_2 = \begin{pmatrix}
      b_1    \\
      \vdots \\
      b_n
    \end{pmatrix},
  \]
  we have \(x = x_1 + i x_2\), where \(x_1\) and \(x_2\) have real entries.
  Note that at least one of \(x_1\) or \(x_2\) is nonzero since \(x \neq \zv\).
  Hence
  \begin{align*}
    \U(x) & = \lambda x                                                            \\
          & = (\lambda_1 + i \lambda_2) (x_1 + i x_2)                              \\
          & = (\lambda_1 x_1 - \lambda_2 x_2) + i (\lambda_1 x_2 + \lambda_2 x_1).
  \end{align*}
  Similarly,
  \[
    \U(x) = A (x_1 + i x_2) = A x_1 + i A x_2.
  \]
  Comparing the real and imaginary parts of these two expressions for \(\U(x)\), we conclude that
  \[
    A x_1 = \lambda_1 x_1 - \lambda_2 x_2 \quad \text{and} \quad A x_2 = \lambda_1 x_2 + \lambda_2 x_1.
  \]
  Finally, let \(\vs{Z} = \spn{\set{\seq{x}{1,2}}}\), the span being taken as a subspace of \(\R^n\) over \(\R\).
  Since \(x_1 \neq \zv\) or \(x_2 \neq \zv\), \(\vs{Z}\) is a nonzero subspace.
  Thus \(1 \leq \dim(\vs{Z}) \leq 2\), and the preceding pair of equations shows that \(\vs{Z}\) is \(\L_A\)-invariant.
\end{proof}

\begin{thm}\label{6.46}
  Let \(\T\) be an orthogonal operator on a nonzero finite-dimensional real inner product space \(\V\).
  Then there exists a collection of pairwise orthogonal \(\T\)-invariant subspaces \(\set{\seq{\W}{1,,m}}\) of \(\V\) such that
  \begin{enumerate}
    \item \(1 \leq \dim(\W_i) \leq 2\) for \(i \in \set{1, \dots, m}\).
    \item \(\V = \seq[\oplus]{\W}{1,,m}\).
  \end{enumerate}
\end{thm}

\begin{proof}[\pf{6.46}]
  The proof is by mathematical induction on \(\dim(\V)\).
  If \(\dim(\V) = 1\), the result is obvious.
  So assume that the result is true whenever \(\dim(\V) = n\) for some fixed integer \(n \geq 1\).

  Suppose \(\dim(\V) = n + 1\).
  By \cref{6.11.6}, there exists a \(\T\)-invariant subspace \(\W_1\) of \(\V\) such that \(1 \leq \dim(\W_1) \leq 2\).
  If \(\W_1 = \V\), the result is established.
  Otherwise, \(\W_1^{\perp} \neq \set{\zv}\) (\cref{6.2.11}).
  By \cref{ex:6.11.14}, \(\W_1^{\perp}\) is \(\T\)-invariant and the restriction of \(\T\) to \(\W_1^{\perp}\) is orthogonal.
  Since \(\dim(\W_1^{\perp}) \leq n\), we may apply the induction hypothesis to \(\T_{\W_1^{\perp}}\) and conclude that there exists a collection of pairwise orthogonal \(\T\)-invariant subspaces \(\set{\seq{\W}{2,,m}}\) of \(\W_1^{\perp}\) such that \(1 \leq \dim(\W_i) \leq 2\) for \(i \in \set{2, \dots, m}\) and \(\W_1^{\perp} = \seq[\oplus]{\W}{2,,m}\).
  Thus \(\set{\seq{\W}{1,,m}}\) is pairwise orthogonal, and by \cref{ex:6.2.13}(d),
  \[
    \V = \W_1 \oplus \W_1^{\perp} = \seq[\oplus]{\W}{1,,m}.
  \]
\end{proof}

\begin{note}
  Applying \cref{6.11.3,6.45} in the context of \cref{6.46}, we conclude that the restriction of \(\T\) to \(\W_i\) is either a rotation or a reflection for each \(i \in \set{1, \dots, m}\).
  Thus, in some sense, \(\T\) is composed of rotations and reflections.
  Unfortunately, very little can be said about the uniqueness of the decomposition of \(\V\) in \cref{6.46}.
  For example, the \(\W_i\)'s, the number \(m\) of \(\W_i\)'s, and the number of \(\W_i\)'s for which \(\T_{\W_i}\) is a reflection are not unique.
  Although the number of \(\W_i\)'s for which \(\T_{\W_i}\) is a reflection is not unique, whether this number is even or odd is an intrinsic property of \(\T\).
  Moreover, we can always decompose \(\V\) so that \(\T_{\W_i}\) is a reflection for at most one \(\W_i\).
  These facts are established in \cref{6.47}.
\end{note}

\begin{thm}\label{6.47}
  Let \(\T, \V, \seq{\W}{1,,m}\) be as in \cref{6.46}.
  \begin{enumerate}
    \item The number of \(\W_i\)'s for which \(\T_{\W_i}\) is a reflection is even or odd according to whether \(\det(\T) = 1\) or \(\det(\T) = -1\).
    \item It is always possible to decompose \(\V\) as in \cref{6.46} so that the number of \(\W_i\)'s for which \(\T_{\W_i}\) is a reflection is zero or one according to whether \(\det(\T) = 1\) or \(\det(\T) = -1\).
          Furthermore, if \(\T_{\W_i}\) is a reflection, then \(\dim(\W_i) = 1\).
  \end{enumerate}
\end{thm}

\begin{proof}[\pf{6.47}(a)]
  Let \(r\) denote the number of \(\W_i\)'s in the decomposition for which \(\T_{\W_i}\) is a reflection.
  Then we have
  \begin{align*}
    \det(\T) & = \det(\T_{\W_1}) \cdots \det(\T_{\W_m}) &  & \by{ex:6.11.15}  \\
             & = 1^{m - r} \cdot (-1)^r                 &  & \by{6.11.3,6.45} \\
             & = (-1)^r.
  \end{align*}
\end{proof}

\begin{proof}[\pf{6.47}(b)]
  Let \(E = \set{x \in \V : \T(x) = -x}\);
  then \(E\) is a \(\T\)-invariant subspace of \(\V\) over \(\R\).
  If \(\W = E^{\perp}\), then \(\W\) is \(\T\)-invariant (\cref{ex:6.11.14}(b)).
  So by applying \cref{6.46} to \(\T_{\W}\), we obtain a collection of pairwise orthogonal \(\T\)-invariant subspaces \(\set{\seq{\W}{1,,k}}\) of \(\W\) such that \(\W = \seq[\oplus]{\W}{1,,k}\) and for \(i \in \set{1, \dots, k}\), the dimension of each \(\W_i\) is either \(1\) or \(2\).
  Observe that, for each \(i \in \set{1, \dots, k}\), \(\T_{\W_i}\) is a rotation.
  For otherwise, if \(\T_{\W_i}\) is a reflection, there exists a nonzero \(x \in \W_i\) for which \(\T(x) = -x\).
  But then, \(x \in \W_i \cap E \subseteq E^{\perp} \cap E = \set{\zv}\), a contradiction.
  If \(E = \set{\zv}\), the result follows.
  Otherwise, choose an orthonormal basis \(\beta\) for \(E\) over \(\R\) containing \(p\) vectors (\(p > 0\)).
  It is possible to decompose \(\beta\) into a pairwise disjoint union \(\beta = \seq[\cup]{\beta}{1,,r}\) such that each \(\beta_i\) contains exactly two vectors for \(i \in \set{1, \dots, r - 1}\), and \(\beta_r\) contains two vectors if \(p\) is even and one vector if \(p\) is odd.
  For each \(i \in \set{1, \dots, r}\), let \(\W_{k + i} = \spn{\beta_i}\).
  Then, clearly, \(\set{\seq{\W}{1,,k,,k+r}}\) is pairwise orthogonal, and
  \[
    \V = \seq[\oplus]{\W}{1,,k,,k+r}.
  \]
  Moreover, if any \(\beta_i\) contains two vectors, then
  \begin{align*}
    \det(\T_{\W_{k + i}}) & = \det([\T_{\W_{k + i}}]_{\beta_i}) &  & \by{ex:5.1.7} \\
                          & = \det\begin{pmatrix}
                                    -1 & 0  \\
                                    0  & -1
                                  \end{pmatrix}               &  & (\T(x) = -x)    \\
                          & = 1.                                &  & \by{4.1.1}
  \end{align*}
  So \(\T_{\W_{k + i}}\) is a rotation, and hence \(\T_{\W_j}\) is a rotation for \(j \in \set{1, \dots, k + r - 1}\).
  If \(\beta_r\) consists of one vector, then \(\dim(\W_{k + r}) = 1\) and
  \begin{align*}
    \det(\T_{\W_{k + r}}) & = \det([\T_{\W_{k + r}}]_{\beta_r}) &  & \by{ex:5.1.7} \\
                          & = \det(-1)                          &  & (\T(x) = -x)  \\
                          & = -1.                               &  & \by{4.2.2}
  \end{align*}
  Thus \(\T_{\W_{k + r}}\) is a reflection by \cref{6.46}, and we conclude that the decomposition
  \[
    \V = \seq[\oplus]{\W}{1,,k,,k+r}
  \]
  satisfies the condition of (b).
\end{proof}

\begin{cor}\label{6.11.7}
  Let \(\T\) be an orthogonal operator on a finite-dimensional real inner product space \(\V\).
  Then there exists a collection \(\set{\seq{\T}{1,,m}}\) of orthogonal operators on \(\V\) such that the following statements are true.
  \begin{enumerate}
    \item For each \(i \in \set{1, \dots, m}\), \(\T_i\) is either a reflection or a rotation.
    \item For at most one \(i \in \set{1, \dots, m}\), \(\T_i\) is a reflection.
    \item \(\T_i \T_j = \T_j \T_i\) for all \(i, j \in \set{1, \dots, m}\).
    \item \(\T = \seq[]{\T}{1,,m}\).
    \item \[
            \det(\T) = \begin{dcases}
              1  & \text{if } \T_i \text{ is a rotation for each } i \in \set{1, \dots, m} \\
              -1 & \text{otherwise}.
            \end{dcases}
          \]
  \end{enumerate}
\end{cor}

\begin{proof}[\pf{6.11.7}]
  As in the proof of \cref{6.47}(b), we can write
  \[
    \V = \seq[\oplus]{\W}{1,,m},
  \]
  where \(\T_{\W_i}\) is a rotation for \(i \in \set{1, \dots, m - 1}\).
  For each \(i \in \set{1, \dots, m}\), define \(\T_i : \V \to \V\) by
  \[
    \T_i(\seq[+]{x}{1,,m}) = \seq[+]{x}{1,,i-1} + \T(x_i) + \seq[+]{x}{i+1,,m},
  \]
  where \(x_j \in \W_j\) for all \(j \in \set{1, \dots, m}\).

  First we show that \(\T_i\) is an orthogonal operator on \(\V\) for all \(i \in \set{1, \dots, m}\).
  Clearly \(\T_i \in \ls(\V)\).
  Since
  \begin{align*}
             & \forall x \in \V, \exists \tuple{x}{1,,m} \in \seq[\times]{\W}{1,,m} : x = \seq[+]{x}{1,,m}                  &  & \by{5.2.7}     \\
    \implies & \forall x \in \V, \norm{\T_i(x)}^2                                                                                               \\
             & = \norm{\seq[+]{x}{1,,i-1} + \T(x_i) + \seq[+]{x}{i+1,,m}}^2                                                                     \\
             & = \norm{x_1}^2 + \cdots + \norm{x_{i - 1}}^2 + \norm{\T(x_i)}^2 + \norm{x_{i + 1}}^2 + \cdots + \norm{x_m}^2 &  & \by{ex:6.1.10} \\
             & = \norm{x_1}^2 + \cdots + \norm{x_{i - 1}}^2 + \norm{x_i}^2 + \norm{x_{i + 1}}^2 + \cdots + \norm{x_m}^2     &  & \by{6.5.1}     \\
             & = \norm{\seq[+]{x}{1,,m}}^2                                                                                  &  & \by{ex:6.1.10} \\
             & = \norm{x}^2                                                                                                 &  & \by{5.2.7}     \\
    \implies & \forall x \in \V, \norm{\T(x)} = \norm{x},                                                                   &  & \by{6.2}[b]
  \end{align*}
  by \cref{6.5.1} we know that \(\T_i\) is orthogonal for all \(i \in \set{1, \dots, m}\).

  Next we claim that \(\T_i\) is a rotation or a reflection according to whether \(\T_{\W_i}\) is a rotation or a reflection.
  We split into two cases:
  \begin{itemize}
    \item If \(\T_{\W_i}\) is a rotation, then by \cref{6.11.1} there exists an orthonormal basis \(\beta = \set{\seq{v}{1,2}}\) for \(\W_i\) over \(\R\), and a \(\theta \in \R\) such that
          \begin{align*}
            \T_{\W_i}(v_1) & = \cos(\theta) v_1 + \sin(\theta) v_2;  \\
            \T_{\W_i}(v_2) & = -\sin(\theta) v_1 + \cos(\theta) v_2.
          \end{align*}
          Note that \(\dim(\W_i) = 2\) and thus \(\W_i^{\perp} = \set{\zv}\) when we only consider vectors in \(\W_i\) itself.
          Since \(\seq{\W}{1,,k}\) are pairwise orthogonal, we see that
          \[
            \W_i^{\perp} = \pa{\sum_{j = 1}^m \W_j} \setminus \W_i \quad \text{and} \quad \forall x \in \pa{\sum_{j = 1}^m \W_j} \setminus \W_i, \T_i(x) = x.
          \]
          Thus by \cref{6.11.1} \(\T_i\) is rotation when \(\T_{\W_i}\) is a rotation.
    \item If \(\T_{\W_i}\) is a reflection, then by \cref{6.47}(b) we have \(\dim(\W_i) = 1\).
          By \cref{6.11.2} we see that \(\T_{\W_i}(x) = -x\) for all \(x \in \W_i\).
          Note that \(\dim(\W_i) = 1\) and thus \(\W_i^{\perp} = \set{\zv}\) when we only consider vectors in \(\W_i\) itself.
          Since \(\seq{\W}{1,,k}\) are pairwise orthogonal, we see that
          \[
            \W_i^{\perp} = \pa{\sum_{j = 1}^m \W_j} \setminus \W_i \quad \text{and} \quad \forall x \in \pa{\sum_{j = 1}^m \W_j} \setminus \W_i, \T_i(x) = x.
          \]
          Thus by \cref{6.11.2} \(\T_i\) is reflection when \(\T_{\W_i}\) is a reflection.
  \end{itemize}
  This establishes (a).

  Next we proof (b).
  Since there is at most one \(\T_{\W_i}\) is a reflection by \cref{6.47}(b), we know that at most one \(\T_i\) is a reflection.
  This establish (b).

  Next we proof (c).
  Thus case for \(i = j\) is trivial.
  So we proof the case \(i \neq j\).
  Without the loss of generality, suppose that \(i < j\).
  Since
  \begin{align*}
    \forall x \in \V, (\T_i \T_j)(x) & = \T_i(\T_j(x))                                                                   \\
                                     & = \T_i(\T_j(\seq[+]{x}{1,,m}))                                 &  & \by{5.2.7}    \\
                                     & = \T_i(\T_j(x_1)) + \cdots + \T_i(\T_j(x_i))                                      \\
                                     & \quad + \cdots \T_i(\T_j(x_j)) + \cdots + \T_i(\T_j(x_m))      &  & \by{2.1.1}[a] \\
                                     & = x_1 + \cdots + \T_i(x_i) + \cdots + \T_j(x_j) + \cdots + x_m                    \\
                                     & = \T_j(\T_i(x_1)) + \cdots + \T_j(\T_i(x_i))                                      \\
                                     & \quad + \cdots \T_j(\T_i(x_j)) + \cdots + \T_j(\T_i(x_m))      &  & \by{2.1.1}[a] \\
                                     & = \T_j(\T_i(\seq[+]{x}{1,,m}))                                 &  & \by{2.1.1}[a] \\
                                     & = \T_j(\T_i(x))                                                &  & \by{5.2.7}    \\
                                     & = (\T_j \T_i)(x),
  \end{align*}
  we see that \(\T_i \T_j = \T_j \T_i\).
  This establish (c).

  Next we proof (d).
  Since
  \begin{align*}
    \forall x \in \V, (\seq[]{\T}{1,,m})(x) & = (\seq[]{\T}{1,,m-1})(\T_m(x))                                                  \\
                                            & = (\seq[]{\T}{1,,m-1})\pa{\sum_{i = 1}^{m - 1} x_i + \T(x_m)} &  & \by{5.2.7}    \\
                                            & = (\seq[]{\T}{1,,m-1})\pa{\sum_{i = 1}^{m - 1} x_i} + \T(x_m) &  & \by{2.1.1}[a] \\
                                            & = \sum_{i = 1}^m \T(x_i)                                                         \\
                                            & = \T\pa{\sum_{i = 1}^m x_i}                                   &  & \by{2.1.1}[a] \\
                                            & = \T(x),                                                      &  & \by{5.2.7}
  \end{align*}
  we see that \(\T = \seq[]{\T}{1,,m}\).
  This establish (d).

  Finally we proof (e).
  By \cref{ex:5.1.7}(d) and \cref{6.11.7}(d) we have \(\det(\T) = \det(\T_1) \cdots \det(\T_m)\).
  The rest follows from \cref{6.47}(b).
\end{proof}

\begin{eg}\label{6.11.8}
  Orthogonal Operators on a Three-Dimensional Real Inner Product Space.

  Let \(\T\) be an orthogonal operator on a three-dimensional real inner product space \(\V\).
  We show that \(\T\) can be decomposed into the composite of a rotation and at most one reflection.
  Let
  \[
    \V = \seq[\oplus]{\W}{1,,m}
  \]
  be a decomposition as in \cref{6.47}(b).
  Clearly, \(m = 2\) or \(m = 3\).

  If \(m = 2\), then \(\V = \W_1 \oplus \W_2\).
  Without loss of generality, suppose that \(\dim(\W_1) = 1\) and \(\dim(\W_2) = 2\).
  Thus \(\T_{\W_1}\) is a reflection or the identity on \(\W_1\), and \(\T_{\W_2}\) is a rotation.
  Defining \(\T_1\) and \(\T_2\) as in the proof of \cref{6.11.7}, we have that \(\T = \T_1 \T_2\) is the composite of a rotation and at most one reflection.
  (Note that if \(\T_{\W_1}\) is not a reflection, then \(\T_1\) is the identity on \(\V\) and \(\T = \T_2\).)

  If \(m = 3\), then \(\V = \W_1 \oplus \W_2 \oplus \W_3\) and \(\dim(\W_i) = 1\) for all \(i \in \set{1, 2, 3}\).
  For each \(i \in \set{1, 2, 3}\), let \(\T_i\) be as in the proof of \cref{6.11.7}.
  If \(\T_{\W_i}\) is not a reflection, then \(\T_i\) is the identity on \(\W_i\).
  Otherwise, \(\T_i\) is a reflection.
  Since \(\T_{\W_i}\) is a reflection for at most one \(i \in \set{1, 2, 3}\), we conclude that \(\T\) is either a single reflection or the identity (a rotation).
\end{eg}

\exercisesection

\setcounter{ex}{1}
\begin{ex}\label{ex:6.11.2}
  Prove that rotations, reflections, and composites of rotations and reflections are orthogonal operators.
\end{ex}

\begin{proof}[\pf{ex:6.11.2}]
  Let \(\T\) be a linear operator on a \(n\)-dimensional real inner product space \(\V\).

  First we show that rotations are orthogonal operators.
  Suppose that \(\T\) is a rotation.
  By \cref{6.11.1} there exists a two-dimensional subspace \(\W\) of \(\V\) over \(\R\), an orthonormal basis \(\beta = \set{\seq{x}{1,2}}\) for \(\W\) over \(\R\), and a \(\theta \in \R\) such that
  \begin{align*}
    \T(x_1)                         & = \cos(\theta) x_1 + \sin(\theta) x_2;  \\
    \T(x_2)                         & = -\sin(\theta) x_1 + \cos(\theta) x_2; \\
    \forall y \in \W^{\perp}, \T(y) & = y.
  \end{align*}
  Let \(\gamma = \set{\seq{v}{1,,n-2}}\) be an orthonormal basis for \(\W^{\perp}\) over \(\R\).
  By \cref{6.2.4} we know that \(\beta \cup \gamma\) is an orthonormal basis for \(\V\) over \(\R\).
  Since
  \begin{align*}
    \inn{\T(x_1), \T(x_1)} & = \inn{\cos(\theta) x_1 + \sin(\theta) x_2, \cos(\theta) x_1 + \sin(\theta) x_2}        &  & \by{6.11.1}     \\
                           & = \cos(\theta) \inn{x_1, \cos(\theta) x_1 + \sin(\theta) x_2}                           &  & \by{6.1.1}[a,b] \\
                           & \quad + \sin(\theta) \inn{x_2, \cos(\theta) x_1 + \sin(\theta) x_2}                     &  & \by{6.1.1}[a,b] \\
                           & = \cos(\theta) \inn{x_1, \cos(\theta) x_1} + \sin(\theta) \inn{x_2, \sin(\theta) x_2}   &  & \by{6.1.12}     \\
                           & = (\cos(\theta))^2 \inn{x_1, x_1} + (\sin(\theta))^2 \inn{x_2, x_2}                     &  & \by{6.1.1}[b]   \\
                           & = (\cos(\theta))^2 + (\sin(\theta))^2                                                   &  & \by{6.1.12}     \\
                           & = 1;                                                                                                         \\
    \inn{\T(x_1), \T(x_2)} & = \inn{\cos(\theta) x_1 + \sin(\theta) x_2, -\sin(\theta) x_1 + \cos(\theta) x_2}       &  & \by{6.11.1}     \\
                           & = \cos(\theta) \inn{x_1, -\sin(\theta) x_1 + \cos(\theta) x_2}                          &  & \by{6.1.1}[a,b] \\
                           & \quad + \sin(\theta) \inn{x_2, -\sin(\theta) x_1 + \cos(\theta) x_2}                    &  & \by{6.1.1}[a,b] \\
                           & = \cos(\theta) \inn{x_1, -\sin(\theta) x_1} + \sin(\theta) \inn{x_2, \cos(\theta) x_2}  &  & \by{6.1.12}     \\
                           & = -\sin(\theta) \cos(\theta) \inn{x_1, x_1} + \sin(\theta) \cos(\theta) \inn{x_2, x_2}  &  & \by{6.1}[b]     \\
                           & = -\sin(\theta) \cos(\theta) + \sin(\theta) \cos(\theta)                                &  & \by{6.1.12}     \\
                           & = 0;                                                                                                         \\
    \inn{\T(x_2), \T(x_2)} & = \inn{-\sin(\theta) x_1 + \cos(\theta) x_2, -\sin(\theta) x_1 + \cos(\theta) x_2}      &  & \by{6.11.1}     \\
                           & = -\sin(\theta) \inn{x_1, -\sin(\theta) x_1 + \cos(\theta) x_2}                         &  & \by{6.1.1}[a,b] \\
                           & \quad + \cos(\theta) \inn{x_2, -\sin(\theta) x_1 + \cos(\theta) x_2}                    &  & \by{6.1.1}[a,b] \\
                           & = -\sin(\theta) \inn{x_1, -\sin(\theta) x_1} + \cos(\theta) \inn{x_2, \cos(\theta) x_2} &  & \by{6.1.12}     \\
                           & = (\sin(\theta))^2 \inn{x_1, x_1} + (\cos(\theta))^2 \inn{x_2, x_2}                     &  & \by{6.1.1}[b]   \\
                           & = (\sin(\theta))^2 + (\cos(\theta))^2                                                   &  & \by{6.1.12}     \\
                           & = 1,
  \end{align*}
  we know that \(\T(\beta) \subseteq \W\) is orthonormal.
  Since \(\T(\gamma) = \gamma\), we know that \(\T(\beta) \cup \T(\gamma)\) is an orthonormal basis for \(\V\) over \(\R\).
  Thus by \cref{6.18}(c)(e) we know that \(\T\) is an orthogonal operator.

  Next we show that reflections are orthogonal operators.
  Suppose that \(\T\) is a reflection.
  By \cref{6.11.2} there exists a one-dimensional subspace \(\W\) of \(\V\) over \(\R\) such that
  \begin{align*}
     & \forall x \in \W, \T(x) = -x;        \\
     & \forall x \in \W^{\perp}, \T(x) = x.
  \end{align*}
  Let \(y \in \V\).
  By \cref{6.6} there exists an unique tuple \(\tuple{y}{1,2} \in \W \times \W^{\perp}\) such that \(y = y_1 + y_2\).
  Since
  \begin{align*}
    \norm{\T(y)}^2 & = \norm{\T(y_1 + y_2)}^2       &  & \by{6.6}       \\
                   & = \norm{\T(y_1) + \T(y_2)}^2   &  & \by{2.1.1}[a]  \\
                   & = \norm{-y_1 + y_2}^2          &  & \by{6.11.2}    \\
                   & = \norm{-y_1}^2 + \norm{y_2}^2 &  & \by{ex:6.1.10} \\
                   & = \norm{y_1}^2 + \norm{y_2}^2  &  & \by{6.2}[a]    \\
                   & = \norm{y_1 + y_2}^2           &  & \by{ex:6.1.10} \\
                   & = \norm{y}^2,                  &  & \by{6.6}
  \end{align*}
  by \cref{6.2}(b) we have \(\norm{\T(y)} = \norm{y}\).
  Thus by \cref{6.5.1} we know that \(\T\) is an orthogonal operator.

  Finally we show that composites of rotations and reflections are orthogonal operators.
  From the proof above we see that rotations and reflections are orthogonal operators.
  Thus we only need to prove that the composites of orthogonal operators are orthogonal operators.
  This is done by \cref{ex:6.5.3}.
\end{proof}

\setcounter{ex}{3}
\begin{ex}\label{ex:6.11.4}
  For any real number \(\phi\), let
  \[
    A = \begin{pmatrix}
      \cos(\phi) & \sin(\phi)  \\
      \sin(\phi) & -\cos(\phi)
    \end{pmatrix}.
  \]
  \begin{enumerate}
    \item Prove that \(\L_A\) is a reflection.
    \item Find the axis in \(\R^2\) about which \(\L_A\) reflects.
  \end{enumerate}
\end{ex}

\begin{proof}[\pf{ex:6.11.4}(a)]
  Since
  \begin{align*}
    \det(\L_A) & = \det(A)                          &  & \by{ex:5.1.7} \\
               & = -(\cos(\phi))^2 - (\sin(\phi))^2 &  & \by{4.1.1}    \\
               & = -1,
  \end{align*}
  by \cref{6.45} we know that \(\L_A\) is a reflection.
\end{proof}

\begin{proof}[\pf{ex:6.11.4}(b)]
  Since
  \begin{align*}
             & A \begin{pmatrix}
                   x \\
                   y
                 \end{pmatrix} = \begin{pmatrix}
                                   x \\
                                   y
                                 \end{pmatrix}                                 &  & \by{6.11.2} \\
    \implies & \begin{pmatrix}
                 \cos(\phi) x + \sin(\phi) y \\
                 \sin(\phi) x - \cos(\phi) y
               \end{pmatrix} = \begin{pmatrix}
                                 x \\
                                 y
                               \end{pmatrix}                                 &  & \by{2.3.1}    \\
    \implies & \begin{dcases}
                 \sin(\phi) y = (1 - \cos(\phi)) x \\
                 \sin(\phi) x = (1 + \cos(\phi)) y
               \end{dcases}                                                \\
    \implies & (\sin(\phi), 1 - \cos(\phi)) \in \set{x \in \R^2 : Ax = x},                      \\
    \implies & \set{x \in \R^2 : Ax = x} = \spn{(\sin(\phi), 1 - \cos(\phi))}, &  & \by{6.11.2}
  \end{align*}
  by \cref{6.11.2} we know that \(\L_A\) is a reflection of \(\R^2\) about \(\spn{(\sin(\phi), 1 - \cos(\phi))}\).
\end{proof}

\begin{ex}\label{ex:6.11.5}
  For any real number \(\phi\), define \(\T_{\phi} = \L_A\), where
  \[
    A = \begin{pmatrix}
      \cos(\phi) & -\sin(\phi) \\
      \sin(\phi) & \cos(\phi)
    \end{pmatrix}.
  \]
  \begin{enumerate}
    \item Prove that any rotation on \(\R^2\) is of the form \(\T_{\phi}\) for some \(\phi\).
    \item Prove that \(\T_{\phi} \T_{\psi} = \T_{(\phi + \psi)}\) for any \(\phi, \psi \in \R\).
    \item Deduce that any two rotations on \(\R^2\) commute.
  \end{enumerate}
\end{ex}

\begin{proof}[\pf{ex:6.11.5}(a)]
  See \cref{6.11.4}.
\end{proof}

\begin{proof}[\pf{ex:6.11.5}(b)]
  Since
  \begin{align*}
     & \begin{pmatrix}
         \cos(\phi) & -\sin(\phi) \\
         \sin(\phi) & \cos(\phi)
       \end{pmatrix} \begin{pmatrix}
                       \cos(\psi) & -\sin(\psi) \\
                       \sin(\psi) & \cos(\psi)
                     \end{pmatrix}                                                            \\
     & = \begin{pmatrix}
           \cos(\phi) \cos(\psi) - \sin(\phi) \sin(\psi) & - \cos(\phi) \sin(\psi) - \sin(\phi) \cos(\psi) \\
           \sin(\phi) \cos(\psi) + \cos(\phi) \sin(\psi) & - \sin(\phi) \sin(\psi) + \cos(\phi) \cos(\psi)
         \end{pmatrix} &  & \by{2.3.1} \\
     & = \begin{pmatrix}
           \cos(\phi + \psi) & -\sin(\phi + \psi) \\
           \sin(\phi + \psi) & \cos(\phi + \psi)
         \end{pmatrix},
  \end{align*}
  by \cref{2.15}(e) we know that \(\T_{\phi} \T_{\psi} = \T_{\phi + \psi}\).
\end{proof}

\begin{proof}[\pf{ex:6.11.5}(c)]
  We have
  \begin{align*}
    \T_{\phi} \T_{\psi} & = \T_{\phi + \psi}    &  & \by{ex:6.11.5}[b] \\
                        & = \T_{\psi + \phi}                           \\
                        & = \T_{\psi} \T_{\phi} &  & \by{ex:6.11.5}[b]
  \end{align*}
  and thus any two rotations on \(\R^2\) commute.
\end{proof}

\begin{ex}\label{ex:6.11.6}
  Prove that the composite of any two rotations on \(\R^3\) is a rotation on \(\R^3\).
\end{ex}

\begin{proof}[\pf{ex:6.11.6}]

\end{proof}

\begin{ex}\label{ex:6.11.7}
  Given real numbers \(\phi\) and \(\psi\), define matrices
  \[
    A = \begin{pmatrix}
      1 & 0          & 0           \\
      0 & \cos(\phi) & -\sin(\phi) \\
      0 & \sin(\phi) & \cos(\phi)
    \end{pmatrix} \quad \text{and} \quad B = \begin{pmatrix}
      \cos(\psi) & -\sin(\psi) & 0 \\
      \sin(\psi) & \cos(\psi)  & 0 \\
      0          & 0           & 1
    \end{pmatrix}.
  \]
  \begin{enumerate}
    \item Prove that \(\L_{A}\) and \(\L_{B}\) are rotations.
    \item Prove that \(\L_{AB}\) is a rotation.
    \item Find the axis of rotation for \(\L_{AB}\).
  \end{enumerate}
\end{ex}

\begin{proof}[\pf{ex:6.11.7}(a)]

\end{proof}

\begin{proof}[\pf{ex:6.11.7}(b)]

\end{proof}

\begin{proof}[\pf{ex:6.11.7}(c)]

\end{proof}

\begin{ex}\label{ex:6.11.8}
  Prove \cref{6.45} using the hints preceding the statement of the theorem.
\end{ex}

\begin{proof}[\pf{ex:6.11.8}]
  See \cref{6.45}.
\end{proof}

\begin{ex}\label{ex:6.11.9}
  Prove that no orthogonal operator can be both a rotation and a reflection.
\end{ex}

\begin{proof}[\pf{ex:6.11.9}]

\end{proof}

\begin{ex}\label{ex:6.11.10}
  Prove that if \(\V\) is a two- or three-dimensional real inner product space, then the composite of two reflections on \(\V\) is a rotation of \(\V\).
\end{ex}

\begin{proof}[\pf{ex:6.11.10}]

\end{proof}

\begin{ex}\label{ex:6.11.11}
  Give an example of an orthogonal operator that is neither a reflection nor a rotation.
\end{ex}

\begin{proof}[\pf{ex:6.11.11}]

\end{proof}

\begin{ex}\label{ex:6.11.12}
  Let \(\V\) be a finite-dimensional real inner product space.
  Define \(\T : \V \to \V\) by \(\T(x) = -x\).
  Prove that \(\T\) is a product of rotations iff \(\dim(\V)\) is even.
\end{ex}

\begin{proof}[\pf{ex:6.11.12}]

\end{proof}

\begin{ex}\label{ex:6.11.13}
  Complete the proof of \cref{6.11.6} by showing that \(\W = \phi_{\beta}^{-1}(\vs{Z})\) satisfies the required conditions.
\end{ex}

\begin{proof}[\pf{ex:6.11.13}]
  By \cref{6.11.6} we need to show that \(\W\) is a subspace of \(\V\) over \(\R\), \(\W\) is \(\T\)-invariant and \(1 \leq \dim(\W) \leq 2\).

  First we show that \(\W\) is a subspace of \(\V\) over \(\R\).
  Since \(\vs{Z}\) is a subspace of \(\R^n\) over \(\R\) and \(\phi_{\beta}^{-1} \in \ls(\R^n, \V)\), by \cref{2.1} we know that \(\W = \phi_{\beta}^{-1}(\vs{Z})\) is a subspace of \(\V\) over \(\R\).

  Next we show that \(\W\) is \(\T\)-invariant.
  Let \(x \in \W\).
  Then there exists a \(y \in \vs{Z}\) such that \(\phi_{\beta}^{-1}(y) = x\) or \(\phi_{\beta}(x) = y\).
  By \cref{6.11.6} we know that \(\vs{Z}\) is \(\L_A\)-invariant.
  Thus by \cref{5.4.1} \(\L_A(y) \in \vs{Z}\).
  By \cref{2.21} this means \((\phi_{\beta}^{-1} \L_A)(y) \in \W\).
  Thus we have
  \begin{align*}
    \T(x) & = (\phi_{\beta}^{-1} \phi_{\beta} \T)(x)   &  & \by{2.21}   \\
          & = (\phi_{\beta}^{-1} \L_A \phi_{\beta})(x) &  & \by{2.4.12} \\
          & = (\phi_{\beta}^{-1} \L_A)(y)                               \\
          & \in \W.
  \end{align*}
  By \cref{5.4.1} \(\W\) is \(\T\)-invariant.

  Finally we show that \(1 \leq \dim(\W) \leq 2\).
  Since \(1 \leq \dim(\vs{Z}) \leq 2\) and \(\W = \phi_{\beta}^{-1}(\vs{Z})\), by \cref{2.19} we have \(1 \leq \dim(\W) \leq 2\).
\end{proof}

\begin{ex}\label{ex:6.11.14}
  Let \(\T\) be an orthogonal (unitary) operator on a finite-dimensional real (complex) inner product space \(\V\).
  If \(\W\) is a \(\T\)-invariant subspace of \(\V\) over \(\R\), prove the following results.
  \begin{enumerate}
    \item \(\T_{\W}\) is an orthogonal (unitary) operator on \(\W\).
    \item \(\W^{\perp}\) is a \(\T\)-invariant subspace of \(\V\).
    \item \(\T_{\W^{\perp}}\) is an orthogonal (unitary) operator on \(\W\).
  \end{enumerate}
\end{ex}

\begin{proof}[\pf{ex:6.11.14}(a)]
  We have
  \begin{align*}
    \forall x \in \W, \norm{\T_{\W}(x)} & = \norm{\T(x)} &  & \by{b.0.4} \\
                                        & = \norm{x}     &  & \by{6.5.1}
  \end{align*}
  and thus by \cref{6.5.1} \(\T_{\W}\) is orthogonal (unitary).
\end{proof}

\begin{proof}[\pf{ex:6.11.14}(b)]
  See \cref{ex:6.5.15}.
\end{proof}

\begin{proof}[\pf{ex:6.11.14}(c)]
  This follows from \cref{ex:6.11.14}(a)(b).
\end{proof}

\begin{ex}\label{ex:6.11.15}
  Let \(\T\) be a linear operator on a finite-dimensional vector space \(\V\) over \(\F\), where \(\V\) is a direct sum of \(\T\)-invariant subspaces, say, \(\V = \seq[\oplus]{\W}{1,,k}\).
  Prove that \(\det(\T) = \det(\T_{\W_1}) \cdots \det(\T_{\W_k})\).
\end{ex}

\begin{proof}[\pf{ex:6.11.15}]
  By \cref{5.25} we have
  \[
    [\T]_{\beta} = \begin{pmatrix}
      [\T_{\W_1}]_{\beta_1} & \zm                   & \cdots & \zm                   \\
      \zm                   & [\T_{\W_2}]_{\beta_2} & \cdots & \zm                   \\
      \zm                   & \zm                   & \cdots & [\T_{\W_k}]_{\beta_k}
    \end{pmatrix},
  \]
  where \(\beta\) is an ordered basis for \(\V\) over \(\F\) and \(\beta_i = \beta \cap \W_i\) for all \(i \in \set{1, \dots, k}\).
  Thus we have
  \begin{align*}
    \det(\T) & = \det([\T]_{\beta})                                             &  & \by{ex:5.1.7}  \\
             & = \det([\T_{\W_1}]_{\beta_1}) \cdots \det([\T_{\W_k}]_{\beta_k}) &  & \by{ex:4.3.21} \\
             & = \det(\T_{\W_1}) \cdots \det(\T_{\W_k}).                        &  & \by{ex:5.1.7}
  \end{align*}
\end{proof}


\chapter{Canonical Forms}\label{ch:7}

% All sections are in separated files.  We include them here.
\section{The Jordan Canonical Form I}\label{sec:7.1}

\begin{defn}\label{7.1.1}
  Let \(\T\) be a linear operator on a finite-dimensional vector space \(\V\) over \(\F\), and suppose that the characteristic polynomial of \(\T\) splits.
  Recall from \cref{5.9} that the diagonalizability of \(\T\) depends on whether the union of ordered bases for the distinct eigenspaces of \(\T\) is an ordered basis for \(\V\) over \(\F\).
  So a lack of diagonalizability means that at least one eigenspace of \(\T\) is too ``small.''

  In this section, we extend the definition of eigenspace to \emph{generalized eigenspace}.
  From these subspaces, we select ordered bases whose union is an ordered basis \(\beta\) for \(\V\) over \(\F\) such that
  \[
    [\T]_{\beta} = \begin{pmatrix}
      A_1    & \zm    & \cdots & \zm    \\
      \zm    & A_2    & \cdots & \zm    \\
      \vdots & \vdots &        & \vdots \\
      \zm    & \zm    & \cdots & A_k
    \end{pmatrix},
  \]
  where each \(\zm\) is a zero matrix, and each \(A_i\) is a square matrix of the form \((\lambda)\) or
  \[
    \begin{pmatrix}
      \lambda & 1       & 0      & \cdots & 0       & 0       \\
      0       & \lambda & 1      & \cdots & 0       & 0       \\
      \vdots  & \vdots  & \vdots &        & \vdots  & \vdots  \\
      0       & 0       & 0      & \cdots & \lambda & 1       \\
      0       & 0       & 0      & \cdots & 0       & \lambda
    \end{pmatrix}
  \]
  for some eigenvalue \(\lambda\) of \(\T\).
  Such a matrix \(A_i\) is called a \textbf{Jordan block} corresponding to \(\lambda\), and the matrix \([\T]_{\beta}\) is called a \textbf{Jordan canonical form} of \(\T\).
  We also say that the ordered basis \(\beta\) is a \textbf{Jordan canonical basis} for \(\T\).
  Observe that each Jordan block \(A_i\) is ``almost'' a diagonal matrix
  ---
  in fact, \([\T]_{\beta}\) is a diagonal matrix iff each \(A_i\) is of the form \((\lambda)\).
\end{defn}

\begin{note}
  In \cref{sec:7.1,sec:7.2}, we prove that every linear operator whose characteristic polynomial splits has a Jordan canonical form that is unique up to the order of the Jordan blocks.
  Nevertheless, it is not the case that the Jordan canonical form is completely determined by the characteristic polynomial of the operator.
\end{note}

\begin{defn}\label{7.1.2}
  Let \(\T\) be a linear operator on a vector space \(\V\) over \(\F\), and let \(\lambda \in \F\).
  Because of the structure of each Jordan block in a Jordan canonical form, we can generalize these observations:
  If \(v\) lies in a Jordan canonical basis for a linear operator \(\T\) and is associated with a Jordan block with diagonal entry \(\lambda\), then \((\T - \lambda \IT[\V])^p(v) = \zv\) for sufficiently large \(p\).
  Eigenvectors satisfy this condition for \(p = 1\).
  A nonzero vector \(x\) in \(\V\) is called a \textbf{generalized eigenvector of \(\T\) corresponding to \(\lambda\)} if \((\T - \lambda \IT[\V])^p(x) = \zv\) for some positive integer \(p\).
\end{defn}

\begin{cor}\label{7.1.3}
  Let \(\T\) be a linear operator on a vector space \(\V\) over \(\F\), and let \(\lambda \in \F\).
  If \(x\) is a generalized eigenvector of \(\T\) corresponding to \(\lambda\), and \(p\) is the smallest positive integer for which \((\T - \lambda \IT[\V])^p(x) = \zv\), then \((\T - \lambda \IT[\V])^{p - 1}(x)\) is an eigenvector of \(\T\) corresponding to \(\lambda\).
  Therefore \(\lambda\) is an eigenvalue of \(\T\).
\end{cor}

\begin{proof}[\pf{7.1.3}]
  Observe that
  \begin{align*}
             & (\T - \lambda \IT[\V])^p(x) = \zv                                   \\
    \implies & (\T - \lambda \IT[\V])\pa{(\T - \lambda \IT[\V])^{p - 1}(x)} = \zv.
  \end{align*}
  Since \(p\) is the smallest positive integer for which \((\T - \lambda \IT[\V])^p(x) = \zv\), we know that \((\T - \lambda \IT[\V])^{p - 1}(x) \neq \zv\).
  Thus by \cref{5.4} \((\T - \lambda \IT[\V])^{p - 1}(x)\) is an eigenvector of \(\T\) and therefore \(\lambda\) is an eigenvalue of \(\T\).
\end{proof}

\begin{defn}\label{7.1.4}
  Let \(\T\) be a linear operator on a vector space \(\V\) over \(\F\), and let \(\lambda\) be an eigenvalue of \(\T\).
  The \textbf{generalized eigenspace of \(\T\) corresponding to \(\lambda\)}, denoted \(\vs{K}_{\lambda}\), is the subset of \(\V\) defined by
  \[
    \vs{K}_{\lambda} = \set{x \in \V : (\T - \lambda \IT[\V])^p(x) = \zv \text{ for some } p \in \Z^+}.
  \]
  Note that \(\vs{K}_{\lambda}\) consists of the zero vector (\cref{2.1.2}(a)) and all generalized eigenvectors corresponding to \(\lambda\) (\cref{7.1.2}).
\end{defn}

\begin{thm}\label{7.1}
  Let \(\T\) be a linear operator on a vector space \(\V\) over \(\F\), and let \(\lambda \in \F\) be an eigenvalue of \(\T\).
  Then
  \begin{enumerate}
    \item \(\vs{K}_{\lambda}\) is a \(\T\)-invariant subspace of \(\V\) containing \(\vs{E}_{\lambda}\)
          (the eigenspace of \(\T\) corresponding to \(\lambda\)).
    \item For any scalar \(\mu \neq \lambda\), the restriction of \(\T - \mu \IT[\V]\) to \(\vs{K}_{\lambda}\) is one-to-one.
  \end{enumerate}
\end{thm}

\begin{proof}[\pf{7.1}(a)]
  Clearly, \(\zv \in \vs{K}_{\lambda}\).
  Suppose that \(x\) and \(y\) are in \(\vs{K}_{\lambda}\).
  Then by \cref{7.1.4} there exist positive integers \(p\) and \(q\) such that
  \[
    (\T - \lambda \IT[\V])^p(x) = (\T - \lambda \IT[\V])^q(y) = \zv.
  \]
  Therefore
  \begin{align*}
    (\T - \lambda \IT[\V])^{p + q}(x + y) & = (\T - \lambda \IT[\V])^{p + q}(x) + (\T - \lambda \IT[\V])^{p + q}(y) &  & \by{2.9}      \\
                                          & = (\T - \lambda \IT[\V])^q(\zv) + (\T - \lambda \IT[\V])^p(\zv)         &  & \by{7.1.4}    \\
                                          & = \zv,                                                                  &  & \by{2.1.2}[a]
  \end{align*}
  and hence \(x + y \in \vs{K}_{\lambda}\).
  Similarly,
  \begin{align*}
    \forall c \in \F, (\T - \lambda \IT[\V])^p(cx) & = c (\T - \lambda \IT[\V])^p(x) &  & \by{2.9}    \\
                                                   & = c \zv                         &  & \by{7.1.4}  \\
                                                   & = \zv                           &  & \by{1.2}[c]
  \end{align*}
  and hence \(cx \in \vs{K}_{\lambda}\).

  To show that \(\vs{K}_{\lambda}\) is \(\T\)-invariant, consider any \(x \in \vs{K}_{\lambda}\).
  Choose a positive integer \(p\) such that \((\T - \lambda \IT[\V])^p(x) = \zv\).
  Then
  \begin{align*}
    (\T - \lambda \IT[\V])^p \T(x) & = \T (\T - \lambda \IT[\V])^p(x) &  & \by{ex:5.4.4} \\
                                   & = \T(\zv)                        &  & \by{7.1.4}    \\
                                   & = \zv.                           &  & \by{2.1.2}[a]
  \end{align*}
  Therefore \(\T(x) \in \vs{K}_{\lambda}\).

  Finally, it is a simple observation that \(\vs{E}_{\lambda}\) is contained in \(\vs{K}_{\lambda}\).
\end{proof}

\begin{proof}[\pf{7.1}(b)]
  Let \(x \in \vs{K}_{\lambda}\) and \((\T - \mu \IT[\V])(x) = \zv\).
  By way of contradiction, suppose that \(x \neq \zv\).
  Let \(p\) be the smallest positive integer for which \((\T - \lambda \IT[\V])^p(x) = \zv\), and let \(y = (\T - \lambda \IT[\V])^{p - 1}(x)\).
  Then
  \[
    (\T - \lambda \IT[\V])(y) = (\T - \lambda \IT[\V])^p(x) = \zv,
  \]
  and hence \(y \in \vs{E}_{\lambda}\) (\cref{7.1.3}).
  Furthermore,
  \begin{align*}
    (\T - \mu \IT[\V])(y) & = (\T - \mu \IT[\V]) (\T - \lambda \IT[\V])^{p - 1}(x)  &  & \by{7.1.3}    \\
                          & = (\T - \lambda \IT[\V])^{p - 1} (\T - \mu \IT[\V]) (x) &  & \by{ex:5.4.4} \\
                          & = \zv,                                                  &  & \by{2.1.2}[a]
  \end{align*}
  so that \(y \in \vs{E}_{\mu}\).
  But by \cref{5.8} \(\vs{E}_{\lambda} \cap \vs{E}_{\mu} = \set{\zv}\), and thus \(y = \zv\), contrary to the hypothesis of \(p\).
  So \(x = \zv\), and the restriction of \(\T - \mu \IT[\V]\) to \(\vs{K}_{\lambda}\) is one-to-one.
\end{proof}

\begin{thm}\label{7.2}
  Let \(\T\) be a linear operator on a finite-dimensional vector space \(\V\) over \(\F\) such that the characteristic polynomial of \(\T\) splits.
  Suppose that \(\lambda\) is an eigenvalue of \(\T\) with multiplicity \(m\).
  Then
  \begin{enumerate}
    \item \(\dim(\vs{K}_{\lambda}) \leq m\).
    \item \(\vs{K}_{\lambda} = \ns{(\T - \lambda \IT[\V])^m}\).
  \end{enumerate}
\end{thm}

\begin{proof}[\pf{7.2}(a)]
  Let \(h\) be the characteristic polynomial of \(\T_{\vs{K}_{\lambda}}\).
  By \cref{5.21}, \(h\) divides the characteristic polynomial of \(\T\), and by \cref{7.1}(b), \(\lambda\) is the only eigenvalue of \(\T_{\vs{K}_{\lambda}}\).
  Hence \(h(t) = (-1)^d (t - \lambda)^d\), where \(d = \dim(\vs{K}_{\lambda})\) (\cref{2.5}), and \(d \leq m\).
\end{proof}

\begin{proof}[\pf{7.2}(b)]
  Clearly \(\ns{(\T - \lambda \IT[\V])^m} \subseteq \vs{K}_{\lambda}\).
  Now let \(h\) and \(d\) be as in (a).
  Then \(h(\T_{\vs{K}_{\lambda}})\) is identically zero by the Cayley--Hamilton theorem (\cref{5.23});
  therefore \((\T - \lambda \IT[\V])^d(x) = \zv\) for all \(x \in \vs{K}_{\lambda}\).
  Since \(d \leq m\), we have \(\vs{K}_{\lambda} \subseteq \ns{(\T - \lambda \IT[\V])^m}\).
\end{proof}

\begin{thm}\label{7.3}
  Let \(\T\) be a linear operator on a finite-dimensional vector space \(\V\) over \(\F\) such that the characteristic polynomial of \(\T\) splits, and let \(\seq{\lambda}{1,,k} \in \F\) be the distinct eigenvalues of \(\T\).
  Then, for every \(x \in \V\), there exist vectors \(v_i \in \vs{K}_{\lambda_i}\), \(i \in \set{1, \dots, k}\), such that
  \[
    x = \seq[+]{v}{1,,k}.
  \]
\end{thm}

\begin{proof}[\pf{7.3}]
  The proof is by mathematical induction on the number \(k\) of distinct eigenvalues of \(\T\).
  First suppose that \(k = 1\), and let \(m\) be the multiplicity of \(\lambda_1\).
  Then \((\lambda_1 - t)^m\) is the characteristic polynomial of \(\T\), and hence \((\lambda_1 \IT[\V] - \T)^m = \zT\) by the Cayley--Hamilton theorem (\cref{5.23}).
  Thus
  \begin{align*}
    \V & = \ns{\zT}                        &  & \by{2.3}    \\
       & = \ns{(\T - \lambda_1 \IT[\V])^m} &  & \by{5.23}   \\
       & = \vs{K}_{\lambda_1},             &  & \by{7.2}(b)
  \end{align*}
  and the result follows.

  Now suppose that for some integer \(k \geq 1\), the result is established whenever \(\T\) has fewer than \(k + 1\) distinct eigenvalues, and suppose that \(\T\) has \(k + 1\) distinct eigenvalues.
  Let \(m\) be the multiplicity of \(\lambda_{k + 1}\), and let \(f\) be the characteristic polynomial of \(\T\).
  Then \(f(t) = (t - \lambda_{k + 1})^m g(t)\) for some polynomial \(g\) not divisible by \((t - \lambda_{k + 1})\).
  Let \(\W = \rg{(\T - \lambda_{k + 1} \IT[\V])^m}\).
  Then \(\W\) is \(\T\)-invariant since
  \begin{align*}
             & \forall y \in \rg{(\T - \lambda_{k + 1} \IT[\V])^m}, \exists x \in \V : (\T - \lambda_{k + 1} \IT[\V])^m(x) = y \\
    \implies & \forall y \in \rg{(\T - \lambda_{k + 1} \IT[\V])^m}, \T(y) = \T ((\T - \lambda_{k + 1} \IT[\V])^m(x))           \\
             & = ((\T - \lambda_{k + 1} \IT[\V])^m \T)(x) = ((\T - \lambda_{k + 1} \IT[\V])^m)(\T(x))                          \\
             & \in \rg{(\T - \lambda_{k + 1} \IT[\V])^m}                                                                       \\
    \implies & \T(\rg{(\T - \lambda_{k + 1} \IT[\V])^m}) \subseteq \rg{(\T - \lambda_{k + 1} \IT[\V])^m}.
  \end{align*}
  We claim that \((\T - \lambda_{k + 1} \IT[\V])^m\) maps \(\vs{K}_{\lambda_i}\) onto itself for \(i \in \set{1, \dots, k}\).
  Suppose that \(i \in \set{1, \dots, k}\).
  First we show that \((\T - \lambda_{k + 1} \IT[\V])^m\) maps \(\vs{K}_{\lambda_i}\) into itself.
  This is true since
  \begin{align*}
             & \forall x \in \vs{K}_{\lambda_i}, \exists p \in \Z^+ : (\T - \lambda_i \IT[\V])^p(x) = \zv        &  & \by{7.1.4}    \\
    \implies & \forall x \in \vs{K}_{\lambda_i}, (\T - \lambda_i \IT[\V])^p((\T - \lambda_{k + 1} \IT[\V])^m(x))                    \\
             & = (\T - \lambda_{k + 1} \IT[\V])^m((\T - \lambda_i \IT[\V])^p(x))                                 &  & \by{ex:5.4.4} \\
             & = \zv                                                                                             &  & \by{2.1.2}[a] \\
    \implies & \forall x \in \vs{K}_{\lambda_i}, (\T - \lambda_{k + 1} \IT[\V])^m(x) \in \vs{K}_{\lambda_i}.     &  & \by{7.1.4}
  \end{align*}
  Now we show the claim is true.
  Since \((\T - \lambda_{k + 1} \IT[\V])^m\) maps \(\vs{K}_{\lambda_i}\) into itself and \(\lambda_{k + 1} \neq \lambda_i\), the restriction of \(\T - \lambda_{k + 1} \IT[\V]\) to \(\vs{K}_{\lambda_i}\) is one-to-one (by \cref{7.1}(b)) and hence is onto (\cref{2.4,2.5}).
  One consequence of this is that for \(i \in \set{1, \dots, k}\), \(\vs{K}_{\lambda_i}\) is contained in \(\W\);
  hence \(\lambda_i\) is an eigenvalue of \(\T_{\W}\) for \(i \in \set{1, \dots, k}\).

  Next, observe that \(\lambda_{k + 1}\) is not an eigenvalue of \(\T_{\W}\).
  For suppose that \(\T(v) = \lambda_{k + 1} v\) for some \(v \in \W\).
  Then \(v = (\T - \lambda_{k + 1} \IT[\V])^m(y)\) for some \(y \in \V\), and it follows from \cref{5.4} that
  \[
    \zv = (\T - \lambda_{k + 1} \IT[\V])(v) = (\T - \lambda_{k + 1} \IT[\V])^{m + 1}(y).
  \]
  Therefore \(y \in \vs{K}_{\lambda_{k + 1}}\).
  So by \cref{7.2}(b), \(v = (\T - \lambda_{k + 1} \IT[\V])^m(y) = \zv\).

  Since every eigenvalue of \(\T_{\W}\) is an eigenvalue of \(\T\), the distinct eigenvalues of \(\T_{\W}\) are \(\seq{\lambda}{1,,k}\).
  For each \(i \in \set{1, \dots, k}\), we defined the generalized eigenspace for \(\T_{\W}\) corresponding to \(\lambda_i\) as follow:
  \[
    \vs{K}_{\lambda_i}' = \set{x \in \W | \exists p \in \Z^+ : (\T_{\W} - \lambda_i)^p(x) = \zv}.
  \]
  Clearly we have \(\vs{K}_{\lambda_i}' \subseteq \vs{K}_{\lambda_i}\) for each \(i \in \set{1, \dots, k}\).

  Now let \(x \in \V\).
  Then \((\T - \lambda_{k + 1} \IT[\V])^m(x) \in \W\).
  Since \(\T_{\W}\) has the \(k\) distinct eigenvalues \(\seq{\lambda}{1,,k}\), the induction hypothesis applies.
  Hence there are vectors \(w_i \in \vs{K}_{\lambda_i}'\), \(i \in \set{1, \dots, k}\), such that
  \[
    (\T - \lambda_{k + 1} \IT[\V])^m(x) = \seq[+]{w}{1,,k}.
  \]
  Since \(\vs{K}_{\lambda_i}' \subseteq \vs{K}_{\lambda_i}\) for \(i \in \set{1, \dots, k}\) and \((\T - \lambda_{k + 1} \IT[\V])^m\) maps \(\vs{K}_{\lambda_i}\) onto itself for \(i \in \set{1, \dots, k}\), there exist vectors \(v_i \in \vs{K}_{\lambda_i}\) such that \((\T - \lambda_{k + 1} \IT[\V])^m(v_i) = w_i\) for \(i \in \set{1, \dots, k}\).
  Thus we have
  \[
    (\T - \lambda_{k + 1} \IT[\V])^m(x) = (\T - \lambda_{k + 1} \IT[\V])^m(v_1) + \cdots + (\T - \lambda_{k + 1} \IT[\V])^m(v_{k}),
  \]
  and it follows that \(x - (\seq[+]{v}{1,,k}) \in \vs{K}_{\lambda_{k + 1}}\) since
  \begin{align*}
    \zv & = (\T - \lambda_{k + 1} \IT[\V])^m(x) - (\T - \lambda_{k + 1} \IT[\V])^m(v_1) + \cdots + (\T - \lambda_{k + 1} \IT[\V])^m(v_{k}) \\
        & = (\T - \lambda_{k + 1} \IT[\V])^m(x - (\seq[+]{v}{1,,k})).
  \end{align*}
  Therefore there exists a vector \(v_k \in \vs{K}_{\lambda_{k + 1}}\) such that
  \[
    x = \seq[+]{v}{1,,k+1}.
  \]
\end{proof}

\begin{thm}\label{7.4}
  Let \(\T\) be a linear operator on a finite-dimensional vector space \(\V\) over \(\F\) such that the characteristic polynomial of \(\T\) splits, and let \(\seq{\lambda}{1,,k} \in \F\) be the distinct eigenvalues of \(\T\) with corresponding multiplicities \(\seq{m}{1,,k}\).
  For \(i \in \set{1, \dots, k}\), let \(\beta_i\) be an ordered basis for \(\vs{K}_{\lambda_i}\) over \(\F\).
  Then the following statements are true.
  \begin{enumerate}
    \item \(\beta_i \cap \beta_j = \varnothing\) for \(i \neq j\).
    \item \(\beta = \seq[\cup]{\beta}{1,,k}\) is an ordered basis for \(\V\) over \(\F\).
    \item \(\dim(\vs{K}_{\lambda_i}) = m_i\) for all \(i \in \set{1, \dots, k}\).
  \end{enumerate}
\end{thm}

\begin{proof}[\pf{7.4}(a)]
  Suppose for sake of contradiction that \(x \in \beta_i \cap \beta_j \subseteq \vs{K}_{\lambda_i} \cap \vs{K}_{\lambda_j}\), where \(i \neq j\).
  By \cref{7.1}(b), \(\T - \lambda_i \IT[\V]\) is one-to-one on \(\vs{K}_{\lambda_j}\), and therefore \((\T - \lambda_i \IT[\V])^p(x) \neq \zv\) for any positive integer \(p\).
  But this contradicts the fact that \(x \in \vs{K}_{\lambda_i}\), and the result follows.
\end{proof}

\begin{proof}[\pf{7.4}(b)]
  Let \(x \in \V\).
  By \cref{7.3}, for \(i \in \set{1, \dots, k}\), there exist vectors \(v_i \in \vs{K}_{\lambda_i}\) such that \(x = \seq[+]{v}{1,,k}\).
  Since each \(v_i\) is a linear combination of the vectors of \(\beta_i\), it follows that \(x\) is a linear combination of the vectors of \(\beta\).
  Therefore \(\beta\) spans \(\V\).
  Let \(q\) be the number of vectors in \(\beta\).
  Then \(\dim(\V) \leq q\).
  For each \(i \in \set{1, \dots, k}\), let \(d_i = \dim(\vs{K}_{\lambda_i})\).
  Then, by \cref{5.3} and \cref{7.2}(a),
  \[
    q = \sum_{i = 1}^k d_i \leq \sum_{i = 1}^k m_i = \dim(\V).
  \]
  Hence \(q = \dim(\V)\).
  Consequently \(\beta\) is a basis for \(\V\) over \(\F\) by \cref{1.6.15}(a).
\end{proof}

\begin{proof}[\pf{7.4}(c)]
  Using the notation and result of (b), we see that \(\sum_{i = 1}^k d_i = \sum_{i = 1}^k m_i\).
  But \(d_i \leq m_i\) by \cref{7.2}(a), and therefore \(d_i = m_i\) for all \(i \in \set{1, \dots, k}\).
\end{proof}

\begin{cor}\label{7.1.5}
  Let \(\T\) be a linear operator on a finite-dimensional vector space \(\V\) over \(\F\) such that the characteristic polynomial of \(\T\) splits.
  Then \(\T\) is diagonalizable iff \(\vs{E}_{\lambda} = \vs{K}_{\lambda}\) for every eigenvalue \(\lambda\) of \(\T\).
\end{cor}

\begin{proof}[\pf{7.1.5}]
  Combining \cref{7.4}(c) and \cref{5.9}(a), we see that \(\T\) is diagonalizable iff \(\dim(\vs{E}_{\lambda}) = \dim(\vs{K}_{\lambda})\) for each eigenvalue \(\lambda\) of \(\T\).
  But \(\vs{E}_{\lambda} \subseteq \vs{K}_{\lambda}\) (\cref{7.1}(a)), and hence these subspaces have the same dimension iff they are equal (\cref{1.11}).
\end{proof}

\begin{defn}\label{7.1.6}
  Let \(\T\) be a linear operator on a vector space \(\V\) over \(\F\), and let \(x\) be a generalized eigenvector of \(\T\) corresponding to the eigenvalue \(\lambda\).
  Suppose that \(p\) is the smallest positive integer for which \((\T - \lambda \IT[\V])^p(x) = \zv\).
  Then the ordered set
  \[
    \set{(\T - \lambda \IT[\V])^{p - 1}(x), (\T - \lambda \IT[\V])^{p - 2}(x), \dots, (\T - \lambda \IT[\V])(x), x}
  \]
  is called a \textbf{cycle of generalized eigenvectors} of \(\T\) corresponding to \(\lambda\).
  The vectors \((\T - \lambda \IT[\V])^{p - 1}(x)\) and \(x\) are called the \textbf{initial vector} and the \textbf{end vector} of the cycle, respectively.
  We say that the \textbf{length} of the cycle is \(p\).
\end{defn}

\begin{note}
  The initial vector of a cycle of generalized eigenvectors of a linear operator \(\T\) is the only eigenvector of \(\T\) in the cycle.
  Also observe that if \(x\) is an eigenvector of \(\T\) corresponding to the eigenvalue \(\lambda\), then the set \(\set{x}\) is a cycle of generalized eigenvectors of \(\T\) corresponding to \(\lambda\) of length \(1\).
\end{note}

\begin{thm}\label{7.5}
  Let \(\T\) be a linear operator on a finite-dimensional vector space \(\V\) over \(\F\) whose characteristic polynomial splits, and suppose that \(\beta\) is a basis for \(\V\) over \(\F\) such that \(\beta\) is a disjoint union of cycles of generalized eigenvectors of \(\T\).
  Then the following statements are true.
  \begin{enumerate}
    \item For each cycle \(\gamma\) of generalized eigenvectors contained in \(\beta\), \(\W = \spn{\gamma}\) is \(\T\)-invariant, and \([\T_{\W}]_{\gamma}\) is a Jordan block.
    \item \(\beta\) is a Jordan canonical basis for \(\V\) over \(\F\).
  \end{enumerate}
\end{thm}

\begin{proof}[\pf{7.5}(a)]
  Suppose that \(\gamma\) corresponds to \(\lambda\), \(\gamma\) has length \(p\), and \(x\) is the end vector of \(\gamma\).
  Then \(\gamma = \set{\seq{v}{1,,p}}\), where
  \[
    v_i = (\T - \lambda \IT[\V])^{p - i}(x) \quad \text{ for } i \in \set{1, \dots, p - 1}
  \]
  and \(v_p = x\).
  So
  \[
    (\T - \lambda \IT[\V])(v_1) = (\T - \lambda \IT[\V])^p(x) = \zv,
  \]
  and hence \(\T(v_1) = \lambda v_1\).
  For \(i \in \set{2, \dots, p}\),
  \[
    (\T - \lambda \IT[\V])(v_i) = (\T - \lambda \IT[\V])^{p - (i - 1)}(x) = v_{i - 1}.
  \]
  Therefore \(\T\) maps \(\W\) into itself, and, by the preceding equations, we see that \([\T_{\W}]_{\gamma}\) is a Jordan block.
\end{proof}

\begin{proof}[\pf{7.5}(b)]
  Let \(\seq{\gamma}{1,,k}\) be cycles in \(\beta\) and let \(\W_i = \spn{\gamma_i}\) for each \(i \in \set{1, \dots, k}\).
  By \cref{5.25} we see that \([\T]_{\beta} = [\T_{\W_1}]_{\beta_1} \oplus \cdots \oplus [\T_{\W_k}]_{\beta_k}\).
  By \cref{7.5}(a) we know that \([\T_{\W_i}]_{\gamma_i}\) is a Jordan block for each \(i \in \set{1, \dots, k}\).
  Thus by \cref{7.1.1} \(\beta\) is a Jordan canonical basis for \(\V\) over \(\F\).
\end{proof}

\begin{thm}\label{7.6}
  Let \(\T\) be a linear operator on a vector space \(\V\) over \(\F\), and let \(\lambda\) be an eigenvalue of \(\T\).
  Suppose that \(\seq{\gamma}{1,,q}\) are cycles of generalized eigenvectors of \(\T\) corresponding to \(\lambda\) such that the initial vectors of the \(\gamma_i\)'s are distinct and form a linearly independent set.
  Then the \(\gamma_i\)'s are disjoint, and their union \(\gamma = \bigcup_{i = 1}^q \gamma_i\) is linearly independent.
\end{thm}

\begin{proof}[\pf{7.6}]
  \cref{ex:7.1.5} shows that the \(\gamma_i\)'s are disjoint.

  The proof that \(\gamma\) is linearly independent is by mathematical induction on the number of vectors in \(\gamma\).
  Denote the number as \(n\).
  If \(n = 1\), then the result is clear (\cref{1.5.4}(b)).
  So assume that for some integer \(n \geq 1\) the result is valid.
  Suppose that \(\gamma\) has \(n + 1\) vectors.
  Let \(\W\) be the subspace of \(\V\) over \(\F\) generated by \(\gamma\).
  Clearly \(\W\) is \((\T - \lambda \IT[\V])\)-invariant, and \(\dim(\W) \leq n + 1\).
  Let \(\U\) denote the restriction of \(\T - \lambda \IT[\V]\) to \(\W\).

  For each \(i \in \set{1, \dots, q}\), let \(\gamma_i'\) denote the cycle obtained from \(\gamma_i\) by deleting the end vector.
  Note that if \(\gamma_i\) has length one, then \(\gamma_i' = \varnothing\).
  In the case that \(\gamma_i' \neq \varnothing\), each vector of \(\gamma_i'\) is the image under \(\U\) of a vector in \(\gamma_i\), and conversely, every nonzero image under \(\U\) of a vector of \(\gamma_i\) is contained in \(\gamma_i'\).
  Let \(\gamma' = \bigcup_{i = 1}^q \gamma_i'\).
  Then by the last statement, \(\gamma'\) generates \(\rg{\U}\).
  Furthermore, \(\gamma'\) consists of \(n + 1 - q\) vectors, and the initial vectors of the \(\gamma_i'\)'s are also initial vectors of the \(\gamma_i\)'s.
  Thus we may apply the induction hypothesis to conclude that \(\gamma'\) is linearly independent.
  Therefore \(\gamma'\) is a basis for \(\rg{\U}\) over \(\F\).
  Hence \(\rk{\U} = n + 1 - q\).
  Since the \(q\) initial vectors of the \(\gamma_i\)'s form a linearly independent set and lie in \(\ns{\U}\) (\cref{7.1.6}), we have \(\nt{\U} \geq q\).
  From these inequalities and the dimension theorem, we obtain
  \begin{align*}
    n + 1 & \geq \dim(\W)                      \\
          & = \rk{\U} + \nt{\U}  &  & \by{2.3} \\
          & \geq (n + 1 - q) + q               \\
          & = n + 1.
  \end{align*}
  We conclude that \(\dim(\W) = n + 1\).
  Since \(\gamma\) generates \(\W\) and consists of \(n + 1\) vectors, it must be a basis for \(\W\) over \(\F\).
  Hence \(\gamma\) is linearly independent.
\end{proof}

\begin{cor}\label{7.1.7}
  Every cycle of generalized eigenvectors of a linear operator is linearly independent.
\end{cor}

\begin{proof}[\pf{7.1.7}]
  By \cref{7.6} we simply set \(q = 1\) to conclude that every cycle of generalized eigenvectors of a linear operator is linearly independent.
  Below we provide an alternative proof.

  Let \(\V\) be a vector space over \(\F\) and let \(\T \in \ls(\V)\).
  Let \(\gamma\) be a cycle of generalized eigenvectors of \(\T\) corresponding to \(\lambda \in \F\).
  Let \(p\) be the length of \(\gamma\) and let \(x\) be the end vector of \(\gamma\).
  By \cref{7.1.6} we can define \(v_i = (\T - \lambda \IT[\V])^{p - i}(x)\) for each \(i \in \set{1, \dots, p}\).
  Let \(\seq{a}{1,,p} \in \F\) such that \(\sum_{i = 1}^p a_i v_i = \zv\).
  Since
  \begin{align*}
             & \sum_{i = 1}^p a_i v_i = \zv                                                                                        \\
    \implies & \zv = (\T - \lambda \IT[\V])^{p - 1}\pa{\sum_{i = 1}^p a_i v_i}   &  & \by{2.1.2}[a]                                \\
             & = \sum_{i = 1}^p a_i (\T - \lambda \IT[\V])^{p - 1}(v_i)          &  & \by{2.1.2}[d]                                \\
             & = \sum_{i = 1}^p a_i (\T - \lambda \IT[\V])^{2p - 1 - i}(x)       &  & \by{7.1.6}                                   \\
             & = a_p (\T - \lambda \IT[\V])^{p - 1}(x)                           &  & \by{7.1.6}                                   \\
    \implies & a_p = 0                                                           &  & ((\T - \lambda \IT[\V])^{p - 1}(x) \neq \zv) \\
    \implies & \zv = (\T - \lambda \IT[\V])^{p - 2}\pa{\sum_{i = 1}^p a_i v_i}   &  & \by{2.1.2}[a]                                \\
             & = (\T - \lambda \IT[\V])^{p - 2}\pa{\sum_{i = 1}^{p - 1} a_i v_i} &  & (a_p = 0)                                    \\
             & = \sum_{i = 1}^{p - 1} a_i (\T - \lambda \IT[\V])^{2p - 2 - i}(x) &  & \by{7.1.6}                                   \\
             & = a_{p - 1} (\T - \lambda \IT[\V])^{p - 1}(x)                     &  & \by{7.1.6}                                   \\
    \implies & \seq[=]{a}{p-1,p} = 0                                             &  & ((\T - \lambda \IT[\V])^{p - 1}(x) \neq \zv) \\
    \implies & \vdots                                                                                                              \\
    \implies & \seq[=]{a}{1,,p} = 0,
  \end{align*}
  by \cref{1.5.3} we see that \(\gamma\) is linearly independent.
\end{proof}

\begin{thm}\label{7.7}
  Let \(\T\) be a linear operator on a finite-dimensional vector space \(\V\) over \(\F\), and let \(\lambda\) be an eigenvalue of \(\T\).
  Then \(\vs{K}_\lambda\) has an ordered basis consisting of a union of disjoint cycles of generalized eigenvectors corresponding to \(\lambda\).
\end{thm}

\begin{proof}[\pf{7.7}]
  The proof is by mathematical induction on \(n = \dim(\vs{K}_\lambda)\).
  The result is clear for \(n = 1\).
  So suppose that for some integer \(n \geq 1\) the result is valid whenever \(\dim(\vs{K}_\lambda) \leq n\), and assume that \(\dim(\vs{K}_\lambda) = n + 1\).
  Let \(\U\) denote the restriction of \(\T - \lambda \IT[\V]\) to \(\vs{K}_\lambda\).
  Then \(\rg{\U}\) is a subspace of \(\vs{K}_\lambda\) of lesser dimension (\cref{7.1.6}), and \(\rg{\U}\) is the space of generalized eigenvectors corresponding to \(\lambda\) for the restriction of \(\T\) to \(\rg{\U}\).
  Therefore, by the induction hypothesis, there exist disjoint cycles \(\seq{\gamma}{1,,q}\) of generalized eigenvectors of this restriction, and hence of \(\T\) itself, corresponding to \(\lambda\) for which \(\gamma = \bigcup_{i = 1}^q \gamma_i\) is a basis for \(\rg{\U}\).
  For \(i \in \set{1, \dots, q}\), the end vector of \(\gamma_i\) is the image under \(\U\) of a vector \(v_i \in \vs{K}_\lambda\), and so we can extend each \(\gamma_i\) to a larger cycle \(\tilde{\gamma}_i = \gamma_i \cup \set{v_i}\) of generalized eigenvectors of \(\T\) corresponding to \(\lambda\).
  For \(i \in \set{1 \dots, q}\), let \(w_i\) be the initial vector of \(\tilde{\gamma}_i\) (and hence of \(\gamma_i\)).
  Since \(\set{\seq{w}{1,,q}}\) is a linearly independent subset of \(\vs{E}_\lambda\) (\cref{7.1.3}), this set can be extended to a basis \(\set{\seq{w}{1,,q}, \seq{u}{1,,s}}\) for \(\vs{E}_\lambda\).
  Then \(\seq{\tilde{\gamma}}{1,,q}, \set{u_1}, \dots, \set{u_s}\) are disjoint cycles of generalized eigenvectors of \(\T\) corresponding to \(\lambda\) (\((\T - \lambda \IT[\V](u_j)) = \zv\) for all \(j \in \set{1, \dots, s}\)) such that the initial vectors of these cycles are linearly independent.
  Therefore their union \(\tilde{\gamma}\) is a linearly independent subset of \(\vs{K}_\lambda\) by \cref{7.6}.

  We show that \(\tilde{\gamma}\) is a basis for \(\vs{K}_\lambda\).
  Suppose that \(\gamma\) consists of \(r = \rk{\U}\) vectors.
  Then \(\tilde{\gamma}\) consists of \(r + q + s\) vectors.
  Furthermore, since \(\set{\seq{w}{1,,q}, \seq{u}{1,,s}}\) is a basis for \(\vs{E}_\lambda = \ns{\U}\) over \(\F\), it follows that \(\nt{\U} = q + s\).
  Therefore
  \[
    \dim(\vs{K}_{\lambda}) = \rk{\U} + \nt{\U} = r + q + s.
  \]
  So \(\tilde{\gamma}\) is a linearly independent subset of \(\vs{K}_\lambda\) containing \(\dim(\vs{K}_\lambda)\) vectors.
  It follows that \(\tilde{\gamma}\) is a basis for \(\vs{K}_\lambda\) (\cref{1.6.15}(b)).
\end{proof}

\begin{cor}\label{7.1.8}
  Let \(\T\) be a linear operator on a finite-dimensional vector space \(\V\) over \(\F\) whose characteristic polynomial splits.
  Then \(\T\) has a Jordan canonical form.
\end{cor}

\begin{proof}[\pf{7.1.8}]
  Let \(\seq{\lambda}{1,,k} \in \F\) be the distinct eigenvalues of \(\T\).
  By \cref{7.7}, for each \(i \in \set{1, \dots, k}\), there is an ordered basis \(\beta_i\) for \(\V\) over \(\F\) consisting of a disjoint union of cycles of generalized eigenvectors corresponding to \(\lambda_i\).
  Let \(\beta = \seq[\cup]{\beta}{1,,k}\).
  Then, by \cref{7.4}(b), \(\beta\) is an ordered basis for \(\V\) over \(\F\).
\end{proof}

\begin{defn}\label{7.1.9}
  Let \(A \in \ms[n][n][\F]\) be such that the characteristic polynomial of \(A\) (and hence of \(\L_A\)) splits.
  Then the \textbf{Jordan canonical form} of \(A\) is defined to be the Jordan canonical form of the linear operator \(\L_A\) on \(\vs{F}^n\).
\end{defn}

\begin{cor}\label{7.1.10}
  Let \(A \in \ms[n][n][\F]\) whose characteristic polynomial splits.
  Then \(A\) has a Jordan canonical form \(J\), and \(A\) is similar to \(J\).
\end{cor}

\begin{proof}[\pf{7.1.10}]
  By \cref{5.1.6} we know that the characteristic polynomial of \(\L_A\) splits.
  Thus by \cref{7.1.8,7.1.9} \(A\) has a Jordan canonical form \(J\).
  Let \(\beta\) be the standard ordered basis for \(\vs{F}^n\) over \(\F\).
  Let \(\gamma\) be a Jordan canonical basis for \(A\).
  Then we have
  \begin{align*}
    A & = [\L_A]_{\beta}                                                        &  & \by{2.15}[a] \\
      & = [\IT[\V]]_{\gamma}^{\beta} [\L_A]_{\gamma} [\IT[\V]]_{\beta}^{\gamma} &  & \by{2.23}    \\
      & = J                                                                     &  & \by{7.1.1}
  \end{align*}
  and thus by \cref{2.5.4} \(A, J\) are similar.
\end{proof}

\exercisesection

\setcounter{ex}{3}
\begin{ex}\label{ex:7.1.4}
  Let \(\T\) be a linear operator on a vector space \(\V\) over \(\F\), and let \(\gamma\) be a cycle of generalized eigenvectors that corresponds to the eigenvalue \(\lambda\).
  Prove that \(\spn{\gamma}\) is a \(\T\)-invariant subspace of \(\V\) over \(\F\).
\end{ex}

\begin{proof}[\pf{ex:7.1.4}]
  Suppose that \(\gamma\) has length \(p\) and end vector \(x \in \V\).
  By \cref{7.1.6} we see that
  \[
    \gamma = \set{(\T - \lambda \IT[\V])^{p - 1}(x), \dots, (\T - \lambda \IT[\V])(x), x}.
  \]
  For each \(i \in \set{1, \dots, p}\), we define \(v_i = (\T - \lambda \IT[\V])^{p - i}(x)\).
  Observe that
  \begin{align*}
    \forall i \in \set{1, \dots, p}, (\T - \lambda \IT[\V])(v_i) & = (\T - \lambda \IT[\V])^{p - (i - 1)}(x)        \\
                                                                 & = \begin{dcases}
                                                                       \zv       & \text{if } i = 1                   \\
                                                                       v_{i - 1} & \text{if } i \in \set{2, \dots, p}
                                                                     \end{dcases} &  & \by{7.1.6}
  \end{align*}
  and
  \[
    \forall i \in \set{1, \dots, p}, \T(v_i) = \begin{dcases}
      \lambda v_1             & \text{if } i = 1                   \\
      \lambda v_i + v_{i - 1} & \text{if } i \in \set{2, \dots, p}
    \end{dcases}.
  \]
  Let \(y \in \spn{\gamma}\).
  By \cref{1.4.3} there exist some \(\seq{a}{1,,p} \in \F\) such that \(y = \sum_{i = 1}^p a_i v_i\).
  Since
  \begin{align*}
    \T(y) & = \T\pa{\sum_{i = 1}^p a_i v_i}                                  &  & \by{1.4.3}    \\
          & = \sum_{i = 1}^p a_i \T(v_i)                                     &  & \by{2.1.2}[d] \\
          & = a_1 \lambda v_1 + \sum_{i = 2}^p a_i (\lambda v_i + v_{i - 1}) &  & \by{7.1.6}    \\
          & \in \spn{\gamma},                                                &  & \by{1.4.3}
  \end{align*}
  by \cref{5.4.1} we see that \(\spn{\gamma}\) is \(\T\)-invariant.
\end{proof}

\begin{ex}\label{ex:7.1.5}
  Let \(\seq{\gamma}{1,,p}\) be cycles of generalized eigenvectors of a linear operator \(\T\) corresponding to an eigenvalue \(\lambda\).
  Prove that if the initial eigenvectors are distinct, then the cycles are disjoint.
\end{ex}

\begin{proof}[\pf{ex:7.1.5}]
  Suppose for sake of contradiction that there exist \(i, j \in \set{1, \dots, p}\) such that \(i \neq j\) and \(\gamma_i \cap \gamma_j \neq \varnothing\).
  Let \(m, n\) be lengths of \(\gamma_i, \gamma_j\), respectively.
  Let \(x, y\) be the end vector of \(\gamma_i, \gamma_j\), respectively.
  Then we have
  \begin{align*}
    \gamma_i & = \set{(\T - \lambda \IT[\V])^{m - 1}(x), \dots, (\T - \lambda \IT[\V])(x), x}; &  & \by{7.1.6} \\
    \gamma_j & = \set{(\T - \lambda \IT[\V])^{n - 1}(y), \dots, (\T - \lambda \IT[\V])(y), y}. &  & \by{7.1.6}
  \end{align*}
  By hypothesis we have \((\T - \lambda \IT[\V])^{m - 1}(x) \neq (\T - \lambda \IT[\V])^{n - 1}(y)\).
  Let \(z \in \beta_i \cap \beta_j\).
  Then there exist \(m' \in \set{0, \dots, m - 1}\) and \(n' \in \set{0, \dots, n - 1}\) such that
  \[
    z = (\T - \lambda \IT[\V])^{m'}(x) = (\T - \lambda \IT[\V])^{n'}(y).
  \]
  We cannot have \(m - m' < n - n'\) since
  \begin{align*}
             & (\T - \lambda \IT[\V])^{m - m'}(z) = (\T - \lambda \IT[\V])^m(x) = \zv     &  & \by{7.1.6}    \\
    \implies & (\T - \lambda \IT[\V])^{n - 1}(y) = (\T - \lambda \IT[\V])^{n - 1 - n'}(z) &  & \by{7.1.6}    \\
             & = (\T - \lambda \IT[\V])^{n - n' - m + m' + m - m' - 1}(z)                                    \\
             & = (\T - \lambda \IT[\V])^{n - n' - m + m' - 1}(\zv)                        &  & \by{7.1.6}    \\
             & = \zv                                                                      &  & \by{2.1.2}[a] \\
    \implies & \gamma_j \text{ has length } \leq n - 1,                                   &  & \by{7.1.6}
  \end{align*}
  a contradiction.
  Similarly we cannot have \(n - n' < m - m'\).
  Thus we must have \(n - n = m - m'\).
  But this means
  \[
    (\T - \lambda \IT[\V])^{m - 1}(x) = (\T - \lambda \IT[\V])^{m - 1 - m'}(z) = (\T - \lambda \IT[\V])^{n - 1 - n'}(z) = (\T - \lambda \IT[\V])^{n - 1}(y),
  \]
  a contradiction.
  Thus \(\gamma_i \cap \gamma_j = \varnothing\).
\end{proof}

\section{The Jordan Canonical Form II}\label{sec:7.2}

\begin{defn}\label{7.2.1}
  For the purposes of this section, we fix a linear operator \(\T\) on an \(n\)-dimensional vector space \(\V\) over \(\F\) such that the characteristic polynomial of \(\T\) splits.
  Let \(\seq{\lambda}{1,,k}\) be the distinct eigenvalues of \(\T\).

  By \cref{7.7}, each generalized eigenspace \(\vs{K}_{\lambda_i}\) contains an ordered basis \(\beta_i\) consisting of a union of disjoint cycles of generalized eigenvectors corresponding to \(\lambda_i\).
  So by \cref{7.4}(b) and \cref{7.5}, the union \(\beta = \bigcup_{i = 1}^k \beta_i\) is a Jordan canonical basis for \(\T\).
  For each \(i \in \set{1, \dots, k}\), let \(\T_i\) be the restriction of \(\T\) to \(\vs{K}_{\lambda_i}\), and let \(A_i = [\T_i]_{\beta_i}\).
  Then \(A_i\) is a Jordan canonical form of \(\T_i\), and
  \[
    J = [\T]_{\beta} = \begin{pmatrix}
      A_1    & \zm    & \cdots & \zm    \\
      \zm    & A_2    & \cdots & \zm    \\
      \vdots & \vdots &        & \vdots \\
      \zm    & \zm    & \cdots & A_k
    \end{pmatrix}
  \]
  is a Jordan canonical form of \(\T\).
  In this matrix, each \(\zm\) is a zero matrix of appropriate size.

  In this section, we compute the matrices \(A_i\) and the bases \(\beta_i\), thereby computing \(J\) and \(\beta\) as well.
  While developing a method for finding \(J\), it becomes evident that in some sense the matrices \(A_i\) are unique.

  To aid in formulating the uniqueness theorem for \(J\), we adopt the following convention:
  The basis \(\beta_i\) for \(\vs{K}_{\lambda_i}\) will henceforth be ordered in such a way that the cycles appear in order of decreasing length.
  That is, if \(\beta_i\) is a disjoint union of cycles \(\seq{\gamma}{1,,n_i}\) and if the length of the cycle \(\gamma_j\) is \(p_j\), we index the cycles so that \(\seq[\geq]{p}{1,,n_i}\).
  This ordering of the cycles limits the possible orderings of vectors in \(\beta_i\), which in turn determines the matrix \(A_i\).
  It is in this sense that \(A_i\) is unique.
  It then follows that the Jordan canonical form for \(\T\) is unique up to an ordering of the eigenvalues of \(\T\).
  As we will see, there is no uniqueness theorem for the bases \(\beta_i\) or for \(\beta\)
  (See \cref{ex:7.2.8}, it is for this reason that we associate the dot diagram with \(\T_i\) rather than with \(\beta_i\)).
  Specifically, we show that for each \(i \in \set{1, \dots, k}\), the number \(n_i\) of cycles that form \(\beta_i\), and the length \(p_j\) (\(j \in \set{1, \dots, n_i}\)) of each cycle, is completely determined by \(\T\).

  To help us visualize each of the matrices \(A_i\) and ordered bases \(\beta_i\), we use an array of dots called a \textbf{dot diagram} of \(\T_i\), where \(\T_i\) is the restriction of \(\T\) to \(\vs{K}_{\lambda_i}\).
  Suppose that \(\beta_i\) is a disjoint union of cycles of generalized eigenvectors \(\seq{\gamma}{1,,n_i}\) with lengths \(\seq[\geq]{p}{1,,n_i}\), respectively.
  The dot diagram of \(\T_i\) contains one dot for each vector in \(\beta_i\), and the dots are configured according to the following rules.
  \begin{itemize}
    \item The array consists of \(n_i\) columns (one column for each cycle).
    \item Counting from left to right, the \(j\)th column consists of the \(p_j\) dots that correspond to the vectors of \(\gamma_j\) starting with the initial vector at the top and continuing down to the end vector.
  \end{itemize}

  Denote the end vectors of the cycles by \(\seq{v}{1,,n_i}\).
  In the following dot diagram of \(\T_i\), each dot is labeled with the name of the vector in \(\beta_i\) to which it corresponds.
  \begin{align*}
     & \bullet (\T - \lambda_i \IT[\V])^{p_1 - 1}(v_1) &  & \bullet (\T - \lambda_i \IT[\V])^{p_2 - 1}(v_2) &  & \cdots &  & \bullet (\T - \lambda_i \IT[\V])^{p_{n_i} - 1}(v_{n_i}) \\
     & \bullet (\T - \lambda_i \IT[\V])^{p_1 - 2}(v_1) &  & \bullet (\T - \lambda_i \IT[\V])^{p_2 - 2}(v_2) &  & \cdots &  & \bullet (\T - \lambda_i \IT[\V])^{p_{n_i} - 2}(v_{n_i}) \\
     & \vdots                                          &  & \vdots                                          &  &        &  & \vdots                                                  \\
     &                                                 &  &                                                 &  &        &  & \bullet (\T - \lambda_i \IT[\V])(v_{n_i})               \\
     &                                                 &  &                                                 &  &        &  & \bullet v_{n_i}                                         \\
     &                                                 &  & \bullet (\T - \lambda_i \IT[\V])(v_2)           &  &        &  &                                                         \\
     &                                                 &  & \bullet v_2                                     &  &        &  &                                                         \\
     & \bullet (\T - \lambda_i \IT[\V])(v_1)           &  &                                                 &  &        &  &                                                         \\
     & \bullet v_1                                     &  &                                                 &  &        &  &
  \end{align*}
  Notice that the dot diagram of \(\T_i\) has \(n_i\) columns (one for each cycle) and \(p_1\) rows.
  Since \(\seq[\geq]{p}{1,,n_i}\), the columns of the dot diagram become shorter (or at least not longer) as we move from left to right.

  Now let \(r_j\) denote the number of dots in the \(j\)th row of the dot diagram.
  Observe that \(\seq[\geq]{r}{1,,p_1}\).
  Furthermore, the diagram can be reconstructed from the values of the \(r_i\)'s.
  The proofs of these facts, which are combinatorial in nature, are treated in \cref{ex:7.2.9}.
\end{defn}

\begin{thm}\label{7.9}
  Using the notations in \cref{7.2.1}.
  For any positive integer \(r\), the vectors in \(\beta_i\) that are associated with the dots in the first \(r\) rows of the dot diagram of \(\T_i\) constitute a basis for \(\ns{(\T - \lambda_i \IT[\V])^r}\) over \(\F\).
  Hence the number of dots in the first \(r\) rows of the dot diagram equals \(\nt{(\T - \lambda_i \IT[\V])^r}\).
\end{thm}

\begin{proof}[\pf{7.9}]
  By \cref{7.1.4} \(\ns{(\T - \lambda_i \IT[\V])^r} \subseteq \vs{K}_{\lambda_i}\), and \(\vs{K}_{\lambda_i}\) is invariant under \((\T - \lambda_i \IT[\V])^r\).
  Let \(\U\) denote the restriction of \((\T - \lambda_i \IT[\V])^r\) to \(\vs{K}_{\lambda_i}\).
  By \cref{7.1}(b), \(\ns{(\T - \lambda_i \IT[\V])^r} = \ns{\U}\), and hence it suffices to establish the theorem for \(\U\).
  Now define
  \[
    S_1 = \set{x \in \beta_i : \U(x) = \zv} \quad \text{and} \quad S_2 = \set{x \in \beta_i : \U(x) \neq \zv}.
  \]
  Let \(a\) and \(b\) denote the number of vectors in \(S_1\) and \(S_2\), respectively, and let \(m_i = \dim(\vs{K}_{\lambda_i})\).
  Then \(a + b = m_i\).
  For any \(x \in \beta_i\), \(x \in S_1\) iff \(x\) is one of the first \(r\) vectors of a cycle (\cref{7.1.6}), and this is true iff \(x\) corresponds to a dot in the first \(r\) rows of the dot diagram (\cref{7.2.1}).
  Hence \(a\) is the number of dots in the first \(r\) rows of the dot diagram.
  For any \(x \in S_2\), the effect of applying \(\U\) to \(x\) is to move the dot corresponding to \(x\) exactly \(r\) places up its column to another dot.
  It follows that \(\U\) maps \(S_2\) in a one-to-one fashion into \(\beta_i\) (since \(\beta_i\) is linearly independent).
  Thus \(\set{\U(x) : x \in S_2}\) is a basis for \(\rg{\U}\) over \(\F\) consisting of \(b\) vectors (\cref{2.2}).
  Hence \(\rk{\U} = b\), and so \(\nt{\U} = m_i - b = a\) (\cref{2.3}).
  But \(S_1\) is a linearly independent subset of \(\ns{\U}\) consisting of \(a\) vectors;
  therefore \(S_1\) is a basis for \(\ns{\U}\).
\end{proof}

\begin{note}
  In the case that \(r = 1\), \cref{7.9} yields \cref{7.2.2}.
\end{note}

\begin{cor}\label{7.2.2}
  Using the notations in \cref{7.2.1}.
  The dimension of \(\vs{E}_{\lambda_i}\) is \(n_i\).
  Hence in a Jordan canonical form of \(\T\), the number of Jordan blocks corresponding to \(\lambda_i\) equals the dimension of \(\vs{E}_{\lambda_i}\).
\end{cor}

\begin{proof}[\pf{7.2.2}]
  By \cref{5.4} we have \(\vs{E}_{\lambda_i} = \ns{\T - \lambda_i \IT[\V]}\).
  Thus by \cref{7.9} the first row of the dot diagram of \(\T_i\) constitute a basis for \(\vs{E}_{\lambda_i}\) over \(\F\).
\end{proof}

\begin{thm}\label{7.10}
  Using the notations in \cref{7.2.1}.
  Let \(r_j\) denote the number of dots in the \(j\)th row of the dot diagram of \(\T_i\), the restriction of \(\T\) to \(\vs{K}_{\lambda_i}\).
  Then the following statements are true.
  \begin{enumerate}
    \item \(r_1 = \dim(\V) - \rk{\T - \lambda_i \IT[\V]}\).
    \item \(r_j = \rk{(\T - \lambda_i \IT[\V])^{j - 1}} - \rk{(\T - \lambda_i \IT[\V])^j}\) if \(j \in \set{2, \dots, p_1}\).
  \end{enumerate}
\end{thm}

\begin{proof}[\pf{7.10}]
  By \cref{7.9}, for \(j \in \set{1, \dots, p_1}\), we have
  \begin{align*}
    \seq[+]{r}{1,,j} & = \nt{(\T - \lambda_i \IT[\V])^j}             &  & \by{7.9} \\
                     & = \dim(\V) - \rk{(\T - \lambda_i \IT[\V])^j}. &  & \by{2.3}
  \end{align*}
  Hence \(r_1 = \dim(\V) - \rk{\T - \lambda_i \IT[\V]}\), and for \(j \in \set{2, \dots, p_1}\),
  \begin{align*}
    r_j & = (\seq[+]{r}{1,,j}) - (\seq[+]{r}{1,,j-1})                                                               \\
        & = \pa{\dim(\V) - \rk{(\T - \lambda_i \IT[\V])^j}} - \pa{\dim(\V) - \rk{(\T - \lambda_i \IT[\V])^{j - 1}}} \\
        & = \rk{(\T - \lambda_i \IT[\V])^{j - 1}} - \rk{(\T - \lambda_i \IT[\V])^j}.
  \end{align*}
\end{proof}

\begin{cor}\label{7.2.3}
  Using the notations in \cref{7.2.1}.
  For any eigenvalue \(\lambda_i\) of \(\T\), the dot diagram of \(\T_i\) is unique.
  Thus, subject to the convention that the cycles of generalized eigenvectors for the bases of each generalized eigenspace are listed in order of decreasing length, the Jordan canonical form of a linear operator or a matrix is unique up to the ordering of the eigenvalues.
\end{cor}

\begin{proof}[\pf{7.2.3}]
  \cref{7.10} shows that the dot diagram of \(\T_i\) is completely determined by \(\T\) and \(\lambda_i\).
\end{proof}

\begin{thm}\label{7.11}
  Let \(A\) and \(B\) be \(n \times n\) matrices, each having Jordan canonical forms computed according to the conventions of this section.
  Then \(A\) and \(B\) are similar iff they have (up to an ordering of their eigenvalues) the same Jordan canonical form.
\end{thm}

\begin{proof}[\pf{7.11}]
  If \(A\) and \(B\) have the same Jordan canonical form \(J\), then \(A\) and \(B\) are each similar to \(J\) and hence are similar to each other.

  Conversely, suppose that \(A\) and \(B\) are similar.
  Then \(A\) and \(B\) have the same eigenvalues (\cref{ex:5.1.12}).
  Let \(J_A\) and \(J_B\) denote the Jordan canonical forms of \(A\) and \(B\), respectively, with the same ordering of their eigenvalues.
  Then \(A\) is similar to both \(J_A\) and \(J_B\), and therefore, by the \cref{2.5.3}, \(J_A\) and \(J_B\) are matrix representations of \(\L_A\).
  Hence \(J_A\) and \(J_B\) are Jordan canonical forms of \(\L_A\). Thus \(J_A = J_B\) by the \cref{7.2.3}.
\end{proof}

\begin{cor}\label{7.2.4}
  A linear operator \(\T\) on a finite-dimensional vector space \(\V\) over \(\F\) is diagonalizable iff its Jordan canonical form is a diagonal matrix.
  Hence \(\T\) is diagonalizable iff the Jordan canonical basis for \(\T\) consists of eigenvectors of \(\T\).
\end{cor}

\begin{proof}[\pf{7.2.4}]
  By \cref{7.1.5,7.2.3} we see that this is true.
\end{proof}

\exercisesection

\setcounter{ex}{5}
\begin{ex}\label{ex:7.2.6}
  Let \(A \in \ms[n][n][\F]\) whose characteristic polynomial splits.
  Prove that \(A\) and \(\tp{A}\) have the same Jordan canonical form, and conclude that \(A\) and \(\tp{A}\) are similar.
\end{ex}

\begin{proof}[\pf{ex:7.2.6}]
  Let \(\lambda \in \F\) be an eigenvalue of \(A\).
  Since
  \begin{align*}
    \forall r \in \Z^+, \rk{\pa{A - \lambda I_n}^r} & = \rk{\tp{\pa{(A - \lambda I_n)^r}}} &  & \by{3.2.5}[a] \\
                                                    & = \rk{\pa{\tp{(A - \lambda I_n)}}^r} &  & \by{2.3.2}    \\
                                                    & = \rk{\pa{\tp{A} - \lambda I_n}^r},  &  & \by{ex:1.3.3}
  \end{align*}
  by \cref{7.10} we see that the dot diagrams of \(A\) and \(\tp{A}\) correspond to \(\lambda\) are the same.
  Since \(\lambda\) is arbitrary, by \cref{7.2.3} we conclude that \(A\) and \(\tp{A}\) have the same Jordan canonical form.
  By \cref{7.11} this means \(A\) and \(\tp{A}\) are similar.
\end{proof}

\begin{ex}\label{ex:7.2.7}
  Let \(\T\) be a linear operator on a finite-dimensional vector space \(\V\) over \(\F\) such that the characteristic polynomial of \(\T\) splits.
  Let \(\gamma\) be a cycle of generalized eigenvectors corresponding to an eigenvalue \(\lambda\), and \(\W\) be the subspace spanned by \(\gamma\).
  Define \(\gamma'\) to be the ordered set obtained from \(\gamma\) by reversing the order of the vectors in \(\gamma\).
  \begin{enumerate}
    \item Prove that \([\T_{\W}]_{\gamma'} = \tp{([\T_{\W}]_{\gamma})}\).
    \item Let \(J\) be the Jordan canonical form of \(\T\).
          Use (a) to prove that \(J\) and \(\tp{J}\) are similar.
    \item Let \(A \in \ms[n][n][\F]\) whose characteristic polynomial splits.
          Use (b) to prove that \(A\) and \(\tp{A}\) are similar.
  \end{enumerate}
\end{ex}

\begin{proof}[\pf{ex:7.2.7}(a)]
  Let \(\gamma = \set{\seq{v}{1,,m}}\).
  By \cref{7.1.6} we have
  \[
    \forall i \in \set{1, \dots, m}, \T(v_i) = \begin{dcases}
      \lambda v_i             & \text{if } i = 1                   \\
      \lambda v_i + v_{i - 1} & \text{if } i \in \set{2, \dots, m}
    \end{dcases}.
  \]
  Now define \(\gamma' = \set{v_1', \dots, v_m'}\).
  By definition we have \(v_i' = v_{m + 1 - i}\) for all \(i \in \set{1, \dots, m}\) and
  \begin{align*}
    \forall i \in \set{1, \dots, m}, \T(v_i') & = \T(v_{m + 1 - i})                                                              \\
                                              & = \begin{dcases}
                                                    \lambda v_{m + 1 - i}             & \text{if } m + 1 - i = 1                   \\
                                                    \lambda v_{m + 1 - i} + v_{m - i} & \text{if } m + 1 - i \in \set{2, \dots, m}
                                                  \end{dcases} \\
                                              & = \begin{dcases}
                                                    \lambda v_i'              & \text{if } i = m                       \\
                                                    \lambda v_i' + v_{i + 1}' & \text{if } i \in \set{1, \dots, m - 1}
                                                  \end{dcases}.
  \end{align*}
  Thus we have
  \begin{align*}
    \forall i, j \in \set{1, \dots, m}, \pa{\tp{([\T_{\W}]_{\gamma})}}_{i j} & = ([\T_{\W}]_{\gamma})_{j i}      &  & \by{1.3.3} \\
                                                                             & = \begin{dcases}
                                                                                   \lambda & \text{if } i = j     \\
                                                                                   1       & \text{if } i + 1 = j \\
                                                                                   0       & \text{otherwise}
                                                                                 \end{dcases} &  & \by{7.1.1}                  \\
                                                                             & = ([\T_{\W}]_{\gamma'})_{i j}.    &  & \by{2.2.4}
  \end{align*}
  By \cref{1.2.8} this means \(\tp{([\T_{\W}]_{\beta})} = [\T_{\W}]_{\gamma'}\).
\end{proof}

\begin{proof}[\pf{ex:7.2.7}(b)]
  Let \(\beta\) be an Jordan canonical basis following the convention in \cref{7.2.1}.
  Let \(J = [\T]_{\beta}\).
  Let \(\beta'\) be the ordered set obtained from \(\beta\) by reversing the ordered of each disjoint cycles in \(\beta\).
  By \cref{5.25} and \cref{ex:7.2.7}(a) we see that \([\T]_{\beta'} = \tp{([\T]_{\beta})} = \tp{J}\).
  By \cref{2.23} we hve \([\T]_{\beta'} = \pa{[\IT[\V]]_{\beta'}^{\beta}}^{-1} [\T]_{\beta} [\IT[\V]]_{\beta'}^{\beta}\).
  Thus by \cref{2.5.4} \(J\) and \(\tp{J}\) are similar.
\end{proof}

\begin{proof}[\pf{ex:7.2.7}(c)]
  Let \(\beta\) be the standard ordered basis for \(\vs{F}^n\) over \(\F\) and let \(\alpha\) be a Jordan canonical basis for \(A\).
  By \cref{ex:7.2.7}(b) we see that \([\L_A]_{\alpha}\) and \(\tp{([\L_A]_{\alpha})}\) are similar.
  By \cref{2.23} we know that \(A = [\L_A]_{\beta}\) and \([\L_A]_{\alpha}\) are similar.
  Thus by \cref{ex:2.5.9} we know that \(A\) and \(\tp{([\L_A]_{\alpha})}\) are similar.
  If we can show that \(\tp{A}\) and \(\tp{([\L_A]_{\alpha})}\) are similar, then by \cref{ex:2.5.9} again we see that \(A\) and \(\tp{A}\) are similar.
  This is true since
  \begin{align*}
    \tp{A} & = \tp{([\L_A]_{\beta})}                                                                                       &  & \by{2.15}[a]  \\
           & = \tp{\pa{\pa{[\IT[\V]]_{\beta}^{\alpha}}^{-1} [\L_A]_{\alpha} [\IT[\V]]_{\beta}^{\alpha}}}                   &  & \by{2.23}     \\
           & = \tp{([\IT[\V]]_{\beta}^{\alpha})} \tp{([\L_A]_{\alpha})} \tp{\pa{\pa{[\IT[\V]]_{\beta}^{\alpha}}^{-1}}}     &  & \by{2.3.2}    \\
           & = \tp{\pa{\pa{[\IT[\V]]_{\alpha}^{\beta}}^{-1}}} \tp{([\L_A]_{\alpha})} \tp{\pa{[\IT[\V]]_{\alpha}^{\beta}}}  &  & \by{2.23}     \\
           & = \pa{\tp{\pa{[\IT[\V]]_{\alpha}^{\beta}}}}^{-1} \tp{([\L_A]_{\alpha})} \tp{\pa{[\IT[\V]]_{\alpha}^{\beta}}}. &  & \by{ex:2.4.5}
  \end{align*}
\end{proof}

\begin{ex}\label{ex:7.2.8}
  Let \(\T\) be a linear operator on a finite-dimensional vector space \(\V\) over \(\F\), and suppose that the characteristic polynomial of \(\T\) splits.
  Let \(\beta\) be a Jordan canonical basis for \(\T\).
  \begin{enumerate}
    \item Prove that for any nonzero scalar \(c\), \(\set{cx : x \in \beta}\) is a Jordan canonical basis for \(\T\).
    \item Suppose that \(\gamma\) is one of the cycles of generalized eigenvectors that forms \(\beta\), and suppose that \(\gamma\) corresponds to the eigenvalue \(\lambda\) and has length greater than \(1\).
          Let \(x\) be the end vector of \(\gamma\), and let \(y\) be a nonzero vector in \(\vs{E}_{\lambda}\).
          Let \(\gamma'\) be the ordered set obtained from \(\gamma\) by replacing \(x\) by \(x + y\).
          Prove that \(\gamma'\) is a cycle of generalized eigenvectors corresponding to \(\lambda\), and that if \(\gamma'\) replaces \(\gamma\) in the union that defines \(\beta\), then the new union is also a Jordan canonical basis for \(\T\).
  \end{enumerate}
\end{ex}

\begin{proof}[\pf{ex:7.2.8}(a)]
  Let \(\lambda\) be an eigenvalue of \(\T\) and let \(x \in \beta\) be an generalized eigenvector of \(\T\) corresponding to \(\lambda\).
  If \(x \in \vs{E}_{\lambda}\), then by \cref{5.1.2} we have \(\T(x) = \lambda x\).
  If \(x \in \vs{K}_{\lambda} \setminus \vs{E}_{\lambda}\), then by \cref{7.1.1} there exists a \(v \in \beta\) such that \(\T(x) = \lambda x + v\).
  In either cases we have
  \[
    \T(cx) = c \T(x) = \begin{dcases}
      c \lambda x \\
      c \lambda x + cv
    \end{dcases} = \begin{dcases}
      \lambda (cx) \\
      \lambda (cx) + (cv)
    \end{dcases}.
  \]
  Thus by \cref{7.1.1} \(\set{cx : x \in \beta}\) is a Jordan canonical basis for \(\T\).
\end{proof}

\begin{proof}[\pf{ex:7.2.8}(b)]
  Since
  \begin{align*}
    (\T - \lambda \IT[\V])(x + y) & = (\T - \lambda \IT[\V])(x) + (\T - \lambda \IT[\V])(y) &  & \by{2.1.1}[a] \\
                                  & = (\T - \lambda \IT[\V])(x) + \zv                       &  & \by{5.4}      \\
                                  & = (\T - \lambda \IT[\V])(x),                            &  & \by{1.2.1}
  \end{align*}
  we see that \((\T - \lambda \IT[\V])^p(x + y) = (\T - \lambda \IT[\V])^p(x)\) for any \(p \in \Z^+\).
  Thus \(\gamma'\) is a cycle of generalized eigenvectors corresponding to \(\lambda\), and the rest claim follows.
\end{proof}

\begin{ex}\label{ex:7.2.9}
  Suppose that a dot diagram has \(k\) columns and \(m\) rows with \(p_j\) dots in column \(j\) and \(r_i\) dots in row \(i\).
  Prove the following results.
  \begin{enumerate}
    \item \(m = p_1\) and \(k = r_1\).
    \item We have
          \begin{align*}
             & \forall j \in \set{1, \dots, k}, p_j = \max \set{i \in \set{1, \dots, m} : r_i \geq j}; \\
             & \forall i \in \set{1, \dots, m}, r_i = \max \set{j \in \set{1, \dots, k} : p_j \geq i}.
          \end{align*}
    \item \(\seq[\geq]{r}{1,,m}\).
    \item Deduce that the number of dots in each column of a dot diagram is completely determined by the number of dots in the rows.
  \end{enumerate}
\end{ex}

\begin{proof}[\pf{ex:7.2.9}(a)]
  By \cref{7.2.1} we know that \(p_1\) is the column with the most number of dots.
  Thus \(m = p_1\).
  Since the number of columns equals the number of disjoint cycles, we have \(k = r_1\).
\end{proof}


\begin{proof}[\pf{ex:7.2.9}(b)]
  We first fix \(k\) and use induction on \(m\) to prove that
  \[
    \forall i \in \set{1, \dots, m}, r_i = \max \set{j \in \set{1, \dots, k} : p_j \geq i}.
  \]
  For \(m = 1\), our dot diagram have \(1\) row and \(k\) columns.
  This means the first row has \(k\) dots and each column has one dot.
  Thus
  \begin{align*}
    \forall i \in \set{1}, r_i & = k                                                    \\
                               & = \max \set{1, \dots, k}                               \\
                               & = \max \set{j \in \set{1, \dots, k} : 1 = p_j \geq i}.
  \end{align*}
  So the base case holds.
  Suppose inductively that for some \(m \geq 1\) the statement is true.
  We need to show that for \(m + 1\) the statement is also true.
  So suppose that there are \(k\) columns and \(m + 1\) rows in a dot diagram.
  By \cref{7.2.1} we see that columns in dot diagram are ordered by decreasing length.
  Thus the first \(r_{m + 1}\) columns are the longest columns of all.
  Since there are \(m + 1\) rows, we know that the first \(r_{m + 1}\) columns have \(m + 1\) dots and the rest \((k - r_{m + 1})\) columns have less than \(m + 1\) dots.
  Thus we have
  \begin{align*}
             & \begin{dcases}
                 \forall j \in \set{1, \dots, r_{m + 1}}, p_j = m + 1 \\
                 \forall j \in \set{r_{m + 1} + 1, \dots, k}, p_j < m + 1
               \end{dcases} \\
    \implies & r_{m + 1} = \max \set{1, \dots, r_{m + 1}}                                   \\
             & = \max \set{j \in \set{1, \dots, r_{m + 1}} : p_j \geq m + 1}                \\
             & = \max \set{j \in \set{1, \dots, k} : p_j \geq m + 1}.
  \end{align*}
  By induction hypothesis we have
  \[
    \forall i \in \set{1, \dots, m + 1}, r_i = \max \set{j \in \set{1, \dots, k} : p_j \geq i}.
  \]
  This closes the induction.

  Now we fix \(m\) and use induction on \(k\) to prove that
  \[
    \forall j \in \set{1, \dots, k}, p_j = \max \set{i \in \set{1, \dots, m} : r_i \geq j}.
  \]
  For \(k = 1\), our dot diagram have \(m\) rows and \(1\) column.
  This means the first column has \(m\) dots and each row has one dot.
  Thus
  \begin{align*}
    \forall j \in \set{1}, p_j & = m                                                    \\
                               & = \max \set{1, \dots, m}                               \\
                               & = \max \set{i \in \set{1, \dots, m} : 1 = r_i \geq j}.
  \end{align*}
  So the base case holds.
  Suppose inductively that for some \(k \geq 1\) the statement is true.
  We need to show that for \(k + 1\) the statement is also true.
  So suppose that there are \(k + 1\) columns and \(m\) rows in a dot diagram.
  By \cref{7.2.1} we see that columns in dot diagram are ordered by decreasing length.
  Thus the first \(p_{k + 1}\) rows are the longest rows of all.
  Since there are \(k + 1\) columns, we know that the first \(p_{k + 1}\) rows have \(k + 1\) dots and the rest \((m - p_{k + 1})\) rows have less than \(k + 1\) dots.
  Thus we have
  \begin{align*}
             & \begin{dcases}
                 \forall i \in \set{1, \dots, p_{k + 1}}, r_i = k + 1 \\
                 \forall i \in \set{p_{k + 1} + 1, \dots, m}, r_i < k + 1
               \end{dcases} \\
    \implies & p_{k + 1} = \max \set{1, \dots, p_{k + 1}}                                   \\
             & = \max \set{i \in \set{1, \dots, p_{k + 1}} : r_i \geq k + 1}                \\
             & = \max \set{i \in \set{1, \dots, m} : r_i \geq k + 1}.
  \end{align*}
  By induction hypothesis we have
  \[
    \forall j \in \set{1, \dots, k + 1}, p_j = \max \set{i \in \set{1, \dots, m} : r_i \geq j}.
  \]
  This closes the induction.
\end{proof}

\begin{proof}[\pf{ex:7.2.9}(c)]
  Suppose that \(i_1, i_2 \in \set{1, \dots, m}\) and \(i_1 < i_2\).
  Since
  \begin{align*}
             & i_1 < i_2                                                                                                                      \\
    \implies & \set{j \in \set{1, \dots, k} : p_j \geq i_2} \subseteq \set{j \in \set{1, \dots, k} : p_j \geq i_1}                            \\
    \implies & \max \set{j \in \set{1, \dots, k} : p_j \geq i_2} \leq \max \set{j \in \set{1, \dots, k} : p_j \geq i_1} &  & \by{7.2.1}       \\
    \implies & r_{i_2} \leq r_{i_1}                                                                                     &  & \by{ex:7.2.9}[b]
  \end{align*}
  and \(i_1, i_2\) are arbitrary, we have \(\seq[\geq]{r}{1,,m}\).
\end{proof}

\begin{proof}[\pf{ex:7.2.9}(d)]
  By \cref{ex:7.2.9}(b) we have
  \[
    \forall j \in \set{1, \dots, k}, p_j = \max{i \in \set{1, \dots, m} : r_i \geq j}.
  \]
  This means when number of dots in the rows of a dot diagram is defined, the number of dots in the columns of the same dot diagram is also defined.
\end{proof}

\begin{ex}\label{ex:7.2.10}
  Let \(\T\) be a linear operator whose characteristic polynomial splits, and let \(\lambda\) be an eigenvalue of \(\T\).
  \begin{enumerate}
    \item Prove that \(\dim(\vs{K}_{\lambda})\) is the sum of the lengths of all the cycles corresponding to \(\lambda\) in the Jordan canonical form of \(\T\).
    \item Deduce that \(\vs{E}_{\lambda} = \vs{K}_{\lambda}\) iff all the Jordan blocks corresponding to \(\lambda\) are \(1 \times 1\) matrices.
  \end{enumerate}
\end{ex}

\begin{proof}[\pf{ex:7.2.10}(a)]
  By \cref{7.6} we know that the union of disjoint cycles are linearly independent.
  By \cref{7.5} each cycle forms a Jordan block corresponding to \(\lambda\), and the union of cycles is a basis for \(\vs{K}_{\lambda}\).
  Thus \(\dim(\vs{K}_{\lambda})\) is the sum of the lengths of all cycles corresponding to \(\lambda\).
\end{proof}

\begin{proof}[\pf{ex:7.2.10}(b)]
  We have
  \begin{align*}
         & \vs{E}_{\lambda} = \vs{K}_{\lambda}                                                         \\
    \iff & \text{each cycle has length } 1                                             &  & \by{5.4}   \\
    \iff & \text{all Jordan blocks corresponding to } \lambda \text{ are } 1 \times 1. &  & \by{7.1.1}
  \end{align*}
\end{proof}

\begin{defn}\label{7.2.5}
  A linear operator \(\T\) on a vector space \(\V\) is called \textbf{nilpotent} if \(\T^p = \zT\) for some positive integer \(p\).
  An \(n \times n\) matrix \(A\) is called \textbf{nilpotent} if \(A^p = \zm\) for some positive integer \(p\).
\end{defn}

\begin{ex}\label{ex:7.2.11}
  Let \(\T\) be a linear operator on a finite-dimensional vector space \(\V\) over \(\F\), and let \(\beta\) be an ordered basis for \(\V\) over \(\F\).
  Prove that \(\T\) is nilpotent iff \([\T]_{\beta}\) is nilpotent.
\end{ex}

\begin{proof}[\pf{ex:7.2.11}]
  We have
  \begin{align*}
         & \T \text{ is nilpotent}                                                                            \\
    \iff & \exists p \in \Z^+ : \T^p = \zT                                              &  & \by{7.2.5}       \\
    \iff & \exists p \in \Z^+ : ([\T]_{\beta})^p = [\T^p]_{\beta} = [\zT]_{\beta} = \zm &  & \by{2.11,2.1.13} \\
    \iff & [\T]_{\beta} \text{ is nilpotent}.                                           &  & \by{7.2.5}
  \end{align*}
\end{proof}

\begin{ex}\label{ex:7.2.12}
  Prove that any square upper triangular matrix with each diagonal entry equal to zero is nilpotent.
\end{ex}

\begin{proof}[\pf{ex:7.2.12}]
  Let \(A \in \ms[n][n][\F]\) be an upper triangular matrix with each diagonal entry equal to zero.
  Let \(\beta = \set{\seq{e}{1,,n}}\) be the standard ordered basis for \(\vs{F}^n\) over \(\F\).
  By \cref{2.2.4} we have \(\L_A(e_1) = \zv\) and
  \[
    \forall i \in \set{2, \dots, n}, \L_A(e_i) = \sum_{k = 1}^{i - 1} A_{k i} e_k.
  \]
  This means \(\L_A^n(e_i) = \zv\) for all \(i \in \set{1, \dots, n}\).
  Thus by \cref{2.1.13} \(\L_A^p = \zT\).
  By \cref{7.2.5} this means \(\L_A\) is nilpotent.
  By \cref{ex:7.2.11} we conclude that \(A\) is nilpotent.
\end{proof}

\begin{ex}\label{ex:7.2.13}
  Let \(\T\) be a nilpotent operator on an \(n\)-dimensional vector space \(\V\) over \(\F\), and suppose that \(p\) is the smallest positive integer for which \(\T^p = \zT\).
  Prove the following results.
  \begin{enumerate}
    \item \(\ns{\T^i} \subseteq \ns{\T^{i + 1}}\) for every positive integer \(i\).
    \item There is a sequence of ordered bases \(\seq{\beta}{1,,p}\) such that \(\beta_i\) is a basis for \(\ns{\T^i}\) over \(\F\) and \(\beta_{i + 1}\) contains \(\beta_i\) for \(i \in \set{1, \dots, p - 1}\).
    \item Let \(\beta = \beta_p\) be the ordered basis for \(\ns{\T^p} = \V\) over \(\F\) in (b).
          Then \([\T]_{\beta}\) is an upper triangular matrix with each diagonal entry equal to zero.
    \item The characteristic polynomial of \(\T\) is \((-1)^n t^n\).
          Hence the characteristic polynomial of \(\T\) splits, and \(0\) is the only eigenvalue of \(\T\).
  \end{enumerate}
\end{ex}

\begin{proof}[\pf{ex:7.2.13}(a)]
  See \cref{ex:7.1.7}(a).
\end{proof}

\begin{proof}[\pf{ex:7.2.13}(b)]
  Let \(\beta_1\) be a basis for \(\ns{\T}\) over \(\F\).
  Since \(\beta_1 \subseteq \ns{\T} \subseteq \ns{\T^2}\), by \cref{1.6.15}(c) we can extend \(\beta_1\) to an ordered basis \(\beta_2\) for \(\ns{\T^2}\) over \(\F\).
  Similarly we can extend \(\beta_2\) to an ordered basis \(\beta_3\) for \(\ns{\T^3}\) over \(\F\).
  Continue this process we can construct the sequence \(\seq{\beta}{1,,p}\) which satisfy the requirement of \cref{ex:7.2.13}(b).
\end{proof}

\begin{proof}[\pf{ex:7.2.13}(c)]
  Let \(i \in \set{1, \dots, p - 1}\).
  By \cref{ex:7.1.7}(c) we have \(\ns{\T^i} \neq \ns{\T^{i + 1}}\).
  Thus
  \begin{align*}
             & \beta_{i + 1} \setminus \beta_i \neq \varnothing                   &  & \by{ex:7.2.13}[b]                 \\
    \implies & \forall v \in \beta_{i + 1} \setminus \beta_i, \begin{dcases}
                                                                \T^i(v) \neq \zv \\
                                                                \T^{i + 1}(v) = \zv
                                                              \end{dcases}      &  & \by{2.1.10}                         \\
    \implies & \forall v \in \beta_{i + 1} \setminus \beta_i, \T(v) \in \ns{\T^i} &  & \by{2.1.10}                       \\
    \implies & \forall v \in \beta_{i + 1}, \T(v) \in \ns{\T^i}                   &  & (\beta_i \subseteq \beta_{i + 1}) \\
    \implies & \T(\beta_{i + 1}) \subseteq \ns{\T^i} = \spn{\beta_i}.             &  & \by{ex:7.2.13}[b]
  \end{align*}
  By \cref{2.2.4} this means \([\T]_{\beta}\) is an upper triangular matrix with each diagonal entry equal to \(0\).
\end{proof}

\begin{proof}[\pf{ex:7.2.13}(d)]
  By \cref{ex:7.2.13}[c] we know that \([\T]_{\beta}\) is an upper triangular matrix with each diagonal entry equal to \(0\).
  Thus by \cref{ex:4.2.23} we see that the characteristic polynomial of \(\T\) is \((-1)^n t^n\).
  By \cref{5.2} we know that \(0\) is the only eigenvalue of \(\T\).
\end{proof}

\begin{ex}\label{ex:7.2.14}
  Prove the converse of \cref{ex:7.2.13}(d):
  If \(\T\) is a linear operator on an \(n\)-dimensional vector space \(\V\) over \(\F\) and \((-1)^n t^n\) is the characteristic polynomial of \(\T\), then \(\T\) is nilpotent.
\end{ex}

\begin{proof}[\pf{ex:7.2.14}]
  Since the characteristic polynomial of \(\T\) splits, by \cref{7.1.8} we know that \(\T\) has a Jordan form and a Jordan canonical basis \(\beta\).
  By \cref{5.2} we know that \(0\) is the only eigenvalue of \(\T\), thus by \cref{7.1.1} \([\T]_{\beta}\) is an upper triangular matrix with each diagonal entry equal to \(0\).
  By \cref{ex:7.2.12} we know that \([\T]_{\beta}\) is nilpotent.
  Thus by \cref{ex:7.2.11} we conclude that \(\T\) is nilpotent.
\end{proof}

\begin{ex}\label{ex:7.2.15}
  Give an example of a linear operator \(\T\) on a finite-dimensional vector space \(\V\) over \(\F\) such that \(\T\) is not nilpotent, but zero is the only eigenvalue of \(\T\).
  Characterize all such operators.
\end{ex}

\begin{proof}[\pf{ex:7.2.15}]
  First we give an example as required.
  Let \(\V = \R^3\) and let \(\F = \R\).
  Define
  \[
    A = \begin{pmatrix}
      0 & 0  & 0 \\
      0 & 0  & 1 \\
      0 & -1 & 0
    \end{pmatrix}.
  \]
  Then \(\det(A - t I_3) = (-t)(t^2 + 1)\).
  Thus the characteristic polynomial of \(A\) does not split and \(0\) is the only eigenvalue of \(A\).
  Since \(A^3 = -A\), we know that \(A^p \neq \zm\) for all \(p \in \Z^+\).
  Thus by \cref{7.2.5} \(A\) is not nilpotent.

  Now we characterize all such operator.
  By \cref{ex:7.2.13}(d) and \cref{ex:7.2.14} we know that \(\T\) is nilpotent iff the characteristic polynomial of \(\T\) is \((-1)^n t^n\).
  Thus \(\T\) is not nilpotent iff the characteristic polynomial of \(\T\) is not \((-1)^n t^n\).
\end{proof}

\begin{ex}\label{ex:7.2.16}
  Let \(\T\) be a nilpotent linear operator on a finite-dimensional vector space \(\V\) over \(\F\).
  Recall from \cref{ex:7.2.13} that \(\lambda = 0\) is the only eigenvalue of \(\T\), and hence \(\V = \vs{K}_0\).
  Let \(\beta\) be a Jordan canonical basis for \(\T\).
  Prove that for any positive integer \(i\), if we delete from \(\beta\) the vectors corresponding to the last \(i\) dots in each column of a dot diagram of \(\beta\), the resulting set is a basis for \(\rg{\T^i}\) over \(\F\).
  (If a column of the dot diagram contains fewer than \(i\) dots, all the vectors associated with that column are removed from \(\beta\).)
\end{ex}

\begin{proof}[\pf{ex:7.2.16}]
  Let \(\gamma\) be a cycle in \(\beta\) with length \(p\) and end vector \(x\).
  By \cref{7.1.6} we have
  \[
    \gamma = \set{\T^{p - 1}(x), \dots, \T^i(x), \T^{i - 1}(x), \dots, \T(x), x}.
  \]
  Let \(\alpha\) be the set obtained from removing the last \(i\) vector in \(\gamma\), i.e.,
  \[
    \alpha = \set{\T^{p - 1}(x), \dots, \T^i(x)}.
  \]
  Note that \(\alpha = \varnothing\) when \(p \leq i\).
  If \(\alpha \neq \varnothing\), then each vector in \(\alpha\) is mapped by \(\T^i\) from exactly one vector in \(\gamma \setminus \alpha\).
  Thus by \cref{2.2} \(\alpha\) is linearly independent and the union of all \(\alpha\) is a basis for \(\rg{\T^i}\) over \(\F\).
\end{proof}

\begin{ex}\label{ex:7.2.20}
\end{ex}

\begin{ex}\label{ex:7.2.21}
\end{ex}

\section{The Minimal Polynomial}\label{sec:7.3}



\section{The Rational Canonical Form}\label{sec:7.4}


% Begin of appendix.
\appendix

% All appendices are in separated files.  We include them here.
\chapter{Sets}\label{ch:a}

\begin{defn}\label{a.0.1}
  A \textbf{set} is a collection of objects, called \textbf{elements} of the set.
  If \(x\) is an element of the set \(A\), then we write \(x \in A\);
  otherwise, we write \(x \notin A\).
\end{defn}

\begin{note}
  One set that appears frequently is the set of real numbers, which we denote by \(\R\) throughout this text.
\end{note}

\begin{defn}\label{a.0.2}
  Two sets \(A\) and \(B\) are called equal, written \(A = B\), if they contain exactly the same elements.
  Sets may be described in one of two ways:
  \begin{enumerate}
    \item By listing the elements of the set between set braces \(\set{}\).
    \item By describing the elements of the set in terms of some characteristic property.
  \end{enumerate}
\end{defn}

\begin{note}
  The order in which the elements of a set are listed is immaterial.
\end{note}

\begin{defn}\label{a.0.3}
  A set \(B\) is called a \textbf{subset} of a set \(A\), written \(B \subseteq A\) or \(A \supseteq B\), if every element of \(B\) is an element of \(A\).
  If \(B \subseteq A\), and \(B \neq A\), then \(B\) is called a \textbf{proper subset} of \(A\).
  Observe that \(A = B\) iff \(A \subseteq B\) and \(B \subseteq A\), a fact that is often used to prove that two sets are equal.
\end{defn}

\begin{defn}\label{a.0.4}
  The \textbf{empty set}, denoted by \(\varnothing\), is the set containing no elements.
  The empty set is a subset of every set.
\end{defn}

\begin{defn}\label{a.0.5}
  Sets may be combined to form other sets in two basic ways.
  The \textbf{union} of two sets \(A\) and \(B\), denoted \(A \cup B\), is the set of elements that are in \(A\), or \(B\), or both;
  that is,
  \[
    A \cup B = \set{x : x \in A \text{ or } x \in B}.
  \]
  The \textbf{intersection} of two sets \(A\) and \(B\), denoted \(A \cap B\), is the set of elements that are in both \(A\) and \(B\);
  that is,
  \[
    A \cap B = \set{x : x \in A \text{ and } x \in B}.
  \]
  Two sets are called \textbf{disjoint} if their intersection equals the empty set.
  The union and intersection of more than two sets can be defined analogously.
  Specifically, if \(\seq{A}{1,,n}\) are sets, then the union and intersections of these sets are defined, respectively, by
  \[
    \bigcup_{i = 1}^n A_i = \set{x : x \in A_i \text{ for some } i \in \set{1, \dots, n}}
  \]
  and
  \[
    \bigcap_{i = 1}^n A_i = \set{x : x \in A_i \text{ for all } i \in \set{1, \dots, n}}.
  \]
  Similarly, if \(\Lambda\) is an index set and \(\set{A_{\alpha} : \alpha \in \Lambda}\) is a collection of sets, the union and intersection of these sets are defined, respectively, by
  \[
    \bigcup_{\alpha \in \Lambda} A_{\alpha} = \set{x : x \in A_{\alpha} \text{ for some } \alpha \in \Lambda}
  \]
  and
  \[
    \bigcap_{\alpha \in \Lambda} A_{\alpha} = \set{x : x \in A_{\alpha} \text{ for all } \alpha \in \Lambda}.
  \]
\end{defn}

\begin{defn}\label{a.0.6}
  By a relation on a set \(A\), we mean a rule for determining whether or not, for any elements \(x\) and \(y\) in \(A\), \(x\) stands in a given relationship to \(y\).
  More precisely, a \textbf{relation} on \(A\) is a set \(S\) of ordered pairs of elements of \(A\) such that \((x, y) \in S\) iff \(x\) stands in the given relationship to \(y\).
  If \(S\) is a relation on a set \(A\), we often write \(x \sim y\) in place of \((x, y) \in S\).
\end{defn}

\begin{defn}\label{a.0.7}
  A relation \(S\) on a set \(A\) is called an \textbf{equivalence relation} on \(A\) if the following three conditions hold:
  \begin{description}
    \item[Reflexivity:]
      For each \(x \in A\), \(x \sim x\).
    \item[Symmetry:]
      If \(x \sim y\), then \(y \sim x\).
    \item[Transitivity:]
      If \(x \sim y\) and \(y \sim z\), then \(x \sim z\).
  \end{description}
\end{defn}

\chapter{Functions}\label{ch:b}

\begin{defn}\label{b.0.1}
  If \(A\) and \(B\) are sets, then a \textbf{function} \(f\) from \(A\) to \(B\), written \(f : A \to B\), is a rule that associates to each element \(x\) in \(A\) an unique element denoted \(f(x)\) in \(B\).
  The element \(f(x)\) is called the \textbf{image} of \(x\) (under \(f\)), and \(x\) is called a preimage of \(f(x)\) (under \(f\)).
  If \(f : A \to B\), then \(A\) is called the \textbf{domain} of \(f\), \(B\) is called the \textbf{codomain} of \(f\), and the set \(\set{f(x) : x \in A}\) is called the \textbf{range} of \(f\).
  Note that the range of \(f\) is a subset of \(B\).
  If \(S \subseteq A\), we denote by \(f(S)\) the set \(\set{f(x) : x \in S}\) of all images of elements of \(S\).
  Likewise, if \(T \subseteq B\), we denote by \(f^{-1}(T)\) the set \(\set{x \in A : f(x) \in T}\) of all preimages of elements in \(T\).
  Finally, two functions \(f : A \to B\) and \(g : A \to B\) are \textbf{equal}, written \(f = g\), if \(f(x) = g(x)\) for all \(x \in A\).
\end{defn}

\begin{defn}\label{b.0.2}
  the preimage of an element in the range need not be unique.
  Functions such that each element of the range has an unique preimage are called \textbf{one-to-one};
  that is \(f : A \to B\) is one-to-one if \(f(x) = f(y)\) implies \(x = y\) or, equivalently, if \(x \neq y\) implies \(f(x) \neq f(y)\).
\end{defn}

\begin{defn}\label{b.0.3}
  If \(f : A \to B\) is a function with range \(B\), that is, if \(f(A) = B\), then \(f\) is called \textbf{onto}.
  So \(f\) is onto iff the range of \(f\) equals the codomain of \(f\).
\end{defn}

\begin{defn}\label{b.0.4}
  Let \(f : A \to B\) be a function and \(S \subseteq A\).
  Then a function \(f_S : S \to B\), called the \textbf{restriction} of \(f\) to \(S\), can be formed by defining \(f_S(x) = f(x)\) for each \(x \in S\).
\end{defn}

\begin{defn}\label{b.0.5}
  Let \(A\), \(B\), and \(C\) be sets and \(f : A \to B\) and \(g : B \to C\) be functions.
  By following \(f\) with \(g\), we obtain a function \(g \circ f : A \to C\) called the \textbf{composite} of \(g\) and \(f\).
  Thus \((g \circ f)(x) = g(f(x))\) for all \(x \in A\).
  Functional composition is associative, however;
  that is, if \(h : C \to D\) is another function, then \(h \circ (g \circ f) = (h \circ g) \circ f\).
\end{defn}

\begin{defn}\label{b.0.6}
  A function \(f : A \to B\) is said to be \textbf{invertible} if there exists a function \(g : B \to A\) such that \((f \circ g)(y) = y\) for all \(y \in B\) and \((g \circ f)(x) = x\) for all \(x \in A\).
  If such a function \(g\) exists, then it is unique and is called the \textbf{inverse} of \(f\).
  We denote the inverse of \(f\) (when it exists) by \(f^{-1}\).
  It can be shown that \(f\) is invertible iff \(f\) is both one-to-one and onto.
  The following facts about invertible functions are easily proved.
  \begin{itemize}
    \item If \(f : A \to B\) is invertible, then \(f^{-1}\) is invertible, and \((f^{-1})^{-1} = f\).
    \item If \(f : A \to B\) and \(g : B \to C\) are invertible, then \(g \circ f\) is invertible, and \((g \circ f)^{-1} = f^{-1} \circ g^{-1}\).
  \end{itemize}
\end{defn}

\chapter{Fields}\label{ch:c}

\begin{defn}\label{c.0.1}
  A field \(\F\) is a set on which two operations \(+\) and \(\cdot\) (called \textbf{addition} and \textbf{multiplication}, respectively) are defined so that, for each pair of elements \(x, y\) in \(F\), there are unique elements \(x + y\) and \(x \cdot y\) in \(\F\) for which the following conditions hold for all elements \(a, b, c\) in \(\F\).
  \begin{enumerate}[label=(F \arabic*), ref=F \arabic*]
    \item\label{f1} \(a + b = b + a\) and \(a \cdot b = b \cdot a\)
    (commutativity of addition and multiplication).
    \item\label{f2} \((a + b) + c = a + (b + c)\) and \((a \cdot b) \cdot c = a \cdot (b \cdot c)\)
    (associativity of addition and multiplication).
    \item\label{f3} There exist distinct elements \(0\) and \(1\) in \(\F\) such that
    \[
      0 + a = a \quad \text{and} \quad 1 \cdot a = a
    \]
    (existence of identity elements for addition and multiplication).
    \item\label{f4} For each element \(a\) in \(F\) and each nonzero element \(b\) in \(\F\), there exist elements \(c\) and \(d\) in \(\F\) such that
    \[
      a + c = 0 \quad \text{and} \quad b \cdot d = 1
    \]
    (existence of inverses for addition and multiplication).
    \item\label{f5} \(a \cdot (b + c) = a \cdot b + a \cdot c\)
    (distributivity of multiplication over addition).
  \end{enumerate}
  The elements \(x + y\) and \(x \cdot y\) are called the \textbf{sum} and \textbf{product}, respectively, of \(x\) and \(y\).
  The elements \(0\) (read ``zero'') and \(1\) (read ``one'') mentioned in \ref{f3} are called \textbf{identity elements} for addition and multiplication, respectively, and the elements \(c\) and \(d\) referred to in \ref{f4} are called an \textbf{additive inverse} for \(a\) and a \textbf{multiplicative inverse} for \(b\), respectively.
\end{defn}

\begin{eg}\label{c.0.2}
  The set of real numbers \(\R\) with the usual definitions of addition and multiplication is a field.
\end{eg}

\begin{eg}\label{c.0.3}
  The set of rational numbers \(\Q\) with the usual definitions of addition and multiplication is a field.
\end{eg}

\begin{eg}\label{c.0.4}
  The field \(\Z_2\) consists of two elements \(0\) and \(1\) with the operations of addition and multiplication defined by the equations
  \begin{align*}
    0 + 0     & = 0             \\
    0 + 1     & = 1 + 0 = 1     \\
    1 + 1     & = 0             \\
    0 \cdot 0 & = 0             \\
    0 \cdot 1 & = 1 \cdot 0 = 0 \\
    1 \cdot 1 & = 1.
  \end{align*}
\end{eg}

\begin{thm}[Cancellation Laws]\label{c.1}
  For arbitrary elements \(a\), \(b\), and \(c\) in a field, the following statements are true.
  \begin{enumerate}
    \item If \(a + b = c + b\), then \(a = c\).
    \item If \(a \cdot b = c \cdot b\) and \(b \neq 0\), then \(a = c\).
  \end{enumerate}
\end{thm}

\begin{proof}[\pf{c.1}(a)]
  We have
  \begin{align*}
             & \exists d \in \F : b + d = 0 &  & \text{(by \ref{f4})} \\
    \implies & (a + b) + d = (c + b) + d    &  & \by{c.0.1}           \\
    \implies & a + (b + d) = c + (b + d)    &  & \text{(by \ref{f2})} \\
    \implies & a + 0 = c + 0                &  & \text{(by \ref{f4})} \\
    \implies & a = c.                       &  & \text{(by \ref{f3})}
  \end{align*}
\end{proof}

\begin{proof}[\pf{c.1}(b)]
  If \(b \neq 0\), then \ref{f4} guarantees the existence of an element \(d\) in the field such that \(b \cdot d = 1\).
  Multiply both sides of the equality \(a \cdot b = c \cdot b\) by \(d\) to obtain \((a \cdot b) \cdot d = (c \cdot b) \cdot d\).
  Consider the left side of this equality:
  By \ref{f2} and \ref{f3}, we have
  \[
    (a \cdot b) \cdot d = a \cdot (b \cdot d) = a \cdot 1 = a.
  \]
  Similarly, the right side of the equality reduces to \(c\).
  Thus \(a = c\).
\end{proof}

\begin{cor}\label{c.0.5}
  The elements \(0\) and \(1\) mentioned in \ref{f3}, and the elements \(c\) and \(d\) mentioned in \ref{f4}, are unique.
\end{cor}

\begin{proof}[\pf{c.0.5}]
  Suppose that \(0' \in \F\) satisfies \(0' + a = a\) for each \(a \in \F\).
  Since \(0 + a = a\) for each \(a \in \F\) , we have \(0' + a = 0 + a\) for each \(a \in \F\).
  Thus \(0' = 0\) by \cref{c.1}.
  The proofs of the remaining parts are similar.
\end{proof}

\begin{defn}\label{c.0.6}
  The additive inverse and the multiplicative inverse of \(b\) are denoted by \(-b\) and \(b^{-1}\), respectively.
  Note that \(-(-b) = b\) and \((b^{-1})^{-1} = b\).
\end{defn}

\begin{defn}\label{c.0.7}
  \textbf{Subtraction} and \textbf{division} can be defined in terms of addition and multiplication by using the additive and multiplicative inverses.
  Specifically, subtraction of \(b\) is defined to be addition of \(-b\) and division by \(b \neq 0\) is defined to be multiplication by \(b^{-1}\);
  that is,
  \[
    a - b = a + (-b) \quad \text{and} \quad \dfrac{a}{b} = a \cdot b^{-1}.
  \]
  In particular, the symbol \(\dfrac{1}{b}\) denotes \(b^{-1}\).
  Division by zero is undefined, but, with this exception, the sum, product, difference, and quotient of any two elements of a field are defined.
\end{defn}

\begin{thm}\label{c.2}
  Let \(a\) and \(b\) be arbitrary elements of a field.
  Then each of the following statements are true.
  \begin{enumerate}
    \item \(a \cdot 0 = 0\).
    \item \((-a) \cdot b = a \cdot (-b) = -(a \cdot b)\).
    \item \((-a) \cdot (-b) = a \cdot b\).
  \end{enumerate}
\end{thm}

\begin{proof}[\pf{c.2}(a)]
  Since \(0 + 0 = 0\), \ref{f5} shows that
  \[
    0 + a \cdot 0 = a \cdot 0 = a \cdot (0 + 0) = a \cdot 0 + a \cdot 0.
  \]
  Thus \(0 = a \cdot 0\) by \cref{c.1}.
\end{proof}

\begin{proof}[\pf{c.2}(b)]
  By definition, \(-(a \cdot b)\) is the unique element of \(\F\) with the property \(a \cdot b + [-(a \cdot b)] = 0\).
  So in order to prove that \((-a) \cdot b = -(a \cdot b)\), it suffices to show that \(a \cdot b + (-a) \cdot b = 0\).
  But \(-a\) is the element of \(\F\) such that \(a + (-a) = 0\);
  so
  \[
    a \cdot b + (-a) \cdot b = [a + (-a)] \cdot b = 0 \cdot b = b  \cdot 0 = 0
  \]
  by \ref{f5} and \cref{c.2}(a).
  Thus \((-a) \cdot b = -(a \cdot b)\).
  The proof that \(a \cdot (-b) = -(a \cdot b)\) is similar.
\end{proof}

\begin{proof}[\pf{c.2}(c)]
  By applying \cref{c.2}(b) twice, we find that
  \[
    (-a) \cdot (-b) = -[a \cdot (-b)] = -[-(a \cdot b)] = a \cdot b.
  \]
\end{proof}

\begin{cor}\label{c.0.8}
  The additive identity of a field has no multiplicative inverse.
\end{cor}

\begin{proof}[\pf{c.0.8}]
  Suppose for sake of contradiction that \(0^{-1}\) exists.
  But then we have
  \begin{align*}
    \forall a \in \F \setminus \set{1}, a & = a \cdot 1                &  & \text{(by \ref{f3})} \\
                                          & = a \cdot (0 \cdot 0^{-1}) &  & \text{(by \ref{f4})} \\
                                          & = (a \cdot 0) \cdot 0^{-1} &  & \text{(by \ref{f2})} \\
                                          & = 0 \cdot 0^{-1}           &  & \by{c.2}[a]          \\
                                          & = 1,                       &  & \text{(by \ref{f4})}
  \end{align*}
  a contradiction.
  Thus \(0^{-1}\) does not exist.
\end{proof}

\begin{defn}\label{c.0.9}
  In an arbitrary field \(\F\), it may happen that a sum \(1 + 1 + \cdots + 1\) (\(p\) summands) equals \(0\) for some positive integer \(p\).
  For example, in the field \(\Z_2\) (defined in \cref{c.0.4}), \(1 + 1 = 0\).
  In this case, the smallest positive integer \(p\) for which a sum of \(p\) \(1\)'s equals \(0\) is called the \textbf{characteristic} of \(\F\);
  if no such positive integer exists, then \(\F\) is said to have \textbf{characteristic zero}.
  Thus \(\Z_2\) has characteristic two, and \(\R\) has characteristic zero.
  Observe that if \(\F\) is a field of characteristic \(p \neq 0\), then \(x + x + \cdots + x\) (\(p\) summands) equals \(0\) for all \(x \in \F\).
  In a field having nonzero characteristic (especially characteristic two), many unnatural problems arise.
  For this reason, some of the results about vector spaces stated in this book require that the field over which the vector space is defined be of characteristic zero (or, at least, of some characteristic other than two).
\end{defn}

\chapter{Complex Numbers}\label{ch:d}

\chapter{Polynomial}\label{ch:e}

\begin{defn}\label{e.0.1}
	A polynomial \(f\) \textbf{divides} a polynomial \(g\) if there exists a polynomial \(q\) such that \(g(x) = f(x) q(x)\).
\end{defn}

\begin{thm}[The Division Algorithm for Polynomials]\label{e.1}
	Let \(f \in \ps[n]{\F}\) and let \(g \in \ps[m]{\F}\).
	Then there exist unique polynomials \(q\) and \(r\) such that
	\begin{equation}\label{eq:e.0.1}
		\forall x \in \F, f(x) = q(x) g(x) + r(x),
	\end{equation}
	where \(r \in \ps[m - 1]{\F}\).
\end{thm}

\begin{proof}[\pf{e.1}]
	We begin by establishing the existence of \(q\) and \(r\) that satisfy \cref{eq:e.0.1}.
	\begin{description}
		\item[Case 1.]
			If \(n < m\), take \(q = \zv\) and \(r = f\) to satisfy \cref{eq:e.0.1}.
		\item[Case 2.]
			When \(0 \leq m \leq n\), we apply mathematical induction on \(n\).
			First suppose that \(n = 0\).
			Then \(m = 0\), and it follows that \(f\) and \(g\) are nonzero constants.
			Hence we may take \(q = f / g\) and \(r = \zv\) to satisfy \cref{eq:e.0.1}.

			Now suppose that the result is valid for all polynomials with degree less than \(n\) for some fixed \(n > 0\), and assume that \(f\) has degree \(n\).
			Suppose that
			\[
				\forall x \in \F, f(x) = a_n x^n + a_{n - 1} x^{n - 1} + \cdots + a_1 x + a_0
			\]
			and
			\[
				\forall x \in \F, g(x) = b_m x^m + b_{m - 1} x^{m - 1} + \cdots + b_1 x + b_0
			\]
			and let \(h\) be the polynomial defined by
			\begin{equation}\label{eq:e.0.2}
				\forall x \in \F, h(x) = f(x) - a_n b_m^{-1} x^{n - m} g(x).
			\end{equation}
			Then \(h\) is a polynomial of degree less than \(n\), and therefore we may apply the induction hypothesis or Case 1 (whichever is relevant) to obtain polynomials \(q_1\) and \(r\) such that \(r\) has degree less than \(m\) and
			\begin{equation}\label{eq:e.0.3}
				\forall x \in \F, h(x) = q_1(x) g(x) + r(x).
			\end{equation}
			Combining \cref{eq:e.0.2,eq:e.0.3} and solving for \(f\) gives us \(f = q g + r\) with
			\[
				\forall x \in \F, q(x) = a_n b_m^{-1} x^{n - m} + q_1(x),
			\]
			which establishes Case I and Case II for any \(n \geq 0\) by mathematical induction.
			This establishes the existence of \(q\) and \(r\).

			We now show the uniqueness of \(q\) and \(r\).
			Suppose that \(q_1, q_2, r_1\), and \(r_2\) exist such that \(r_1\) and \(r_2\) each has degree less than \(m\) and
			\[
				\forall x \in \F, f(x) = q_1(x) g(x) + r_1(x) = q_2(x) g(x) + r_2(x).
			\]
			Then
			\begin{equation}\label{eq:e.0.4}
				\forall x \in \F, (q_1(x) - q_2(x)) g(x) = r_2(x) - r_1(x).
			\end{equation}
			The right side of \cref{eq:e.0.4} is a polynomial of degree less than \(m\).
			Since \(g\) has degree \(m\), it must follow that \(q_1 - q_2\) is the zero polynomial.
			Hence \(q_1 = q_2\);
			thus \(r_1 = r_2\) by \cref{eq:e.0.4}.
	\end{description}
\end{proof}

\begin{defn}\label{e.0.2}
	In the context of \cref{e.1}, we call \(q\) and \(r\) the \textbf{quotient} and \textbf{remainder}, respectively, for the division of \(f\) by \(g\).
\end{defn}

\begin{cor}\label{e.0.3}
	Let \(f\) be a polynomial of positive degree, and let \(a \in \F\).
	Then \(f(a) = 0\) iff \(x - a\) divides \(f\).
\end{cor}

\begin{proof}[\pf{e.0.3}]
	Suppose that \(x - a\) divides \(f\).
	Then there exists a polynomial \(q\) such that \(f(x) = (x - a) q(x)\) for all \(x \in \F\).
	Thus \(f(a) = (a - a) q(a) = 0 \cdot q(a) = 0\).

	Conversely, suppose that \(f(a) = 0\).
	By the division algorithm, there exist polynomials \(q\) and \(r\) such that \(r\) has degree less than one and
	\[
		\forall x \in \F, f(x) = q(x) (x - a) + r(x).
	\]
	Substituting \(a\) for \(x\) in the equation above, we obtain \(r(a) = 0\).
	Since \(r\) has degree less than \(1\), it must be the constant polynomial \(r = \zv\).
	Thus \(f(x) = q(x) (x - a)\).
\end{proof}

\begin{defn}\label{e.0.4}
	For any polynomial \(f\) with coefficients from a field \(\F\), an element \(a \in \F\) is called a \textbf{zero} of \(f\) if \(f(a) = 0\).
	With this terminology, \cref{e.0.3} states that \(a\) is a zero of \(f\) iff \(x - a\) divides \(f\).
\end{defn}

\begin{cor}\label{e.0.5}
	Any polynomial of degree \(n \geq 1\) has at most \(n\) distinct zeros.
\end{cor}

\begin{proof}[\pf{e.0.5}]
	The proof is by mathematical induction on \(n\).
	The result is obvious if \(n = 1\).
	Now suppose that the result is true for some positive integer \(n\), and let \(f\) be a polynomial of degree \(n + 1\).
	If \(f\) has no zeros, then there is nothing to prove.
	Otherwise, if \(a\) is a zero of \(f\), then by \cref{e.0.5} we may write \(f(x) = (x - a) q(x)\) for some polynomial \(q\).
	Note that \(q\) must be of degree \(n\);
	therefore, by the induction hypothesis, \(q\) can have at most \(n\) distinct zeros.
	Since any zero of \(q\) distinct from \(a\) is also a zero of \(f\), it follows that \(f\) can have at most \(n + 1\) distinct zeros.
\end{proof}

\begin{defn}\label{e.0.6}
	Two nonzero polynomials are called \textbf{relatively prime} if no polynomial of positive degree divides each of them.
\end{defn}

\begin{thm}\label{e.2}
	If \(f_1\) and \(f_2\) are relatively prime polynomials over \(\F\), there exist polynomials \(q_1\) and \(q_2\) such that
	\[
		\forall x \in \F, q_1(x) f_1(x) + q_2(x) f_2(x) = 1,
	\]
	where \(1 \in \F\).
\end{thm}

\begin{proof}[\pf{e.2}]
	Without loss of generality, assume that the degree of \(f_1\) is greater than or equal to the degree of \(f_2\).
	The proof is by mathematical induction on the degree of \(f_2\).
	If \(f_2\) has degree \(0\), then \(f_2\) is a nonzero constant \(c\).
	In this case, we can take \(q_1 = \zv\) and \(q_2 = 1 / c\).

	Now suppose that the theorem holds whenever the polynomial of lesser degree has degree less than \(n\) for some positive integer \(n\), and suppose that \(f_2\) has degree \(n\).
	By the division algorithm, there exist polynomials \(q\) and \(r\) such that \(r\) has degree less than \(n\) and
	\begin{equation}\label{eq:e.0.5}
		\forall x \in \F, f_1(x) = q(x) f_2(x) + r(x).
	\end{equation}
	Since \(f_1\) and \(f_2\) are relatively prime, \(r\) is not the zero polynomial.
	We claim that \(f_2\) and \(r\) are relatively prime.
	Suppose otherwise;
	then there exists a polynomial \(g\) of positive degree that divides both \(f_2\) and \(r\).
	Hence, by (5), \(g\) also divides \(f_1\), contradicting the fact that \(f_1\) and \(f_2\) are relatively prime.
	Since \(r\) has degree less than \(n\), we may apply the induction hypothesis to \(f_2\) and \(r\).
	Thus there exist polynomials \(g_1\) and \(g_2\) such that
	\begin{equation}\label{eq:e.0.6}
		\forall x \in \F, g_1(x) f_2(x) + g_2(x) r(x) = 1.
	\end{equation}
	Combining \cref{eq:e.0.5,eq:e.0.6}, we have
	\begin{align*}
		1 & = g_1(x) f_2(x) + g_2(x) (f_1(x) - q(x) f_2(x))  \\
		  & = g_2(x) f_1(x) + (g_1(x) - g_2(x) q(x)) f_2(x).
	\end{align*}
	Thus, setting \(q_1 = g_2\) and \(q_2 = g_1 - g_2 q\), we obtain the desired result.
\end{proof}

\begin{defn}\label{e.0.7}
	Let
	\[
		f(x) = a_0 + a_1 x + \cdots + a_n x^n
	\]
	be a polynomial with coefficients from a field \(\F\).
	If \(\T\) is a linear operator on a vector space \(\V\) over \(\F\), we define
	\[
		f(\T) = a_0 \IT[\V] + a_1 \T + \cdots + a_n \T^n.
	\]
	Similarly, if \(A \in \ms[n][n][\F]\), we define
	\[
		f(A) = a_0 I_n + a_1 A + \cdots + a_n A^n.
	\]
\end{defn}

\begin{thm}\label{e.3}
	Let \(f\) be a polynomial with coefficients from a field \(\F\), and let \(\T\) be a linear operator on a vector space \(\V\) over \(\F\).
	Then the following statements are true.
	\begin{enumerate}
		\item \(f(\T)\) is a linear operator on \(\V\).
		\item If \(\beta\) is a finite ordered basis for \(\V\) over \(\F\) and \(A = [\T]_{\beta}\), then \([f(\T)]_{\beta} = f(A)\).
	\end{enumerate}
\end{thm}

\begin{proof}[\pf{e.3}]
	Let \(f(x) = a_0 + a_1 x + \cdots + a_n x^n\).
	By \cref{e.0.7} we have
	\[
		f(\T) = a_0 \IT[\V] + a_1 \T + \cdots + a_n \T^n \quad \text{and} \quad f(A) = a_0 I_n + a_1 A + \cdots + a_n A^n.
	\]
	By \cref{2.7,2.9} we see that \(f(\T) \in \ls(\V)\).
	Suppose that \(A = [\T]_{\beta} \in \ms[n][n][\F]\).
	Then we have
	\begin{align*}
		[f(\T)]_{\beta} & = [a_0 \IT[\V] + a_1 \T + \cdots + a_n \T^n]_{\beta}                       &  & \by{e.0.7}   \\
		                & = a_0 [\IT[\V]]_{\beta} + a_1 [\T]_{\beta} + \cdots + a_n [\T^n]_{\beta}   &  & \by{2.8}     \\
		                & = a_0 [\IT[\V]]_{\beta} + a_1 [\T]_{\beta} + \cdots + a_n ([\T]_{\beta})^n &  & \by{2.11}    \\
		                & = a_0 I_n + a_1 A + \cdots + a_n A^n                                       &  & \by{2.12}[d] \\
		                & = f(A).                                                                    &  & \by{e.0.7}
	\end{align*}
\end{proof}

\begin{thm}\label{e.4}
	Let \(\T\) be a linear operator on a vector space \(\V\) over field \(\F\), and let \(A \in \ms[n][n][\F]\).
	Then, for any polynomials \(f_1\) and \(f_2\) with coefficients from \(\F\),
	\begin{enumerate}
		\item \(f_1(\T) f_2(\T) = f_2(\T) f_1(\T)\);
		\item \(f_1(A) f_2(A) = f_2(A) f_1(A)\).
	\end{enumerate}
\end{thm}

\begin{proof}[\pf{e.4}(a)]
	Let \(f_1(x) = a_0 + a_1 x + \cdots + a_n x^n\) and \(f_2(x) = b_0 + b_1 x + \cdots + b_m x^m\).
	Since
	\begin{align*}
		\forall i \in \set{0, \dots, n}, & (a_i \T^i) f_2(\T)                                                                  \\
		                                 & = (a_i \T^i) (b_0 \IT[\V] + b_1 \T + \cdots + b_m \T^m)       &  & \by{e.0.7}       \\
		                                 & = b_0 a_i \T^i + b_1 a_i \T^{1 + i} + \cdots + b_m \T^{m + i} &  & \by{2.10}[a,c,d] \\
		                                 & = (b_0 \IT[\V] + b_1 \T + \cdots + b_m \T^m) (a_i \T^i)       &  & \by{2.10}[b,c]   \\
		                                 & = f_2(\T) (a_i \T^i),                                         &  & \by{e.0.7}
	\end{align*}
	we have
	\begin{align*}
		f_1(\T) f_2(\T) & = (a_0 \IT[\V] + a_1 \T + \cdots + a_n \T^n) f_2(\T)                     &  & \by{e.0.7}                    \\
		                & = (a_0 \IT[\V]) f_2(\T) + (a_1 \T) f_2(\T) + \cdots + (a_n \T^n) f_2(\T) &  & \by{2.10}[a,d]                \\
		                & = f_2(\T) (a_0 \IT[\V]) + f_2(\T) (a_1 \T) + \cdots + f_2(\T) (a_n \T^n) &  & \text{(from the proof above)} \\
		                & = f_2(\T) (a_0 \IT[\V] + a_1 \T + \cdots + a_n \T^n)                     &  & \by{2.10}[a,d]                \\
		                & = f_2(\T) f_1(\T).                                                       &  & \by{e.0.7}
	\end{align*}
\end{proof}

\begin{proof}[\pf{e.4}(b)]
	Let \(\beta\) be the standard ordered basis for \(\vs{F}^n\) over \(\F\).
	Then we have
	\begin{align*}
		f_1(A) f_2(A) & = f_1([\L_A]_{\beta}) f_2([\L_A]_{\beta}) &  & \by{2.15}[a] \\
		              & = [f_1(\L_A)]_{\beta} [f_2(\L_A)]_{\beta} &  & \by{e.3}[b]  \\
		              & = [f_1(\L_A) f_2(\L_A)]_{\beta}           &  & \by{2.3.3}   \\
		              & = [f_2(\L_A) f_1(\L_A)]_{\beta}           &  & \by{e.4}[a]  \\
		              & = [f_2(\L_A)]_{\beta} [f_1(\L_A)]_{\beta} &  & \by{2.3.3}   \\
		              & = f_2([\L_A]_{\beta}) f_1([\L_A]_{\beta}) &  & \by{2.15}[a] \\
		              & = f_2(A) f_1(A).                          &  & \by{e.3}[b]
	\end{align*}
\end{proof}

\begin{thm}\label{e.5}
	Let \(\T\) be a linear operator on a vector space \(\V\) over field \(\F\), and let \(A \in \ms[n][n][\F]\).
	If \(f_1\) and \(f_2\) are relatively prime polynomials with entries from \(\F\), then there exist polynomials \(q_1\) and \(q_2\) with entries from \(\F\) such that
	\begin{enumerate}
		\item \(q_1(\T) f_1(\T) + q_2(\T) f_2(\T) = \IT[\V]\);
		\item \(q_1(A) f_1(A) + q_2(A) f_2(A) = I_n\).
	\end{enumerate}
\end{thm}

\begin{proof}[\pf{e.5}]
	By \cref{e.2} there exist polynomials \(q_1\) and \(q_2\) such that \(q_1(x) f_1(x) + q_2(x) f_2(x) = 1\).
	Thus by \cref{e.0.7} we have \(q_1(\T) f_1(\T) + q_2(\T) f_2(\T) = \IT[\V]\) and \(q_1(A) f_1(A) + q_2(A) f_2(A) = I_n\).
\end{proof}

\begin{defn}\label{e.0.8}
	A polynomial \(f\) with coefficients from a field \(\F\) is called \textbf{monic} if its leading coefficient is \(1\).
	If \(f\) has positive degree and cannot be expressed as a product of polynomials with coefficients from \(\F\) each having positive degree, then \(f\) is called \textbf{irreducible}.

	Observe that whether a polynomial is irreducible depends on the field \(\F\) from which its coefficients come.
	Clearly any polynomial of degree \(1\) is irreducible.
	Moreover, for polynomials with coefficients from an algebraically closed field, the polynomials of degree \(1\) are the only irreducible polynomials.
\end{defn}

\begin{thm}\label{e.6}
	Let \(\phi\) and \(f\) be polynomials.
	If \(\phi\) is irreducible and \(\phi\) does not divide \(f\), then \(\phi\) and \(f\) are relatively prime.
\end{thm}

\begin{proof}[\pf{e.6}]
	Suppose for sake of contradiction that \(\phi\) and \(f\) are not relatively prime.
	Then there exists a polynomial \(g\) such that \(g\) divides \(\phi\) and \(f\).
	Since \(\phi\) is irreducible, by \cref{e.0.8} we must have \(\phi = cg\) for some \(c \in \F\).
	Since \(g\) divides \(f\), we know that \(\phi\) divides \(f\).
	But this contradict to the fact that \(\phi\) does not divide \(f\).
	Thus \(\phi\) and \(f\) are relatively prime.
\end{proof}

\begin{thm}\label{e.7}
	Any two distinct irreducible monic polynomials are relatively prime.
\end{thm}

\begin{proof}[\pf{e.7}]
	Let \(\phi\) and \(f\) be distinct irreducible monic polynomials.
	Since \(\phi\) and \(f\) are irreducible and monic, by \cref{e.0.8} we know that \(\phi\) does not divide \(f\).
	Thus by \cref{e.6} \(\phi\) and \(f\) are relatively prime.
\end{proof}

\begin{thm}\label{e.8}
	Let \(f, g\), and \(\phi\) be polynomials.
	If \(\phi\) is irreducible and divides the product \(fg\), then \(\phi\) divides \(f\) or \(\phi\) divides \(g\).
\end{thm}

\begin{proof}[\pf{e.8}]
	Suppose that \(\phi\) does not divide \(f\).
	Then \(\phi\) and \(f\) are relatively prime by \cref{e.6}, and so there exist polynomials \(q_1\) and \(q_2\) such that
	\[
		1 = q_1(x) \phi(x) + q_2(x) f(x).
	\]
	Multiplying both sides of this equation by \(g\) yields
	\begin{equation}\label{eq:e.0.7}
		g(x) = q_1(x) \phi(x) g(x) + q_2(x) f(x) g(x).
	\end{equation}
	Since \(\phi\) divides \(fg\), there is a polynomial \(h\) such that \(fg = \phi h\).
	Thus \cref{eq:e.0.7} becomes
	\[
		g(x) = q_1(x) \phi(x) g(x) + q_2(x) \phi(x) h(x) = \phi(x) (q_1(x) g(x) + q_2(x) h(x)).
	\]
	So \(\phi\) divides \(g\).
\end{proof}

\begin{cor}\label{e.0.9}
	Let \(\phi, \seq{\phi}{1,,n}\) be irreducible monic polynomials.
	If \(\phi\) divides the product \(\seq[]{\phi}{1,,n}\), then \(\phi = \phi_i\) for some \(i \in \set{1, 2, . . . , n}\).
\end{cor}

\begin{proof}[\pf{e.0.9}]
	We prove the corollary by mathematical induction on \(n\).
	For \(n = 1\), the result is an immediate consequence of \cref{e.7}.
	Suppose then that for some \(n > 1\), the corollary is true for any \(n - 1\) irreducible monic polynomials, and let \(\seq{\phi}{1,,n}\) be \(n\) irreducible polynomials.
	If \(\phi\) divides
	\[
		\seq[]{\phi}{1,,n} = (\seq[]{\phi}{1,,n-1}) \phi_n,
	\]
	then \(\phi\) divides the product \(\seq[]{\phi}{1,,n-1}\) or \(\phi\) divides \(\phi_n\) by \cref{e.8}.
	In the first case, \(\phi = \phi_i\) for some \(i \in \set{1, \dots, n - 1}\) by the induction hypothesis;
	in the second case, \(\phi = \phi_n\) by \cref{e.7}.
\end{proof}

\begin{thm}[Unique Factorization Theorem for Polynomials]\label{e.9}
	For any polynomial \(f\) of positive degree, there exist a unique constant \(c\);
	unique distinct irreducible monic polynomials \(\seq{\phi}{1,,k}\);
	and unique positive integers \(\seq{n}{1,,k}\) such that
	\[
		f(x) = c (\phi_1(x))^{n_1} \cdots (\phi_k(x))^{n_k}.
	\]
\end{thm}

\begin{proof}[\pf{e.9}]
	We begin by showing the existence of such a factorization using mathematical induction on the degree of \(f\).
	If \(f\) is of degree \(1\), then \(f(x) = ax + b\) for some constants \(a\) and \(b\) with \(a \neq 0\).
	Setting \(\phi(x) = x + b / a\), we have \(f = a \phi\).
	Since \(\phi\) is an irreducible monic polynomial, the result is proved in this case.
	Now suppose that the conclusion is true for any polynomial with positive degree less than some integer \(n > 1\), and let \(f\) be a polynomial of degree \(n\).
	Then
	\[
		f(x) = a_n x^n + \cdots + a_1 x + a_0
	\]
	for some constants \(a_i\) with \(a_n \neq 0\).
	If \(f\) is irreducible, then
	\[
		f(x) = a_n \pa{x^n + \dfrac{a_{n - 1}}{a_n} x^{n - 1} + \cdots + \dfrac{a_1}{a_n} x + \dfrac{a_0}{a_n}}
	\]
	is a representation of \(f\) as a product of \(a_n\) and an irreducible monic polynomial.
	If \(f\) is not irreducible, then \(f = gh\) for some polynomials \(g\) and \(h\), each of positive degree less than \(n\).
	The induction hypothesis guarantees that both \(g\) and \(h\) factor as products of a constant and powers of distinct irreducible monic polynomials.
	Consequently \(f = gh\) also factors in this way.
	Thus, in either case, \(f\) can be factored as a product of a constant and powers of distinct irreducible monic polynomials.

	It remains to establish the uniqueness of such a factorization.
	Suppose that
	\begin{equation}\label{eq:e.0.8}
		\begin{aligned}
			f(x) & = c (\phi_1(x))^{n_1} \cdots (\phi_k(x))^{n_k}  \\
			     & = d (\psi_1(x))^{m_1} \cdots (\psi_r(x))^{m_r},
		\end{aligned}
	\end{equation}
	where \(c\) and \(d\) are constants, \(\phi_i\) and \(\psi_j\) are irreducible monic polynomials, and \(n_i\) and \(m_j\) are positive integers for \(i \in \set{1, \dots, k}\) and \(j \in \set{1, \dots, r}\).
	Clearly both \(c\) and \(d\) must be the leading coefficient of \(f\);
	hence \(c = d\).
	Dividing by \(c\), we find that \cref{eq:e.0.8} becomes
	\begin{equation}\label{eq:e.0.9}
		(\phi_1)^{n_1} \cdots (\phi_k)^{n_k} = (\psi_1)^{m_1} \cdots (\psi_r)^{m_r}.
	\end{equation}
	So \(\phi_i\) divides the right side of \cref{eq:e.0.9} for \(i \in \set{1, \dots, k}\).
	Consequently, by \cref{e.0.9}, each \(\phi_i\) equals some \(\psi_j\), and similarly, each \(\psi_j\) equals some \(\phi_i\).
	We conclude that \(r = k\) and that, by renumbering if necessary, \(\phi_i = \psi_i\) for \(i \in \set{1, \dots, k}\).
	Suppose that \(n_i \neq m_i\) for some \(i \in \set{1, \dots, k}\).
	Without loss of generality, we may suppose that \(i = 1\) and \(n_1 > m_1\).
	Then by canceling \((\phi_1)^{m_1}\) from both sides of \cref{eq:e.0.9}, we obtain
	\begin{equation}\label{eq:e.0.10}
		(\phi_1)^{n_1 - m_1} (\phi_2)^{n_2} \cdots (\phi_k)^{n_k} = (\phi_2)^{m_2} \cdots (\phi_k)^{m_k}.
	\end{equation}
	Since \(n_1 - m_1 > 0\), \(\phi_1\) divides the left side of \cref{eq:e.0.10} and hence divides the right side also.
	So \(\phi_1 = \phi_i\) for some \(i \in \set{2, \dots, k}\) by \cref{e.0.9}.
	But this contradicts that \(\seq{\phi}{1,,k}\) are distinct.
	Hence the factorizations of \(f\) in \cref{eq:e.0.8} are the same.
\end{proof}

\begin{note}
	It is often useful to regard a polynomial \(f(x) = a_n x^n + \cdots + a_1 x + a_0\) with coefficients from a field \(\F\) as a function \(f : \F \to \F\).
	In this case, the value of \(f\) at \(c \in \F\) is \(f(c) = a_n c^n + \cdots + a_1 c + a_0\).
	Unfortunately, for arbitrary fields there is not a one-to-one correspondence between polynomials and polynomial functions.
	For example, if \(f(x) = x^2\) and \(g(x) = x\) are two polynomials over the field \(\Z^2\) (defined in \cref{c.0.4}), then \(f\) and \(g\) have different degrees and hence are not equal as polynomials.
	But \(f(a) = g(a)\) for all \(a \in \Z^2\), so that \(f\) and \(g\) are equal polynomial functions.
	Our final result shows that this anomaly cannot occur over an infinite field.
\end{note}

\begin{thm}\label{e.10}
	Let \(f\) and \(g\) be polynomials with coefficients from an infinite field \(\F\).
	If \(f(a) = g(a)\) for all \(a \in \F\), then \(f\) and \(g\) are equal.
\end{thm}

\begin{proof}[\pf{e.10}]
	Suppose that \(f(a) = g(a)\) for all \(a \in \F\).
	Define \(h = f - g\), and suppose that \(h\) is of degree \(n \geq 1\).
	It follows from \cref{e.0.5} that \(h\) can have at most \(n\) zeroes.
	But
	\[
		h(a) = f(a) - g(a) = 0
	\]
	for every \(a \in \F\), contradicting the assumption that \(h\) has positive degree.
	Thus \(h\) is a constant polynomial, and since \(h(a) = 0\) for each \(a \in \F\), it follows that \(h\) is the zero polynomial.
	Hence \(f = g\).
\end{proof}


%------------------------------------------------------------------------------
% Back matters.
%------------------------------------------------------------------------------

\backmatter

\end{document}

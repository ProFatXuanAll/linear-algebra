\section{Systems of Linear Equations --- Theoretical Aspects}\label{sec:3.3}

\begin{defn}\label{3.3.1}
  The system of equations
  \[
    (S) \quad \begin{matrix}
      a_{1 1} x_1 + a_{1 2} x_2 + \cdots + a_{1 n} x_n = b_1 \\
      a_{2 1} x_1 + a_{2 2} x_2 + \cdots + a_{2 n} x_n = b_2 \\
      \vdots                                                 \\
      a_{m 1} x_1 + a_{m 2} x_2 + \cdots + a_{m n} x_n = b_m
    \end{matrix}
  \]
  where \(a_{i j}\) and \(b_i\) (\(1 \leq i \leq m\) and \(1 \leq j \leq n\)) are scalars in a field \(\F\) and \(\seq{x}{1,,n}\) are \(n\) variables taking values in \(\F\), is called a \textbf{system of \(m\) linear equations in \(n\) unknowns over the field \(\F\)}.

  The \(m \times n\) matrix
  \[
    A = \begin{pmatrix}
      a_{1 1} & a_{1 2} & \cdots & a_{1 n} \\
      a_{2 1} & a_{2 2} & \cdots & a_{2 n} \\
      \vdots  & \vdots  &        & \vdots  \\
      a_{m 1} & a_{m 2} & \cdots & a_{m n}
    \end{pmatrix}
  \]
  is called the \textbf{coefficient matrix} of the system \((S)\).

  If we let
  \[
    x = \begin{pmatrix}
      x_1    \\
      x_2    \\
      \vdots \\
      x_n
    \end{pmatrix} \quad \text{and} \quad b = \begin{pmatrix}
      b_1    \\
      b_2    \\
      \vdots \\
      b_m
    \end{pmatrix},
  \]
  then the system \((S)\) may be rewritten as a single matrix equation
  \[
    Ax = b.
  \]
  To exploit the results that we have developed, we often consider a system of linear equations as a single matrix equation.

  A \textbf{solution} to the system \((S)\) is an \(n\)-tuple
  \[
    s = \begin{pmatrix}
      s_1    \\
      s_2    \\
      \vdots \\
      s_n
    \end{pmatrix} \in \vs{F}^n
  \]
  such that \(As = b\).
  The set of all solutions to the system \((S)\) is called the \textbf{solution set} of the system.
  System \((S)\) is called \textbf{consistent} if its solution set is nonempty;
  otherwise it is called \textbf{inconsistent}.
\end{defn}

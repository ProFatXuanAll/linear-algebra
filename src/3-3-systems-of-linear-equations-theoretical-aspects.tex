\section{Systems of Linear Equations --- Theoretical Aspects}\label{sec:3.3}

\begin{defn}\label{3.3.1}
  The system of equations
  \[
    (S) \quad \begin{matrix}
      a_{1 1} x_1 + a_{1 2} x_2 + \cdots + a_{1 n} x_n = b_1 \\
      a_{2 1} x_1 + a_{2 2} x_2 + \cdots + a_{2 n} x_n = b_2 \\
      \vdots                                                 \\
      a_{m 1} x_1 + a_{m 2} x_2 + \cdots + a_{m n} x_n = b_m
    \end{matrix}
  \]
  where \(a_{i j}\) and \(b_i\) (\(1 \leq i \leq m\) and \(1 \leq j \leq n\)) are scalars in a field \(\F\) and \(\seq{x}{1,,n}\) are \(n\) variables taking values in \(\F\), is called a \textbf{system of \(m\) linear equations in \(n\) unknowns over the field \(\F\)}.

  The \(m \times n\) matrix
  \[
    A = \begin{pmatrix}
      a_{1 1} & a_{1 2} & \cdots & a_{1 n} \\
      a_{2 1} & a_{2 2} & \cdots & a_{2 n} \\
      \vdots  & \vdots  &        & \vdots  \\
      a_{m 1} & a_{m 2} & \cdots & a_{m n}
    \end{pmatrix}
  \]
  is called the \textbf{coefficient matrix} of the system \((S)\).

  If we let
  \[
    x = \begin{pmatrix}
      x_1    \\
      x_2    \\
      \vdots \\
      x_n
    \end{pmatrix} \quad \text{and} \quad b = \begin{pmatrix}
      b_1    \\
      b_2    \\
      \vdots \\
      b_m
    \end{pmatrix},
  \]
  then the system \((S)\) may be rewritten as a single matrix equation
  \[
    Ax = b.
  \]
  To exploit the results that we have developed, we often consider a system of linear equations as a single matrix equation.

  A \textbf{solution} to the system \((S)\) is an \(n\)-tuple
  \[
    s = \begin{pmatrix}
      s_1    \\
      s_2    \\
      \vdots \\
      s_n
    \end{pmatrix} \in \vs{F}^n
  \]
  such that \(As = b\).
  The set of all solutions to the system \((S)\) is called the \textbf{solution set} of the system.
  System \((S)\) is called \textbf{consistent} if its solution set is nonempty;
  otherwise it is called \textbf{inconsistent}.
\end{defn}

\begin{defn}\label{3.3.2}
  A system \(Ax = b\) of \(m\) linear equations in \(n\) unknowns is said to be \textbf{homogeneous} if \(b = \zv\).
  Otherwise the system is said to be \textbf{nonhomogeneous}.
\end{defn}

\begin{note}
  Any homogeneous system has at least one solution, namely, the zero vector.
\end{note}

\begin{thm}\label{3.8}
  Let \(Ax = \zv\) be a homogeneous system of \(m\) linear equations in \(n\) unknowns over a field \(\F\).
  Let \(\vs{K}\) denote the set of all solutions to \(Ax = \zv\).
  Then \(\vs{K} = \ns{\L_A}\);
  hence \(\vs{K}\) is a subspace of \(\vs{F}^n\) of dimension \(n - \rk{\L_A} = n - \rk{A}\).
\end{thm}

\begin{proof}[\pf{3.8}]
  Clearly, \(\vs{K} = \set{s \in \vs{F}^n : As = \zv} = \ns{\L_A}\).
  The second part now follows from the dimension theorem (\cref{2.3}).
\end{proof}

\begin{cor}\label{3.3.3}
  If \(m < n\), the system \(Ax = \zv\) has a nonzero solution.
\end{cor}

\begin{proof}[\pf{3.3.3}]
  Suppose that \(m < n\).
  Then we have \(\rk{A} = \rk{\L_A} \leq m\) by \cref{3.6}.
  Hence
  \[
    \dim(\vs{K}) = n - \rk{\L_A} \geq n - m > 0,
  \]
  where \(\vs{K} = \ns{\L_A}\).
  Since \(\dim(\vs{K}) > 0\), \(\vs{K} \neq \set{\zv}\).
  Thus there exists a nonzero vector \(s \in \vs{K}\);
  so \(s\) is a nonzero solution to \(Ax = \zv\).
\end{proof}

\begin{defn}\label{3.3.4}
  We refer to the equation \(Ax = \zv\) as the \textbf{homogeneous system corresponding to} \(Ax = b\).
\end{defn}

\begin{thm}\label{3.9}
  Let \(K\) be the solution set of a system of linear equations \(Ax = b\), and let \(\vs{K}_{\vs{H}}\) be the solution set of the corresponding homogeneous system \(Ax = \zv\).
  Then for any solution \(s\) to \(Ax = b\)
  \[
    K = \set{s} + \vs{K}_{\vs{H}} = \set{s + k : k \in \vs{K}_{\vs{H}}}.
  \]
\end{thm}

\begin{proof}[\pf{3.9}]
  Let \(s\) be any solution to \(Ax = b\).
  We must show that \(K = \set{s} + \vs{K}_{\vs{H}}\) .
  If \(w \in K\), then \(Aw = b\).
  Hence
  \[
    A(w - s) = Aw - As = b - b = \zv.
  \]
  So \(w - s \in \vs{K}_{\vs{H}}\).
  Thus there exists \(k \in \vs{K}_{\vs{H}}\) such that \(w - s = k\).
  It follows that \(w = s + k \in \set{s} + \vs{K}_{\vs{H}}\), and therefore
  \[
    K \subseteq \set{s} + \vs{K}_{\vs{H}}.
  \]

  Conversely, suppose that \(w \in \set{s} + \vs{K}_{\vs{H}}\);
  then \(w = s + k\) for some \(k \in \vs{K}_{\vs{H}}\).
  But then \(Aw = A(s + k) = As + Ak = b + \zv = b\);
  so \(w \in K\).
  Therefore \(\set{s} + \vs{K}_{\vs{H}} \subseteq K\), and thus \(K = \set{s} + \vs{K}_{\vs{H}}\).
\end{proof}

\begin{thm}\label{3.10}
  Let \(Ax = b\) be a system of \(n\) linear equations in \(n\) unknowns.
  If \(A\) is invertible, then the system has exactly one solution, namely, \(A^{-1} b\).
  Conversely, if the system has exactly one solution, then \(A\) is invertible.
\end{thm}

\begin{proof}[\pf{3.10}]
  Suppose that \(A\) is invertible.
  Substituting \(A^{-1} b\) into the system, we have \(A (A^{-1} b) = (A A^{-1}) b = b\).
  Thus \(A^{-1} b\) is a solution.
  If \(s\) is an arbitrary solution, then \(As = b\).
  Multiplying both sides by \(A^{-1}\) gives \(s = A^{-1} b\).
  Thus the system has one and only one solution, namely, \(A^{-1} b\).

  Conversely, suppose that the system has exactly one solution \(s\).
  Let \(\vs{K}_{\vs{H}}\) denote the solution set for the corresponding homogeneous system \(Ax = \zv\).
  By \cref{3.9}, \(\set{s} = \set{s} + \vs{K}_{\vs{H}}\).
  But this is so only if \(\vs{K}_{\vs{H}} = \set{\zv}\).
  Thus \(\ns{\L_A} = \set{\zv}\), and hence \(A\) is invertible (by \cref{2.5}).
\end{proof}

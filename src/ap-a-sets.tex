\chapter{Sets}\label{ch:a}

\begin{defn}\label{a.0.1}
  A \textbf{set} is a collection of objects, called \textbf{elements} of the set.
  If \(x\) is an element of the set \(A\), then we write \(x \in A\);
  otherwise, we write \(x \notin A\).
\end{defn}

\begin{note}
  One set that appears frequently is the set of real numbers, which we denote by \(\R\) throughout this text.
\end{note}

\begin{defn}\label{a.0.2}
  Two sets \(A\) and \(B\) are called equal, written \(A = B\), if they contain exactly the same elements.
  Sets may be described in one of two ways:
  \begin{enumerate}
    \item By listing the elements of the set between set braces \(\set{}\).
    \item By describing the elements of the set in terms of some characteristic property.
  \end{enumerate}
\end{defn}

\begin{note}
  The order in which the elements of a set are listed is immaterial.
\end{note}

\begin{defn}\label{a.0.3}
  A set \(B\) is called a \textbf{subset} of a set \(A\), written \(B \subseteq A\) or \(A \supseteq B\), if every element of \(B\) is an element of \(A\).
  If \(B \subseteq A\), and \(B \neq A\), then \(B\) is called a \textbf{proper subset} of \(A\).
  Observe that \(A = B\) iff \(A \subseteq B\) and \(B \subseteq A\), a fact that is often used to prove that two sets are equal.
\end{defn}

\begin{defn}\label{a.0.4}
  The \textbf{empty set}, denoted by \(\varnothing\), is the set containing no elements.
  The empty set is a subset of every set.
\end{defn}

\begin{defn}\label{a.0.5}
  Sets may be combined to form other sets in two basic ways.
  The \textbf{union} of two sets \(A\) and \(B\), denoted \(A \cup B\), is the set of elements that are in \(A\), or \(B\), or both;
  that is,
  \[
    A \cup B = \set{x : x \in A \text{ or } x \in B}.
  \]
  The \textbf{intersection} of two sets \(A\) and \(B\), denoted \(A \cap B\), is the set of elements that are in both \(A\) and \(B\);
  that is,
  \[
    A \cap B = \set{x : x \in A \text{ and } x \in B}.
  \]
  Two sets are called \textbf{disjoint} if their intersection equals the empty set.
  The union and intersection of more than two sets can be defined analogously.
  Specifically, if \(\seq{A}{1,,n}\) are sets, then the union and intersections of these sets are defined, respectively, by
  \[
    \bigcup_{i = 1}^n A_i = \set{x : x \in A_i \text{ for some } i \in \set{1, \dots, n}}
  \]
  and
  \[
    \bigcap_{i = 1}^n A_i = \set{x : x \in A_i \text{ for all } i \in \set{1, \dots, n}}.
  \]
  Similarly, if \(\Lambda\) is an index set and \(\set{A_{\alpha} : \alpha \in \Lambda}\) is a collection of sets, the union and intersection of these sets are defined, respectively, by
  \[
    \bigcup_{\alpha \in \Lambda} A_{\alpha} = \set{x : x \in A_{\alpha} \text{ for some } \alpha \in \Lambda}
  \]
  and
  \[
    \bigcap_{\alpha \in \Lambda} A_{\alpha} = \set{x : x \in A_{\alpha} \text{ for all } \alpha \in \Lambda}.
  \]
\end{defn}

\begin{defn}\label{a.0.6}
  By a relation on a set \(A\), we mean a rule for determining whether or not, for any elements \(x\) and \(y\) in \(A\), \(x\) stands in a given relationship to \(y\).
  More precisely, a \textbf{relation} on \(A\) is a set \(S\) of ordered pairs of elements of \(A\) such that \((x, y) \in S\) iff \(x\) stands in the given relationship to \(y\).
  If \(S\) is a relation on a set \(A\), we often write \(x \sim y\) in place of \((x, y) \in S\).
\end{defn}

\begin{defn}\label{a.0.7}
  A relation \(S\) on a set \(A\) is called an \textbf{equivalence relation} on \(A\) if the following three conditions hold:
  \begin{description}
    \item[Reflexivity:]
      For each \(x \in A\), \(x \sim x\).
    \item[Symmetry:]
      If \(x \sim y\), then \(y \sim x\).
    \item[Transitivity:]
      If \(x \sim y\) and \(y \sim z\), then \(x \sim z\).
  \end{description}
\end{defn}

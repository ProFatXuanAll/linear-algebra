\section{Conditioning and the Rayleigh Quotient}\label{sec:6.10}

\begin{defn}\label{6.10.1}
  In \cref{sec:3.4}, we studied specific techniques that allow us to solve systems of linear equations in the form \(Ax = b\), where \(A \in \ms\) and \(b \in \ms[m][1][\F]\).
  Such systems often arise in applications to the real world.
  The coefficients in the system are frequently obtained from experimental data, and, in many cases, both \(m\) and \(n\) are so large that a computer must be used in the calculation of the solution.
  Thus two types of errors must be considered.
  First, experimental errors arise in the collection of data since no instruments can provide completely accurate measurements.
  Second, computers introduce roundoff errors.
  One might intuitively feel that small relative changes in the coefficients of the system cause small relative errors in the solution.
  A system that has this property is called \textbf{well-conditioned};
  otherwise, the system is called \textbf{ill-conditioned}.
\end{defn}

\begin{note}
  We now consider several examples of these types of errors, concentrating primarily on changes in \(b\) rather than on changes in the entries of \(A\).
  In addition, we assume that \(A\) is a square, complex (or real), invertible matrix since this is the case most frequently encountered in applications.

  Of course, we are really interested in \emph{relative changes} since a change in the solution of, say, \(10\), is considered large if the original solution is of the order \(10^{-2}\), but small if the original solution is of the order \(10^6\).
\end{note}

\begin{defn}\label{6.10.2}
  We use the notation \(\delta b\) to denote the vector \(b' - b\), where \(b\) is the vector in the original system and \(b'\) is the vector in the modified system.
  We now define the \textbf{relative change} in \(b\) to be the scalar \(\dfrac{\norm{\delta b}}{\norm{b}}\), where \(\norm{\cdot}\) denotes the standard norm on \(\C^n\) (or \(\R^n\));
  that is, \(\norm{b} = \sqrt{\inn{b, b}}\).
  Most of what follows, however, is true for any norm.
  Similar definitions hold for the \textbf{relative change} in \(x\).
\end{defn}

\begin{note}
  If the lines defined by the two equations are nearly coincident, then a small change in either line could greatly alter the point of intersection, that is, the solution to the system.
\end{note}

\begin{defn}\label{6.10.3}
  Let \(A\) be a complex (or real) \(n \times n\) matrix.
  Define the \textbf{(Euclidean) norm} of \(A\) by
  \[
    \norm{A} = \max_{x \neq \zv} \dfrac{\norm{Ax}}{\norm{x}},
  \]
  where \(x \in \C^n\) or \(x \in \R^n\).
\end{defn}

\begin{note}
  Intuitively, \(\norm{A}\) represents the maximum \emph{magnification} of a vector by the matrix \(A\).
  The question of whether or not this maximum exists, as well as the problem of how to compute it, can be answered by the use of the so-called \emph{Rayleigh quotient}.
\end{note}

\begin{defn}\label{6.10.4}
  Let \(B\) be an \(n \times n\) self-adjoint matrix.
  The \textbf{Rayleigh quotient} for \(x \neq \zv\) is defined to be the scalar \(R(x) = \dfrac{\inn{Bx, x}}{\norm{x}^2}\).
\end{defn}

\begin{thm}\label{6.43}
  For a self-adjoint matrix \(B \in \ms[n][n][\F]\), we have that \(\max_{x \neq \zv} R(x)\) is the largest eigenvalue of \(B\) and \(\min_{x \neq \zv} R(x)\) is the smallest eigenvalue of \(B\).
\end{thm}

\begin{proof}[\pf{6.10.4}]
  By \cref{6.19,6.20}, we may choose an orthonormal basis \(\set{\seq{v}{1,,n}}\) of eigenvectors of \(B\) over \(\F\) such that \(B v_i = \lambda_i v_i\) (\(i \in \set{1, \dots, n}\)), where \(\seq[\geq]{\lambda}{1,,n}\).
  (Recall that by \cref{6.4.10}(a), the eigenvalues of \(B\) are real.)
  Now, for \(x \in \vs{F}^n\), there exist scalars \(\seq{a}{1,,n} \in \F\) such that
  \[
    x = \sum_{i = 1}^n a_i v_i.
  \]
  Hence
  \begin{align*}
    R(x) & = \dfrac{\inn{Bx, x}}{\norm{x}^2}                                                           &  & \by{6.10.4}     \\
         & = \dfrac{\inn{\sum_{i = 1}^n a_i \lambda_i v_i, \sum_{j = 1}^n a_j v_j}}{\norm{x}^2}        &  & \by{5.1.2}      \\
         & = \dfrac{\sum_{i = 1}^n a_i \lambda_i \inn{v_i, \sum_{j = 1}^n a_j v_j}}{\norm{x}^2}        &  & \by{6.1.1}[a,b] \\
         & = \dfrac{\sum_{i = 1}^n \sum_{j = 1}^n a_i \conj{a_j} \lambda_i \inn{v_i, v_j}}{\norm{x}^2} &  & \by{6.1}[a,b]   \\
         & = \dfrac{\sum_{i = 1}^n a_i \conj{a_i} \lambda_i}{\norm{x}^2}                               &  & \by{6.1.12}     \\
         & = \dfrac{\sum_{i = 1}^n \lambda_i \abs{a_i}^2}{\norm{x}^2}                                  &  & \by{d.0.5}      \\
         & \leq \dfrac{\lambda_1 \sum_{i = 1}^n \abs{a_i}^2}{\norm{x}^2}                               &  & \by{6.2}[b]     \\
         & = \dfrac{\lambda_1 \norm{x}^2}{\norm{x}^2}                                                  &  & \by{ex:6.1.12}  \\
         & = \lambda_1.
  \end{align*}
  It is easy to see that \(R(v_1) = \lambda_1\), so we have demonstrated the first half of the theorem.
  The second half is proved similarly, i.e.,
  \begin{align*}
    R(x) & = \dfrac{\sum_{i = 1}^n \lambda_i \abs{a_i}^2}{\norm{x}^2}    &  & \text{(from the proof above)} \\
         & \geq \dfrac{\lambda_n \sum_{i = 1}^n \abs{a_i}^2}{\norm{x}^2} &  & \by{6.2}[b]                   \\
         & = \dfrac{\lambda_n \norm{x}^2}{\norm{x}^2}                    &  & \by{ex:6.1.12}                \\
         & = \lambda_n.
  \end{align*}
\end{proof}

\begin{cor}\label{6.10.5}
  For any square matrix \(A\), \(\norm{A}\) is finite and, in fact, equals \(\sqrt{\lambda}\), where \(\lambda\) is the largest eigenvalue of \(A^* A\).
\end{cor}

\begin{proof}[\pf{6.10.5}]
  Let \(B\) be the self-adjoint matrix \(A^* A\) (\cref{ex:6.4.18}(a)), and let \(\lambda\) be the largest eigenvalue of \(B\) (\cref{ex:6.4.17}(a)).
  Since, for \(x \neq \zv\),
  \begin{align*}
    0 & \leq \dfrac{\norm{Ax}^2}{\norm{x}^2}   &  & \by{6.2}[b] \\
      & = \dfrac{\inn{Ax, Ax}}{\norm{x}^2}     &  & \by{6.1.9}  \\
      & = \dfrac{\inn{A^* A x, x}}{\norm{x}^2} &  & \by{6.3.4}  \\
      & = \dfrac{\inn{Bx, x}}{\norm{x}^2}                       \\
      & = R(x),                                &  & \by{6.10.4}
  \end{align*}
  it follows from \cref{6.43} that \(\norm{A}^2 = \lambda\).
\end{proof}

\begin{note}
  Observe that the proof of \cref{6.10.5} shows that all the eigenvalues of \(A^* A\) are nonnegative.
  (positive semidefinite by \cref{ex:6.4.17}(a).)
\end{note}

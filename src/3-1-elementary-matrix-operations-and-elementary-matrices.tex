\section{Elementary Matrix Operations and Elementary Matrices}\label{sec:3.1}

\begin{defn}\label{3.1.1}
  Let \(A \in \MS\).
  Any one of the following three operations on the rows (columns) of \(A\) is called an \textbf{elementary row (column) operation}:
  \begin{enumerate}[label=(\arabic*)]
    \item interchanging any two rows (columns) of \(A\);
    \item multiplying any row (column) of \(A\) by a nonzero scalar;
    \item adding any scalar multiple of a row (column) of \(A\) to another row (column).
  \end{enumerate}
  Any of these three operations is called an \textbf{elementary operation}.
  Elementary operations are of \textbf{type 1}, \textbf{type 2}, or \textbf{type 3} depending on whether they are obtained by (1), (2), or (3).
\end{defn}

\begin{note}
  Notice that if a matrix \(Q\) can be obtained from a matrix \(P\) by means of an elementary row operation, then \(P\) can be obtained from \(Q\) by an elementary row operation of the same type.
  (See \cref{ex:3.1.8}.)
\end{note}

\begin{defn}\label{3.1.2}
  An \(n \times n\) \textbf{elementary matrix} is a matrix obtained by performing an elementary operation on \(I_n\).
  The elementary matrix is said to be of \textbf{type 1}, \textbf{2}, or \textbf{3} according to whether the elementary operation performed on \(I_n\) is a type 1, 2, or 3 operation, respectively.
\end{defn}

\begin{note}
  Any elementary matrix can be obtained in at least two ways ---
  either by performing an elementary row operation on \(I_n\) or by performing an elementary column operation on \(I_n\).
  (See \cref{ex:3.1.4}.)
\end{note}

\begin{thm}\label{3.1}
  Let \(A \in \MS\), and suppose that \(B\) is obtained from \(A\) by performing an elementary row (column) operation.
  Then there exists an \(m \times m\) (\(n \times n\)) elementary matrix \(E\) such that \(B = EA\) (\(B = AE\)).
  In fact, \(E\) is obtained from \(I_m\) (\(I_n\)) by performing the same elementary row (column) operation as that which was performed on \(A\) to obtain \(B\).
  Conversely, if \(E\) is an elementary \(m \times m\) (\(n \times n\)) matrix, then \(EA\) (\(AE\)) is the matrix obtained from \(A\) by performing the same elementary row (column) operation as that which produces \(E\) from \(I_m\) (\(I_n\)).
\end{thm}

\begin{proof}[\pf{3.1}]
  First we prove that \cref{3.1} is true for a type 1 row operation.
  Let \(A, B \in \MS\) such that \(B\) is obtained by interchanging two rows of \(A\).
  Let the interchanging rows be \(i, j \in \set{1, \dots, m}\).
  If \(E \in \ms{m}{m}{\F}\) is the elementary matrix performing the same operation, then by \cref{2.3.4} we have
  \[
    \forall p, q \in \set{1, \dots, m}, E_{p q} = \begin{dcases}
      \delta_{p q} & \text{if } (p \neq i) \land (p \neq j) \\
      \delta_{j q} & \text{if } p = i                       \\
      \delta_{i q} & \text{if } p = j
    \end{dcases}.
  \]
  Thus
  \begin{align*}
     & \forall (p, k) \in \set{1, \dots, m} \times \set{1, \dots, n},                                                \\
     & (EA)_{p k} = \sum_{q = 1}^m E_{p q} A_{q k}                                     &  & \text{(by \cref{2.3.1})} \\
     & = \begin{dcases}
           \sum_{q = 1}^m \delta_{p q} A_{q k} & \text{if } (p \neq i) \land (p \neq j) \\
           \sum_{q = 1}^m \delta_{j q} A_{q k} & \text{if } p = i                       \\
           \sum_{q = 1}^m \delta_{i q} A_{q k} & \text{if } p = j
         \end{dcases}                                \\
     & = \begin{dcases}
           A_{p k} & \text{if } (p \neq i) \land (p \neq j) \\
           A_{j k} & \text{if } p = i                       \\
           A_{i k} & \text{if } p = j
         \end{dcases}                                                            \\
     & = B_{p k}                                                                       &  & \text{(by \cref{3.1.1})}
  \end{align*}
  and we have \(B = EA\).

  Next we prove that \cref{3.1} is true for a type 1 column operation.
  Let \(A, B \in \MS\) such that \(B\) is obtained by interchanging two columns of \(A\).
  Let the interchanging columns be \(i, j \in \set{1, \dots, n}\).
  If \(E \in \ms{n}{n}{\F}\) is the elementary matrix performing the same operation, then by \cref{2.3.4} we have
  \[
    \forall p, q \in \set{1, \dots, n}, E_{p q} = \begin{dcases}
      \delta_{p q} & \text{if } (q \neq i) \land (q \neq j) \\
      \delta_{p j} & \text{if } q = i                       \\
      \delta_{p i} & \text{if } q = j
    \end{dcases}.
  \]
  Thus
  \begin{align*}
     & \forall (k, q) \in \set{1, \dots, m} \times \set{1, \dots, n},                                                \\
     & (AE)_{k q} = \sum_{p = 1}^n A_{k p} E_{p q}                                     &  & \text{(by \cref{2.3.1})} \\
     & = \begin{dcases}
           \sum_{p = 1}^n A_{k p} \delta_{p q} & \text{if } (q \neq i) \land (q \neq j) \\
           \sum_{p = 1}^n A_{k p} \delta_{p j} & \text{if } q = i                       \\
           \sum_{p = 1}^n A_{k p} \delta_{p i} & \text{if } q = j
         \end{dcases}                                \\
     & = \begin{dcases}
           A_{k q} & \text{if } (q \neq i) \land (q \neq j) \\
           A_{k j} & \text{if } q = i                       \\
           A_{k i} & \text{if } q = j
         \end{dcases}                                                            \\
     & = B_{k q}                                                                       &  & \text{(by \cref{3.1.1})}
  \end{align*}
  and we have \(B = AE\).

  Next we prove that \cref{3.1} is true for a type 2 row operation.
  Let \(c \in \F\) and let \(A, B \in \MS\) such that \(B\) is obtained by multiplying the \(i\)th row of \(A\) with \(c\).
  If \(E \in \ms{m}{m}{\F}\) is the elementary matrix performing the same operation, then by \cref{2.3.4} we have
  \[
    \forall p, q \in \set{1, \dots, m}, E_{p q} = \begin{dcases}
      \delta_{p q}   & \text{if } p \neq i \\
      c \delta_{i q} & \text{if } p = i
    \end{dcases}.
  \]
  Thus
  \begin{align*}
     & \forall (p, k) \in \set{1, \dots, m} \times \set{1, \dots, n},                               \\
     & (EA)_{p k} = \sum_{q = 1}^m E_{p q} A_{q k}                    &  & \text{(by \cref{2.3.1})} \\
     & = \begin{dcases}
           \sum_{q = 1}^m \delta_{p q} A_{q k}   & \text{if } p \neq i \\
           \sum_{q = 1}^m c \delta_{i q} A_{q k} & \text{if } p = i
         \end{dcases}                                \\
     & = \begin{dcases}
           A_{p k}   & \text{if } p \neq i \\
           c A_{i k} & \text{if } p = i
         \end{dcases}                                                            \\
     & = B_{p k}                                                      &  & \text{(by \cref{3.1.1})}
  \end{align*}
  and we have \(B = EA\).

  Next we prove that \cref{3.1} is true for a type 2 column operation.
  Let \(c \in \F\) and let \(A, B \in \MS\) such that \(B\) is obtained by multiplying the \(i\)th column of \(A\) with \(c\).
  If \(E \in \ms{n}{n}{\F}\) is the elementary matrix performing the same operation, then by \cref{2.3.4} we have
  \[
    \forall p, q \in \set{1, \dots, n}, E_{p q} = \begin{dcases}
      \delta_{p q}   & \text{if } q \neq i \\
      c \delta_{p i} & \text{if } q = i
    \end{dcases}.
  \]
  Thus
  \begin{align*}
     & \forall (k, q) \in \set{1, \dots, m} \times \set{1, \dots, n},                               \\
     & (AE)_{k q} = \sum_{p = 1}^n A_{k p} E_{p q}                    &  & \text{(by \cref{2.3.1})} \\
     & = \begin{dcases}
           \sum_{p = 1}^n A_{k p} \delta_{p q}   & \text{if } q \neq i \\
           \sum_{p = 1}^n A_{k p} c \delta_{p i} & \text{if } q = i
         \end{dcases}                                \\
     & = \begin{dcases}
           A_{k q}   & \text{if } q \neq i \\
           c A_{k i} & \text{if } q = i
         \end{dcases}                                                            \\
     & = B_{k q}                                                      &  & \text{(by \cref{3.1.1})}
  \end{align*}
  and we have \(B = AE\).

  Next we prove that \cref{3.1} is true for a type 3 row operation.
  Let \(i, j \in \set{1, \dots, m}\) such that \(i \neq j\).
  Let \(A, B \in \MS\) such that \(B\) is obtained by multiplying the \(i\)th row of \(A\) with \(c \in \F\) and add it to the \(j\)th row of \(A\).
  If \(E \in \ms{m}{m}{\F}\) is the elementary matrix performing the same operation, then by \cref{2.3.4} we have
  \[
    \forall p, q \in \set{1, \dots, m}, E_{p q} = \begin{dcases}
      \delta_{p q}                  & \text{if } p \neq j \\
      c \delta_{i q} + \delta_{j q} & \text{if } p = j
    \end{dcases}.
  \]
  Thus
  \begin{align*}
     & \forall (p, k) \in \set{1, \dots, m} \times \set{1, \dots, n},                                                                      \\
     & (EA)_{p k} = \sum_{q = 1}^m E_{p q} A_{q k}                                                           &  & \text{(by \cref{2.3.1})} \\
     & = \begin{dcases}
           \sum_{q = 1}^m \delta_{p q} A_{q k}                    & \text{if } p \neq j \\                    \\
           \sum_{q = 1}^m (c \delta_{i q} + \delta_{j q}) A_{q k} & \text{if } p = j
         \end{dcases}                                \\
     & = \begin{dcases}
           A_{p k}             & \text{if } p \neq j \\
           c A_{i k} + A_{j k} & \text{if } p = j
         \end{dcases}                                                                                         \\
     & = B_{p k}                                                                                             &  & \text{(by \cref{3.1.1})}
  \end{align*}
  and we have \(B = EA\).

  Finally we prove that \cref{3.1} is true for a type 3 column operation.
  Let \(i, j \in \set{1, \dots, m}\) such that \(i \neq j\).
  Let \(A, B \in \MS\) such that \(B\) is obtained by multiplying the \(i\)th column of \(A\) with \(c \in \F\) and add it to the \(j\)th column of \(A\).
  If \(E \in \ms{n}{n}{\F}\) is the elementary matrix performing the same operation, then by \cref{2.3.4} we have
  \[
    \forall p, q \in \set{1, \dots, n}, E_{p q} = \begin{dcases}
      \delta_{p q}                  & \text{if } q \neq j \\
      c \delta_{p i} + \delta_{p j} & \text{if } q = j
    \end{dcases}.
  \]
  Thus
  \begin{align*}
     & \forall (k, q) \in \set{1, \dots, m} \times \set{1, \dots, n},                                                \\
     & (AE)_{k q} = \sum_{p = 1}^n A_{k p} E_{p q}                                     &  & \text{(by \cref{2.3.1})} \\
     & = \begin{dcases}
           \sum_{p = 1}^n A_{k p} \delta_{p q}                    & \text{if } q \neq j \\
           \sum_{p = 1}^n A_{k p} (c \delta_{p i} + \delta_{p j}) & \text{if } q = j
         \end{dcases}                                \\
     & = \begin{dcases}
           A_{k q}             & \text{if } q \neq j \\
           c A_{k i} + A_{k j} & \text{if } q = j
         \end{dcases}                                                                   \\
     & = B_{k q}                                                                       &  & \text{(by \cref{3.1.1})}
  \end{align*}
  and we have \(B = AE\).
\end{proof}

\begin{thm}\label{3.2}
  Elementary matrices are invertible, and the inverse of an elementary matrix is an elementary matrix of the same type.
\end{thm}

\begin{proof}[\pf{3.2}]
  Let \(E\) be an elementary \(n \times n\) matrix.
  Then \(E\) can be obtained by an elementary row operation on \(I_n\).
  By reversing the steps used to transform \(I_n\) into \(E\), we can transform \(E\) back into \(I_n\).
  The result is that \(I_n\) can be obtained from \(E\) by an elementary row operation of the same type.
  By \cref{3.1} there is an elementary matrix \(\overline{E}\) such that \(\overline{E} E = I_n\).
  Therefore, by \cref{ex:2.4.10} \(E\) is invertible and \(E^{-1} = \overline{E}\).
\end{proof}

\exercisesection

\setcounter{ex}{3}
\begin{ex}\label{ex:3.1.4}
  Prove the following statement:
  Any elementary \(n \times n\) matrix can be obtained in at least two ways.
  More precisely, let \(E \in \ms{n}{n}{\F}\) and let \(c \in \F \setminus \set{0}\).
  For distinct \(i, j \in \set{1, \dots, n}\), we have
  \begin{itemize}
    \item \(E\) is an elementary matrix of type 1 which interchange the \(i\)th row with \(j\)th row, iff, \(E\) is an elementary matrix of type 1 which interchange the \(i\)th column with \(j\)th column.
    \item \(E\) is an elementary matrix of type 2 which multiply the \(i\)th row with \(c\), iff, \(E\) is an elementary matrix of type 2 which multiply the \(i\)th column with \(c\).
    \item \(E\) is an elementary matrix of type 3 which multiply the \(i\)th row with \(c\) and add it to the \(j\)th row, iff, \(E\) is an elementary matrix of type 3 which multiply the \(j\)th column with \(c\) and add it to the \(i\)th column.
  \end{itemize}
\end{ex}

\begin{proof}[\pf{ex:3.1.4}]
  First we show that a type 1 elementary matrix can be obtained in at least two ways.
  Fix \(i, j \in \set{1, \dots, n}\) and \(i \neq j\).
  Let \(E \in \ms{n}{n}{\F}\) be an elementary matrix of type 1 which interchange the \(i\)th row with \(j\)th row, and let \(F \in \ms{n}{n}{\F}\) be an elementary matrix of type 1 which interchange the \(i\)th column with \(j\)th column.
  Observe that
  \begin{align*}
    \forall p, q \in \set{1, \dots, n}, E_{p q} & = \begin{dcases}
                                                      \delta_{p q} & \text{if } (p \neq i) \land (p \neq j) \\
                                                      \delta_{j q} & \text{if } p = i                       \\
                                                      \delta_{i q} & \text{if } p = j
                                                    \end{dcases}    &  & \text{(by \cref{2.3.4})}    \\
                                                & = \begin{dcases}
                                                      1 & \text{if } (p \neq i) \land (p \neq j) \land (p = q) \\
                                                      1 & \text{if } (p = i) \land (q = j)                     \\
                                                      1 & \text{if } (p = j) \land (q = i)                     \\
                                                      0 & \text{otherwise}
                                                    \end{dcases} &  & \text{(by \cref{2.3.4})} \\
                                                & = \begin{dcases}
                                                      1 & \text{if } (q \neq i) \land (q \neq j) \land (p = q) \\
                                                      1 & \text{if } (p = i) \land (q = j)                     \\
                                                      1 & \text{if } (p = j) \land (q = i)                     \\
                                                      0 & \text{otherwise}
                                                    \end{dcases} \\
                                                & = \begin{dcases}
                                                      \delta_{p q} & \text{if } (q \neq i) \land (q \neq j) \\
                                                      \delta_{p i} & \text{if } q = j                       \\
                                                      \delta_{p j} & \text{if } q = i
                                                    \end{dcases}    &  & \text{(by \cref{2.3.4})}    \\
                                                & = F_{p q}.
  \end{align*}
  Thus a type 1 elementary matrix can be obtained in at least two ways.

  Next we show that a type 2 elementary matrix can be obtained in at least two ways.
  Fix \(i \in \set{1, \dots, n}\).
  Let \(c \in \F\), let \(E \in \ms{n}{n}{\F}\) be an elementary matrix of type 2 which multiply the \(i\)th row with \(c\), and let \(F \in \ms{n}{n}{\F}\) be an elementary matrix of type 2 which multiply the \(i\)th column with \(c\).
  Observe that
  \begin{align*}
    \forall p, q \in \set{1, \dots, n}, E_{p q} & = \begin{dcases}
                                                      \delta_{p q}   & \text{if } p \neq i \\
                                                      c \delta_{i q} & \text{if } p = i
                                                    \end{dcases}    &  & \text{(by \cref{2.3.4})}                          \\
                                                & = \begin{dcases}
                                                      1 & \text{if } (p \neq i) \land (p = q) \\
                                                      c & \text{if } (p = i) \land (q = i)    \\
                                                      0 & \text{otherwise}
                                                    \end{dcases} &  & \text{(by \cref{2.3.4})}                             \\
                                                & = \begin{dcases}
                                                      1 & \text{if } (q \neq i) \land (p = q) \\
                                                      c & \text{if } (p = i) \land (q = i)    \\
                                                      0 & \text{otherwise}
                                                    \end{dcases}                                \\
                                                & = \begin{dcases}
                                                      \delta_{p q}   & \text{if } q \neq i \\
                                                      c \delta_{p i} & \text{if } q = i
                                                    \end{dcases}    &  & \text{(by \cref{2.3.4})}                          \\
                                                & = F_{p q}.                                 &  & \text{(by \cref{2.3.4})}
  \end{align*}
  Thus a type 2 elementary matrix can be obtained in at least two ways.

  Finally we show that a type 3 elementary matrix can be obtained in at least two ways.
  Fix \(i, j \in \set{1, \dots, n}\) and \(i \neq j\).
  Let \(c \in \F\), let \(E \in \ms{n}{n}{\F}\) be an elementary matrix of type 3 which multiply the \(i\)th row with \(c\) and add it to the \(j\)th row, and let \(F \in \ms{n}{n}{\F}\) be an elementary matrix of type 3 which multiply the \(j\)th column with \(c\) and add it to the \(i\)th column.
  Observe that
  \begin{align*}
    \forall p, q \in \set{1, \dots, n}, E_{p q} & = \begin{dcases}
                                                      \delta_{p q}                  & \text{if } p \neq j \\
                                                      c \delta_{i q} + \delta_{j q} & \text{if } p = j
                                                    \end{dcases} &  & \text{(by \cref{2.3.4})}                                \\
                                                & = \begin{dcases}
                                                      1 & \text{if } (p \neq j) \land (p = q) \\
                                                      c & \text{if } (p = j) \land (q = i)    \\
                                                      1 & \text{if } (p = j) \land (q = j)    \\
                                                      0 & \text{otherwise}
                                                    \end{dcases}             &  & \text{(by \cref{2.3.4})}                             \\
                                                & = \begin{dcases}
                                                      1 & \text{if } p = q                 \\
                                                      c & \text{if } (p = j) \land (q = i) \\
                                                      0 & \text{otherwise}
                                                    \end{dcases}                                               \\
                                                & = \begin{dcases}
                                                      1 & \text{if } (p = q) \land (q \neq i) \\
                                                      c & \text{if } (p = j) \land (q = i)    \\
                                                      1 & \text{if } (p = i) \land (q = i)    \\
                                                      0 & \text{otherwise}
                                                    \end{dcases}                                            \\
                                                & = \begin{dcases}
                                                      \delta_{p q}                  & \text{if } q \neq i \\
                                                      c \delta_{p j} + \delta_{p i} & \text{if } q = i
                                                    \end{dcases} &  & \text{(by \cref{2.3.4})}                                \\
                                                & = F_{p q}.                                             &  & \text{(by \cref{2.3.4})}
  \end{align*}
  Thus a type 3 elementary matrix can be obtained in at least two ways.
\end{proof}

\begin{ex}\label{ex:3.1.5}
  Prove that \(E\) is an elementary matrix iff \(\tp{E}\) is.
  More precisely, let \(E \in \ms{n}{n}{\F}\) and let \(c \in \F \setminus \set{0}\).
  For distinct \(i, j \in \set{1, \dots, n}\), we have
  \begin{itemize}
    \item \(\tp{E} = E\) iff \(E\) is an elementary matrix of type 1 or type 2.
    \item \(E\) is an elementary matrix of type 3 which multiply the \(i\)th row (column) with \(c\) and add it to the \(j\)th row (column), iff, \(\tp{E}\) is an elementary matrix of type 3 which multiply the \(i\)th row (column) with \(c\) and add it to the \(j\)th row (column).
  \end{itemize}
\end{ex}

\begin{proof}[\pf{ex:3.1.5}]
  First we show that the transpose of a type 1 elementary matrix is a type 1 elementary matrix.
  Fix \(i, j \in \set{1, \dots, n}\) and \(i \neq j\).
  Let \(E \in \ms{n}{n}{\F}\) be an elementary matrix of type 1 which interchange the \(i\)th row with \(j\)th row.
  Observe that
  \begin{align*}
    \forall p, q \in \set{1, \dots, n}, \pa{\tp{E}}_{p q} & = E_{q p}                                                   &  & \text{(by \cref{1.3.3})} \\
                                                          & = \begin{dcases}
                                                                \delta_{q p} & \text{if } (q \neq i) \land (q \neq j) \\
                                                                \delta_{j p} & \text{if } q = i                       \\
                                                                \delta_{i p} & \text{if } q = j
                                                              \end{dcases}    &  & \text{(by \cref{2.3.4})}                                   \\
                                                          & = \begin{dcases}
                                                                1 & \text{if } (q \neq i) \land (q \neq j) \land (q = p) \\
                                                                1 & \text{if } (q = i) \land (p = j)                     \\
                                                                1 & \text{if } (q = j) \land (p = i)                     \\
                                                                0 & \text{otherwise}
                                                              \end{dcases} &  & \text{(by \cref{2.3.4})}                                \\
                                                          & = \begin{dcases}
                                                                1 & \text{if } (p \neq i) \land (p \neq j) \land (q = p) \\
                                                                1 & \text{if } (q = i) \land (p = j)                     \\
                                                                1 & \text{if } (q = j) \land (p = i)                     \\
                                                                0 & \text{otherwise}
                                                              \end{dcases}                                \\
                                                          & = \begin{dcases}
                                                                \delta_{p q} & \text{if } (p \neq i) \land (p \neq j) \\
                                                                \delta_{i q} & \text{if } p = j                       \\
                                                                \delta_{j q} & \text{if } p = i
                                                              \end{dcases}    &  & \text{(by \cref{2.3.4})}                                   \\
                                                          & = E_{p q}.                                                  &  & \text{(by \cref{2.3.4})}
  \end{align*}
  Thus we have \(\tp{E} = E\).
  By \cref{ex:3.1.4} we see that \(E\) is also an elementary matrix of type 1 which interchange the \(i\)th column with \(j\)th column.
  Thus the transpose of a type 1 elementary matrix is a type 1 elementary matrix.

  Next we show that the transpose of a type 2 elementary matrix is a type 2 elementary matrix.
  Fix \(i \in \set{1, \dots, n}\).
  Let \(E \in \ms{n}{n}{\F}\) be an elementary matrix of type 2 which multiply the \(i\)th row with \(c \in \F\).
  Observe that
  \begin{align*}
    \forall p, q \in \set{1, \dots, n}, \pa{\tp{E}}_{p q} & = E_{q p}                                  &  & \text{(by \cref{1.3.3})} \\
                                                          & = \begin{dcases}
                                                                \delta_{q p}   & \text{if } q \neq i \\
                                                                c \delta_{i p} & \text{if } q = i
                                                              \end{dcases}    &  & \text{(by \cref{2.3.4})}                          \\
                                                          & = \begin{dcases}
                                                                1 & \text{if } (q \neq i) \land (q = p) \\
                                                                c & \text{if } (q = i) \land (p = i)    \\
                                                                0 & \text{otherwise}
                                                              \end{dcases} &  & \text{(by \cref{2.3.4})}                             \\
                                                          & = \begin{dcases}
                                                                1 & \text{if } (p \neq i) \land (q = p) \\
                                                                c & \text{if } (q = i) \land (p = i)    \\
                                                                0 & \text{otherwise}
                                                              \end{dcases}                                \\
                                                          & = \begin{dcases}
                                                                \delta_{p q}   & \text{if } p \neq i \\
                                                                c \delta_{i q} & \text{if } p = i
                                                              \end{dcases}    &  & \text{(by \cref{2.3.4})}                          \\
                                                          & = E_{p q}.                                 &  & \text{(by \cref{2.3.4})}
  \end{align*}
  Thus we have \(\tp{E} = E\).
  By \cref{ex:3.1.4} we see that \(E\) is also an elementary matrix of type 2 which multiply the \(i\)th column with \(c\).
  Thus the transpose of a type 2 elementary matrix is a type 2 elementary matrix.

  Finally we show that the transpose of a type 3 elementary matrix is a type 3 elementary matrix.
  Fix \(i, j \in \set{1, \dots, n}\) and \(i \neq j\).
  Let \(c \in \F\), let \(E \in \ms{n}{n}{\F}\) be an elementary matrix of type 3 which multiply the \(i\)th row with \(c\) and add it to the \(j\)th row, and let \(F \in \ms{n}{n}{\F}\) be an elementary matrix of type 3 which multiply the \(j\)th row with \(c\) and add it to the \(i\)th row.
  Observe that
  \begin{align*}
    \forall p, q \in \set{1, \dots, n}, \pa{\tp{E}}_{p q} & = E_{q p}                                              &  & \text{(by \cref{1.3.3})} \\
                                                          & = \begin{dcases}
                                                                \delta_{q p}                  & \text{if } q \neq j \\
                                                                c \delta_{i p} + \delta_{j p} & \text{if } q = j
                                                              \end{dcases} &  & \text{(by \cref{2.3.4})}                                \\
                                                          & = \begin{dcases}
                                                                1 & \text{if } (q \neq j) \land (q = p) \\
                                                                c & \text{if } (q = j) \land (p = i)    \\
                                                                1 & \text{if } (q = j) \land (p = j)    \\
                                                                0 & \text{otherwise}
                                                              \end{dcases}             &  & \text{(by \cref{2.3.4})}                             \\
                                                          & = \begin{dcases}
                                                                1 & \text{if } q = p                 \\
                                                                c & \text{if } (q = j) \land (p = i) \\
                                                                0 & \text{otherwise}
                                                              \end{dcases}                                               \\
                                                          & = \begin{dcases}
                                                                1 & \text{if } (p \neq i) \land (p = q) \\
                                                                c & \text{if } (p = i) \land (q = j)    \\
                                                                1 & \text{if } (p = i) \land (q = i)    \\
                                                                0 & \text{otherwise}
                                                              \end{dcases}                                            \\
                                                          & = \begin{dcases}
                                                                \delta_{p q}                  & \text{if } p \neq i \\
                                                                c \delta_{j q} + \delta_{i q} & \text{if } p = i
                                                              \end{dcases} &  & \text{(by \cref{2.3.4})}                                \\
                                                          & = F_{p q}.                                             &  & \text{(by \cref{2.3.4})}
  \end{align*}
  Thus we have \(\tp{E} = F\).
  By \cref{ex:3.1.4} we see that \(E\) is also an elementary matrix of type 3 which multiply the \(j\)th column with \(c\) and add it to the \(i\)th column.
  Thus the transpose of a type 3 elementary matrix is a type 3 elementary matrix.
\end{proof}

\begin{ex}\label{ex:3.1.6}
  Let \(A \in \MS\).
  Prove that if \(B\) can be obtained from \(A\) by an elementary row (column) operation, then \(\tp{B}\) can be obtained from \(\tp{A}\) by the corresponding elementary column (row) operation.
\end{ex}

\begin{proof}[\pf{ex:3.1.6}]
  First suppose that \(B\) is obtained from \(A\) by an elementary row operation.
  By \cref{3.1} we know that there exists an elementary matrix \(E\) such that \(B = EA\).
  By \cref{2.3.2} we know that \(\tp{B} = \tp{A} \tp{E}\).
  By \cref{ex:3.1.5} we know that \(\tp{E}\) is an elementary operation.
  Thus by \cref{3.1} we know that \(\tp{B}\) can be obtained from \(\tp{A}\) by an elementary column operation.

  Now suppose that \(B\) is obtained from \(A\) by an elementary column operation.
  By \cref{3.1} we know that there exists an elementary matrix \(E\) such that \(B = AE\).
  By \cref{2.3.2} we know that \(\tp{B} = \tp{E} \tp{A}\).
  By \cref{ex:3.1.5} we know that \(\tp{E}\) is an elementary operation.
  Thus by \cref{3.1} we know that \(\tp{B}\) can be obtained from \(\tp{A}\) by an elementary row operation.
\end{proof}

\setcounter{ex}{7}
\begin{ex}\label{ex:3.1.8}
  Prove that if a matrix \(Q\) can be obtained from a matrix \(P\) by an elementary row operation, then \(P\) can be obtained from \(Q\) by an elementary matrix of the same type.
\end{ex}

\begin{proof}[\pf{ex:3.1.8}]
  First suppose that \(Q\) is obtained from \(P\) by an elementary row operation.
  By \cref{3.1} we know that there exists an elementary matrix \(E\) such that \(Q = EP\).
  By \cref{3.2} we know that \(E^{-1}\) exists and \(E^{-1}\) is an elementary row operation of the same type.
  Thus we have \(P = E^{-1} E P = E^{-1} Q\).
  By \cref{3.1} this means \(P\) can be obtained from \(Q\) by an elementary row operation of the same type.

  Now suppose that \(Q\) is obtained from \(P\) by an elementary column operation.
  By \cref{3.1} we know that there exists an elementary matrix \(E\) such that \(Q = PE\).
  By \cref{3.2} we know that \(E^{-1}\) exists and \(E^{-1}\) is an elementary column operation of the same type.
  Thus we have \(P = P E E^{-1} = Q E^{-1}\).
  By \cref{3.1} this means \(P\) can be obtained from \(Q\) by an elementary column operation of the same type.
\end{proof}

\section{Normal and Self-Adjoint Operators}\label{sec:6.4}

\begin{lem}\label{6.4.1}
  Let \(\T\) be a linear operator on a finite-dimensional inner product space \(\V\) over \(\F\).
  If \(\T\) has an eigenvector, then so does \(\T^*\).
\end{lem}

\begin{proof}[\pf{6.4.1}]
  Suppose that \(v\) is an eigenvector of \(\T\) with corresponding eigenvalue \(\lambda\).
  Then for any \(x \in \V\),
  \begin{align*}
    0 & = \inn{\zv, x}                                 &  & \text{(by \cref{6.1}(c))}        \\
      & = \inn{(\T - \lambda \IT[\V])(v), x}           &  & \text{(by \cref{5.2.4})}         \\
      & = \inn{v, (\T - \lambda \IT[\V])^*(x)}         &  & \text{(by \cref{6.9})}           \\
      & = \inn{v, (\T^* - \conj{\lambda} \IT[\V])(x)}, &  & \text{(by \cref{6.11}(a)(b)(e))}
  \end{align*}
  and hence \(v\) is orthogonal to the range of \(\T^* - \conj{\lambda} \IT[\V]\).
  So \(\T^* - \conj{\lambda} \IT[\V]\) is not onto (\(v \notin \rg{\T^* - \conj{\lambda} \IT[\V]}\)) and hence is not one-to-one (by \cref{2.5}).
  Thus \(\T^* - \conj{\lambda} \IT[\V]\) has a nonzero null space, and any nonzero vector in this null space is an eigenvector of \(\T^*\) with corresponding eigenvalue \(\conj{\lambda}\).
\end{proof}

\begin{thm}[Schur's theorem]\label{6.14}
  Let \(\T\) be a linear operator on a finite-dimensional inner product space \(\V\) over \(\F\).
  Suppose that the characteristic polynomial of \(\T\) splits.
  Then there exists an orthonormal basis \(\beta\) for \(\V\) over \(\F\) such that the matrix \([\T]_{\beta}\) is upper triangular.
\end{thm}

\begin{proof}[\pf{6.14}]
  The proof is by mathematical induction on the dimension \(n\) of \(\V\).
  The result is immediate if \(n = 1\).
  So suppose that the result is true for linear operators on \(n\)-dimensional inner product spaces whose characteristic polynomials split.
  By \cref{6.4.1}, we can assume that \(\T^*\) has a unit eigenvector \(z\).
  Suppose that \(\T^*(z) = \lambda z\) and that \(\W = \spn{\set{z}}\).
  We show that \(\W^{\perp}\) is \(\T\)-invariant.
  If \(y \in \W^{\perp}\) and \(x = cz \in \W\), then
  \begin{align*}
    \inn{\T(y), x} & = \inn{\T(y), cz}                                              \\
                   & = \inn{y, \T^*(cz)}           &  & \text{(by \cref{6.9})}      \\
                   & = \inn{y, c \T^*(z)}          &  & \text{(by \cref{2.1.1}(b))} \\
                   & = \inn{y, c \lambda z}        &  & \text{(by \cref{5.1.2})}    \\
                   & = \conj{c \lambda} \inn{y, z} &  & \text{(by \cref{6.1}(b))}   \\
                   & = \conj{c \lambda} (0)        &  & \text{(by \cref{6.2.9})}    \\
                   & = 0.
  \end{align*}
  So \(\T(y) \in \W^{\perp}\).
  It is easy to show (see \cref{5.21}, or as a consequence of \cref{ex:4.4.6}) that the characteristic polynomial of \(\T_{\W^{\perp}}\) divides the characteristic polynomial of \(\T\) and hence splits.
  By \cref{6.7}(c), \(\dim(\W^{\perp}) = n\), so we may apply the induction hypothesis to \(\T_{\W^{\perp}}\) and obtain an orthonormal basis \(\gamma\) for \(\W^{\perp}\) over \(\F\) such that \([\T_{\W^{\perp}}]_{\gamma}\) is upper triangular.
  Clearly, \(\beta = \gamma \cup \set{z}\) is an orthonormal basis for \(\V\) over \(\F\) (by \cref{1.6.15,6.2.4}) such that \([\T]_{\beta}\) is upper triangular.
\end{proof}

\begin{cor}\label{6.4.2}
  Let \(\V\) be a finite-dimensional inner product space over \(\F\) and let \(\T \in \ls(\V)\).
  Suppose that there exists an orthonormal basis of eigenvectors of \(\T\).
  Then \(\T \T^* = \T^* \T\).
\end{cor}

\begin{proof}[\pf{6.4.2}]
  If such an orthonormal basis \(\beta\) exists, then \([\T]_{\beta}\) is a diagonal matrix, and hence \([\T^*]_{\beta} = [\T]_{\beta}^*\) is also a diagonal matrix.
  Because diagonal matrices commute, we conclude that \(\T\) and \(\T^*\) commute.
\end{proof}

\begin{defn}\label{6.4.3}
  Let \(\V\) be an inner product space over \(\F\), and let \(\T \in \ls(\V)\).
  We say that \(\T\) is \textbf{normal} if \(\T \T^* = \T^* \T\).
  An \(n \times n\) real or complex matrix \(A\) is \textbf{normal} if \(A A^* = A^* A\).
\end{defn}

\begin{cor}\label{6.4.4}
  Let \(\V\) be an inner product space over \(\F\), and let \(\T \in \ls(\V)\).
  \(\T\) is normal iff \([\T]_{\beta}\) is normal, where \(\beta\) is an orthonormal basis.
\end{cor}

\begin{proof}[\pf{6.4.4}]
  We have
  \begin{align*}
         & \T \text{ is normal}                                                                    \\
    \iff & \T \T^* = \T^* \T                                         &  & \text{(by \cref{6.4.3})} \\
    \iff & [\T]_{\beta} [\T]_{\beta}^* = [\T]_{\beta}^* [\T]_{\beta} &  & \text{(by \cref{6.10})}  \\
    \iff & [\T]_{\beta} \text{ is normal}.                           &  & \text{(by \cref{6.4.3})}
  \end{align*}
\end{proof}

\begin{eg}\label{6.4.5}
  Let \(\T : \R^2 \to \R^2\) be rotation by \(\theta\), where \(0 < \theta < \pi\).
  The matrix representation of \(\T\) in the standard ordered basis is given by
  \[
    A = \begin{pmatrix}
      \cos\theta & -\sin\theta \\
      \sin\theta & \cos\theta
    \end{pmatrix}.
  \]
  Note that \(A A^* = I_2 = A^* A\);
  so \(A\), and hence \(\T\), is normal.
\end{eg}

\begin{eg}\label{6.4.6}
  Suppose that \(A\) is a real skew-symmetric matrix;
  that is, \(\tp{A} = -A\).
  Then \(A\) is normal because both \(A \tp{A}\) and \(\tp{A} A\) are equal to \(-A^2\).
\end{eg}

\begin{note}
  Clearly, the operator \(\T\) in \cref{6.4.5} does not even possess one eigenvector.
  So in the case of a real inner product space, we see that normality is not sufficient to guarantee an orthonormal basis of eigenvectors.
  All is not lost, however.
  We show that normality suffices if \(\V\) is a complex inner product space.
\end{note}

\begin{thm}\label{6.15}
  Let \(\V\) be an inner product space over \(\F\), and let \(\T\) be a normal operator on \(\V\).
  Then the following statements are true.
  \begin{enumerate}
    \item \(\norm{\T(x)} = \norm{\T^*(x)}\) for all \(x \in \V\).
    \item \(\T - c \IT[\V]\) is normal for every \(c \in \F\).
    \item If \(x\) is an eigenvector of \(\T\), then \(x\) is also an eigenvector of \(\T^*\).
          In fact, if \(\T(x) = \lambda x\), then \(\T^*(x) = \conj{\lambda} x\).
    \item If \(\lambda_1\) and \(\lambda_2\) are distinct eigenvalues of \(\T\) with corresponding eigenvectors \(x_1\) and \(x_2\), then \(x_1\) and \(x_2\) are orthogonal.
  \end{enumerate}
\end{thm}

\begin{proof}[\pf{6.15}(a)]
  For any \(x \in \V\), we have
  \begin{align*}
    \norm{\T(x)}^2 & = \inn{\T(x), \T(x)}     &  & \text{(by \cref{6.1.9})} \\
                   & = \inn{\T^* \T(x), x}    &  & \text{(by \cref{6.9})}   \\
                   & = \inn{\T \T^*(x), x}    &  & \text{(by \cref{6.4.3})} \\
                   & = \inn{\T^*(x), \T^*(x)} &  & \text{(by \cref{6.9})}   \\
                   & = \norm{\T^*(x)}^2.      &  & \text{(by \cref{6.1.9})}
  \end{align*}
\end{proof}

\begin{proof}[\pf{6.15}(b)]
  We have
  \begin{align*}
    (\T - c \IT[\V]) (\T - c \IT[\V])^* & = (\T - c \IT[\V]) (\T^* - \conj{c} \IT[\V])                                  &  & \text{(by \cref{6.11}(a)(b)(e))} \\
                                        & = \T \T^* - \conj{c} \T \IT[\V] - c \IT[\V] \T^* + c \conj{c} \IT[\V] \IT[\V] &  & \text{(by \cref{2.10})}          \\
                                        & = \T^* \T - \conj{c} \IT[\V] \T - c \T^* \IT[\V] + c \conj{c} \IT[\V] \IT[\V] &  & \text{(by \cref{6.4.3})}         \\
                                        & = (\T^* - \conj{c} \IT[\V]) (\T - c \IT[\V])                                  &  & \text{(by \cref{2.10})}          \\
                                        & = (\T - c \IT[\V])^* (\T - c \IT[\V])                                         &  & \text{(by \cref{6.11}(a)(b)(e))}
  \end{align*}
  and thus by \cref{6.4.3} \(\T - c \IT[\V]\) is normal for all \(c \in \F\).
\end{proof}

\begin{proof}[\pf{6.15}(c)]
  Suppose that \(\T(x) = \lambda x\) for some \(x \in \V\).
  Let \(\U = \T - \lambda \IT[\V]\).
  Then \(\U(x) = \zv\), and \(\U\) is normal by (b).
  Thus (a) implies that
  \begin{align*}
    0 & = \norm{\U(x)}                              &  & \text{(by \cref{6.2}(b))}        \\
      & = \norm{\U^*(x)}                            &  & \text{(by \cref{6.15}(a))}       \\
      & = \norm{(\T^* - \conj{\lambda} \IT[\V])(x)} &  & \text{(by \cref{6.11}(a)(b)(e))} \\
      & = \norm{\T^*(x) - \conj{\lambda} x}.        &  & \text{(by \cref{2.2.5})}
  \end{align*}
  Hence by \cref{6.2}(b) \(\T^*(x) = \conj{\lambda} x\).
  So \(x\) is an eigenvector of \(\T^*\).
\end{proof}

\begin{proof}[\pf{6.15}(d)]
  Let \(\lambda_1\) and \(\lambda_2\) be distinct eigenvalues of \(\T\) with corresponding eigenvectors \(x_1\) and \(x_2\).
  Then, using (c), we have
  \begin{align*}
    \lambda_1 \inn{x_1, x_2} & = \inn{\lambda_1 x_1, x_2}        &  & \text{(by \cref{6.1.1}(b))} \\
                             & = \inn{\T(x_1), x_2}              &  & \text{(by \cref{5.1.2})}    \\
                             & = \inn{x_1, \T^*(x_2)}            &  & \text{(by \cref{6.9})}      \\
                             & = \inn{x_1, \conj{\lambda_2} x_2} &  & \text{(by \cref{6.15}(c))}  \\
                             & = \lambda_2 \inn{x_1, x_2}.       &  & \text{(by \cref{6.1}(b))}
  \end{align*}
  Since \(\lambda_1 \neq \lambda_2\), we conclude that \(\inn{x_1, x_2} = 0\).
\end{proof}

\begin{thm}\label{6.16}
  Let \(\T\) be a linear operator on a finite-dimensional complex inner product space \(\V\).
  Then \(\T\) is normal iff there exists an orthonormal basis for \(\V\) consisting of eigenvectors of \(\T\).
\end{thm}

\begin{proof}[\pf{6.16}]
  Suppose that \(\T\) is normal.
  By the fundamental theorem of algebra (\cref{d.4}), the characteristic polynomial of \(\T\) splits.
  So we may apply Schur's theorem (\cref{6.14}) to obtain an orthonormal basis \(\beta = \set{\seq{v}{1,,n}}\) for \(\V\) over \(\F\) such that \([\T]_{\beta} = A\) is upper triangular.
  We know that \(v_1\) is an eigenvector of \(\T\) because \(A\) is upper triangular.
  Assume that \(\seq{v}{1,,k-1}\) are eigenvectors of \(\T\).
  We claim that \(v_k\) is also an eigenvector of \(\T\).
  It then follows by mathematical induction on \(k\) that all of the \(v_i\)'s are eigenvectors of \(\T\).
  Consider any \(j < k\), and let \(\lambda_j\) denote the eigenvalue of \(\T\) corresponding to \(v_j\).
  By \cref{6.15}(c), \(\T^*(v_j) = \conj{\lambda_j} v_j\).
  Since \(A\) is upper triangular, by \cref{2.2.4} we have
  \[
    \T(v_k) = A_{1 k} v_1 + \cdots + A_{j k} v_j + \cdots + A_{k k} v_k.
  \]
  Furthermore,
  \begin{align*}
    A_{j k} & = \inn{\T(v_k), v_j}              &  & \text{(by \cref{6.2.6})}   \\
            & = \inn{v_k, \T^*(v_j)}            &  & \text{(by \cref{6.9})}     \\
            & = \inn{v_k, \conj{\lambda_j} v_j} &  & \text{(by \cref{6.15}(c))} \\
            & = \lambda_j \inn{v_k, v_j}        &  & \text{(by \cref{6.1}(b))}  \\
            & = 0.                              &  & \text{(by \cref{6.1.12})}
  \end{align*}
  It follows that \(\T(v_k) = A_{k k} v_k\), and hence \(v_k\) is an eigenvector of \(\T\).
  So by induction, all the vectors in \(\beta\) are eigenvectors of \(\T\).

  The converse was already proved by \cref{6.4.2}.
\end{proof}

\begin{note}
  Interestingly, as \cref{6.4.7} shows, \cref{6.16} does not extend to infinite-dimensional complex inner product spaces.
\end{note}

\begin{eg}\label{6.4.7}
  Consider the inner product space \(\vs{H}\) with the orthonormal set \(S\) from \cref{6.1.13}.
  Let \(\V = \spn{S}\), and let \(\T, \U \in \ls(\V)\) defined by \(\T(f) = f_1 f\) and \(\U(f) = f_{-1} f\).
  Then
  \[
    \T(f_n) = f_{n + 1} \quad \text{and} \quad \U(f_n) = f_{n - 1}
  \]
  for all integers \(n\).
  Thus
  \begin{align*}
    \inn{\T(f_m), f_n} & = \inn{f_{m + 1}, f_n}                                \\
                       & = \delta_{(m + 1), n}  &  & \text{(by \cref{6.1.13})} \\
                       & = \delta_{m, (n - 1)}                                 \\
                       & = \inn{f_m, f_{n - 1}} &  & \text{(by \cref{6.1.13})} \\
                       & = \inn{f_m, \U(f_n)}.
  \end{align*}
  It follows that \(\U = \T^*\).
  Furthermore, \(\T \T^* = \IT[\V] = \T^* \T\);
  so \(\T\) is normal.

  We show that \(\T\) has no eigenvectors.
  Suppose that \(f\) is an eigenvector of \(\T\), say, \(\T(f) = \lambda f\) for some \(\lambda\).
  Since \(\V\) equals the span of \(S\), we may write
  \[
    f = \sum_{i = n}^m a_i f_i, \quad \text{where } a_m \neq 0.
  \]
  Hence
  \[
    \sum_{i = n}^m a_i f_{i + 1} = \T(f) = \lambda f = \sum_{i = n}^m \lambda a_i f_i.
  \]
  Since \(a_m \neq 0\), we can write \(f_{m + 1}\) as a linear combination of \(\seq{f}{n,n+1,,m}\).
  But this is a contradiction because \(S\) is linearly independent.
\end{eg}

\begin{note}
  \cref{6.4.5} illustrates that normality is not sufficient to guarantee the existence of an orthonormal basis of eigenvectors for real inner product spaces.
  For real inner product spaces, we must replace normality by the stronger condition that \(\T = \T^*\) in order to guarantee such a basis.
\end{note}

\begin{defn}\label{6.4.8}
  Let \(\T\) be a linear operator on an inner product space \(\V\) over \(\F\).
  We say that \(\T\) is \textbf{self-adjoint} (\textbf{Hermitian}) if \(\T = \T^*\).
  An \(n \times n\) real or complex matrix \(A\) is \textbf{self-adjoint} (\textbf{Hermitian}) if \(A = A^*\).
\end{defn}

\begin{cor}\label{6.4.9}
  If \(\beta\) is an orthonormal basis, then \(\T\) is self-adjoint iff \([\T]_{\beta}\) is self-adjoint.
  For real matrices, this condition reduces to the requirement that \(A\) be symmetric.
\end{cor}

\begin{proof}[\pf{6.4.9}]
  We have
  \begin{align*}
         & \T \text{ is self-adjoint}                                                   \\
    \iff & \T = \T^*                                      &  & \text{(by \cref{6.4.8})} \\
    \iff & [\T]_{\beta} = [\T^*]_{\beta} = [\T]_{\beta}^* &  & \text{(by \cref{6.10})}  \\
    \iff & [\T]_{\beta} \text{ is self-adjoint}.          &  & \text{(by \cref{6.4.8})}
  \end{align*}
\end{proof}

\begin{lem}\label{6.4.10}
  Let \(\T\) be a self-adjoint operator on a finite-dimensional inner product space \(\V\) over \(\F\).
  Then
  \begin{enumerate}
    \item Every eigenvalue of \(\T\) is real.
    \item Suppose that \(\V\) is a real inner product space.
          Then the characteristic polynomial of \(\T\) splits.
  \end{enumerate}
\end{lem}

\begin{proof}[\pf{6.4.10}(a)]
  Suppose that \(\T(x) = \lambda x\) for \(x \neq \zv\).
  Because a self-adjoint operator is also normal, we have
  \begin{align*}
    \lambda x & = \T(x)             &  & \text{(by \cref{5.1.2})}   \\
              & = \T^*(x)           &  & \text{(by \cref{6.4.8})}   \\
              & = \conj{\lambda} x. &  & \text{(by \cref{6.15}(c))}
  \end{align*}
  So \(\lambda = \conj{\lambda}\);
  that is, \(\lambda\) is real.
\end{proof}

\begin{proof}[\pf{6.4.10}(b)]
  Let \(n = \dim(\V)\), \(\beta\) be an orthonormal basis for \(\V\) over \(\R\), and \(A = [\T]_{\beta}\).
  Then \(A\) is self-adjoint by \cref{6.4.9}.
  Let \(\T_A\) be the linear operator on \(\C^n\) defined by \(\T_A(x) = Ax\) for all \(x \in \C^n\).
  Note that \(\T_A\) is self-adjoint because \([\T_A]_{\gamma} = A\), where \(\gamma\) is the standard ordered (orthonormal) basis for \(\C^n\) over \(\C\) (see \cref{2.15}(a)).
  So, by (a), the eigenvalues of \(\T_A\) are real.
  By the fundamental theorem of algebra (\cref{d.4}), the characteristic polynomial of \(\T_A\) splits into factors of the form \(t - \lambda\).
  Since each \(\lambda\) is real, the characteristic polynomial splits over \(\R\).
  But \(\T_A\) has the same characteristic polynomial as \(A\), which has the same characteristic polynomial as \(\T\) (see \cref{5.1.6}).
  Therefore the characteristic polynomial of \(\T\) splits.
\end{proof}

\begin{thm}\label{6.17}
  Let \(\T\) be a linear operator on a finite-dimensional real inner product space \(\V\).
  Then \(\T\) is self-adjoint iff there exists an orthonormal basis \(\beta\) for \(\V\) over \(\R\) consisting of eigenvectors of \(\T\).
\end{thm}

\begin{proof}[\pf{6.17}]
  Suppose that \(\T\) is self-adjoint.
  By \cref{6.4.10}(b), we may apply Schur's theorem (\cref{6.14}) to obtain an orthonormal basis \(\beta\) for \(\V\) over \(\R\) such that the matrix \(A = [\T]_{\beta}\) is upper triangular.
  But
  \begin{align*}
    A^* & = [\T]_{\beta}^*                               \\
        & = [\T^*]_{\beta} &  & \text{(by \cref{6.10})}  \\
        & = [\T]_{\beta}   &  & \text{(by \cref{6.4.8})} \\
        & = A.
  \end{align*}
  So \(A\) and \(A^*\) are both upper triangular, and therefore \(A\) is a diagonal matrix.
  Thus \(\beta\) must consist of eigenvectors of \(\T\).

  Now suppose that there exists an orthonormal basis \(\beta\) for \(\V\) over \(\R\) consisting of eigenvectors of \(\T\).
  By \cref{5.1.1} we know that \([\T]_{\beta}\) is a diagonal matrix.
  Since \(\V\) is over \(\R\), we have
  \begin{align*}
    [\T^*]_{\beta} & = [\T]_{\beta}^*      &  & \text{(by \cref{6.10})}                        \\
                   & = \tp{([\T]_{\beta})} &  & \text{(\(\V\) is over \(\R\))}                 \\
                   & = [\T]_{\beta}.       &  & \text{(\([\T]_{\beta}\) is a diagonal matrix)}
  \end{align*}
  Thus by \cref{6.4.8} \([\T]_{\beta}\) is self-adjoint and by \cref{6.4.9} \(\T\) is self-adjoint.
\end{proof}

\begin{note}
  \cref{6.17} is used extensively in many areas of mathematics and statistics.
  We restate this theorem in matrix form in \cref{sec:6.5}.
\end{note}

\exercisesection

\setcounter{ex}{3}
\begin{ex}\label{ex:6.4.4}
  Let \(\T\) and \(\U\) be self-adjoint operators on an inner product space \(\V\) over \(\F\).
  Prove that \(\T \U\) is self-adjoint iff \(\T \U = \U \T\).
\end{ex}

\begin{proof}[\pf{ex:6.4.4}]
  We have
  \begin{align*}
         & \T \U \text{ is self-adjoint}                                 \\
    \iff & \T \U = (\T \U)^*             &  & \text{(by \cref{6.4.8})}   \\
         & = \U^* \T^*                   &  & \text{(by \cref{6.11}(c))} \\
         & = \U \T.                      &  & \text{(by \cref{6.4.8})}
  \end{align*}
\end{proof}

\setcounter{ex}{5}
\begin{ex}\label{ex:6.4.6}
  Let \(\V\) be a complex inner product space, and let \(\T \in \ls(\V)\).
  Define
  \[
    \T_1 = \frac{1}{2} (\T + \T^*) \quad \text{and} \quad \T_2 = \frac{1}{2i} (\T - \T^*).
  \]
  \begin{enumerate}
    \item Prove that \(\T_1\) and \(\T_2\) are self-adjoint and that \(\T = \T_1 + i \T_2\).
    \item Suppose also that \(\T = \U_1 + i \U_2\), where \(\U_1\) and \(\U_2\) are self-adjoint.
          Prove that \(\U_1 = \T_1\) and \(\U_2 = \T_2\).
    \item Prove that \(\T\) is normal iff \(\T_1 \T_2 =  \T_2 \T_1\).
  \end{enumerate}
\end{ex}

\begin{proof}[\pf{ex:6.4.6}(a)]
  We have
  \begin{align*}
    \T_1^* & = \pa{\frac{1}{2} (\T + \T^*)}^*                                        \\
           & = \frac{1}{2} (\T^* + \T)         &  & \text{(by \cref{6.11}(a)(b)(d))} \\
           & = \T_1                                                                  \\
    \T_2^* & = \pa{\frac{1}{2i} (\T - \T^*)}^*                                       \\
           & = \frac{-1}{2i} (\T^* - \T)       &  & \text{(by \cref{6.11}(a)(b)(d))} \\
           & = \frac{1}{2i} (\T - \T^*)                                              \\
           & = \T_2.
  \end{align*}
  Thus by \cref{6.4.8} \(\T_1, \T_2\) are self-adjoint.
  By definition we have
  \begin{align*}
    \T_1 + i \T_2 & = \frac{1}{2} (\T + \T^*) + \frac{i}{2i} (\T - \T^*) \\
                  & = \T.
  \end{align*}
\end{proof}

\begin{proof}[\pf{ex:6.4.6}(b)]
  We have
  \begin{align*}
             & \T = \begin{dcases}
                      \T_1 + i \T_2 \\
                      \U_1 + i \U_2
                    \end{dcases}                      &  & \text{(by \cref{ex:6.4.6}(a))} \\
    \implies & \T^* = \begin{dcases}
                        \T_1^* - i \T_2^* \\
                        \U_1^* - i \U_2^*
                      \end{dcases}                    &  & \text{(by \cref{6.11}(a)(b))}  \\
             & = \begin{dcases}
                   \T_1 - i \T_2 \\
                   \U_1 - i \U_2
                 \end{dcases}                         &  & \text{(by \cref{6.4.8})}       \\
    \implies & \begin{dcases}
                 \T_1 = \frac{1}{2} (\T + \T^*) = \U_1 \\
                 \T_2 = \frac{1}{2i} (\T - \T^*) = \U_2
               \end{dcases}.
  \end{align*}
\end{proof}

\begin{proof}[\pf{ex:6.4.6}(c)]
  We have
  \begin{align*}
         & \T \text{ is normal}                                                                          \\
    \iff & \T \T^* = \T^* \T                                               &  & \text{(by \cref{6.4.3})} \\
    \iff & (\T + \T^*) (\T - \T^*) = \T^2 - \T \T^* + \T^* \T - (\T^*)^2                                 \\
         & = \T^2 + \T \T^* - \T^* \T - (\T^*)^2 = (\T - \T^*) (\T + \T^*) &  & \text{(by \cref{2.10})}  \\
    \iff & \T_1 \T_2 = \frac{1}{4i} (\T + \T^*) (\T - \T^*)                                              \\
         & = \frac{1}{4i} (\T - \T^*) (\T + \T^*) = \T_2 \T_1.
  \end{align*}
\end{proof}

\begin{ex}\label{ex:6.4.7}
  Let \(\T\) be a linear operator on an inner product space \(\V\) over \(\F\), and let \(\W\) be a \(\T\)-invariant subspace of \(\V\) over \(\F\).
  Prove the following results.
  \begin{enumerate}
    \item If \(\T\) is self-adjoint, then \(\T_{\W}\) is self-adjoint.
    \item \(\W^{\perp}\) is \(\T^*\)-invariant.
    \item If \(\W\) is both \(\T\)- and \(\T^*\)-invariant, then \((\T_{\W})^* = (\T^*)_{\W}\).
    \item If \(\W\) is both \(\T\)- and \(\T^*\)-invariant and \(\T\) is normal, then \(\T_{\W}\) is normal.
  \end{enumerate}
\end{ex}

\begin{proof}[\pf{ex:6.4.7}(a)]
  We have
  \begin{align*}
    \forall x, y \in \W, \inn{x, (\T_{\W})^*(y)} & = \inn{\T_{\W}(x), y} &  & \text{(by \cref{6.9})}   \\
                                                 & = \inn{\T(x), y}      &  & \text{(by \cref{5.4.1})} \\
                                                 & = \inn{x, \T^*(y)}    &  & \text{(by \cref{6.9})}   \\
                                                 & = \inn{x, \T(y)}      &  & \text{(by \cref{6.4.8})} \\
                                                 & = \inn{x, \T_{\W}(y)} &  & \text{(by \cref{5.4.1})}
  \end{align*}
  and thus by \cref{6.1}(e) \(\T_{\W} = (\T_{\W})^*\).
  By \cref{6.4.8} this means \(\T_{\W}\) is self-adjoint.
\end{proof}

\begin{proof}[\pf{ex:6.4.7}(b)]
  Since
  \begin{align*}
             & v \in \W^{\perp}                                                     \\
    \implies & \forall x \in \W, \inn{v, x} = 0       &  & \text{(by \cref{6.2.9})} \\
    \implies & \forall x \in \W, \inn{v, \T(x)} = 0   &  & \text{(by \cref{5.4.1})} \\
    \implies & \forall x \in \W, \inn{\T^*(v), x} = 0 &  & \text{(by \cref{6.9})}   \\
    \implies & \T^*(v) \in \W^{\perp},                &  & \text{(by \cref{6.2.9})}
  \end{align*}
  by \cref{5.4.1} we see that \(\W^{\perp}\) is \(\T^*\)-invariant.
\end{proof}

\begin{proof}[\pf{ex:6.4.7}(c)]
  Since
  \begin{align*}
    \forall x, y \in \W, \inn{x, (\T^*)_{\W}(y)} & = \inn{x, \T^*(y)}         &  & \text{(by \cref{5.4.1})} \\
                                                 & = \inn{\T(x), y}           &  & \text{(by \cref{6.9})}   \\
                                                 & = \inn{\T_{\W}(x), y}      &  & \text{(by \cref{5.4.1})} \\
                                                 & = \inn{x, (\T_{\W})^*(y)}, &  & \text{(by \cref{6.9})}
  \end{align*}
  by \cref{6.1}(e) we see that \((\T^*)_{\W} = (\T_{\W})^*\).
\end{proof}

\begin{proof}[\pf{ex:6.4.7}(d)]
  We have
  \begin{align*}
    \forall x \in \W, (\T_{\W} (\T_{\W})^*)(x) & = (\T_{\W} (\T^*)_{\W})(x) &  & \text{(by \cref{ex:6.4.7}(c))} \\
                                               & = (\T \T^*)(x)             &  & \text{(by \cref{5.4.1})}       \\
                                               & = (\T^* \T)(x)             &  & \text{(by \cref{6.4.5})}       \\
                                               & = ((\T^*)_{\W} \T_{\W})(x) &  & \text{(by \cref{5.4.1})}       \\
                                               & = ((\T_{\W})^* \T_{\W})(x) &  & \text{(by \cref{ex:6.4.7}(c))}
  \end{align*}
  and thus \(\T_{\W} (\T_{\W})^* = (\T_{\W})^* \T_{\W}\).
  By \cref{6.4.5} this means \(\T_{\W}\) is normal.
\end{proof}

\begin{ex}\label{ex:6.4.8}
  Let \(\T\) be a normal operator on a finite-dimensional complex inner product space \(\V\), and let \(\W\) be a subspace of \(\V\) over \(\C\).
  Prove that if \(\W\) is \(\T\)-invariant, then \(\W\) is also \(\T^*\)-invariant.
\end{ex}

\begin{proof}[\pf{ex:6.4.8}]
  If \(\W = \set{\zv}\), then by \cref{2.1.2}(a) we have \(\T^*(\set{\zv}) = \set{\zv}\) (this is true since \(\T^* \in \ls(\V)\) by \cref{6.9}).
  Thus \(\W\) is \(\T^*\)-invariant in this case.

  Now suppose that \(\W \neq \set{\zv}\).
  Since \(\T\) is normal and \(\V\) is over \(\C\), by \cref{6.16} there exists an orthonormal basis \(\beta\) for \(\V\) over \(\C\) consisting of eigenvectors of \(\T\).
  By \cref{ex:5.4.24} this means \(\T_{\W}\) is diagonalizable.
  If we let \(\beta_{\W} = \beta \cap \W\), then by \cref{ex:5.4.24} we know that \(\beta_{\W}\) is an orthonormal basis for \(\W\) over \(\C\) consisting of eigenvectors of \(\T_{\W}\).
  By \cref{6.15}(c) we know that \(\beta_{\W}\) is a set of eigenvectors of \(\T^*\).
  Thus by \cref{5.4.2}(e) we see that \(\W\) is \(\T^*\)-invariant.
\end{proof}

\begin{ex}\label{ex:6.4.9}
  Let \(\T\) be a normal operator on a finite-dimensional inner product space \(\V\) over \(\F\).
  Prove that \(\ns{\T} = \ns{\T^*}\) and \(\rg{\T} = \rg{\T^*}\).
\end{ex}

\begin{proof}[\pf{ex:6.4.9}]
  Since
  \begin{align*}
         & x \in \ns{\T}                                      \\
    \iff & \T(x) = \zv        &  & \text{(by \cref{2.1.10})}  \\
    \iff & \norm{\T(x)} = 0   &  & \text{(by \cref{6.2}(b))}  \\
    \iff & \norm{\T^*(x)} = 0 &  & \text{(by \cref{6.15}(a))} \\
    \iff & \T^*(x) = \zv      &  & \text{(by \cref{6.2}(b))}  \\
    \iff & x \in \ns{\T^*},   &  & \text{(by \cref{2.1.10})}
  \end{align*}
  we have \(\ns{\T} = \ns{\T^*}\).
  Thus
  \begin{align*}
    \rg{\T^*} & = \ns{\T}^{\perp}   &  & \text{(by \cref{ex:6.3.12}(b))} \\
              & = \ns{\T^*}^{\perp} &  & \text{(from the proof above)}   \\
              & = \rg{\T^{**}}      &  & \text{(by \cref{ex:6.3.12}(b))} \\
              & = \rg{\T}.          &  & \text{(by \cref{6.11}(d))}
  \end{align*}
\end{proof}

\begin{ex}\label{ex:6.4.10}
  Let \(\T\) be a self-adjoint operator on a finite-dimensional inner product space \(\V\) over \(\F\).
  Prove that for all \(x \in \V\)
  \[
    \norm{\T(x) \pm ix}^2 = \norm{\T(x)}^2 + \norm{x}^2.
  \]
  Deduce that \(\T - i \IT[\V]\) is invertible and that \(\pa{(\T - i \IT[\V])^{-1}}^* = (\T + i \IT[\V])^{-1}\).
\end{ex}

\begin{proof}[\pf{ex:6.4.10}]
  Since
  \begin{align*}
    \inn{\T(x), x} & = \inn{\T^*(x), x}       &  & \text{(by \cref{6.4.8})}    \\
                   & = \inn{x, \T(x)}         &  & \text{(by \cref{6.9})}      \\
                   & = \conj{\inn{\T(x), x}}, &  & \text{(by \cref{6.1.1}(c))}
  \end{align*}
  we know that \(\inn{\T(x), x} \in \R\) for all \(x \in \V\).
  Thus
  \begin{align*}
    \norm{\T(x) \pm ix}^2 & = \norm{\T(x)}^2 \pm 2 \Re(\inn{\T(x), ix}) + \norm{ix}^2   &  & \text{(by \cref{ex:6.1.19}(a))} \\
                          & = \norm{\T(x)}^2 \pm 2 \Re(-i \inn{\T(x), x}) + \norm{ix}^2 &  & \text{(by \cref{6.1}(b))}       \\
                          & = \norm{\T(x)}^2 + \norm{ix}^2                              &  & \text{(from the proof above)}   \\
                          & = \norm{\T(x)}^2 + \norm{x}^2.                              &  & \text{(by \cref{6.2}(a))}
  \end{align*}

  Next we show that \(\T - i \IT[\V]\) is invertible.
  Observe that
  \begin{align*}
             & x \in \ns{\T - i \IT[\V]}                                          \\
    \implies & \T(x) - ix = \zv                &  & \text{(by \cref{2.1.10})}     \\
    \implies & \norm{\T(x) - ix} = 0           &  & \text{(by \cref{6.2}(b))}     \\
    \implies & \norm{\T(x)}^2 + \norm{x}^2 = 0 &  & \text{(from the proof above)} \\
    \implies & \norm{\T(x)}^2 = \norm{x}^2 = 0 &  & \text{(by \cref{6.2}(b))}     \\
    \implies & x = 0.                          &  & \text{(by \cref{6.2}(b))}
  \end{align*}
  Thus by \cref{2.5} \(\T - i \IT[\V]\) is invertible.

  Finally we show that \(\pa{(\T - i \IT[\V])^{-1}}^* = (\T + i \IT[\V])^{-1}\).
  Since
  \begin{align*}
    \forall x, y \in \V, & \inn{x, \pa{\pa{(\T - i \IT[\V])^{-1}}^* (\T + i \IT[\V])}(y)}                                       \\
                         & = \inn{(\T - i \IT[\V])^{-1}(x), (\T + i \IT[\V])(y)}          &  & \text{(by \cref{6.9})}           \\
                         & = \inn{(\T - i \IT[\V])^{-1}(x), (\T^* + i \IT[\V])(y)}        &  & \text{(by \cref{6.4.8})}         \\
                         & = \inn{(\T - i \IT[\V])^{-1}(x), (\T - i \IT[\V])^*(y)}        &  & \text{(by \cref{6.11}(a)(b)(e))} \\
                         & = \inn{\pa{(\T - i \IT[\V]) (\T - i \IT[\V])^{-1}}(x), y}      &  & \text{(by \cref{6.9})}           \\
                         & = \inn{x, y}                                                                                         \\
                         & = \inn{x, \IT[\V](y)},
  \end{align*}
  by \cref{6.1}(e) we have \(\pa{(\T - i \IT[\V])^{-1}}^* (\T + i \IT[\V]) = \IT[\V]\).
  Thus \(\pa{(\T - i \IT[\V])^{-1}}^* = (\T + i \IT[\V])^{-1}\).
\end{proof}

\begin{ex}\label{ex:6.4.11}
  Assume that \(\T\) is a linear operator on a complex (not necessarily finite-dimensional) inner product space \(\V\) with an adjoint \(\T^*\).
  Prove the following results.
  \begin{enumerate}
    \item If \(\T\) is self-adjoint, then \(\inn{\T(x), x}\) is real for all \(x \in \V\).
    \item If \(\T\) satisfies \(\inn{\T(x), x} = 0\) for all \(x \in \V\), then \(\T = \zT\).
    \item If \(\inn{\T(x), x}\) is real for all \(x \in \V\), then \(\T = \T^*\).
  \end{enumerate}
\end{ex}

\begin{proof}[\pf{ex:6.4.11}(a)]
  Since
  \begin{align*}
    \forall x \in \V, \inn{\T(x), x} & = \inn{\T^*(x), x}       &  & \text{(by \cref{6.4.8})}    \\
                                     & = \inn{x, \T(x)}         &  & \text{(by \cref{6.9})}      \\
                                     & = \conj{\inn{\T(x), x}}, &  & \text{(by \cref{6.1.1}(c))}
  \end{align*}
  we see that \(\inn{\T(x), x} \in \R\) for all \(x \in \V\).
\end{proof}

\begin{proof}[\pf{ex:6.4.11}(b)]
  Let \(x, y \in \V\).
  Since
  \begin{align*}
    0 & = \inn{\T(x + y), x + y}                                                                             \\
      & = \inn{\T(x) + \T(y), x + y}                                        &  & \text{(by \cref{2.1.1}(a))} \\
      & = \inn{\T(x), x + y} + \inn{\T(y), x + y}                           &  & \text{(by \cref{6.1.1}(a))} \\
      & = \inn{\T(x), x} + \inn{\T(x), y} + \inn{\T(y), x} + \inn{\T(y), y} &  & \text{(by \cref{6.1}(a))}   \\
      & = \inn{\T(x), y} + \inn{\T(y), x},
  \end{align*}
  we have \(\inn{\T(x), y} = - \inn{\T(y), x}\).
  Since
  \begin{align*}
    0 & = \inn{\T(x + iy), x + iy}                                                                                  \\
      & = \inn{\T(x) + i \T(y), x + iy}                                         &  & \text{(by \cref{2.1.2}(b))}    \\
      & = \inn{\T(x), x + iy} + i \inn{\T(y), x + iy}                           &  & \text{(by \cref{6.1.1}(a)(b))} \\
      & = \inn{\T(x), x} - i \inn{\T(x), y} + i \inn{\T(y), x} + \inn{\T(y), y} &  & \text{(by \cref{6.1}(a)(b))}   \\
      & = -i \inn{\T(x), y} + i \inn{\T(y), x},
  \end{align*}
  we have \(\inn{\T(x), y} = \inn{\T(y), x}\).
  Thus
  \begin{align*}
             & \forall x, y \in \V, \begin{dcases}
                                      \inn{\T(x), y} = - \inn{\T(y), x} \\
                                      \inn{\T(x), y} = \inn{\T(y), x}
                                    \end{dcases}                                  \\
    \implies & \forall x, y \in \V, \inn{\T(x), y} = \inn{\T(y), x} = 0                                \\
    \implies & \T = \zT.                                                &  & \text{(by \cref{6.1}(e))}
  \end{align*}
\end{proof}

\begin{proof}[\pf{ex:6.4.11}(c)]
  We have
  \begin{align*}
    \forall x \in \V, \inn{x, \T(x)} & = \conj{\inn{\T(x), x}} &  & \text{(by \cref{6.1.1}(c))} \\
                                     & = \inn{\T(x), x}        &  & (\inn{\T(x), x} \in \R)     \\
                                     & = \inn{x, \T^*(x)}      &  & \text{(by \cref{6.9})}
  \end{align*}
  and
  \begin{align*}
    \forall x \in \V, 0 & = \inn{x, \T(x)} - \inn{x, \T^*(x)} &  & \text{(from the proof above)} \\
                        & = \inn{x, \T(x) - \T^*(x)}          &  & \text{(by \cref{6.1}(a)(b))}  \\
                        & = \inn{x, (\T - \T^*)(x)}           &  & \text{(by \cref{2.10})}       \\
                        & = \inn{(\T - \T^*)(x), x}.          &  & \text{(by \cref{6.1.1}(c))}
  \end{align*}
  Thus
  \begin{align*}
             & \T - \T^* = \zT &  & \text{(by \cref{ex:6.4.11}(b))} \\
    \implies & \T = \T^*.
  \end{align*}
\end{proof}

\begin{ex}\label{ex:6.4.12}
  Let \(\T\) be a normal operator on a finite-dimensional real inner product space \(\V\) whose characteristic polynomial splits.
  Prove that \(\V\) has an orthonormal basis of eigenvectors of \(\T\).
  Hence prove that \(\T\) is self-adjoint.
\end{ex}

\begin{proof}[\pf{ex:6.4.12}]
  By \cref{6.14} we know that there exists an orthonormal basis \(\beta = \set{\seq{v}{1,,n}}\) for \(\V\) over \(\F\) such that \([\T]_{\beta}\) is upper triangular.
  We claim that for all \(k \in \set{1, \dots, n}\), the set \(\set{\seq{v}{1,,k}} \subseteq \beta\) is consist of eigenvectors of \(\T\).
  We use induction on \(k\) to proof the claim.
  For \(k = 1\), since \([\T]_{\beta}\) is upper triangular, by \cref{2.2.4} this means \(v_1\) is an eigenvector of \(\T\).
  Thus the base case holds.
  Suppose inductively that \(\set{\seq{v}{1,,k}}\) is consist of eigenvectors of \(\T\) for some \(k \geq 1\).
  We show that the claim is true for \(k + 1\).
  By induction hypothesis we know that \(\seq{v}{1,,k}\) are eigenvectors of \(\T\) corresponding to eigenvalues \(\seq{\lambda}{1,,k} \in \R\).
  Since \(\T\) is normal, by \cref{6.15}(c) we know that \(\T^*(v_i) = \conj{\lambda_i} v_i\) for all \(i \in \set{1, \dots, k}\).
  Then we have
  \begin{align*}
    \T(v_{k + 1}) & = \sum_{j = 1}^n \inn{\T(v_{k + 1}), v_j} v_j                                                           &  & \text{(by \cref{6.2.6})}         \\
                  & = \sum_{j = 1}^{k + 1} \inn{\T(v_{k + 1}), v_j} v_j                                                     &  & \text{(by \cref{6.14})}          \\
                  & = \sum_{j = 1}^{k + 1} \inn{v_{k + 1}, \T^*(v_j)} v_j                                                   &  & \text{(by \cref{6.9})}           \\
                  & = \inn{v_{k + 1}, \T^*(v_{k + 1})} v_{k + 1} + \sum_{j = 1}^k \inn{v_{k + 1}, \T^*(v_j)} v_j                                                  \\
                  & = \inn{v_{k + 1}, \T^*(v_{k + 1})} v_{k + 1} + \sum_{j = 1}^k \inn{v_{k + 1}, \conj{\lambda_j} v_j} v_j &  & \text{(by induction hypothesis)} \\
                  & = \inn{v_{k + 1}, \T^*(v_{k + 1})} v_{k + 1} + \sum_{j = 1}^k \lambda_j \inn{v_{k + 1}, v_j} v_j        &  & \text{(by \cref{6.1}(b))}        \\
                  & = \inn{v_{k + 1}, \T^*(v_{k + 1})} v_{k + 1}                                                            &  & \text{(by \cref{6.1.12})}
  \end{align*}
  and thus \(v_{k + 1}\) is an eigenvector of \(\T\).
  This closes the induction.
  Hence \(\beta\) is consist of eigenvectors of \(\T\).
  By \cref{5.1.1} this means \([\T]_{\beta}\) is a diagonal matrix.
  Thus \([\T]_{\beta}^* = [\T]_{\beta}\) and by \cref{6.4.9} \(\T\) is self-adjoint.
\end{proof}

\begin{ex}\label{ex:6.4.13}
  An \(A \in \ms{n}{n}{\R}\) is said to be a \textbf{Gramian} matrix if there exists a \(B \in \ms{n}{n}{\R}\) such that \(A = \tp{B} B\).
  Prove that \(A\) is a Gramian matrix iff \(A\) is symmetric and all of its eigenvalues are nonnegative.
\end{ex}

\begin{proof}[\pf{ex:6.4.13}]
  First suppose that \(A\) is Gramian.
  By \cref{ex:6.4.13} there exists a \(B \in \ms{n}{n}{\R}\) such that \(A = \tp{B} B\).
  Then we have
  \begin{align*}
    \tp{A} & = \tp{(\tp{B} B)}                                       \\
           & = \tp{B} \tp{(\tp{B})} &  & \text{(by \cref{2.3.2})}    \\
           & = \tp{B} B             &  & \text{(by \cref{ex:1.3.4})} \\
           & = A.
  \end{align*}
  Thus by \cref{1.3.4} \(A\) is symmetric.
  If \(\lambda\) is an eigenvalue of \(A\) and \(x\) is an eigenvector corresponding to \(\lambda\) with unit length, then we have
  \begin{align*}
    \lambda & = \lambda \norm{x}^2  &  & \text{(by \cref{6.1.12})}   \\
            & = \lambda \inn{x, x}  &  & \text{(by \cref{6.1.9})}    \\
            & = \inn{\lambda x, x}  &  & \text{(by \cref{6.1.1}(b))} \\
            & = \inn{Ax, x}         &  & \text{(by \cref{5.1.2})}    \\
            & = \inn{\tp{B} B x, x}                                  \\
            & = \inn{B^* B x, x}    &  & \text{(by \cref{6.1.5})}    \\
            & = \inn{Bx, Bx}        &  & \text{(by \cref{6.9})}      \\
            & = \norm{Bx}^2         &  & \text{(by \cref{6.1.9})}    \\
            & \geq 0.               &  & \text{(by \cref{6.2}(b))}
  \end{align*}
  Thus all eigenvalues of \(A\) are nonnegative.

  Now suppose that \(A\) is symmetric and all eigenvalues of \(A\) are nonnegative.
  Since
  \begin{align*}
    A & = \tp{A} &  & \text{(by \cref{1.3.4})} \\
      & = A^*,   &  & \text{(by \cref{6.1.5})}
  \end{align*}
  by \cref{6.4.8} we know that \(A\) is self-adjoint.
  By \cref{6.17} there exists an orthonormal basis \(\beta = \set{\seq{v}{1,,n}}\) for \(\vs{\R}^n\) over \(\R\) consisting of eigenvectors of \(A\).
  For each \(i \in \set{1, \dots, n}\), let \(\lambda_i\) be an eigenvalue of \(A\) such that \(A v_i = \lambda v_i\).
  By \cref{2.6} we can define an \(\U \in \ls(\vs{\R}^n)\) such that \(\U(v_i) = \sqrt{v_i} v_i\) for all \(i \in \set{1, \dots, n}\).
  Then we have
  \begin{align*}
             & [\U]_{\beta} = \begin{pmatrix}
                                \sqrt{v_1} & 0          & \cdots & 0          \\
                                0          & \sqrt{v_2} & \cdots & 0          \\
                                \vdots     & \vdots     &        & \vdots     \\
                                0          & 0          & \cdots & \sqrt{v_n}
                              \end{pmatrix}                      &  & \text{(by \cref{2.2.4})}                     \\
    \implies & [\U]_{\beta}^* = \tp{([\U]_{\beta})} = [\U]_{\beta}                   &  & \text{(by \cref{6.1.5})} \\
    \implies & \tp{([\U]_{\beta})} [\U]_{\beta} = ([\U]_{\beta})^2 = [\L_A]_{\beta}. &  & \text{(by \cref{2.2.4})}
  \end{align*}
  If \(\gamma\) is the standard ordered basis for \(\vs{\R}^n\) over \(\R\) (which is orthonormal), then we have
  \begin{align*}
    A & = [\L_A]_{\gamma}                                                                                              &  & \text{(by \cref{2.15}(a))}      \\
      & = \pa{[\IT[\vs{F}^n]]_{\gamma}^{\beta}}^{-1} [\L_A]_{\beta} [\IT[\vs{F}^n]]_{\gamma}^{\beta}                   &  & \text{(by \cref{2.23})}         \\
      & = \pa{[\IT[\vs{F}^n]]_{\gamma}^{\beta}}^{-1} \tp{([\U]_{\beta})} [\U]_{\beta} [\IT[\vs{F}^n]]_{\gamma}^{\beta} &  & \text{(from the proof above)}   \\
      & = \pa{[\IT[\vs{F}^n]]_{\gamma}^{\beta}}^* \tp{([\U]_{\beta})} [\U]_{\beta} [\IT[\vs{F}^n]]_{\gamma}^{\beta}    &  & \text{(by \cref{ex:6.1.23}(c))} \\
      & = \tp{\pa{[\IT[\vs{F}^n]]_{\gamma}^{\beta}}} \tp{([\U]_{\beta})} [\U]_{\beta} [\IT[\vs{F}^n]]_{\gamma}^{\beta} &  & \text{(by \cref{6.1.5})}        \\
      & = \tp{\pa{[\U]_{\beta} [\IT[\vs{F}^n]]_{\gamma}^{\beta}}} [\U]_{\beta} [\IT[\vs{F}^n]]_{\gamma}^{\beta}.       &  & \text{(by \cref{2.3.2})}
  \end{align*}
  Thus \(A\) is Gramian.
\end{proof}

\begin{ex}[Simultaneous Diagonalization]\label{ex:6.4.14}
  Let \(\V\) be a finite-dimensional real inner product space, and let \(\U\) and \(\T\) be self-adjoint linear operators on \(\V\) such that \(\U \T = \T \U\).
  Prove that there exists an orthonormal basis for \(\V\) over \(\R\) consisting of vectors that are eigenvectors of both \(\U\) and \(\T\).
  (The complex version of this result appears as \cref{ex:6.6.10}.)
\end{ex}

\begin{proof}[\pf{ex:6.4.14}]
  Let \(n = \dim(\V)\).
  We use induction on \(n\).
  For \(n = 1\), any orthonormal basis for \(\V\) over \(\R\) makes \([\T]_{\beta}\) and \([\U]_{\beta}\) diagonal matrices.
  Thus by \cref{5.2.8} \(\T, \U\) are simultaneously diagonalizable by some orthonormal bases and the base case holds.
  Suppose inductively that for some \(n \geq 1\), self-adjoint linear operators on \(\V\) which commute are simultaneously diagonalizable by some orthonormal bases.
  We need to show that this is true for \(n + 1\).
  Let \(\dim(\V) = n + 1\), let \(\T, \U \in \ls(\V)\) such that \(\T^* = \T\), \(\U^* = \U\) and \(\U \T = \T \U\).
  Since \(\T\) is self-adjoint, by \cref{6.17} there exists an orthonormal basis \(\beta\) for \(\V\) over \(\R\) consisting of eigenvectors of \(\T\).
  Let \(\lambda \in \R\) be an eigenvalue of \(\T\) and let \(E_{\lambda}\) be the eigenspace of \(\lambda\).
  Now we split into two cases:
  \begin{itemize}
    \item If \(E_{\lambda} = \V\), then any basis for \(\V\) over \(\R\) is consist of eigenvectors of \(\T\).
          Since \(\U\) is self-adjoint, by \cref{6.17} we know that there exists an orthonormal basis for \(\V\) over \(\R\) consist of eigenvectors of \(\U\).
          Thus by \cref{5.2.8} \(\T, \U\) are simultaneously diagonalizable by some orthonormal bases for \(\V\) over \(\R\).
    \item If \(E_{\lambda} \neq \V\), then by \cref{5.7} we have \(1 \leq \dim(E_{\lambda}) < n + 1\).
          By \cref{5.4.2}(e) we know that \(E_{\lambda}\) is \(\T\)-invariant.
          we claim that \(E_{\lambda}\) is \(\U\)-invariant.
          Since
          \begin{align*}
                     & \forall v \in E_{\lambda}, \T(v) = \lambda v             &  & \text{(by \cref{5.2.4})}    \\
            \implies & \forall v \in E_{\lambda}, \lambda \U(v) = \U(\lambda v) &  & \text{(by \cref{2.1.1}(b))} \\
                     & = \U(\T(v)) = \T(\U(v))                                  &  & \text{(by \cref{5.1.2})}    \\
            \implies & \forall v \in E_{\lambda}, \U(v) \in E_{\lambda}         &  & \text{(by \cref{5.1.2})}    \\
            \implies & \U(E_{\lambda}) \subseteq E_{\lambda},
          \end{align*}
          by \cref{5.4.1} we know that \(E_{\lambda}\) is \(\U\)-invariant.
          Since \(\T, \U\) are self-adjoint, by \cref{ex:6.4.7}(b) we know that \(E_{\lambda}^{\perp}\) is both \(\T\)- and \(\U\)-invariant.
          By \cref{6.7}(c) we know that \(\dim(E_{\lambda}^{\perp}) < n + 1\).
          Thus by induction hypothesis we can find some orthonormal bases \(\beta_1\) and \(\beta_2\) for \(E_{\lambda}\) and \(E_{\lambda}^{\perp}\) over \(\R\), respectively, consist of eigenvectors of both \(\T\) and \(\U\).
          By \cref{5.10}(d) and \cref{6.7}(c) we know that \(\beta = \beta_1 \cup \beta_2\) is an orthonormal basis for \(\V\) over \(\R\) consist of eigenvectors of both \(\T\) and \(\U\).
          Thus by \cref{5.2.8} \(\T, \U\) are simultaneously diagonalizable by some orthonormal bases for \(\V\) over \(\R\).
  \end{itemize}
  From all cases above the induction is closed.
\end{proof}

\begin{ex}\label{ex:6.4.15}
  Let \(A, B \in \ms{n}{n}{\R}\) be symmetric such that \(AB = BA\).
  Use \cref{ex:6.4.14} to prove that there exists an orthogonal matrix \(P\) such that \(\tp{P} A P\) and \(\tp{P} B P\) are both diagonal matrices.
\end{ex}

\begin{proof}[\pf{ex:6.4.15}]
  Since
  \begin{align*}
    A & = \tp{A} &  & \text{(by \cref{1.3.4})} \\
      & = A^*    &  & \text{(by \cref{6.1.5})} \\
    B & = \tp{B} &  & \text{(by \cref{1.3.4})} \\
      & = B^*    &  & \text{(by \cref{6.1.5})}
  \end{align*}
  and
  \begin{align*}
    \L_A \L_B & = \L_{AB}    &  & \text{(by \cref{2.15}(e))} \\
              & = \L_{BA}                                    \\
              & = \L_B \L_A, &  & \text{(by \cref{2.15}(e))}
  \end{align*}
  by \cref{ex:6.4.14} there exists an orthonormal basis \(\gamma\) for \(\vs{R}^n\) over \(\R\) such that \([\L_A]_{\gamma}, [\L_B]_{\gamma}\) are diagonal matrices.
  If we let \(\beta\) be the standard ordered basis for \(\vs{R}^n\) over \(\R\) (which is orthonormal), then we have
  \begin{align*}
    [\L_A]_{\gamma} & = \pa{[\IT[\vs{F}^n]]_{\gamma}^{\beta}}^{-1} [\L_A]_{\beta} [\IT[\vs{F}^n]]_{\gamma}^{\beta} &  & \text{(by \cref{2.23})}         \\
                    & = \pa{[\IT[\vs{F}^n]]_{\gamma}^{\beta}}^{-1} A [\IT[\vs{F}^n]]_{\gamma}^{\beta}              &  & \text{(by \cref{2.15}(a))}      \\
                    & = \pa{[\IT[\vs{F}^n]]_{\gamma}^{\beta}}^* A [\IT[\vs{F}^n]]_{\gamma}^{\beta}                 &  & \text{(by \cref{ex:6.1.23}(c))} \\
                    & = \tp{\pa{[\IT[\vs{F}^n]]_{\gamma}^{\beta}}} A [\IT[\vs{F}^n]]_{\gamma}^{\beta}.             &  & \text{(by \cref{6.1.5})}
  \end{align*}
  Similarly we have \([\L_A]_{\gamma} = \tp{\pa{[\IT[\vs{F}^n]]_{\gamma}^{\beta}}} A [\IT[\vs{F}^n]]_{\gamma}^{\beta}\).
  By setting \(P = [\IT[\vs{F}^n]]_{\gamma}^{\beta}\) we are done.
\end{proof}

\begin{ex}\label{ex:6.4.16}
  Prove the \emph{Cayley--Hamilton theorem} for a complex \(n \times n\) matrix \(A\).
  That is, if \(f\) is the characteristic polynomial of \(A\), prove that \(f(A) = \zm\).
  (The general case is proved in \cref{5.23}.)
\end{ex}

\begin{proof}[\pf{ex:6.4.16}]
  Since \(A \in \ms{n}{n}{\C}\), by \cref{d.4} we know that \(\det(A - t I_n)\) splits.
  By \cref{6.14} there exists an orthonormal basis \(\gamma\) for \(\vs{C}^n\) over \(\C\) such that \([\L_A]_{\gamma}\) is upper triangular.
  By \cref{ex:4.2.23} we have
  \[
    \forall t \in \C, f(t) = \prod_{i = 1}^n (([\L_A]_{\gamma})_{i i} - t)
  \]
  and thus by \cref{e.0.7}
  \[
    f([\L_A]_{\gamma}) = \prod_{i = 1}^n (([\L_A]_{\gamma})_{i i} I_n - [\L_A]_{\gamma}).
  \]
  Observe that for each \(j \in \set{1, \dots, n}\), we have
  \begin{align*}
     & f([\L_A]_{\gamma})(e_j)                                       \\
     & = \pa{\prod_{\substack{i = 1                                  \\ i \neq j}}^n (([\L_A]_{\gamma})_{i i} I_n - [\L_A]_{\gamma})} \cdot (([\L_A]_{\gamma})_{j j} I_n - [\L_A]_{\gamma} e_j))         \\
     & = \pa{\prod_{\substack{i = 1                                  \\ i \neq j}}^n (([\L_A]_{\gamma})_{i i} I_n - [\L_A]_{\gamma})} \cdot (([\L_A]_{\gamma})_{j j} e_j - ([\L_A]_{\gamma})_{j j} e_j)) && \text{(\([\L_A]_{\gamma}\) is diagonal)} \\
     & = \pa{\prod_{\substack{i = 1                                  \\ i \neq j}}^n (([\L_A]_{\gamma})_{i i} I_n - [\L_A]_{\gamma})} \cdot \zv \\
     & = \zv.                       &  & \text{(by \cref{2.1.2}(a))}
  \end{align*}
  Thus
  \begin{align*}
             & \forall j \in \set{1, \dots, n}, f([\L_A]_{\gamma})(e_j) = \zv                                \\
    \implies & f([\L_A]_{\gamma}) = \zm.                                      &  & \text{(by \cref{2.1.13})}
  \end{align*}
  If we let \(\beta\) be the standard ordered basis for \(\C^n\) over \(\C\), then we have
  \begin{align*}
    f(A) & = f([\L_A]_{\beta})                                                                                                                                                               &  & \text{(by \cref{2.15}(a))}    \\
         & = f([\IT[\C^n]]_{\gamma}^{\beta} [\L_A]_{\gamma} [\IT[\C^n]]_{\beta}^{\gamma})                                                                                                    &  & \text{(by \cref{2.23})}       \\
         & = \prod_{i = 1}^n (([\L_A]_{\gamma})_{i i} I_n - [\IT[\C^n]]_{\gamma}^{\beta} [\L_A]_{\gamma} [\IT[\C^n]]_{\beta}^{\gamma})                                                                                          \\
         & = \prod_{i = 1}^n (([\L_A]_{\gamma})_{i i} [\IT[\C^n]]_{\gamma}^{\beta} [\IT[\C^n]]_{\beta}^{\gamma} - [\IT[\C^n]]_{\gamma}^{\beta} [\L_A]_{\gamma} [\IT[\C^n]]_{\beta}^{\gamma}) &  & \text{(by \cref{2.23})}       \\
         & = \prod_{i = 1}^n [\IT[\C^n]]_{\gamma}^{\beta} (([\L_A]_{\gamma})_{i i} I_n - [\L_A]_{\gamma}) [\IT[\C^n]]_{\beta}^{\gamma}                                                                                          \\
         & = [\IT[\C^n]]_{\gamma}^{\beta} \pa{\prod_{i = 1}^n (([\L_A]_{\gamma})_{i i} I_n - [\L_A]_{\gamma})} [\IT[\C^n]]_{\beta}^{\gamma}                                                                                     \\
         & = [\IT[\C^n]]_{\gamma}^{\beta} f([\L_A]_{\gamma}) [\IT[\C^n]]_{\beta}^{\gamma}                                                                                                                                       \\
         & = \zm.                                                                                                                                                                            &  & \text{(from the proof above)}
  \end{align*}
\end{proof}

\begin{defn}\label{6.4.11}
  A linear operator \(\T\) on a finite-dimensional inner product space is called \textbf{positive definite} (\textbf{positive semidefinite}) if \(\T\) is self-adjoint and \(\inn{\T(x), x} > 0\) (\(\inn{\T(x), x} \geq 0\)) for all \(x \neq \zv\).

  An \(n \times n\) matrix \(A\) with entries from \(\R\) or \(\C\) is called \textbf{positive definite} (\textbf{positive semidefinite}) if \(\L_A\) is positive definite (positive semidefinite).
\end{defn}

\begin{ex}\label{ex:6.4.17}
  Let \(\T\) and \(\U\) be a self-adjoint linear operators on an \(n\)-dimensional inner product space \(\V\) over \(\F\), and let \(A = [\T]_{\beta}\), where \(\beta\) is an orthonormal basis for \(\V\) over \(\F\).
  Prove the following results.
  \begin{enumerate}
    \item \(\T\) is positive definite (semidefinite) iff all of its eigenvalues are positive (nonnegative).
    \item \(\T\) is positive definite (semidefinite) iff
          \[
            \sum_{i = 1}^n \sum_{j = 1}^n A_{i j} a_j \conj{a_i} > 0 \; (\geq 0) \text{ for all nonzero } n\text{-tuples } \tuple{a}{1,,n}.
          \]
    \item \(\T\) is positive semidefinite iff \(A = B^* B\) for some square matrix \(B\).
    \item If \(\T\) and \(\U\) are positive semidefinite operators such that \(\T^2 = \U^2\), then \(\T = \U\).
    \item If \(\T\) and \(\U\) are positive definite operators such that \(\T \U = \U \T\), then \(\T \U\) is positive definite.
    \item \(\T\) is positive definite (semidefinite) iff \(A\) is positive definite (semidefinite).
  \end{enumerate}
  Because of (f), results analogous to items (a) through (d) hold for matrices as well as operators.
\end{ex}

\begin{proof}[\pf{ex:6.4.17}(a)]
  First suppose that \(\T\) is positive definite (semidefinite).
  Suppose that \(\lambda \in \F\) is an eigenvalue of \(\T\) and \(v \in \V\) is an eigenvector of \(\T\) corresponding to \(\lambda\).
  Then we have
  \begin{align*}
    \lambda & = \lambda \frac{\norm{v}^2}{\norm{v}^2}                &  & (v \neq \zv)                \\
            & = \frac{\lambda \inn{v, v}}{\norm{v}^2}                &  & \text{(by \cref{6.1.9})}    \\
            & = \frac{\inn{\lambda v, v}}{\norm{v}^2}                &  & \text{(by \cref{6.1.1}(b))} \\
            & = \frac{\inn{\T(v), v}}{\norm{v}^2}                    &  & \text{(by \cref{5.1.2})}    \\
            & \begin{dcases}
                > 0    & \text{if } \T \text{ is positive definite} \\
                \geq 0 & \text{if } \T \text{ is semidefinite}
              \end{dcases}. &  & \text{(by \cref{6.2}(b) and \cref{6.4.11})}
  \end{align*}

  Now suppose that eigenvalues of \(\T\) are positive (nonnegative).
  Since \(\T\) is self-adjoint (by hypothesis), by \cref{6.4.8} we have \(\T = \T^*\).
  Thus we have \(\T^* \T = \T^2 = \T \T^*\) and by \cref{6.4.3} \(\T\) is normal.
  By \cref{6.16} we know that there exists an orthonormal basis \(\gamma = \set{\seq{v}{1,,n}}\) for \(\V\) over \(\F\) consist of eigenvectors of \(\T\).
  For each \(i \in \set{1, \dots, n}\), let \(\lambda_i\) be the eigenvalue of \(\T\) correspond to \(v_i\).
  Then we have
  \begin{align*}
     & \forall x \in \V \setminus \set{\zv}, \inn{\T(x), x}                                                                   \\
     & = \inn{\T\pa{\sum_{i = 1}^n \inn{x, v_i} v_i}, \sum_{j = 1}^n \inn{x, v_j} v_j}    &  & \text{(by \cref{6.5})}         \\
     & = \inn{\sum_{i = 1}^n \inn{x, v_i} \T(v_i), \sum_{j = 1}^n \inn{x, v_j} v_j}       &  & \text{(by \cref{2.1.2}(d))}    \\
     & = \inn{\sum_{i = 1}^n \inn{x, v_i} \lambda_i v_i, \sum_{j = 1}^n \inn{x, v_j} v_j} &  & \text{(by \cref{5.1.2})}       \\
     & = \sum_{i = 1}^n \inn{x, v_i} \lambda_i \inn{v_i, \sum_{j = 1}^n \inn{x, v_j} v_j} &  & \text{(by \cref{6.1.1}(a)(b))} \\
     & = \sum_{i = 1}^n \inn{x, v_i} \lambda_i \inn{v_i, \inn{x, v_i} v_i}                &  & \text{(by \cref{6.1.12})}      \\
     & = \sum_{i = 1}^n \abs{\inn{x, v_i}}^2 \lambda_i \inn{v_i, v_i}                     &  & \text{(by \cref{6.1}(b))}      \\
     & = \sum_{i = 1}^n \abs{\inn{x, v_i}}^2 \lambda_i \norm{v_i}^2                       &  & \text{(by \cref{6.1.9})}       \\
     & = \sum_{i = 1}^n \abs{\inn{x, v_i}}^2 \lambda_i                                    &  & \text{(by \cref{6.1.12})}      \\
     & \begin{dcases}
         > 0    & \text{if } \lambda_i > 0 \text{ for all } i \in \set{1, \dots, n}    \\
         \geq 0 & \text{if } \lambda_i \geq 0 \text{ for all } i \in \set{1, \dots, n}
       \end{dcases}.   &  & (x \neq \zv)
  \end{align*}
  Thus by \cref{6.4.11} \(\T\) is positive definite (semidefinite).
  We conclude that \(\T\) is positive definite (semidefinite) iff all of its eigenvalues are positive (nonnegative).
\end{proof}

\begin{proof}[\pf{ex:6.4.17}(b)]
  Let \(\beta = \set{\seq{v}{1,,n}}\).
  Observe that
  \begin{align*}
     & \inn{\T\pa{\sum_{j = 1}^n a_j v_j}, \sum_{k = 1}^n a_k v_k}                                                            \\
     & = \inn{\sum_{j = 1}^n a_j \T(v_j), \sum_{k = 1}^n a_k v_k}                         &  & \text{(by \cref{2.1.2}(d))}    \\
     & = \inn{\sum_{j = 1}^n a_j \pa{\sum_{i = 1}^n A_{i j} v_i}, \sum_{k = 1}^n a_k v_k} &  & \text{(by \cref{2.2.4})}       \\
     & = \sum_{j = 1}^n \sum_{i = 1}^n a_j A_{i j} \inn{v_i, \sum_{k = 1}^n a_k v_k}      &  & \text{(by \cref{6.1.1}(a)(b))} \\
     & = \sum_{j = 1}^n \sum_{i = 1}^n a_j A_{i j} \inn{v_i, a_i v_i}                     &  & \text{(by \cref{6.1.12})}      \\
     & = \sum_{j = 1}^n \sum_{i = 1}^n a_j A_{i j} \conj{a_i} \inn{v_i, v_i}              &  & \text{(by \cref{6.1}(b))}      \\
     & = \sum_{j = 1}^n \sum_{i = 1}^n a_j A_{i j} \conj{a_i}.                            &  & \text{(by \cref{6.1.12})}
  \end{align*}
  Thus we have
  \begin{align*}
         & \T \text{ is positive definite (semidefinite)}                                                                                                  \\
    \iff & \forall x \in \V \setminus \set{\zv}, 0 < (\leq) \inn{\T(x), x}                                              &  & \text{(by \cref{6.4.11})}     \\
         & = \inn{\T\pa{\sum_{j = 1}^n \inn{x, v_j} v_j}, \sum_{k = 1}^n \inn{x, v_k} v_k}                              &  & \text{(by \cref{6.5})}        \\
         & = \sum_{j = 1}^n \sum_{i = 1}^n \inn{x, v_j} A_{i j} \conj{\inn{x, v_i}}                                     &  & \text{(from the proof above)} \\
    \iff & \forall a \in \vs{F}^n \setminus \set{\zv}, \sum_{i = 1}^n \sum_{j = 1}^n A_{i j} a_j \conj{a_i} > (\geq) 0.
  \end{align*}
\end{proof}

\begin{proof}[\pf{ex:6.4.17}(c)]
  First suppose that \(\T\) is positive semidefinite.
  As in the proof of \cref{ex:6.4.17}(a), \(\T\) is self-adjoint implies there exists an orthonormal basis \(\gamma = \set{\seq{v}{1,,n}}\) for \(\V\) over \(\F\) consist of eigenvectors of \(\T\).
  For each \(i \in \set{1, \dots, n}\), let \(\lambda_i\) be the eigenvalue of \(\T\) corresponding to \(v_i\).
  By \cref{ex:6.4.17}(a) we know that \(\lambda_i \geq 0\) for all \(i \in \set{1, \dots, n}\).
  If we let \(\alpha\) be the standard ordered basis for \(\vs{F}^n\) over \(\F\) and
  \[
    C = \begin{pmatrix}
      \sqrt{\lambda_1} & 0                & \cdots & 0                \\
      0                & \sqrt{\lambda_2} & \cdots & 0                \\
      \vdots           & \vdots           &        & \vdots           \\
      0                & 0                & \cdots & \sqrt{\lambda_n}
    \end{pmatrix},
  \]
  then we have
  \begin{align*}
             & C^* = \tp{C} = C                                                                                                                          &  & \text{(by \cref{6.1.5})}      \\
    \implies & [\T]_{\gamma} = C^* C                                                                                                                     &  & \text{(by \cref{5.1.1})}      \\
    \implies & A = [\T]_{\beta} = \pa{[\IT[\V]]_{\beta}^{\gamma}}^{-1} [\T]_{\gamma} [\IT[\V]]_{\beta}^{\gamma}                                          &  & \text{(by \cref{2.23})}       \\
             & = \pa{[\IT[\V]]_{\beta}^{\gamma}}^{-1} C^* C [\IT[\V]]_{\beta}^{\gamma}                                                                   &  & \text{(from the proof above)} \\
             & = \pa{[\IT[\V]]_{\alpha}^{\gamma} [\IT[\V]]_{\beta}^{\alpha}}^{-1} C^* C [\IT[\V]]_{\alpha}^{\gamma} [\IT[\V]]_{\beta}^{\alpha}           &  & \text{(by \cref{2.11})}       \\
             & = \pa{[\IT[\V]]_{\beta}^{\alpha}}^{-1} \pa{[\IT[\V]]_{\alpha}^{\gamma}}^{-1} C^* C [\IT[\V]]_{\alpha}^{\gamma} [\IT[\V]]_{\beta}^{\alpha} &  & \text{(by \cref{ex:2.4.4})}   \\
             & = \pa{[\IT[\V]]_{\beta}^{\alpha}}^* \pa{[\IT[\V]]_{\alpha}^{\gamma}}^* C^* C [\IT[\V]]_{\alpha}^{\gamma} [\IT[\V]]_{\beta}^{\alpha}       &  & \text{(by \cref{ex:6.1.23})}  \\
             & = (C [\IT[\V]]_{\alpha}^{\gamma} [\IT[\V]]_{\beta}^{\alpha})^* C [\IT[\V]]_{\alpha}^{\gamma} [\IT[\V]]_{\beta}^{\alpha}.                  &  & \text{(by \cref{6.3.2}(c))}
  \end{align*}
  By setting \(B = C [\IT[\V]]_{\alpha}^{\gamma} [\IT[\V]]_{\beta}^{\alpha}\) we see that \(A = B^* B\).

  Now suppose that there exists a \(B \in \ms{n}{n}{\F}\) such that \(A = B^* B\).
  Let \(\beta = \set{\seq{v}{1,,n}}\) and let \([\cdot, \cdot]\) be the standard inner product on \(\vs{F}^n\) over \(\F\) as defined in \cref{6.1.2}.
  For each \(i \in \set{1, \dots, n}\), let \(b_i\) denote the \(i\)th column of \(B\).
  Then we have
  \begin{align*}
     & \forall x \in \V \setminus \set{\zv}, \inn{\T(x), x}                                                                                       \\
     & = \inn{\T\pa{\sum_{j = 1}^n \inn{x, v_j} v_j}, \sum_{k = 1}^n \inn{x, v_k} v_k}                        &  & \text{(by \cref{6.5})}         \\
     & = \inn{\sum_{j = 1}^n \inn{x, v_j} \T(v_j), \sum_{k = 1}^n \inn{x, v_k} v_k}                           &  & \text{(by \cref{2.1.2}(d))}    \\
     & = \inn{\sum_{j = 1}^n \inn{x, v_j} \pa{\sum_{i = 1}^n A_{i j} v_i}, \sum_{k = 1}^n \inn{x, v_k} v_k}   &  & \text{(by \cref{2.2.4})}       \\
     & = \sum_{i = 1}^n \sum_{j = 1}^n \inn{x, v_j} A_{i j} \inn{v_i, \sum_{k = 1}^n \inn{x, v_k} v_k}        &  & \text{(by \cref{6.1.1}(a)(b))} \\
     & = \sum_{i = 1}^n \sum_{j = 1}^n \inn{x, v_j} A_{i j} \inn{v_i, \inn{x, v_i} v_i}                       &  & \text{(by \cref{6.1.12})}      \\
     & = \sum_{i = 1}^n \sum_{j = 1}^n \inn{x, v_j} A_{i j} \conj{\inn{x, v_i}} \inn{v_i, v_i}                &  & \text{(by \cref{6.1}(b))}      \\
     & = \sum_{i = 1}^n \sum_{j = 1}^n \inn{x, v_j} A_{i j} \conj{\inn{x, v_i}}                               &  & \text{(by \cref{6.1.12})}      \\
     & = \sum_{i = 1}^n \sum_{j = 1}^n \inn{x, v_j} (B^* B)_{i j} \conj{\inn{x, v_i}}                         &  & \text{(by hypothesis)}         \\
     & = \sum_{i = 1}^n \sum_{j = 1}^n \sum_{k = 1}^n \inn{x, v_j} (B^*)_{i k} B_{k j} \conj{\inn{x, v_i}}    &  & \text{(by \cref{2.3.1})}       \\
     & = \sum_{i = 1}^n \sum_{j = 1}^n \sum_{k = 1}^n \inn{x, v_j} \conj{B_{k i}} B_{k j} \conj{\inn{x, v_i}} &  & \text{(by \cref{6.1.5})}       \\
     & = \sum_{i = 1}^n \sum_{j = 1}^n \sum_{k = 1}^n \inn{x, v_j} B_{k j} \conj{\inn{x, v_i} B_{k i}}        &  & \text{(by \cref{d.2}(c))}      \\
     & = \sum_{i = 1}^n \sum_{j = 1}^n [\inn{x, v_j} b_j, \inn{x, v_i} b_i]                                   &  & \text{(by \cref{6.1.2})}       \\
     & = \sum_{i = 1}^n \br{\sum_{j = 1}^n \inn{x, v_j} b_j, \inn{x, v_i} b_i}                                &  & \text{(by \cref{6.1.1}(a))}    \\
     & = \br{\sum_{j = 1}^n \inn{x, v_j} b_j, \sum_{i = 1}^n \inn{x, v_i} b_i}                                &  & \text{(by \cref{6.1}(a))}      \\
     & = \norm{\sum_{i = 1}^n \inn{x, v_i} b_i}^2                                                             &  & \text{(by \cref{6.1.9})}       \\
     & \geq 0.                                                                                                &  & \text{(by \cref{6.2}(b))}
  \end{align*}
  Thus by \cref{6.4.11} \(\T\) is positive semidefinite.
\end{proof}

\begin{proof}[\pf{ex:6.4.17}(d)]
  As in the proof of \cref{ex:6.4.17}(a), \(\T\) is self-adjoint implies there exists an orthonormal basis \(\gamma\) for \(\V\) over \(\F\) consist of eigenvectors of \(\T\).
  Let \(v \in \gamma\) and let \(\lambda \in \F\) be an eigenvalue of \(\T\) corresponding to \(v\).
  By \cref{ex:6.4.17}(a) we know that \(\lambda \geq 0\).
  Now we split into two cases:
  \begin{itemize}
    \item If \(\lambda = 0\), then we have
          \begin{align*}
                     & \T(v) = 0v = \zv                  &  & \text{(by \cref{5.1.2})}    \\
            \implies & \U^2(v) = \T^2(v) = \T(\zv) = \zv &  & \text{(by \cref{2.1.2}(a))} \\
            \implies & 0 = \inn{\U^2(v), v}              &  & \text{(by \cref{6.1}(c))}   \\
                     & = \inn{(\U^* \U)(v), v}           &  & \text{(by \cref{6.4.8})}    \\
                     & = \inn{\U(v), \U(v)}              &  & \text{(by \cref{6.9})}      \\
            \implies & \U(v) = \zv = \T(v).              &  & \text{(by \cref{6.1}(d))}
          \end{align*}
    \item If \(\lambda > 0\), then we have
          \begin{align*}
                     & \U^2(v) = \T^2(v) = \lambda^2 v                                                                           &  & \text{(by \cref{5.1.2})} \\
            \implies & \zv = \U^2(v) - \lambda^2 v = ((\U + \lambda \IT[\V]) (\U - \lambda \IT[\V]))(v)                          &  & \text{(by \cref{2.10})}  \\
            \implies & \begin{dcases}
                         \ns{\U - (-\lambda) \IT[\V]} = \ns{\U + \lambda \IT[\V]} = \set{\zv} \\
                         (\U - \lambda \IT[\V])(v) \in \ns{\U + \lambda \IT[\V]}
                       \end{dcases} &  & \text{(by \cref{ex:6.4.17}(a))}                                  \\
            \implies & (\U - \lambda \IT[\V])(v) = \zv                                                                                                         \\
            \implies & \U(v) = \lambda v = \T(v).
          \end{align*}
  \end{itemize}
  From all cases above we see that \(\U(v) = \T(v)\).
  Since \(v\) is arbitrary, we see that \(\T, \U\) argee on \(\gamma\).
  Thus by \cref{2.1.13} we have \(\T = \U\).
\end{proof}

\begin{proof}[\pf{ex:6.4.17}(e)]
  Since \(\T, \U\) are self-adjoint and \(\T \U = \U \T\), by \cref{ex:6.4.14} there exists an orthonormal basis \(\gamma\) for \(\V\) over \(\F\) consist of eigenvectors of \(\T\) and \(\U\).
  Let \(v \in \gamma\) and let \(\lambda_{\T}, \lambda_{\U} \in \F\) be eigenvalues of \(\T\) and \(\U\) corresponding to \(v\), respectively.
  Since
  \begin{align*}
    \U(\T(v)) & = \U(\lambda_{\T} v)           &  & \text{(by \cref{5.1.2})}    \\
              & = \lambda_{\T} \U(v)           &  & \text{(by \cref{2.1.1}(b))} \\
              & = \lambda_{\T} \lambda_{\U} v, &  & \text{(by \cref{5.1.2})}
  \end{align*}
  we know that \(v\) is an eigenvector of \(\U \T\) corresponding to eigenvalue \(\lambda_{\T} \lambda_{\U}\).
  By \cref{ex:6.4.17}(a) we know that \(\lambda_{\T} > 0\) and \(\lambda_{\U} > 0\), thus \(\lambda_{\T} \lambda_{\U} > 0\).
  Since \(v\) is arbitrary, we know that \(\gamma\) is consist of eigenvectors of \(\U \T\) and all eigenvalues of \(\U \T\) are positive.
  Since
  \begin{align*}
    (\U \T)^* & = \T^* \U^* &  & \text{(by \cref{6.11}(c))} \\
              & = \T \U     &  & \text{(by \cref{6.4.8})}   \\
              & = \U \T,
  \end{align*}
  by \cref{6.4.8} we know that \(\U \T\) is self-adjoint.
  Thus by \cref{ex:6.4.17}(a) \(\U \T = \T \U\) is positive definite.
\end{proof}

\begin{proof}[\pf{ex:6.4.17}(f)]
  Let \([\cdot, \cdot]\) be the standard inner product on \(\vs{F}^n\) over \(\F\).
  Then we have
  \begin{align*}
         & \T \text{ is positive definite (semidefinite)}                                                                               \\
    \iff & \forall x \in \V, \inn{\T(x), x} > (\geq) 0                                             &  & \text{(by \cref{6.4.11})}       \\
    \iff & \forall x \in \vs{F}^n, \sum_{i = 1}^n \sum_{j = 1}^n A_{i j} x_j \conj{x_i} > (\geq) 0 &  & \text{(by \cref{ex:6.4.17}(b))} \\
    \iff & \forall x \in \vs{F}^n, [Ax, x] > (\geq) 0                                              &  & \text{(by \cref{6.1.2})}        \\
    \iff & \forall x \in \vs{F}^n, [\L_A(x), x] > (\geq) 0                                         &  & \text{(by \cref{2.3.8})}        \\
    \iff & A \text{ is positive definite (semidefinite)}.                                          &  & \text{(by \cref{6.4.11})}
  \end{align*}
\end{proof}

\begin{ex}\label{ex:6.4.18}
  Let \(\T \in \ls(\V, \W)\) where \(\V\) and \(\W\) are finite-dimensional inner product spaces.
  Prove the following results.
  \begin{enumerate}
    \item \(\T^* \T\) and \(\T \T^*\) are positive semidefinite.
    \item \(\rk{\T^* \T} = \rk{\T \T^*} = \rk{\T}\).
  \end{enumerate}
\end{ex}

\begin{proof}[\pf{ex:6.4.18}(a)]
  Let \(\inn{\cdot, \cdot}_1, \inn{\cdot, \cdot}_2\) be inner products on \(\V\) and \(\W\) over \(\F\), respectively.
  Since
  \begin{align*}
    (\T^* \T)^* & = \T^* \T^{**} &  & \text{(by \cref{ex:6.3.16}(c))} \\
                & = \T^* \T      &  & \text{(by \cref{ex:6.3.16}(d))} \\
    (\T \T^*)^* & = \T^{**} \T^* &  & \text{(by \cref{ex:6.3.16}(c))} \\
                & = \T \T^*      &  & \text{(by \cref{ex:6.3.16}(d))}
  \end{align*}
  and
  \begin{align*}
    \forall x \in \V \setminus \set{\zv}, \inn{(\T^* \T)(x), x}_1 & = \inn{\T^*(\T(x)), x}_1                                        \\
                                                                  & = \inn{\T(x), \T(x)}_2     &  & \text{(by \cref{ex:6.3.15}(a))} \\
                                                                  & \geq 0                     &  & \text{(by \cref{6.1}(d))}       \\
    \forall y \in \W \setminus \set{\zv}, \inn{(\T \T^*)(y), y}_2 & = \inn{\T(\T^*(y)), y}_2                                        \\
                                                                  & = \inn{\T^*(y), \T^*(y)}_1 &  & \text{(by \cref{ex:6.3.15}(a))} \\
                                                                  & \geq 0,                    &  & \text{(by \cref{6.1}(d))}
  \end{align*}
  by \cref{6.4.11} we know that \(\T^* \T\) and \(\T \T^*\) are positive semidefinite.
\end{proof}

\begin{proof}[\pf{ex:6.4.18}(b)]
  By \cref{ex:6.3.15}(e) we have \(\ns{\T^* \T} = \ns{\T}\).
  Since
  \begin{align*}
             & \dim(\V) = \begin{dcases}
                            \rk{\T} + \nt{\T} \\
                            \rk{\T^* \T} + \nt{\T^* \T}
                          \end{dcases} &  & \text{(by \cref{2.3})}            \\
    \implies & \rk{\T} = \rk{\T^* \T},     &  & \text{(from the proof above)}
  \end{align*}
  we have
  \begin{align*}
    \rk{\T \T^*} & = \rk{\T^{**} \T^*} &  & \text{(by \cref{ex:6.3.16}(d))} \\
                 & = \rk{\T^*}         &  & \text{(from the proof above)}   \\
                 & = \rk{\T}.          &  & \text{(by \cref{ex:6.3.15}(c))}
  \end{align*}
\end{proof}

\begin{ex}\label{ex:6.4.19}
  Let \(\T\) and \(\U\) be positive definite operators on an inner product space \(\V\) over \(\F\).
  Prove the following results.
  \begin{enumerate}
    \item \(\T + \U\) is positive definite.
    \item If \(c > 0\), then \(c \T\) is positive definite.
    \item \(\T^{-1}\) is positive definite.
  \end{enumerate}
\end{ex}

\begin{proof}[\pf{ex:6.4.19}(a)]
  We have
  \begin{align*}
    (\T + \U)^* & = \T^* + \U^* &  & \text{(by \cref{6.11}(a))} \\
                & = \T + \U     &  & \text{(by \cref{6.4.11})}
  \end{align*}
  and
  \begin{align*}
    \forall x \in \V \setminus \set{\zv}, \inn{(\T + \U)(x), x} & = \inn{\T(x) + \U(x), x}                                           \\
                                                                & = \inn{\T(x), x} + \inn{\U(x), x} &  & \text{(by \cref{6.1.1}(a))} \\
                                                                & > 0.                              &  & \text{(by \cref{6.4.11})}
  \end{align*}
  Thus by \cref{6.4.11} \(\T + \U\) is positive definite.
\end{proof}

\begin{proof}[\pf{ex:6.4.19}(b)]
  We have
  \begin{align*}
    (c\T)^* & = \conj{c} \T^* &  & \text{(by \cref{6.11}(b))} \\
            & = \conj{c} \T   &  & \text{(by \cref{6.4.11})}  \\
            & = c \T          &  & (c > 0)
  \end{align*}
  and
  \begin{align*}
    \forall x \in \V \setminus \set{\zv}, \inn{(c \T)(x), x} & = \inn{c \T(x), x}                                  \\
                                                             & = c \inn{\T(x), x} &  & \text{(by \cref{6.1.1}(b))} \\
                                                             & > 0.               &  & \text{(by \cref{6.4.11})}
  \end{align*}
  Thus by \cref{6.4.11} \(c \T\) is positive definite.
\end{proof}

\begin{proof}[\pf{ex:6.4.19}(c)]
  We have
  \begin{align*}
    (\T^{-1})^* & = (\T^*)^{-1} &  & \text{(by \cref{ex:6.3.8})} \\
                & = \T^{-1}     &  & \text{(by \cref{6.4.11})}
  \end{align*}
  and
  \begin{align*}
             & \lambda \in \F \text{ is an eigenvalue of } \T^{-1}                                      \\
    \implies & \lambda^{-1} \in \F \text{ is an eigenvalue of } \T &  & \text{(by \cref{ex:5.1.8}(b))}  \\
    \implies & \lambda^{-1} > 0                                    &  & \text{(by \cref{ex:6.4.17}(a))} \\
    \implies & \lambda > 0.
  \end{align*}
  Thus by \cref{ex:6.4.17}(a) \(\T^{-1}\) is positive definite.
\end{proof}

\begin{ex}\label{ex:6.4.20}
  Let \(\V\) be an inner product space over \(\F\) with inner product \(\inn{\cdot, \cdot}\), and let \(\T\) be a positive definite linear operator on \(\V\).
  Prove that \(\inn{x, y}' = \inn{\T(x), y}\) defines another inner product on \(\V\) over \(\F\).
\end{ex}

\begin{proof}[\pf{ex:6.4.20}]
  Let \(x, y, z \in \V\) and let \(c \in \F\).
  Then we have
  \begin{align*}
    \inn{x + y, z}'    & = \inn{\T(x + y), z}              &  & \text{(by \cref{ex:6.4.20})} \\
                       & = \inn{\T(x) + \T(y), z}          &  & \text{(by \cref{2.1.1}(a))}  \\
                       & = \inn{\T(x), z} + \inn{\T(y), z} &  & \text{(by \cref{6.1.1}(a))}  \\
                       & = \inn{x, z}' + \inn{y, z}'       &  & \text{(by \cref{ex:6.4.20})} \\
    \inn{cx, y}'       & = \inn{\T(cx), y}                 &  & \text{(by \cref{ex:6.4.20})} \\
                       & = \inn{c \T(x), y}                &  & \text{(by \cref{2.1.1}(b))}  \\
                       & = c \inn{\T(x), y}                &  & \text{(by \cref{6.1.1}(b))}  \\
                       & = c \inn{x, y}'                   &  & \text{(by \cref{ex:6.4.20})} \\
    \conj{\inn{x, y}'} & = \conj{\inn{\T(x), y}}           &  & \text{(by \cref{ex:6.4.20})} \\
                       & = \inn{y, \T(x)}                  &  & \text{(by \cref{6.1.1}(c))}  \\
                       & = \inn{\T^*(y), x}                &  & \text{(by \cref{6.9})}       \\
                       & = \inn{\T(y), x}                  &  & \text{(by \cref{6.4.11})}    \\
                       & = \inn{y, x}'.                    &  & \text{(by \cref{ex:6.4.20})}
  \end{align*}
  If \(x \neq \zv\), then we have
  \begin{align*}
    \inn{x, x}' & = \inn{\T(x), x} &  & \text{(by \cref{ex:6.4.20})} \\
                & > 0.             &  & \text{(by \cref{6.4.11})}
  \end{align*}
  Thus by \cref{6.1.1} \(\inn{\cdot, \cdot}'\) is an inner product on \(\V\) over \(\F\).
\end{proof}

\begin{ex}\label{ex:6.4.21}
  Let \(\V\) be a finite-dimensional inner product space over \(\F\), and let \(\T\) and \(\U\) be self-adjoint operators on \(\V\) such that \(\T\) is positive definite.
  Prove that both \(\T \U\) and \(\U \T\) are diagonalizable linear operators that have only real eigenvalues.
\end{ex}

\begin{proof}[\pf{ex:6.4.21}]
  Let \(\inn{\cdot, \cdot}\) be an inner product on \(\V\) over \(\F\).
  For each \(\lt{S} \in \ls(\V)\), we denote the adjoint of \(\lt{S}\) with respect to \(\inn{\cdot, \cdot}\) as \(\vs{S}^*\).
  Suppose that \(\T, \U\) are self-adjoint with respect to \(\inn{\cdot, \cdot}\) and \(\T\) is positive definite with respect to \(\inn{\cdot, \cdot}\).

  Define inner product \(\inn{\cdot, \cdot}'\) on \(\V\) over \(\F\) by setting \(\inn{x, y}' = \inn{\T(x), y}\) for all \(x, y \in \V\).
  By \cref{ex:6.4.20} we see that \(\inn{\cdot, \cdot}'\) is indeed an inner product on \(\V\) over \(\F\).
  For each \(\lt{S} \in \ls(\V)\), we denote the adjoint of \(\lt{S}\) with respect to \(\inn{\cdot, \cdot}'\) as \(\vs{S}^{\dag}\).
  These adjoint operators are well-defined thanks to \cref{6.9}.
  Now we show that \(\U \T\) is self-adjoint with respect to \(\inn{\cdot, \cdot}'\).
  Since
  \begin{align*}
    \forall x, y \in \V, \inn{x, (\U \T)^{\dag}(y)}' & = \inn{(\U \T)(x), y}'       &  & \text{(by \cref{6.9})}   \\
                                                     & = \inn{\T(\U(\T(x))), y}                                   \\
                                                     & = \inn{\T(x), \U^*(\T^*(y))} &  & \text{(by \cref{6.9})}   \\
                                                     & = \inn{\T(x), \U(\T(y))}     &  & \text{(by \cref{6.4.8})} \\
                                                     & = \inn{x, (\U \T)(y)}',
  \end{align*}
  by \cref{6.1}(e) we have \((\U \T)^{\dag} = \U \T\).
  Thus by \cref{6.4.8} \(\U \T\) is self-adjoint with respect to \(\inn{\cdot, \cdot}'\), and by \cref{6.16,6.17} we know that \(\U \T\) is diagonalizable.
  By \cref{6.4.10}(a) this means \(\U \T\) has only real eigenvalues.

  Since \(\T\) is self-adjoint with respect to \(\inn{\cdot, \cdot}\), by \cref{6.16,6.17} we know that there exists an orthonormal basis for \(\V\) over \(\F\) consist of eigenvectors of \(\T\).
  Since \(\T\) is positive definite with respect to \(\inn{\cdot, \cdot}\), by \cref{ex:6.4.17}(a) we know that every eigenvalue of \(\T\) is positive.
  Thus \(\T\) is invertible, and we can define inner product \(\inn{x, y}'' = \inn{\T^{-1}(x), y}\) for all \(x, y\).
  By \cref{ex:6.4.19}(c) we know that \(\T^{-1}\) is positive definite with respect to \(\inn{\cdot, \cdot}\), thus by \cref{ex:6.4.20} \(\inn{\cdot, \cdot}''\) is indeed an inner product on \(\V\) over \(\F\).
  For each \(\lt{S} \in \ls(\V)\), we denote the adjoint of \(\lt{S}\) with respect to \(\inn{\cdot, \cdot}''\) as \(\vs{S}^{\ddag}\).
  These adjoint operators are well-defined thanks to \cref{6.9}.
  Now we show that \(\T \U\) is self-adjoint with respect to \(\inn{\cdot, \cdot}''\).
  Since
  \begin{align*}
    \forall x, y \in \V, \inn{x, (\T \U)^{\ddag}(y)}'' & = \inn{(\T \U)(x), y}''           &  & \text{(by \cref{6.9})}      \\
                                                       & = \inn{\T^{-1}(\T(\U(x))), y}                                      \\
                                                       & = \inn{\U(x), y}                                                   \\
                                                       & = \inn{x, \U^*(y)}                &  & \text{(by \cref{6.9})}      \\
                                                       & = \inn{x, \U(y)}                  &  & \text{(by \cref{6.4.8})}    \\
                                                       & = \inn{x, \T^{-1}(\T(\U(y)))}                                      \\
                                                       & = \inn{(\T^{-1})^*(x), \T(\U(y))} &  & \text{(by \cref{6.9})}      \\
                                                       & = \inn{(\T^*)^{-1}(x), \T(\U(y))} &  & \text{(by \cref{ex:6.3.8})} \\
                                                       & = \inn{\T^{-1}(x), \T(\U(y))}     &  & \text{(by \cref{6.4.8})}    \\
                                                       & = \inn{x, (\T \U)(y)}'',
  \end{align*}
  by \cref{6.1}(e) we have \((\T \U)^{\ddag} = \T \U\).
  Thus by \cref{6.4.8} \(\T \U\) is self-adjoint with respect to \(\inn{\cdot, \cdot}''\), and by \cref{6.16,6.17} we know that \(\T \U\) is diagonalizable.
  By \cref{6.4.10}(a) this means \(\T \U\) has only real eigenvalues.
\end{proof}

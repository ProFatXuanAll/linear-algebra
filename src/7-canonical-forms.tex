\chapter{Canonical Forms}\label{ch:7}

\begin{note}
  As we learned in \cref{ch:5}, the advantage of a diagonalizable linear operator lies in the simplicity of its description.
  Such an operator has a diagonal matrix representation, or, equivalently, there is an ordered basis for the underlying vector space consisting of eigenvectors of the operator.
  However, not every linear operator is diagonalizable, even if its characteristic polynomial splits.

  It is the purpose of this chapter to consider alternative matrix representations for nondiagonalizable operators.
  These representations are called \emph{canonical forms}.
  There are different kinds of canonical forms, and their advantages and disadvantages depend on how they are applied.
  The choice of a canonical form is determined by the appropriate choice of an ordered basis.
  Naturally, the canonical forms of a linear operator are not diagonal matrices if the linear operator is not diagonalizable.

  In this chapter, we treat two common canonical forms.
  The first of these, the \emph{Jordan canonical form}, requires that the characteristic polynomial of the operator splits.
  This form is always available if the underlying field is algebraically closed, that is, if every polynomial with coefficients from the field splits.
  For example, the field of complex numbers is algebraically closed by the fundamental theorem of algebra (see \cref{d.4}).
  The first two sections deal with this form.
  The \emph{rational canonical form}, treated in \cref{sec:7.4}, does not require such a factorization.
\end{note}

% All sections are in separated files.  We include them here.
\section{The Jordan Canonical Form I}\label{sec:7.1}

\begin{defn}\label{7.1.1}
  Let \(\T\) be a linear operator on a finite-dimensional vector space \(\V\) over \(\F\), and suppose that the characteristic polynomial of \(\T\) splits.
  Recall from \cref{5.9} that the diagonalizability of \(\T\) depends on whether the union of ordered bases for the distinct eigenspaces of \(\T\) is an ordered basis for \(\V\) over \(\F\).
  So a lack of diagonalizability means that at least one eigenspace of \(\T\) is too ``small.''

  In this section, we extend the definition of eigenspace to \emph{generalized eigenspace}.
  From these subspaces, we select ordered bases whose union is an ordered basis \(\beta\) for \(\V\) over \(\F\) such that
  \[
    [\T]_{\beta} = \begin{pmatrix}
      A_1    & \zm    & \cdots & \zm    \\
      \zm    & A_2    & \cdots & \zm    \\
      \vdots & \vdots &        & \vdots \\
      \zm    & \zm    & \cdots & A_k
    \end{pmatrix},
  \]
  where each \(\zm\) is a zero matrix, and each \(A_i\) is a square matrix of the form \((\lambda)\) or
  \[
    \begin{pmatrix}
      \lambda & 1       & \zv    & \cdots & \zv     & \zv     \\
      \zv     & \lambda & 1      & \cdots & \zv     & \zv     \\
      \vdots  & \vdots  & \vdots &        & \vdots  & \vdots  \\
      \zv     & \zv     & \zv    & \cdots & \lambda & 1       \\
      \zv     & \zv     & \zv    & \cdots & \zv     & \lambda
    \end{pmatrix}
  \]
  for some eigenvalue \(\lambda\) of \(\T\).
  Such a matrix \(A_i\) is called a \textbf{Jordan block} corresponding to \(\lambda\), and the matrix \([\T]_{\beta}\) is called a \textbf{Jordan canonical form} of \(\T\).
  We also say that the ordered basis \(\beta\) is a \textbf{Jordan canonical basis} for \(\T\).
  Observe that each Jordan block \(A_i\) is ``almost'' a diagonal matrix
  ---
  in fact, \([\T]_{\beta}\) is a diagonal matrix iff each \(A_i\) is of the form \((\lambda)\).
\end{defn}

\begin{note}
  In \cref{sec:7.1,sec:7.2}, we prove that every linear operator whose characteristic polynomial splits has a Jordan canonical form that is unique up to the order of the Jordan blocks.
  Nevertheless, it is not the case that the Jordan canonical form is completely determined by the characteristic polynomial of the operator.
\end{note}

\begin{defn}\label{7.1.2}
  Let \(\T\) be a linear operator on a vector space \(\V\) over \(\F\), and let \(\lambda \in \F\).
  Because of the structure of each Jordan block in a Jordan canonical form, we can generalize these observations:
  If \(v\) lies in a Jordan canonical basis for a linear operator \(\T\) and is associated with a Jordan block with diagonal entry \(\lambda\), then \((\T - \lambda \IT[\V])^p(v) = \zv\) for sufficiently large \(p\).
  Eigenvectors satisfy this condition for \(p = 1\).
  A nonzero vector \(x\) in \(\V\) is called a \textbf{generalized eigenvector of \(\T\) corresponding to \(\lambda\)} if \((\T - \lambda \IT[\V])^p(x) = \zv\) for some positive integer \(p\).
\end{defn}

\begin{cor}\label{7.1.3}
  Let \(\T\) be a linear operator on a vector space \(\V\) over \(\F\), and let \(\lambda \in \F\).
  If \(x\) is a generalized eigenvector of \(\T\) corresponding to \(\lambda\), and \(p\) is the smallest positive integer for which \((\T - \lambda \IT[\V])^p(x) = \zv\), then \((\T - \lambda \IT[\V])^{p - 1}(x)\) is an eigenvector of \(\T\) corresponding to \(\lambda\).
  Therefore \(\lambda\) is an eigenvalue of \(\T\).
\end{cor}

\begin{proof}[\pf{7.1.3}]
  Observe that
  \begin{align*}
             & (\T - \lambda \IT[\V])^p(x) = \zv                                   \\
    \implies & (\T - \lambda \IT[\V])\pa{(\T - \lambda \IT[\V])^{p - 1}(x)} = \zv.
  \end{align*}
  Since \(p\) is the smallest positive integer for which \((\T - \lambda \IT[\V])^p(x) = \zv\), we know that \((\T - \lambda \IT[\V])^{p - 1}(x) \neq \zv\).
  Thus by \cref{5.4} \((\T - \lambda \IT[\V])^{p - 1}(x)\) is an eigenvector of \(\T\) and therefore \(\lambda\) is an eigenvalue of \(\T\).
\end{proof}

\begin{defn}\label{7.1.4}
  Let \(\T\) be a linear operator on a vector space \(\V\) over \(\F\), and let \(\lambda\) be an eigenvalue of \(\T\).
  The \textbf{generalized eigenspace of \(\T\) corresponding to \(\lambda\)}, denoted \(\vs{K}_{\lambda}\), is the subset of \(\V\) defined by
  \[
    \vs{K}_{\lambda} = \set{x \in \V : (\T - \lambda \IT[\V])^p(x) = \zv \text{ for some } p \in \Z^+}.
  \]
  Note that \(\vs{K}_{\lambda}\) consists of the zero vector (\cref{2.1.2}(a)) and all generalized eigenvectors corresponding to \(\lambda\) (\cref{7.1.2}).
\end{defn}

\begin{thm}\label{7.1}
  Let \(\T\) be a linear operator on a vector space \(\V\) over \(\F\), and let \(\lambda \in \F\) be an eigenvalue of \(\T\).
  Then
  \begin{enumerate}
    \item \(\vs{K}_{\lambda}\) is a \(\T\)-invariant subspace of \(\V\) containing \(\vs{E}_{\lambda}\)
          (the eigenspace of \(\T\) corresponding to \(\lambda\)).
    \item For any scalar \(\mu \neq \lambda\), the restriction of \(\T - \mu \IT[\V]\) to \(\vs{K}_{\lambda}\) is one-to-one.
  \end{enumerate}
\end{thm}

\begin{proof}[\pf{7.1}(a)]
  Clearly, \(\zv \in \vs{K}_{\lambda}\).
  Suppose that \(x\) and \(y\) are in \(\vs{K}_{\lambda}\).
  Then by \cref{7.1.4} there exist positive integers \(p\) and \(q\) such that
  \[
    (\T - \lambda \IT[\V])^p(x) = (\T - \lambda \IT[\V])^q(y) = \zv.
  \]
  Therefore
  \begin{align*}
    (\T - \lambda \IT[\V])^{p + q}(x + y) & = (\T - \lambda \IT[\V])^{p + q}(x) + (\T - \lambda \IT[\V])^{p + q}(y) &  & \by{2.9}      \\
                                          & = (\T - \lambda \IT[\V])^q(\zv) + (\T - \lambda \IT[\V])^p(\zv)         &  & \by{7.1.4}    \\
                                          & = \zv,                                                                  &  & \by{2.1.2}[a]
  \end{align*}
  and hence \(x + y \in \vs{K}_{\lambda}\).
  Similarly,
  \begin{align*}
    \forall c \in \F, (\T - \lambda \IT[\V])^p(cx) & = c (\T - \lambda \IT[\V])^p(x) &  & \by{2.9}    \\
                                                   & = c \zv                         &  & \by{7.1.4}  \\
                                                   & = \zv                           &  & \by{1.2}[c]
  \end{align*}
  and hence \(cx \in \vs{K}_{\lambda}\).

  To show that \(\vs{K}_{\lambda}\) is \(\T\)-invariant, consider any \(x \in \vs{K}_{\lambda}\).
  Choose a positive integer \(p\) such that \((\T - \lambda \IT[\V])^p(x) = \zv\).
  Then
  \begin{align*}
    (\T - \lambda \IT[\V])^p \T(x) & = \T (\T - \lambda \IT[\V])^p(x) &  & \by{ex:5.4.4} \\
                                   & = \T(\zv)                        &  & \by{7.1.4}    \\
                                   & = \zv.                           &  & \by{2.1.2}[a]
  \end{align*}
  Therefore \(\T(x) \in \vs{K}_{\lambda}\).

  Finally, it is a simple observation that \(\vs{E}_{\lambda}\) is contained in \(\vs{K}_{\lambda}\).
\end{proof}

\begin{proof}[\pf{7.1}(b)]
  Let \(x \in \vs{K}_{\lambda}\) and \((\T - \mu \IT[\V])(x) = \zv\).
  By way of contradiction, suppose that \(x \neq \zv\).
  Let \(p\) be the smallest positive integer for which \((\T - \lambda \IT[\V])^p(x) = \zv\), and let \(y = (\T - \lambda \IT[\V])^{p - 1}(x)\).
  Then
  \[
    (\T - \lambda \IT[\V])(y) = (\T - \lambda \IT[\V])^p(x) = \zv,
  \]
  and hence \(y \in \vs{E}_{\lambda}\) (\cref{7.1.3}).
  Furthermore,
  \begin{align*}
    (\T - \mu \IT[\V])(y) & = (\T - \mu \IT[\V]) (\T - \lambda \IT[\V])^{p - 1}(x)  &  & \by{7.1.3}    \\
                          & = (\T - \lambda \IT[\V])^{p - 1} (\T - \mu \IT[\V]) (x) &  & \by{ex:5.4.4} \\
                          & = \zv,                                                  &  & \by{2.1.2}[a]
  \end{align*}
  so that \(y \in \vs{E}_{\mu}\).
  But by \cref{5.8} \(\vs{E}_{\lambda} \cap \vs{E}_{\mu} = \set{\zv}\), and thus \(y = \zv\), contrary to the hypothesis of \(p\).
  So \(x = \zv\), and the restriction of \(\T - \mu \IT[\V]\) to \(\vs{K}_{\lambda}\) is one-to-one.
\end{proof}

\begin{thm}\label{7.2}
  Let \(\T\) be a linear operator on a finite-dimensional vector space \(\V\) over \(\F\) such that the characteristic polynomial of \(\T\) splits.
  Suppose that \(\lambda\) is an eigenvalue of \(\T\) with multiplicity \(m\).
  Then
  \begin{enumerate}
    \item \(\dim(\vs{K}_{\lambda}) \leq m\).
    \item \(\vs{K}_{\lambda} = \ns{(\T - \lambda \IT[\V])^m}\).
  \end{enumerate}
\end{thm}

\begin{proof}[\pf{7.2}(a)]
  Let \(h\) be the characteristic polynomial of \(\T_{\vs{K}_{\lambda}}\).
  By \cref{5.21}, \(h\) divides the characteristic polynomial of \(\T\), and by \cref{7.1}(b), \(\lambda\) is the only eigenvalue of \(\T_{\vs{K}_{\lambda}}\).
  Hence \(h(t) = (-1)^d (t - \lambda)^d\), where \(d = \dim(\vs{K}_{\lambda})\) (\cref{2.5}), and \(d \leq m\).
\end{proof}

\begin{proof}[\pf{7.2}(b)]
  Clearly \(\ns{(\T - \lambda \IT[\V])^m} \subseteq \vs{K}_{\lambda}\).
  Now let \(h\) and \(d\) be as in (a).
  Then \(h(\T_{\vs{K}_{\lambda}})\) is identically zero by the Cayley--Hamilton theorem (\cref{5.23});
  therefore \((\T - \lambda \IT[\V])^d(x) = \zv\) for all \(x \in \vs{K}_{\lambda}\).
  Since \(d \leq m\), we have \(\vs{K}_{\lambda} \subseteq \ns{(\T - \lambda \IT[\V])^m}\).
\end{proof}

\begin{thm}\label{7.3}
  Let \(\T\) be a linear operator on a finite-dimensional vector space \(\V\) over \(\F\) such that the characteristic polynomial of \(\T\) splits, and let \(\seq{\lambda}{1,,k} \in \F\) be the distinct eigenvalues of \(\T\).
  Then, for every \(x \in \V\), there exist vectors \(v_i \in \vs{K}_{\lambda_i}\), \(i \in \set{1, \dots, k}\), such that
  \[
    x = \seq[+]{v}{1,,k}.
  \]
\end{thm}

\begin{proof}[\pf{7.3}]
  The proof is by mathematical induction on the number \(k\) of distinct eigenvalues of \(\T\).
  First suppose that \(k = 1\), and let \(m\) be the multiplicity of \(\lambda_1\).
  Then \((\lambda_1 - t)^m\) is the characteristic polynomial of \(\T\), and hence \((\lambda_1 \IT[\V] - \T)^m = \zT\) by the Cayley--Hamilton theorem (\cref{5.23}).
  Thus
  \begin{align*}
    \V & = \ns{\zT}                        &  & \by{2.3}    \\
       & = \ns{(\T - \lambda_1 \IT[\V])^m} &  & \by{5.23}   \\
       & = \vs{K}_{\lambda_1},             &  & \by{7.2}(b)
  \end{align*}
  and the result follows.

  Now suppose that for some integer \(k \geq 1\), the result is established whenever \(\T\) has fewer than \(k + 1\) distinct eigenvalues, and suppose that \(\T\) has \(k + 1\) distinct eigenvalues.
  Let \(m\) be the multiplicity of \(\lambda_{k + 1}\), and let \(f\) be the characteristic polynomial of \(\T\).
  Then \(f(t) = (t - \lambda_{k + 1})^m g(t)\) for some polynomial \(g\) not divisible by \((t - \lambda_{k + 1})\).
  Let \(\W = \rg{(\T - \lambda_{k + 1} \IT[\V])^m}\).
  Then \(\W\) is \(\T\)-invariant since
  \begin{align*}
             & \forall y \in \rg{(\T - \lambda_{k + 1} \IT[\V])^m}, \exists x \in \V : (\T - \lambda_{k + 1} \IT[\V])^m(x) = y \\
    \implies & \forall y \in \rg{(\T - \lambda_{k + 1} \IT[\V])^m}, \T(y) = \T ((\T - \lambda_{k + 1} \IT[\V])^m(x))           \\
             & = ((\T - \lambda_{k + 1} \IT[\V])^m \T)(x) = ((\T - \lambda_{k + 1} \IT[\V])^m)(\T(x))                          \\
             & \in \rg{(\T - \lambda_{k + 1} \IT[\V])^m}                                                                       \\
    \implies & \T(\rg{(\T - \lambda_{k + 1} \IT[\V])^m}) \subseteq \rg{(\T - \lambda_{k + 1} \IT[\V])^m}.
  \end{align*}
  We claim that \((\T - \lambda_{k + 1} \IT[\V])^m\) maps \(\vs{K}_{\lambda_i}\) onto itself for \(i \in \set{1, \dots, k}\).
  Suppose that \(i \in \set{1, \dots, k}\).
  First we show that \((\T - \lambda_{k + 1} \IT[\V])^m\) maps \(\vs{K}_{\lambda_i}\) into itself.
  This is true since
  \begin{align*}
             & \forall x \in \vs{K}_{\lambda_i}, \exists p \in \Z^+ : (\T - \lambda_i \IT[\V])^p(x) = \zv        &  & \by{7.1.4}    \\
    \implies & \forall x \in \vs{K}_{\lambda_i}, (\T - \lambda_i \IT[\V])^p((\T - \lambda_{k + 1} \IT[\V])^m(x))                    \\
             & = (\T - \lambda_{k + 1} \IT[\V])^m((\T - \lambda_i \IT[\V])^p(x))                                 &  & \by{ex:5.4.4} \\
             & = \zv                                                                                             &  & \by{2.1.2}[a] \\
    \implies & \forall x \in \vs{K}_{\lambda_i}, (\T - \lambda_{k + 1} \IT[\V])^m(x) \in \vs{K}_{\lambda_i}.     &  & \by{7.1.4}
  \end{align*}
  Now we show the claim is true.
  Since \((\T - \lambda_{k + 1} \IT[\V])^m\) maps \(\vs{K}_{\lambda_i}\) into itself and \(\lambda_{k + 1} \neq \lambda_i\), the restriction of \(\T - \lambda_{k + 1} \IT[\V]\) to \(\vs{K}_{\lambda_i}\) is one-to-one (by \cref{7.1}(b)) and hence is onto (\cref{2.4,2.5}).
  One consequence of this is that for \(i \in \set{1, \dots, k}\), \(\vs{K}_{\lambda_i}\) is contained in \(\W\);
  hence \(\lambda_i\) is an eigenvalue of \(\T_{\W}\) for \(i \in \set{1, \dots, k}\).

  Next, observe that \(\lambda_{k + 1}\) is not an eigenvalue of \(\T_{\W}\).
  For suppose that \(\T(v) = \lambda_{k + 1} v\) for some \(v \in \W\).
  Then \(v = (\T - \lambda_{k + 1} \IT[\V])^m(y)\) for some \(y \in \V\), and it follows from \cref{5.4} that
  \[
    \zv = (\T - \lambda_{k + 1} \IT[\V])(v) = (\T - \lambda_{k + 1} \IT[\V])^{m + 1}(y).
  \]
  Therefore \(y \in \vs{K}_{\lambda_{k + 1}}\).
  So by \cref{7.2}(b), \(v = (\T - \lambda_{k + 1} \IT[\V])^m(y) = \zv\).

  Since every eigenvalue of \(\T_{\W}\) is an eigenvalue of \(\T\), the distinct eigenvalues of \(\T_{\W}\) are \(\seq{\lambda}{1,,k}\).
  For each \(i \in \set{1, \dots, k}\), we defined the generalized eigenspace for \(\T_{\W}\) corresponding to \(\lambda_i\) as follow:
  \[
    \vs{K}_{\lambda_i}' = \set{x \in \W | \exists p \in \Z^+ : (\T_{\W} - \lambda_i)^p(x) = \zv}.
  \]
  Clearly we have \(\vs{K}_{\lambda_i}' \subseteq \vs{K}_{\lambda_i}\) for each \(i \in \set{1, \dots, k}\).

  Now let \(x \in \V\).
  Then \((\T - \lambda_{k + 1} \IT[\V])^m(x) \in \W\).
  Since \(\T_{\W}\) has the \(k\) distinct eigenvalues \(\seq{\lambda}{1,,k}\), the induction hypothesis applies.
  Hence there are vectors \(w_i \in \vs{K}_{\lambda_i}'\), \(i \in \set{1, \dots, k}\), such that
  \[
    (\T - \lambda_{k + 1} \IT[\V])^m(x) = \seq[+]{w}{1,,k}.
  \]
  Since \(\vs{K}_{\lambda_i}' \subseteq \vs{K}_{\lambda_i}\) for \(i \in \set{1, \dots, k}\) and \((\T - \lambda_{k + 1} \IT[\V])^m\) maps \(\vs{K}_{\lambda_i}\) onto itself for \(i \in \set{1, \dots, k}\), there exist vectors \(v_i \in \vs{K}_{\lambda_i}\) such that \((\T - \lambda_{k + 1} \IT[\V])^m(v_i) = w_i\) for \(i \in \set{1, \dots, k}\).
  Thus we have
  \[
    (\T - \lambda_{k + 1} \IT[\V])^m(x) = (\T - \lambda_{k + 1} \IT[\V])^m(v_1) + \cdots + (\T - \lambda_{k + 1} \IT[\V])^m(v_{k}),
  \]
  and it follows that \(x - (\seq[+]{v}{1,,k}) \in \vs{K}_{\lambda_{k + 1}}\) since
  \begin{align*}
    \zv & = (\T - \lambda_{k + 1} \IT[\V])^m(x) - (\T - \lambda_{k + 1} \IT[\V])^m(v_1) + \cdots + (\T - \lambda_{k + 1} \IT[\V])^m(v_{k}) \\
        & = (\T - \lambda_{k + 1} \IT[\V])^m(x - (\seq[+]{v}{1,,k})).
  \end{align*}
  Therefore there exists a vector \(v_k \in \vs{K}_{\lambda_{k + 1}}\) such that
  \[
    x = \seq[+]{v}{1,,k+1}.
  \]
\end{proof}

\begin{thm}\label{7.4}
  Let \(\T\) be a linear operator on a finite-dimensional vector space \(\V\) over \(\F\) such that the characteristic polynomial of \(\T\) splits, and let \(\seq{\lambda}{1,,k} \in \F\) be the distinct eigenvalues of \(\T\) with corresponding multiplicities \(\seq{m}{1,,k}\).
  For \(i \in \set{1, \dots, k}\), let \(\beta_i\) be an ordered basis for \(\vs{K}_{\lambda_i}\) over \(\F\).
  Then the following statements are true.
  \begin{enumerate}
    \item \(\beta_i \cap \beta_j = \varnothing\) for \(i \neq j\).
    \item \(\beta = \seq[\cup]{\beta}{1,,k}\) is an ordered basis for \(\V\) over \(\F\).
    \item \(\dim(\vs{K}_{\lambda_i}) = m_i\) for all \(i \in \set{1, \dots, k}\).
  \end{enumerate}
\end{thm}

\begin{proof}[\pf{7.4}(a)]
  Suppose for sake of contradiction that \(x \in \beta_i \cap \beta_j \subseteq \vs{K}_{\lambda_i} \cap \vs{K}_{\lambda_j}\), where \(i \neq j\).
  By \cref{7.1}(b), \(\T - \lambda_i \IT[\V]\) is one-to-one on \(\vs{K}_{\lambda_j}\), and therefore \((\T - \lambda_i \IT[\V])^p(x) \neq \zv\) for any positive integer \(p\).
  But this contradicts the fact that \(x \in \vs{K}_{\lambda_i}\), and the result follows.
\end{proof}

\begin{proof}[\pf{7.4}(b)]
  Let \(x \in \V\).
  By \cref{7.3}, for \(i \in \set{1, \dots, k}\), there exist vectors \(v_i \in \vs{K}_{\lambda_i}\) such that \(x = \seq[+]{v}{1,,k}\).
  Since each \(v_i\) is a linear combination of the vectors of \(\beta_i\), it follows that \(x\) is a linear combination of the vectors of \(\beta\).
  Therefore \(\beta\) spans \(\V\).
  Let \(q\) be the number of vectors in \(\beta\).
  Then \(\dim(\V) \leq q\).
  For each \(i \in \set{1, \dots, k}\), let \(d_i = \dim(\vs{K}_{\lambda_i})\).
  Then, by \cref{5.3} and \cref{7.2}(a),
  \[
    q = \sum_{i = 1}^k d_i \leq \sum_{i = 1}^k m_i = \dim(\V).
  \]
  Hence \(q = \dim(\V)\).
  Consequently \(\beta\) is a basis for \(\V\) over \(\F\) by \cref{1.6.15}(a).
\end{proof}

\begin{proof}[\pf{7.4}(c)]
  Using the notation and result of (b), we see that \(\sum_{i = 1}^k d_i = \sum_{i = 1}^k m_i\).
  But \(d_i \leq m_i\) by \cref{7.2}(a), and therefore \(d_i = m_i\) for all \(i \in \set{1, \dots, k}\).
\end{proof}

\begin{cor}\label{7.1.5}
  Let \(\T\) be a linear operator on a finite-dimensional vector space \(\V\) over \(\F\) such that the characteristic polynomial of \(\T\) splits.
  Then \(\T\) is diagonalizable iff \(\vs{E}_{\lambda} = \vs{K}_{\lambda}\) for every eigenvalue \(\lambda\) of \(\T\).
\end{cor}

\begin{proof}[\pf{7.1.5}]
  Combining \cref{7.4}(c) and \cref{5.9}(a), we see that \(\T\) is diagonalizable iff \(\dim(\vs{E}_{\lambda}) = \dim(\vs{K}_{\lambda})\) for each eigenvalue \(\lambda\) of \(\T\).
  But \(\vs{E}_{\lambda} \subseteq \vs{K}_{\lambda}\) (\cref{7.1}(a)), and hence these subspaces have the same dimension iff they are equal (\cref{1.11}).
\end{proof}

\section{The Jordan Canonical Form II}\label{sec:7.2}

\begin{defn}\label{7.2.1}
  For the purposes of this section, we fix a linear operator \(\T\) on an \(n\)-dimensional vector space \(\V\) over \(\F\) such that the characteristic polynomial of \(\T\) splits.
  Let \(\seq{\lambda}{1,,k}\) be the distinct eigenvalues of \(\T\).

  By \cref{7.7}, each generalized eigenspace \(\vs{K}_{\lambda_i}\) contains an ordered basis \(\beta_i\) consisting of a union of disjoint cycles of generalized eigenvectors corresponding to \(\lambda_i\).
  So by \cref{7.4}(b) and \cref{7.5}, the union \(\beta = \bigcup_{i = 1}^k \beta_i\) is a Jordan canonical basis for \(\T\).
  For each \(i \in \set{1, \dots, k}\), let \(\T_i\) be the restriction of \(\T\) to \(\vs{K}_{\lambda_i}\), and let \(A_i = [\T_i]_{\beta_i}\).
  Then \(A_i\) is a Jordan canonical form of \(\T_i\), and
  \[
    J = [\T]_{\beta} = \begin{pmatrix}
      A_1    & \zm    & \cdots & \zm    \\
      \zm    & A_2    & \cdots & \zm    \\
      \vdots & \vdots &        & \vdots \\
      \zm    & \zm    & \cdots & A_k
    \end{pmatrix}
  \]
  is a Jordan canonical form of \(\T\).
  In this matrix, each \(\zm\) is a zero matrix of appropriate size.

  In this section, we compute the matrices \(A_i\) and the bases \(\beta_i\), thereby computing \(J\) and \(\beta\) as well.
  While developing a method for finding \(J\), it becomes evident that in some sense the matrices \(A_i\) are unique.

  To aid in formulating the uniqueness theorem for \(J\), we adopt the following convention:
  The basis \(\beta_i\) for \(\vs{K}_{\lambda_i}\) will henceforth be ordered in such a way that the cycles appear in order of decreasing length.
  That is, if \(\beta_i\) is a disjoint union of cycles \(\seq{\gamma}{1,,n_i}\) and if the length of the cycle \(\gamma_j\) is \(p_j\), we index the cycles so that \(\seq[\geq]{p}{1,,n_i}\).
  This ordering of the cycles limits the possible orderings of vectors in \(\beta_i\), which in turn determines the matrix \(A_i\).
  It is in this sense that \(A_i\) is unique.
  It then follows that the Jordan canonical form for \(\T\) is unique up to an ordering of the eigenvalues of \(\T\).
  As we will see, there is no uniqueness theorem for the bases \(\beta_i\) or for \(\beta\)
  (See \cref{ex:7.2.8}, it is for this reason that we associate the dot diagram with \(\T_i\) rather than with \(\beta_i\)).
  Specifically, we show that for each \(i \in \set{1, \dots, k}\), the number \(n_i\) of cycles that form \(\beta_i\), and the length \(p_j\) (\(j \in \set{1, \dots, n_i}\)) of each cycle, is completely determined by \(\T\).

  To help us visualize each of the matrices \(A_i\) and ordered bases \(\beta_i\), we use an array of dots called a \textbf{dot diagram} of \(\T_i\), where \(\T_i\) is the restriction of \(\T\) to \(\vs{K}_{\lambda_i}\).
  Suppose that \(\beta_i\) is a disjoint union of cycles of generalized eigenvectors \(\seq{\gamma}{1,,n_i}\) with lengths \(\seq[\geq]{p}{1,,n_i}\), respectively.
  The dot diagram of \(\T_i\) contains one dot for each vector in \(\beta_i\), and the dots are configured according to the following rules.
  \begin{itemize}
    \item The array consists of \(n_i\) columns (one column for each cycle).
    \item Counting from left to right, the \(j\)th column consists of the \(p_j\) dots that correspond to the vectors of \(\gamma_j\) starting with the initial vector at the top and continuing down to the end vector.
  \end{itemize}

  Denote the end vectors of the cycles by \(\seq{v}{1,,n_i}\).
  In the following dot diagram of \(\T_i\), each dot is labeled with the name of the vector in \(\beta_i\) to which it corresponds.
  \begin{align*}
     & \bullet (\T - \lambda_i \IT[\V])^{p_1 - 1}(v_1) &  & \bullet (\T - \lambda_i \IT[\V])^{p_2 - 1}(v_2) &  & \cdots &  & \bullet (\T - \lambda_i \IT[\V])^{p_{n_i} - 1}(v_{n_i}) \\
     & \bullet (\T - \lambda_i \IT[\V])^{p_1 - 2}(v_1) &  & \bullet (\T - \lambda_i \IT[\V])^{p_2 - 2}(v_2) &  & \cdots &  & \bullet (\T - \lambda_i \IT[\V])^{p_{n_i} - 2}(v_{n_i}) \\
     & \vdots                                          &  & \vdots                                          &  &        &  & \vdots                                                  \\
     &                                                 &  &                                                 &  &        &  & \bullet (\T - \lambda_i \IT[\V])(v_{n_i})               \\
     &                                                 &  &                                                 &  &        &  & \bullet v_{n_i}                                         \\
     &                                                 &  & \bullet (\T - \lambda_i \IT[\V])(v_2)           &  &        &  &                                                         \\
     &                                                 &  & \bullet v_2                                     &  &        &  &                                                         \\
     & \bullet (\T - \lambda_i \IT[\V])(v_1)           &  &                                                 &  &        &  &                                                         \\
     & \bullet v_1                                     &  &                                                 &  &        &  &
  \end{align*}
  Notice that the dot diagram of \(\T_i\) has \(n_i\) columns (one for each cycle) and \(p_1\) rows.
  Since \(\seq[\geq]{p}{1,,n_i}\), the columns of the dot diagram become shorter (or at least not longer) as we move from left to right.

  Now let \(r_j\) denote the number of dots in the \(j\)th row of the dot diagram.
  Observe that \(\seq[\geq]{r}{1,,p_1}\).
  Furthermore, the diagram can be reconstructed from the values of the \(r_i\)'s.
  The proofs of these facts, which are combinatorial in nature, are treated in \cref{ex:7.2.9}.
\end{defn}

\begin{thm}\label{7.9}
  Using the notations in \cref{7.2.1}.
  For any positive integer \(r\), the vectors in \(\beta_i\) that are associated with the dots in the first \(r\) rows of the dot diagram of \(\T_i\) constitute a basis for \(\ns{(\T - \lambda_i \IT[\V])^r}\) over \(\F\).
  Hence the number of dots in the first \(r\) rows of the dot diagram equals \(\nt{(\T - \lambda_i \IT[\V])^r}\).
\end{thm}

\begin{proof}[\pf{7.9}]
  By \cref{7.1.4} \(\ns{(\T - \lambda_i \IT[\V])^r} \subseteq \vs{K}_{\lambda_i}\), and \(\vs{K}_{\lambda_i}\) is invariant under \((\T - \lambda_i \IT[\V])^r\).
  Let \(\U\) denote the restriction of \((\T - \lambda_i \IT[\V])^r\) to \(\vs{K}_{\lambda_i}\).
  By \cref{7.1}(b), \(\ns{(\T - \lambda_i \IT[\V])^r} = \ns{\U}\), and hence it suffices to establish the theorem for \(\U\).
  Now define
  \[
    S_1 = \set{x \in \beta_i : \U(x) = \zv} \quad \text{and} \quad S_2 = \set{x \in \beta_i : \U(x) \neq \zv}.
  \]
  Let \(a\) and \(b\) denote the number of vectors in \(S_1\) and \(S_2\), respectively, and let \(m_i = \dim(\vs{K}_{\lambda_i})\).
  Then \(a + b = m_i\).
  For any \(x \in \beta_i\), \(x \in S_1\) iff \(x\) is one of the first \(r\) vectors of a cycle (\cref{7.1.6}), and this is true iff \(x\) corresponds to a dot in the first \(r\) rows of the dot diagram (\cref{7.2.1}).
  Hence \(a\) is the number of dots in the first \(r\) rows of the dot diagram.
  For any \(x \in S_2\), the effect of applying \(\U\) to \(x\) is to move the dot corresponding to \(x\) exactly \(r\) places up its column to another dot.
  It follows that \(\U\) maps \(S_2\) in a one-to-one fashion into \(\beta_i\) (since \(\beta_i\) is linearly independent).
  Thus \(\set{\U(x) : x \in S_2}\) is a basis for \(\rg{\U}\) over \(\F\) consisting of \(b\) vectors (\cref{2.2}).
  Hence \(\rk{\U} = b\), and so \(\nt{\U} = m_i - b = a\) (\cref{2.3}).
  But \(S_1\) is a linearly independent subset of \(\ns{\U}\) consisting of \(a\) vectors;
  therefore \(S_1\) is a basis for \(\ns{\U}\).
\end{proof}

\begin{note}
  In the case that \(r = 1\), \cref{7.9} yields \cref{7.2.2}.
\end{note}

\begin{cor}\label{7.2.2}
  Using the notations in \cref{7.2.1}.
  The dimension of \(\vs{E}_{\lambda_i}\) is \(n_i\).
  Hence in a Jordan canonical form of \(\T\), the number of Jordan blocks corresponding to \(\lambda_i\) equals the dimension of \(\vs{E}_{\lambda_i}\).
\end{cor}

\begin{proof}[\pf{7.2.2}]
  By \cref{5.4} we have \(\vs{E}_{\lambda_i} = \ns{\T - \lambda_i \IT[\V]}\).
  Thus by \cref{7.9} the first row of the dot diagram of \(\T_i\) constitute a basis for \(\vs{E}_{\lambda_i}\) over \(\F\).
\end{proof}

\begin{thm}\label{7.10}
  Using the notations in \cref{7.2.1}.
  Let \(r_j\) denote the number of dots in the \(j\)th row of the dot diagram of \(\T_i\), the restriction of \(\T\) to \(\vs{K}_{\lambda_i}\).
  Then the following statements are true.
  \begin{enumerate}
    \item \(r_1 = \dim(\V) - \rk{\T - \lambda_i \IT[\V]}\).
    \item \(r_j = \rk{(\T - \lambda_i \IT[\V])^{j - 1}} - \rk{(\T - \lambda_i \IT[\V])^j}\) if \(j \in \set{2, \dots, p_1}\).
  \end{enumerate}
\end{thm}

\begin{proof}[\pf{7.10}]
  By \cref{7.9}, for \(j \in \set{1, \dots, p_1}\), we have
  \begin{align*}
    \seq[+]{r}{1,,j} & = \nt{(\T - \lambda_i \IT[\V])^j}             &  & \by{7.9} \\
                     & = \dim(\V) - \rk{(\T - \lambda_i \IT[\V])^j}. &  & \by{2.3}
  \end{align*}
  Hence \(r_1 = \dim(\V) - \rk{\T - \lambda_i \IT[\V]}\), and for \(j \in \set{2, \dots, p_1}\),
  \begin{align*}
    r_j & = (\seq[+]{r}{1,,j}) - (\seq[+]{r}{1,,j-1})                                                               \\
        & = \pa{\dim(\V) - \rk{(\T - \lambda_i \IT[\V])^j}} - \pa{\dim(\V) - \rk{(\T - \lambda_i \IT[\V])^{j - 1}}} \\
        & = \rk{(\T - \lambda_i \IT[\V])^{j - 1}} - \rk{(\T - \lambda_i \IT[\V])^j}.
  \end{align*}
\end{proof}

\begin{cor}\label{7.2.3}
  Using the notations in \cref{7.2.1}.
  For any eigenvalue \(\lambda_i\) of \(\T\), the dot diagram of \(\T_i\) is unique.
  Thus, subject to the convention that the cycles of generalized eigenvectors for the bases of each generalized eigenspace are listed in order of decreasing length, the Jordan canonical form of a linear operator or a matrix is unique up to the ordering of the eigenvalues.
\end{cor}

\begin{proof}[\pf{7.2.3}]
  \cref{7.10} shows that the dot diagram of \(\T_i\) is completely determined by \(\T\) and \(\lambda_i\).
\end{proof}

\begin{thm}\label{7.11}
  Let \(A\) and \(B\) be \(n \times n\) matrices, each having Jordan canonical forms computed according to the conventions of this section.
  Then \(A\) and \(B\) are similar iff they have (up to an ordering of their eigenvalues) the same Jordan canonical form.
\end{thm}

\begin{proof}[\pf{7.11}]
  If \(A\) and \(B\) have the same Jordan canonical form \(J\), then \(A\) and \(B\) are each similar to \(J\) and hence are similar to each other.

  Conversely, suppose that \(A\) and \(B\) are similar.
  Then \(A\) and \(B\) have the same eigenvalues (\cref{ex:5.1.12}).
  Let \(J_A\) and \(J_B\) denote the Jordan canonical forms of \(A\) and \(B\), respectively, with the same ordering of their eigenvalues.
  Then \(A\) is similar to both \(J_A\) and \(J_B\), and therefore, by the \cref{2.5.3}, \(J_A\) and \(J_B\) are matrix representations of \(\L_A\).
  Hence \(J_A\) and \(J_B\) are Jordan canonical forms of \(\L_A\). Thus \(J_A = J_B\) by the \cref{7.2.3}.
\end{proof}

\begin{cor}\label{7.2.4}
  A linear operator \(\T\) on a finite-dimensional vector space \(\V\) over \(\F\) is diagonalizable iff its Jordan canonical form is a diagonal matrix.
  Hence \(\T\) is diagonalizable iff the Jordan canonical basis for \(\T\) consists of eigenvectors of \(\T\).
\end{cor}

\begin{proof}[\pf{7.2.4}]
  By \cref{7.1.5,7.2.3} we see that this is true.
\end{proof}

\exercisesection

\setcounter{ex}{5}
\begin{ex}\label{ex:7.2.6}
  Let \(A \in \ms[n][n][\F]\) whose characteristic polynomial splits.
  Prove that \(A\) and \(\tp{A}\) have the same Jordan canonical form, and conclude that \(A\) and \(\tp{A}\) are similar.
\end{ex}

\begin{proof}[\pf{ex:7.2.6}]
  Let \(\lambda \in \F\) be an eigenvalue of \(A\).
  Since
  \begin{align*}
    \forall r \in \Z^+, \rk{\pa{A - \lambda I_n}^r} & = \rk{\tp{\pa{(A - \lambda I_n)^r}}} &  & \by{3.2.5}[a] \\
                                                    & = \rk{\pa{\tp{(A - \lambda I_n)}}^r} &  & \by{2.3.2}    \\
                                                    & = \rk{\pa{\tp{A} - \lambda I_n}^r},  &  & \by{ex:1.3.3}
  \end{align*}
  by \cref{7.10} we see that the dot diagrams of \(A\) and \(\tp{A}\) correspond to \(\lambda\) are the same.
  Since \(\lambda\) is arbitrary, by \cref{7.2.3} we conclude that \(A\) and \(\tp{A}\) have the same Jordan canonical form.
  By \cref{7.11} this means \(A\) and \(\tp{A}\) are similar.
\end{proof}

\begin{ex}\label{ex:7.2.7}
  Let \(\T\) be a linear operator on a finite-dimensional vector space \(\V\) over \(\F\) such that the characteristic polynomial of \(\T\) splits.
  Let \(\gamma\) be a cycle of generalized eigenvectors corresponding to an eigenvalue \(\lambda\), and \(\W\) be the subspace spanned by \(\gamma\).
  Define \(\gamma'\) to be the ordered set obtained from \(\gamma\) by reversing the order of the vectors in \(\gamma\).
  \begin{enumerate}
    \item Prove that \([\T_{\W}]_{\gamma'} = \tp{([\T_{\W}]_{\gamma})}\).
    \item Let \(J\) be the Jordan canonical form of \(\T\).
          Use (a) to prove that \(J\) and \(\tp{J}\) are similar.
    \item Let \(A \in \ms[n][n][\F]\) whose characteristic polynomial splits.
          Use (b) to prove that \(A\) and \(\tp{A}\) are similar.
  \end{enumerate}
\end{ex}

\begin{proof}[\pf{ex:7.2.7}(a)]
  Let \(\gamma = \set{\seq{v}{1,,m}}\).
  By \cref{7.1.6} we have
  \[
    \forall i \in \set{1, \dots, m}, \T(v_i) = \begin{dcases}
      \lambda v_i             & \text{if } i = 1                   \\
      \lambda v_i + v_{i - 1} & \text{if } i \in \set{2, \dots, m}
    \end{dcases}.
  \]
  Now define \(\gamma' = \set{v_1', \dots, v_m'}\).
  By definition we have \(v_i' = v_{m + 1 - i}\) for all \(i \in \set{1, \dots, m}\) and
  \begin{align*}
    \forall i \in \set{1, \dots, m}, \T(v_i') & = \T(v_{m + 1 - i})                                                              \\
                                              & = \begin{dcases}
                                                    \lambda v_{m + 1 - i}             & \text{if } m + 1 - i = 1                   \\
                                                    \lambda v_{m + 1 - i} + v_{m - i} & \text{if } m + 1 - i \in \set{2, \dots, m}
                                                  \end{dcases} \\
                                              & = \begin{dcases}
                                                    \lambda v_i'              & \text{if } i = m                       \\
                                                    \lambda v_i' + v_{i + 1}' & \text{if } i \in \set{1, \dots, m - 1}
                                                  \end{dcases}.
  \end{align*}
  Thus we have
  \begin{align*}
    \forall i, j \in \set{1, \dots, m}, \pa{\tp{([\T_{\W}]_{\gamma})}}_{i j} & = ([\T_{\W}]_{\gamma})_{j i}      &  & \by{1.3.3} \\
                                                                             & = \begin{dcases}
                                                                                   \lambda & \text{if } i = j     \\
                                                                                   1       & \text{if } i + 1 = j \\
                                                                                   0       & \text{otherwise}
                                                                                 \end{dcases} &  & \by{7.1.1}                  \\
                                                                             & = ([\T_{\W}]_{\gamma'})_{i j}.    &  & \by{2.2.4}
  \end{align*}
  By \cref{1.2.8} this means \(\tp{([\T_{\W}]_{\beta})} = [\T_{\W}]_{\gamma'}\).
\end{proof}

\begin{proof}[\pf{ex:7.2.7}(b)]
  Let \(\beta\) be an Jordan canonical basis following the convention in \cref{7.2.1}.
  Let \(J = [\T]_{\beta}\).
  Let \(\beta'\) be the ordered set obtained from \(\beta\) by reversing the ordered of each disjoint cycles in \(\beta\).
  By \cref{5.25} and \cref{ex:7.2.7}(a) we see that \([\T]_{\beta'} = \tp{([\T]_{\beta})} = \tp{J}\).
  By \cref{2.23} we hve \([\T]_{\beta'} = \pa{[\IT[\V]]_{\beta'}^{\beta}}^{-1} [\T]_{\beta} [\IT[\V]]_{\beta'}^{\beta}\).
  Thus by \cref{2.5.4} \(J\) and \(\tp{J}\) are similar.
\end{proof}

\begin{proof}[\pf{ex:7.2.7}(c)]
  Let \(\beta\) be the standard ordered basis for \(\vs{F}^n\) over \(\F\) and let \(\alpha\) be a Jordan canonical basis for \(A\).
  By \cref{ex:7.2.7}(b) we see that \([\L_A]_{\alpha}\) and \(\tp{([\L_A]_{\alpha})}\) are similar.
  By \cref{2.23} we know that \(A = [\L_A]_{\beta}\) and \([\L_A]_{\alpha}\) are similar.
  Thus by \cref{ex:2.5.9} we know that \(A\) and \(\tp{([\L_A]_{\alpha})}\) are similar.
  If we can show that \(\tp{A}\) and \(\tp{([\L_A]_{\alpha})}\) are similar, then by \cref{ex:2.5.9} again we see that \(A\) and \(\tp{A}\) are similar.
  This is true since
  \begin{align*}
    \tp{A} & = \tp{([\L_A]_{\beta})}                                                                                       &  & \by{2.15}[a]  \\
           & = \tp{\pa{\pa{[\IT[\V]]_{\beta}^{\alpha}}^{-1} [\L_A]_{\alpha} [\IT[\V]]_{\beta}^{\alpha}}}                   &  & \by{2.23}     \\
           & = \tp{([\IT[\V]]_{\beta}^{\alpha})} \tp{([\L_A]_{\alpha})} \tp{\pa{\pa{[\IT[\V]]_{\beta}^{\alpha}}^{-1}}}     &  & \by{2.3.2}    \\
           & = \tp{\pa{\pa{[\IT[\V]]_{\alpha}^{\beta}}^{-1}}} \tp{([\L_A]_{\alpha})} \tp{\pa{[\IT[\V]]_{\alpha}^{\beta}}}  &  & \by{2.23}     \\
           & = \pa{\tp{\pa{[\IT[\V]]_{\alpha}^{\beta}}}}^{-1} \tp{([\L_A]_{\alpha})} \tp{\pa{[\IT[\V]]_{\alpha}^{\beta}}}. &  & \by{ex:2.4.5}
  \end{align*}
\end{proof}

\begin{ex}\label{ex:7.2.8}
  Let \(\T\) be a linear operator on a finite-dimensional vector space \(\V\) over \(\F\), and suppose that the characteristic polynomial of \(\T\) splits.
  Let \(\beta\) be a Jordan canonical basis for \(\T\).
  \begin{enumerate}
    \item Prove that for any nonzero scalar \(c\), \(\set{cx : x \in \beta}\) is a Jordan canonical basis for \(\T\).
    \item Suppose that \(\gamma\) is one of the cycles of generalized eigenvectors that forms \(\beta\), and suppose that \(\gamma\) corresponds to the eigenvalue \(\lambda\) and has length greater than \(1\).
          Let \(x\) be the end vector of \(\gamma\), and let \(y\) be a nonzero vector in \(\vs{E}_{\lambda}\).
          Let \(\gamma'\) be the ordered set obtained from \(\gamma\) by replacing \(x\) by \(x + y\).
          Prove that \(\gamma'\) is a cycle of generalized eigenvectors corresponding to \(\lambda\), and that if \(\gamma'\) replaces \(\gamma\) in the union that defines \(\beta\), then the new union is also a Jordan canonical basis for \(\T\).
  \end{enumerate}
\end{ex}

\begin{proof}[\pf{ex:7.2.8}(a)]
  Let \(\lambda\) be an eigenvalue of \(\T\) and let \(x \in \beta\) be an generalized eigenvector of \(\T\) corresponding to \(\lambda\).
  If \(x \in \vs{E}_{\lambda}\), then by \cref{5.1.2} we have \(\T(x) = \lambda x\).
  If \(x \in \vs{K}_{\lambda} \setminus \vs{E}_{\lambda}\), then by \cref{7.1.1} there exists a \(v \in \beta\) such that \(\T(x) = \lambda x + v\).
  In either cases we have
  \[
    \T(cx) = c \T(x) = \begin{dcases}
      c \lambda x \\
      c \lambda x + cv
    \end{dcases} = \begin{dcases}
      \lambda (cx) \\
      \lambda (cx) + (cv)
    \end{dcases}.
  \]
  Thus by \cref{7.1.1} \(\set{cx : x \in \beta}\) is a Jordan canonical basis for \(\T\).
\end{proof}

\begin{proof}[\pf{ex:7.2.8}(b)]
  Since
  \begin{align*}
    (\T - \lambda \IT[\V])(x + y) & = (\T - \lambda \IT[\V])(x) + (\T - \lambda \IT[\V])(y) &  & \by{2.1.1}[a] \\
                                  & = (\T - \lambda \IT[\V])(x) + \zv                       &  & \by{5.4}      \\
                                  & = (\T - \lambda \IT[\V])(x),                            &  & \by{1.2.1}
  \end{align*}
  we see that \((\T - \lambda \IT[\V])^p(x + y) = (\T - \lambda \IT[\V])^p(x)\) for any \(p \in \Z^+\).
  Thus \(\gamma'\) is a cycle of generalized eigenvectors corresponding to \(\lambda\), and the rest claim follows.
\end{proof}

\begin{ex}\label{ex:7.2.9}
  Suppose that a dot diagram has \(k\) columns and \(m\) rows with \(p_j\) dots in column \(j\) and \(r_i\) dots in row \(i\).
  Prove the following results.
  \begin{enumerate}
    \item \(m = p_1\) and \(k = r_1\).
    \item We have
          \begin{align*}
             & \forall j \in \set{1, \dots, k}, p_j = \max \set{i \in \set{1, \dots, m} : r_i \geq j}; \\
             & \forall i \in \set{1, \dots, m}, r_i = \max \set{j \in \set{1, \dots, k} : p_j \geq i}.
          \end{align*}
    \item \(\seq[\geq]{r}{1,,m}\).
    \item Deduce that the number of dots in each column of a dot diagram is completely determined by the number of dots in the rows.
  \end{enumerate}
\end{ex}

\begin{proof}[\pf{ex:7.2.9}(a)]
  By \cref{7.2.1} we know that \(p_1\) is the column with the most number of dots.
  Thus \(m = p_1\).
  Since the number of columns equals the number of disjoint cycles, we have \(k = r_1\).
\end{proof}


\begin{proof}[\pf{ex:7.2.9}(b)]
  We first fix \(k\) and use induction on \(m\) to prove that
  \[
    \forall i \in \set{1, \dots, m}, r_i = \max \set{j \in \set{1, \dots, k} : p_j \geq i}.
  \]
  For \(m = 1\), our dot diagram have \(1\) row and \(k\) columns.
  This means the first row has \(k\) dots and each column has one dot.
  Thus
  \begin{align*}
    \forall i \in \set{1}, r_i & = k                                                    \\
                               & = \max \set{1, \dots, k}                               \\
                               & = \max \set{j \in \set{1, \dots, k} : 1 = p_j \geq i}.
  \end{align*}
  So the base case holds.
  Suppose inductively that for some \(m \geq 1\) the statement is true.
  We need to show that for \(m + 1\) the statement is also true.
  So suppose that there are \(k\) columns and \(m + 1\) rows in a dot diagram.
  By \cref{7.2.1} we see that columns in dot diagram are ordered by decreasing length.
  Thus the first \(r_{m + 1}\) columns are the longest columns of all.
  Since there are \(m + 1\) rows, we know that the first \(r_{m + 1}\) columns have \(m + 1\) dots and the rest \((k - r_{m + 1})\) columns have less than \(m + 1\) dots.
  Thus we have
  \begin{align*}
             & \begin{dcases}
                 \forall j \in \set{1, \dots, r_{m + 1}}, p_j = m + 1 \\
                 \forall j \in \set{r_{m + 1} + 1, \dots, k}, p_j < m + 1
               \end{dcases} \\
    \implies & r_{m + 1} = \max \set{1, \dots, r_{m + 1}}                                   \\
             & = \max \set{j \in \set{1, \dots, r_{m + 1}} : p_j \geq m + 1}                \\
             & = \max \set{j \in \set{1, \dots, k} : p_j \geq m + 1}.
  \end{align*}
  By induction hypothesis we have
  \[
    \forall i \in \set{1, \dots, m + 1}, r_i = \max \set{j \in \set{1, \dots, k} : p_j \geq i}.
  \]
  This closes the induction.

  Now we fix \(m\) and use induction on \(k\) to prove that
  \[
    \forall j \in \set{1, \dots, k}, p_j = \max \set{i \in \set{1, \dots, m} : r_i \geq j}.
  \]
  For \(k = 1\), our dot diagram have \(m\) rows and \(1\) column.
  This means the first column has \(m\) dots and each row has one dot.
  Thus
  \begin{align*}
    \forall j \in \set{1}, p_j & = m                                                    \\
                               & = \max \set{1, \dots, m}                               \\
                               & = \max \set{i \in \set{1, \dots, m} : 1 = r_i \geq j}.
  \end{align*}
  So the base case holds.
  Suppose inductively that for some \(k \geq 1\) the statement is true.
  We need to show that for \(k + 1\) the statement is also true.
  So suppose that there are \(k + 1\) columns and \(m\) rows in a dot diagram.
  By \cref{7.2.1} we see that columns in dot diagram are ordered by decreasing length.
  Thus the first \(p_{k + 1}\) rows are the longest rows of all.
  Since there are \(k + 1\) columns, we know that the first \(p_{k + 1}\) rows have \(k + 1\) dots and the rest \((m - p_{k + 1})\) rows have less than \(k + 1\) dots.
  Thus we have
  \begin{align*}
             & \begin{dcases}
                 \forall i \in \set{1, \dots, p_{k + 1}}, r_i = k + 1 \\
                 \forall i \in \set{p_{k + 1} + 1, \dots, m}, r_i < k + 1
               \end{dcases} \\
    \implies & p_{k + 1} = \max \set{1, \dots, p_{k + 1}}                                   \\
             & = \max \set{i \in \set{1, \dots, p_{k + 1}} : r_i \geq k + 1}                \\
             & = \max \set{i \in \set{1, \dots, m} : r_i \geq k + 1}.
  \end{align*}
  By induction hypothesis we have
  \[
    \forall j \in \set{1, \dots, k + 1}, p_j = \max \set{i \in \set{1, \dots, m} : r_i \geq j}.
  \]
  This closes the induction.
\end{proof}

\begin{proof}[\pf{ex:7.2.9}(c)]
  Suppose that \(i_1, i_2 \in \set{1, \dots, m}\) and \(i_1 < i_2\).
  Since
  \begin{align*}
             & i_1 < i_2                                                                                                                      \\
    \implies & \set{j \in \set{1, \dots, k} : p_j \geq i_2} \subseteq \set{j \in \set{1, \dots, k} : p_j \geq i_1}                            \\
    \implies & \max \set{j \in \set{1, \dots, k} : p_j \geq i_2} \leq \max \set{j \in \set{1, \dots, k} : p_j \geq i_1} &  & \by{7.2.1}       \\
    \implies & r_{i_2} \leq r_{i_1}                                                                                     &  & \by{ex:7.2.9}[b]
  \end{align*}
  and \(i_1, i_2\) are arbitrary, we have \(\seq[\geq]{r}{1,,m}\).
\end{proof}

\begin{proof}[\pf{ex:7.2.9}(d)]
  By \cref{ex:7.2.9}(b) we have
  \[
    \forall j \in \set{1, \dots, k}, p_j = \max{i \in \set{1, \dots, m} : r_i \geq j}.
  \]
  This means when number of dots in the rows of a dot diagram is defined, the number of dots in the columns of the same dot diagram is also defined.
\end{proof}

\begin{ex}\label{ex:7.2.10}
  Let \(\T\) be a linear operator whose characteristic polynomial splits, and let \(\lambda\) be an eigenvalue of \(\T\).
  \begin{enumerate}
    \item Prove that \(\dim(\vs{K}_{\lambda})\) is the sum of the lengths of all the cycles corresponding to \(\lambda\) in the Jordan canonical form of \(\T\).
    \item Deduce that \(\vs{E}_{\lambda} = \vs{K}_{\lambda}\) iff all the Jordan blocks corresponding to \(\lambda\) are \(1 \times 1\) matrices.
  \end{enumerate}
\end{ex}

\begin{proof}[\pf{ex:7.2.10}(a)]
  By \cref{7.6} we know that the union of disjoint cycles are linearly independent.
  By \cref{7.5} each cycle forms a Jordan block corresponding to \(\lambda\), and the union of cycles is a basis for \(\vs{K}_{\lambda}\).
  Thus \(\dim(\vs{K}_{\lambda})\) is the sum of the lengths of all cycles corresponding to \(\lambda\).
\end{proof}

\begin{proof}[\pf{ex:7.2.10}(b)]
  We have
  \begin{align*}
         & \vs{E}_{\lambda} = \vs{K}_{\lambda}                                                         \\
    \iff & \text{each cycle has length } 1                                             &  & \by{5.4}   \\
    \iff & \text{all Jordan blocks corresponding to } \lambda \text{ are } 1 \times 1. &  & \by{7.1.1}
  \end{align*}
\end{proof}

\begin{defn}\label{7.2.5}
  A linear operator \(\T\) on a vector space \(\V\) is called \textbf{nilpotent} if \(\T^p = \zT\) for some positive integer \(p\).
  An \(n \times n\) matrix \(A\) is called \textbf{nilpotent} if \(A^p = \zm\) for some positive integer \(p\).
\end{defn}

\begin{ex}\label{ex:7.2.11}
  Let \(\T\) be a linear operator on a finite-dimensional vector space \(\V\) over \(\F\), and let \(\beta\) be an ordered basis for \(\V\) over \(\F\).
  Prove that \(\T\) is nilpotent iff \([\T]_{\beta}\) is nilpotent.
\end{ex}

\begin{proof}[\pf{ex:7.2.11}]
  We have
  \begin{align*}
         & \T \text{ is nilpotent}                                                                            \\
    \iff & \exists p \in \Z^+ : \T^p = \zT                                              &  & \by{7.2.5}       \\
    \iff & \exists p \in \Z^+ : ([\T]_{\beta})^p = [\T^p]_{\beta} = [\zT]_{\beta} = \zm &  & \by{2.11,2.1.13} \\
    \iff & [\T]_{\beta} \text{ is nilpotent}.                                           &  & \by{7.2.5}
  \end{align*}
\end{proof}

\begin{ex}\label{ex:7.2.12}
  Prove that any square upper triangular matrix with each diagonal entry equal to zero is nilpotent.
\end{ex}

\begin{proof}[\pf{ex:7.2.12}]
  Let \(A \in \ms[n][n][\F]\) be an upper triangular matrix with each diagonal entry equal to zero.
  Let \(\beta = \set{\seq{e}{1,,n}}\) be the standard ordered basis for \(\vs{F}^n\) over \(\F\).
  By \cref{2.2.4} we have \(\L_A(e_1) = \zv\) and
  \[
    \forall i \in \set{2, \dots, n}, \L_A(e_i) = \sum_{k = 1}^{i - 1} A_{k i} e_k.
  \]
  This means \(\L_A^n(e_i) = \zv\) for all \(i \in \set{1, \dots, n}\).
  Thus by \cref{2.1.13} \(\L_A^p = \zT\).
  By \cref{7.2.5} this means \(\L_A\) is nilpotent.
  By \cref{ex:7.2.11} we conclude that \(A\) is nilpotent.
\end{proof}

\begin{ex}\label{ex:7.2.13}
  Let \(\T\) be a nilpotent operator on an \(n\)-dimensional vector space \(\V\) over \(\F\), and suppose that \(p\) is the smallest positive integer for which \(\T^p = \zT\).
  Prove the following results.
  \begin{enumerate}
    \item \(\ns{\T^i} \subseteq \ns{\T^{i + 1}}\) for every positive integer \(i\).
    \item There is a sequence of ordered bases \(\seq{\beta}{1,,p}\) such that \(\beta_i\) is a basis for \(\ns{\T^i}\) over \(\F\) and \(\beta_{i + 1}\) contains \(\beta_i\) for \(i \in \set{1, \dots, p - 1}\).
    \item Let \(\beta = \beta_p\) be the ordered basis for \(\ns{\T^p} = \V\) over \(\F\) in (b).
          Then \([\T]_{\beta}\) is an upper triangular matrix with each diagonal entry equal to zero.
    \item The characteristic polynomial of \(\T\) is \((-1)^n t^n\).
          Hence the characteristic polynomial of \(\T\) splits, and \(0\) is the only eigenvalue of \(\T\).
  \end{enumerate}
\end{ex}

\begin{proof}[\pf{ex:7.2.13}(a)]
  See \cref{ex:7.1.7}(a).
\end{proof}

\begin{proof}[\pf{ex:7.2.13}(b)]
  Let \(\beta_1\) be a basis for \(\ns{\T}\) over \(\F\).
  Since \(\beta_1 \subseteq \ns{\T} \subseteq \ns{\T^2}\), by \cref{1.6.15}(c) we can extend \(\beta_1\) to an ordered basis \(\beta_2\) for \(\ns{\T^2}\) over \(\F\).
  Similarly we can extend \(\beta_2\) to an ordered basis \(\beta_3\) for \(\ns{\T^3}\) over \(\F\).
  Continue this process we can construct the sequence \(\seq{\beta}{1,,p}\) which satisfy the requirement of \cref{ex:7.2.13}(b).
\end{proof}

\begin{proof}[\pf{ex:7.2.13}(c)]
  Let \(i \in \set{1, \dots, p - 1}\).
  By \cref{ex:7.1.7}(c) we have \(\ns{\T^i} \neq \ns{\T^{i + 1}}\).
  Thus
  \begin{align*}
             & \beta_{i + 1} \setminus \beta_i \neq \varnothing                   &  & \by{ex:7.2.13}[b]                 \\
    \implies & \forall v \in \beta_{i + 1} \setminus \beta_i, \begin{dcases}
                                                                \T^i(v) \neq \zv \\
                                                                \T^{i + 1}(v) = \zv
                                                              \end{dcases}      &  & \by{2.1.10}                         \\
    \implies & \forall v \in \beta_{i + 1} \setminus \beta_i, \T(v) \in \ns{\T^i} &  & \by{2.1.10}                       \\
    \implies & \forall v \in \beta_{i + 1}, \T(v) \in \ns{\T^i}                   &  & (\beta_i \subseteq \beta_{i + 1}) \\
    \implies & \T(\beta_{i + 1}) \subseteq \ns{\T^i} = \spn{\beta_i}.             &  & \by{ex:7.2.13}[b]
  \end{align*}
  By \cref{2.2.4} this means \([\T]_{\beta}\) is an upper triangular matrix with each diagonal entry equal to \(0\).
\end{proof}

\begin{proof}[\pf{ex:7.2.13}(d)]
  By \cref{ex:7.2.13}[c] we know that \([\T]_{\beta}\) is an upper triangular matrix with each diagonal entry equal to \(0\).
  Thus by \cref{ex:4.2.23} we see that the characteristic polynomial of \(\T\) is \((-1)^n t^n\).
  By \cref{5.2} we know that \(0\) is the only eigenvalue of \(\T\).
\end{proof}

\begin{ex}\label{ex:7.2.14}
  Prove the converse of \cref{ex:7.2.13}(d):
  If \(\T\) is a linear operator on an \(n\)-dimensional vector space \(\V\) over \(\F\) and \((-1)^n t^n\) is the characteristic polynomial of \(\T\), then \(\T\) is nilpotent.
\end{ex}

\begin{proof}[\pf{ex:7.2.14}]
  Since the characteristic polynomial of \(\T\) splits, by \cref{7.1.8} we know that \(\T\) has a Jordan form and a Jordan canonical basis \(\beta\).
  By \cref{5.2} we know that \(0\) is the only eigenvalue of \(\T\), thus by \cref{7.1.1} \([\T]_{\beta}\) is an upper triangular matrix with each diagonal entry equal to \(0\).
  By \cref{ex:7.2.12} we know that \([\T]_{\beta}\) is nilpotent.
  Thus by \cref{ex:7.2.11} we conclude that \(\T\) is nilpotent.
\end{proof}

\begin{ex}\label{ex:7.2.15}
  Give an example of a linear operator \(\T\) on a finite-dimensional vector space \(\V\) over \(\F\) such that \(\T\) is not nilpotent, but zero is the only eigenvalue of \(\T\).
  Characterize all such operators.
\end{ex}

\begin{proof}[\pf{ex:7.2.15}]
  First we give an example as required.
  Let \(\V = \R^3\) and let \(\F = \R\).
  Define
  \[
    A = \begin{pmatrix}
      0 & 0  & 0 \\
      0 & 0  & 1 \\
      0 & -1 & 0
    \end{pmatrix}.
  \]
  Then \(\det(A - t I_3) = (-t)(t^2 + 1)\).
  Thus the characteristic polynomial of \(A\) does not split and \(0\) is the only eigenvalue of \(A\).
  Since \(A^3 = -A\), we know that \(A^p \neq \zm\) for all \(p \in \Z^+\).
  Thus by \cref{7.2.5} \(A\) is not nilpotent.

  Now we characterize all such operator.
  By \cref{ex:7.2.13}(d) and \cref{ex:7.2.14} we know that \(\T\) is nilpotent iff the characteristic polynomial of \(\T\) is \((-1)^n t^n\).
  Thus \(\T\) is not nilpotent iff the characteristic polynomial of \(\T\) is not \((-1)^n t^n\).
\end{proof}

\begin{ex}\label{ex:7.2.16}
  Let \(\T\) be a nilpotent linear operator on a finite-dimensional vector space \(\V\) over \(\F\).
  Recall from \cref{ex:7.2.13} that \(\lambda = 0\) is the only eigenvalue of \(\T\), and hence \(\V = \vs{K}_0\).
  Let \(\beta\) be a Jordan canonical basis for \(\T\).
  Prove that for any positive integer \(i\), if we delete from \(\beta\) the vectors corresponding to the last \(i\) dots in each column of a dot diagram of \(\beta\), the resulting set is a basis for \(\rg{\T^i}\) over \(\F\).
  (If a column of the dot diagram contains fewer than \(i\) dots, all the vectors associated with that column are removed from \(\beta\).)
\end{ex}

\begin{proof}[\pf{ex:7.2.16}]
  Let \(\gamma\) be a cycle in \(\beta\) with length \(p\) and end vector \(x\).
  By \cref{7.1.6} we have
  \[
    \gamma = \set{\T^{p - 1}(x), \dots, \T^i(x), \T^{i - 1}(x), \dots, \T(x), x}.
  \]
  Let \(\alpha\) be the set obtained from removing the last \(i\) vector in \(\gamma\), i.e.,
  \[
    \alpha = \set{\T^{p - 1}(x), \dots, \T^i(x)}.
  \]
  Note that \(\alpha = \varnothing\) when \(p \leq i\).
  If \(\alpha \neq \varnothing\), then each vector in \(\alpha\) is mapped by \(\T^i\) from exactly one vector in \(\gamma \setminus \alpha\).
  Thus by \cref{2.2} \(\alpha\) is linearly independent and the union of all \(\alpha\) is a basis for \(\rg{\T^i}\) over \(\F\).
\end{proof}

\begin{ex}\label{ex:7.2.20}
\end{ex}

\begin{ex}\label{ex:7.2.21}
\end{ex}

\section{The Minimal Polynomial}\label{sec:7.3}

\begin{defn}\label{7.3.1}
  Let \(\T\) be a linear operator on a finite-dimensional vector space.
  A polynomial \(p\) is called a \textbf{minimal polynomial} of \(\T\) if \(p\) is a monic polynomial of least positive degree for which \(p(\T) = \zT\).
\end{defn}

\begin{cor}\label{7.3.2}
  Every linear operator on a finite-dimensional vector space has a minimal polynomial.
\end{cor}

\begin{proof}[\pf{7.3.2}]
  The Cayley--Hamilton theorem (\cref{5.23}) tells us that for any linear operator \(\T\) on an \(n\)-dimensional vector space, there is a polynomial \(f\) of degree \(n\) such that \(f(\T) = \zT\), namely, the characteristic polynomial of \(\T\).
  Hence there is a polynomial of least degree with this property, and this degree is at most \(n\).
  If \(g\) is such a polynomial, we can divide \(g\) by its leading coefficient to obtain another polynomial \(p\) of the same degree with leading coefficient \(1\), that is, \(p\) is a \emph{monic} polynomial.
\end{proof}

\begin{thm}\label{7.12}
  Let \(p\) be a minimal polynomial of a linear operator \(\T\) on a finite-dimensional vector space \(\V\) over \(\F\).
  \begin{enumerate}
    \item For any polynomial \(g\), if \(g(\T) = \zT\), then \(p\) divides \(g\).
          In particular, \(p\) divides the characteristic polynomial of \(\T\).
    \item The minimal polynomial of \(\T\) is unique.
  \end{enumerate}
\end{thm}

\begin{proof}[\pf{7.12}(a)]
  Let \(g\) be a polynomial for which \(g(\T) = \zT\).
  By the division algorithm for polynomials (\cref{e.1}), there exist polynomials \(q\) and \(r\) such that
  \begin{equation}\label{eq:7.3.1}
    g(t) = q(t) p(t) + r(t),
  \end{equation}
  where \(r\) has degree less than the degree of \(p\).
  Substituting \(\T\) into \cref{eq:7.3.1} and using that \(g(\T) = p(\T) = \zT\), we have \(r(\T) = \zT\).
  Since \(r\) has degree less than \(p\) and \(p\) is the minimal polynomial of \(\T\), \(r\) must be the zero polynomial.
  Thus \cref{eq:7.3.1} simplifies to \(g = qp\), proving (a).
\end{proof}

\begin{proof}[\pf{7.12}(b)]
  Suppose that \(p_1\) and \(p_2\) are each minimal polynomials of \(\T\).
  Then \(p_1\) divides \(p_2\) by (a).
  Since \(p_1\) and \(p_2\) have the same degree, we have that \(p_2 = c p_1\) for some nonzero scalar \(c\).
  Because \(p_1\) and \(p_2\) are monic, \(c = 1\);
  hence \(p_1 = p_2\).
\end{proof}

\begin{defn}\label{7.3.3}
  Let \(A \in \ms[n][n][\F]\).
  The \textbf{minimal polynomial} \(p\) of \(A\) is the monic polynomial of least positive degree for which \(p(A) = \zm\).
\end{defn}

\begin{thm}\label{7.13}
  Let \(\T\) be a linear operator on a finite-dimensional vector space \(\V\) over \(\F\), and let \(\beta\) be an ordered basis for \(\V\) over \(\F\).
  Then the minimal polynomial of \(\T\) is the same as the minimal polynomial of \([\T]_{\beta}\).
\end{thm}

\begin{proof}[\pf{7.13}]
  Let \(p\) be the minimal polynomial of \(\T\).
  By \cref{7.3.1} we have \(p(\T) = \zT\).
  Thus by \cref{e.3}(b) we have \(\zm = [p(\T)]_{\beta} = p([\T]_{\beta})\).
  Suppose for sake of contradiction that \(q\) is the minimal polynomial of \(A\) and \(q \neq p\).
  Then the degree of \(q\) must be less than the degree of \(p\).
  Thus by \cref{e.3}(b) this means \(\zm = q([\T]_{\beta}) = [q(\T)]_{\beta}\).
  But by \cref{2.2.4} this means \(q(\T) = \zT\), which contradict to the uniqueness of \(p\) (\cref{7.12}(b)).
  Thus \(p = q\).
\end{proof}

\begin{cor}\label{7.3.4}
  For any \(A \in \ms[n][n][\F]\), the minimal polynomial of \(A\) is the same as the minimal polynomial of \(\L_A\).
\end{cor}

\begin{proof}[\pf{7.3.4}]
  Let \(\beta\) be the standard ordered basis for \(\vs{F}^n\) over \(\F\).
  By \cref{2.15}(a) we have \([\L_A]_{\beta} = A\).
  Thus by \cref{7.13} the minimal polynomial of \(A\) is the same as the minimal polynomial of \(\L_A\).
\end{proof}

\begin{note}
  In view of \cref{7.13,7.3.4}, \cref{7.12} and all subsequent theorems in this section that are stated for operators are also valid for matrices.
\end{note}

\begin{thm}\label{7.14}
  Let \(\T\) be a linear operator on a finite-dimensional vector space \(\V\) over \(\F\), and let \(p\) be the minimal polynomial of \(\T\).
  A scalar \(\lambda\) is an eigenvalue of \(\T\) iff \(p(\lambda) = 0\).
  Hence the characteristic polynomial and the minimal polynomial of \(\T\) have the same zeros.
\end{thm}

\begin{proof}[\pf{7.14}]
  Let \(f\) be the characteristic polynomial of \(\T\).
  Since \(p\) divides \(f\) (\cref{7.12}(a)), there exists a polynomial \(q\) such that \(f = qp\).
  If \(\lambda\) is a zero of \(p\), then
  \[
    f(\lambda) = q(\lambda) p(\lambda) = q(\lambda) \cdot 0 = 0.
  \]
  So \(\lambda\) is a zero of \(f\);
  that is, \(\lambda\) is an eigenvalue of \(\T\).

  Conversely, suppose that \(\lambda\) is an eigenvalue of \(\T\), and let \(x \in \V\) be an eigenvector corresponding to \(\lambda\).
  By \cref{ex:5.1.22}, we have
  \[
    \zv = \zT(x) = p(\T)(x) = p(\lambda)(x).
  \]
  Since \(x \neq \zv\), it follows that \(p(\lambda) = 0\), and so \(\lambda\) is a zero of \(p\).
\end{proof}

\begin{cor}\label{7.3.5}
  Let \(\T\) be a linear operator on a finite-dimensional vector space \(\V\) over \(\F\) with minimal polynomial \(p\) and characteristic polynomial \(f\).
  Suppose that \(f\) factors as
  \[
    f(t) = (\lambda_1 - t)^{n_1} \cdots (\lambda_k - t)^{n_k},
  \]
  where \(\seq{\lambda}{1,,k}\) are the distinct eigenvalues of \(\T\).
  Then there exist integers \(\seq{m}{1,,k}\) such that \(1 \leq m_i \leq n_i\) for all \(i \in \set{1, \dots, k}\) and
  \[
    p(t) = (t - \lambda_1)^{m_1} \cdots (t - \lambda_k)^{m_k}.
  \]
\end{cor}

\begin{proof}[\pf{7.3.5}]
  By \cref{7.14} we know that \(f\) and \(p\) have the same zeros, and by \cref{5.2} these zeros are exactly the eigenvalues of \(\T\).
  Thus \(\seq{m}{1,,k}\) exist and \(m_i \geq 1\) for all \(i \in \set{1, \dots, k}\).
  Since \(p\) divides \(f\) (\cref{7.12}(a)), we know that \(m_i \leq n_i\) for all \(i \in \set{1, \dots, k}\).
\end{proof}

\section{The Rational Canonical Form}\label{sec:7.4}

\exercisesection

\begin{ex}\label{ex:7.4.7}

\end{ex}


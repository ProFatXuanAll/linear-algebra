\section{Determinants of Order \textrm{2}}\label{sec:4.1}

\begin{defn}\label{4.1.1}
	If
	\[
		A = \begin{pmatrix}
			a & b \\
			c & d
		\end{pmatrix}
	\]
	is a \(2 \times 2\) matrix with entries from a field \(\F\), then we define the \textbf{determinant} of \(A\), denoted \(\det(A)\) or \(\abs{A}\), to be the scalar \(ad - bc\).
\end{defn}

\begin{note}
	There exist \(A, B \in \ms{2}{2}{\F}\) such that \(\det(A + B) \neq \det(A) + \det(B)\), the function \(\det : \ms{2}{2}{\F} \to \F\) is \emph{not} a linear transformation.
\end{note}

\begin{thm}\label{4.1}
	The function \(\det : \ms{2}{2}{\F} \to \F\) is a linear function of each row of a \(2 \times 2\) matrix when the other row is held fixed.
	That is, if \(u, v, w \in \vs{F}^2\) and \(k \in \F\), then
	\[
		\det\begin{pmatrix}
			u + kv \\
			w
		\end{pmatrix} = \det\begin{pmatrix}
			u \\
			w
		\end{pmatrix} + k \det\begin{pmatrix}
			v \\
			w
		\end{pmatrix}
	\]
	and
	\[
		\det\begin{pmatrix}
			w \\
			u + kv
		\end{pmatrix} = \det\begin{pmatrix}
			w \\
			u
		\end{pmatrix} + k \det\begin{pmatrix}
			w \\
			v
		\end{pmatrix}.
	\]
\end{thm}

\begin{proof}[\pf{4.1}]
	Let \(u = \tuple{a}{1,2}, v = \tuple{b}{1,2}, w = \tuple{c}{1,2} \in \vs{F}^2\) and let \(k \in \F\).
	Then
	\begin{align*}
		\det\begin{pmatrix}
			    u \\
			    w
		    \end{pmatrix} + k \det\begin{pmatrix}
			                          v \\
			                          w
		                          \end{pmatrix} & = \det\begin{pmatrix}
			                                                a_1 & a_2 \\
			                                                c_1 & c_2
		                                                \end{pmatrix} + k \det\begin{pmatrix}
			                                                                      b_1 & b_2 \\
			                                                                      c_1 & c_2
		                                                                      \end{pmatrix}  \\
		                                      & = (a_1 c_2 - a_2 c_1) + k (b_1 c_2 - b_2 c_1) \\
		                                      & = (a_1 + k b_1) c_2 - (a_2 + k b_2) c_1       \\
		                                      & = \det\begin{pmatrix}
			                                              a_1 + k b_1 & a_2 + k b_2 \\
			                                              c_1         & c_2
		                                              \end{pmatrix}               \\
		                                      & = \det\begin{pmatrix}
			                                              u + kv \\
			                                              w
		                                              \end{pmatrix}.
	\end{align*}
	A similar calculation shows that
	\[
		\det\begin{pmatrix}
			w \\
			u
		\end{pmatrix} + k \det\begin{pmatrix}
			w \\
			v
		\end{pmatrix} = \det\begin{pmatrix}
			w \\
			u + kv
		\end{pmatrix}.
	\]
\end{proof}

\begin{thm}\label{4.2}
	Let \(A \in \ms{2}{2}{\F}\).
	Then the determinant of \(A\) is nonzero iff \(A\) is invertible.
	Moreover, if \(A\) is invertible, then
	\[
		A^{-1} = \frac{1}{\det(A)} \begin{pmatrix}
			A_{2 2}  & -A_{1 2} \\
			-A_{2 1} & A_{1 1}
		\end{pmatrix}.
	\]
\end{thm}

\begin{proof}[\pf{4.2}]
	If \(\det(A) \neq 0\), then we can define a matrix
	\[
		M = \frac{1}{\det(A)} \begin{pmatrix}
			A_{2 2}  & -A_{1 2} \\
			-A_{2 1} & A_{1 1}
		\end{pmatrix}.
	\]
	A straightforward calculation shows that \(AM = MA = I\), and so \(A\) is invertible and \(M = A^{-1}\).

	Conversely, suppose that \(A\) is invertible.
	\cref{3.2.2} shows that the rank of
	\[
		A = \begin{pmatrix}
			A_{1 1} & A_{1 2} \\
			A_{2 1} & A_{2 2}
		\end{pmatrix}
	\]
	must be \(2\).
	Hence \(A_{1 1} \neq 0\) or \(A_{2 1} \neq 0\).
	If \(A_{1 1} \neq 0\), add \(-A_{2 1} / A_{1 1}\) times row \(1\) of \(A\) to row \(2\) to obtain the matrix
	\[
		\begin{pmatrix}
			A_{1 1} & A_{1 2}                                   \\
			0       & A_{2 2} - \frac{A_{1 2} A_{2 1}}{A_{1 1}}
		\end{pmatrix}.
	\]
	Because elementary row operations are rank-preserving by \cref{3.2.3}, it follows that
	\[
		A_{2 2} - \frac{A_{1 2} A_{2 1}}{A_{1 1}} \neq 0.
	\]
	Therefore \(\det(A) = A_{1 1} A_{2 2} - A_{1 2} A_{2 1} \neq 0\).
	On the other hand, if \(A_{2 1} \neq 0\), we see that \(\det(A) \neq 0\) by adding \(-A_{1 1} / A_{2 1}\) times row \(2\) of \(A\) to row \(1\) and applying a similar argument.
	Thus, in either case, \(\det(A) \neq 0\).
\end{proof}

\begin{defn}\label{4.1.2}
	By the \textbf{angle} between two vectors in \(\R^2\), we mean the angle with measure \(\theta\) (\(0 \leq \theta < \pi\)) that is formed by the vectors having the same magnitude and direction as the given vectors but emanating from the origin.

	If \(\beta = \set{u, v}\) is an ordered basis for \(\R^2\), we define the \textbf{orientation} of \(\beta\) to be the real number
	\[
		\mathbf{O}\begin{pmatrix}
			u \\
			v
		\end{pmatrix} = \frac{\det\begin{pmatrix}
				u \\
				v
			\end{pmatrix}}{\abs{\det\begin{pmatrix}
					u \\
					v
				\end{pmatrix}}}.
	\]
	(The denominator of this fraction is nonzero by \cref{4.2}.)
	Clearly
	\[
		\mathbf{O}\begin{pmatrix}
			u \\
			v
		\end{pmatrix} = \pm 1.
	\]
	Notice that
	\[
		\mathbf{O}\begin{pmatrix}
			e_1 \\
			e_2
		\end{pmatrix} = 1 \quad \text{and} \quad \mathbf{O}\begin{pmatrix}
			e_1 \\
			-e_2
		\end{pmatrix} = -1.
	\]

	Recall that a coordinate system \(\set{u, v}\) is called \textbf{right-handed} if \(u\) can be rotated in a counterclockwise direction through an angle \(\theta\) (\(0 < \theta < \pi\)) to coincide with \(v\).
	Otherwise \(\set{u, v}\) is called a \textbf{left-handed} system.
	In general (see \cref{ex:4.1.12}),
	\[
		\mathbf{O}\begin{pmatrix}
			u \\
			v
		\end{pmatrix} = 1
	\]
	iff the ordered basis \(\set{u, v}\) forms a right-handed coordinate system.
	For convenience, we also define
	\[
		\mathbf{O}\begin{pmatrix}
			u \\
			v
		\end{pmatrix} = 1
	\]
	if \(\set{u, v}\) is linearly dependent.
\end{defn}

\begin{defn}\label{4.1.3}
	Any ordered set \(\set{u, v}\) in \(\R^2\) determines a parallelogram in the following manner.
	Regarding \(u\) and \(v\) as arrows emanating from the origin of \(\R^2\), we call the parallelogram having \(u\) and \(v\) as adjacent sides the \textbf{parallelogram determined by \(u\) and \(v\)}.
	Observe that if the set \(\set{u, v}\) is linearly dependent (i.e., if \(u\) and \(v\) are parallel), then the ``parallelogram'' determined by \(u\) and \(v\) is actually a line segment, which we consider to be a degenerate parallelogram having area zero.
\end{defn}

\begin{prop}\label{4.1.4}
	Let \(\set{u, v} \subseteq \R^2\).
	If we define
	\[
		\mathbf{A}\begin{pmatrix}
			u \\
			v
		\end{pmatrix}
	\]
	as the area of the parallelogram determined by \(u\) and \(v\), then
	\[
		\mathbf{A}\begin{pmatrix}
			u \\
			v
		\end{pmatrix} = \mathbf{O}\begin{pmatrix}
			u \\
			v
		\end{pmatrix} \cdot \det\begin{pmatrix}
			u \\
			v
		\end{pmatrix} = \abs{\det\begin{pmatrix}
				u \\
				v
			\end{pmatrix}}.
	\]
\end{prop}

\begin{proof}[\pf{4.1.4}]
	First, since
	\[
		\mathbf{O}\begin{pmatrix}
			u \\
			v
		\end{pmatrix} = \pm 1,
	\]
	we may multiply both sides of the desired equation by
	\[
		\mathbf{O}\begin{pmatrix}
			u \\
			v
		\end{pmatrix}
	\]
	to obtain the equivalent form
	\[
		\mathbf{O}\begin{pmatrix}
			u \\
			v
		\end{pmatrix} \cdot \mathbf{A}\begin{pmatrix}
			u \\
			v
		\end{pmatrix} = \det\begin{pmatrix}
			u \\
			v
		\end{pmatrix}.
	\]
	We establish this equation by verifying that the three conditions of \cref{ex:4.1.11} are satisfied by the function
	\[
		\delta\begin{pmatrix}
			u \\
			v
		\end{pmatrix} = \mathbf{O}\begin{pmatrix}
			u \\
			v
		\end{pmatrix} \cdot \mathbf{A}\begin{pmatrix}
			u \\
			v
		\end{pmatrix}.
	\]
	\begin{enumerate}
		\item We begin by showing that for any real number \(c\)
		      \[
			      \delta\begin{pmatrix}
				      u \\
				      cv
			      \end{pmatrix} = c \cdot \delta\begin{pmatrix}
				      u \\
				      v
			      \end{pmatrix}.
		      \]
		      Observe that this equation is valid if \(c = 0\) because
		      \[
			      \delta\begin{pmatrix}
				      u \\
				      cv
			      \end{pmatrix} = \mathbf{O}\begin{pmatrix}
				      u \\
				      \zv
			      \end{pmatrix} \cdot \mathbf{A}\begin{pmatrix}
				      u \\
				      \zv
			      \end{pmatrix} = 1 \cdot 0 = 0.
		      \]
		      So assume that \(c \neq 0\).
		      Regarding \(cv\) as the base of the parallelogram determined by \(u\) and \(cv\), we see that
		      \[
			      \mathbf{A}\begin{pmatrix}
				      u \\
				      cv
			      \end{pmatrix} = \text{base } \times \text{ altitude} = \abs{c} (\text{length of } v) (\text{altitude}) = \abs{c} \cdot \mathbf{A}\begin{pmatrix}
				      u \\
				      v
			      \end{pmatrix},
		      \]
		      since the altitude of the parallelogram determined by \(u\) and \(cv\) is the same as that in the parallelogram determined by \(u\) and \(v\).
		      Hence
		      \begin{align*}
			      \delta\begin{pmatrix}
				            u \\
				            cv
			            \end{pmatrix} & = \mathbf{O}\begin{pmatrix}
				                                        u \\
				                                        cv
			                                        \end{pmatrix} \cdot \mathbf{A}\begin{pmatrix}
				                                                                      u \\
				                                                                      cv
			                                                                      \end{pmatrix}                                              \\
			                            & = \pa{\frac{c}{\abs{c}} \mathbf{O}\begin{pmatrix}
					                                                                u \\
					                                                                v
				                                                                \end{pmatrix}} \pa{\abs{c} \mathbf{A}\begin{pmatrix}
					                                                                                                     u \\
					                                                                                                     v
				                                                                                                     \end{pmatrix}} &  & \by{4.1} \\
			                            & = c \cdot \mathbf{O}\begin{pmatrix}
				                                                  u \\
				                                                  v
			                                                  \end{pmatrix} \cdot \mathbf{A}\begin{pmatrix}
				                                                                                u \\
				                                                                                v
			                                                                                \end{pmatrix}                                    \\
			                            & = c \cdot \delta\begin{pmatrix}
				                                              u \\
				                                              v
			                                              \end{pmatrix}.
		      \end{align*}
		      A similar argument shows that
		      \[
			      \delta\begin{pmatrix}
				      cu \\
				      v
			      \end{pmatrix} = c \cdot \delta\begin{pmatrix}
				      u \\
				      v
			      \end{pmatrix}.
		      \]

		      We next prove that
		      \[
			      \delta\begin{pmatrix}
				      u \\
				      au + bw
			      \end{pmatrix} = b \cdot \delta\begin{pmatrix}
				      u \\
				      w
			      \end{pmatrix}
		      \]
		      for any \(u, w \in \R^2\) and any real numbers \(a\) and \(b\).
		      Because the parallelograms determined by \(u\) and \(w\) and by \(u\) and \(u + w\) have a common base \(u\) and the same altitude, it follows that
		      \[
			      \mathbf{A}\begin{pmatrix}
				      u \\
				      w
			      \end{pmatrix} = \mathbf{A}\begin{pmatrix}
				      u \\
				      u + w
			      \end{pmatrix}.
		      \]
		      If \(a = 0\), then
		      \[
			      \delta\begin{pmatrix}
				      u \\
				      au + bw
			      \end{pmatrix} = \delta\begin{pmatrix}
				      u \\
				      bw
			      \end{pmatrix} = b \cdot \delta\begin{pmatrix}
				      u \\
				      w
			      \end{pmatrix}
		      \]
		      by the first paragraph of (a).
		      Otherwise, if \(a \neq 0\), then
		      \[
			      \delta\begin{pmatrix}
				      u \\
				      au + bw
			      \end{pmatrix} = a \cdot \delta\begin{pmatrix}
				      u \\
				      u + \frac{b}{a} w
			      \end{pmatrix} = a \cdot \delta\begin{pmatrix}
				      u \\
				      \frac{b}{a} w
			      \end{pmatrix} = b \cdot \delta\begin{pmatrix}
				      u \\
				      w
			      \end{pmatrix}.
		      \]
		      So the desired conclusion is obtained in either case.

		      We are now able to show that
		      \[
			      \delta\begin{pmatrix}
				      u \\
				      v_1 + v_2
			      \end{pmatrix} = \delta\begin{pmatrix}
				      u \\
				      v_1
			      \end{pmatrix} + \delta\begin{pmatrix}
				      u \\
				      v_2
			      \end{pmatrix}
		      \]
		      for all \(u, v_1, v_2 \in \R^2\).
		      Since the result is immediate if \(u = 0\), we assume that \(u \neq 0\).
		      Choose any vector \(w \in \R^2\) such that \(\set{u, w}\) is linearly independent.
		      Then for any vectors \(v_1, v_2 \in \R^2\) there exist scalars \(a_i\) and \(b_i\) such that \(v_i = a_i u + b_i w\) (\(i \in \set{1, 2}\)).
		      Thus
		      \begin{align*}
			      \delta\begin{pmatrix}
				            u \\
				            v_1 + v_2
			            \end{pmatrix} & = \delta\begin{pmatrix}
				                                    u \\
				                                    (a_1 + a_2) u + (b_1 + b_2) w
			                                    \end{pmatrix}           \\
			                            & = (b_1 + b_2) \delta\begin{pmatrix}
				                                                  u \\
				                                                  w
			                                                  \end{pmatrix}           \\
			                            & = \delta\begin{pmatrix}
				                                      u \\
				                                      a_1 u + b_1 w
			                                      \end{pmatrix} + \delta\begin{pmatrix}
				                                                            u \\
				                                                            a_2 u + b_2 w
			                                                            \end{pmatrix} \\
			                            & = \delta\begin{pmatrix}
				                                      u \\
				                                      v_1
			                                      \end{pmatrix} + \delta\begin{pmatrix}
				                                                            u \\
				                                                            v_2
			                                                            \end{pmatrix}.
		      \end{align*}
		      A similar argument shows that
		      \[
			      \delta\begin{pmatrix}
				      u_1 + u_2 \\
				      v
			      \end{pmatrix} = \delta\begin{pmatrix}
				      u_1 \\
				      v
			      \end{pmatrix} + \delta\begin{pmatrix}
				      u_2 \\
				      v
			      \end{pmatrix}
		      \]
		      for all \(u_1, u_2, v \in \R^2\).
		\item Since
		      \[
			      \mathbf{A}\begin{pmatrix}
				      u \\
				      u
			      \end{pmatrix} = 0,
		      \]
		      it follows that
		      \[
			      \delta\begin{pmatrix}
				      u \\
				      u
			      \end{pmatrix} = \mathbf{O}\begin{pmatrix}
				      u \\
				      u
			      \end{pmatrix} \cdot \mathbf{A}\begin{pmatrix}
				      u \\
				      u
			      \end{pmatrix} = 0
		      \]
		      for any \(u \in \R^2\).
		\item Because the parallelogram determined by \(e_1\) and \(e_2\) is the unit square,
		      \[
			      \delta\begin{pmatrix}
				      e_1 \\
				      e_2
			      \end{pmatrix} = \mathbf{O}\begin{pmatrix}
				      e_1 \\
				      e_2
			      \end{pmatrix} \cdot \mathbf{A}\begin{pmatrix}
				      e_1 \\
				      e_2
			      \end{pmatrix} = 1 \cdot 1 = 1.
		      \]
		      Therefore \(\delta\) satisfies the three conditions of \cref{ex:4.1.11}, and hence \(\delta = \det\).
		      So the area of the parallelogram determined by \(u\) and \(v\) equals
		      \[
			      \mathbf{O}\begin{pmatrix}
				      u \\
				      v
			      \end{pmatrix} \cdot \det\begin{pmatrix}
				      u \\
				      v
			      \end{pmatrix}.
		      \]
	\end{enumerate}
\end{proof}

\exercisesection

\setcounter{ex}{4}
\begin{ex}\label{ex:4.1.5}
	Prove that if \(B\) is the matrix obtained by interchanging the rows of \(A \in \ms{2}{2}{\F}\), then \(\det(B) = -\det(A)\).
\end{ex}

\begin{proof}[\pf{ex:4.1.5}]
	We have
	\begin{align*}
		\det(B) & = \det\begin{pmatrix}
			                A_{2 1} & A_{2 2} \\
			                A_{1 1} & A_{1 2}
		                \end{pmatrix}                                \\
		        & = A_{2 1} A_{1 2} - A_{2 2} A_{1 1}    &  & \by{4.1.1} \\
		        & = -(A_{1 1} A_{2 2} - A_{1 2} A_{2 1})                 \\
		        & = -\det(A).                            &  & \by{4.1.1}
	\end{align*}
\end{proof}

\begin{ex}\label{ex:4.1.6}
	Prove that if the two columns of \(A \in \ms{2}{2}{\F}\) are identical, then \(\det(A) = 0\).
\end{ex}

\begin{proof}[\pf{ex:4.1.6}]
	We have
	\begin{align*}
		         & (A_{1 1}, A_{2 1}) = (A_{1 2}, A_{2 2})                                                              \\
		\implies & \det(A) = A_{1 1} A_{2 2} - A_{1 2} A_{2 1} = A_{1 1} A_{2 2} - A_{1 1} A_{2 2} = 0. &  & \by{4.1.1}
	\end{align*}
\end{proof}

\begin{ex}\label{ex:4.1.7}
	Prove that \(\det(\tp{A}) = \det(A)\) for any \(A \in \ms{2}{2}{\F}\).
\end{ex}

\begin{proof}[\pf{ex:4.1.7}]
	We have
	\begin{align*}
		\det(\tp{A}) & = \det\begin{pmatrix}
			                     A_{1 1} & A_{2 1} \\
			                     A_{1 2} & A_{2 2}
		                     \end{pmatrix}               &  & \by{1.3.3}   \\
		             & = A_{1 1} A_{2 2} - A_{2 1} A_{1 2} &  & \by{4.1.1} \\
		             & = A_{1 1} A_{2 2} - A_{1 2} A_{2 1}                 \\
		             & = \det(A).                          &  & \by{4.1.1}
	\end{align*}
\end{proof}

\begin{ex}\label{ex:4.1.8}
	Prove that if \(A \in \ms{2}{2}{\F}\) is upper triangular, then \(\det(A)\) equals the product of the diagonal entries of \(A\).
\end{ex}

\begin{proof}[\pf{ex:4.1.8}]
	We have
	\begin{align*}
		         & A_{2 1} = 0                                                    &  & \by{ex:1.3.12} \\
		\implies & \det(A) = A_{1 1} A_{2 2} - A_{1 2} A_{2 1} = A_{1 1} A_{2 2}. &  & \by{4.1.1}
	\end{align*}
\end{proof}

\begin{ex}\label{ex:4.1.9}
	Prove that \(\det(AB) = \det(A) \cdot \det(B)\) for any \(A, B \in \ms{2}{2}{\F}\).
\end{ex}

\begin{proof}[\pf{ex:4.1.9}]
	We have
	\begin{align*}
		 & \det(AB)                                                                                                                        \\
		 & = \det\begin{pmatrix}
			         A_{1 1} B_{1 1} + A_{1 2} B_{2 1} & A_{1 1} B_{1 2} + A_{1 2} B_{2 2} \\
			         A_{2 1} B_{1 1} + A_{2 2} B_{2 1} & A_{2 1} B_{1 2} + A_{2 2} B_{2 2}
		         \end{pmatrix}                                        &  & \by{2.3.1}                                                     \\
		 & = (A_{1 1} B_{1 1} + A_{1 2} B_{2 1}) (A_{2 1} B_{1 2} + A_{2 2} B_{2 2})                                                       \\
		 & \quad - (A_{1 1} B_{1 2} + A_{1 2} B_{2 2}) (A_{2 1} B_{1 1} + A_{2 2} B_{2 1})                                 &  & \by{4.1.1} \\
		 & = (A_{1 1} A_{2 2}) (B_{1 1} B_{2 2} - B_{1 2} B_{2 1}) - (A_{1 2} A_{2 1}) (B_{1 1} B_{2 2} - B_{1 2} B_{2 1})                 \\
		 & = (A_{1 1} A_{2 2} - A_{1 2} A_{2 1}) (B_{1 1} B_{2 2} - B_{1 2} B_{2 1})                                                       \\
		 & = \det(A) \det(B).                                                                                              &  & \by{4.1.1}
	\end{align*}
\end{proof}

\begin{ex}\label{ex:4.1.10}
	The \textbf{classical adjoint} of a matrix \(A \in \ms{2}{2}{\F}\) is the matrix
	\[
		C = \begin{pmatrix}
			A_{2 2}  & -A_{1 2} \\
			-A_{2 1} & A_{1 1}
		\end{pmatrix}.
	\]
	Prove that
	\begin{enumerate}
		\item \(CA = AC = \det(A) I\).
		\item \(\det(C) = \det(A)\).
		\item The classical adjoint of \(\tp{A}\) is \(\tp{C}\).
		\item If \(A\) is invertible, then \(A^{-1} = (\det(A))^{-1} C\).
	\end{enumerate}
\end{ex}

\begin{proof}[\pf{ex:4.1.10}(a)]
	We have
	\begin{align*}
		CA & = \begin{pmatrix}
			       C_{1 1} A_{1 1} + C_{1 2} A_{2 1} & C_{1 1} A_{1 2} + C_{1 2} A_{2 2} \\
			       C_{2 1} A_{1 1} + C_{2 2} A_{2 1} & C_{2 1} A_{1 2} + C_{2 2} A_{2 2}
		       \end{pmatrix}   &  & \by{2.3.1}                    \\
		   & = \begin{pmatrix}
			       A_{2 2} A_{1 1} - A_{1 2} A_{2 1}  & A_{2 2} A_{1 2} - A_{1 2} A_{2 2}  \\
			       -A_{2 1} A_{1 1} + A_{1 1} A_{2 1} & -A_{2 1} A_{1 2} + A_{1 1} A_{2 2}
		       \end{pmatrix} &  & \by{ex:4.1.10}                  \\
		   & = \begin{pmatrix}
			       \det(A) & 0       \\
			       0       & \det(A)
		       \end{pmatrix}                                                       &  & \by{4.1.1}      \\
		   & = \det(A) I                                                                &  & \by{1.2.9}
	\end{align*}
	and
	\begin{align*}
		AC & = \begin{pmatrix}
			       A_{1 1} C_{1 1} + A_{1 2} C_{2 1} & A_{1 1} C_{1 2} + A_{1 2} C_{2 2} \\
			       A_{2 1} C_{1 1} + A_{2 2} C_{2 1} & A_{2 1} C_{1 2} + A_{2 2} C_{2 2}
		       \end{pmatrix}  &  & \by{2.3.1}                   \\
		   & = \begin{pmatrix}
			       A_{1 1} A_{2 2} - A_{1 2} A_{2 1} & -A_{1 1} A_{1 2} + A_{1 2} A_{1 1} \\
			       A_{2 1} A_{2 2} - A_{2 2} A_{2 1} & -A_{2 1} A_{1 2} + A_{2 2} A_{1 1}
		       \end{pmatrix} &  & \by{ex:4.1.10}                  \\
		   & = \begin{pmatrix}
			       \det(A) & 0       \\
			       0       & \det(A)
		       \end{pmatrix}                                                      &  & \by{4.1.1}      \\
		   & = \det(A) I.                                                              &  & \by{1.2.9}
	\end{align*}
\end{proof}

\begin{proof}[\pf{ex:4.1.10}(b)]
	We have
	\begin{align*}
		\det(C) & = A_{2 2} A_{1 1} - (-A_{1 2}) (-A_{2 1}) &  & \by{4.1.1} \\
		        & = A_{1 1} A_{2 2} - A_{1 2} A_{2 1}                       \\
		        & = \det(A).                                &  & \by{4.1.1}
	\end{align*}
\end{proof}

\begin{proof}[\pf{ex:4.1.10}(c)]
	We have
	\begin{align*}
		 & \text{the classical adjoint of } \tp{A}                     \\
		 & = \begin{pmatrix}
			     (\tp{A})_{2 2}  & -(\tp{A})_{1 2} \\
			     -(\tp{A})_{2 1} & (\tp{A})_{1 1}
		     \end{pmatrix}    &  & \by{ex:4.1.10}                      \\
		 & = \begin{pmatrix}
			     A_{2 2}  & -A_{2 1} \\
			     -A_{1 2} & A_{1 1}
		     \end{pmatrix}                  &  & \by{1.3.3}            \\
		 & = \tp{C}.                               &  & \by{ex:4.1.10}
	\end{align*}
\end{proof}

\begin{proof}[\pf{ex:4.1.10}(d)]
	We have
	\begin{align*}
		\frac{1}{\det(A)} C & = \frac{1}{\det(A)} \begin{pmatrix}
			                                          A_{2 2}  & -A_{1 2} \\
			                                          -A_{2 1} & A_{1 1}
		                                          \end{pmatrix} &  & \by{ex:4.1.10} \\
		                    & = A^{-1}.                           &  & \by{4.2}
	\end{align*}
\end{proof}

\begin{ex}\label{ex:4.1.11}
	Let \(\delta : \ms{2}{2}{\F} \to \F\) be a function with the following three properties.
	\begin{enumerate}
		\item \(\delta\) is a linear function of each row of the matrix when the other row is held fixed.
		\item If the two rows of \(A \in \ms{2}{2}{\F}\) are identical, then \(\delta(A) = 0\).
		\item \(\delta(I_2) = 1\).
	\end{enumerate}
	Prove that \(\delta(A) = \det(A)\) for all \(A \in \ms{2}{2}{\F}\).
\end{ex}

\begin{proof}[\pf{ex:4.1.11}]
	Let \(A \in \ms{n}{n}{\F}\), let \(e_1 = (1, 0)\) and let \(e_2 = (0, 1)\).
	Since
	\begin{align*}
		0 & = \delta\begin{pmatrix}
			            e_1 + e_2 \\
			            e_1 + e_2
		            \end{pmatrix}               &  & \by{ex:4.1.11}[b]                                                     \\
		  & = \delta\begin{pmatrix}
			            e_1 \\
			            e_1 + e_2
		            \end{pmatrix} + \delta\begin{pmatrix}
			                                  e_2 \\
			                                  e_1 + e_2
		                                  \end{pmatrix} &  & \by{ex:4.1.11}[a]                                             \\
		  & = \delta\begin{pmatrix}
			            e_1 \\
			            e_1
		            \end{pmatrix} + \delta\begin{pmatrix}
			                                  e_1 \\
			                                  e_2
		                                  \end{pmatrix} + \delta\begin{pmatrix}
			                                                        e_2 \\
			                                                        e_1
		                                                        \end{pmatrix} + \delta\begin{pmatrix}
			                                                                              e_2 \\
			                                                                              e_2
		                                                                              \end{pmatrix} &  & \by{ex:4.1.11}[a] \\
		  & = 0 + 1 + \delta\begin{pmatrix}
			                    e_2 \\
			                    e_1
		                    \end{pmatrix} + 0,       &  & \by{ex:4.1.11}[bc]
	\end{align*}
	we know that
	\[
		\delta\begin{pmatrix}
			e_2 \\
			e_1
		\end{pmatrix} = -1
	\]
	and thus
	\begin{align*}
		\delta(A) & = \delta\begin{pmatrix}
			                    A_{1 1} & A_{1 2} \\
			                    A_{2 1} & A_{2 2}
		                    \end{pmatrix}                                                                                     \\
		          & = \delta\begin{pmatrix}
			                    A_{1 1} e_1 + A_{1 2} e_2 \\
			                    A_{2 1} e_1 + A_{2 2} e_2
		                    \end{pmatrix}                                                                             \\
		          & = A_{1 1} \delta\begin{pmatrix}
			                            e_1 \\
			                            A_{2 1} e_1 + A_{2 2} e_2
		                            \end{pmatrix} + A_{1 2} \delta\begin{pmatrix}
			                                                          e_2 \\
			                                                          A_{2 1} e_1 + A_{2 2} e_2
		                                                          \end{pmatrix}             &  & \by{ex:4.1.11}[a]                \\
		          & = A_{1 1} \pa{A_{2 1} \delta\begin{pmatrix}
				                                        e_1 \\
				                                        e_1
			                                        \end{pmatrix} + A_{2 2} \delta\begin{pmatrix}
				                                                                      e_1 \\
				                                                                      e_2
			                                                                      \end{pmatrix}}             &  & \by{ex:4.1.11}[a]   \\
		          & \quad + A_{1 2} \pa{A_{2 1} \delta\begin{pmatrix}
				                                              e_2 \\
				                                              e_1
			                                              \end{pmatrix} + A_{2 2} \delta\begin{pmatrix}
				                                                                            e_2 \\
				                                                                            e_2
			                                                                            \end{pmatrix}}         &  & \by{ex:4.1.11}[a] \\
		          & = A_{1 1} A_{2 2} \delta\begin{pmatrix}
			                                    e_1 \\
			                                    e_2
		                                    \end{pmatrix} + A_{1 2} A_{2 1} \delta\begin{pmatrix}
			                                                                          e_2 \\
			                                                                          e_1
		                                                                          \end{pmatrix}     &  & \by{ex:4.1.11}[b]        \\
		          & = A_{1 1} A_{2 2} + A_{1 2} A_{2 1} \delta\begin{pmatrix}
			                                                      e_2 \\
			                                                      e_1
		                                                      \end{pmatrix} &  & \by{ex:4.1.11}[c]                                \\
		          & = A_{1 1} A_{2 2} - A_{1 2} A_{2 1}                       &  & \text{(from the proof above)}                  \\
		          & = \det(A).                                                &  & \by{4.1.1}
	\end{align*}
	Since \(A\) is arbitrary, we conclude that \(\delta = \det\).
\end{proof}

\begin{ex}\label{ex:4.1.12}
	Let \(\set{u, v}\) be an ordered basis for \(\R^2\).
	Prove that
	\[
		\mathbf{O}\begin{pmatrix}
			u \\
			v
		\end{pmatrix} = 1
	\]
	iff \(\set{u, v}\) forms a right-handed coordinate system.
\end{ex}

\begin{proof}[\pf{ex:4.1.12}]
	First suppose that \(\set{u, v}\) forms a right-handed coordinate system.
	By \cref{4.1.2} there exists a \(\theta \in (0, \pi)\) such that by rotating \(u\) with angle \(\theta\) and scaling \(u\) with some \(t > 0\) we get \(v\).
	Note that \(\theta \neq 0\) since \(\set{u, v}\) is a basis for \(\R^2\) over \(\R\).
	By \cref{2.1.3} this means
	\[
		v = t (u_1 \cos(\theta) - u_2 \sin(\theta), u_1 \sin(\theta) + u_2 \cos(\theta)).
	\]
	Thus we have
	\begin{align*}
		\det\begin{pmatrix}
			    u \\
			    v
		    \end{pmatrix} & = \det\begin{pmatrix}
			                          u_1                                     & u_2                                     \\
			                          t u_1 \cos(\theta) - t u_2 \sin(\theta) & t u_1 \sin(\theta) + t u_2 \cos(\theta)
		                          \end{pmatrix}                           \\
		                    & = t \det\begin{pmatrix}
			                              u_1                                 & u_2                                 \\
			                              u_1 \cos(\theta) - u_2 \sin(\theta) & u_1 \sin(\theta) + u_2 \cos(\theta)
		                              \end{pmatrix}         &  & \by{4.1}                               \\
		                    & = t (u_1^2 \sin(\theta) + u_2^2 \sin(\theta))                                        &  & \by{4.1.1}            \\
		                    & > 0.                                                                                 &  & (\theta \in (0, \pi))
	\end{align*}
	By \cref{4.1.2} we have
	\[
		\mathbf{O}\begin{pmatrix}
			u \\
			v
		\end{pmatrix} = \frac{\det\begin{pmatrix}
				u \\
				v
			\end{pmatrix}}{\abs{\det\begin{pmatrix}
					u \\
					v
				\end{pmatrix}}} = \frac{\det\begin{pmatrix}
				u \\
				v
			\end{pmatrix}}{\det\begin{pmatrix}
				u \\
				v
			\end{pmatrix}} = 1.
	\]

	Now suppose that
	\[
		\mathbf{O}\begin{pmatrix}
			u \\
			v
		\end{pmatrix} = 1.
	\]
	Since \(\set{u, v}\) is a basis for \(\R^2\), by rotating \(u\) with angle \(\theta > 0\) and scaling \(u\) with \(t > 0\) we get \(v\).
	By \cref{2.1.3} this means
	\[
		v = t (u_1 \cos(\theta) - u_2 \sin(\theta), u_1 \sin(\theta) + u_2 \cos(\theta)).
	\]
	Then we have
	\begin{align*}
		         & \mathbf{O}\begin{pmatrix}
			                     u \\
			                     v
		                     \end{pmatrix} = 1                                                        \\
		\implies & \det\begin{pmatrix}
			               u \\
			               v
		               \end{pmatrix} > 0                             &  & \by{4.1.2}                  \\
		\implies & u_1 v_2 - u_2 v_1 > 0                           &  & \by{4.1.1}                    \\
		\implies & t u_1^2 \sin(\theta) + t u_2^2 \sin(\theta) > 0 &  & \text{(from the proof above)} \\
		\implies & \sin(\theta) > 0                                &  & (t > 0)                       \\
		\implies & \theta \in (0, \pi)
	\end{align*}
	and thus by \cref{4.1.2} \(\set{u, v}\) forms a right-handed coordinate system.
\end{proof}

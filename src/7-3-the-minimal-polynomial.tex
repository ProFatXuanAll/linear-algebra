\section{The Minimal Polynomial}\label{sec:7.3}

\begin{defn}\label{7.3.1}
  Let \(\T\) be a linear operator on a finite-dimensional vector space.
  A polynomial \(p\) is called a \textbf{minimal polynomial} of \(\T\) if \(p\) is a monic polynomial of least positive degree for which \(p(\T) = \zT\).
\end{defn}

\begin{cor}\label{7.3.2}
  Every linear operator on a finite-dimensional vector space has a minimal polynomial.
\end{cor}

\begin{proof}[\pf{7.3.2}]
  The Cayley--Hamilton theorem (\cref{5.23}) tells us that for any linear operator \(\T\) on an \(n\)-dimensional vector space, there is a polynomial \(f\) of degree \(n\) such that \(f(\T) = \zT\), namely, the characteristic polynomial of \(\T\).
  Hence there is a polynomial of least degree with this property, and this degree is at most \(n\).
  If \(g\) is such a polynomial, we can divide \(g\) by its leading coefficient to obtain another polynomial \(p\) of the same degree with leading coefficient \(1\), that is, \(p\) is a \emph{monic} polynomial.
\end{proof}

\begin{thm}\label{7.12}
  Let \(p\) be a minimal polynomial of a linear operator \(\T\) on a finite-dimensional vector space \(\V\) over \(\F\).
  \begin{enumerate}
    \item For any polynomial \(g\), if \(g(\T) = \zT\), then \(p\) divides \(g\).
          In particular, \(p\) divides the characteristic polynomial of \(\T\).
    \item The minimal polynomial of \(\T\) is unique.
  \end{enumerate}
\end{thm}

\begin{proof}[\pf{7.12}(a)]
  Let \(g\) be a polynomial for which \(g(\T) = \zT\).
  By the division algorithm for polynomials (\cref{e.1}), there exist polynomials \(q\) and \(r\) such that
  \begin{equation}\label{eq:7.3.1}
    g(t) = q(t) p(t) + r(t),
  \end{equation}
  where \(r\) has degree less than the degree of \(p\).
  Substituting \(\T\) into \cref{eq:7.3.1} and using that \(g(\T) = p(\T) = \zT\), we have \(r(\T) = \zT\).
  Since \(r\) has degree less than \(p\) and \(p\) is the minimal polynomial of \(\T\), \(r\) must be the zero polynomial.
  Thus \cref{eq:7.3.1} simplifies to \(g = qp\), proving (a).
\end{proof}

\begin{proof}[\pf{7.12}(b)]
  Suppose that \(p_1\) and \(p_2\) are each minimal polynomials of \(\T\).
  Then \(p_1\) divides \(p_2\) by (a).
  Since \(p_1\) and \(p_2\) have the same degree, we have that \(p_2 = c p_1\) for some nonzero scalar \(c\).
  Because \(p_1\) and \(p_2\) are monic, \(c = 1\);
  hence \(p_1 = p_2\).
\end{proof}

\begin{defn}\label{7.3.3}
  Let \(A \in \ms[n][n][\F]\).
  The \textbf{minimal polynomial} \(p\) of \(A\) is the monic polynomial of least positive degree for which \(p(A) = \zm\).
\end{defn}

\begin{thm}\label{7.13}
  Let \(\T\) be a linear operator on a finite-dimensional vector space \(\V\) over \(\F\), and let \(\beta\) be an ordered basis for \(\V\) over \(\F\).
  Then the minimal polynomial of \(\T\) is the same as the minimal polynomial of \([\T]_{\beta}\).
\end{thm}

\begin{proof}[\pf{7.13}]
  Let \(p\) be the minimal polynomial of \(\T\).
  By \cref{7.3.1} we have \(p(\T) = \zT\).
  Thus by \cref{e.3}(b) we have \(\zm = [p(\T)]_{\beta} = p([\T]_{\beta})\).
  Suppose for sake of contradiction that \(q\) is the minimal polynomial of \(A\) and \(q \neq p\).
  Then the degree of \(q\) must be less than the degree of \(p\).
  Thus by \cref{e.3}(b) this means \(\zm = q([\T]_{\beta}) = [q(\T)]_{\beta}\).
  But by \cref{2.2.4} this means \(q(\T) = \zT\), which contradict to the uniqueness of \(p\) (\cref{7.12}(b)).
  Thus \(p = q\).
\end{proof}

\begin{cor}\label{7.3.4}
  For any \(A \in \ms[n][n][\F]\), the minimal polynomial of \(A\) is the same as the minimal polynomial of \(\L_A\).
\end{cor}

\begin{proof}[\pf{7.3.4}]
  Let \(\beta\) be the standard ordered basis for \(\vs{F}^n\) over \(\F\).
  By \cref{2.15}(a) we have \([\L_A]_{\beta} = A\).
  Thus by \cref{7.13} the minimal polynomial of \(A\) is the same as the minimal polynomial of \(\L_A\).
\end{proof}

\begin{note}
  In view of \cref{7.13,7.3.4}, \cref{7.12} and all subsequent theorems in this section that are stated for operators are also valid for matrices.
\end{note}

\begin{thm}\label{7.14}
  Let \(\T\) be a linear operator on a finite-dimensional vector space \(\V\) over \(\F\), and let \(p\) be the minimal polynomial of \(\T\).
  A scalar \(\lambda\) is an eigenvalue of \(\T\) iff \(p(\lambda) = 0\).
  Hence the characteristic polynomial and the minimal polynomial of \(\T\) have the same zeros.
\end{thm}

\begin{proof}[\pf{7.14}]
  Let \(f\) be the characteristic polynomial of \(\T\).
  Since \(p\) divides \(f\) (\cref{7.12}(a)), there exists a polynomial \(q\) such that \(f = qp\).
  If \(\lambda\) is a zero of \(p\), then
  \[
    f(\lambda) = q(\lambda) p(\lambda) = q(\lambda) \cdot 0 = 0.
  \]
  So \(\lambda\) is a zero of \(f\);
  that is, \(\lambda\) is an eigenvalue of \(\T\).

  Conversely, suppose that \(\lambda\) is an eigenvalue of \(\T\), and let \(x \in \V\) be an eigenvector corresponding to \(\lambda\).
  By \cref{ex:5.1.22}, we have
  \[
    \zv = \zT(x) = p(\T)(x) = p(\lambda)(x).
  \]
  Since \(x \neq \zv\), it follows that \(p(\lambda) = 0\), and so \(\lambda\) is a zero of \(p\).
\end{proof}

\begin{cor}\label{7.3.5}
  Let \(\T\) be a linear operator on a finite-dimensional vector space \(\V\) over \(\F\) with minimal polynomial \(p\) and characteristic polynomial \(f\).
  Suppose that \(f\) factors as
  \[
    f(t) = (\lambda_1 - t)^{n_1} \cdots (\lambda_k - t)^{n_k},
  \]
  where \(\seq{\lambda}{1,,k}\) are the distinct eigenvalues of \(\T\).
  Then there exist integers \(\seq{m}{1,,k}\) such that \(1 \leq m_i \leq n_i\) for all \(i \in \set{1, \dots, k}\) and
  \[
    p(t) = (t - \lambda_1)^{m_1} \cdots (t - \lambda_k)^{m_k}.
  \]
\end{cor}

\begin{proof}[\pf{7.3.5}]
  By \cref{7.14} we know that \(f\) and \(p\) have the same zeros, and by \cref{5.2} these zeros are exactly the eigenvalues of \(\T\).
  Thus \(\seq{m}{1,,k}\) exist and \(m_i \geq 1\) for all \(i \in \set{1, \dots, k}\).
  Since \(p\) divides \(f\) (\cref{7.12}(a)), we know that \(m_i \leq n_i\) for all \(i \in \set{1, \dots, k}\).
\end{proof}

\begin{thm}\label{7.15}
  Let \(\T\) be a linear operator on an \(n\)-dimensional vector space \(\V\) such that \(\V\) is a \(\T\)-cyclic subspace of itself.
  Then the characteristic polynomial \(f\) and the minimal polynomial \(p\) have the same degree, and hence \(f = (-1)^n p\).
\end{thm}

\begin{proof}[\pf{7.15}]
  Since \(\V\) is a \(\T\)-cyclic space, there exists an \(x \in \V\) such that
  \[
    \beta = \set{x, \T(x), \dots, \T^{n - 1}(x)}
  \]
  is a basis for \(\V\) over \(\F\) (\cref{5.22}).
  Let
  \[
    g(t) = a_0 + a_1 t + \cdots + a_k t^k
  \]
  be a polynomial of degree \(k < n\).
  Then \(a_k \neq 0\) and
  \[
    g(\T)(x) = a_0 x + a_1 \T(x) + \cdots a_k \T^k(x),
  \]
  and so \(g(\T)(x)\) is a linear combination of the vectors of \(\beta\) having at least one nonzero coefficient, namely, \(a_k\).
  Since \(\beta\) is linearly independent, it follows that \(g(\T)(x) \neq 0\);
  hence \(g(\T) \neq \zT\).
  Therefore the minimal polynomial of \(\T\) has degree \(n\), which is also the degree of the characteristic polynomial of \(\T\).
\end{proof}

\begin{note}
  \cref{7.15} gives a condition under which the degree of the minimal polynomial of an operator is as large as possible.
  By \cref{7.14}, the degree of the minimal polynomial of an operator must be greater than or equal to the number of distinct eigenvalues of the operator.
  \cref{7.16} shows that the operators for which the degree of the minimal polynomial is as small as possible are precisely the diagonalizable operators.
\end{note}

\begin{thm}\label{7.16}
  Let \(\T\) be a linear operator on a finite-dimensional vector space \(\V\) over \(\F\).
  Then \(\T\) is diagonalizable iff the minimal polynomial of \(\T\) is of the form
  \[
    p(t) = (t - \lambda_1) \cdots (t - \lambda_k),
  \]
  where \(\seq{\lambda}{1,,k}\) are the distinct eigenvalues of \(\T\).
\end{thm}

\begin{proof}[\pf{7.16}]
  Suppose that \(\T\) is diagonalizable.
  Let \(\seq{\lambda}{1,,k}\) be the distinct eigenvalues of \(\T\), and define
  \[
    p(t) = (t - \lambda_1) \cdots (t - \lambda_k).
  \]
  By \cref{7.14}, \(p\) divides the minimal polynomial of \(\T\).
  Let \(\beta = \set{\seq{v}{1,,n}}\) be a basis for \(\V\) over \(\F\) consisting of eigenvectors of \(\T\), and consider any \(v_i \in \beta\).
  Then \((\T - \lambda_j \IT[\V])(v_i) = \zv\) for some eigenvalue \(\lambda_j\).
  Since \(t - \lambda_j\) divides \(p\), there is a polynomial \(q_j\) such that \(p(t) = q_j(t) (t - \lambda_j)\).
  Hence
  \[
    p(\T)(v_i) = q_j(\T) (\T - \lambda_j \IT[\V])(v_i) = \zv.
  \]
  It follows that \(p(\T) = \zT\), since \(p(\T)\) takes each vector in a basis for \(\V\) over \(\F\) into \(\zv\).
  Therefore \(p\) is the minimal polynomial of \(\T\).

  Conversely, suppose that there are distinct scalars \(\seq{\lambda}{1,,k}\) such that the minimal polynomial \(p\) of \(\T\) factors as
  \[
    p(t) = (t - \lambda_1) \cdots (t - \lambda_k).
  \]
  By \cref{7.14}, the \(\lambda_i\)'s are eigenvalues of \(\T\).
  We apply mathematical induction on \(n = \dim(\V)\).
  Clearly \(\T\) is diagonalizable for \(n = 1\).
  Now assume that \(\T\) is diagonalizable whenever \(\dim(\V) < n\) for some \(n > 1\), and let \(\dim(\V) = n\) and \(\W = \rg{\T - \lambda_k \IT[\V]}\).
  Obviously \(\W \neq \V\), because \(\lambda_k\) is an eigenvalue of \(\T\).
  If \(\W = \set{\zv}\), then \(\T = \lambda_k \IT[\V]\), which is clearly diagonalizable.
  So suppose that \(0 < \dim(\W) < n\).
  Then \(\W\) is \(\T\)-invariant, and for any \(x \in \W\),
  \[
    (\T - \lambda_1 \IT[\V]) \cdots (\T - \lambda_{k - 1} \IT[\V])(x) = \zv.
  \]
  It follows that the minimal polynomial of \(\T_{\W}\) divides the polynomial \((t - \lambda_1) \cdots (t - \lambda_{k - 1})\).
  Hence by the induction hypothesis, \(\T_{\W}\) is diagonalizable.
  Furthermore, \(\lambda_k\) is not an eigenvalue of \(\T_{\W}\) by \cref{7.14}.
  Therefore \(\W \cap \ns{\T - \lambda_k \IT[\V]} = \set{\zv}\).
  Now let \(\beta_1 = \set{\seq{v}{1,,m}}\) be a basis for \(\W\) over \(\F\) consisting of eigenvectors of \(\T_{\W}\) (and hence of \(\T\)), and let \(\beta_2 = \set{\seq{w}{1,,p}}\) be a basis for \(\ns{\T - \lambda_k \IT[\V]}\) over \(\F\), the eigenspace of \(\T\) corresponding to \(\lambda_k\).
  Then \(\beta_1\) and \(\beta_2\) are disjoint by the previous comment.
  Moreover, \(m + p = n\) by the dimension theorem applied to \(\T - \lambda_k \IT[\V]\).
  We show that \(\beta = \beta_1 \cup \beta_2\) is linearly independent.
  Consider scalars \(\seq{a}{1,,m}\) and \(\seq{b}{1,,p}\) such that
  \[
    \seq[+]{a,v}{1,,m} + \seq[+]{b,w}{1,,p} = \zv.
  \]
  Let
  \[
    x = \sum_{i = 1}^m a_i v_i \quad \text{and} \quad y = \sum_{i = 1}^p b_i w_i.
  \]
  Then \(x \in \W\), \(y \in \ns{\T - \lambda_k \IT[\V]}\), and \(x + y = \zv\).
  It follows that \(x = -y \in \W \cap \ns{\T - \lambda_k \IT[\V]}\), and therefore \(x = \zv\).
  Since \(\beta_1\) is linearly independent, we have that \(\seq[=]{a}{1,,m} = 0\).
  Similarly, \(\seq[=]{b}{1,,p} = 0\), and we conclude that \(\beta\) is a linearly independent subset of \(\V\) consisting of \(n\) eigenvectors.
  It follows that \(\beta\) is a basis for \(\V\) over \(\F\) consisting of eigenvectors of \(\T\), and consequently \(\T\) is diagonalizable.
\end{proof}

\begin{note}
  In addition to diagonalizable operators, there are methods for determining the minimal polynomial of any linear operator on a finite-dimensional vector space.
  In the case that the characteristic polynomial of the operator splits, the minimal polynomial can be described using the Jordan canonical form of the operator.
  (See \cref{ex:7.3.13}.)
  In the case that the characteristic polynomial does not split, the minimal polynomial can be described using the rational canonical form, which we study in \cref{sec:7.4}.
  (See \cref{ex:7.4.7}.)
\end{note}

\exercisesection

\begin{ex}\label{ex:7.3.13}

\end{ex}

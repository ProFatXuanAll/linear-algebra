\section{The Minimal Polynomial}\label{sec:7.3}

\begin{defn}\label{7.3.1}
  Let \(\T\) be a linear operator on a finite-dimensional vector space.
  A polynomial \(p\) is called a \textbf{minimal polynomial} of \(\T\) if \(p\) is a monic polynomial of least positive degree for which \(p(\T) = \zT\).
\end{defn}

\begin{cor}\label{7.3.2}
  Every linear operator on a finite-dimensional vector space has a minimal polynomial.
\end{cor}

\begin{proof}[\pf{7.3.2}]
  The Cayley--Hamilton theorem (\cref{5.23}) tells us that for any linear operator \(\T\) on an \(n\)-dimensional vector space, there is a polynomial \(f\) of degree \(n\) such that \(f(\T) = \zT\), namely, the characteristic polynomial of \(\T\).
  Hence there is a polynomial of least degree with this property, and this degree is at most \(n\).
  If \(g\) is such a polynomial, we can divide \(g\) by its leading coefficient to obtain another polynomial \(p\) of the same degree with leading coefficient \(1\), that is, \(p\) is a \emph{monic} polynomial.
\end{proof}

\begin{thm}\label{7.12}
  Let \(p\) be a minimal polynomial of a linear operator \(\T\) on a finite-dimensional vector space \(\V\) over \(\F\).
  \begin{enumerate}
    \item For any polynomial \(g\), if \(g(\T) = \zT\), then \(p\) divides \(g\).
          In particular, \(p\) divides the characteristic polynomial of \(\T\).
    \item The minimal polynomial of \(\T\) is unique.
  \end{enumerate}
\end{thm}

\begin{proof}[\pf{7.12}(a)]
  Let \(g\) be a polynomial for which \(g(\T) = \zT\).
  By the division algorithm for polynomials (\cref{e.1}), there exist polynomials \(q\) and \(r\) such that
  \begin{equation}\label{eq:7.3.1}
    g(t) = q(t) p(t) + r(t),
  \end{equation}
  where \(r\) has degree less than the degree of \(p\).
  Substituting \(\T\) into \cref{eq:7.3.1} and using that \(g(\T) = p(\T) = \zT\), we have \(r(\T) = \zT\).
  Since \(r\) has degree less than \(p\) and \(p\) is the minimal polynomial of \(\T\), \(r\) must be the zero polynomial.
  Thus \cref{eq:7.3.1} simplifies to \(g = qp\), proving (a).
\end{proof}

\begin{proof}[\pf{7.12}(b)]
  Suppose that \(p_1\) and \(p_2\) are each minimal polynomials of \(\T\).
  Then \(p_1\) divides \(p_2\) by (a).
  Since \(p_1\) and \(p_2\) have the same degree, we have that \(p_2 = c p_1\) for some nonzero scalar \(c\).
  Because \(p_1\) and \(p_2\) are monic, \(c = 1\);
  hence \(p_1 = p_2\).
\end{proof}

\begin{defn}\label{7.3.3}
  Let \(A \in \ms[n][n][\F]\).
  The \textbf{minimal polynomial} \(p\) of \(A\) is the monic polynomial of least positive degree for which \(p(A) = \zm\).
\end{defn}

\begin{thm}\label{7.13}
  Let \(\T\) be a linear operator on a finite-dimensional vector space \(\V\) over \(\F\), and let \(\beta\) be an ordered basis for \(\V\) over \(\F\).
  Then the minimal polynomial of \(\T\) is the same as the minimal polynomial of \([\T]_{\beta}\).
\end{thm}

\begin{proof}[\pf{7.13}]
  Let \(p\) be the minimal polynomial of \(\T\).
  By \cref{7.3.1} we have \(p(\T) = \zT\).
  Thus by \cref{e.3}(b) we have \(\zm = [p(\T)]_{\beta} = p([\T]_{\beta})\).
  Suppose for sake of contradiction that \(q\) is the minimal polynomial of \(A\) and \(q \neq p\).
  Then the degree of \(q\) must be less than the degree of \(p\).
  Thus by \cref{e.3}(b) this means \(\zm = q([\T]_{\beta}) = [q(\T)]_{\beta}\).
  But by \cref{2.2.4} this means \(q(\T) = \zT\), which contradict to the uniqueness of \(p\) (\cref{7.12}(b)).
  Thus \(p = q\).
\end{proof}

\begin{cor}\label{7.3.4}
  For any \(A \in \ms[n][n][\F]\), the minimal polynomial of \(A\) is the same as the minimal polynomial of \(\L_A\).
\end{cor}

\begin{proof}[\pf{7.3.4}]
  Let \(\beta\) be the standard ordered basis for \(\vs{F}^n\) over \(\F\).
  By \cref{2.15}(a) we have \([\L_A]_{\beta} = A\).
  Thus by \cref{7.13} the minimal polynomial of \(A\) is the same as the minimal polynomial of \(\L_A\).
\end{proof}

\begin{note}
  In view of \cref{7.13,7.3.4}, \cref{7.12} and all subsequent theorems in this section that are stated for operators are also valid for matrices.
\end{note}

\begin{thm}\label{7.14}
  Let \(\T\) be a linear operator on a finite-dimensional vector space \(\V\) over \(\F\), and let \(p\) be the minimal polynomial of \(\T\).
  A scalar \(\lambda\) is an eigenvalue of \(\T\) iff \(p(\lambda) = 0\).
  Hence the characteristic polynomial and the minimal polynomial of \(\T\) have the same zeros.
\end{thm}

\begin{proof}[\pf{7.14}]
  Let \(f\) be the characteristic polynomial of \(\T\).
  Since \(p\) divides \(f\) (\cref{7.12}(a)), there exists a polynomial \(q\) such that \(f = qp\).
  If \(\lambda\) is a zero of \(p\), then
  \[
    f(\lambda) = q(\lambda) p(\lambda) = q(\lambda) \cdot 0 = 0.
  \]
  So \(\lambda\) is a zero of \(f\);
  that is, \(\lambda\) is an eigenvalue of \(\T\).

  Conversely, suppose that \(\lambda\) is an eigenvalue of \(\T\), and let \(x \in \V\) be an eigenvector corresponding to \(\lambda\).
  By \cref{ex:5.1.22}, we have
  \[
    \zv = \zT(x) = p(\T)(x) = p(\lambda)(x).
  \]
  Since \(x \neq \zv\), it follows that \(p(\lambda) = 0\), and so \(\lambda\) is a zero of \(p\).
\end{proof}

\begin{cor}\label{7.3.5}
  Let \(\T\) be a linear operator on a finite-dimensional vector space \(\V\) over \(\F\) with minimal polynomial \(p\) and characteristic polynomial \(f\).
  Suppose that \(f\) factors as
  \[
    f(t) = (\lambda_1 - t)^{n_1} \cdots (\lambda_k - t)^{n_k},
  \]
  where \(\seq{\lambda}{1,,k}\) are the distinct eigenvalues of \(\T\).
  Then there exist integers \(\seq{m}{1,,k}\) such that \(1 \leq m_i \leq n_i\) for all \(i \in \set{1, \dots, k}\) and
  \[
    p(t) = (t - \lambda_1)^{m_1} \cdots (t - \lambda_k)^{m_k}.
  \]
\end{cor}

\begin{proof}[\pf{7.3.5}]
  By \cref{7.14} we know that \(f\) and \(p\) have the same zeros, and by \cref{5.2} these zeros are exactly the eigenvalues of \(\T\).
  Thus \(\seq{m}{1,,k}\) exist and \(m_i \geq 1\) for all \(i \in \set{1, \dots, k}\).
  Since \(p\) divides \(f\) (\cref{7.12}(a)), we know that \(m_i \leq n_i\) for all \(i \in \set{1, \dots, k}\).
\end{proof}

\section{The Adjoint of a Linear Operator}\label{sec:6.3}

\begin{thm}\label{6.8}
  Let \(\V\) be a finite-dimensional inner product space over \(\F\), and let \(g \in \ls(\V, \F)\).
  Then there exists a unique vector \(y \in \V\) such that \(g(x) = \inn{x, y}\) for all \(x \in \V\).
\end{thm}

\begin{proof}[\pf{6.8}]
  Let \(\beta = \set{\seq{v}{1,,n}}\) be an orthonormal basis for \(\V\) over \(\F\), and let
  \[
    y = \sum_{i = 1}^n \conj{g(v_i)} v_i.
  \]
  Define \(h : \V \to \F\) by \(h(x) = \inn{x, y}\), which is clearly linear (by \cref{6.1.1}(a)(b)).
  Furthermore, for \(j \in \set{1, \dots, n}\) we have
  \begin{align*}
    h(v_j) & = \inn{v_j, y}                                                   \\
           & = \inn{v_j, \sum_{i = 1}^n \conj{g(v_i)} v_i}                    \\
           & = \sum_{i = 1}^n g(v_i) \inn{v_j, v_i}        &  & \by{6.1}[a,b] \\
           & = \sum_{i = 1}^n g(v_i) \delta_{j i}          &  & \by{6.1.12}   \\
           & = g(v_j).
  \end{align*}
  Since \(g\) and \(h\) both agree on \(\beta\), we have that \(g = h\) by \cref{2.1.13}.
  To show that \(y\) is unique, suppose that \(g(x) = \inn{x, y'}\) for all \(x \in \V\).
  Then \(\inn{x, y} = \inn{x, y'}\) for all \(x \in \V\);
  so by \cref{6.1}(e), we have \(y = y'\).
\end{proof}

\begin{thm}\label{6.9}
  Let \(\V\) be a finite-dimensional inner product space over \(\F\), and let \(\T \in \ls(\V)\).
  Then there exists a unique \(\T^* \in \ls(\V)\) such that \(\inn{\T(x), y} = \inn{x, \T^*(y)}\) for all \(x, y \in \V\).
  \(\T^*\) is called the \textbf{adjoint} of the operator \(\T\).
  The symbol \(\T^*\) is read ``\(\T\) star.''
\end{thm}

\begin{proof}[\pf{6.9}]
  Let \(y \in \V\).
  Define \(g : \V \to \F\) by \(g(x) = \inn{\T(x), y}\) for all \(x \in \V\).
  We first show that \(g\) is linear.
  Let \(x_1, x_2 \in \V\) and \(c \in \F\).
  Then
  \begin{align*}
    g(c x_1 + x_2) & = \inn{\T(c x_1 + x_2), y}                                   \\
                   & = \inn{c \T(x_1) + \T(x_2), y}          &  & \by{2.1.2}[b]   \\
                   & = c \inn{\T(x_1), y} + \inn{\T(x_2), y} &  & \by{6.1.1}[a,b] \\
                   & = c g(x_1) + g(x_2).
  \end{align*}
  Hence by \cref{2.1.2}(b) \(g \in \ls(\V, \F)\).

  We now apply \cref{6.8} to obtain a unique vector \(y' \in \V\) such that \(g(x) = \inn{x, y'}\);
  that is, \(\inn{\T(x), y} = \inn{x, y'}\) for all \(x \in \V\).
  Defining \(\T^* : \V \to \V\) by \(\T^*(y) = y'\), we have \(\inn{\T(x), y} = \inn{x, \T^*(y)}\).

  To show that \(\T^*\) is linear, let \(y_1, y_2 \in \V\) and \(c \in \F\).
  Then for any \(x \in \V\),
  we have
  \begin{align*}
    \inn{x, \T^*(c y_1 + y_2)} & = \inn{\T(x), c y_1 + y_2}                                            \\
                               & = \conj{c} \inn{\T(x), y_1} + \inn{\T(x), y_2}     &  & \by{6.1}[a,b] \\
                               & = \conj{c} \inn{x, \T^*(y_1)} + \inn{x, \T^*(y_2)}                    \\
                               & = \inn{x, c \T^*(y_1) + \T^*(y_2)}.                &  & \by{6.1}[a,b]
  \end{align*}
  Since \(x\) is arbitrary, \(\T^*(c y_1 + y_2) = c \T^*(y_1) + \T^*(y_2)\) by \cref{6.1}(e).

  Finally, we need to show that \(\T^*\) is unique. Suppose that \(\U \in \ls(\V)\) and that it satisfies \(\inn{\T(x), y} = \inn{x, \U(y)}\) for all \(x, y \in \V\).
  Then \(\inn{x, \T^*(y)} = \inn{x, \U(y)}\) for all \(x, y \in \V\), so \(\T^* = \U\).
\end{proof}

\begin{note}
  Thus by \cref{6.9} \(\T^*\) is the unique operator on \(\V\) satisfying \(\inn{\T(x), y} = \inn{x, \T^*(y)}\) for all \(x, y \in \V\).
  Note that we also have
  \begin{align*}
    \inn{x, \T(y)} & = \conj{\inn{\T(y), x}}   &  & \by{6.1.1}[c] \\
                   & = \conj{\inn{y, \T^*(x)}} &  & \by{6.9}      \\
                   & = \inn{\T^*(x), y};       &  & \by{6.1.1}[c]
  \end{align*}
  so \(\inn{x, \T(y)} = \inn{\T^*(x), y}\) for all \(x, y \in \V\).
  We may view these equations symbolically as adding a \(*\) to \(\T\) when shifting its position inside the inner product symbol.
\end{note}

\begin{note}
  For an infinite-dimensional inner product space, the adjoint of a linear operator \(\T\) may be defined to be the function \(\T^*\) such that \(\inn{\T(x), y} = \inn{x, \T^*(y)}\) for all \(x, y \in \V\), provided it exists.
  Although the uniqueness and linearity of \(\T^*\) follow as before, the existence of the adjoint is not guaranteed (see \cref{ex:6.3.24}).
  The reader should observe the necessity of the hypothesis of finite-dimensionality in the proof of \cref{6.8}.
  Many of the theorems we prove about adjoints, nevertheless, do not depend on \(\V\) being finite-dimensional.
  \emph{Thus, unless stated otherwise, for the remainder of this chapter we adopt the convention that a reference to the adjoint of a linear operator on an infinite-dimensional inner product space assumes its existence}.
\end{note}

\begin{thm}\label{6.10}
  Let \(\V\) be a finite-dimensional inner product space over \(\F\), and let \(\beta\) be an orthonormal basis for \(\V\) over \(\F\).
  If \(\T \in \ls(\V)\), then
  \[
    [\T^*]_{\beta} = [\T]_{\beta}^*.
  \]
\end{thm}

\begin{proof}[\pf{6.10}]
  Let \(A = [\T]_{\beta}\), \(B = [\T^*]_{\beta}\), and \(\beta = \set{\seq{v}{1,,n}}\).
  Then we have
  \begin{align*}
    \forall i, j \in \set{1, \dots, n}, B_{i j} & = \inn{\T^*(v_j), v_i}        &  & \by{6.2.6}    \\
                                                & = \conj{\inn{v_i, \T^*(v_j)}} &  & \by{6.1.1}[c] \\
                                                & = \conj{\inn{\T(v_i), v_j}}   &  & \by{6.9}      \\
                                                & = \conj{A_{j i}}              &  & \by{6.2.6}    \\
                                                & = (A^*)_{i j}.                &  & \by{6.1.5}
  \end{align*}
  Hence \(B = A^*\).
\end{proof}

\begin{cor}\label{6.3.1}
  Let \(A \in \ms[n][n][\F]\).
  Then \(\L_{A^*} = (\L_A)^*\).
\end{cor}

\begin{proof}[\pf{6.3.1}]
  If \(\beta\) is the standard ordered basis for \(\vs{F}^n\), then, by \cref{2.15}(a), we have \([\L_A]_{\beta} = A\).
  Hence \([(\L_A)^*]_{\beta} = [\L_A]_{\beta}^* = A^* = [\L_{A^*}]_{\beta}\), and so \((\L_A)^* = \L_{A^*}\).
\end{proof}

\begin{thm}\label{6.11}
  Let \(\V\) be an inner product space over \(\F\) and let \(\T, \U \in \ls(\V)\).
  Then
  \begin{enumerate}
    \item \((\T + \U)^* = \T^* + \U^*\);
    \item \((c \T)^* = \conj{c} \T^*\) for any \(c \in \F\);
    \item \((\T \U)^* = \U^* \T^*\);
    \item \(\T^{**} = \T\);
    \item \(\IT[\V]^* = \IT[\V]\).
  \end{enumerate}
\end{thm}

\begin{proof}[\pf{6.11}(a)]
  Since
  \begin{align*}
    \forall x, y \in \V, \inn{x, (\T + \U)^*(y)} & = \inn{(\T + \U)(x), y}               &  & \by{6.9}      \\
                                                 & = \inn{\T(x) + \U(x), y}              &  & \by{2.2.5}    \\
                                                 & = \inn{\T(x), y} + \inn{\U(x), y}     &  & \by{6.1.1}[a] \\
                                                 & = \inn{x, \T^*(y)} + \inn{x, \U^*(y)} &  & \by{6.9}      \\
                                                 & = \inn{x, \T^*(y) + \U^*(y)}          &  & \by{6.1}[a]   \\
                                                 & = \inn{x, (\T^* + \U^*)(y)},          &  & \by{2.2.5}
  \end{align*}
  by \cref{6.1}(e) we know that \(\T^* + \U^* = (\T + \U)^*\).
\end{proof}

\begin{proof}[\pf{6.11}(b)]
  Since
  \begin{align*}
    \forall x, y \in \V, \inn{x, (c \T)^*(y)} & = \inn{(c \T)(x), y}           &  & \by{6.9}      \\
                                              & = \inn{c \T(x), y}             &  & \by{2.2.5}    \\
                                              & = c \inn{\T(x), y}             &  & \by{6.1.1}[b] \\
                                              & = c \inn{x, \T^*(y)}           &  & \by{6.9}      \\
                                              & = \inn{x, \conj{c} \T^*(y)}    &  & \by{6.1}[b]   \\
                                              & = \inn{x, (\conj{c} \T^*)(y)}, &  & \by{2.2.5}
  \end{align*}
  by \cref{6.1}(e) we know that \((c \T)^* = \conj{c} \T^*\).
\end{proof}

\begin{proof}[\pf{6.11}(c)]
  Since
  \begin{align*}
    \forall x, y \in \V, \inn{x, (\T \U)^*(y)} & = \inn{(\T \U)(x), y}      &  & \by{6.9} \\
                                               & = \inn{\T(\U(x)), y}                     \\
                                               & = \inn{\U(x), \T^*(y)}     &  & \by{6.9} \\
                                               & = \inn{x, \U^*(\T^*(y))}   &  & \by{6.9} \\
                                               & = \inn{x, (\U^* \T^*)(y)},
  \end{align*}
  by \cref{6.1}(e) we know that \((\T \U)^* = \U^* \T^*\).
\end{proof}

\begin{proof}[\pf{6.11}(d)]
  Since
  \begin{align*}
    \forall x, y \in \V, \inn{x, \T(y)} & = \inn{\T^*(x), y}     &  & \by{6.9} \\
                                        & = \inn{x, \T^{**}(y)}, &  & \by{6.9}
  \end{align*}
  by \cref{6.1}(e) we know that \(\T^{**} = \T\).
\end{proof}

\begin{proof}[\pf{6.11}(e)]
  Since
  \begin{align*}
    \forall x, y \in \V, \inn{x, \IT[\V](y)} & = \inn{x, y}             &  & \by{2.1.9} \\
                                             & = \inn{\IT[\V](x), y}    &  & \by{2.1.9} \\
                                             & = \inn{x, \IT[\V]^*(y)}, &  & \by{6.9}
  \end{align*}
  by \cref{6.1}(e) we know that \(\IT[\V]^* = \IT[\V]\).
\end{proof}

\begin{note}
  The same proof works for \cref{6.11} in the infinite-dimensional case, provided that the existence of \(\T^*\) and \(\U^*\) is assumed.
\end{note}

\begin{cor}\label{6.3.2}
  Let \(A, B \in \ms[n][n][\F]\).
  Then
  \begin{enumerate}
    \item \((A + B)^* = A^* + B^*\);
    \item \((cA)^* = \conj{c} A^*\) for all \(c \in \F\).
    \item \((AB)^* = B^* A^*\);
    \item \(A^{**} = A\);
    \item \(I_n^* = I_n\).
  \end{enumerate}
\end{cor}

\begin{proof}[\pf{6.3.2}(a)]
  Since
  \begin{align*}
    \L_{(A + B)^*} & = (\L_{A + B})^*      &  & \by{6.3.1}   \\
                   & = (\L_A + \L_B)^*     &  & \by{2.15}[c] \\
                   & = (\L_A)^* + (\L_B)^* &  & \by{6.11}[a] \\
                   & = \L_{A^*} + \L_{B^*} &  & \by{6.3.1}   \\
                   & = \L_{A^* + B^*},     &  & \by{2.15}[c]
  \end{align*}
  by \cref{2.15}(b) we have \((A + B)^* = A^* + B^*\).
\end{proof}

\begin{proof}[\pf{6.3.2}(b)]
  Since
  \begin{align*}
    \L_{(cA)^*} & = (\L_{cA})^*        &  & \by{6.3.1}   \\
                & = (c \L_A)^*         &  & \by{2.15}[c] \\
                & = \conj{c} (\L_A)^*  &  & \by{6.11}[b] \\
                & = \conj{c} \L_{A^*}  &  & \by{6.3.1}   \\
                & = \L_{\conj{c} A^*}, &  & \by{2.15}[c]
  \end{align*}
  by \cref{2.15}(b) we have \((cA)^* = \conj{c} A^*\).
\end{proof}

\begin{proof}[\pf{6.3.2}(c)]
  Since
  \begin{align*}
    \L_{(AB)^*} & = (\L_{AB})^*       &  & \by{6.3.1}   \\
                & = (\L_A \L_B)^*     &  & \by{2.15}[e] \\
                & = (\L_B)^* (\L_A)^* &  & \by{6.11}[c] \\
                & = \L_{B^*} \L_{A^*} &  & \by{6.3.1}   \\
                & = \L_{B^* A^*},     &  & \by{2.15}[e]
  \end{align*}
  by \cref{2.15}(b) we have \((AB)^* = B^* A^*\).
\end{proof}

\begin{proof}[\pf{6.3.2}(d)]
  Since
  \begin{align*}
    \L_A & = (\L_A)^{**}  &  & \by{6.11}[d] \\
         & = (\L_{A^*})^* &  & \by{6.3.1}   \\
         & = \L_{A^{**}}, &  & \by{6.3.1}
  \end{align*}
  by \cref{2.15}(b) we have \(A = A^{**}\).
\end{proof}

\begin{proof}[\pf{6.3.2}(e)]
  Since
  \begin{align*}
    \L_{I_n} & = \IT[\vs{F}^n]   &  & \by{2.15}[f] \\
             & = \IT[\vs{F}^n]^* &  & \by{6.11}[e] \\
             & = (\L_{I_n})^*    &  & \by{2.15}[f] \\
             & = \L_{I_n^*},     &  & \by{6.3.1}
  \end{align*}
  by \cref{2.15}(b) we have \(I_n = I_n^*\).
\end{proof}

\begin{defn}\label{6.3.3}
  For \(x, y \in \vs{F}^n\), let \(\inn{x, y}_n\) denote the standard inner product of \(x\) and \(y\) in \(\vs{F}^n\).
  Recall that if \(x\) and \(y\) are regarded as column vectors, then \(\inn{x, y}_n = y^* x\).
\end{defn}

\begin{lem}\label{6.3.4}
  Let \(A \in \ms\), \(x \in \vs{F}^n\), and \(y \in \vs{F}^m\).
  Then
  \[
    \inn{Ax, y}_m = \inn{x, A^* y}_n.
  \]
\end{lem}

\begin{proof}[\pf{6.3.4}]
  We have
  \begin{align*}
    \inn{Ax, y}_m & = y^* (Ax)          &  & \by{6.3.3}         \\
                  & = (y^* A) x         &  & \by{2.16}          \\
                  & = (A^* y)^* x       &  & \by{ex:6.3.5}[c,d] \\
                  & = \inn{x, A^* y}_n. &  & \by{6.3.3}
  \end{align*}
\end{proof}

\begin{lem}\label{6.3.5}
  Let \(A \in \ms\).
  Then \(\rk{A^* A} = \rk{A}\).
\end{lem}

\begin{proof}[\pf{6.3.5}]
  By the dimension theorem (\cref{2.3}), we need only show that, for \(x \in \vs{F}^n\), we have \(A^* Ax = \zv\) iff \(Ax = \zv\).
  Clearly, \(Ax = \zv\) implies that \(A^* Ax = \zv\).
  So assume that \(A^* Ax = \zv\).
  Then
  \begin{align*}
    0 & = \inn{A^* Ax, x}_n    &  & \by{6.1}[c]      \\
      & = \inn{Ax, A^{**} x}_m &  & \by{6.3.4}       \\
      & = \inn{Ax, Ax}_m,      &  & \by{ex:6.3.5}[d]
  \end{align*}
  so that \(Ax = \zv\) by \cref{6.1}(d).
\end{proof}

\begin{cor}\label{6.3.6}
  If \(A \in \ms\) such that \(\rk{A} = n\), then \(A^* A\) is invertible.
\end{cor}

\begin{proof}[\pf{6.3.6}]
  By \cref{6.3.5} we have \(n = \rk{A} = \rk{A^* A}\).
  Since \(A^* A \in \ms[n][n][\F]\), by \cref{3.2.2} we know that \(A^* A\) is invertible.
\end{proof}

\begin{thm}\label{6.12}
  Let \(A \in \ms\) and \(y \in \vs{F}^m\).
  Then there exists \(x_0 \in \vs{F}^n\) such that \((A^* A) x_0 = A^* y\) and \(\norm{A x_0 - y} \leq \norm{Ax - y}\) for all \(x \in \vs{F}^n\).
  Furthermore, if \(\rk{A} = n\), then \(x_0 = (A^* A)^{-1} A^* y\).
\end{thm}

\begin{proof}[\pf{6.12}]
  Define \(\W = \set{Ax : x \in \vs{F}^n}\);
  that is, \(\W = \rg{\L_A}\).
  By \cref{6.2.12}, there exists a unique vector in \(\W\) that is closest to \(y\).
  Call this vector \(A x_0\), where \(x_0 \in \vs{F}^n\).
  Then \(\norm{A x_0 - y} \leq \norm{Ax - y}\) for all \(x \in \vs{F}^n\).

  To develop a practical method for finding such an \(x_0\), we note from \cref{6.6,6.2.12} that \(A x_0 - y \in \W^{\perp}\);
  so \(\inn{Ax, A x_0 - y}_m = 0\) for all \(x \in \vs{F}^n\).
  Thus, by \cref{6.3.4}, we have that \(\inn{x, A^* (A x_0 - y)}_n = 0\) for all \(x \in \vs{F}^n\).
  By \cref{6.1}(e) we have \(A^* (A x_0 - y) = 0\).
  So we need only find a solution \(x_0\) to \(A^* Ax = A^* y\).
  If, in addition, we assume that \(\rk{A} = n\), then by \cref{6.3.5} we have \(x_0 = (A^* A)^{-1} A^* y\).
\end{proof}

\begin{note}
  Consider the following problem:
  An experimenter collects data by taking measurements \(\seq{y}{1,,m}\) at times \(\seq{t}{1,,m}\), respectively.
  Suppose that the data \((t_1, y_1), \dots, (t_m, y_m)\) are plotted as points in the plane.
  From this plot, the experimenter feels that there exists an essentially linear relationship between \(y\) and \(t\), say \(y = ct + d\), and would like to find the constants \(c\) and \(d\) so that the line \(y = ct + d\) represents the best possible fit to the data collected.
  One such estimate of fit is to calculate the error \(E\) that represents the sum of the squares of the vertical distances from the points to the line;
  that is,
  \[
    E = \sum_{i = 1}^m (y_i - (c t_i + d))^2.
  \]
  Thus the problem is reduced to finding the constants \(c\) and \(d\) that minimize \(E\).
  (For this reason the line \(y = ct + d\) is called the \textbf{least squares line}.)
  If we let
  \[
    A = \begin{pmatrix}
      t_1    & 1      \\
      \vdots & \vdots \\
      t_m    & 1
    \end{pmatrix}, x = \begin{pmatrix}
      c \\
      d
    \end{pmatrix}, y = \begin{pmatrix}
      y_1    \\
      \vdots \\
      y_m
    \end{pmatrix},
  \]
  then it follows that \(E = \norm{y - Ax}^2\).

  \cref{6.12} develop a general method for finding an explicit vector \(x_0 \in \vs{F}^n\) that minimizes \(E\);
  that is, given \(A \in \ms\), we find \(x_0 \in \vs{F}^n\) such that \(\norm{y - A x_0} \leq \norm{y - Ax}\) for all vectors \(x \in \vs{F}^n\).
  This method not only allows us to find the linear function that best fits the data, but also, for any positive integer \(n\), the best fit using a polynomial of degree at most \(n\).

  Suppose that the experimenter chose the times \(t_i\) for \(i \in \set{1, \dots, m}\) to satisfy
  \[
    \sum_{i = 1}^m t_i = 0.
  \]
  Then the two columns of \(A\) would be orthogonal, so \(A^* A\) would be a diagonal matrix (see \cref{ex:6.3.19}).
  In this case, the computations are greatly simplified.

  In practice, the \(m \times 2\) matrix \(A\) in our least squares application has rank equal to two, and hence \(A^* A\) is invertible by \cref{6.3.6}.
  For, otherwise, the first column of \(A\) is a multiple of the second column, which consists only of ones.
  But this would occur only if the experimenter collects all the data at exactly one time.

  Finally, the method in \cref{6.12} may also be applied if, for some \(k\), the experimenter wants to fit a polynomial of degree at most \(k\) to the data.
  For instance, if a polynomial \(y = a_0 + a_1 t + \cdots + a_k t^k\) of degree at most \(k\) is desired, the appropriate model is
  \[
    A = \begin{pmatrix}
      t_1^k  & \cdots & t_1    & 1      \\
      \vdots &        & \vdots & \vdots \\
      t_m^k  & \cdots & t_m    & 1
    \end{pmatrix}, x = \begin{pmatrix}
      a_k    \\
      \vdots \\
      a_0
    \end{pmatrix}, y = \begin{pmatrix}
      y_1    \\
      \vdots \\
      y_m
    \end{pmatrix}.
  \]
\end{note}

\begin{defn}\label{6.3.7}
  A solution \(s\) to \(Ax = b\) is called a \textbf{minimal solution} if \(\norm{s} \leq \norm{u}\) for all other solutions \(u\).
\end{defn}

\begin{thm}\label{6.13}
  Let \(A \in \ms\) and \(b \in \vs{F}^m\).
  Suppose that \(Ax = b\) is consistent.
  Then the following statements are true.
  \begin{enumerate}
    \item There exists exactly one minimal solution \(s\) of \(Ax = b\), and \(s \in \rg{\L_{A^*}}\).
    \item The vector \(s\) is the only solution to \(Ax = b\) that lies in \(\rg{\L_{A^*}}\);
          that is, if \(u\) satisfies \((A A^*) u = b\), then \(s = A^* u\).
  \end{enumerate}
\end{thm}

\begin{proof}[\pf{6.13}(a)]
  For simplicity of notation, we let \(\W = \rg{\L_{A^*}}\) and \(\W' = \ns{\L_A}\).
  Let \(x\) be any solution to \(Ax = b\).
  By \cref{6.6}, \(x = s + y\) for some \(s \in \W\) and \(y \in \W^{\perp}\).
  But \(\W^{\perp} = \W'\) by \cref{ex:6.3.12}, and therefore \(b = Ax = As + Ay = As\).
  So \(s\) is a solution to \(Ax = b\) that lies in \(\W\).
  To prove (a), we need only show that \(s\) is the unique minimal solution.
  Let \(v\) be any solution to \(Ax = b\).
  By \cref{3.9}, we have that \(v = s + u\), where \(u \in \W'\).
  Since \(s \in \W\), which equals \((\W')^{\perp}\) by \cref{ex:6.3.12}, we have
  \[
    \norm{v}^2 = \norm{s + u}^2 = \norm{s}^2 + \norm{u}^2 \geq \norm{s}^2
  \]
  by \cref{ex:6.1.10}.
  Thus \(s\) is a minimal solution.
  We can also see from the preceding calculation that if \(\norm{v} = \norm{s}\), then \(u = \zv\);
  hence \(v = s\).
  Therefore \(s\) is the unique minimal solution to \(Ax = b\), proving (a).
\end{proof}

\begin{proof}[\pf{6.13}(b)]
  Assume that \(v\) is also a solution to \(Ax = b\) that lies in \(\W\).
  Then
  \[
    v - s \in \W \cap \W' = \W \cap \W^{\perp} = \set{\zv};
  \]
  so \(v = s\).

  Finally, suppose that \((A A^*) u = b\), and let \(v = A^* u\).
  Then \(v \in \W\) and \(Av = b\).
  Therefore \(s = v = A^* u\) by the discussion above.
\end{proof}

\begin{note}
  \cref{6.13} assures that every consistent system of linear equations has a unique minimal solution and provides a method for computing it.
\end{note}

\exercisesection

\setcounter{ex}{4}
\begin{ex}\label{ex:6.3.5}
  Let \(A, B \in \ms\) and let \(C \in \ms[n][p][\F]\).
  Then
  \begin{enumerate}
    \item \((A + B)^* = A^* + B^*\);
    \item \((cA)^* = \conj{c} A^*\) for all \(c \in \F\).
    \item \((AC)^* = C^* A^*\);
    \item \(A^{**} = A\);
  \end{enumerate}
\end{ex}

\begin{proof}[\pf{ex:6.3.5}(a)]
  We have
  \begin{align*}
    ((A + B)^*)_{i j} & = \conj{(A + B)_{j i}}            &  & \by{6.1.5}  \\
                      & = \conj{A_{j i} + B_{j i}}        &  & \by{1.2.9}  \\
                      & = \conj{A_{j i}} + \conj{B_{j i}} &  & \by{d.2}[b] \\
                      & = (A^*)_{i j} + (B^*)_{i j}       &  & \by{6.1.5}  \\
                      & = (A^* + B^*)_{i j}               &  & \by{1.2.9}
  \end{align*}
  where \(i \in \set{1, \dots, n}\) and \(j \in \set{1, \dots, m}\).
  Thus by \cref{1.2.8} \((A + B)^* = A^* + B^*\).
\end{proof}

\begin{proof}[\pf{ex:6.3.5}(b)]
  We have
  \begin{align*}
    ((cA)^*)_{i j} & = \conj{(cA)_{j i}}       &  & \by{6.1.5}  \\
                   & = \conj{c A_{j i}}        &  & \by{1.2.9}  \\
                   & = \conj{c} \conj{A_{j i}} &  & \by{d.2}[c] \\
                   & = \conj{c} (A^*)_{i j}    &  & \by{6.1.5}  \\
                   & = (\conj{c} A^*)_{i j}    &  & \by{1.2.9}
  \end{align*}
  where \(i \in \set{1, \dots, n}\) and \(j \in \set{1, \dots, m}\).
  Thus by \cref{1.2.8} \((cA)^* = \conj{c} A^*\).
\end{proof}

\begin{proof}[\pf{ex:6.3.5}(c)]
  We have
  \begin{align*}
    ((AC)^*)_{i j} & = \conj{(AC)_{j i}}                            &  & \by{6.1.5}    \\
                   & = \conj{\sum_{k = 1}^n A_{j k} C_{k i}}        &  & \by{2.3.1}    \\
                   & = \sum_{k = 1}^n \conj{A_{j k}} \conj{C_{k i}} &  & \by{d.2}[b,c] \\
                   & = \sum_{k = 1}^n (A^*)_{k j} (C^*)_{i k}       &  & \by{6.1.5}    \\
                   & = (C^* A^*)_{i j}                              &  & \by{2.3.1}
  \end{align*}
  where \(i \in \set{1, \dots, n}\) and \(j \in \set{1, \dots, m}\).
  Thus by \cref{1.2.8} \((AC)^* = C^* A^*\).
\end{proof}

\begin{proof}[\pf{ex:6.3.5}(d)]
  We have
  \begin{align*}
    (A^{**})_{i j} & = \conj{(A^*)_{j i}}    &  & \by{6.1.5}  \\
                   & = \conj{\conj{A_{i j}}} &  & \by{6.1.5}  \\
                   & = A_{i j}               &  & \by{d.2}[a]
  \end{align*}
  where \(i \in \set{1, \dots, m}\) and \(j \in \set{1, \dots, n}\).
  Thus by \cref{1.2.8} \(A = A^{**}\).
\end{proof}

\begin{ex}\label{ex:6.3.6}
  Let \(\T\) be a linear operator on an inner product space \(\V\) over \(\F\).
  Let \(\U_1 = \T + \T^*\) and \(\U_2 = \T \T^*\).
  Prove that \(\U_1 = \U_1^*\) and \(\U_2 = \U_2^*\).
\end{ex}

\begin{proof}[\pf{ex:6.3.6}]
  We have
  \begin{align*}
    \U_1^* & = (\T + \T^*)^*                    \\
           & = \T^* + \T^{**} &  & \by{6.11}[a] \\
           & = \T^* + \T      &  & \by{6.11}[d] \\
           & = \U_1
  \end{align*}
  and
  \begin{align*}
    \U_2^* & = (\T \T^*)^*                    \\
           & = \T^{**} \T^* &  & \by{6.11}[c] \\
           & = \T \T^*      &  & \by{6.11}[d] \\
           & = \U_2.
  \end{align*}
\end{proof}

\begin{ex}\label{ex:6.3.7}
  Give an example of a linear operator \(\T\) on an inner product space \(\V\) over \(\F\) such that \(\ns{\T} \neq \ns{\T^*}\).
\end{ex}

\begin{proof}[\pf{ex:6.3.7}]
  Let \(A = \begin{pmatrix}
    0 & 0 \\
    1 & 0
  \end{pmatrix}\).
  Then we have
  \[
    A^* = \begin{pmatrix}
      0 & 1 \\
      0 & 0
    \end{pmatrix}
  \]
  and by \cref{6.3.1}
  \[
    \ns{\L_A} = \spn{e_2} \neq \spn{e_1} = \ns{\L_{A^*}} = \ns{(\L_A)^*}.
  \]
\end{proof}

\begin{ex}\label{ex:6.3.8}
  Let \(\V\) be a finite-dimensional inner product space over \(\F\), and let \(\T \in \ls(\V)\).
  Prove that if \(\T\) is invertible, then \(\T^*\) is invertible and \((\T^*)^{-1} = (\T^{-1})^*\).
\end{ex}

\begin{proof}[\pf{ex:6.3.8}]
  We have
  \begin{align*}
    \IT[\V] & = \IT[\V]^*        &  & \by{6.11}[e] \\
            & = (\T \T^{-1})^*                     \\
            & = (\T^{-1})^* \T^* &  & \by{6.11}[c] \\
            & = (\T^{-1} \T)^*                     \\
            & = \T^* (\T^{-1})^*
  \end{align*}
  and thus \((\T^*)^{-1} = (\T^{-1})^*\).
\end{proof}

\begin{ex}\label{ex:6.3.9}
  Prove that if \(\V = \W \oplus \W^{\perp}\) and \(\T\) is the projection on \(\W\) along \(\W^{\perp}\), then \(\T = \T^*\).
\end{ex}

\begin{proof}[\pf{ex:6.3.9}]
  We have
  \begin{align*}
             & \V = \W \oplus \W^{\perp}                                                                        \\
    \implies & \forall x \in \V, \exists (x_1, x_2) \in \W \times \W^{\perp} : x = x_1 + x_2 &  & \by{5.2.7}    \\
    \implies & \forall x, y \in \V, \inn{x, \T^*(y)} = \inn{\T(x), y}                        &  & \by{6.9}      \\
             & = \inn{x_1, y}                                                                &  & \by{2.1.14}   \\
             & = \inn{x_1, y_1 + y_2} = \inn{x_1, y_1} + \inn{x_1, y_2}                      &  & \by{6.1}[a]   \\
             & = \inn{x_1, y_1}                                                              &  & \by{6.2.9}    \\
             & = \inn{x_1, y_1} + \inn{x_2, y_1}                                             &  & \by{6.2.9}    \\
             & = \inn{x_1 + x_2, y_1}                                                        &  & \by{6.1.1}[a] \\
             & = \inn{x, \T(y)}                                                              &  & \by{2.1.14}   \\
    \implies & \T = \T^*.                                                                    &  & \by{6.1}[e]
  \end{align*}
\end{proof}

\begin{ex}\label{ex:6.3.10}
  Let \(\T\) be a linear operator on an inner product space \(\V\) over \(\F\).
  Prove that \(\norm{\T(x)} = \norm{x}\) for all \(x \in \V\) iff \(\inn{\T(x), \T(y)} = \inn{x,y}\) for all \(x, y \in \V\).
\end{ex}

\begin{proof}[\pf{ex:6.3.10}]
  First suppose that \(\norm{\T(x)} = \norm{x}\) for all \(x \in \V\).
  If \(\F = \C\), then we have
  \begin{align*}
    \forall x, y \in \V, \inn{\T(x), \T(y)} & = \frac{1}{4} \sum_{k = 1}^4 i^k \norm{\T(x) + i^k \T(y)}^2 &  & \by{ex:6.1.20}[b]      \\
                                            & = \frac{1}{4} \sum_{k = 1}^4 i^k \norm{\T(x + i^k y)}^2     &  & \by{2.1.2}[b]          \\
                                            & = \frac{1}{4} \sum_{k = 1}^4 i^k \norm{x + i^k y}^2         &  & \text{(by hypothesis)} \\
                                            & = \inn{x, y}.                                               &  & \by{ex:6.1.20}[b]
  \end{align*}
  If \(\F = \R\), then we have
  \begin{align*}
     & \forall x, y \in \V, \inn{\T(x), \T(y)}                                                        \\
     & = \frac{1}{4} \pa{\norm{\T(x) + \T(y)}^2 - \norm{\T(x) - \T(y)}^2} &  & \by{ex:6.1.20}[a]      \\
     & = \frac{1}{4} \pa{\norm{\T(x + y)}^2 - \norm{\T(x - y)}^2}         &  & \by{2.1.2}[b]          \\
     & = \frac{1}{4} \pa{\norm{x + y}^2 - \norm{x - y}^2}                 &  & \text{(by hypothesis)} \\
     & = \inn{x, y}.                                                      &  & \by{ex:6.1.20}[a]
  \end{align*}
  In either cases we have \(\inn{\T(x), \T(y)} = \inn{x, y}\) for all \(x, y \in \V\).

  Now suppose that \(\inn{\T(x), \T(y)} = \inn{x, y}\) for all \(x, y \in \V\).
  Then we have
  \begin{align*}
             & \forall x \in \V, \norm{\T(x)}^2 = \inn{\T(x), \T(x)} = \inn{x, x} = \norm{x}^2 &  & \by{6.1.9}  \\
    \implies & \forall x \in \V, \norm{\T(x)} = \norm{x}.                                      &  & \by{6.2}[b]
  \end{align*}
\end{proof}

\begin{ex}\label{ex:6.3.11}
  For a linear operator \(\T\) on an inner product space \(\V\) over \(\F\), prove that \(\T^* \T = \zT\) implies \(\T = \zT\).
  Is the same result true if we assume that \(\T \T^* = \zT\)?
\end{ex}

\begin{proof}[\pf{ex:6.3.11}]
  First we show that \(\T^* \T = \zT \implies \T = \zT\).
  This is true since
  \begin{align*}
             & \T^* \T = \zT                                                                  \\
    \implies & \forall x, y \in \V, 0 = \inn{\zv, y} = \inn{(\T^* \T)(x), y} &  & \by{6.1}[c] \\
             & = \inn{\T^*(\T(x)), y} = \inn{\T(x), \T(y)}                   &  & \by{6.9}    \\
    \implies & \forall x \in \V, \inn{\T(x), \T(x)} = 0                                       \\
    \implies & \forall x \in \V, \T(x) = \zv                                 &  & \by{6.1}[d] \\
    \implies & \T = \zT.
  \end{align*}

  Now we show that \(\T \T^* = \zT \implies \T = \zT\).
  Since
  \begin{align*}
             & \forall x, y \in \V, 0 = \inn{\zv, y} = \inn{\zT(x), y} &  & \by{6.1}[c]       \\
             & = \inn{x, \zT^*(y)}                                     &  & \by{6.9}          \\
    \implies & \forall x \in \V, \inn{\zT^*(x), \zT^*(x)} = 0          &  & (\zT^*(x) \in \V) \\
    \implies & \forall x \in \V, \zT^*(x) = \zv                        &  & \by{6.1}[d]       \\
    \implies & \zT^* = \zT,                                            &  & \by{2.1.9}
  \end{align*}
  we have
  \begin{align*}
             & \T^{**} \T^* = \T \T^* = \zT &  & \by{6.11}[d]                  \\
    \implies & \T^* = \zT                   &  & \text{(from the proof above)} \\
    \implies & \T = \T^{**} = \zT^* = \zT.  &  & \text{(from the proof above)}
  \end{align*}
\end{proof}

\begin{ex}\label{ex:6.3.12}
  Let \(\V\) be an inner product space over \(\F\), and let \(\T \in \ls(\V)\).
  Prove the following results.
  \begin{enumerate}
    \item \(\rg{\T^*}^{\perp} = \ns{\T}\).
    \item If \(\V\) is finite-dimensional, then \(\rg{\T^*} = \ns{\T}^{\perp}\).
  \end{enumerate}
\end{ex}

\begin{proof}[\pf{ex:6.3.12}(a)]
  We have
  \begin{align*}
         & v \in \rg{\T^*}^{\perp}                                    \\
    \iff & \forall y \in \rg{\T^*}, \inn{v, y} = 0 &  & \by{6.2.9}    \\
    \iff & \forall x \in \V, \inn{v, \T^*(x)} = 0  &  & \by{2.1.10}   \\
    \iff & \forall x \in \V, \inn{\T(v), x} = 0    &  & \by{6.9}      \\
    \iff & \forall x \in \V, \inn{x, \T(v)} = 0    &  & \by{6.1.1}[c] \\
    \iff & \T(v) = \zv                             &  & \by{6.1}[c,e] \\
    \iff & v \in \ns{\T}                           &  & \by{2.1.10}
  \end{align*}
  and thus \(\rg{\T^*}^{\perp} = \ns{\T}\).
\end{proof}

\begin{proof}[\pf{ex:6.3.12}(b)]
  We have
  \begin{align*}
    \rg{\T^*} & = (\rg{\T^*}^{\perp})^{\perp} &  & \by{ex:6.2.13}[c] \\
              & = \ns{\T}^{\perp}.            &  & \by{ex:6.3.12}[a]
  \end{align*}
\end{proof}

\begin{ex}\label{ex:6.3.13}
  Let \(\T\) be a linear operator on a finite-dimensional inner product space \(\V\) over \(\F\).
  Prove the following results.
  \begin{enumerate}
    \item \(\ns{\T^* \T} = \ns{\T}\).
          Deduce that \(\rk{\T^* \T} = \rk{\T}\).
    \item \(\rk{\T} = \rk{\T^*}\).
          Deduce from (a) that \(\rk{\T \T^*} = \rk{\T}\).
    \item For any \(A \in \ms[n][n][\F]\), \(\rk{A^* A} = \rk{A A^*} = \rk{A}\).
  \end{enumerate}
\end{ex}

\begin{proof}[\pf{ex:6.3.13}(a)]
  We have
  \begin{align*}
         & v \in \ns{\T^* \T}                                            \\
    \iff & (\T^* \T)(v) = \T^*(\T(v)) = \zv           &  & \by{2.1.10}   \\
    \iff & \forall x \in \V, \inn{x, \T^*(\T(v))} = 0 &  & \by{6.1}[c,d] \\
    \iff & \forall x \in \V, \inn{\T(x), \T(v)} = 0   &  & \by{6.9}      \\
    \iff & \T(v) = \zv                                &  & \by{6.1}[c,d] \\
    \iff & v \in \ns{\T}                              &  & \by{2.1.10}
  \end{align*}
  and thus \(\ns{\T^* \T} = \ns{\T}\).
  Since \(\V\) is finite-dimensional, we have
  \begin{align*}
             & \dim(\V) = \begin{dcases}
                            \rk{\T^* \T} + \nt{\T^* \T} \\
                            \rk{\T} + \nt{\T}
                          \end{dcases}                  &  & \by{2.3}                             \\
    \implies & \rk{\T^* \T} = \rk{\T} + \nt{\T} - \nt{\T^* \T}                                    \\
             & = \rk{\T}.                                      &  & \text{(from the proof above)}
  \end{align*}
\end{proof}

\begin{proof}[\pf{ex:6.3.13}(b)]
  Since
  \begin{align*}
    \rg{\T^* \T} & = \rg{(\T^* \T)^*}     &  & \by{6.11}[c,d]    \\
                 & = \ns{\T^* \T}^{\perp} &  & \by{ex:6.3.12}[b] \\
                 & = \ns{\T}^{\perp}      &  & \by{ex:6.3.13}[a] \\
                 & = \rg{\T^*},           &  & \by{ex:6.3.12}[b]
  \end{align*}
  by \cref{ex:6.3.13}(a) we have \(\rk{\T^* \T} = \rk{\T^*} = \rk{\T}\).
  Thus
  \begin{align*}
             & \rk{(\T^*)^* \T^*} = \rk{(\T^*)^*} &  & \text{(from the proof above)} \\
    \implies & \rk{\T \T^*} = \rk{\T}.            &  & \by{6.11}[d]
  \end{align*}
\end{proof}

\begin{proof}[\pf{ex:6.3.13}(c)]
  We have
  \begin{align*}
    \rk{A^* A} & = \rk{\L_{A^* A}}    &  & \by{3.2.1}        \\
               & = \rk{\L_{A^*} \L_A} &  & \by{2.15}[e]      \\
               & = \rk{(\L_A)^* \L_A} &  & \by{6.3.1}        \\
               & = \rk{\L_A}          &  & \by{ex:6.3.13}[a] \\
               & = \rk{A}             &  & \by{3.2.1}        \\
               & = \rk{\L_A (\L_A)^*} &  & \by{ex:6.3.13}[b] \\
               & = \rk{\L_A \L_{A^*}} &  & \by{6.3.1}        \\
               & = \rk{\L_{A A^*}}    &  & \by{2.15}[e]      \\
               & = \rk{A A^*}.        &  & \by{3.2.1}
  \end{align*}
\end{proof}

\begin{ex}\label{ex:6.3.14}
  Let \(\V\) be an inner product space over \(\F\), and let \(y, z \in \V\).
  Define \(\T : \V \to \V\) by \(\T(x) = \inn{x, y} z\) for all \(x \in \V\).
  First prove that \(\T \in \ls(\V)\).
  Then show that \(\T^*\) exists, and find an explicit expression for it.
\end{ex}

\begin{proof}[\pf{ex:6.3.14}]
  Let \(x_1, x_2 \in \V\) and let \(c \in \F\).
  Since
  \begin{align*}
    \T(c x_1 + x_2) & = \inn{c x_1 + x_2, y} z            &  & \by{ex:6.3.14}  \\
                    & = (c \inn{x_1, y} + \inn{x_2, y}) z &  & \by{6.1.1}[a,b] \\
                    & = c \inn{x_1, y} z + \inn{x_2, y} z &  & \by{1.2.1}      \\
                    & = c \T(x_1) + \T(x_2),              &  & \by{ex:6.3.14}
  \end{align*}
  by \cref{2.1.2}(b) we see that \(\T \in \ls(\V)\).
  Now we define \(\U : \V \to \V\) as follow:
  \[
    \forall x \in \V, \U(x) = \inn{x, z} y.
  \]
  Since
  \begin{align*}
    \inn{\T(x_1), x_2} & = \inn{\inn{x_1, y} z, x_2}               &  & \by{ex:6.3.14} \\
                       & = \conj{\inn{x_2, \inn{x_1, y} z}}        &  & \by{6.1.1}[c]  \\
                       & = \conj{\conj{\inn{x_1, y}} \inn{x_2, z}} &  & \by{6.1}[b]    \\
                       & = \inn{x_1, y} \conj{\inn{x_2, z}}        &  & \by{d.2}[a,c]  \\
                       & = \conj{\inn{x_2, z}} \inn{x_1, y}                            \\
                       & = \inn{x_1, \inn{x_2, z} y}               &  & \by{6.1}[b]    \\
                       & = \inn{x_1, \U(x_2)},
  \end{align*}
  by \cref{6.9} we see that \(\T^*\) exists and \(\T^* = \U\).
\end{proof}

\begin{defn}\label{6.3.8}
  Let \(\T \in \ls(\V, \W)\), where \(\V\) and \(\W\) are finite-dimensional inner product spaces over \(\F\) with inner products \(\inn{\cdot, \cdot}_1\) and \(\inn{\cdot, \cdot}_2\), respectively.
  A function \(\T^* : \W \to \V\) is called an \textbf{adjoint} of \(\T\) if \(\inn{\T(x), y}_2 = \inn{x, \T^*(y)}_1\) for all \(x \in \V\) and \(y \in \W\).
\end{defn}

\begin{ex}\label{ex:6.3.15}
  Let \(\T \in \ls(\V, \W)\), where \(\V\) and \(\W\) are finite-dimensional inner product spaces over \(\F\) with inner products \(\inn{\cdot, \cdot}_1\) and \(\inn{\cdot, \cdot}_2\), respectively.
  Prove the following results.
  \begin{enumerate}
    \item There is a unique adjoint \(\T^*\) of \(\T\), and \(\T^* \in \ls(\W, \V)\).
    \item If \(\beta\) and \(\gamma\) are orthonormal bases for \(\V\) and \(\W\) over \(\F\), respectively, then \([\T^*]_{\gamma}^{\beta} = ([\T]_{\beta}^{\gamma})^*\).
    \item \(\rk{\T^*} = \rk{\T}\).
    \item \(\inn{\T^*(x), y}_1 = \inn{x, \T(y)}_2\) for all \(x \in \W\) and \(y \in \V\).
    \item For all \(x \in \V\), \(\T^* \T(x) = \zv\) iff \(\T(x) = \zv\).
  \end{enumerate}
\end{ex}

\begin{proof}[\pf{ex:6.3.15}(a)]
  Let \(y \in \W\).
  Define \(g : \V \to \F\) by \(g(x) = \inn{\T(x), y}_2\) for all \(x \in \V\).
  We first show that \(g \in \ls(\V, \F)\).
  Let \(x_1, x_2 \in \V\) and let \(c \in \F\).
  Then
  \begin{align*}
    g(c x_1 + x_2) & = \inn{\T(c x_1 + x_2), y}_2                                     \\
                   & = \inn{c \T(x_1) + \T(x_2), y}_2            &  & \by{2.1.2}[b]   \\
                   & = c \inn{\T(x_1), y}_2 + \inn{\T(x_2), y}_2 &  & \by{6.1.1}[a,b] \\
                   & = c g(x_1) + g(x_2).
  \end{align*}
  Hence by \cref{2.1.2}(b) we see that \(g \in \ls(\V, \F)\).

  We now apply \cref{6.8} to obtain a unique vector \(y' \in \V\) such that \(g(x) = \inn{x, y'}_1\);
  that is, \(\inn{\T(x), y}_2 = \inn{x, y'}_1\) for all \(x \in \V\).
  Defining \(\T^* : \W \to \V\) by \(\T^*(y) = y'\), we have \(\inn{\T(x), y}_2 = \inn{x, \T^*(y)}_1\).

  To show that \(\T^*\) is linear, let \(y_1, y_2 \in \W\) and \(c \in \F\).
  Then for any \(x \in \V\),
  we have
  \begin{align*}
    \inn{x, \T^*(c y_1 + y_2)}_1 & = \inn{\T(x), c y_1 + y_2}_2                                              \\
                                 & = \conj{c} \inn{\T(x), y_1}_2 + \inn{\T(x), y_2}_2     &  & \by{6.1}[a,b] \\
                                 & = \conj{c} \inn{x, \T^*(y_1)}_1 + \inn{x, \T^*(y_2)}_1                    \\
                                 & = \inn{x, c \T^*(y_1) + \T^*(y_2)}_1.                  &  & \by{6.1}[a,b]
  \end{align*}
  Since \(x\) is arbitrary, \(\T^*(c y_1 + y_2) = c \T^*(y_1) + \T^*(y_2)\) by \cref{6.1}(e).

  Finally, we need to show that \(\T^*\) is unique. Suppose that \(\U \in \ls(\W, \V)\) and that it satisfies \(\inn{\T(x), y}_2 = \inn{x, \U(y)}_1\) for all \((x, y) \in \V \times \W\).
  Then \(\inn{x, \T^*(y)}_1 = \inn{x, \U(y)}_1\) for all \((x, y) \in \V \times \W\), so \(\T^* = \U\).
\end{proof}

\begin{proof}[\pf{ex:6.3.15}(b)]
  Let \(i \in \set{1, \dots, m}\) and let \(j \in \set{1, \dots, n}\).
  Then we have
  \begin{align*}
    \pa{[\T^*]_{\gamma}^{\beta}}_{i j} & = \inn{\T^*(w_j), v_i}_1                   &  & \by{6.5}      \\
                                       & = \conj{\inn{v_i, \T^*(w_j)}_1}            &  & \by{6.1.1}[c] \\
                                       & = \conj{\inn{\T(v_i), w_j}_2}              &  & \by{6.3.8}    \\
                                       & = \conj{([\T]_{\beta}^{\gamma})_{j i}}     &  & \by{6.5}      \\
                                       & = \pa{\pa{[\T]_{\beta}^{\gamma}}^*}_{i j}. &  & \by{6.1.5}
  \end{align*}
  Hence \([\T^*]_{\gamma}^{\beta} = \pa{[\T]_{\beta}^{\gamma}}^*\).
\end{proof}

\begin{proof}[\pf{ex:6.3.15}(c)]
  First we show that \(\ns{\T} = \rg{\T^*}^{\perp}\).
  This is true since
  \begin{align*}
    \iff & v \in \ns{\T}                                                \\
    \iff & \T(v) = \zv_{\W}                          &  & \by{2.1.10}   \\
    \iff & \forall y \in \W, \inn{y, \T(v)}_2 = 0    &  & \by{6.1}[c,d] \\
    \iff & \forall y \in \W, \inn{\T(v), y}_2 = 0    &  & \by{6.1.1}[c] \\
    \iff & \forall y \in \W, \inn{v, \T^*(y)}_1 = 0  &  & \by{6.3.8}    \\
    \iff & \forall x \in \rg{\T^*}, \inn{v, x}_1 = 0                    \\
    \iff & v \in \rg{\T^*}^{\perp}.                  &  & \by{6.2.9}
  \end{align*}

  Now we show that \(\rk{\T^*} = \rk{\T}\).
  This is true since
  \begin{align*}
    \rk{\T} & = \dim(\V) - \nt{\T}                 &  & \by{2.3}                      \\
            & = \dim(\V) - \dim(\ns{\T})           &  & \by{2.1.12}                   \\
            & = \dim(\V) - \dim(\rg{\T^*}^{\perp}) &  & \text{(from the proof above)} \\
            & = \dim(\rg{\T^*})                    &  & \by{ex:6.2.13}[d]             \\
            & = \rk{\T^*}.                         &  & \by{2.1.12}
  \end{align*}
\end{proof}

\begin{proof}[\pf{ex:6.3.15}(d)]
  We have
  \begin{align*}
    \inn{\T^*(x), y}_1 & = \conj{\inn{y, \T^*(x)}_1} &  & \by{6.1.1}[c] \\
                       & = \conj{\inn{\T(y), x}_2}   &  & \by{6.3.8}    \\
                       & = \inn{x, \T(y)}_2.         &  & \by{6.1.1}[c]
  \end{align*}
\end{proof}

\begin{proof}[\pf{ex:6.3.15}(e)]
  We have
  \begin{align*}
         & v \in \ns{\T^* \T}                                              \\
    \iff & (\T^* \T)(v) = \T^*(\T(v)) = \zv_{\V}        &  & \by{2.1.10}   \\
    \iff & \forall x \in \V, \inn{x, \T^*(\T(v))}_1 = 0 &  & \by{6.1}[c,d] \\
    \iff & \forall x \in \V, \inn{\T(x), \T(v)}_2 = 0   &  & \by{6.3.8}    \\
    \iff & \T(v) = \zv_{\W}                             &  & \by{6.1}[c,d] \\
    \iff & v \in \ns{\T}                                &  & \by{2.1.10}
  \end{align*}
  and thus \(\ns{\T^* \T} = \ns{\T}\).
\end{proof}

\begin{ex}\label{ex:6.3.16}
  Let \(\V, \W, \vs{X}\) be finite-dimensional inner product spaces over \(\F\).
  Let \(\seq{\inn{\cdot, \cdot}}{1,2,3}\) be inner products on \(\V, \W, \vs{X}\) over \(\F\), respectively.
  Let \(\T, \U \in \ls(\V, \W)\) and let \(\lt{S} \in \ls(\W, \vs{X})\).
  Then
  \begin{enumerate}
    \item \((\T + \U)^* = \T^* + \U^*\);
    \item \((c \T)^* = \conj{c} \T^*\) for any \(c \in \F\);
    \item \((\lt{S} \T)^* = \T^* \lt{S}^*\);
    \item \(\T^{**} = \T\);
  \end{enumerate}
\end{ex}

\begin{proof}[\pf{ex:6.3.16}(a)]
  Since
  \begin{align*}
     & \forall (x, y) \in \V \times \W, \inn{x, (\T + \U)^*(y)}_1                    \\
     & = \inn{(\T + \U)(x), y}_2                                  &  & \by{6.3.8}    \\
     & = \inn{\T(x) + \U(x), y}_2                                 &  & \by{2.2.5}    \\
     & = \inn{\T(x), y}_2 + \inn{\U(x), y}_2                      &  & \by{6.1.1}[a] \\
     & = \inn{x, \T^*(y)}_1 + \inn{x, \U^*(y)}_1                  &  & \by{6.3.8}    \\
     & = \inn{x, \T^*(y) + \U^*(y)}_1                             &  & \by{6.1}[a]   \\
     & = \inn{x, (\T^* + \U^*)(y)}_1,                             &  & \by{2.2.5}
  \end{align*}
  by \cref{6.1}(e) we know that \(\T^* + \U^* = (\T + \U)^*\).
\end{proof}

\begin{proof}[\pf{ex:6.3.16}(b)]
  Since
  \begin{align*}
    \forall (x, y) \in \V \times \W, \inn{x, (c \T)^*(y)}_1 & = \inn{(c \T)(x), y}_2           &  & \by{6.3.8}    \\
                                                            & = \inn{c \T(x), y}_2             &  & \by{2.2.5}    \\
                                                            & = c \inn{\T(x), y}_2             &  & \by{6.1.1}[b] \\
                                                            & = c \inn{x, \T^*(y)}_1           &  & \by{6.3.8}    \\
                                                            & = \inn{x, \conj{c} \T^*(y)}_1    &  & \by{6.1}[b]   \\
                                                            & = \inn{x, (\conj{c} \T^*)(y)}_1, &  & \by{2.2.5}
  \end{align*}
  by \cref{6.1}(e) we know that \((c \T)^* = \conj{c} \T^*\).
\end{proof}

\begin{proof}[\pf{ex:6.3.16}(c)]
  Since
  \begin{align*}
    \forall (x, y) \in \V \times \vs{X}, \inn{x, (\lt{S} \T)^*(y)}_1 & = \inn{(\lt{S} \T)(x), y}_3      &  & \by{6.3.8} \\
                                                                     & = \inn{\lt{S}(\T(x)), y}_3                       \\
                                                                     & = \inn{\T(x), \lt{S}^*(y)}_2     &  & \by{6.3.8} \\
                                                                     & = \inn{x, \T^*(\lt{S}^*(y))}_1   &  & \by{6.3.8} \\
                                                                     & = \inn{x, (\T^* \lt{S}^*)(y)}_1,
  \end{align*}
  by \cref{6.1}(e) we know that \((\lt{S} \T)^* = \T^* \lt{S}^*\).
\end{proof}

\begin{proof}[\pf{ex:6.3.16}(d)]
  Since
  \begin{align*}
    \forall (x, y) \in \V \times \W, \inn{x, \T(y)}_1 & = \inn{\T^*(x), y}_2     &  & \by{ex:6.3.15}[d] \\
                                                      & = \inn{x, \T^{**}(y)}_1, &  & \by{6.3.8}
  \end{align*}
  by \cref{6.1}(e) we know that \(\T^{**} = \T\).
\end{proof}

\begin{ex}\label{ex:6.3.17}
  Let \(T \in \ls(\V, \W\), where \(\V\) and \(\W\) are finite-dimensional inner product spaces over \(\F\).
  Prove that \(\rg{\T^*}^{\perp} = \ns{\T}\), using \cref{6.3.8}.
\end{ex}

\begin{proof}[\pf{ex:6.3.17}]
  See proof of \cref{ex:6.3.15}(c).
\end{proof}

\begin{ex}\label{ex:6.3.18}
  Let \(A \in \ms[n][n][\F]\).
  Prove that \(\det(A^*) = \conj{\det(A)}\).
\end{ex}

\begin{proof}[\pf{ex:6.3.18}]
  We have
  \begin{align*}
    \det(A^*) & = \det(\conj{\tp{A}}) &  & \by{6.1.5}        \\
              & = \conj{\det(\tp{A})} &  & \by{ex:4.3.13}[a] \\
              & = \conj{\det(A)}.     &  & \by{4.8}
  \end{align*}
\end{proof}

\begin{ex}\label{ex:6.3.19}
  Suppose that \(A \in \ms\) and no two columns of \(A\) are identical.
  Prove that \(A^* A\) is a diagonal matrix iff every pair of columns of \(A\) is orthogonal.
\end{ex}

\begin{proof}[\pf{ex:6.3.19}]
  For each \(i \in \set{1, \dots, n}\), let \(a_i\) be the \(i\)th column of \(A\).
  Then we have
  \begin{align*}
         & A^* A \text{ is diagonal}                                                                                                  \\
    \iff & \forall i, j \in \set{1, \dots, n}, (A^* A)_{i j} = (A^* A)_{i j} \delta_{i j}                         &  & \by{ex:2.3.10} \\
    \iff & \forall i, j \in \set{1, \dots, n}, \sum_{k = 1}^m (A^*)_{i k} A_{k j} = (A^* A)_{i j} \delta_{i j}    &  & \by{2.3.1}     \\
    \iff & \forall i, j \in \set{1, \dots, n}, \sum_{k = 1}^m \conj{A_{k i}} A_{k j} = (A^* A)_{i j} \delta_{i j} &  & \by{6.1.5}     \\
    \iff & \forall i, j \in \set{1, \dots, n}, \inn{a_j, a_i} = (A^* A)_{i j} \delta_{i j}                        &  & \by{6.1.2}     \\
    \iff & \text{the set of columns of } A \text{ is orthogonal}.                                                 &  & \by{6.1.12}
  \end{align*}
\end{proof}

\setcounter{ex}{22}
\begin{ex}\label{ex:6.3.23}
  Consider the problem of finding the least squares line \(y = ct + d\) corresponding to the \(m\) observations \((t_1, y_1), \dots, (t_m, y_m)\).
  \begin{enumerate}
    \item Show that the equation \((A^* A) x_0 = A^* y\) of \cref{6.12} takes the form of the \emph{normal equations}:
          \[
            \pa{\sum_{i = 1}^m t_i^2} c + \pa{\sum_{i = 1}^m t_i} d = \sum_{i = 1}^m t_i y_i
          \]
          and
          \[
            \pa{\sum_{i = 1}^m t_i} c + md = \sum_{i = 1}^m y_i.
          \]
          These equations may also be obtained from the error \(E\) by setting the partial derivatives of \(E\) with respect to both \(c\) and \(d\) equal to zero.
    \item Use the second normal equation of (a) to show that the least squares line must pass through the \emph{center of mass}, \((\overline{t}, \overline{y})\), where
          \[
            \overline{t} = \frac{1}{m} \sum_{i = 1}^m t_i \quad \text{and} \quad \overline{y} = \frac{1}{m} \sum_{i = 1}^m y_i.
          \]
  \end{enumerate}
\end{ex}

\begin{proof}[\pf{ex:6.3.23}(a)]
  Since
  \begin{align*}
    A         & = \begin{pmatrix}
                    t_1    & 1      \\
                    \vdots & \vdots \\
                    t_m    & 1
                  \end{pmatrix}                                                               &  & \by{6.12}    \\
    A^*       & = \begin{pmatrix}
                    t_1 & \cdots & t_m \\
                    1   & \cdots & 1
                  \end{pmatrix}                                                            &  & \by{6.1.5}      \\
    A^* A     & = \begin{pmatrix}
                    \sum_{i = 1}^m t_i^2 & \sum_{i = 1}^m t_i \\
                    \sum_{i = 1}^m t_i   & m
                  \end{pmatrix}                                     &  & \by{2.3.1}                             \\
    A^* A x_0 & = \begin{pmatrix}
                    \sum_{i = 1}^m t_i^2 & \sum_{i = 1}^m t_i \\
                    \sum_{i = 1}^m t_i   & m
                  \end{pmatrix} \begin{pmatrix}
                                  c \\
                                  d
                                \end{pmatrix}                                     &  & \by{6.12}                \\
              & = \begin{pmatrix}
                    \pa{\sum_{i = 1}^m t_i^2} c + \pa{\sum_{i = 1}^m t_i} d \\
                    \pa{\sum_{i = 1}^m t_i} c + md
                  \end{pmatrix} &  & \by{2.3.1}                 \\
    A^* y     & = \begin{pmatrix}
                    t_1 & \cdots & t_m \\
                    1   & \cdots & 1
                  \end{pmatrix} \begin{pmatrix}
                                  y_1    \\
                                  \vdots \\
                                  y_m
                                \end{pmatrix}                                                    &  & \by{6.12} \\
              & = \begin{pmatrix}
                    \sum_{i = 1}^m t_i y_i \\
                    \sum_{i = 1}^m y_i
                  \end{pmatrix},                                                        &  & \by{2.3.1}
  \end{align*}
  we have
  \begin{align*}
             & A^* A x_0 = A^* y                                                                                         &  & \by{6.12} \\
    \implies & \begin{dcases}
                 \pa{\sum_{i = 1}^m t_i^2} c + \pa{\sum_{i = 1}^m t_i} d = \sum_{i = 1}^m t_i y_i \\
                 \pa{\sum_{i = 1}^m t_i} c + md = \sum_{i = 1}^m y_i
               \end{dcases}.
  \end{align*}
\end{proof}

\begin{proof}[\pf{ex:6.3.23}(b)]
  We have
  \begin{align*}
             & \frac{1}{m} \pa{\pa{\sum_{i = 1}^m t_i} c + md} = \frac{1}{m} \pa{\sum_{i = 1}^m y_i} &  & \by{ex:6.3.23}[a] \\
    \implies & \overline{t} c + d = \overline{y}                                                                            \\
    \implies & (\overline{t}, \overline{y}) \text{ pass through } y = ct + d.                        &  & \by{6.12}
  \end{align*}
\end{proof}

\begin{ex}\label{ex:6.3.24}
  Let \(\V\) and \(\set{\seq{e}{1,2,}}\) be defined as in \cref{ex:6.2.23}.
  Define \(\T : \V \to \V\) by
  \[
    \T(\sigma)(k) = \sum_{i = k}^\infty \sigma(i) \quad \text{for } k \in \Z^+.
  \]
  Notice that the infinite series in the definition of \(\T\) converges because \(\sigma(i) \neq 0\) for only finitely many \(i \in \Z^+\).
  \begin{enumerate}
    \item Prove that \(\T \in \ls(\V)\).
    \item Prove that for any positive integer \(n\), \(\T(e_n) = \sum_{i = 1}^n e_i\).
    \item Prove that \(\T\) has no adjoint
  \end{enumerate}
\end{ex}

\begin{proof}[\pf{ex:6.3.24}(a)]
  Let \(a, b \in \V\) and let \(c \in \F\).
  Since
  \begin{align*}
    \forall k \in \Z^+, \T(ca + b)(k) & = \sum_{i = k}^\infty (ca + b)(i)                       &  & \by{ex:6.3.24} \\
                                      & = c \sum_{i = k}^\infty a(i) + \sum_{i = k}^\infty b(i)                     \\
                                      & = c \T(a)(k) + \T(b)(k)                                 &  & \by{ex:6.3.24} \\
                                      & = (c \T(a) + \T(b))(k),
  \end{align*}
  by \cref{1.2.13} and \cref{2.1.2}(b) we see that \(\T \in \ls(\V)\).
\end{proof}

\begin{proof}[\pf{ex:6.3.24}(b)]
  Let \(n \in \Z^+\).
  Then we have
  \begin{align*}
    \forall k \in \Z^+, \T(e_n)(k) & = \sum_{i = k}^\infty e_n(i)       &  & \by{ex:6.3.24}    \\
                                   & = \sum_{i = k}^\infty \delta_{n i} &  & \by{ex:6.2.23}[b] \\
                                   & = \begin{dcases}
                                         1 & \text{if } k \leq n \\
                                         0 & \text{if } k > n
                                       \end{dcases}         &  & \by{2.3.4}                    \\
                                   & = \sum_{i = 1}^n e_i(k).           &  & \by{ex:6.2.23}[b]
  \end{align*}
\end{proof}

\begin{proof}[\pf{ex:6.3.24}(c)]
  Suppose for sake of contradiction that \(\T^* \in \ls(\V)\) exists.
  Then we have
  \begin{align*}
    \forall k \in \Z^+, \T^*(e_1)(k) & = \conj{\conj{\T^*(e_1)(k)}}                                  &  & \by{d.2}[a]       \\
                                     & = \conj{\delta_{k k} \cdot \conj{\T^*(e_1)(k)}}               &  & \by{2.3.4}        \\
                                     & = \conj{\sum_{i = 1}^\infty e_k(i) \cdot \conj{\T^*(e_1)(i)}} &  & \by{ex:6.2.23}[b] \\
                                     & = \conj{\inn{e_k, \T^*(e_1)}}                                 &  & \by{ex:6.2.23}    \\
                                     & = \conj{\inn{\T(e_k), e_1}}                                   &  & \by{6.9}          \\
                                     & = \conj{\inn{\sum_{i = 1}^k e_k, e_1}}                        &  & \by{ex:6.3.24}[b] \\
                                     & = \conj{\sum_{i = 1}^k \inn{e_k, e_1}}                        &  & \by{6.1.1}[a]     \\
                                     & = \conj{1} = 1.                                               &  & \by{ex:6.2.23}[b]
  \end{align*}
  But \(\T^*(e_1) \in \V\) implies there are only finite number of nonzeros in \(\T^*(e_1)\), a contradiction.
  Thus \(\T^*\) does not exist.
\end{proof}

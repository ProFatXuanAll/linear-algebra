\section{Invariant Subspaces and the Cayley-Hamilton Theorem}\label{sec:5.4}

\begin{defn}\label{5.4.1}
  Let \(\T\) be a linear operator on a vector space \(\V\) over \(\F\).
  A subspace \(\W\) of \(\V\) over \(\F\) is called a \textbf{\(\T\)-invariant subspace} of \(\V\) over \(\F\) if \(\T(\W) \subseteq \W\), that is, if \(\T(v) \in \W\) for all \(v \in \W\).
\end{defn}

\begin{eg}\label{5.4.2}
  Suppose that \(\T\) is a linear operator on a vector space \(\V\) over \(\F\).
  Then the following subspaces of \(\V\) are \(\T\)-invariant:
  \begin{enumerate}
    \item \(\set{\zv}\)
    \item \(\V\)
    \item \(\rg{\T}\)
    \item \(\ns{\T}\)
    \item \(E_{\lambda}\), for any eigenvalue \(\lambda\) of \(\T\).
  \end{enumerate}
\end{eg}

\begin{proof}[\pf{5.4.2}]
  We have
  \begin{align*}
    \T(\set{\zv}) & = \set{\zv} \subseteq \set{\zv} &  & \text{(by \cref{2.1.2}(a))} \\
    \T(\V)        & = \rg{\T} \subseteq \V          &  & \text{(by \cref{2.5.2})}    \\
    \T(\rg{\T})   & \subseteq \T(\V) = \rg{\T}      &  & \text{(by \cref{2.1.10})}   \\
    \T(\ns{\T})   & = \set{\zv} \subseteq \ns{\T}   &  & \text{(by \cref{2.1.10})}
  \end{align*}
  and
  \begin{align*}
             & \forall v \in E_{\lambda}, \T(v) = \lambda v \in E_{\lambda} &  & \text{(by \cref{5.2.4})} \\
    \implies & \T(E_{\lambda}) \subseteq E_{\lambda}.
  \end{align*}
  Thus by \cref{5.4.1} \(\set{\zv}, \V, \rg{\T}, \ns{\T}, E_{\lambda}\) are \(\T\)-invariant subspace of \(\V\) over \(\F\).
\end{proof}

\begin{defn}\label{5.4.3}
  Let \(\T\) be a linear operator on a vector space \(\V\) over \(\F\), and let \(x\) be a nonzero vector in \(\V\).
  The subspace
  \[
    \W = \spn{\set{x, \T(x), \T^2(x), \dots}}
  \]
  is called the \textbf{\(\T\)-cyclic subspace of \(\V\) generated by \(x\)}.
  It is a simple matter to show that \(\W\) is \(\T\)-invariant.
  In fact, \(\W\) is the ``smallest'' \(\T\)-invariant subspace of \(\V\) containing \(x\).
  That is, any \(\T\)-invariant subspace of \(\V\) containing \(x\) must also contain \(\W\)
  (see \cref{ex:5.4.11}).
\end{defn}

\begin{thm}\label{5.21}
  Let \(\T\) be a linear operator on a finite-dimensional vector space \(\V\) over \(\F\), and let \(\W\) be a \(\T\)-invariant subspace of \(\V\) over \(\F\).
  Then the characteristic polynomial of \(\T_{\W}\) divides the characteristic polynomial of \(\T\).
\end{thm}

\begin{proof}[\pf{5.21}]
  Choose an ordered basis \(\gamma = \set{\seq{v}{1,,k}}\) for \(\W\) over \(\F\), and extend it to an ordered basis \(\beta = \set{\seq{v}{1,,k,k+1,n}}\) for \(\V\) over \(\F\).
  Let \(A = [\T]_{\beta}\) and \(B_1 = [\T_{\W}]_{\gamma}\).
  Then, by \cref{ex:5.4.12}, \(A\) can be written in the form
  \[
    A = \begin{pmatrix}
      B_1 & B_2 \\
      \zm & B_3
    \end{pmatrix}.
  \]
  Let \(f\) be the characteristic polynomial of \(\T\) and \(g\) the characteristic polynomial of \(\T_{\W}\).
  Then
  \[
    f(t) = \det(A - t I_n) = \det\begin{pmatrix}
      B_1 - t I_k & B_2               \\
      \zm         & B_3 - t I_{n - k}
    \end{pmatrix} = g(t) \cdot \det(B_3 - t I_{n - k})
  \]
  by \cref{ex:4.3.21}.
  Thus \(g\) divides \(f\).
\end{proof}

\exercisesection

\setcounter{ex}{3}
\begin{ex}\label{ex:5.4.4}
  Let \(\T\) be a linear operator on a vector space \(\V\) over \(\F\), and let \(\W\) be a \(\T\)-invariant subspace of \(\V\) over \(\F\).
  Prove that \(\W\) is \(g(\T)\)-invariant for any polynomial \(g\).
\end{ex}

\begin{proof}[\pf{ex:5.4.4}]
  Let \(g(x) = a_0 + a_1 x + \cdots + a_n x^n\) for some \(\seq{a}{0,,n} \in \F\).
  Then we have
  \begin{align*}
    \forall w \in \W, g(\T)(w) & = (a_0 \IT[\V] + a_1 \T + \cdots + a_n \T^n)(w) &  & \text{(by \cref{e.0.7})} \\
                               & = a_0 w + a_1 \T(w) + \cdots + a_n \T^n(w)                                    \\
                               & \subseteq \W                                    &  & \text{(by \cref{5.4.1})}
  \end{align*}
  and thus by \cref{5.4.1} \(\W\) is \(g(\T)\)-invariant.
\end{proof}

\begin{ex}\label{ex:5.4.5}
  Let \(\T\) be a linear operator on a vector space \(\V\) over \(\F\).
  Prove that the intersection of any collection of \(\T\)-invariant subspaces of \(\V\) over \(\F\) is a \(\T\)-invariant subspace of \(\V\) over \(\F\).
\end{ex}

\begin{proof}[\pf{ex:5.4.5}]
  Let \(K\) be an index set and let \(\set{U_{\alpha} : \alpha \in K}\) be a set of \(\T\)-invariant subspaces of \(\V\) over \(\F\).
  Then we have
  \begin{align*}
             & \forall \alpha \in K, \begin{dcases}
                                       U_{\alpha} \text{ is a subspace of } \V \text{ over } \F \\
                                       \T(U_{\alpha}) \subseteq U_{\alpha}
                                     \end{dcases}                                             &  & \text{(by \cref{5.4.1})}                          \\
    \implies & \begin{dcases}
                 \bigcap_{\alpha \in K} U_{\alpha} \text{ is a subspace of } \V \text{ over } \F \\
                 \T\pa{\bigcap_{\alpha \in K} U_{\alpha}} \subseteq \bigcap_{\alpha \in K} U_{\alpha}
               \end{dcases} &  & \text{(by \cref{1.4})}                                \\
    \implies & \bigcap_{\alpha \in K} U_{\alpha} \text{ is } \T\text{-invariant}.                                      &  & \text{(by \cref{5.4.1})}
  \end{align*}
\end{proof}

\setcounter{ex}{6}
\begin{ex}\label{ex:5.4.7}
  Prove that the restriction of a linear operator \(\T\) to a \(\T\)-invariant subspace is a linear operator on that subspace.
\end{ex}

\begin{proof}[\pf{ex:5.4.7}]
  Let \(\V\) be a vector space over \(\F\), let \(\T \in \ls(\V)\) and let \(\W\) be a \(\T\)-invariant subspace of \(\V\) over \(\F\).
  Let \(x, y \in \W\) and let \(c \in \F\).
  Since
  \begin{align*}
    \T_{\W}(cx + y) & = \T(cx + y)                &  & \text{(by \cref{b.0.4})}    \\
                    & = c \T(x) + \T(y)           &  & \text{(by \cref{2.1.2}(b))} \\
                    & = c \T_{\W}(x) + \T_{\W}(y) &  & \text{(by \cref{b.0.4})}    \\
                    & \in \W,                     &  & \text{(by \cref{1.3})}
  \end{align*}
  by \cref{2.1.2}(b) we know that \(\T_{\W} \in \ls(\W)\).
\end{proof}

\begin{ex}\label{ex:5.4.8}
  Let \(\T\) be a linear operator on a vector space \(\V\) over \(\F\) with a \(\T\)-invariant subspace \(\W\) over \(\F\).
  Prove that if \(v\) is an eigenvector of \(\T_{\W}\) with corresponding eigenvalue \(\lambda\), then the same is true for \(\T\).
\end{ex}

\begin{proof}[\pf{ex:5.4.8}]
  We have
  \begin{align*}
             & \T_{\W}(v) = \lambda v &  & \text{(by \cref{5.1.2})} \\
    \implies & \T(v) = \lambda v.     &  & \text{(by \cref{b.0.4})}
  \end{align*}
\end{proof}

\setcounter{ex}{10}
\begin{ex}\label{ex:5.4.11}
  Let \(\T\) be a linear operator on a vector space \(\V\) over \(\F\), let \(v\) be a nonzero vector in \(\V\), and let \(\W\) be the \(\T\)-cyclic subspace of \(\V\) generated by \(v\).
  Prove that
  \begin{enumerate}
    \item \(\W\) is \(\T\)-invariant.
    \item Any \(\T\)-invariant subspace of \(\V\) over \(\F\) containing \(v\) also contains \(\W\).
  \end{enumerate}
\end{ex}

\begin{proof}[\pf{ex:5.4.11}(a)]
  Let \(x \in \W\).
  Then we have
  \begin{align*}
             & \begin{dcases}
                 \exists \seq{v}{1,,k} \in \W \\
                 \exists \seq{a}{1,,k} \in \F
               \end{dcases} : x = \seq[+]{a,v}{1,,k}                                     &  & \text{(by \cref{1.4.3})} \\
    \implies & \begin{dcases}
                 \exists \seq{i}{1,,k} \in \N \\
                 \exists \seq{a}{1,,k} \in \F
               \end{dcases} : x = a_1 \T^{i_1}(v) + \cdots + a_k \T^{i_k}(v)             &  & \text{(by \cref{5.4.3})} \\
    \implies & \begin{dcases}
                 \exists \seq{i}{1,,k} \in \N \\
                 \exists \seq{a}{1,,k} \in \F
               \end{dcases} : \T(x) = a_1 \T^{i_1 + 1}(v) + \cdots + a_k \T^{i_k + 1}(v) &  & \text{(by \cref{2.10})}  \\
    \implies & \T(x) \in \W.                                                             &  & \text{(by \cref{5.4.3})}
  \end{align*}
  Since \(x\) is arbitrary, we see that \(\T(\W) \subseteq \W\) and by \cref{5.4.1} \(\W\) is \(\T\)-invariant.
\end{proof}

\begin{proof}[\pf{ex:5.4.11}(b)]

\end{proof}

\begin{ex}\label{ex:5.4.12}

\end{ex}

\begin{ex}\label{ex:5.4.25}

\end{ex}

\begin{ex}\label{ex:5.4.32}

\end{ex}
